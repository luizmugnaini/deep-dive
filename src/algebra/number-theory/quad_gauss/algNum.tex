\section{Algebraic Numbers and Algebraic Integers}

\begin{definition}
   We say that \(\alpha \in \mathbb{C}\) is an algebraic number if for some \(f
   \in \mathbb{Q}[x],\ \deg(f) > 0\) we have \(f(\alpha) = 0\). 

   On the other hand \(\omega \in \mathbb{C}\) is an algebraic integer if for
   some monic polynomial \(g \in \mathbb{Z}[x]\) we have \(g(\omega) = 0\).
\end{definition}

\begin{proposition}
   Let \(q \in \mathbb{Q}\). Then \(q\) is an algebraic integer if and only if
   \(q \in \mathbb{Z}\).
\end{proposition}

\begin{proof}
   (\(\Rightarrow\)) Let \(q := b/c \in \mathbb{Q}\), such that \(b, c \in
   \mathbb{Z}\) and \((b, c) = 1\), be an algebraic integer, that is, for some
   monic \(g(x) = x^n + a_{n-1}x^{n-1} + \dots + a_1 x + a_0 \in \mathbb{Z}[x]\)
   we have 
   \begin{gather*}
      g(q) = q^n + a_{n-1}q^{n-1} + \dots + a_1 q + a_0 = 0 \\
      \Rightarrow
      b^n + a_{n-1} b^{n-1}c + \dots + a_1 b c^{n-1} + a_0 c^n = 0
   \end{gather*}
   Now, since \(c\) divides every term other than the leader, it should be such
   that \(c \mid b^n\) bet from construction we had \((b, c) = 1 \Rightarrow c
   \mid b^n\) and thus this can be true if and only if \(|c| = 1\), which
   implies immediately that \(q = b/c \in \mathbb{Z}\).

   (\(\Leftarrow\)) Suppose now that \(q \in \mathbb{Z}\) so that \(x - q \in
   \mathbb{Z}[x]\) is such that \(q\) is a root and therefore is an algebraic
   integer.
\end{proof}

\subsection{Algebraic numbers form a field}

\begin{definition}[\(\mathbb{Q}\)-module]
   We say that a set \(V \subseteq \mathbb{C}\) is a \(\mathbb{Q}\)-module if:
   \begin{enumerate}[I.]
      \item  \(\forall a, b \in V\) implies that \(a + b, a - b \in V\).
      \item \(\forall a \in V\) and  \(\forall q \in \mathbb{Q}\) implies \(q a
         \in V\).
      \item \(\exists \{b_1, \dots, b_\ell\} \subseteq V\) such that \(\forall a
         \in V\) there exists \(\{q_1, \dots, q_\ell\} \subseteq \mathbb{Q}\) 
         such that \(a = \sum_{j=1}^{\ell} q_j b_j\).
   \end{enumerate}
   That is, \(V\) is a finite dimensional vector space over \(\mathbb{Q}\). Also,
   from the last property, we denote \(V = [b_1, \dots, b_\ell]\).
\end{definition}

\begin{proposition}
   Let \(V \subseteq \mathbb{C}\) be a \(\mathbb{Q}\)-module and suppose that
   for a given \(\alpha \in \mathbb{C}\) we have that \(\forall \gamma \in V\)
   then \(\alpha \gamma \in V\). Then \(\alpha\) is an algebraic number.
\end{proposition}

\begin{proof}
   Since the statement holds for every element of \(V := [\gamma_1, \dots,
   \gamma_\ell]\), let 
   \[
      \alpha\gamma_i = \sum_{j=1}^{\ell} a_{ij} \gamma_j 
      \Rightarrow 
      \sum_{j = 1}^\ell (a_{ij} - \delta_{ij}\alpha) \gamma_j = 0
   \]
   where \(\delta_{ij}\) is the Kronecker delta. Notice that we can write that
   in matrix language as
   \[
      \begin{pmatrix} 
         a_{11} & \dots & a_{1\ell} \\
         \vdots & \ddots & \vdots \\
         a_{\ell 1} & \dots & a_{\ell \ell}
      \end{pmatrix} 
      - \alpha \cdot \mathrm{Id}_{\ell \times \ell}
   \] 
   but notice that since all rows of such matrix are zero, we can use  column
   reduction to show that there is an equivalent matrix with the last row equal
   to \(0\) (we just sum all rows on the last row) and therefore the determinant
   of such matrix is necessarily equal to \(0\). Since the determinant is a
   polynomial of degree \(\ell\) on \(\alpha\) and has rational coefficients, we
   conclude that \(\alpha\) is indeed an algebraic number.
\end{proof}

\begin{proposition}
   The set of algebraic numbers form a field.
\end{proposition}

\begin{proof}
   Let \(\alpha, \beta \in \mathbb{C}\) be algebraic numbers and by construction
   let \(\sum_{i=0}^n a_i \alpha^i = 0\) and \(\sum_{j=0}^m b_j \beta^j = 0\) be
   such that \(a_i, b_j \in \mathbb{Q}\). Then we can write \(\alpha^n =
   -\sum_{i=0}^{n-1} (a_i/a_n)\alpha^i\) and \(\beta^n = - \sum_{j=0}^{n-1}
   (b_j/b_n) \beta^j\).

   Now we construct a \(\mathbb{Q}\)-module \(V\) with basis  \(\{\alpha^i
   \beta^j : 1 \leqslant i \leqslant n-1 \text{ and } 1 \leqslant j \leqslant
   m-1\}\). Then, given an element \(\gamma \in V\) we have \(c_{ij} \in
   \mathbb{Q}\) such that
   \[
      \gamma = \sum_{\substack{1 \leqslant i \leqslant n-1 \\ 1 \leqslant j
      \leqslant m-1}} c_{ij} \alpha^i \beta^j
   \] 
   and therefore we have that
   \begin{align*}
      \gamma \alpha 
      &= \sum_{\substack{1 \leqslant i \leqslant n-1 \\ 1 \leqslant
      j \leqslant m-1}} c_{ij} \alpha^{i+1} \beta^j 
      \\
      &= \sum_{\substack{1 \leqslant i \leqslant n - 2 \\ 1 \leqslant j \leqslant
      m-1}} c_{ij}\alpha^i \beta^j 
      + \sum_{\substack{i=n-1 \\ 1 \leqslant j \leqslant m-1}} c_{ij}
      \alpha^{i+1} \beta^j 
      \\
      &= \sum_{\substack{1 \leqslant i \leqslant n - 2 \\ 1 \leqslant j
      \leqslant m-1}} c_{ij} \alpha^i \beta^j \sum_{\substack{i = n-1 \\ 1
      \leqslant j \leqslant m-1}} c_{ij} \beta^j \left( - \sum_{k=0}^{n-1} (a_k/a_n)
      \alpha^k \right) 
   \end{align*}
   which is again a polynomial with \(1 \leqslant i \leqslant n-1\) and \(1
   \leqslant j \leqslant m-1\) and therefore we conclude that \(\alpha \gamma
   \in V\). The exact same proof shows that \(\beta \gamma \in V\) for all
   \(\gamma \in V\). Notice therefore that \((\alpha\beta)\gamma \in V\) and
   \((\alpha + \beta) \gamma \in V\) for all \(\gamma \in V\) and therefore from
   the last proposition we conclude that \(\alpha\beta\) and  \(\alpha + \beta\)
   are algebraic numbers.

   Let now \(\lambda\) be a non-zero algebraic number, then let
   \(\sum_{k=0}^\ell d_k \lambda^k = 0\), where \(d_k \in \mathbb{Q}\) for all
   index \(k\). Dividing the polynomial by \(\lambda^n\) we get \(\sum_{k =
   0}^\ell d_K \lambda^{n-k} = 0\) and therefore \(\lambda^{-1}\) is also an
   algebraic number.
\end{proof}

\subsection{Algebraic integers form a ring}


\begin{definition}[\(\mathbb{Z}\)-module]
   A subset \(W \subseteq \mathbb{C}\) is a \(\mathbb{Z}\)-module if:
   \begin{enumerate}[I.]
      \item \(\forall a, b \in W\) implies \(a + b, a - b \in W\).
      \item \(\exists \{b_1, \dots, b_\ell\} \subseteq W\) such that \(\forall a
         \in W\) we have \(\{c_1, \dots, c_{\ell}\} \subseteq \mathbb{Z} \) in
         such a way that \(a = \sum_{i = 1}^\ell c_i b_i\).
   \end{enumerate}
\end{definition}

\begin{proposition}
   Let \(W\) be a \(\mathbb{Z}\)-module and \(\omega \in \mathbb{C}\) be such
   that \(\forall \gamma \in W\) the element \(\omega\gamma \in W\). Then
   \(\omega\) is an algebraic integer.
\end{proposition}

\begin{proposition}
   The set of algebraic integers forms a ring \(\Omega\).
\end{proposition}

I'll not write down the proofs formally since they are basically just the same
proofs as for the algebraic numbers except that we deal with monic polynomials
in \(\mathbb{Z}[x]\).

\begin{proposition}
   Let \(\omega_1, \omega_2 \in \Omega\) and \(p \in \mathbb{Z}\) a prime, then 
   \[
      (\omega_1 + \omega_2)^p \equiv \omega_1^p + \omega_2^p \pmod{p}
   \] 
\end{proposition}

\begin{proof}
   Trivially we have (since \(p \mid \binom{p}{k}\) for all \(1 \leqslant k
   \leqslant p-1\)):
   \[
      (\omega_1 + \omega_2)^p = \binom{p}{0} \omega_1^p + \sum_{k=1}^{p-1}
      \binom{p}{k} \omega_1^{k}\omega_2^{(p-1)-k} + \binom{p}{p} \omega_2^p
      \equiv \omega_1^p + \omega_2^p.
   \]
\end{proof}

\subsection{More on Algebraic Numbers}

\begin{proposition}
   Let \(\alpha\) be a root of a unique monic irreducible \(f \in
   \mathbb{Q}[x]\) (minimal polynomial of \(\alpha\)). Furthermore if \(g \in
   \mathbb{Q}[x]\) such that \(g(\alpha) = 0\) then \(f(x) \mid g(x)\).
\end{proposition}

\begin{proof}
   Define \(f \in \mathbb{Q}[x]\) to be a monic irreducible polynomial with
   \(f(\alpha) = 0\). Now, define \(g \in \mathbb{Q}[x]\) such that \(g(\alpha)
   = 0\). Suppose \(f(x) \nmid g(x)\) then \(\exists s, r \in \mathbb{Q}[x]\) 
   such that \(f(x)r(x) + g(x)s(x) = 1\) but \(f(\alpha) = g(\alpha) = 0\),
   where we get a trivial contradiction, thus \(f(x) \mid g(x)\).

   Let \(f, h \in \mathbb{Q}[x]\) be two monic irreducible polynomials such that
   \(f(\alpha) = h(\alpha) = 0\), then, since \(\deg(h) = \deg(f)\) and both are
   monic, we also have that \(f(\alpha) - h(\alpha) = 0\) which has degree
   strictly less than \(h, f\), but as proved above we should have \(f(x), h(x)
   \mid f(x) - h(x)\), which cannot be true by any means, therefore  \(f(x) -
   h(x) = 0\), proving the uniqueness.
\end{proof}

\begin{definition}
   If \(\alpha\) is an algebraic number and \(f \in \mathbb{Q}[x]\) is its
   minimal polynomial, then if \(\deg(f) = n\) we say that \(\alpha\) is an
   algebraic number of degree \(n\).
\end{definition}

\begin{definition}[Conjugate]
   Let \(f \in \mathbb{Q}[x]\) be an irreducible polynomial of degree \(n\),
   then \(f\) is the minimal polynomial of each of its roots. Let \(\alpha,
   \beta\) be roots of \(f\), then we say that \(\alpha\) and \(\beta\) are
   conjugate.
\end{definition}

\begin{definition}
   Let \(\alpha \in \Omega\). We define the field \(\mathbb{Q}(\alpha)\) of
   complex numbers to be composed of elements \(g(\alpha)/h(\alpha)\) where
   \(g, h \in \mathbb{Q}[x]\) and \(h(\alpha) \neq 0\). 
\end{definition}

\begin{proposition}
   If \(\alpha \in \Omega\) then \(\mathbb{Q}(\alpha) = \mathbb{Q}[\alpha]\).
\end{proposition}

\begin{proof}
   Obviously we have that \(\mathbb{Q}[\alpha] \subseteq \mathbb{Q}(\alpha)\).
   Define \(f \in \mathbb{Q}[x]\) to be the minimal polynomial of \(\alpha\).
   Let \(h \in \mathbb{Q}[x]\) be such that \(h(\alpha) \neq 0\) then
   we have that \(f(x) \nmid h(x)\), implying in the existence of \(s, r \in
   \mathbb{Q}[x]\) such that \(f(x)r(x) + h(x)s(x) = 1\) and therefore we see
   that \(f(\alpha)r(\alpha) + h(\alpha)s(\alpha) = h(\alpha)s(\alpha) = 1\)
   then we conclude that \(s = h^{-1} \in \mathbb{Q}[\alpha]\). From this we can
   trivially construct all elements of \(\mathbb{Q}(\alpha)\) on
   \(\mathbb{Q}[\alpha]\) and thus \(\mathbb{Q}(\alpha) \subseteq
   \mathbb{Q}[\alpha]\).
\end{proof}

\begin{corollary}
   Let \(\alpha\) be an algebraic number of degree \(n\), then
   \([\mathbb{Q}(\alpha) : \mathbb{Q}] = n\).
\end{corollary}

\begin{proof}
   Let \(f(x) := x^n + \dots + a_0\) be the minimal polynomial of
   \(\alpha\). Since \(f(\alpha) = 0\) we see that \(\alpha^n\) is a linear
   combination of \(B = \{1,\dots,\alpha^{n-1}\}\) and also \(B\) is a base for
   \(Q[\alpha]\) since \(B\) is linearly independent from construction of \(f\). 
   This shows that \([\mathbb{Q}[\alpha] : \mathbb{Q}] = n\) but since
   \(\mathbb{Q}[\alpha] = \mathbb{Q}(\alpha)\) then the proposition is already
   proven.
\end{proof}


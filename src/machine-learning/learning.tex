\section{Learnability}

\begin{definition}[True error]
\label{def:true-error}
Consider a target function \(f: \mathcal{X} \to \mathcal{Y}\) and a probability
distribution \(\mathcal{D}\) over \(\mathcal{X}\). If
\(h: \mathcal{X} \to \mathcal{Y}\) is a prediction rule for the learning problem
\((\mathcal{D}, f)\), we define the \emph{true error} of \(h\) as
\[
L_{(\mathcal{D}, f)}(h)
\coloneq \Prob_{x \sim \mathcal{D}}(h(x) \neq f(x))
= \mathcal{D}(\{x \in \mathcal{X} \colon h(x) \neq f(x)\}),
\]
which is the probability of sampling a point \(x \in \mathcal{X}\), according to
the distribution \(\mathcal{D}\), on which the predictor \(h\) fails to meet \(f\).
\end{definition}

\begin{definition}[Training error]
\label{def:training-error}
Let \(S = (x_i, y_i)_{i=1}^m\) be a training set, sampled from an unknown
probability distribution \(\mathcal{D}\) and labelled by a target function
\(f: \mathcal{X} \to \mathcal{Y}\). Let \(h_S: \mathcal{X} \to \mathcal{Y}\) be the hypothesis returned by a learning
algorithm based on the training set \(S\). We define the \emph{training error}
\(L_S: \mathcal{Y}^{\mathcal{X}} \to \R\) to be the map
\[
L_S(h) \coloneq \frac{|\{i \in [m] \colon h(x_i) \neq y_i\}|}{m}.
\]
\end{definition}

\begin{definition}[Empirical risk minimisation]
\label{def:empirical-risk-minimisation}
The learning paradigm known as \emph{empirical risk minimisation}
(ERM) is defined to have the goal of minimising the training
error. As this may cause overfitting, we restrict the set of possible predictors
to a hypothesis collection \(\mathcal{H}\)---this approach induces a bias to the
model. Given a training sample \(S\), we define the learner
\(\text{ERM}_{\mathcal{H}}: 2^{\mathcal{X}} \to \mathcal{Y}^{\mathcal{X}}\) to
choose a predictor \(\text{ERM}_{\mathcal{H}}(S) \coloneq h_S\) such that
\[
h_S \in \argmin_{h \in \mathcal{H}} L_S(h).
\]
\end{definition}

\begin{definition}[Realisability assumption]
\label{def:realisability-assumption}
Within the hypothesis collection \(\mathcal{H}\) there exists \(h^{\star}\) for
which the training error is null:
\[
L_S(h^{\star}) = 0.
\]
\end{definition}

\begin{definition}[Independently identically distributed assumption (i.i.d.)]
\label{def:iid-assumption}
We shall assume that the training set \(S\) has points which are independently
and identically distributed (i.i.d.) with respect to the probability distribution
\(\mathcal{D}\), shortly we write that \(S \sim \mathcal{D}^{|S|}\).
\end{definition}

Since \(S\) is chosen by a randomised procedure, the predictor \(h_S\) and risk
\(L_{(\mathcal{D}, f)}(h_S)\) are both random variables. This allows us to talk
about the probability of \(L_{(\mathcal{D}, f)}(h_S)\) not being large.

\begin{definition}[Confidence parameter]
\label{def:confidence-parameter}
We shall denote by \(\delta\) the probability of \(S\) being a
\emph{non-representative} sample of \(\mathcal{X}\) with respect to the
distribution \(\mathcal{D}\), and \(1 - \delta\) the \emph{confidence parameter}
of the predictor \(h_S\).
\end{definition}

\begin{definition}[Accuracy parameter]
\label{def:accuracy-parameter}
We denote by \(\varepsilon\) the \emph{accuracy parameter}, which measures how
large is \(L_{(\mathcal{D}, f)}(h_S)\) as follows: if
\(L_{(\mathcal{D}, f)}(h_S) > \varepsilon\) we interpret the result of the
algorithm as a \emph{failure}, while
\(L_{(\mathcal{D}, f)}(h_S) \leq \varepsilon\) as a \emph{approximately correct}
predictor.
\end{definition}

\begin{proposition}[Failure bound]
\label{prop:finite-hypothesis-class-failure-bound}
Consider a learner with a finite hypothesis class \(\mathcal{H}\). The
probability of obtaining a sample \(S\), of size \(m\), on which the learner
fails the accuracy parameter is bound by
\[
\mathcal{D}^m(\{\pi_{\mathcal{X}} S \colon L_{(\mathcal{D}, f)}(h_S) > \varepsilon\})
\leq |\mathcal{H}| e^{-\varepsilon m}
\]
where \(\pi_{\mathcal{X}}\) is the projection of the first coordinate.
\end{proposition}

\begin{proof}
Let
\(\mathcal{H}_{\text{bad}} \coloneq \{h \in \mathcal{H} \colon L_{(\mathcal{D},
  f)} > \varepsilon\}\) be the collection of all hypothesis which fail to meet
the expected accuracy of the learner, and define
\[
M \coloneq
\{\pi_{\mathcal{X}} S \colon \exists h \in \mathcal{H}_{\text{bad}},\, L_S(h) = 0\}
= \bigcup_{h \in \mathcal{H}_{\text{bad}}}
\{\pi_{\mathcal{X}} S \colon L_S(h) = 0\},
\]
the collection of all possible misleading non-representative samples for which
the training error can be nullified by the learner.

The realisability assumption says that \(L_S(h_S) = 0\), therefore
\(L_{(\mathcal{D}, f)}(h_S) > \varepsilon\) is only possible in the situation
where there exists a predictor \(h \in \mathcal{H}_{\text{bad}}\) such that
\(L_S(h) = 0\), that is
\[
\{\pi_{\mathcal{X}} S \colon L_{(\mathcal{D}, f)}(h_S) > \varepsilon\} \subseteq M.
\]
Therefore, by the union bound applied to the distribution \(\mathcal{D}^m\) we have
\begin{align*}
  \mathcal{D}^m(\{\pi_{\mathcal{X}} S \colon L_{(\mathcal{D}, f)}(h_S) > \varepsilon\})
  &\leq \mathcal{D}^m(M) \\
  &= \mathcal{D}^m \Big(
      \bigcup_{h \in \mathcal{H}_{\text{bad}}}
      \{\pi_{\mathcal{X}} S \colon L_S(h) = 0\}
    \Big) \\
  &\leq \sum_{h \in \mathcal{H}_{\text{bad}}}
    \mathcal{D}^m(\{\pi_{\mathcal{X}} S \colon L_S(h) = 0\}).
\end{align*}
Now notice that for any \(h \in \mathcal{H}_{\text{bad}}\) we have
\[
\{\pi_{\mathcal{X}} S \colon L_S(h) = 0\}
= \{\pi_{\mathcal{X}} S \colon \forall j \in [m],\, h(x_j) = f(x_j)\},
\]
where \(\pi_{\mathcal{X}} S = (x_j)_{j=1}^m\). Since each point \(x_j \in
\mathcal{X}\) composing \(S\) is chosen in an independently and identically
distributed fashion, it follows that
\begin{align*}
  \mathcal{D}^m(\{\pi_{\mathcal{X}} S \colon L_S(h) = 0\})
  &= \mathcal{D}^m(
    \{\pi_{\mathcal{X}} S \colon \forall j \in [m],\, h(x_j) = f(x_j)\}) \\
  &= \prod_{j=1}^m \mathcal{D}(\{x_j \in \mathcal{X} \colon h(x_j) = f(x_j)\}).
\end{align*}
Moreover, since we define
\(L_{(\mathcal{D}, f)} = \mathcal{D}(\{x \colon h(x) \neq f(x)\})\) we obtain
for each \(j \in [m]\) the bound
\[
\mathcal{D}(\{x_j \colon h(x_j) = f(x_j)\})
= 1 - L_{(\mathcal{D}, f)}(h)
\leq 1 - \varepsilon,
\]
since \(L_{(\mathcal{D}, f)}(h) > \varepsilon\) for any
\(h \in \mathcal{H}_{\text{bad}}\). Now if we use the inequality \(1 -
\varepsilon \leq e^{-\varepsilon}\) we obtain
\[
\mathcal{D}^m(\{\pi_{\mathcal{X}} S \colon L_S(h) = 0\})
\leq (1 - \varepsilon)^m
\leq e^{-m \varepsilon}.
\]

From this analysis we can conclude that
\[
\mathcal{D}^m(\{\pi_{\mathcal{X}} S \colon L_{(\mathcal{D}, f)}(h) > \varepsilon\})
\leq |\mathcal{H}_{\text{bad}}| e^{-m \varepsilon}
\leq |\mathcal{H}| e^{-m \varepsilon},
\]
which is the bound that we settled to prove.
\end{proof}

\begin{corollary}
\label{cor:finite-hypothesis-pac-learnability}
Let \(\mathcal{H}\) be a finite hypothesis collection, and consider parameters
\(0 < \delta < 1\) and \(\varepsilon > 0\). If \(m \in \N\) is such that
\[
m \geq \frac{\log(|\mathcal{H}| / \delta)}{\varepsilon},
\]
then for any labelling map \(f: \mathcal{X} \to \mathcal{Y}\) together with a
distribution \(\mathcal{D}\) following the realisability assumption, we have a
probability of at least \(1 - \delta\) of choosing an i.i.d.~sample \(S\) with
size \(m\) for which every ERM hypothesis \(h_S\) satisfies
\[
L_{(\mathcal{D}, f)}(h_S) \leq \varepsilon.
\]

In other words, for a sufficiently large \(m\), the learner
\(\text{ERM}_{\mathcal{H}}\) will be \emph{probably} (with a confidence of at
least \(1 - \delta\)) \emph{approximately} (up to an error of \(\varepsilon\))
\emph{correct} (PAC).
\end{corollary}

\section{PAC Learning Model}

\begin{definition}[PAC learning]
\label{def:pac-learning}
A hypothesis class \(\mathcal{H}\) is said to be \emph{PAC learnable} if there
exists a map
\[
m_{\mathcal{H}}: (0, 1)^2 \longrightarrow \N,
\]
and a learning algorithm \(A\) such that for every tuple
\((\varepsilon, \delta, f, \mathcal{D}, S)\) where
\begin{itemize}\setlength\itemsep{0em}
\item \(\varepsilon, \delta \in (0, 1)\) are the accuracy and confidence
  parameters, respectively.
\item \(f: \mathcal{X} \to \{0, 1\}\) is a binary labelling map on \(\mathcal{X}\).
\item \(\mathcal{D}\) is a probability distribution on \(\mathcal{X}\).
\item The realisability assumption holds with respect to the triple
  \((\mathcal{H}, \mathcal{D}, f)\).
\item The set \(S = (x_j, f(x_j))_{j=1}^m\) with \(m \geq
  m_{\mathcal{H}}(\varepsilon, \delta)\) is composed of i.i.d.~samples of
  \(\mathcal{X}\) generated by \(\mathcal{D}\) and labelled by \(f\).
\end{itemize}
The algorithm \(A\) returns a predictor \(A(S) = h\) with a probability of at
least \(1 - \delta\), over the choice of \(S\), such that
\[
L_{(\mathcal{D}, f)}(h) \leq \varepsilon.
\]

The map \(m_{\mathcal{H}}: (0, 1)^2 \to \N\) determines the \emph{sample
  complexity} necessary to ensure that the learning algorithm will result in a
PAC solution. We shall impose \(m_{\mathcal{H}}\) to return the minimal sample
complexity such that \(\mathcal{H}\) is PAC learnable with accuracy
\(\varepsilon\) and confidence \(\delta\).
\end{definition}

Rephrasing \cref{cor:finite-hypothesis-pac-learnability} with our new jargon:

\begin{corollary}
\label{cor:finite-hypothesis-pac-learnability-succint}
Every finite hypothesis class \(\mathcal{H}\) is PAC learnable with sample complexity
\[
m_{\mathcal{H}}(\varepsilon, \delta) \leq
\left\lceil \frac{\log(|\mathcal{H}| / \delta)}{\varepsilon} \right\rceil.
\]
\end{corollary}

\subsection{Agnostic PAC Learning Model}

Let \(\mathcal{X}\) be our domain space of features and \(\mathcal{Y}\) be the
space of labels. We shall now consider \(\mathcal{D}\) to be the joint
distribution over the product space \(\mathcal{X} \times \mathcal{Y}\). This
allows for two samples corresponding to the same point \(x \in \mathcal{X}\) to
assume different labels \(y_1, y_2 \in \mathcal{Y}\). With this in mind, we need
to revise the true error to be in accordance with this new distribution
\(\mathcal{D}\).

\begin{definition}[Revising the true error]
\label{def:true-error-revised}
Given a probability distribution \(\mathcal{D}\) over \(\mathcal{X} \times
\mathcal{Y}\), and a predictor \(h: \mathcal{X} \to \mathcal{Y}\), the
\emph{true error} of \(h\) is given by
\[
L_{\mathcal{D}}(h) \coloneq \Prob_{(x, y) \sim \mathcal{D}}(h(x) \neq y)
= \mathcal{D}(\{(x, y) \in \mathcal{X} \times \mathcal{Y} \colon h(x) \neq y\}).
\]
\end{definition}

\begin{proposition}[Bayes optimal predictor]
\label{prop:bayes-optimal-predictor}
Let \(\mathcal{D}\) be any distribution on \(\mathcal{X} \times \{0, 1\}\). The
best predictor \(f_{\mathcal{D}}: \mathcal{X} \to \{0, 1\}\), called the
\emph{Bayes optimal predictor}\footnote{In reality, this predictor cannot be
  used since the learner does not have access to the probability distribution
  \(\mathcal{D}\).}, will be given by
\[
f_{\mathcal{D}}(x) \coloneq
\begin{cases}
  1, &\text{if } \Prob_{(x, y) \sim \mathcal{D}}(y = 1 \mid x) \geq 1/2 \\
  0, &\text{otherwise.}
\end{cases}
\]
That is, for any predictor \(h: \mathcal{X} \to \{0, 1\}\) we have
\(L_{\mathcal{D}}(f_{\mathcal{D}}) \leq L_{\mathcal{D}}(h)\).
\end{proposition}

\begin{definition}[Agnostic PAC learning model]
\label{def:agnostic-pac-learning-model}
A hypothesis class \(\mathcal{H}\) is said to be \emph{agnostic PAC learnable}
if there exists a minimal map \(m_{\mathcal{H}}: (0, 1)^2 \to \N\) and leaning
algorithm \(A\) such that for every tuple \((\varepsilon, \delta,
\mathcal{D}, S)\) where
\begin{itemize}\setlength\itemsep{0em}
\item \(\varepsilon, \delta \in (0, 1)\) are the accuracy and confidence
  parameters, respectively.
\item \(\mathcal{D}\) is a probability distribution over
  \(\mathcal{X} \times \mathcal{Y}\).
\item \(S \in (\mathcal{X} \times \mathcal{Y})^m\) is an i.i.d.~sample of size
  \(m \geq m_{\mathcal{H}}(\varepsilon, \delta)\) generated by \(\mathcal{D}\).
\end{itemize}
The algorithm \(A\) returns a predictor \(A(S) \coloneq g \in \mathcal{H}\) with
a probability of at least \(1 - \delta\) over the choice of \(S\), such that
\[
L_{\mathcal{D}}(g) \leq \min_{h \in \mathcal{H}} L_{\mathcal{D}}(h) + \varepsilon.
\]
\end{definition}

\section{Extending The PAC Model}

\begin{definition}[Generalised loss function]
\label{def:generalised-loss-function}
Let \(\mathcal{H}\) be a collection of hypothesis, and \(\mathcal{Z}\) a domain
of interest. Any map of the type
\[
\ell: \mathcal{H} \times \mathcal{Z} \to \R_{\geq 0}
\]
is a \emph{loss function}.
\end{definition}

\begin{definition}[Generalised error functions]
\label{def:generalised-error-functions}
Consider a hypothesis class \(\mathcal{H}\), a domain \(\mathcal{Z}\) with
associated distribution \(\mathcal{D}\), and a loss function \(\ell: \mathcal{H}
\times \mathcal{Z} \to \R_{\geq 0}\). We define the following:
\begin{enumerate}[(a)]\setlength\itemsep{0em}
\item The \emph{true error} of a predictor \(h \in \mathcal{H}\) is given by
  \[
  L_{\mathcal{D}}(h) \coloneq \Expect_{z \sim \mathcal{D}}[\ell(h, z)].
  \]
\item Given an i.i.d.~sample \(S \in \mathcal{Z}^m\), we define the \emph{empirical
    error} of a predictor \(h \in \mathcal{H}\) to be
  \[
  L_S(h) \coloneq \frac{1}{m} \sum_{z \in S} \ell(h, z)
  \]
\end{enumerate}
\end{definition}

\begin{example}
\label{exp:loss-functions}
The following are extensively used loss functions:
\begin{enumerate}[(a)]\setlength\itemsep{0em}
\item (\emph{Classifier loss}) Given a domain \(\mathcal{Z} = \mathcal{X} \times
  \mathcal{Y}\) and a predictor \(h\), we define the classifier loss function
  \[
  \ell_{\text{class}}(h, (x, y)) \coloneq
  \begin{cases}
    0, &\text{if } h(x) = y \\
    1, &\text{if } h(x) \neq y.
  \end{cases}
  \]

\item (\emph{Square loss}) Consider a domain \(\mathcal{Z} = \mathcal{X} \times
  \mathcal{Y}\) and a predictor \(h\), we define the square loss function as
  \[
  \ell_{\text{sq}}(h, (x, y)) \coloneq (h(x) - y)^2,
  \]
  which is commonly used in regression learning problems.
\end{enumerate}
\end{example}

\begin{definition}[Agnostic PAC learning model for generalised loss functions]
\label{def:agnostic-pac-learning-model-loss-function}
A hypothesis class \(\mathcal{H}\) is said to be \emph{agnostic PAC learnable}
with respect to a domain \(\mathcal{Z}\) and loss function
\(\ell: \mathcal{H} \times \mathcal{Z} \to \R_{\geq 0}\) if there exists a
minimal map \(m_{\mathcal{H}}: (0, 1)^2 \to \N\) and leaning algorithm \(A\)
such that for every tuple \((\varepsilon, \delta, \mathcal{D}, S)\) where
\begin{itemize}\setlength\itemsep{0em}
\item \(\varepsilon, \delta \in (0, 1)\) are the accuracy and confidence
  parameters, respectively.
\item \(\mathcal{D}\) is a probability distribution over
  \(\mathcal{Z}\).
\item \(S \in \mathcal{Z}^m\) is an i.i.d.~sample of size
  \(m \geq m_{\mathcal{H}}(\varepsilon, \delta)\) generated by \(\mathcal{D}\).
\end{itemize}
The algorithm \(A\) returns a predictor \(A(S) \coloneq g \in \mathcal{H}\) with
a probability of at least \(1 - \delta\) over the choice of \(S\), such that
\[
L_{\mathcal{D}}(g) \leq \min_{h \in \mathcal{H}} L_{\mathcal{D}}(h) + \varepsilon,
\]
where the true error function \(L_{\mathcal{D}}\) is given by
\(L_{\mathcal{D}}(h) = \Expect_{z \sim \mathcal{D}}[\ell(h, z)]\).
\end{definition}

\section{Learning via Uniform Convergence}

\begin{definition}[\(\varepsilon\)-representative sample]
\label{def:epsilon-representative-sample}
Let \(\mathcal{Z}\) be a domain with distribution \(\mathcal{D}\), and
associated hypothesis class \(\mathcal{H}\) and loss function \(\ell\). We say
that a sample \(S\) is \(\varepsilon\)-representative if for every predictor \(h
\in \mathcal{H}\) we have
\[
|L_S(h) - L_{\mathcal{D}}(h)| \leq \varepsilon.
\]
\end{definition}

\begin{lemma}
\label{lem:epsilon-representative-good-erm-output}
Consider a context \((\mathcal{Z}, \mathcal{D}, \mathcal{H}, \ell)\), and an
\((\varepsilon/2)\)-representative sample \(S\). Any output of the ERM learner
\[
\text{ERM}_{\mathcal{H}}(S) = h_S \in \argmin_{h \in \mathcal{H}} L_S(h)
\]
will be such that
\[
L_{\mathcal{D}}(h_S) \leq \min_{h \in \mathcal{H}} L_{\mathcal{D}}(h) + \varepsilon.
\]
\end{lemma}

\begin{proof}
Let \(h \in \mathcal{H}\) be any predictor, then
\(|L_S(h) - L_{\mathcal{D}}(h)| \leq \varepsilon/2\) therefore
\[
L_{\mathcal{D}}(h) \leq L_S(h) + \varepsilon/2
\, \text{ and }\,
L_S(h) \leq L_{\mathcal{D}}(h) + \varepsilon/2
\]
In particular, since \(h_S\)
minimises the empirical error we have
\begin{align*}
  L_{\mathcal{D}}(h_S)
  &\leq L_S(h_S) + \varepsilon/2 \\
  &\leq L_S(h) + \varepsilon/2 \\
  &\leq (L_{\mathcal{D}}(h) + \varepsilon/2) + \varepsilon/2 \\
  &= L_{\mathcal{D}}(h) + \varepsilon,
\end{align*}
which proves the proposition.
\end{proof}

\begin{corollary}
\label{cor:erm-is-agnostic-pac-learner-sufficient-condition}
The \(\text{ERM}_{\mathcal{H}}\) learner is an agnostic PAC learner if with a
probability of at least \(1- \delta\) over the random choice of a sample \(S\),
the training set \(S\) is \(\varepsilon\)-representative.
\end{corollary}

\begin{definition}[Uniform convergence property]
\label{def:learner-uniform-convergence-property}
A hypothesis class \(\mathcal{H}\) is said to \emph{satisfy the uniform convergence
property}---with respect to a domain \(\mathcal{Z}\) and a loss function
\(\ell\)---if there exists a minimal function
\[
m_{\mathcal{H}}^{\text{UC}}: (0, 1)^2 \longrightarrow \N
\]
such that: for every distribution \(\mathcal{D}\) over \(\mathcal{Z}\), and
parameters \(\varepsilon, \delta \in (0, 1)\), if \(S \in \mathcal{Z}^m\) is an
i.i.d.~sample with size
\(m \geq m_{\mathcal{H}}^{\text{UC}}(\varepsilon, \delta)\), then with a
probability of at least \(1 - \delta\) the sample \(S\) is
\(\varepsilon\)-representative.
\end{definition}

\begin{corollary}
\label{cor:unif-conv-hypothesis-is-agnostically-pac-learnable}
If a hypothesis class \(\mathcal{H}\) has the uniform convergence property---with
respect to \((\mathcal{Z}, \ell)\)---with a function
\(m_{\mathcal{H}}^{\text{UC}}\) then \(\mathcal{H}\) is agnostically PAC
learnable with a sample complexity
\[
m_{\mathcal{H}}(\varepsilon, \delta) \leq
m_{\mathcal{H}}^{\text{UC}}(\varepsilon/2, \delta).
\]
\end{corollary}

\section{Finite Hypothesis Classes are Uniform Convergent}

\subsection{Important Measure Concentration Inequalities}

\begin{proposition}[Strong law of large numbers]
\label{prop:strong-law-of-large-numbers}
Let \((Z_1, \dots, Z_m)\) be a sequence of i.i.d.~random variables with equal
mean \(\mu\). The strong law of large numbers states that when \(m \to \infty\)
the empirical average \(\overline Z \coloneq \frac{1}{m} \sum_{j=1}^m Z_j\)
converges to the expected value \(\mu\) with a probability of \(1\).
\end{proposition}

\begin{lemma}[Markov's inequality]
\label{lem:markov-inequality}
Let \(Z\) be a non-negative random variable. Then for any \(a \geq 0\) we have
\[
\Prob(Z \geq a) \leq \Expect[Z] / a,
\]
known as the Markov's inequality.
\end{lemma}

\begin{proof}
Since \(\Prob(Z \geq x)\) is monotonically non-increasing as a function of
\(x\), for any \(a \geq 0\) we have
\begin{align*}
  \Expect[Z]
  &= \int_0^{\infty} \Prob(Z \geq x)\, \diff x \\
  &\geq \int_0^a \Prob(Z \geq x)\, \diff x \\
  &\geq \int_0^a \Prob(Z \geq a)\, \diff x \\
  &= a \Prob(Z \geq a),
\end{align*}
therefore \(\Prob(Z \geq a) \leq \Expect[Z] / a\) as wanted.
\end{proof}

\begin{lemma}
\label{lem:markov-inequality-lemma}
Given a random variable \(Z\) with values in \([0, 1]\), denote \(\mu \coloneq
\Expect[Z]\). Then for any \(a \in (0, 1)\) we have the following two upper-bounds
\begin{gather}
  \Prob(Z > 1 - a) \geq \frac{\mu - (1 - a)}{a}, \\
  \Prob(Z > a) \geq \frac{\mu - a}{1 - a} \geq \mu - a.
\end{gather}
\end{lemma}

\begin{proof}
Define a random variable \(X \coloneq 1 - Z\). Since \(Z \in [0, 1]\) then \(Y\)
is non-negative and has \(\Expect[Y] = 1 - \mu\). By the Markov's inequality on
\(Y\) we have
\[
\Prob(Z \leq 1 - a)
= \Prob(1 - Z \geq a)
= \Prob(Y \geq a)
\leq \frac{\Expect[Y]}{a}
= \frac{1 - \mu}{a}.
\]
Therefore
\[
\Prob(Z > 1 - a) \geq 1 - \frac{1 - \mu}{a} = \frac{\mu - (1 - a)}{a}.
\]
\end{proof}

\begin{lemma}[Hoeffding's lemma]
\label{lem:hoeffding-lemma}
Let \(X\) be a random variable taking values in the interval \([a, b]\),
and such that \(\Expect[X] = 0\). Then for every \(\lambda > 0\) we have
\[
\Expect[e^{\lambda X}] \leq \exp\Big( \frac{\lambda^2 (b - a)^2}{8} \Big).
\]
\end{lemma}

\begin{proof}
Notice that the exponential map \(f(x) \coloneq e^{\lambda x}\) is convex,
therefore by definition it follows that for any \(t \in (0, 1)\) and \(x \in [a,
b]\) we have
\[
f(x) \leq t f(a) + (1 - t) f(b).
\]
Letting \(t \coloneq \frac{b - x}{b - a} \in [0, 1]\) we find
\[
e^{\lambda x} \leq \frac{b-x}{b-a} e^{\lambda a} + \frac{x-a}{b-a} e^{\lambda b}.
\]
If we now consider the expectation of \(f\) with respect to the random variable
\(X\) we find the following inequality:
\begin{align}\label{eq:hoeffding-lemma}
  \nonumber
  \Expect[e^{\lambda X}]
  &\leq \frac{b - \Expect[X]}{b - a} e^{\lambda a}
    + \frac{\Expect[X] - a}{b - a} e^{\lambda b} \\
  &= \frac{b}{b - a} e^{\lambda a} - \frac{a}{b - a} e^{\lambda b}
\end{align}
since the expected value of \(X\) is null. Given \(p \coloneq -\frac{a}{b - a}\)
define a map
\[
L(h) \coloneq -h p + \log(1 - p + p e^h)
\]
so that, for \(h_0 \coloneq \lambda(b - a)\) we have that \(e^{L(h_0)}\) is
exactly the right-hand side of \cref{eq:hoeffding-lemma}. Using the fact that
\(L(0) = L'(0) = 0\) and that \(L''(h) \leq 1/4\) for any \(h\), we find by the
Taylor expansion of \(L\) about \(0\) that
\[
L(h_0) \approx L(0) + L'(0) h_0 + \frac{L''(h)}{2} h_0^2 = \frac{L''(h_0)}{2}
\leq \frac{h_0^2}{8}.
\]
With this we have
\[
\Expect[e^{\lambda X}] \leq e^{L(h_0)} \leq e^{h_0^2/8}
= e^{\frac{\lambda^2 (b - a)^2}{8}},
\]
which is the bound we wanted to prove.
\end{proof}

\begin{lemma}[Hoeffding's inequality]
\label{lem:hoeffding-inequality}
Let \((Z_1, \dots, Z_m)\) be a collection of i.i.d.~random variables,
and let \(\overline Z \coloneq \frac{1}{m} \sum_{j=1}^m Z_j\).  If for
all \(1 \leq j \leq m\) we have \(\Expect[Z_j] = \mu\) and
\(\Prob[a \leq Z_j \leq b] = 1\), then for any \(\varepsilon > 0\) we have
an upper-bound
\[
\Prob\big( \big| \overline{Z} - \mu \big| > \varepsilon \big)
\leq 2 \exp\Big( -\frac{2 m \varepsilon^{2}}{(b - a)^2} \Big).
\]
\end{lemma}

\begin{proof}
For each \(j\), let \(X_j \coloneq Z_j - \Expect[Z_j]\) and define
\(\overline{X} \coloneq \frac{1}{m} \sum_{j=1}^m X_j\). Since the exponential
function is monotonically increasing, we have that for every pair
\(\lambda, \varepsilon > 0\):
\[
\Prob\big( \overline{X} \geq \varepsilon \big)
= \Prob(e^{\lambda X} \geq e^{\lambda \varepsilon})
\leq e^{-\lambda \varepsilon} \Expect[e^{\lambda \overline X}],
\]
where we used the Markov's inequality \cref{lem:markov-inequality}. Since the
random variables are i.i.d.~we have
\[
\Expect[e^{\lambda \overline X}]
= \Expect[e^{\lambda \frac{1}{m} \sum_j X_j}]
= \Expect\Big[ \prod_{j=1}^m e^{\lambda X_j/m} \Big]
= \prod_{j=1}^m \Expect[e^{\lambda X_j/m}].
\]
Via \cref{lem:hoeffding-lemma}, for each index \(j\) we have an upper-bound
\[
\Expect[e^{\lambda X_j/m}] \leq \exp\Big( \frac{\lambda^2 (b - a)^2}{8 m^2} \Big).
\]
With this bound in hands we obtain
\[
\Expect[e^{\lambda \overline X}]
= \prod_{j=1}^m \exp\Big( \frac{\lambda^2 (b - a)^2}{8 m^2} \Big)
= \exp\Big( \frac{\lambda^2 (b - a)^2}{8 m} \Big).
\]
Therefore it follows that
\[
\Prob\big( \overline X \geq \varepsilon \big)
\leq \exp\Big(-\lambda \varepsilon + \frac{\lambda^2 (b - a)^2}{8 m} \Big).
\]
If we let \(\lambda \coloneq \frac{4 m \varepsilon}{(b-a)}^2\) we obtain
\[
\Prob\big( \overline X \geq \varepsilon \big)
\leq \exp\Big( -\frac{2 m \varepsilon}{(b - a)^2} \Big).
\]

If we now consider the case for the random variable \(-\overline{X}\) we shall
find
\[
\Prob(\overline X \leq -\varepsilon)
\leq \exp\Big( -\frac{2 m \varepsilon^{2}}{(b-a)^2} \Big).
\]
Using the union bound
\(\Prob(|\overline X| \geq \varepsilon) \leq \Prob(\overline X \geq \varepsilon)
+ \Prob(\overline X \leq - \varepsilon)\) we obtain the wanted inequality.
\end{proof}

\section{Uniform Convergence for Finite Hypothesis Classes}

\begin{proposition}
\label{prop:finite-hypothesis-agnostically-pac-learnable}
Let \(\mathcal{H}\) be a \emph{finite} hypothesis class with respect to a domain
\(\mathcal{Z}\), and let \(\ell: \mathcal{H} \times \mathcal{Z} \to [0, 1]\) be
a loss function. Then \(\mathcal{H}\) has the uniform convergence property with
a sample complexity of
\[
m_{\mathcal{H}}^{\text{UC}}(\varepsilon, \delta) \leq
\left\lceil \frac{\log(2 |\mathcal{H}| / \delta)}{2 \varepsilon^2} \right\rceil.
\]
Moreover, \(\mathcal{H}\) is agnostically PAC learnable with respect to the ERM
algorithm with a sample complexity
\[
m_{\mathcal{H}}(\varepsilon) \leq m_{\mathcal{H}}^{\text{UC}}(\varepsilon/2, \delta)
\leq
\left\lceil \frac{2 \log(2 |\mathcal{H}| / \delta)}{\varepsilon^2} \right\rceil
\]
\end{proposition}

\begin{proof}
Fix parameters \(\varepsilon, \delta \in (0, 1)\). Lets consider the set
\[
\mathcal{S}_{\text{bad}} \coloneq
\{S \colon \exists h \in \mathcal{H},\, |L_S(h) - L_{\mathcal{D}}(h)| > \varepsilon\}
= \bigcup_{h \in \mathcal{H}} \{S \colon |L_S(h) - L_{\mathcal{D}}(h)| > \varepsilon\}
\]
and for the sake of brevity define the notation
\(\mathcal{S}_{\text{bad}}^h \coloneq \{S \colon |L_S(h) - L_{\mathcal{D}}(h)| >
\varepsilon\}\) for each \(h \in \mathcal{H}\). Via the union bound we find
\[
\mathcal{D}^m(\mathcal{S}_{\text{bad}})
\leq \sum_{h \in \mathcal{H}} \mathcal{D}^m(\mathcal{S}_{\text{bad}}^h)
\]

Fix a predictor \(h \in \mathcal{H}\), chosen prior to the sampling of the
training set. Let \(S = (z_j, f(z_j))_{j=1}^m\) be an i.i.d.~sampled training
set. Since each \(z_j\) is sampled i.i.d.~it follows that
\(\Expect[\ell(h, z_j)] = L_{\mathcal{D}}(h)\) from the definition of the true
error. By the linearity of the expected value, we also have
\[
\Expect[L_S(h)] = \Expect\Big[ \frac{1}{m} \sum_{j=1}^m \ell(h, z_j) \Big]
= \frac{1}{m} \sum_{j=1}^m \Expect[\ell(h, z_j)]
=  L_{\mathcal{D}}(h).
\]
It follows that \(|L_{\mathcal{D}}(h) - L_S(h)| = |\Expect[L_S(h)] - L_S(h)|\)
and we would wish to show that \(L_S(h)\) is \emph{concentrated} around its
expected value.

Since \(L_S(h)\) is composed of the sum of \(m\) i.i.d.~random variables, we can
apply the law of large numbers to conclude that when \(m \to \infty\) the
empirical average \(L_S(h)\) converges to its true expected value
\(L_{\mathcal{D}}(h)\). We would like to quantify this for finite \(m\), and for
that we shall make use of the Hoeffding's inequality. For that, consider the
sequence of random variables \((\ell(h, z_1), \dots, \ell(h, z_m))\), which are
sampled i.i.d.~since \(h\) is a fixed hypothesis and \((z_1, \dots, z_m)\) are
i.i.d.~random variables. Since \(\im \ell \subseteq [0, 1]\) and that
\(z_j \in [0, 1]\) for all indices \(j\). We have satisfied every condition of
\cref{lem:hoeffding-inequality} and thus we may apply the Hoeffding's inequality
as follows:
\[
\mathcal{D}^m(\mathcal{S}_{\text{bad}}^h)
= \Prob(|L_S(h) - L_{\mathcal{D}}(h)| > \varepsilon)
\leq 2 e^{-2 m \varepsilon}.
\]
Therefore going back to \(\mathcal{S}_{\text{bad}}\) we find
\[
\mathcal{D}^m(\mathcal{S}_{\text{bad}})
\leq \sum_{h \in \mathcal{H}} \mathcal{D}^m(\mathcal{S}_{\text{bad}}^{h})
= 2 |\mathcal{H}| e^{-2 m \varepsilon}.
\]

Therefore, by choosing an integer \(m \in \N\)
\[
m \geq \frac{\log(2 |\mathcal{H}| / \delta)}{2 \varepsilon^2}
\]
we ensure that \(\mathcal{D}^m(\mathcal{S}_{\text{bad}}) \leq \delta\),
therefore
\[
\mathcal{D}^m(\{S \colon \forall h \in \mathcal{H},\,
|L_S(h) - L_{\mathcal{D}}(h)| \leq \varepsilon\}) \geq 1 - \delta.
\]
\end{proof}

\section{Universal Learners are Impossible}

\begin{theorem}[No-free-lunch]
\label{thm:no-free-lunch}
Let \(A\) be a learning algorithm assigned for the binary classification of a
domain \(\mathcal{X}\) with respect to the \(\{0, 1\}\)-loss function
\(\ell_{\text{class}}\). Let \(m < |\mathcal{X}|/2\) be the size of a training
set. There exists a distribution \(\mathcal{D}\) over the domain
\(\mathcal{X} \times \{0, 1\}\) for which
\begin{enumerate}[(a)]\setlength\itemsep{0em}
\item There exists a labelling map \(f: \mathcal{X} \to \{0, 1\}\) for which
  \(L_{\mathcal{D}}(f) = 0\).
\item With a probability of at least \(1/7\) over the choice of the training set
  \(S \sim \mathcal{D}^m\) we have
  \[
  L_{\mathcal{D}}(A(S)) \geq \frac{1}{8}.
  \]
\end{enumerate}
In other words, there exists a true labelling map \(f\), with respect to the
distribution \(\mathcal{D}\), underlying the learning problem, and the algorithm
\emph{fails} to output a predictor with a good approximation of \(f\).
\end{theorem}

\begin{proof}
Let \(C \subseteq \mathcal{X}\) be a subset of size \(2 m\). We'll prove that if
a learner has access to only half of \(C\), then it has no information on how to
correctly label the remaining half of \(C\). The number of maps
\(C \to \{0, 1\}\) is given by \(T \coloneq |\{0, 1\}|^{|C|} = 2^{2m}\): let
\((f_1, \dots, f_T)\) denote all such maps. For each \(1 \leq j \leq T\) define
a distribution \(\mathcal{D}_j\) over \(C \times \{0, 1\}\) where
\[
\mathcal{D}_j(\{(x, y)\}) =
\begin{cases}
  \frac{1}{|C|}, &\text{if } y = f_i(x) \\
  0, &\text{otherwise.}
\end{cases}
\]
That is, \(\mathcal{D}_j\) makes \(f_j\) the true labelling function on
\(C\)---so that \(L_{\mathcal{D}_j}(f_j) = 0\).

Let \(K = (2 m)^m\) denote the number of possible sequences consisting of \(m\)
i.i.d.~sampled instances from \(C\), and let \((S_1, \dots, S_K)\) denote all
such sequences. Given a sequence \(S_j = (x_1, \dots, x_m)\), define
\(S_j^i \coloneq (x_j, f_j)_{j=1}^m\) to be the labelling of \(S_j\) by the map
\(f_i\). Given a distribution \(D_i\), the available training sets for the
algorithm \(A\) are \((S_1^i, \dots, S_K^i)\), which are i.i.d.~and labelled by
the same map \(f_i\). Therefore one has
\begin{equation}\label{eq:nfl-expect-AS}
\Expect_{S \sim \mathcal{D}_i^m}[L_{\mathcal{D}_i}(A(S))]
= \frac{1}{K} \sum_{j=1}^K L_{\mathcal{D}_i}(A(S_{j}^i)).
\end{equation}

Considering all \(T\) pairs \((f_j, \mathcal{D}_j)\) we have the following
relations:
\begin{align}\label{eq:nfl-max-i-in-T}
  \nonumber
  \max_{1 \leq i \leq T} \Expect_{S \sim \mathcal{D}_i^m}[L_{\mathcal{D}_i}(A(S))]
  &\geq \frac{1}{T} \sum_{i=i}^{T} \Expect_{S \sim \mathcal{D}_i^m}[L_{\mathcal{D}_i}(A(S))]
  \\
  \nonumber
  &= \frac{1}{T} \sum_{i=i}^{T} \Big(
    \frac{1}{K} \sum_{j=1}^K L_{\mathcal{D}_i}(A(S_j^i))
    \Big) \\
  \nonumber
  &= \frac{1}{K} \sum_{j=1}^{K} \Big(
    \frac{1}{T} \sum_{i=1}^T L_{\mathcal{D}_i}(A(S_j^i))
    \Big) \\
  &\geq \min_{1 \leq j \leq K} \frac{1}{T} \sum_{i=1}^T L_{\mathcal{D}_i}(A(S_j^i)).
\end{align}
Fix any \(1 \leq j \leq K\) and let \(S_j \coloneq (x_1, \dots, x_m)\) and let
\((v_1, \dots, v_p)\) be a sequence containing all instances of \(C\) not
appearing in \(S_j\)---which certainly has \(p \geq m\). Hence, for any
labelling \(h: C \to \{0, 1\}\) and any \(1 \leq i \leq T\) we have a true error
\begin{align*}
L_{\mathcal{D}_i}(h)
&= \frac{1}{2m} \sum_{x \in C} \ell_{\text{class}}(h, (x, f_i(x))) \\
&\geq \frac{1}{2 m} \sum_{k=1}^p \ell_{\text{class}}(h, (v_k, f_i(v_k))) \\
&\geq \frac{1}{2 p} \sum_{k=1}^p \ell_{\text{class}}(h, (v_k, f_i(v_k))).
\end{align*}
From this we obtain that
\begin{align}\label{eq:nfl-avg-true-loss-T}
  \nonumber
  \frac{1}{T} \sum_{i=1}^T L_{\mathcal{D}_i}(A(S_j^i))
  &\geq \frac{1}{T} \sum_{i=1}^T \Big(
    \frac{1}{2 p} \sum_{k=1}^p \ell_{\text{class}}(A(S_j^i), (v_k, f_i(v_k)))
    \Big) \\
  \nonumber
  &= \frac{1}{2 p} \sum_{k=1}^p \Big(
    \frac{1}{T} \sum_{i=1}^T \ell_{\text{class}}(A(S_j^i), (v_k, f_i(v_k)))
    \Big) \\
  &\geq \frac{1}{2} \min_{1 \leq k \leq p} \frac{1}{T}
    \sum_{i=1}^T \ell_{\text{class}}(A(S_j^i), (v_k, f_i(v_k))).
\end{align}
For the last time, fix an index \(1 \leq k \leq p\). Partition \((f_1, \dots,
f_T)\) into a collection of \(T/2\) disjoint pairs of the form \((f_i, f_{i'})\)
for which \(f_i(c) \neq f_{i'}(c)\) if and only if \(c = v_k\). From
construction, such pairs satisfy \(S_j^i = S_j^{i'}\) for any \(1 \leq j \leq
K\), therefore
\[
\ell_{\text{class}}(A(S_j^i), (v_k, f_i(v_k)))
+ \ell_{\text{class}}(A(S_j^{i'}), (v_k, f_{i'}(v_k)))
= 1,
\]
which in turn implies in
\begin{equation}\label{eq:nfl-avg-T-is-half}
\frac{1}{T} \sum_{i=1}^T \ell_{\text{class}}(A(S_j^i), (v_k, f_i(v_k)) = \frac{1}{2}.
\end{equation}

Now substituting \cref{eq:nfl-avg-T-is-half} in \cref{eq:nfl-avg-true-loss-T}
results in
\[
\frac{1}{T} \sum_{i=1}^{T} L_{\mathcal{D}_i}(A(S_j^i)) \geq \frac{1}{4}.
\]
If we now substitute this into \cref{eq:nfl-max-i-in-T} we get
\[
\max_{1 \leq i \leq T} \Expect_{S \sim \mathcal{D}_i^m}[L_{\mathcal{D}_i}(A(S))]
\geq \frac{1}{4},
\]
and say this maximum is attained at an index \(1 \leq i_0 \leq T\). Denote
\((f_{i_0}, \mathcal{D}_{i_0}) \coloneq (f, \mathcal{D})\) for short. Using the
\cref{lem:markov-inequality-lemma} we have
\[
\Prob(L_{\mathcal{D}}(A(S)) \geq 1/8)
\geq \frac{\Expect_{S \sim \mathcal{D}^m}[L_{\mathcal{D}}(A(S))] - 1/8}{1 - 1/8}
= \frac{1}{7}
\]
\end{proof}

\begin{corollary}
\label{cor:full-hypothesis-class-not-pac-learnable}
Let \(\mathcal{X}\) be an infinite domain and define
\(\mathcal{H} \coloneq \{0, 1\}^{\mathcal{X}}\) to be our hypothesis class---that is, the collection of all
possible binary classifiers of \(\mathcal{X}\). Then
\(\mathcal{H}\) is \emph{not} PAC learnable.
\end{corollary}

\begin{proof}
Suppose, for the sake of contradiction, that \(\mathcal{H}\) is PAC learnable. Take
\(\varepsilon < 1/8\) and \(\delta < 1/7\). From assumption, there exists an algorithm
\(A\) and a number \(m \in \N\), depending on the choice of \(\varepsilon\) and
\(\delta\), such that for any distribution \(\mathcal{D}\) over
\(\mathcal{X} \times \{0, 1\}\) and labelling function
\(f: \mathcal{X} \to \{0, 1\}\) with \(L_{\mathcal{D}}(f) = 0\)---then, with a probability of at least
\(1 - \delta\) over a sample \(S \sim \mathcal{D}^m\), the algorithm \(A\) will output a
classifier with \(L_{\mathcal{D}}(A(S)) \leq \varepsilon\). Since
\(\mathcal{X}\) is infinite, then certainly its cardinality overcomes \(2 m\) and by
\cref{thm:no-free-lunch} the algorithm \(A\) fails for some distribution
\(\mathcal{D}_0\) with a probability greater than \(1/7\)---in this case the failure of
\(A\) means \(L_{\mathcal{D}_0}(A(S)) > 1/8\). This is a contradiction and therefore
proves that \(\mathcal{H}\) is not PAC learnable.
\end{proof}

\begin{definition}[Threshold function]
\label{def:threshold-function}
Let \(P\) be a proposition taking values in the set
\(\{\texttt{false}, \texttt{true}\}\). We define a map
\(\mathbf{1}_P: \{\texttt{false}, \texttt{true}\} \to \{0, 1\}\) as
\[
\mathbf{1}_P(x) =
\begin{cases}
  1, &\text{if } P(x) = \texttt{true} \\
  0, &\text{if } P(x) = \texttt{false}.
\end{cases}
\]
A common use of such maps are threshold functions: for each \(a \in \R\) we define
a map \(\mathbf{1}_{< a}\) assuming \(1\) if the argument is less than \(a\) and
\(0\) otherwise.
\end{definition}

\begin{lemma}
\label{lem:threshold-functions-pac-learnable}
Let \(\mathcal{H}\) denote the infinite collection of all threshold functions as in
\cref{def:threshold-function}. Then \(\mathcal{H}\) is PAC learnable with the ERM
algorithm with a sample complexity of
\[
m_{\mathcal{H}}(\varepsilon, \delta) \leq \left\lceil \frac{\log(2/\delta)}{\varepsilon} \right\rceil.
\]
\end{lemma}

\begin{proof}
Via the realisability assumption, let \(a^{\star} \in \R\) be such that
\(L_{\mathcal{D}}(\mathbf{1}_{a^{\star}}) = 0\). Consider the marginal distribution
\(\mathcal{D}_{\mathcal{X}}\) and take points \(a_0 < a^{\star} < a_1\) such that\footnote{If such
  points do not exist, we can set \(a_0 \coloneq -\infty\) and \(a_1 \coloneq \infty\).}
\[
\Prob_{x \sim \mathcal{D}_{\mathcal{X}}}(a_0 < x < a^{\star})
= \Prob_{x \sim \mathcal{D}_{\mathcal{X}}}(a^{\star} < x < a_1)
= \varepsilon.
\]
Let \(S \sim \mathcal{D}^m\) be a training set, and define constants\footnote{If the first
  set is empty, we instead set \(b_0 \coloneq \infty\), and if the latter is empty then
  \(b_1 \coloneq -\infty\).}
\[
b_0 \coloneq \max_{(x, 1) \in S} x
\quad \text{ and } \quad
b_1 \coloneq \min_{(x, 0) \in S} x
\]
Let \(b_S \in \R\) be a threshold obtained by \(\text{ERM}_{\mathcal{H}}\), giving
\(b_0 < b_S < b_1\). Notice that in order to have a true error
\(L_{\mathcal{D}}(\mathbf{1}_{< b_S}) \leq \varepsilon\) it is sufficient to ensure that \(a_0 \leq b_0\)
and \(b_1 \leq a_1\) so that \(b_S \in (a_0, a_1)\). Notice that
\begin{align*}
  \Prob_{S \sim \mathcal{D}^m}(L_{\mathcal{D}}(\mathbf{1}_{< b_S}) > \varepsilon)
  &\leq \Prob_{S \sim \mathcal{D}^m}(b_0 < a_0 \text{ and } b_1 > a_1) \\
  &\leq \Prob_{S \sim \mathcal{D}^m}(b_0 < a_0) + \Prob(b_1 > a_1)
\end{align*}
Notice that in order to have the event \(b_0 < a_0\) one has to sample every point
\((x, y) \in S\) with \(x < a_0\), since the probability of having a sample \(x \in
(a_0, a^{\star})\) is \(\varepsilon\) it follows that
\begin{align*}
  \Prob_{S \sim \mathcal{D}^m}(b_0 < a_0)
  &= \Prob_{S \sim \mathcal{D}^m}(\forall (x, y) \in S,\, x \notin (a_0, a^{\star})) \\
  &= (1 - \varepsilon)^m \\
  &\leq e^{-m \varepsilon} \\
  &\leq e^{- \frac{\log(2/\delta)}{\varepsilon} \varepsilon} \\
  &= \delta/2,
\end{align*}
since \(m > \log(2/\delta)/\varepsilon\). Analogously, \(b_1 > a_1\) only if each point
\((x, y)\) of \(S\) has \(x > a_1\), therefore
\(\Prob_{S \sim \mathcal{D}^m}(b_1 > a_1) \leq \delta/2\).  This shows that
\[
\Prob_{S \sim \mathcal{D}^m}(L_{\mathcal{D}}(\mathbf{1}_{< b_S} > \varepsilon)) \leq \delta.
\]
\end{proof}

\begin{definition}[Restricting the hypothesis class]
\label{def:restrict-hypothesis-class}
Let \(\mathcal{H}\) be a hypothesis class of maps \(\mathcal{X} \to \mathcal{Y}\) and take a finite set
\(C \subseteq \mathcal{X}\). We define \(\mathcal{H}|_C \coloneq (h|_C)_{h \in \mathcal{H}}\) to be the restriction of the
hypothesis class to \(C\). We say that \(\mathcal{H}\) \emph{shatters} \(C\) if it is the
case that \(\mathcal{H}|_C = \mathcal{Y}^{C}\).
\end{definition}

\begin{corollary}
\label{cor:shatters-2m-cannot-learn-with-train-of-size-m}
Let \(\mathcal{H}\) be a hypothesis class of maps \(\mathcal{X} \to \{0, 1\}\), and \(m\) be the
training set size. Assume the existence of a subset \(C \subseteq \mathcal{X}\) with cardinality
\(2 m\) such that \(\mathcal{H}|_C\) shatters \(C\). It follows that for any learning
algorithm \(A\) there exists a distribution \(\mathcal{D}\) over \(\mathcal{X} \times \{0, 1\}\) and a
predictor \(h \in \mathcal{H}\) for which \(L_{\mathcal{D}}(h) = 0\) and having a probability of at
least \(1/7\) over the choice of samples \(S \sim \mathcal{D}^m\) that \(L_{\mathcal{D}}A(S) \geq 1/8\).
\end{corollary}

\begin{definition}[VC-dimension]
\label{def:vc-dimension}
Let \(\mathcal{H}\) be a hypothesis class. We define the \emph{VC-dimension} of
\(\mathcal{H}\) to be:
\[
\vcdim \mathcal{H} \coloneq \max \{|C| \colon C \subseteq \mathcal{X} \text{ is shattered by } \mathcal{H}|_C\}.
\]
\end{definition}

\begin{theorem}
\label{thm:infinite-VC-dim-not-learnable}
If \(\mathcal{H}\) is a class with \emph{infinite} VC-dimension, then
\(\mathcal{H}\) is \emph{not PAC learnable}.
\end{theorem}

\begin{proof}
Let \(S\) be a training set with size \(m\). Since
\(\vcdim \mathcal{H} = \infty\) then there exists a set with size \(2 m\) that is shattered by
\(\mathcal{H}\)---thus by \cref{cor:shatters-2m-cannot-learn-with-train-of-size-m}
\(\mathcal{H}\) isn't PAC learnable.
\end{proof}

\section{Fundamental Theorem of PAC Learning}

\begin{definition}[Growth function]
\label{def:growth-function-learning}
Let \(\mathcal{H}\) be a hypothesis class. We define the growth function of \(\mathcal{H}\) to be
the map \(\Growth_{\mathcal{H}}: \N \to \N\) given by
\[
\Growth_{\mathcal{H}}(m) \coloneq \max \{|\mathcal{H}_C| \colon C \subseteq \mathcal{X} \text{ with } |C| = m\}.
\]
\end{definition}

\begin{corollary}
\label{cor:growth-function-lower-vc-dim}
If \(\vcdim \mathcal{H} = d\) then for any \(m \leq d\) we have \(\Growth_{\mathcal{H}}(m) = 2^m\).
\end{corollary}

\begin{lemma}[Sauer-Shelah-Perles]
\label{lem:sauer-shelah-perles}
Let \(\mathcal{H}\) be a hypothesis class with \(\vcdim \mathcal{H} \leq d\) finite. Then for any
\(m \in \N\) one has
\[
\Growth_{\mathcal{H}}(m) \leq \sum_{j=0}^d \binom m j.
\]
In particular, if \(m > d + 1\) then our bound becomes
\[
\Growth_{\mathcal{H}}(m) \leq \Big(\frac{e m}{d}\Big)^d
\]
\end{lemma}

\begin{theorem}[Fundamental theorem of statistical learning]
\label{thm:fundamental-thm-stat-learning}
Let \(\mathcal{H}\) be a hypothesis class of maps \(\mathcal{X} \to \{0, 1\}\) together with the
binary loss function. The following conditions are equivalent:
\begin{enumerate}[(a)]\setlength\itemsep{0em}
\item The collection \(\mathcal{H}\) has the uniform convergence property.

\item Any ERM rule succeeds at PAC learning \(\mathcal{H}\) agnostically.

\item The collection \(\mathcal{H}\) is agnostic PAC learnable.

\item The collection \(\mathcal{H}\) is PAC learnable.

\item Any ERM rule succeeds at PAC learning \(\mathcal{H}\).

\item The collection \(\mathcal{H}\) has a finite VC-dimension.
\end{enumerate}
\end{theorem}

\begin{theorem}[Sample complexity estimation]
\label{thm:finite-vc-dim-sample-complexity-estimation}
Let \(\mathcal{H}\) be a hypothesis class of maps \(\mathcal{X} \to \{0, 1\}\) together with the
binary loss function, and having a finite \(\vcdim \mathcal{H} = d\). There exists
constants \(c_1, c_2 \in \R\) such that:
\begin{enumerate}[(a)]\setlength\itemsep{0em}
\item \(\mathcal{H}\) has the uniform convergence property and sample complexity:
  \[
  c_1 \frac{d + \log(1/\delta)}{\varepsilon^2} \leq
  m_{\mathcal{H}}^{\text{UC}}(\varepsilon, \delta) \leq
  c_2 \frac{d + \log(1/\delta)}{\varepsilon^2}.
  \]

\item \(\mathcal{H}\) is agnostic PAC learnable and has sample complexity:
  \[
  c_1 \frac{d + \log(1/\delta)}{\varepsilon^2} \leq
  m_{\mathcal{H}}^{\text{UC}}(\varepsilon, \delta) \leq
  c_2 \frac{d + \log(1/\delta)}{\varepsilon^2}.
  \]

\item \(\mathcal{H}\) is PAC learnable with sample complexity of:
  \[
  c_1 \frac{d + \log(1/\delta)}{\varepsilon} \leq
  m_{\mathcal{H}}(\varepsilon, \delta) \leq
  c_2 \frac{d \log(1/\varepsilon) + \log(1/\delta)}{\varepsilon}.
  \]
\end{enumerate}
\end{theorem}

\todo[inline]{Write proof}

\section{Non-Uniform Learnability}

\begin{definition}[Competitive hypothesis]
\label{def:competitive-hypothesis}
Let \(\mathcal{H}\) be a hypothesis class, and let \(h, h' \in \mathcal{H}\) be a pair of predictors,
and consider parameters \(\varepsilon, \delta \in (0, 1)\). We say that \(h\) is
\((\varepsilon, \delta)\)-competitive with respect to \(h'\) if with a probability of at least
\(1 - \delta\) we have
\[
L_{\mathcal{D}}(h) \leq L_{\mathcal{D}}(h') + \varepsilon.
\]
\end{definition}

\begin{definition}
\label{def:non-uniform-learnable-hypothesis-class}
We say that a hypothesis class \(\mathcal{H}\) is \emph{non-uniformly learnable} if there
exists an algorithm \(A\) and a map
\[
m_{\mathcal{H}}^{\text{NUL}}: (0, 1)^2 \times \mathcal{H} \longrightarrow \N
\]
such that, for every triple \((\varepsilon, \delta, h) \in (0, 1)^2 \times \mathcal{H}\) the resulting predictor
\(A(S)\) is \((\varepsilon, \delta)\)-competitive with respect to \(h\). That is, with a
probability of at least \(1 - \delta\) over the choice of sample set \(S \sim \mathcal{D}^m\) we have
\[
L_{\mathcal{D}} A(S) \leq L_{\mathcal{D}}(h) + \varepsilon.
\]
\end{definition}

\begin{lemma}
\label{lem:countable-unif-conv-prop--then-non-unif-learnable}
Let \(\mathcal{H} = \bigcup_{n \in \N} \mathcal{H}_n\) be a hypothesis class. If each \(\mathcal{H}_n\) has the
uniform convergence property, then \(\mathcal{H}\) is non-uniformly learnable.
\end{lemma}

\begin{theorem}
\label{thm:non-uniformly-learnable-iff-countable-union-pac}
Let \(\mathcal{H}\) be a hypothesis class of maps of the form
\(\mathcal{X} \to \{0, 1\}\). The class \(\mathcal{H}\) is non-uniformly learnable if and only if it
consists of a countable union of agnostic PAC learnable hypothesis classes.
\end{theorem}



%%% Local Variables:
%%% mode: latex
%%% TeX-master: "../../deep-dive"
%%% End:

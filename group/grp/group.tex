\section{Group definitions}

Lets recall \cref{def: group} and demystify it in the following definition.

\begin{definition}[Group]
  A group \(G\) is a groupoid \(\cat G\) with one object \(*\). The elements of
  the group are the morphisms \(\Aut_{\cat G}(*)\). From axioms of \cref{def:
  category}:
  \begin{enumerate}[(G1)]
    \item An identity element \(e = \Id_*\).
    \item An associative binary operation \(G \times G \to G\), commonly denoted
      by juxtaposition.
    \item For all \(g \in G\) there exists an inverse element \(g^{-1} \in G\)
      such that \(g g^{-1} = e = g^{-1} g\).
  \end{enumerate}
\end{definition}

\begin{proposition}
  The identity element of a group is unique.
\end{proposition}

\begin{proof}
  See \cref{cor: unique identity}.
\end{proof}

\begin{proposition}
  The inverse of an element of a group is unique.
\end{proposition}

\begin{proof}
  See \cref{prop: iso unique inverse}.
\end{proof}

\begin{proposition}
  Let \(G\) be a group and \(g, h \in G\), then \((g h)^{-1} = h^{-1} g^{-1}\).
\end{proposition}

\begin{proof}
  \((h^{-1} g^{-1})(g h) = h^{-1} (g^{-1} g) h = h^{-1} e h = h^{-1} h = e\)
  hence \(h^{-1} g^{-1} = (g h)^{-1}\).
\end{proof}

\begin{proposition}[Cancellation]
  Let \(G\) be a group and elements \(a, b, c \in G\). If \(a c = b c\) then
  \(a = b\).
\end{proposition}

\begin{proof}
  Notice that \(a = a e = (a c) c^{-1} = (b c) c^{-1} = b e = b\).
\end{proof}

\begin{proposition}
  Let \(G\) be a group. If \(g \in G\), then the collection \(\{g h: h \in G\}\)
  is equal to \(G\).
\end{proposition}

\begin{proof}
  Denote \(G' := \{g h: h \in G\}\). Clearly we have \(G' \subseteq G\). On the
  other hand, if \(\ell \in G\), the element \(g (g^{-1} \ell) = e \ell = \ell
  \in G'\), hence \(G \subseteq G'\). Thus \(G = G'\).
\end{proof}

\begin{definition}[Commutative group]
  A group \(G\) is said to be commutative (or abelian) if for all \(g, h \in G\)
  we have \(g h = h g\).
\end{definition}

\begin{corollary}
  Let \(G\) be a group such that for all \(g \in G\), \(g^2 = e\). Then \(G\) is
  a commutative group.
\end{corollary}

\begin{proof}
  Let \(g, h \in G\) be any elements, then \((g h) (h g) = g h^2 g = g e g = g^2
  = e = (g h) (g h)\) and from cancellation law we find that \(g h = h g\).
\end{proof}

\subsection{Orders}

\begin{definition}[Order of an element]\label{def: group elem order}
  Let \(G\) be a group and \(g \in G\) be any element. We say that \(g\) has
  finite order if there exists \(n \in \Z_{> 0}\) such that \(g^n = e\).
  The order \(|g|\) of the element \(g\) is defined as the smallest such
  positive integer. If \(g\) does not have a finite order, it is common to write
   \(|g| = \infty\).
\end{definition}

\begin{lemma}\label{lem: order and multiples}
  Let \(G\) be a group and \(g \in G\) be an element with finite order. Then
  \(g^n = e\) for some \(n \in \Z_{> 0}\) if and only if \(|g|\) divides
  \(n\).
\end{lemma}

\begin{proof}
  Since \(|g| \leq n\), define \(m \in \Z_{> 0}\) such that \(n - m |g|
  \geq 0\) and \(n - (m + 1) |g| < 0\). Define \(r = n - m |g|\) to be the
  remainder, hence \(r < |g|\). Our goal is to show that \(r = 0\). Notice that
  \(g^r = g^n g^{-|g| m} = e e^{-m} = e\), which can only be the case for \(r =
  0\), since \(|g|\) is defined to be the least positive integer such that
  \(g^{|g|} = e\). This proves that \(m |g| = n\).

  For the second part, suppose \(n\) is a multiple of \(|g|\) and denote it by
  \(m|g| = n\). Then \(g^{m |g|} = e^m = e\).
\end{proof}

\begin{definition}[Order of a group]
  Let \(G\) group of finite number of elements. We define the order of \(G\) to
  be the number of its elements and denote it by \(|G|\). If \(G\) is an
  infinite group, then \(|G| = \infty\).
\end{definition}

\subsubsection{Order of Products}

\begin{proposition}[Order of the power]\label{prop: order of the power}
  Let \(G\) be a group and \(g \in G\) be an element with finite order. Then for
  all \(m \in \Z_{\geq 0}\) the element \(g^m\) has finite order. For
  \(m = 0\) we have \(|g^m| = 1\), for \(m > 0\) we have
  \[
    |g^m| = \frac{\operatorname{lcm}(m, |g|)}{m} =
    \frac{|g|}{\operatorname{gcd}(m, |g|)}.
  \]
\end{proposition}

\begin{proof}
  From divisibility arguments, we have \(\operatorname{lcm}(a, b)
  \operatorname{gcd}(a, b) = ab\) for integers \(a\) and \(b\), hence the second
  equality is justified. We prove the equality \(|g^m| =
  \frac{\operatorname{lcm}(m, |g|)} m\). For the sake of notation, let \(d :=
  |g^m|\). Notice that \(g^{m d} = e\) and hence \(|g|\) divides \(m d\). Since
  \(d\) is the least positive integer with such property, it follows that \(m
  d\) is the least common multiple of \(m\) and \(|g|\). Therefore \(m |g^m| =
  \operatorname{lcm}(m, |g|)\), which proves the equation.
\end{proof}

\begin{proposition}\label{prop: order-prod-commutes}
  Let \(G\) be a group, then for any \(g, h \in G\) we have \(|gh| = |hg|\).
\end{proposition}

\begin{proof}
  First, let \(x, y \in G\) be any elements, we prove that \(|y x y^{-1}| =
  |x|\). Notice that for any \(n \geq 1\) we have \((y x y^{-1})^n = y x^n
  y^{-1}\), hence the least element that annihilates the product \(y x y^{-1}\)
  is the order of \(x\) --- that is, \(|y x y^{-1}| = x\). In particular, \(hg =
  g^{-1} gh g\), hence \(|hg| = |g^{-1} (gh) g| = |gh|\).
\end{proof}

\begin{proposition}\label{prop: commutative-order-of-prod}
  Let \(G\) be a group and \(g, h \in G\) be such that \(g h = h g\). Then \(|g
  h|\) divides \(\operatorname{lcm}(|g|, |h|)\).
\end{proposition}

\begin{proof}
  Let \(n\) be a common multiple of \(|g|\) and \(|h|\), then \(g^n = h^n = e\)
  from \cref{lem: order and multiples}. Notice that the commutative property \(g
  h = h g\) allow us to permute the terms of \(g^n h^n = e\) in order to obtain
  \(g^n h^n = (g h)^n = e\). In particular, since \(\operatorname{lcm}(|g|,
  |h|)\) is a common multiple of \(|g|\) and \(|h|\), we find that \((g
  h)^{\operatorname{lcm}(|g|, |h|)} = e\) and again from \cref{lem: order and
  multiples} we find that \(|g h|\) divides \(\operatorname{lcm}(|g|, |h|)\)
\end{proof}

\begin{lemma}\label{lem: ord-prod-rel-prime}
  Let \(G\) be a group and \(g, h \in G\) commute --- \(gh = hg\). If
  \(\operatorname{gcd}(|g|, |h|) = 1\), then \(|gh| = |g| |h|\).
\end{lemma}

\begin{proof}
  Let \(|gh| = \ell, |g| = m\) and \(|h| = n\). From \cref{prop:
  commutative-order-of-prod}, we have that \(\ell \mid \operatorname{lcm}(m,
  n)\), and since \(m n = \operatorname{lcm}(m, n) \operatorname{gcd}(m, n) =
  \operatorname{lcm}(m, n)\), then \(\ell \mid mn\), which implies that \(\ell
  \leq mn\). Moreover, since the elements commute, \((g h)^\ell = g^\ell h^\ell
  = e\) hence \(g^\ell = (h^\ell)^{-1}\). From \cref{prop: order of the power},
  \(|g^\ell| = \frac{|g|}{\operatorname{gcd}(\ell, |g|)} = \frac m m = 1 \) and
  equivalently for \(h\) we have \(|h^\ell| = 1\). This shows that \(g^\ell =
  h^\ell = e\) and therefore both \(m\) and \(n\) divide \(\ell\), hence so does
  the product \(mn\), thus \(mn \leq \ell\). This completes the proof that
  \(\ell = mn\).
\end{proof}

\begin{definition}[Maximal finite order]\label{def: maximal-finite-order}
  Let \(G\) be a group. An element \(g \in G\) is said to be of maximal finite
  order if its order is finite and for all \(h \in G\) with finite order, we
  have \(|h| \leq |g|\).
\end{definition}

\begin{proposition}
  Let \(G\) be a commutative group and \(g \in G\) be of maximal finite order.
  If \(h \in G\) has finite order, then \(|h|\) divides \(|g|\).
\end{proposition}

\begin{proof}
  Define the notation \(|g| = m\) and \(|h| = n\). Let \(P = (p_j)_j\) be a
  finite sequence containing all primes such that less than or equal to \(m\)
  and define a finite sequence of integers of same length \(A = (a_j)_j\) such
  that \(m = \prod_j p_j^{a_j}\). Since \(n \leq m\) it follows that there also
  exists a finite sequence of integers \(B = (b_j)_j\) such that \(n = \prod_j
  p_j^{b_j}\). Suppose for the sake of contradiction that \(n\) doesn't divide
  \(m\), so that there exists an index \(k\) such that \(a_{k} < b_{k}\)
  --- from the fact that \(\frac m n = \prod_j p_j^{a_j - b_j}\). Consider now
  the order --- following from \cref{lem: ord-prod-rel-prime}:
  \[
    |g^{\left( p_k^{a_k} \right)} h^{\left(n / p_k^{b_k}\right)}|
    = |g^{\left( p_k^{a_k} \right)}| |h^{\left(n / p_k^{b_k}\right)}|
    = \frac m {\operatorname{gcd}(m, p_k^{a_k})}
    \frac n {\operatorname{gcd}(n, n / p_k^{b_k})}
    = \frac m {p_k^{a_k}} \frac n {n / p_k^{b_k}}
    = m p_k^{b_k - a_k}
  \]
  and since \(b_k > a_k\), it follows that \(m p_k^{b_k - a_k} > m\) and hence
  there is a contradiction since we assumed that \(m\) was the the maximal
  finite order of the group. We conclude that there does not exist \(k\) for
  which \(b_k\) is less than \(a_k\) --- thus \(n\) divides \(m\).
\end{proof}

\subsubsection{Finite Groups and Elements of Order 2}

\begin{lemma}[Order 2 elements implies commutative]\label{lem: order2-commutes}
  Let \(G\) be a group such that all non-identity elements have order \(2\).
  Then \(G\) is commutative.
\end{lemma}

\begin{proof}
  Let \(g, h \in G\), notice that \(|gh| = 2\) from \cref{prop:
  commutative-order-of-prod} hence \(gh \in G\), then \((gh)^2 = e\) and
  therefore \(gh = g^{-1} h^{-1} = g^{-1}(ghgh)h^{-1} = hg\) commutes.
\end{proof}

\begin{lemma}
  Let \(G\) be a finite group such that any element has order at most \(2\). Let
  \(H\) be any subgroup of \(G\), for any \(t \in G \setminus H\) consider the
  collection \(T = H \cup \{h t : h \in H\}\), then \(T\) is a subgroup of \(G\)
  with \(|T| = 2|H|\).
\end{lemma}

\begin{proof}
  Since \(t \not\in H\), any element \(ht \in T\) with \(h \in H\) is such that
  \(ht \not\in H\). The number of distinct elements of the form is \(|H|\) ---
  one for each \(h \in H\) ---, hence \(|T| = |H| + |H| = 2|H|\).
\end{proof}

\begin{lemma}[Order \(2^n\)]\label{lem: order-2n}
  Let \(G\) be a finite group such that any element has order at most \(2\). The
  order of \(G\) is of the form \(2^n\) for some \(n \geq 0\). Moreover, if
  \(|G| > 1\), then there exists a subgroup \(H\) of \(G\) with order \(|H| =
  2^{n-1}\).
\end{lemma}

\begin{proof}
  We create a recursive algorithm to find the collection of elements of \(G\)
  that generate any other element contained in \(G\):
  \begin{enumerate}
    \item (Base case) If \(G = H_j\), return \(H_j\).
    \item (Recursion) Let \(g \in G \setminus H_j\) and construct \(H_{j + 1} =
      H_j \cup \{h g: h \in H_j\}\), recursively call the algorithm with \(H_{j
      + 1}\).
  \end{enumerate}
  Such algorithm is certain to terminate since \(|G|\) is finite. Notice that
  the second step always doubles the order of the set \(H_j\), so that --- if
  \(H_n\) is the result of the algorithm --- then \(|H_n| = 2^n\). This shows
  that \(|G| = |H_n| = 2^n\). The second part of the statement follows
  immediately from the construction of the algorithm.
\end{proof}

\begin{proposition}
  Let \(G\) be a commutative group, if there exists exactly one element of order
  \(2\) --- say \(f \in G\) --- then the product of all elements of the group is
  \[
    \prod_{g \in G} g = f.
  \]
  Otherwise, we have \(\prod_{g \in G} g = e\).
\end{proposition}

\begin{proof}
  Since \(G\) is finite, let \(G = \{e, g_1, g_1^{-1}, \dots, g_n, g_n^{-1}\}\).
  Suppose there exists one and only one element of order \(2\) and denote it by
  \(f\) --- that is, \(f = f^{-1}\). Since the group is commutative, we can
  rearrange the product of elements so that we can take the pairwise product of
  each element and its inverse --- this being possible only in the case where
  \(g \neq g^{-1}\) ---, taking the product we find
  \[
    \prod_{g \in G} g = e (g_1 g_1^{-1}) \dots (g_j g_j^{-1}) f \dots (g_{j + 2}
    g_{j + 2}^{-1}) (g_n g_n^{-1}) = e^{j + 1} f e^{n - (j + 1)} = f
  \]
  since \(f\) has no inverse in \(G \setminus \{f\}\).

  Let there be no element with order \(2\) in \(G\), then the rearrangement of
  pairwise element and respective inverse is possible with each \(g \in G\),
  making
  \[
    \prod_{g \in G} g = e (g_1 g_1^{-1}) \dots (g_j g_j^{-1}) \dots (g_n
    g_n^{-1}) = e^{n + 1} = e.
  \]

  Suppose there exists \(m > 1\) distinct elements in \(G\) with order \(2\) and
  define \(T = \{f \in G: |f| \leq 2\}\) and \(S = G \setminus T\). From the
  last case we know that \(\prod_{s \in S} s = e\). Note that \(T\) forms a
  subgroup of \(G\):
  \begin{itemize}
    \item \(e \in T\).
    \item If \(g, h \in T\), then from \cref{prop: commutative-order-of-prod}
      \(|gh|\) divides \(\operatorname{lcm}(|g|, |h|) \leq 2\) hence \(|gh| \leq
      2\) and \(gh \in T\).
    \item Since \(g \in T\) implies \(g g = e\) then \(g = g^{-1}\) and
      \(|g^{-1}| = 2\), so \(g^{-1} \in T\).
  \end{itemize}
  Moreover, the order of the group \(T\) has to be of the form \(2^k\) for some
  \(k \geq 2\), from \cref{lem: order-2n}. From the same lemma, choose a
  subgroup \(H\) with order \(2^{k-1}\) and take \(u \in T \setminus H\), so
  that \(T = H \cup \{hu: h \in H\}\) --- such \(u\) exists from the algorithm
  constructed in the lemma. Since \(T\) is commutative (see \cref{lem:
  order2-commutes}), we can write
  \[
    \prod_{t \in T} t = \prod_{h \in H} h \prod_{h \in H} hu = \prod_{h \in H} u
    h^2 = \prod_{h \in H} u = u^{2^{k-1}} = (u^2)^{2^{k-2}} = e.
  \]
  Therefore we can finally conclude that
  \[
    \prod_{g \in G} g = \prod_{s \in S} s \prod_{t \in T} t = e.
  \]
\end{proof}

\section{Examples of Groups}

\begin{definition}[Symmetric groups]\label{def: sym-group}
  Let \(A \in \Set\). The symmetric group of \(A\) --- also referred to as the
  permutation group of \(A\) ---, \(\Aut_\Set(A)\), is denoted by \(\mathcal
  S_A\). The symmetric group of the range set \(\{1, \dots, n\}\) is denoted
  \(\mathcal S_n\).
\end{definition}

\begin{notation}[Permutations]
  A permutation \(\sigma \in \mathcal S_n\) is denoted by a table of the form
  \[
    \sigma =
    \begin{pmatrix}
      1 &2 &\dots &n-1 &n \\
      \sigma(1) &\sigma(2) &\dots &\sigma(n-1) &\sigma(n)
    \end{pmatrix}
  \]
\end{notation}

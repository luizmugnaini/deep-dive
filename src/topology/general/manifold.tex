\section{Topological Manifolds}

\subsection{Locally Euclidean}

\begin{definition}[Locally euclidean]\label{def: locally euclidean}
Let \(M\) be a topological space. We say that \(M \) is locally euclidean of
dimension \(n\) if for all points \(x \in X\) there exists a neighbourhood
\(U_x \subseteq M\) of \(x\) such that exists an open set \(V \subseteq
\R^n\) for which \(U_x \iso V\), that is, \(U_x\) is isomorphic to
\(V\).
\end{definition}

\begin{proposition}\label{prop: open ball homeomorphic to Rn}
Let \(B^n \subseteq \R^n\) be any open ball, then the morphism
\[
  \varphi: B^n \to \R^n,\ x \longmapsto \frac{x}{1 - |x|}
\]
is an isomorphism \(B^n \iso \R^n\).
\end{proposition}

\begin{lemma}\label{lem:locally-euclidean-equivalences}
A topological space \(M\) is locally euclidean of dimension \(n\) if and only
if either of the following is true:
\begin{enumerate}[(a)]
  \item Every point of \(M\) has a neighbourhood in \(M\) that is isomorphic
    to an open ball in \(\R^n\).
  \item Every point of \(M\) has a neighbourhood in \(M\) that is isomorphic
    to \(\R^n\).
\end{enumerate}
\end{lemma}

\begin{proof}
From \cref{prop: open ball homeomorphic to Rn}, we find that the proof of one
of the two statements proves the other. Lets prove the first one of them.
(\(\Leftarrow\)) Since \(B^n \subseteq \R^n\) is an open of
\(\R^n\) we conclude from the definition \cref{def: locally euclidean}
that \(M\) is indeed locally euclidean.
(\(\Rightarrow\)) Suppose \(M\) is a locally euclidean \(n\)-dimensional
topological space, then let any point \(p \in M\) and define \(U_p \subseteq
M\) be a neighbourhood of \(p\) such that exists \(V \subseteq \R^n\)
for which we can define an isomorphism \(\phi : U_p \to V\). From the fact
that the collection of open balls in \(\R^n\) form a basis for
\(\R^n\), we conclude that there exists an open ball \(B^n \subseteq
V\) for which \(p \in B^n\). Thus the open set \(\psi^{-1}(B) \subseteq M\) is
a neighbourhood of \(p\). Since \(\phi\) is an isomorphism, then the
morphism \(\psi|_B^{-1} : B \to \phi^{-1}(B)\) is an isomorphism, thus \(B
\iso \phi^{-1}(B)\) as wanted.
\end{proof}

\begin{proposition}
\label{prop:locally-euclidean-first-countable}
Every locally euclidean space is first countable.
\end{proposition}

\todo[inline]{prove}

\begin{definition}[Miscelaneous]
Let \(M\) be a locally euclidean \(n\)-dimensional topological space. We
define:
\begin{enumerate}[(C1)]
  \item\label{def: coordinate domain}
    (Coordinate domain) An open set \(U \subseteq M\) is called a coordinate
    domain of \(M\) if it is isomorphic to an open set of \(\R^n\).
  \item\label{def: coordinate map}
    (Coordinate map) An isomorphism \(\phi\) from a coordinate domain to an
    open set of \(\R^n\) is called a coordinate map.
  \item\label{def: coordinate chart}
    (Coordinate chart) The pair \((U, \phi)\) of a coordinate domain and one
    of its coordinate maps is called a coordinate chart for \(M\).
  \item\label{def: coordinate ball}
    (Coordinate ball) A coordinate domain isomorphic to an open ball of
    \(\R^n\) is called a coordinate ball.
  \item\label{def: euclidean neighbourhood}
    (Coordinate neighbourhood or euclidean neighbourhood) Given a point \(p
    \in M\), if \(U \subseteq M\) is a coordinate domain of \(M\) such that
    \(p \in U\), then we say that \(U\) is a coordinate neighbourhood of
    \(p\).
\end{enumerate}
\end{definition}

\begin{proposition}
\label{prop:coordinate-ball-second-countable}
Every coordinate ball is second countable.
\end{proposition}

\begin{proof}
Let \(M\) be a locally euclidiean \(n\)-dimensional topological space and let
\(U \subseteq M\) be a coordinate ball of \(M\) and \(\phi: U \isoto B^n\) be an
isomorphism. From \cref{cor:euclidean-space-second-countable} we find that there
exists a countable basis \(\mathcal B \subseteq 2^{\R^n}\) for \(B^n\). Since
\(\phi\) is an isomorphism, then the collection of preimages \(\{\phi^{-1}(B)
\subseteq U : B \in \mathcal B\}\) is also a basis for \(U\) and is clearly
countable --- hence \(U\) is second countable.
\end{proof}

\begin{proposition}[Locally euclidean from surjective morphism]
\label{prop:locally-euclidean-from-surjective-map}
Let \(X\) be locally euclidean of dimension \(n\), and \(f: X \to Y\) be a
surjective local isomorphism. Then \(Y\) is locally euclidean of dimension
\(n\).
\end{proposition}

\begin{proof}
Let \(\mathcal B\) be a basis for \(X\). Since \(f\) is continuous, surjective
and a open (see \cref{prop:properties-local-homeomorphism}), we can use
\cref{prop: basis image surjective} to conclude that \(f(\mathcal B)\) is a
basis for \(Y\). Since \(f\) is a local isomorphism, given any point \(y \in
Y\), choose a neighbourhood \(V_y \in f(\mathcal B)\), so that there exists a
\(B \in \mathcal B\) such that \(f(B) = V_y\). Consider the isomorphism
\(f|_B: B \to V_y\). Since \(X\) is locally euclidean of dimension \(n\), there
exists an open set \(W \subseteq \R^n\) such that \(B \iso W\). Moreover, since
\(B \iso V_y\), then \(V_y \iso W\). We conclude that \(Y\) is locally euclidean
of dimension \(n\).
\end{proof}

\subsection{Topological Manifold}

\begin{definition}[Topological manifold]\label{def: topological manifold}
An \(n\)-dimensional topological manifold is a second countable Hausdorff
space that is locally euclidean \(n\)-dimensional.
\end{definition}

\begin{proposition}\label{prop:coordinate-ball-basis}
Every topological manifold admits a basis of coordinate balls.
\end{proposition}

\begin{proof}
Let \(M\) be a \(n\)-dimensional topological manifold. From
\cref{lem:locally-euclidean-equivalences} we can take, for every point \(p \in
M\), a neighbourhood \(U \subseteq M\) such that \(U \iso B^n\) (an open ball of
\(\R^n\)). Define \(\mathcal{U}\) to be the collection of all coordinate balls
on \(M\), from our last argument it follows that \(\mathcal U\) covers
\(M\). Let \(U, U' \in \mathcal U\) be intersecting coordinate balls and
\(\phi\) be a coordinate isomorphism of either \(U\) or \(U'\). Let \(p \in U
\cap U'\) be any point and define a ball \(B_{p}^{n} \subseteq \phi(U
\cap U') \subseteq \R^n\) that is a neighbouhood of \(\phi(p) \in \R^n\).
Let \(V \subseteq M\) be defined as \(V \coloneq \phi^{-1}(B^n)\) so that \(V
\subseteq U \cap U'\) and also \(p \in V\). Notice that the induced map \(\psi:
V \to B^n\), defined by \(\psi(x) \coloneq \phi(x)\), is an isomorphism --- that
is, \(V \in \mathcal U\) and hence \(\mathcal U\) is a basis for \(M\) (see
\cref{prop:equivalent-basis}).
\end{proof}

\begin{proposition}
Every open subset of an \(n\)-manifold is itself an \(n\)-manifold.
\end{proposition}

\begin{proof}
Consider \(U \subseteq M\) an open set and let \(p \in U\). Then consider \(V
\subseteq M\) to be a coordinate neighbourhood of \(x\) such that \(V \iso B
\subseteq \R^n\), then the set \(U \cap V\) is open and isomorphic
to a subset of \(B\) and hence \(V\) is \(n\)-dimensional locally euclidean.
Moreover, since a subset of a Hausdorff space is Hausdorff and a subset of a
second countable space is second countable, then \(U\) is indeed a
\(n\)-manifold.
\end{proof}

\begin{definition}
The empty topological space is an \(n\)-manifold for all \(n > 0\).
\end{definition}

\begin{theorem}[Dimension invariance]\label{def: manifold dimension invariance}
If \(m \neq n\), then a nonempty topological space cannot be both
\(n\)-manifold and \(m\)-manifold.
\end{theorem}

\begin{proposition}
A separable metric space that is locally euclidean of dimension \(n\) is an
\(n\)-manifold.
\end{proposition}

\begin{proof}
From \cref{prop: metric space properties}, since \(M\) is separable then it is
also second countable. Moreover, from \cref{prop: metric space T2} we find
that \(M\) is Hausdorff. Hence \(M\) is an \(n\)-manifold.
\end{proof}

\begin{proposition}
Every topological manifold is separable and metrizable.
\end{proposition}

\begin{proof}
Notice that since a manifold is second countable, then by \cref{prop: second
countable properties} we find that it is separable.
\todo[inline]{After proving Urysohn metrization theorem, prove the
metrizability property}
\end{proof}

\subsection{Manifolds with Boundary}

\begin{definition}[Upper half-space]\label{def:upper-half-space}
We define the closed \(n\)-dimensional upper half-space \(\Uhs^n \subseteq
\R^n\) as
\[
  \Uhs^n \coloneq \{x \in \R^n : \pi_n(x) \geq 0\}
\]
where \(\pi_n\) is the projection of the \(n\)-th coordinate. We define the
boundary of \(\Uhs^n\) as \(\partial \Uhs^n \coloneqq \{x : \pi_n(x) = 0\}\),
and the interior as \(\Int(\Uhs^n) \coloneqq \{x : \pi_n(x) > 0\}\).
\end{definition}

\begin{definition}[Manifold with boundary]
\label{def: manifold with boundary}
We define an \(n\)-dimensional topological manifold with boundary to be a
second countable Hausdorff space such that each point has a neighbourhood
isomorphic to an open set of \(\R^n\) or \(\Uhs^n\).
\end{definition}

\begin{definition}[Miscelaneous]
Let \(M\) be an \(n\)-manifold with boundary. We define
\begin{enumerate}[(MB1)]
  \item A coordinate chart for \(M\) is a pair \((U, \phi)\), where \(U
    \subseteq M\) is an open set and \(\phi: U \isoto V\) is an isomorphism,
    where \(V \subseteq \R^n\) or \(V \subseteq \Uhs^n\). We say that the chart
    is an interior chart if \(V\) is an open subset of \(\R^n\). A chart is said
    to be a boundary chart if \(V\) is an open subset of \(\Uhs^n\) with
    \(\im(\phi) \cap \partial \Uhs^n \neq \emptyset\).
  \item A point \(p \in M\) is called an interior point of \(M\) if it is
    contained in the domain of an interior chart. The collection of such
    points is called the interior of \(M\), and is denoted \(\Int(M)\).
  \item A point \(p \in M\) is called a boundary point of \(M\) if it is in
    the domain of a boundary chart that maps \(p\) to a point of
    \(\Uhs^n\). The boundary of \(M\), is defined as the collection of
    such points and is denoted by \(\partial M\).
\end{enumerate}
\end{definition}

\begin{proposition}\label{prop: interior is a manifold}
If \(M\) is an \(n\)-dimensional manifold with boundary, then \(\Int(M)\) is
an open subset of \(M\), which is itself an \(n\)-dimensional manifold
(without boundary).
\end{proposition}

\begin{proof}
Let \(p \in \Int(M)\) be any point, then by definition we have that \(p \in
U\) where \((U, \phi)\) is an interior chart for \(M\) and therefore \(U \iso
V\) where \(V\) is some open subset of \(\R^n\), hence locally
euclidean, which makes \(\Int(M)\) an \(n\)-dimensional manifold (since a
subset of a Hausdorff space is Hausdorff and a subset of a second countable
space is again second countable). To prove that \(\Int(M)\) is open, we can
use the fact that \(M\) is first countable and lemma \cref{lem: sequence
lemma}. First, suppose the converse, so that \((x_i)_{i \in \N}\) is a
sequence of points in \(M\) that is not-eventually in \(\Int(M)\) and, for the
sake of contradiction, \(x_i \to p\). Consider the open neighbourhood \(U\) of
\(p\) (which happens to be the chart domain), if \(x_i \to p\) then \(\exists
N \in \N: x_i \in U\) for all \(i \geq N\), but notice that every
point of \(U\) is an interior point of \(M\), therefore \(x_i \in \Int(M)\),
which is a contradiction to the hypothesis that the sequence is not-eventually
in \(\Int(M)\). Hence such sequences cannot converge to \(p\), and \(\Int(M)\)
is open.
\end{proof}

\begin{theorem}[Boundary invariance]\label{thm: boundary invariance}
Let \(M\) be a manifold with boundary, then
\[
  \partial M \cap \Int(M) = \emptyset
\]
\end{theorem}

\begin{proof}
\todo[inline]{Proof to come far ahead}
\end{proof}

\begin{corollary}
If \(M\) is a nonempty \(n\)-manifold with boundary, then the collection
\(\partial M\) is closed in \(M\), and \(M\) is an \(n\)-manifold if and only
if \(\partial M = \emptyset\).
\end{corollary}

\begin{proof}
Since \(\partial M = M \setminus \Int(M)\) from theorem \cref{thm: boundary
invariance} and \(\Int(M)\) is open from \cref{prop: interior is a manifold},
then it follows that \(\partial M\) is closed. (\(\Rightarrow\)) Moreover,
suppose that \(M\) is a manifold, then given any point \(p \in M\) there
exists an interior chart \((U, \phi)\) such that \(p \in U \subseteq
\Int(M)\), hence \(\Int(M) = M\), and from \cref{thm: boundary invariance} we
conclude that \(\partial M = \emptyset\). (\(\Leftarrow\)) Suppose that
\(\partial M = \emptyset\), then from \cref{thm: boundary invariance} we
conclude that \(M = \Int(M)\), which is a manifold by \cref{prop: interior is a
manifold}.
\end{proof}

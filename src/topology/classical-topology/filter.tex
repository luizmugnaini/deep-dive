\section{Filters and Nets}

\begin{definition}[Filter on a set]\label{def: filter}
    A filter on a set \(X\) is a collection \(\mathcal F \subseteq 2^X\) that
    satisfies the following properties
    \begin{enumerate}[(F1)]
        \item\label{def: filter F1}
        (Downward directed) Given \(A, B \in \mathcal F\), then there exists
        \(C \in \mathcal F\) such that \(C \subseteq A \cap B\).
        \item\label{def: filter F2}
        (Upward closed) If \(A \in \mathcal F\) and \(A \subseteq B\) then \(B
        \in \mathcal F\).
        \item\label{def: filter F3}
        \(\mathcal F\) is non-empty.
    \end{enumerate}
\end{definition}

\begin{definition}[Proper filter]
    If \(\mathcal F\) is a filter of \(X\) such that there exists \(A \subseteq
    X\) for which \(A \not\in \mathcal F\), then we say that \(\mathcal F\) is a
    proper filter. This is equivalent of saying that \(\emptyset \not\in \mathcal
    F\).
\end{definition}

\begin{proposition}[Eventuality filter]
    Let \(X\) be a topological space and \({(x_i)}_{i \in \N}\) be a
    sequence of points of \(X\). The collection of sets such that \((x_i)\) is
    eventually in, explicitly
    \[
        \mathcal E_{(x_i)} = \{U \subseteq X \colon \forall i \geq N, x_i \in U (N \in
        \N)\},
    \]
    is a filter on \(X\).
\end{proposition}

\begin{definition}[Filter base]
    A non-empty downward directed set is called a filter base.
\end{definition}

\begin{proposition}
    Given a filter base \(\mathcal G\) on \(X\), we define the filter generated by
    \(\mathcal G\) as the collection
    \[
        \mathcal G^\uparrow = \{A \subseteq X \colon G \subseteq A (G \in \mathcal G)\}.
    \]
\end{proposition}

\begin{proof}
    Since \(\mathcal G\) is non-empty, \(\mathcal G^\uparrow\) is non-empty. Let
    \(A, B \in \mathcal G^\uparrow\), then there are \(G, H \in \mathcal G\) such
    that \(G \subseteq A\) and \(H \subseteq B\). Since \(G \cap H \in \mathcal
    G\), then \(G \cap H \subseteq A \cap B\) and thus \(A \cap B \in \mathcal
    G^\uparrow\). Suppose that \(A \subseteq C\) where \(C \subseteq X\), then
    since \(G \subseteq A\), it follows that \(G \subseteq C\) and hence \(C \in
    \mathcal G^\uparrow\). Therefore \(\mathcal G^\uparrow\) satisfies all of the
    three requirements.
\end{proof}

\begin{proposition}
    Let \(X\) be a topological space. The base at a point \(p \in X\), namely
    \(\mathcal B_p\), is a filter base.
\end{proposition}

\begin{proof}
    Let \(U, V \in \mathcal B_p\) be neighbourhoods of \(p\). Since \(p \in U \cap
    V\) and the finite intersection of open sets is open, then \(U \cap V\) is a
    neighbourhood of \(p\) and hence \(U \cap V \in \mathcal B_p\).
\end{proof}

\begin{definition}[Convergence of filters]
    \label{def: convergence of filters}
    Let \(\mathcal F\) be a filter on a topological space \((X, \tau)\). We
    say that \(\mathcal F\) converges to \(x\), \(\mathcal F \to x\), if and only
    if \(\mathcal B_x \subseteq \mathcal F\).
\end{definition}

\subsection{Filters determine Hausdorff, closure and continuity}

\begin{proposition}[Hausdorff]\label{prop: hausdorff from filter}
    A topological space is Hausdorff if and only if limits of convergent proper
    filters are unique.
\end{proposition}

\begin{proof}
    (\(\Rightarrow\)) Let \(X\) be a Hausdorff space and \(\mathcal F\) be a
    proper filter on \(X\). Suppose that for distinct \(x, y \in X\) we have
    \(\mathcal F \to x\) and also \(\mathcal F \to y\). Since \(X\) is Hausdorff,
    let \(U, V \subseteq X\) neighbourhoods of \(x\) and \(y\), respectively, and
    such that \(U \cap V\) is empty. Notice that since \(U\) is a neighbourhood of
    \(x\) then \(U \in \mathcal F\), and analogously, \(V \in \mathcal F\), but
    from the downward directness, it follows that \(U \cap V = \emptyset \in
    \mathcal F\), which can't be the case since \(\mathcal F\) is supposed to be
    proper. Hence \(\mathcal F\) cannot converge to distinct points of \(X\).
    (\(\Leftarrow\)) Suppose that \(X\) is not Hausdorff and choose distinct
    points \(x, y\) that are not separable by open sets. Consider the collection
    \(\mathcal B = \mathcal B_x \cap \mathcal B_y\), then given two sets \(A, B
    \in \mathcal B\) we have that \(A \cap B \in \mathcal B\) and therefore
    \(\mathcal B\) is a filter base. Notice that the filter \(\mathcal B^\uparrow\)
    converges both to \(x\) and \(y\).
\end{proof}

\begin{proposition}[Closed]\label{prop: closed from filter}
    Let \(X\) be a topological space and \(A \subseteq X\). A point \(p \in
    \Cl A\) if and only if there exists a proper filter \(\mathcal F\) with
    \(A \in \mathcal F\) such that \(\mathcal F \to p\).
\end{proposition}

\begin{proof}
    (\(\Rightarrow\)) Let \(p \in \Cl A\), then for all neighbourhoods \(U
    \in \mathcal B_p\) the set \(U \cap (A \setminus \{p\})\) is non-empty, in
    particular we have that \(\mathcal B \coloneq \{U \cap A \colon U \in \mathcal B_p\}\)
    does not contain the empty set. Hence the proper filter \(\mathcal
    B^\uparrow\)converges to \(p\). (\(\Leftarrow\)) Let \(\mathcal F\) be a
    proper filter with \(\mathcal F \to p\) and \(A \in \mathcal F\), then in
    particular we have that the downward directness implies \(\mathcal B \subseteq
    \mathcal F\) and thus \(\emptyset \not\in \mathcal B\), hence \(x \in
    \Cl A\).
\end{proof}

\begin{definition}[Pushforward of filters]\label{def: pushforward of filters}
    Let \(f: X \to Y\) be a map of sets. The collection of images \(\{f(A) \colon A \in
    \mathcal F\}\) form a filter base whose generated filter is defined to the be
    the pushforward of \(\mathcal F\) with respect to \(f\), namely
    \(f_\ast(\mathcal F)\). Hence
    \[
        f_\ast(\mathcal F) = \{B \subseteq Y \colon f(A) \subseteq B (A \in \mathcal
        F)\}.
    \]
\end{definition}

\begin{proposition}[Continuity]\label{prop: continuity from filter}
    A map \(f: X \to Y\) is continuous if and only if for every given filter
    \(\mathcal F\) on \(X\) such that \(\mathcal F \to x\), then \(f_\ast(\mathcal
    F) \to f(x)\), where \(x \in X\).
\end{proposition}

\begin{proof}
    (\(\Rightarrow\)) Suppose \(f\) is a continuous map and \(\mathcal F \to x\).
    From continuity of \(f\), for any given neighbourhood of \(f(x)\), \(B \in
    \mathcal B_{f(x)}\), there exists a corresponding neighbourhood of \(x\), \(V
    \in \mathcal B_x\), such that \(f^{-1}(B) \subseteq V\). Since
    \(\mathcal F \to x\) if and only if \(\mathcal B_x \subseteq \mathcal F\)
    (from \cref{def: convergence of filters}), then in particular \(f^{-1}(B) \in
    \mathcal F\), since \(f^{-1}(B) \in \mathcal B_x\), is a neighbourhood of
    \(x\). Hence, if \(B \in \mathcal B_{f(x)}\) is any element and \(A =
    f^{-1}(B)\), then \(f(A) \subseteq B\), which implies that \(\mathcal B_{f(x)}
    \subseteq f_\ast(\mathcal F)\) from the definition of the pushforward,
    therefore \(f_\ast(\mathcal F) \to f(x)\).

    (\(\Leftarrow\)) Suppose now that for any filter \(\mathcal F\) on \(X\) such
    that \(\mathcal F \to x\), implies \(f_\ast(\mathcal F) \to f(x)\). Given any
    open set \(U \subseteq Y\), if \(U \cap \im(f) = \emptyset\) then \(f^{-1}(U)
    = \emptyset\) and hence is open, otherwise there exists \(x \in X\) for which
    \(f(x) \in U\). Given such a point \(x \in X\), take \(\mathcal F = \mathcal
    B_x^\uparrow\) so that from hypothesis \(f_\ast(\mathcal F) \to f(x)\) and
    hence \(\mathcal B_{f(x)} \subseteq f_\ast(\mathcal F)\). Since \(U\) is open,
    then \(U \in \mathcal B_{f(x)}\), which in turn implies that \(U \in
    f_\ast(\mathcal F)\) and from definition there must exist a set \(V \in
    \mathcal F\), which happens to be a neighbourhood of \(x\), such that \(f(V)
    \subseteq B\), thus \(f\) is continuous.
\end{proof}

\todo[inline]{Write on nets later}

%%% Local Variables:
%%% TeX-master: "../../deep-dive"
%%% End:
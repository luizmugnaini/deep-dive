\documentclass[11pt,reqno]{amsart}
\usepackage[english]{babel}

% Page geometry
\usepackage{fullpage}

% Math stuff: do not mess with the ordering!
\usepackage{mathtools}
\usepackage{amsthm}
\usepackage{amssymb}
\usepackage{stmaryrd}
\usepackage{tikz}
\usepackage{tikz-cd}

% Font
\usepackage[no-math]{newpxtext} % last: mathpazo
\usepackage{newpxmath}
% Set arrow tip to that of newpxmath
\tikzset{>=Straight Barb, commutative diagrams/arrow style=tikz}

% Utilities
\usepackage{enumerate}
\usepackage{todonotes}

% Color
\usepackage{xcolor}
\definecolor{brightmaroon}{rgb}{0.76, 0.13, 0.28}

% References
\usepackage{hyperref}
\hypersetup{
  colorlinks = true,
  allcolors  = brightmaroon,
}
\usepackage{cleveref}
\usepackage{csquotes}
\usepackage[
backend=biber,
style=alphabetic,
]{biblatex}
\addbibresource{bibliography.bib}

\linespread{1.05}
\vfuzz=14pt % No more vbox errors all over the place

\newcommand{\HRule}{\rule{\linewidth}{0.5mm}} % Horizontal rule
\usepackage{multirow}

% Environments
\theoremstyle{definition}
\newtheorem{theorem}{Theorem}[section]
\newtheorem{proposition}[theorem]{Proposition}
\newtheorem{lemma}[theorem]{Lemma}
\newtheorem{corollary}[theorem]{Corollary}
\newtheorem{axiom}[theorem]{Axiom}
\newtheorem{definition}[theorem]{Definition}
\newtheorem{remark}[theorem]{Remark}
\newtheorem{example}[theorem]{Example}
\newtheorem{notation}[theorem]{Notation}

% symbols
\renewcommand{\qedsymbol}{\(\natural\)}% blacklozenge
\renewcommand{\leq}{\leqslant}
\renewcommand{\geq}{\geqslant}
\renewcommand{\setminus}{\smallsetminus}
% \newcommand{\coloneq}{\coloneqq}

% ':' for maps and '\colon' for relations on collections
\DeclareMathSymbol{:}{\mathpunct}{operators}{"3A}
\let\colon\relax
\DeclareMathSymbol{\colon}{\mathrel}{operators}{"3A}

% Common collections
\newcommand{\Z}{\mathbf{Z}}
\newcommand{\N}{\mathbf{N}}
\newcommand{\Q}{\mathbf{Q}}
\newcommand{\CC}{\mathbf{C}}
\newcommand{\R}{\mathbf{R}}
\renewcommand{\emptyset}{\varnothing}

\begin{document}
\begin{titlepage}
 \vfill
  \begin{center}
       \textsc{\LARGE \textbf{Universidade de São Paulo}} \\[2.0cm]

       \vskip 0.5cm
       \textsc{\large Luiz Gustavo Mugnaini Anselmo}

       {\normalsize Molecular Sciences \\
         Class 30, No.~USP 11809746

       E-mail: \texttt{luizmugnaini@usp.br}}\\[2.0cm]

       \HRule\\
       \vskip 0.5cm
       {\LARGE \textbf{Discrete Differential Geometry \& Computer Graphics}}
       \HRule\\[1.5cm]

       \hspace{.45\textwidth}
       \begin{minipage}{.5\textwidth}
         % \normalsize \textbf{Advanced cycle project: Molecular Sciences.}\\[0.5cm]

       \textsc{\large Prof.~Sinai Robins}\\
       Universidade de São Paulo \\
       Instituto de Matemática e Estatística \\
       E-mail: \texttt{srobins@ime.usp.br}\\[1cm]

       \normalsize São Paulo, December of 2023
       \end{minipage}
  \end{center}
\end{titlepage}

% Title stuff
\title[Discrete Differential Geometry \& Computer Graphics]{%
{\footnotesize\sl Advanced Cycle Project} \\ \smallskip
    Discrete Differential Geometry \& Computer Graphics
}%

\author{%
    Luiz Gustavo Mugnaini Anselmo \& Sinai Robins
}%

\address{%
    Instituto de Matemática e Estatística, Universidade de São
    Paulo, Rua do Matão 1010, 05508--090~São Paulo, SP
}%

\email{luizmugnaini@usp.br, srobins@ime.usp.br}

\begin{abstract}
    This is the advanced cycle research project of Luiz Gustavo Mugnaini Anselmo, supervised by
    Professor Sinai Robins at the Instituto de matemática e Estatística, USP. The goal of the
    project is to study discrete differential geometry and apply it to the area of computer
    graphics.
\end{abstract}

\maketitle


\section{Introduction}\label{sec:intro}

This project is part of the advanced cycle of the student Luiz Gustavo Mugnaini Anselmo for the
Molecular Sciences bachelor's degree. The project has as its main goal the preparation of the
student for a graduate program in the intersecting fields of computer graphics and discrete
differential geometry.

The field of discrete differential geometry (DDG) has at its core the goal of finding discrete
equivalents of the theory of manifolds. This should enable many smooth problems to be solved by
analyzing only a finite number of elements, such as polyhedron. This interest steams from many
different fields of mathematics and computer science, but we'll be interested in the particular case
of computer graphics. Being able to deal with general surfaces through the lens of a computer is a
big area of research in the field of computer graphics and has been active for many years. The
techniques developed by DDG enables us to reason about these surfaces in a more attainable way, that
would otherwise be impossible.

In classical numerical analysis given a problem in a smooth setting, one is concerned of building
algorithms discretizing the problem and minimizing the approximation errors of this discrete
version. Contrary to that, in DDG our goal is to mimic the expected behaviour of the problem in the
smooth setting but now in a discretized version---this being independent on how small are the
elements that compose this new discrete version. The biggest problem with the discretization of a
smooth theory of manifolds is that we can loose algebraic invariants that where fundamental to the
theory. This opens the question for how we should deal with this process in a way that minimizes the
loss of information when going from the smooth theory to the discrete and backwards. When tackling
this problem we find out that even when starting from equivalent characterizations in the smooth
setting we may end up with nonequivalent discretized versions of the same problem.

Computer graphics is a long running field of research in the major are of computer science. It is
also widely used in the entertainment industry, such as movies (known as \emph{computer-generated
images}, or CGI for short) and video games. We can generally define computer graphics to be
concerned in the use of computer algorithms to synthesize sensory information. Computer graphics
mainly deals with visual information but can also encompass, for instance, sound synthesis. In this
project we are going to be interested in the rendering of images and the geometry processing
underlying three-dimensional models.

The rendering pipeline can be approached in two main ways: rasterization and ray-casting. The former
is the process of computing the mapping of scene geometry to pixels in the screen in the form of
fragments of the image output. The latter is concerned with the physically-based approach of
modeling the casting of light rays from a given light source to the scene objects and simulating
the interaction of these rays with the scene---this allows us to compute the color of each pixel in
the screen based on their physical characteristics, such as Lambertian, dielectric and metal
surfaces. The main advantage of rasterization over ray-tracing is that it's fast since it does not
have to simulate the scattering of millions of rays for each frame being rendered. On the other
hand, ray-casting is the best technique regarding the synthesis of photorealistic images, contrary
to rasterization. Fortunately, graphics processing units (GPUs) are getting faster as time goes by,
so the computationally-expensive task of simulating rays is now more attainable---in fact, we have
already available systems that are capable of making real-time ray-tracing based rendering.

Geometry processing is mostly concerned about developing algorithms to analyze and manipulate
geometric models. Typical use cases of geometry processing include surface reconstruction, noise
removal, shape simplification, and geometric modeling. The study of general smooth surfaces is not
directly possible for a computer algorithm---having only a finite amount of memory makes it
impossible to capture smoothness---hence various techniques are developed in order to fragment such
surfaces into manageable discrete fragments. For instance, one approach to analyzing a surface is
the construction of a polygon mesh that encompasses the object---much in the way that one deals with
triangulations of manifolds in the smooth setting. This discretization of a surface goes hand in
hand with the approach of DDG, making it possible to use techniques coming from differential
geometry to the study of computational geometry.

\section{Literature}

The young field of discrete differential geometry has a limited amount of resources in the form of
textbooks to learn from. This is however a good opportunity to get in touch with the current state
of the art presented in recent papers. In order to lay the fundamental concepts of the theory behind
discrete differential geometry, the student will study the following resources: 
\begin{enumerate}[(a)]\setlength\itemsep{0em}
    \item ``Discrete Differential Geometry: An Applied Introduction'', by Keenan Crane
        \cite{DDG-AAI-2023-Crane}. Accompanied by public recorded lectures, this book explores the
        fundamental concepts underlying discrete differential geometry and their connections to
        geometry processing.
    \item Lecture notes by Albert Chern on Discrete Differential Geometry \cite{DDG-2020-Chern,Wang-2023-ECIG}.
        Covers a wide variety of topics in exterior calculus, DDG, and
        their connections with computer graphics.
\end{enumerate}

The field of computer graphics has long been established in the industry and the academic setting
and already counts with a vast literature. The following books will be studied in order to lay the
foundations of the field:
\begin{enumerate}[(a)]\setlength\itemsep{0em}
    \item ``Advanced Global Illumination'', by P.~Dutr\'{e} et.~al \cite{AGI-2006}. Explores the
        mathematical theory underlying rendering and global illumination, with the goal of
        understanding photorealistic image synthesis.
    \item ``Real-Time Rendering'', by T.~Akanine-M\"{o}ller et.~al \cite{RTR-2018-Moller}. Deals with
        the most highly interactive subarea of computer graphics, where images should be rendered to
        the screen in such a way that the viewer cannot distinguish particular rendering
        frames---but rather a dynamic, real-time, immersive image.
    \item ``Polygon Mesh Processing'', by M.~Botsch et.~al \cite{PMP-2010}. The book deals with the
        components of the geometry processing pipeline based on polygon meshes. This will be
        relevant for the connection between DDG and computer graphics.
\end{enumerate}


\appendix

\section{Proposed Courses}\label{sec:courses}

The planning for the next semester involves both the study of differential geometry from a discrete
point of view, and computer graphics theory and techniques. The former will be studied
independently, while the second will be studied both independently and in conjunction with a formal
introductory course in computer graphics.

\begin{table}[h!]\label{tab:courses}
  \centering
  \caption{
      Proposed disciplines for the next semester (2024/1). Types are divided into undergraduate (U)
      and graduate (G)
  }
  \begin{tabular}{ |c|c|c|c|c| }
    \hline
    Year and semester & Code & Course & Type & Credits \\
    \hline
    \multirow{2}{*}{2024/1}
                      & CCM0428 & Iniciação à Pesquisa IV         & U & 12 \\ 
                      & MAC5744 & Introdução à Computação Gráfica & G & 6 \\
    \hline
\end{tabular}
\end{table}

\printbibliography

\end{document}

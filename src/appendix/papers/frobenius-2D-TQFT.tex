\documentclass[../../../deep-dive]{subfiles}

\author{Luiz Gustavo Mugnaini Anselmo \\ n\(^{\circ}\)USP:~11809746}

\begin{document}

\onlyinsubfile{
\maketitle
\subfile{../../category-theory/monoidal-cat}

\printbibliography
}

% \chapter{Frobenius Algebras \& 2D TQFTs}

% \section{Monoidal Categories}

% \begin{definition}
% \label{def:monoidal-category}
% A \emph{monoidal category} is a tuple
% \((\cat M, \otimes, 1, \alpha, \gamma, \rho)\) consisting of:
% \begin{itemize}\setlength\itemsep{0em}
% \item A \emph{category} \(\cat M\).

% \item A bifunctor \(\otimes: \cat M \times \cat M \to \cat M\)

% \item A distinguished object \(1 \in \cat M\) that is \emph{unitary} with
%   respect to \(\otimes\), that is:
%   \[
%   m \otimes 1 = m = 1 \otimes m
%   \]
%   for any object \(m \in \cat M\).

% \item A \emph{natural isomorphism}
%   \[
%   \alpha: ((- \otimes -) \otimes -)
%   \overset{\iso}\Longrightarrow (- \otimes (- \otimes -)).
%   \]
%   called \emph{associator}, in the sense that given any triple of objects \((a,
%   b, c)\) of \(\cat M\), the image
%   \[
%   \begin{tikzcd}
%   (a \otimes b) \otimes c \ar[r, "{\alpha(a, b, c)}"', "\dis"]
%   &a \otimes (b \otimes c)
%   \end{tikzcd}
%   \]
%   is an isomorphism in \(\cat M\).

% \item Two \emph{natural isomorphisms}
%   \[
%   \lambda: (1 \otimes -)
%   \overset{\iso}\Longrightarrow (-)
%   \quad
%   \text{ and }
%   \quad
%   \rho: (1 \otimes -)
%   \overset{\iso}\Longrightarrow (-)
%   \]
%   called \emph{left and right unitors}, respectively. In other words, given any
%   object \(a \in \cat M\) the morphisms \(\lambda a: 1 \otimes a \isoto a\) and
%   \(\rho a: a \otimes 1 \isoto a\) are \emph{isomorphisms} in \(\cat M\).
% \end{itemize}
% This data should satisfy the following two conditions:
% \begin{itemize}\setlength\itemsep{0em}
% \item (Triangle identity) Given any pair \((a, b)\) of objects in \(\cat M\),
%   the diagram
%   \[
%   \begin{tikzcd}
%   (a \otimes 1) \otimes b \ar[rr, "{\alpha(a, 1, b)}"]
%   \ar[rd, "\rho a \otimes \Id_b"']
%   & &a \otimes (1 \otimes b) \ar[ld, "\Id_a \otimes \lambda b"]
%   \\
%   &a \otimes b &
%   \end{tikzcd}
%   \]
%   commutes in \(\cat M\).

% \item (Pentagon identity) Given any tuple \((a, b, c, d)\) of objects in
%   \(\cat M\), the diagram
%   \[
%   \begin{tikzcd}
%   &
%   &(a \otimes b) \otimes (c \otimes d)
%   \ar[rdd, "{\alpha(a \otimes b, c, d)}"]
%   &
%   \\
%   & & &
%   \\
%   a \otimes (b \otimes (c \otimes d))
%   \ar[dd, "{\Id_a \otimes \alpha(b, c, d)}"']
%   \ar[rruu, "{\alpha(a, b, c \otimes d)}"]
%   &
%   &
%   &((a \otimes b) \otimes c) \otimes d
%   \\
%   & & &
%   \\
%   % empty
%   a \otimes ((b \otimes c) \otimes d)
%   \ar[rrr, "{\alpha(a, b \otimes c, d)}"']
%   &
%   &
%   &(a \otimes (b \otimes c)) \otimes d
%   \ar[uu, "{\alpha(a, b, c) \otimes \Id_d}"']
%   \end{tikzcd}
%   \]
%   is commutative in \(\cat M\).
% \end{itemize}
% \end{definition}
% The tuple \((\cat M, \otimes, 1, \alpha, \lambda, \rho)\) is said to be a
% \emph{strict monoidal category} if the three natural isomorphisms are naturally
% isomorphic to the identity---if this is the case, we shall refer to the category
% simply by the triple \((\cat M, \otimes, 1)\).

% \begin{definition}
% \label{def:monoidal-functor}
% Let \((\cat M, \otimes, 1, \alpha, \lambda, \rho)\) and \((\cat N,
% \widehat\otimes, \widehat 1, \widehat \alpha, \widehat \lambda, \widehat \rho)\)
% be two (strict) monoidal categories. We say that a functor \(F: \cat M \to \cat
% N\) is a (\emph{strict}) \emph{monoidal functor} if it preserves the actions of
% the natural isomorphisms. To put concretely, we have:
% \begin{itemize}\setlength\itemsep{0em}
% \item The unit of \(\cat M\) is mapped to the unit of \(\cat N\), that is,
%   \(F e = \widehat e\).

% \item For any \(a \in \cat M\) one has
%   \(F (\lambda a) = \widehat \lambda (F a)\) and
%   \(F (\rho a) = \widehat \rho(F a)\).

% \item For any pair \((a, b)\) of objects in \(\cat M\) there exists an
%   isomorphism \(F(a \otimes b) \iso F a \widehat \otimes F b\) in \(\cat
%   N\)---in the strict case, the isomorphism is replaced by an equality.

% \item For any triple \((a, b, c)\) of objects in \(\cat M\) we have
%   \(F \alpha(a, b, c) = \widehat \alpha (F a, F b, F c)\).

% \item For every two maps \(f\) and \(g\) in \(\cat M\) there exists an
%   isomorphism \(F(f \otimes g) \iso F f \widehat \otimes F g\) in
%   \(\cat N\)---in the strict case, the isomorphism is replaced by an equality.
% \end{itemize}
% \end{definition}

% \begin{definition}[Monoidal natural transformation]
% \label{def:monoidal-natural-transformation}
% Let \((\cat M, \otimes, 1, \alpha, \lambda, \rho)\) and
% \((\cat N, \widehat\otimes, \widehat 1, \widehat \alpha, \widehat \lambda,
% \widehat \rho)\) be two (strict) monoidal categories, and consider a pair of
% parallel (strict) functors \(F, G: \cat M \para \cat N\). A natural
% transformation \(\eta: F \nat G\) is said to be \emph{monoidal} if
% \(\eta_1 = \widehat 1\), and for any pair of objects \(a, b \in \cat M\) the
% diagram
% \[
% \begin{tikzcd}
% F(a \otimes b) \ar[d, "\dis"']
% \ar[r, "\eta_{a \otimes b}"]
% &G(a \otimes b) \ar[d, "\dis"] \\
% F a \widehat\otimes F b \ar[r, "\eta_a \widehat\otimes \eta_b"']
% &G a \widehat\otimes G b
% \end{tikzcd}
% \]
% commutes in the monoidal category \(\cat N\).
% \end{definition}

% \begin{theorem}[Strictification of monoidal categories]
% \label{thm:strictification}
% Every monoidal category is \emph{monoidally equivalent} to a \emph{strict}
% monoidal category.
% \end{theorem}

% \begin{proof}
% Let \((\cat M, \otimes, 1, \alpha, \lambda, \rho)\) be a monoidal
% category. We shall construct a strict monoidal category out of \(\cat M\). To
% that end, define a category \(\cat N\) where:
% \begin{itemize}\setlength\itemsep{0em}
% \item The objects of \(\cat N\) are pairs \((F, \eta)\) where \(F\) is an
%   \emph{endofunctor} of \(\cat M\) and
%   \[
%   \eta: F(- \otimes -) \isonat F(-) \otimes (-)
%   \]
%   is a \emph{natural isomorphism} such that, for any triple \((a, b, c)\) of
%   objects of \(\cat M\), the pentagonal diagram
%   \[
%   \begin{tikzcd}
%   &
%   &(F(a) \otimes b) \otimes c
%   &
%   \\
%   & & &
%   \\
%   F(a \otimes b) \otimes c
%   \ar[rruu, "\eta_{(a, b)} \otimes \Id_c"]
%   &
%   &
%   & F(a) \otimes (b \otimes c)
%   \ar[luu, "{\alpha(F a, b, c)}"']
%   \\
%   & & &
%   \\
%   F((a \otimes b) \otimes c)
%   \ar[uu, "\eta_{(a \otimes b, c)}"]
%   &
%   &
%   &F(a \otimes (b \otimes c))
%   \ar[lll, "{F \alpha(a, b, c)}"]
%   \ar[uu, "\eta_{(a, b \otimes c)}"']
%   \end{tikzcd}
%   \]

% \item A morphism \(\varepsilon: (F, \eta) \to (F', \eta')\) is a natural
%   transformation \(\varepsilon: F \nat F'\) such that, given any pair \((a, b)\)
%   of objects of \(\cat M\), the diagram
%   \[
%   \begin{tikzcd}
%   F(a \otimes b) \ar[d, "\eta_{(a, b)}"']
%   \ar[r, "\varepsilon_{a \otimes b}"]
%   &F'(a \otimes b) \ar[d, "\eta'_{(a, b)}"]
%   \\
%   F(a) \otimes b \ar[r, "\varepsilon_a \otimes \Id_b"']
%   &F'(a) \otimes b
%   \end{tikzcd}
%   \]
%   commutes in \(\cat M\). Moreover, we define the composition of morphisms in
%   \(\cat N\) to be given by the vertical composition of natural transformations.

% \item Define a bifunctor \(\widehat\otimes: \cat N \times \cat N \to \cat N\) as
%   \((F, \eta) \widehat\otimes (F', \eta') \coloneq (F F', \widehat\eta)\), where
%   \[
%   \widehat\eta: F F'(- \otimes -) \nat F F'(-) \otimes (-)
%   \]
%   is the natural transformation given by the composition
%   \[
%   \begin{tikzcd}
%   F F'(a \otimes b)
%   \ar[rr, "F \eta'_{(a, b)}"']
%   \ar[rrrr, bend left, "\widehat\eta_{(a, b)}"]
%   &
%   &F (F'(a) \otimes b)
%   \ar[rr, "\eta_{(F' a, b)}"']
%   &
%   &F F'(a) \otimes b
%   \end{tikzcd}
%   \]
%   for any pair of objects \((a, b)\) of \(\cat M\). Given morphisms
%   \(\varepsilon, \delta: (F, \eta) \to (F', \eta')\) we also define
%   \(\varepsilon \otimes\).
% \end{itemize}
% From this construction we find that the triple
% \((\cat N, \widehat\otimes, (\Id_{\cat M}, I))\)---where the natural
% transformation \(I: (- \otimes -) \isonat (- \otimes -)\) is the identity
% morphism \(I_{(a, b)} \coloneq \Id_{a \otimes b}\) in \(\cat M\) for any two
% \(a, b \in \cat M\)---is a \emph{strict monoidal category}, since:
% \begin{itemize}\setlength\itemsep{0em}
% \item The bifunctor \(\widehat\otimes\) satisfies \emph{equality} for both left
%   and right unitors.
% \item
% \todo[inline]{Continue proof and monoidal categories at large}
% \end{itemize}
% \end{proof}

\end{document}

%%% Local Variables:
%%% mode: latex
%%% TeX-master: "../../../deep-dive"
%%% End:

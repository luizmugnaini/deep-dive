\section{Cobordisms}

\subsection{Unoriented Cobordisms}

\begin{definition}[Unoriented cobordism]
    \label{def:unoriented-cobordism}
    Given a pair \(\Sigma_0\) and \(\Sigma_1\) of smooth compact
    \((n-1)\)-manifolds without boundary, we define a \emph{cobordism between
        \(\Sigma_0\) and \(\Sigma_1\)} to be a smooth compact \(n\)-manifold \(M\)
    whose boundary is \(\Bd M = \Sigma_0 \disj \Sigma_1\). We thus call the
    manifolds \(\Sigma_0\) and \(\Sigma_1\) \emph{cobordant}.
\end{definition}

\begin{example}
    \label{exp:birth-death-circle}
    Two interesting cobordisms are formed from the empty manifold to the circle,
    which shall be called \emph{birth-of-a-circle}, and from the circle to the empty
    manifold, so called \emph{death-of-a-circle}.
\end{example}

\begin{lemma}[Cobordant zero and one dimensional manifolds]
    \label{lem:cobordant-0-and-1-manifolds}
    Two given compact \(0\)-manifolds without boundary are cobordant if and only if
    they have the same number of points modulo \(2\). Moreover, any two compact
    \(1\)-manifolds without boundary are cobordant.
\end{lemma}

\begin{proof}
    Let's consider the case of a pair of \(0\)-manifolds \(\Sigma_0\) and
    \(\Sigma_1\). Notice that since every pair of points can be connected by a
    smooth curve, and every \(1\)-manifold with boundary has an even number of
    boundary points\footnote{This is due to the fact that \(1\)-manifolds are
        \(C^{\infty}\)-isomorphic to a finite disjoint union of circles or intervals
        (see \cref{cor:1-manifold-boundary-even-number-of-points}).}, it follows that
    \(\Sigma_0\) and \(\Sigma_1\) are cobordant if and only if the disjoint union
    \(\Sigma_0 \disj \Sigma_1\) has an even number of points.

    For the second statement, one should recall that a compact \(1\)-manifold is the
    disjoint union of circles. Then we can choose one of the manifolds to
    attach copies of the death-of-a-circle cobordism for each of its circles, and
    attach birth-of-a-circle cobordisms for each of its respective
    circles of the other \(1\)-manifold. This construction yields a cobordism
    between them.
\end{proof}

\subsection{Oriented Cobordisms}

Consider the following setup: let \(\Sigma\) be a closed submanifold of \(M\)
with codimension \(1\), where \(\dim M = n\). Assume both manifolds to be
oriented.

\begin{definition}[Positive normal]
    \label{def:positive-normal-manifold}
    Let \([v_1, \dots, v_{n-1}]\) be a positive basis for \(T_x \Sigma\) for any
    given point \(x \in \Sigma\). We say that a tangent vector \(v \in T_x M\) is a
    \emph{positive normal} if the induced basis \([v_1, \dots, v_{n-1}, v]\) for
    \(T_x M\) is positive.
\end{definition}

\begin{definition}[In and out boundaries]
    \label{def:in-out-boundaries}
    If \(\Sigma\) is a connected component of \(\Bd M\), we call \(\Sigma\) an
    \emph{in-boundary} if a positive normal points inwards relative to \(M\), and
    otherwise an \emph{out-boundary}---when a positive normal points outward
    relative to \(M\).
\end{definition}

% In order to show that in and out boundaries are not ill-defined, one has to show
% the independence of the choice of the positive normal and the base point
% \(x \in \Sigma\). This comes from the assumption that \(\Sigma\) is connected,
% thus

The notion of an in and out boundary allows us to define the notion of an
oriented cobordism. From now on, a cobordism will always mean an
\emph{oriented} one, unless stated otherwise.

\begin{definition}[Oriented cobordism]
    \label{def:oriented-cobordism}
    Let \(\Sigma_{\text{in}}\) and \(\Sigma_{\text{out}}\) be compact
    \((n-1)\)-manifolds without boundary. We define an \emph{oriented cobordism}
    between them to be a triple \((M, \iota_{\text{in}}, \iota_{\text{out}})\),
    where \(M\) is a smooth compact oriented \(n\)-manifold, and arrows
    \[
        \begin{tikzcd}
            \Sigma_{\text{in}} \ar[r, "\iota_{\text{in}}"] &M
            &\Sigma_{\text{out}} \ar[l, "\iota_{\text{out}}"']
        \end{tikzcd}
    \]
    which are \(C^{\infty}\)-isomorphisms when restricted to the in and out boundary
    of \(M\), respectively. We shall denote the oriented cobordism \(M\) as an
    arrow \(M: \Sigma_{\text{in}} \nat \Sigma_{\text{out}}\).
\end{definition}

\begin{definition}[Equivalence of cobordisms]
    \label{def:equivalence-of-cobordisms}
    Given two cobordisms
    \[
        \begin{tikzcd}
            \Sigma_{\text{in}} \ar[r, shift right=1.5, Rightarrow, "N"']
            \ar[r, shift left=1.5, Rightarrow, "M"]
            &\Sigma_{\text{out}}
        \end{tikzcd}
    \]
    we say that \(M\) is \emph{equivalent} to the cobordism \(N\) if there exists an
    orientation-preserving \(C^{\infty}\)-isomorphism \(\phi: M \isoto N\) such that
    the following diagram commutes in \(\Man\):
    \[
        \begin{tikzcd}
            &N &
            \\
            \Sigma_{\text{in}} \ar[ru] \ar[rd]
            &
            &\Sigma_{\text{out}} \ar[lu] \ar[ld]
            \\
            &M \ar[uu, "\phi", "\dis"']&
        \end{tikzcd}
    \]
\end{definition}

\section{Elements of Morse Theory}

\begin{definition}
    \label{def:non-degenerate-point-and-index}
    Let \(f: M \to I\) be a \(C^{\infty}\)-morphism, and \(p \in M\) be a critical
    point of \(f\). We call \(p\) a \emph{non-degenerate} point if there exists a
    chart about \(p\) for which the local Hessian of \(f\) is
    invertible. Furthermore, define the \emph{index of \(f\) at \(p\)} to be the
    number of \emph{negative eigenvalues} of the local Hessian.
\end{definition}

\begin{definition}[Morse maps]
    \label{def:morse-maps}
    Given a smooth manifold \(M\), we say that a \(C^{\infty}\)-morphism
    \(f: M \to I\) is a \emph{Morse map} if every critical point of \(f\) is
    non-degenerate. If it happens to be the case that \(M\) is a manifold with
    boundary, we shall require that \(f^{-1} \Bd I = \Bd M\) and that the boundary
    points \(\Bd I = \{0, 1\}\) are regular values of \(f\)---this ensures that
    \(\Bd M\) contains no critical points.
\end{definition}

The existence of Morse maps is ensured by the following theorem:

\begin{theorem}
    \label{thm:morse-maps-are-dense}
    For any manifold \(M\) and integer \(2 \leq r \leq \infty\), the collection of
    Morse maps \(M \to I\) is dense in \(C^r(M, I)\).
\end{theorem}

The following is a generalization of the construction of attaching spaces:

\begin{definition}[Gluing]
    \label{def:gluing-topological-spaces}
    Let \(f: X \to Y\) and \(g: X \to Z\) be topological morphisms. We define the
    \emph{gluing of \(Y\) and \(Z\) along \(X\)} to be the pushout
    \[
        \begin{tikzcd}
            X \ar[d, "g"']
            \ar[r, "f"]
            \ar[rd, "\ulcorner", phantom, very near end]
            &Y \ar[d] \\
            Z \ar[r]
            &Y \disj_X Z
        \end{tikzcd}
    \]
    Explicitly, \(Y \disj_X Z\) is the quotient space of \(Y \disj Z\) where
    \(y \sim z\) if and only if there exists a common \(x \in X\) such that
    \(f x = y\) and \(g x = z\).
\end{definition}

%%% Local Variables:
%%% mode: latex
%%% TeX-master: "../../../deep-dive"
%%% End:

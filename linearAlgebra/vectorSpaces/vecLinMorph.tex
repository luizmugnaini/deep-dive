\section{Vector Spaces from Linear Morphism}

\begin{proposition}
  Given \(k\)-vector spaces \(V\) and \(W\), consider the \(k\)-linear morphism
  \(\varphi : V \to W\). Then \(\varphi\) is injective if and only if \(\ker
  \varphi = \{0\}\).
\end{proposition}

\begin{proof}
  (\(\Rightarrow\)) Suppose \(\varphi\) is injective, then, since \(0 \xmapsto
  \varphi 0\), it follows that \(\ker \varphi = \{0\}\). (\(\Leftarrow\))
  Suppose now that \(\ker \varphi = \{0\}\), then from the universal property
  of quotients we find that \(V/\ker \varphi \hookrightarrow W\) is injective,
  since the only possible element in \(V\) that generates the zero class is \(0
  \in V\), therefore, we can regard the above mapping as \([v] \mapsto
  \varphi(v)\), therefore we conclude that \(\varphi\) is injective.
\end{proof}

\begin{theorem}[Universal property for kernels]
  Let \(V, W\) be \(k\)-vector spaces and consider the \(k\)-linear morphism
  \(\varphi : V \to W\). Denote \(\iota : \ker \varphi \hookrightarrow V\) the
  embedding morphism. Then, for any given \(k\)-vector space \(L\) and
  \(k\)-linear morphism \(\psi : L \to V\) such that \(\varphi \circ \psi = 0\),
  there exists a unique \(k\)-linear morphism \(\ell : L \to \ker \varphi\) such
  that the following diagram commutes
  \[
    \begin{tikzcd}
      L \ar[d, dashed, swap, "\ell"] \ar[dr, "\psi"] \ar[drr, bend left,
      "0" description] & & \\
      \ker \varphi \ar[r, hookrightarrow, "\iota"] &V \ar[r, "\varphi"] &W
    \end{tikzcd}
  \] 
\end{theorem}

\begin{proof}
  The condition for the commutativity of the diagram is \(\psi = \iota \circ
  \ell\). Notice that since the definition of \(\psi\) depends uniquely in the
  condition \(\varphi \circ \psi = 0\), we find that since \(\im(\ell)
  \subseteq \ker \varphi\) it follows that indeed \(\varphi \circ (\iota \circ
  \ell) = 0\) and therefore the uniqueness of \(\ell\) comes by defining it as
  \(\ell(l) = \psi(l)\) and the existence comes merely by the fact that \(\psi\) 
  is already a \(k\)-linear morphism.
\end{proof}

\begin{definition}[Cokernel]
  Let \(k\)-vector spaces \(V\) and \(W\) and consider the linear morphism
  \(\varphi : V \to W\). We define the cokernel of \(\varphi\) as
  \[
    \coker \varphi = W/\im(\varphi).
  \] 
\end{definition}

\begin{theorem}[Universal property for cokernels]
   Let \(k\)-vector spaces \(V\) and \(W\) and the linear morphism \(\varphi : V
   \to W\). Denote by \(\pi : W \to \coker \varphi\) the projection morphism.
   For any given \(k\)-vector space \(L\) and linear morphism \(\psi : W \to L\)
   such that  \(\psi \circ \varphi = 0\), then there exists a unique linear
   morphism  \(\ell : \coker \varphi \to L\) such that the diagram commutes
   \[
     \begin{tikzcd}
       V \ar[drr, bend right, "0" description] \ar[r, "\varphi"] &W \ar[r,
       "\pi"] \ar[dr, "\psi"] &\coker \varphi \ar[d, dashed, "\ell"] 
       \\
        & &L
     \end{tikzcd}
   \] 
\end{theorem}

\begin{proof}
  The condition for the diagram to commute is that \(\ell \circ \pi = \psi\).
  For that, notice that since \(\coker \varphi = W/\im(\varphi)\) it follows
  that \(\ell \circ \pi \circ \varphi (v) = \ell([\varphi(v)])\) but notice that
  obviously \(\varphi(v) \in \im(\varphi)\) thus \([\varphi(v)] = 0 \in \coker
  \varphi\) thus indeed \(\ell \circ \pi \circ \varphi = 0\). Moreover, in order
  to define \(\ell\) we ought to have \(\ell([w]) = \psi(w)\), which is well
  defined since \([w] = w + \im(\varphi)\) and \(\forall w' \in \im(\varphi)\) 
  we have \(\psi(w') = 0\) and therefore \(\psi(w + w') = \psi(w)\), which ends
  the proof since \(\ell\) gets the linear structure of \(\psi\) and also is
  unique from construction.
\end{proof}

\begin{proposition}
  Let \(k\)-vector spaces \(V\) and \(W\) and define any linear morphism
  \(\varphi : V \to W\), then there are natural isomorphisms 
  \[
    \ker(W \overset \pi \longrightarrow \coker \varphi) \iso \coker(\ker \varphi
    \overset \iota \longrightarrow V) \iso \im(\varphi)
  \] 
\end{proposition}

\begin{proof}
  First we show the existence of the isomorphism \(\ker \pi \iso \im(\varphi)\),
  which is obtained by taking the mapping \(w \mapsto w\), since \(\ker \pi =
  \im(\pi)\), establishing an obvious isomorphism between the two given objects.
  Now we focus on showing the isomorphism \(\coker \iota = V/\ker \varphi \iso
  \im(\varphi)\), which can be obtained by considering the mapping \([v] \mapsto
  \varphi(v)\), which is well defined from the same reasoning as in the above
  theorem.
\end{proof}

Consider the collections of \(k\)-vector spaces \(\{V_i\}_i\) and \(\{W_i\}_i\)
with a corresponding collection of morphism \(\{f_i : V_i \to W_i\}_i\) and
\(\{\varphi_i : V_i \to V_{i+1}\}_i\) and \(\{\psi_i : W_i \to W_{i+1}\}\) such
that for any index \(i\) the following diagram commutes
\[
  \begin{tikzcd}
    V_i \rar["f_i"] \dar["\varphi_i"] &W_i \dar["\psi_i"] 
    \\
    V_{i+1} \rar["f_{i+1}"] &W_{i+1}
  \end{tikzcd}
\] 

By defining the morphisms \(\iota_i : \ker f_i \to V\) and \(\pi_i : W \to
\coker f_i\) we find that since 
\[
  f_{i+1} \circ \varphi_i \circ \iota_i 
  = \psi_i \circ f_i \circ \iota_i 
  = \psi_i \circ 0 = 0
\] 
then we can use the universal property of kernels in order to find the existence
of a unique morphism \(\ker f_i \to \ker f_{i+1}\), that is
\[
  \begin{tikzcd}
    \ker f_i \dar[dashed] \ar[dr, "\varphi_i \circ \iota_i"] \ar[drr, bend
    left, "0"] & &
    \\
    \ker f_{i+1} \rar[hook, "\iota_{i+1}"] &V_{i+1} \rar["f_{i+1}"] &W_{i+1}
  \end{tikzcd}
\] 

On the other hand we have that 
\[
  \pi_{i+1} \circ \psi_i \circ f_i 
  = \pi_{i+1} \circ f_{i+1} \circ \varphi_i 
  = 0 \circ \varphi_i = 0 
\] 
thus by means of the universal property for cokernels we find that there exists
a unique morphism \(\coker f_i \to \coker f_{i+1}\), that is
\[
  \begin{tikzcd}
    V_i \rar["f_i"] \ar[drr, bend right, "0" description] 
      &W_i \rar["\pi_i"] \ar[dr, "\pi_{i+1} \circ \psi_i" description]
    &\coker f_i \dar[dashed]
    \\
    & &\coker f_{i+1}
  \end{tikzcd}  
\] 

Binding both results together we find that in general we have the following
diagram
\[
  \begin{tikzcd}
    \vdots \dar[dashed] &\vdots \dar &\vdots \dar &\vdots \dar[dashed]
    \\
    \ker f_i \rar["\iota_i"] \dar[dashed]
      &V_i \rar["f_i"] \dar["\varphi_i"] 
    &W_i \dar["\psi_i"] \rar["\pi_i"] 
      &\coker f_i \dar[dashed]
    \\
    \ker f_{i+1} \rar["\iota_{i+1}"] \dar[dashed]
      &V_{i+1} \rar["f_{i+1}"] \dar
    &W_{i+1} \rar["\pi_{i+1}"] \dar
      &\coker f_{i+1} \dar[dashed]
    \\
    \vdots &\vdots &\vdots &\vdots
  \end{tikzcd}
\] 

\begin{definition}[Exact sequence]
  Given a collection of \(k\)-vector spaces \(\{V_i\}_i\) and \(k\)-linear
  morphisms \(\{f_i : V_i \to V_{i+1}\}\), we say that the sequence 
  \[
    \begin{tikzcd}
      V_0 \rar["f_0"] &V_1 \rar["f_1"] &\dots \rar["f_{n-1}"] &V_n
    \end{tikzcd}  
  \] 
  is exact if \(\ker f_i = \im(f_{i-1})\) for all index.
\end{definition}

\begin{proposition}
  Let the \(k\)-linear morphism \(f: V \to W\). Then \(f\) is injective if and
  only if the sequence 
  \begin{tikzcd}[cramped] 
    0 \rar &V \rar["f"] &W 
  \end{tikzcd}
  is exact. On the other hand, the morphism \(f\) is surjective if and only if
  the sequence
  \begin{tikzcd}[cramped] 
    V \rar["f"] &W \rar &0
  \end{tikzcd}
  is exact.
\end{proposition} 

\begin{proof}
  (\(\Rightarrow\)) Suppose \(f\) is injective, then \(\ker f = \{0\}\),
  moreover, surely \(\im(0) = \ker f\) thus the sequence is exact.
  (\(\Leftarrow\)) Suppose the sequence is exact, so that \(\im(0) = \ker f\)
  but then \(\ker f = \{0\}\) which, as we already have proven above, implies in
  \(f\) injective. 

  (\(\Rightarrow\)) Suppose \(f\) is surjective, then \(\im(f)= W = \ker 0\) and
  thus the sequence is exact. (\(\Leftarrow\)) Suppose the sequence is exact, so
  that \(\im(f) = W = \ker 0\), then given any element \(w \in W\) there exists
  at least one corresponding \(v \in V\) such that \(f(v) = w\) and therefore
  the morphism is surjective.
\end{proof}

\begin{definition}[Short exact sequence]   
  Let the \(k\)-vector spaces \(V, W\) and \(L\). A exact sequence 
  \[
    \begin{tikzcd}
      0 \rar &W \rar[hook, "f"] &V \rar[two heads, "g"] &L \rar &0
    \end{tikzcd}
  \] 
  that is, with \(\ker g = \im(f)\), is said to be a short exact sequence of
  \(k\)-vector spaces.
\end{definition}

\begin{proposition}
  In the short exact sequence 
  \begin{tikzcd}[cramped, sep = small]
    0 \rar &W \rar[hook, "f"] &V \rar[two heads, "g"] &L \rar &0
  \end{tikzcd}
  we have \(L \iso V/W\).
\end{proposition}

\begin{proof}
  Firstly, notice that since \(W \hookrightarrow V\) is an embedding, we can
  regard \(W\) as a subspace of \(V\), so that taking the quotient \(V/W\) is
  possible. Moreover, notice that since \(g\) is surjective, from the first
  isomorphism theorem we find that \(V/\ker g \iso L\) which from the fact that
  the sequence is exact, is the same as \(V/\im f \iso L\) but this is exactly
  what we meant by \(V/W\) since \(W\) was regarded as a subspace of \(V\) via
  the embedding \(f\).
\end{proof}

\begin{proposition}
   Let \(f : V \to W\) be a \(k\)-linear morphism, then the following is true
   \begin{enumerate}[(a).]
     \item The sequence \(0 \to \ker f \to V \to \im f \to 0\) is exact.
     \item The sequence \(0 \to \im f \to W \to \coker f \to 0\) is exact.
     \item The sequence \(0 \to \ker f \to V \to W \to \coker f \to 0\) is
       exact.
   \end{enumerate}
\end{proposition}

\begin{proof}
  For (I) notice that trivially \(\ker(\ker f \to V) = \im(0) = \{0\}\), since
  \(\ker f \subseteq V\); moreover, \(\im(\ker f \to V) = \ker f = \ker(V \to
  \im(f))\) from the mere definition of kernel; finally, it is to be noticed
  that \(\im(V \to \im(f)) = \ker(\im(f) \to 0)\) because the zero morphism has
  its whole domain as its kernel. 
  For (II) we have that \(\im(0 \to \im(f)) = \{0\} \subseteq W\), thus indeed
  \(\ker(\im(f) \to W) = \im(0 \to \im(f))\); after that, notice \(\im(\im(f)
  \to W) = \im(f)\) but from definition we have \(\coker f = W/\im(f)\) thus we
  really get \(\im(f) = \ker(\coker f)\); the next morphism trivially obeys
  \(\im(W \to \coker f) = \ker(\coker f \to 0)\).
  For (III) the part \(0 \to \ker f \to V\) is already exact, also from
  definition \(\im(\ker f \to V) = \ker f = \ker(V \to W)\); then, \(\im(V \to
  W) = \im(f) = \ker(\coker f)\); and finally  \(\im(W \to \coker f) =
  \ker(\coker f \to 0)\).
\end{proof}

\begin{definition}[The \(\Hom\) space]
  Let the \(k\)-vector spaces \(V, W\). Then the collection of morphisms
  \(\Hom_{\cat{Vect}_k}(V, W)\) is a \(k\)-vector space with structure given by
  \begin{gather*}
    (f + g)(v) = f(v) + g(v),\ \forall f, g \in \Hom_{\cat{Vect}_k}(V, W) \\
    (cf)(v) = c f(v),\ \forall f \in \Hom_{\cat{Vect}_k}(V, W)\ \forall c \in k
  \end{gather*}
  where \(0\) is the map \(0 : V \to 0 \to W\).
\end{definition}

\begin{remark}
  For the remaining this section we'll be dealing with the category
  \(\cat{Vect}_k\), unless said otherwise, therefore I'll omit the
  category for the sake of notation.
\end{remark}

\begin{proposition}
   For any \(k\)-vector space \(V\) there exists a natural isomorphism
   \[
     \Hom(k, V) \iso V
   \]
\end{proposition}

\begin{proof}
  Map the zero morphism \((0 : k \to V) \longmapsto 0 \in V\). Then we could take
  the morphism \(1_v : k \to V\) being defined as \(k \ni 1 \xmapsto{1_v} v \in
  V\), not restricting the other mappings whatsoever, then we could trivially
  make the mappings \((1_v : k \to V) \longmapsto v \in V\) and then we are
  essentially done, since this is trivially an isomorphism.
\end{proof}

\begin{proposition}
  Let the collection of \(k\)-vector spaces \(\{V_i\}_{i \in I}\), then, for any
  \(L\), \(k\)-vector space, then:
  \begin{enumerate}[I.]
    \item There is a natural isomorphism 
      \[
        \Hom \bigg( L,\ \prod_{i \in I} V_i \bigg) \iso \prod_{i \in
        I} \Hom(L,\ V_i).
      \]
    \item There is a natural isomorphism
      \[
        \Hom\bigg( \bigoplus_{i \in I} V_i,\ L \bigg) \iso
        \prod_{i \in I} \Hom(V_i,\ L)
      \] 
      and therefore, given a set \(S\), there is also a natural isomorphism
      \(\Hom(k^{\oplus S}, V) \iso V^S\).
  \end{enumerate}
\end{proposition}

\begin{proof}
  For the first proposition, we can assign the map of zero mappings \((0 : L \to
  \prod V_i) \longmapsto (0 : L \to V_i)_i\). Next we may consider the functions
  \(g : L \to \prod V_i\), defined as \(g(l) = (f_1(l), f_2(l), \dots)\) for all
  \(l \in L\), where \(f_i : L \to V_i\); then we can simply make the mapping
  \(g \longmapsto (\pi_i \circ f_i)_i\).

  For the second proposition, consider the morphism \(\psi : \Hom\left(
  \bigoplus_{i \in I} V_i,\ L \right) \) with the mapping \(f \mapsto (f \circ
  \iota_i)_{i \in I}\) where \(\iota_j: V_j \hookrightarrow \bigoplus_{i \in
  I} V_i\) is the inclusion map. For the injectivity, consider \(f \in
  \Hom\left( \bigoplus_{i \in I} V_i,\ L \right), f \neq 0\) be a morphism,
  then \(f \circ \iota_i \neq 0\) and therefore \(\ker(\psi) = 0\), which
  implies in the injectivity of \(\psi\). Now, let any tuple of morphisms
  \((g_i)_{i \in I} \in \prod_{i \in I} \Hom(V_i, L)\), then we can define a
  morphism \(f \in \Hom\left( \bigoplus_{i \in I} V_i,\ L \right)\) such that
  \(f \circ \iota_i = g_i\) then definitely \(\psi(f) = (g_i)_{i \in I}\).
\end{proof}

\begin{definition}[Induced \(\Hom\) \(k\)-linear morphism]
  Given a \(k\)-linear morphism \(f : V \to L\) and any \(k\)-vector space
  \(W\), there are induced, uniquely defined, \(k\)-linear morphisms
  \begin{enumerate}[(a)]
    \item (Pushforward) \(f_\ast : \Hom(W, V) \to \Hom(W, L)\) with the mapping
      \(\alpha \mapsto f \circ \alpha\), so that the diagram commutes
      \[
        \begin{tikzcd}
          V \ar[r, "f"]
            &L \\
          W \ar[u, "\alpha"] 
          \ar[ur, dashed, swap, "f_\ast(\alpha) := f \circ \alpha"]
        \end{tikzcd}
      \] 
    \item (Pullback) \(f^\ast : \Hom(L, W) \longrightarrow \Hom(V, W)\) with
      the mapping \(\alpha \longmapsto \alpha \circ f\), so that the diagram
      commutes
      \[
        \begin{tikzcd}
          V \ar[r, "f"] 
            &L \ar[d, "\alpha"] \\
            &W \ar[ul, dashed, "f^\ast(\alpha) := \alpha \circ f"]
        \end{tikzcd}
      \] 
  \end{enumerate}
\end{definition}

\begin{proposition}
  Given \(V_1, V_2, V_3\), \(k\)-vector spaces, the following holds for any
  given \(k\)-vector space \(L\):
  \begin{enumerate}[I.]
    \item If the sequence \(0 \to V_1 \to V_2 \to V_3\) is exact, then 
      \[
          \Hom(L, 0) = 0 \to \Hom(L, V_1) \to \Hom(L, V_2) \to \Hom(L, V_3)
      \] 
      is an exact sequence, that is, covariant \(\Hom\) is left exact.
    \item If the sequence \(V_1 \to V_2 \to V_3 \to 0\) is exact, then 
      \[
          \Hom(0, L) = 0 \to \Hom(V_3, L) \to \Hom(V_2, L) \to \Hom(V_1, L)
      \] 
      is an exact sequence, that is, contravariant \(\Hom\) is left exact.
  \end{enumerate}
\end{proposition}

\begin{proof}
  For each of the propositions, we'll denote the morphisms \(f: V_1 \to V_2\)
  and  \(g: V_2 \to V_3\) such that \(\ker g = \im f\).

  For the first proposition, let \(0 \to V_1 \to V_2 \to V_3\) be a an exact
  sequence, and consider the sequence \(0 \to \Hom(L, V_1) \xrightarrow{f_*}
  \Hom(L, V_2) \xrightarrow{g_*} \Hom(L, V_3)\). Let \(\beta \in \im f_*\), then
  there exists \(\alpha \in \Hom(L, V_1)\) such that \(f_*(\alpha) = f \circ
  \alpha = \beta\), hence \(g_*(\beta) = g \circ \beta = g \circ f \circ \alpha
  = 0\) since \(\ker g = \im f\), which implies in \(\im f_* \subseteq \ker
  g_*\).  Suppose now that \(\beta \in \ker g_*\), so that \(g_*(\beta) = g
  \circ \beta = 0\), then we find that \(\im \beta \subseteq \ker g = \im f\)
  and hence \(\beta \in \im f_*\), which implies in \(\ker g_* \subseteq \im
  f_*\).  Therefore \(\ker g_* = \im f_*\).

  For the second proposition, let \(V_1 \to V_2 \to V_3 \to 0\) be a an exact
  sequence and consider the sequence \(0 \to \Hom(V_3, L) \xrightarrow{g^*}
  \Hom(V_2, L) \xrightarrow{f^*} \Hom(V_1, L)\). Let \(\gamma \in \im g_*\) and
  consider \(\lambda \in \Hom(V_3, L)\) such that \(g^*(\lambda) = \lambda \circ
  g = \gamma\). Then we find that \(f^*(\gamma) = \gamma \circ f = \lambda \circ
  g \circ f = \lambda \circ 0 = 0\) hence \(\gamma \in \ker f^*\), which implies
  in \(\im g^* \subseteq \ker f^*\).

  Let \(\gamma \in \ker f^*\) so that \(f^*(\gamma) = \gamma \circ f = 0\), this
  implies in \(\im f = \ker g \subseteq \ker \gamma\). From \cref{thm: universal
  property for quotients} we find that \(\gamma\) induces a morphism
  \(\overline\gamma\) such that the following diagram commutes
  \[
    \begin{tikzcd}
      V_2 \ar[r, "\gamma"] \ar[d, "\pi"] &L \\
      V/\ker g \ar[ur, swap, "\overline\gamma"]
    \end{tikzcd}
  \] 
  so that \(\gamma = \overline\gamma \circ \pi\). Moreover, since \(g\) is
  surjective we have \(\im g = V_3\), and from \cref{thm: first isomorphism} we
  find that \(g\) induces an isomorphism \(\overline g: V_2/\ker g \isoto
  V_3\), so that the following diagram commutes
  \[
    \begin{tikzcd} 
      V_2 \ar[r, "g"] \ar[d, "\pi"] &V_3 \\
      V_2/\ker g \ar[ru, swap, "\overline g"]
    \end{tikzcd}
  \] 
  so that \(g = \overline g \circ \pi\) and since \(\overline g\) is an
  isomorphims, then \(\pi = g \circ \overline g^{-1}\). Notice now that
  \(g^*(\overline \gamma \circ \overline g^{-1}) = \overline\gamma \circ
  \overline g^{-1} \circ g = \overline\gamma \circ \pi = \gamma\) and therefore
  \(\gamma \in \im g^*\). This shows that \(\ker f^* \subseteq \im g^*\) and
  therefore \(\ker f^* = \im g^*\).
\end{proof}

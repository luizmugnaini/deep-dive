% \section{Topological Vector Spaces}

% \begin{definition}
% \label{def:topological-vector-space}
% A topological vector space is a \(k\)-vector space together with a topology
% such that addition of vectors and the product by scalars are both continuous
% \(k\)-linear maps.

% We denote by \(\TVect_{\R}\) the category consisting of topological
% \(\R\)-vector spaces together with morphisms, which are continuous \(\R\)-linear
% maps (which may also be referenced to by the term ``top-linear'').

% Let \(E\) be a topological vector space. The continuous \(\R\)-linear maps
% corresponding to the dual space \(E^{*} = \Hom_{\TVect_{\R}}(E, \R)\), of a
% topological \(\R\)-vector space \(E\), are called \(\R\) \emph{forms}. The
% collection of forms of the form \(E \to \R\) will be conveniently separated in
% classes and denoted:
% \begin{itemize}\setlength\itemsep{0em}
% \item \(L(E)\): the collection of continuous linear maps \(E \to \R\).
% \item \(L^r(E)\): the collection of continuous \(r\)-multilinear maps \(E^r \to \R\).
% \item \(L^r_{\Sym}(E)\): the collection of continuous \(r\)-multilinear
%   symmetric maps \(E^r \to \R\).
% \item \(L^r_{\Alt}(E)\): the collection of continuous \(r\)-multilinear
%   alternating maps \(E^r \to \R\).
% \end{itemize}
% \end{definition}

% \begin{definition}[Locally convex]
% \label{def:locally-convex}
% A topological vector space \(E\) is said to be locally convex if, for every open
% set \(U \subseteq E\), any pair of points \(x, y \in U\) are such that \(t x +
% (1 - t) y \in U\) for all \(t \in [0, 1]\).
% \end{definition}

% \begin{definition}[Banachable]
% \label{def:banachable}
% A topological \(\R\)-vector space \(E\) is said to be banachable if \(E\) is
% complete and its topology can be defined by a norm.
% \end{definition}

% As a point of order, \emph{every time} we mention a topological \(\R\)-vector
% space in the course of this chapter, we shall mean a \emph{banachable space}.

% \begin{definition}[Norm of a morphism]
% \label{def:norm-morphism-TopVect}
% Let \(E\) and \(F\) be topological \(\R\)-vector spaces. In order to make
% \(\Hom_{\TVect_{\R}}(E, F)\) into a topological \(\R\)-vector space, we can
% construct a norm for which, given a morphism \(A: E \to F\), define \(K \coloneq
% \{k \in \R \colon \norm{Ax}_{F} \leq k \norm{x}_E \text{, for all } x \in E\}\), the
% norm of \(A\) is
% \[
%   \norm{A} \coloneq \sup_{k \in K} k.
% \]

% If \(\Hom_{\TVect_{\R}}(E_1, \dots, E_n; F)\) is the collection of continuous
% \(\R\)-multilinear maps, then we define similarly the norm of a continuous
% multilinear map \(B: \prod_{j=1}^n E_j \to F\) as
% \[
%   \norm{B} \coloneq \sup_{m \in M} m,
% \]
% where \(M \coloneq \{m \in \R \colon \norm{Bx}_F \leq m \prod_{j=1}^n
% \norm{x_j}_{E_j} \text{, for all } x \in E\}\).
% \end{definition}

% \begin{remark}
% \label{rm:Cp-morphism}
% From now on, \emph{\(C^p\)-morphism} will refer to a map \(f: U \to V\) between
% open subsets of Banach spaces such that \(f\) is a continuous map of class
% \(C^p\), where \(p \leq \infty\).
% \end{remark}

\section{Differentiable Manifolds}

\subsection{Charts \& Atlases}

\begin{definition}[Atlas]
\label{def:Cp-atlas}
Let \(X\) be an \(n\)-dimensional topological manifold. An \emph{atlas} of class
\(C^p\) on \(X\) is a collection of charts \(\{(U_j, \phi_j)\}_{j \in J}\) such
that
\begin{enumerate}[(a)]\setlength\itemsep{0em}
\item For all \(j \in J\), \(U_j \subseteq X\) is an open set, and
  the collection \(\{U_j\}_{j \in J}\) is an \emph{open cover} for \(X\).

\item For every \(j \in J\), \(\phi_j: U_j \to V_j\) is a \emph{topological
  isomorphism} from the open set \(U_j \subseteq X\) to an open set \(V_j
  \subseteq R^n\).

\item For each pair \(i, j \in J\), the induced \emph{change of coordinates} map
  \[
  \phi_j \phi_i^{-1}: \phi_i(U_i \cap U_j) \isoto \phi_j(U_i \cap U_j)
  \]
  is of class \(C^p\) --- that is, every chart of the atlas with intersecting
  domain is \emph{compatible}.
\end{enumerate}
\end{definition}

\begin{definition}[Chart \& atlas compatibility]
\label{def:compatible-chart}
Let \(X\) be a topological \(n\)-manifold. If \((U, \phi: U \isoto V)\) and
\((U', \psi: U' \isoto V')\) are \emph{charts} on \(X\), we say that they are
\emph{\(C^p\)-compatible} if the two induced transition maps
\[
\phi \psi^{-1}: \psi(U \cap U') \longrightarrow \phi(U \cap U')
\quad\text{ and }\quad
\psi \phi^{-1}: \phi(U \cap U') \longrightarrow \psi(U \cap U')
\]
are of class \(C^p\).

From this definition, we say that a chart is said to be compatible with a given
atlas if it is compatible with every chart of the atlas.  Moreover, given two
atlases, we say that they are compatible if every chart of one is compatible
with the other atlas.
\end{definition}

\begin{lemma}
\label{lem:compatible-charts-with-atlas-are-compatible}
Let \(\mathcal{A} \coloneq \{(U_j, \phi_j)\}_{j \in J}\) be an atlas on a
topological manifold \(X\). If both \((V, \psi)\) and \((W, \sigma)\) are charts
of \(X\) compatible with the atlas \(\mathcal{A}\), then they are compatible
with each other.
\end{lemma}

\begin{proof}
Let \(p \in V \cap W\) be any point and let \(j \in J\) be such that
\(p \in U_j\) --- thus \(p \in V \cap W \cap U_j\). Since \(\phi_j \psi^{-1}\)
and \(\sigma \phi_j^{-1}\) are \(C^p\) maps, then
\(\sigma \psi = (\sigma \phi_j^{-1}) \circ (\phi_j \psi^{-1})\) is \(C^p\) when
restricted to \(\psi(V \cap W \cap U_j)\). Moreover, since
\(\psi(p) \in \psi(V \cap W \cap U_j)\), it follows that \(\sigma \psi\) is
\(C^p\) on \(\psi(p)\) --- therefore \(\sigma \psi\) is \(C^p\) on every point
of its domain since \(p\) was chosen arbitrarily. The same analogous argument
can be used to show that \(\psi \sigma^{-1}\) is \(C^p\).
\end{proof}

\begin{proposition}
\label{prop:compatible-atlas-equivalence-relation}
The compatibility of atlases form an \emph{equivalence} relation.
\end{proposition}

\begin{proof}
Clearly reflexivity and symmetry are satisfied. Let
\(\mathcal{U} \coloneq \{(U_{j}, \phi_j)\}_{j \in J}\) and
\(\mathcal{V} \coloneq \{(V_{i}, \psi_i)\}_{i \in I}\) be two compatible atlases
for some topological manifold \(X\). If
\(\mathcal{A} \coloneq \{(A_{s}, \mu_s)\}_{s \in S}\) is another atlas for
\(X\), which happens to be compatible to \(\mathcal{U}\), then for every
\(s \in S\) the maps \(\phi_j \mu_{s}^{-1}\) and \(\mu_s \phi_j^{-1}\)
are of class \(C^p\) for any \(j \in J\). Since \(\phi_j \psi_i^{-1}\) and
\(\psi_i \phi_j^{-1}\) are \(C^p\) for all \(i \in I\), then in particular
\[
\mu_s \psi_i^{-1} = (\mu_s \phi_j^{-1}) \circ (\phi_j \psi_i^{-1})
\quad\text{ and }\quad
\psi_i \mu_s^{-1} = (\psi_i \phi_j^{-1}) \circ (\phi_j \mu_s^{-1})
\]
are both maps of class \(C^p\). Therefore we conclude that \(\mathcal{A}\) is
compatible with \(\mathcal{V}\).
\end{proof}

\begin{corollary}
\label{cor:unique-maximal-atlas}
Any atlas on a topological manifold is contained in a \emph{unique maximal
  atlas} --- an atlas is said to be maximal if it isn't contained in any atlas
other than itself.
\end{corollary}

\subsection{\texorpdfstring{\(C^p\)}{Cp}-Manifolds}

In this chapter we shall mostly consider the case of \(C^{\infty}\)-manifolds,
also called \emph{smooth} manifolds, but for generality we shall define
differentiable manifolds for all \(p \in \N \cup \{\infty\}\).

\begin{definition}[\(C^p\)-manifold structure on \(X\)]
\label{def:Cp-manifold}
The equivalence classes of atlases of class \(C^p\) on a topological space \(X\)
define what is called a \emph{\(C^p\)-manifold structure on} \(X\).
\end{definition}

We now give another definition of a \(C^p\)-manifold structure on topological
spaces, to do that, we first introduce the following concept.

\begin{definition}[Functionally structured space]
\label{def:functionally-structured-space}
Let \(X\) be a topological space. A \emph{functional structure} on \(X\) is a
map \(F_X\) on the collection of open sets of \(X\) such that, for any open set
\(U \subseteq X\), we have:
\begin{enumerate}[(a)]\setlength\itemsep{0em}
\item \(F_X(U)\) is a
  \emph{subalgebra} of \(C(U)\), the algebra of continuous real valued maps
  on \(U\).

\item \(F_X(U)\) contains all constant maps.

\item If \(V \subseteq U\) is another open set of \(X\), and \(f \in F_X(U)\),
  then \(f|_V \in F_X(V)\).

\item If \(U = \bigcup_{j \in J} U_j\), and \(f: U \to \R\) is a continuous map
  such that \(f|_{U_j} \in F_X(U_j)\) for each \(j \in J\), then it follows that
  \(f \in F_X(U)\).
\end{enumerate}
The pair \((X, F_X)\) is called a \emph{functionally structured space}.

Let \(U \subseteq X\) be open. For any open set \(V \subseteq U\) we define
\[
F_U(V) \coloneq F_X(V),
\]
and hence \((U, F_U)\) is a functionally structured space.
\end{definition}

\begin{definition}[Morphisms of functionally structured spaces]
\label{def:morphism-functionally-structured-spaces}
A morphism
\[
\phi: (X, F_X) \to (Y, F_Y)
\]
between functionally structured spaces is a map \(\phi: X \to Y\) such that, for
any open set \(V \subseteq Y\) and \(f \in F_Y(V)\), we have
\(f \phi \in F_X(\phi^{-1}(V))\).
\end{definition}

\begin{definition}[Second definition of a \(C^p\)-manifold]
\label{def:second-def-Cp-manifold}
An \(n\)-dimensional differentiable manifold is second countable functionally
structured Hausdorff space \((M, F)\) which is locally isomorphic to
\((\R^n, C^p)\).

The local isomorphism is equivalent to the requirement that, for each
point \(p \in M\), there exists a neighbourhood \(U \subseteq M\) of \(p\) such
that \((U, F_U) \iso (V, C_V^p)\) as functionally structured spaces --- for
some open set \(V \subseteq \R^n\).
\end{definition}

\begin{lemma}
\label{lem:morphism-func-struc-is-Cp}
Let \(U, V \subseteq R^n\) be open subspaces. An isomorphism between
functionally structured spaces
\[
\phi: (U, C_U^p) \isoto (V, C_V^p)
\]
is a map \(\phi: U \to V\) of class \(C^p\) if and only if \(f \phi \in C^p(U)\)
for all \(f \in C^p(V)\).
\end{lemma}

\begin{proof}
Clearly, if \(\phi: U \to V\) is of class \(C^p\) then \(f \phi\) is a
composition of \(C^p\) maps, thus \(f \phi \in C^p(U)\) for any \(f \in
C^p(V)\).

Conversely, if we have the hypothesis that \(f \phi \in C^p(U)\) for all
\(f \in C^p(V)\), one may consider the canonical projections
\(\pi_j: \R^n \epi \R\) and notice that \(\pi_j \phi \in C^p(U)\). Therefore
each component of \(\phi\) is a \(C^p(U)\) map, implying that \(\phi\) itself is
a \(C^p(U, V)\) map.
\end{proof}

\begin{lemma}[Equivalence of the definitions]
\label{lem:equivalence-def-Cp-man}
The constructions on \cref{def:Cp-manifold} and
\cref{def:second-def-Cp-manifold} are equivalent.
\end{lemma}

\begin{proof}
Let \((M, F)\) be a \(C^p\)-manifold in the sense of
\cref{def:second-def-Cp-manifold}. A chart on \(M\) will be interpreted as a
pair \((U, \phi: U \isoto V)\) such that \(\phi\) is an isomorphism
\((U, F) \iso (V, C_V^p)\) of functionally structured spaces. Since every point
of \(M\) has a neighbourhood from which one can define the above mencioned
isomorphism, we see that these charts do cover the whole space \(M\).

It remains to be proven that the transition maps are \(C^p\). From our
interpretation of chart, given any two charts
\(\phi: (U, F_U) \isoto (V, C_V^p)\) and
\(\psi: (U', F_{U'}) \isoto (V', C_{V'}^p)\) in \(M\), since \(\phi \psi^{-1}\)
and \(\psi \phi^{-1}\) are isomorphisms of functionally structured spaces
\((V, C_V^p)\) and \((V', C_{V'}^p)\) --- by \cref{lem:morphism-func-struc-is-Cp}
they are \(C^p\) maps. Therefore, our collection of charts match the
requirements of \cref{def:Cp-atlas}.

For the converse, let \(M\) be a \(C^p\)-manifold with an atlas
\(\mathcal{A}\). For every chart \((U, \phi: U \isoto V) \in \mathcal{A}\)
(where \(\phi\) is a topological morphism), define
\[
F(U) \coloneq \{f \phi \in C^{p}(U) \colon f \in C^p(V)\}.
\]
Let \(x \in M\) be any point and consider a chart \((U, \psi: U \isoto V)
\in \mathcal{A}\), where \(U \subseteq M\) is a neighbourhood of \(x\). Notice
that \(\phi\) naturally induces a morphism of functionally structured spaces
\(\phi: (U, F_U) \to (V, C_V^p)\) --- moreover, since \(\phi\) is a topological
isomorphism, then \(\phi\) is an isomorphism \((U, F_U) \iso (V, C_V^p)\).
\end{proof}

\begin{definition}[Smooth map]
\label{def:smooth-map-between-manifolds}
Let \(M\) and \(N\) be smooth manifolds of dimension \(m\) and \(n\),
respectively. A map\(f: M \to N\) is said to be \emph{smooth at a point
  \(p \in M\)} if there exists a chart \((V, \psi)\) about \(f(p)\) in \(N\),
and a chart \((U, \phi)\) about \(p\) in \(M\) such that the map
\[
\psi f \phi^{-1}: \phi(f^{-1}(V) \cap U) \to \R^m
\]
is smooth at \(\phi(p)\). Naturally, \(f\) is said to be \emph{smooth} when
\(f\) is smooth at every point of \(M\).
\end{definition}

\begin{lemma}
\label{lem:equiv-smooth-map-definition}
Let \((M, F_M)\) and \((N, F_N)\) be smooth manifolds. A map \(f: M \to N\) is
smooth in the sense of \cref{def:smooth-map-between-manifolds} if and only if
\(f\) is smooth in the sense of
\cref{def:morphism-functionally-structured-spaces}.
\end{lemma}

\begin{proof}
Assume that \(M\) and \(N\) are, respectively, \(m\) and \(n\)-dimensional
spaces. First we consider \(f\) as a morphism of functionally structured spaces.
Let \(p \in M\) be any point and consider charts, in the sense of isomorphisms
of functionally structured spaces:
\begin{itemize}\setlength\itemsep{0em}
\item \(\phi: U \isoto V\) in \(M\) --- where \(U\) is a neighbourhood of \(p\)
  and \(V \subseteq R^m\)

\item \(\psi: U' \isoto V'\) in \(N\) --- where \(U'\) is a neighbourhood of
  \(f(p)\) and \(V' \subseteq \R^n\).
\end{itemize}
Since \(\phi\) is an isomorphism, one can consider its inverse and use the
property that, for any open set \(S \subseteq V\) and map \(g \in F_M(S)\), then
\(g \phi^{-1} \in C^{\infty}(\phi(S))\). Since \(f\) is a morphism of functionally
structured spaces, in particular \(\pi_j \psi \in F_N(U')\) for all \(1 \leq j
\leq n\), then \(\pi_j \psi f \in F_M(f^{-1}(U'))\). Therefore, since \(p \in
U\) and \(f(p) \in U'\), the intersection \(f^{-1}(U') \cap U \subseteq M\) is
non-empty and we can conclude that
\[
\pi_j \psi f \phi^{-1} \in C^{\infty}(\phi(f^{-1}(U') \cap U)).
\]
Since this is the case for every \(1 \leq j \leq n\), it follows that
\(\psi f \phi^{-1}\) is \(C^{\infty}\) --- therefore \(f\) is smooth in the
sense of \cref{def:smooth-map-between-manifolds}.

For the converse, suppose \(f: M \to N\) is smooth as in
\cref{def:smooth-map-between-manifolds}. From the last property of the
functional structures on spaces, we can simply consider a chart \(\psi: U' \to
V'\) in \(N\) and define a functional structure on \(N\) as
\[
F_N(U') \coloneq \{g \phi \in C^{\infty}(U') \colon g \in C^{\infty}(V')\}.
\]
If \(h \in F_N(U')\) is any map, then the analogous structure \(F_M\) on \(M\)
has
\[
F_M(f^{-1}(U')) = \{w f \in C^{\infty}(f^{-1}(U')) \colon w \in C^{\infty}(U')\},
\]
therefore \(h f \in F_M(f^{-1}(U'))\) as wanted.
\end{proof}

\begin{definition}[Category of \(C^{\infty}\)-manifolds]
\label{def:smooth-manifolds-category}
We define \(\Man\) to be the category of smooth manifolds and smooth morphisms
between them.
\end{definition}

\begin{corollary}
\label{cor:isomorphism-man}
An isomorphism in the category \(\Man\) is a smooth morphism of manifolds with a
smooth inverse. Some call these isomorphisms by ``diffeomorphisms'', we shall
call them \(C^{\infty}\)-isomorphisms.
\end{corollary}

\subsubsection{Local Coordinates}

Let \(M\) be an smooth \(n\)-manifold and \(f: M \to \R\) be any real valued
map. If \(x: U \to V\) is a chart of \(M\), we consider the induced map
\(\overline{f} \coloneq f x^{-1}: V \to \R\), a multivariable real map from a
subset of \(\R^n\) to \(\R\). For any \(p \in U\), one has
\emph{``coordinates''} \(x(p) = (x_1(p), \dots, x_n(p))\). These coordinates can
be used to compute \(f\) via
\[
f(p) = \overline{f}(x_1(p), \dots, x_n(p)),
\]
working as some kind of local coordinates on the open set \(U\). This an
intuitive view of what the following definition states.

\begin{definition}[Local coordinates]
\label{def:local-coordinates}
If \(X\) is an \(\R^n\) modeled manifold, and \(\phi: U \to \R^n\) is a chart,
where \(U \subseteq X\) is open, then, we say that the collection
\(\{\phi_j\}_{j=1}^n\) are \emph{local coordinates} for \(X\) on \(U\).
\end{definition}

\section{Structures}

\begin{definition}[Induced structure]
\label{def:induced-functional-structure-space}
Let \((X, F_X)\) be a functionally structured space, and \(\phi: X \to Y\) be a
continuous map, where \(Y\) is a topological space. We define the \emph{induced
  functional structure} on \(Y\) via \(F_X\) and \(\phi\) to be given by
\[
F_Y(U) \coloneq \{f \in C(U) \colon f \phi \in F_X(\phi^{-1}(U))\},
\]
for any open set \(U \subseteq Y\).
\end{definition}

\begin{definition}[Induced structure on subspace]
\label{def:induced-func-struc-on-subspace}
Let \((X, F)\) be a functionally structured space, and \(A \subseteq X\) be a
subspace. We construct a functional structure \(F_A\) on \(A\) as follows: for
all open sets \(U \subseteq A\), a continuous map \(f: U \to \R\) is contained
in \(F_A(U)\) if and only if for every \(p \in U\) there exists a neighbourhood
\(W \subseteq X\) of \(p\) such that \(f\) is the restriction to \(W \cap A\) of
some map \(g \in F(W)\).
\end{definition}

\begin{definition}[Binary product of manifolds]
\label{def:product-manifold}
Let \(M\) and \(N\) be manifolds of dimension \(m\) and \(n\), respectively. If
\(\phi: U \to \R^m\) is a chart for \(M\) and \(\psi: V \to \R^n\) is a chart
for \(N\), we take \(\phi \times \psi: U \times V \to \R^{m + n}\) to be a chart
for the product space \(M \times N\). The maximal atlas consisting of such
product charts make \(M \times N\) into a product manifold of dimension
\(m + n\).
\end{definition}

\begin{definition}[Submanifold]
\label{def:submanifold}
Let \(M\) be an \(n\)-manifold. A subset \(N \subseteq M\) is said to be a
\emph{\(k\)-manifold} of \(M\) if for each \(p \in N\) there exists a chart
\(\phi: U \to V\) of \(M\) about \(p\) for which
\[
\phi(U \cap N) = V \cap (\R^k \times \{0\}).
\]
Charts with this property are called \emph{adapted} to \(N\). The maximal atlas
containing all adapted charts to \(N\) makes \(N\) into a manifold.

We define the \emph{codimension} of \(N\) \emph{in} \(M\) to be the difference
\(n - k\). The submanifold \(N\) is said to be \emph{smooth} exactly when about
each point there exists an adapted smooth chart to \(N\) --- so that one can
find a unique maximal smooth atlas for \(N\).
\end{definition}

\begin{definition}[Smooth embedding]
\label{def:smooth-embedding}
A \(C^{\infty}\)-morphism \(f: N \to M\) is said to be a \emph{smooth embedding}
if the image \(f(N) \subseteq M\) is a \emph{smooth submanifold} of \(M\) and
the induced restriction \(f: N \isoto f(N)\) is a \(C^{\infty}\)-isomorphism.
\end{definition}

\begin{example}[Sphere]
\label{exp:sphere-smooth-manifold}
Consider the sphere \(S^n \subseteq \R^{n+1}\), we'll define a smooth atlas on
\(S^n\) using solely two charts. To that end, define two points
\(n \coloneq (0, \dots, 0, 1)\) and \(s \coloneq (0, \dots, 0, -1)\) in
\(\R^{n+1}\) --- these correspon to the north and south poles of \(S^n\),
respectively. Consider the \emph{stereographic projection}
\(\phi_n: S^n \setminus \{n\} \to \R^n\), from the sphere with its north pole
cut out to the hyperplane \(R^n \iso \R^n \times \{0\}\) --- the stereographic
projection maps each point \(x \in S^n\) to the intersection of the line passing
through \(x\) and \(n\), and the hyperplane above-mencioned. The inverse map of
\(\phi_n\) is \(\pi_n: \R^n \to S^n \setminus \{n\}\), mapping
\[
x \longmapsto \frac{(2 x, \| x \|^2 - 1)}{(1 + \| x \|)^2}.
\]
We define the south stereographic projection \(\phi_s: S^n \setminus \{s\} \to
\R^n\) equivalently. This construction has a smooth transition map
\[
\phi_s \phi_n^{-1}(x) = \frac{y}{\| y \|^2}.
\]
\end{example}

\begin{proposition}
% \label{prop:}
Let \(M\) be an \(n\)-manifold and \(\mathcal{U} \coloneq (U_j)_{j \in J}\) be
an open covering of \(M\). Then there exists a collection of charts \((\phi_k:
V_k \to B_k)_{k \in \N}\) such that the following properties hold:
\begin{enumerate}\setlength\itemsep{0em}
\item For all \(k \in \N\), the set \(V_k\) is contained in some member of
  \(\mathcal{U}\).
\item For all \(k \in \N\), we have \(B_k = \{x \in \R^n \colon \| x \| < 3\}\)
  --- the open \(n\)-ball of radius \(3\).
\item The collection of open sets \((V_k)_{k \in \N}\) is a locally finite
  covering of \(M\).
\end{enumerate}
\end{proposition}

\todo[inline]{Continue after some point-set topology}

\section{Oriented Manifolds}

\begin{definition}[Oriented manifold]
\label{def:oriented-manifolds}
A manifold \(M\) with atlas \(\mathcal{A}\) such that, for every pair of charts
\(\phi, \psi \in \mathcal{A}\), the Jacobian of the transition map
\(\phi \psi^{-1}\) has \emph{positive determinant} for all points of the domain,
then \(M\) is called a \emph{oriented manifold}. A maximal atlas on \(M\) with
such a property is said to be an \emph{orientation} of the manifold, and \(M\)
is called \emph{orientable}.
\end{definition}

\begin{proposition}[Two orientations]
\label{prop:two-orientations-man}
Let \(M\) be a connected, orientable, smooth manifold. Then \(M\) admits exactly
two orientations.
\end{proposition}

\todo[inline]{Prove}

%%% Local Variables:
%%% mode: latex
%%% TeX-master: "../../../deep-dive"
%%% End:

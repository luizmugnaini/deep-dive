\section{Tensor Products}

\begin{definition}[Biadditive and bilinear maps]
    \label{def:biadditive-and-bilinear-maps}
    Let \(R\) be a ring and consider a right \(R\)-module \(A\), and a left
    \(R\)-module \(B\), and an abelian group \(G\). A map \(f: A \times B \to G\) is
    said to be \emph{\(R\)-biadditive} if, for any pairs \((a, b), (a', b') \in A \times
    B\) and scalar \(r \in R\) we have the following conditions satisfied:
    \begin{enumerate}[(a)]\setlength\itemsep{0em}
        \item \(f(a + a', b) = f(a, b) + f(a', b)\).
        \item \(f(a, b + b') = f(a, b) + f(a, b')\).
        \item \(f(a r, b) = f(a, r b)\).
    \end{enumerate}
    Further, if \(R\) is \emph{commutative}, we say that \(f\) is
    \emph{\(R\)-bilinear} if it also satisfies:
    \begin{enumerate}[(a)]\setlength\itemsep{0em}\setcounter{enumi}{3}
        \item \(f(a r, b) = f(a, rb) = r f(a, b)\).
    \end{enumerate}

    Now let \(A\) be a \(k\)-algebra, where \(k\) is a commutative ring, and
    consider modules \(L_A\) and \({}_AM\). Then we define a morphism of
    \(k\)-modules \(g: L \times M \to X\) to be \(A\)-bilinear if
    \begin{itemize}\setlength\itemsep{0em}
        \item \(g(\ell \alpha + \ell' \beta, m) = g(\ell, m) \alpha + g(\ell', m) \beta\), for any \(\alpha, \beta \in k\).
        \item \(g(\ell, m \alpha + m' \beta) = g(\ell, m) \alpha + g(\ell, m') \beta\), for any \(\alpha, \beta \in k\).
        \item \(g(\ell a, m) = g(\ell, a m)\).
    \end{itemize}
\end{definition}

\begin{definition}[Tensor product]
    \label{def:tensor-product}
    Let \(R\) be a ring and consider right and left \(R\)-modules \(A_R\) and
    \({}_RB\). The \emph{tensor product} of \(A\) and \(B\) is an abelian group
    \(A \otimes_R B\) together with an \(R\)-biadditive map
    \(t: A \times B \to A \otimes_R B\) such that, for every abelian group \(G\) and
    any \(R\)-biadditive map \(f: A \times B \to G\), there exists a unique abelian
    group morphism \(\phi: A \otimes_R B \to G\) such that the following triangle
    commutes:
    \[
        \begin{tikzcd}
            A \times B \ar[d, "t"'] \ar[r, "f"] &G \\
            A \otimes_R B \ar[ru, dashed, "\phi"', bend right] &
        \end{tikzcd}
    \]
\end{definition}

\begin{lemma}[Uniqueness up to isomorphism]
    \label{lem:tensor-product-unique-up-to-iso}
    Let \(X\) and \(Y\) be tensor products for \(R\)-modules \(A_R\) and \({}_RB\),
    then there exists a natural isomorphism \(X \iso Y\).
\end{lemma}

\begin{proof}
    Let \(t_X: A \times B \to X\) and \(t_Y: A \times B \to Y\) be their
    respectively associated \(R\)-biadditive map. Consider the universal property of
    the tensor product applied to both \((X, t_X)\) and \((Y, t_Y)\): we obtain
    unique morphisms of groups \(\phi_1: X \to Y\) and \(\phi_2: Y \to X\) such that
    the following diagram commutes
    \[
        \begin{tikzcd}
            X \ar[rd, dashed, bend right, "\phi_1"']
            &A \times B \ar[l, "t_X"']
            \ar[d, "t_Y"]
            \ar[r, "t_X"]
            &X
            \\
            &Y \ar[ru, "\phi_2"', dashed, bend right] &
        \end{tikzcd}
        \qquad
        \qquad
        \begin{tikzcd}
            Y \ar[rd, dashed, bend right, "\phi_2"']
            &A \times B \ar[l, "t_Y"']
            \ar[d, "t_X"]
            \ar[r, "t_Y"]
            &Y
            \\
            &X \ar[ru, "\phi_1"', dashed, bend right] &
        \end{tikzcd}
    \]
    From the left diagram we see that \(\Id_X: X \to X\) also makes the diagram
    commute and on the right diagram \(\Id_Y: Y \to Y\) is also a possible morphism
    satisfying commutativity. Since the morphisms \(\phi_1\) and \(\phi_2\) are
    uniquely chosen, then the compositions \(\phi_1 \phi_2\) and \(\phi_2 \phi_1\)
    are also unique. Therefore it follows that \(\phi_1 \phi_2 = \Id_Y\) and
    \(\phi_2 \phi_1 = \Id_X\)---hence \(X \iso Y\).
\end{proof}

\begin{lemma}[Existence]
    \label{lem:tensor-product-exists}
    Given a ring \(R\) together with right and left \(R\)-modules \(A_R\) and
    \({}_RB\), their tensor product \(A \otimes_R B\) exists.
\end{lemma}

\begin{proof}
    Let \(F\) be the free abelian group with basis \(A \times B\), and consider the
    subgroup \(S\) of \(S\) generated by elements of the following form, for any
    \((a, b), (a', b') \in A \times B\) and \(r \in R\):
    \begin{itemize}\setlength\itemsep{0em}
        \item \((a, b + b') - (a, b) - (a, b')\),
        \item \((a + a', b) - (a, b) - (a', b)\),
        \item \((a r, b) - (a, r b)\).
    \end{itemize}
    We'll show that the quotient \(F/S\) is the wanted tensor product
    \(A \otimes_R B\). Define the map \(t: A \otimes B \to F/S\)
    naturally by mapping \((a, b) \mapsto (a, b) + S\). From the definition of
    \(S\) we obtain freely that \(t\) is \(R\)-biadditive.

    Let \(G\) be any abelian group and \(f: A \times B \to G\) any \(R\)-biadditive
    map. Then from the free group theorem there exists a unique morphism of groups
    \(\phi: F \to G\) determined by mapping \((a, b) \mapsto f(a, b)\). Moreover,
    \(S \subseteq \ker \phi\) from the fact that we imposed \(R\)-biadditiveness in
    \(f\). Therefore, by the quotient theorem \(\phi\) induces a unique morphism of
    groups \(\psi: F/S \to G\) satisfying \(\psi t = f\):
    \[
        \begin{tikzcd}
            A \times B \ar[rrd, bend left, "f"]
            \ar[d, hook]
            \ar[dd, bend right=50, "t"']
            &&\\
            F \ar[rr, dashed, "\phi"]
            \ar[d, two heads]
            &&G \\
            F/S \ar[rru, bend right, "\psi"', dashed]
            &&
        \end{tikzcd}
    \]
    This shows that \(F/S\) satisfies the universal property of the tensor product
    and thus \(A \otimes_R B \iso F/S\).
\end{proof}

\begin{proposition}
    \label{prop:tensoring-maps}
    Consider a morphism of left \(R\)-modules \(\lambda: A_R \to B_R\) and a
    morphism of right \(R\)-modules \(\rho: {}_RX \to {}_RY\). The maps \(\lambda\)
    and \(\rho\) induce a unique morphism of abelian groups
    \[
        \lambda \otimes \rho: A \otimes_R X \longrightarrow B \otimes_R Y.
    \]
    This morphism is given by the mapping
    \(a \otimes x \mapsto \lambda(a) \otimes \rho(x)\).
\end{proposition}

\begin{proof}
    Let \(\phi: A \times X \to B \otimes_R Y\) be the map given by
    \(\phi(a, b) \coloneq a \otimes b\). Then for any \(r \in R\) one has
    \[
        \phi(a r, b)
        = \lambda(a r) \otimes \rho(b)
        = \lambda(a) r \otimes \rho(b)
        = \lambda(a) \otimes r \rho(b)
        = \lambda(a) \otimes \rho(r b)
        = \phi(a, r b).
    \]
    Thus \(\phi\) is \(R\)-biadditive. The morphism uniquely induced morphism
    \(A \otimes_R X \to B \otimes_R Y\) is exactly \(\lambda \otimes \rho\) as
    described above.
\end{proof}

\begin{corollary}
    \label{cor:composition-tensored-maps}
    Consider right \(R\)-module morphisms \(A \xrightarrow{\rho_1} B
    \xrightarrow{\rho_2} C\) and morphisms of left \(R\)-modules \(X
    \xrightarrow{\lambda_1} Y \xrightarrow{\lambda_2} Z\). Then we have a composition
    \[
        (\rho_2 \otimes \lambda_2) \circ (\rho_1 \otimes \lambda_1)
        = \rho_2 \rho_1 \otimes \lambda_2 \lambda_1.
    \]
\end{corollary}

\begin{lemma}[Additive tensoring functors]
    \label{lem:addive-tensor-functors}
    Let \(A\) be a right \(R\)-module and \(B\) be a left \(R\)-module.
    \begin{enumerate}[(a)]\setlength\itemsep{0em}
        \item There exists an additive functor \(F_A: \lMod{R} \to \Ab\) given by
              \(F_A(B) = A \otimes_R B\) and \(F_A(\beta) = \Id_A \otimes \beta\). We shall
              denote this functor by \(A \otimes_R -\).

        \item There exists an additive functor \(G_B: \rMod{R} \to \Ab\) mapping \(G_B(A) =
              A \otimes_R B\) and \(G_B(\alpha) = \alpha \otimes \Id_B\). This functor will
              be denoted by \(- \otimes_R B\).
    \end{enumerate}
\end{lemma}

\begin{corollary}[Induced isomorphisms]
    \label{cor:isos-induce-iso-tensor}
    Let \(f \in \Hom_{\rMod{R}}(M, N)\) and \(g \in \Hom_{\lMod{R}}(L, K)\) be
    isomorphisms. The induced morphism
    \(f \otimes g: M \otimes_R L \to N \otimes_R K\) is an isomorphism of abelian
    groups.
\end{corollary}

\begin{proof}
    The functor \(- \otimes_R L\) takes \(f\) to the isomorphism \(f \otimes \Id_L\)
    and \(M \otimes_R -\) takes \(g\) to the isomorphism \(\Id_M \otimes g\). The
    composition \((f \otimes \Id_L) (\Id_M \otimes g) = f \otimes g\) will thus be
    an isomorphism.
\end{proof}

\begin{proposition}[Extending scalars]
    \label{prop:extending-scalars-tensor-prod}
    Let \(R\) be a ring and consider a subring \(S \subseteq R\).
    \begin{enumerate}[(a)]\setlength\itemsep{0em}
        \item Let \(A \in \Mod{(R, S)}\), then for any \(B \in \lMod{S}\) the tensor
              product \(A \otimes_S B\) inherits a natural structure of a left \(R\)-module:
              for any \(r \in R\) and \(a \otimes b \in A \otimes_S B\) we have
              \(r(a \otimes b) = (r a) \otimes b\).

        \item Let \(B \in \Mod{(S, R)}\), then for any \(A \in \rMod{S}\) the tensor
              product \(A \otimes_S B\) inherits a natural structure of a right \(R\)-module:
              for any \(r \in R\) and \(a \otimes b \in A \otimes_S B\) we have
              \((a \otimes b) r = a \otimes (b r)\).
    \end{enumerate}
\end{proposition}

\begin{proof}
    We shall prove only the first proposition since the other follows
    analogously. Given any \(r \in R\), consider the left multiplication morphism
    \({}_r\mul: A \to A\) and the morphism of groups \({}_r\mul \otimes \Id_B\)
    under the functor \(- \otimes_S B\). Notice that this map associates
    \(a \otimes b \mapsto {}_r\mul(a) \otimes b = (r a) \otimes b\). It is trivial
    to verify that \({}_r\mul\) induces a multiplicative left \(R\)-module
    structure on the abelian group \(A \otimes_S B\)---transforming it into a left
    \(R\)-module.
\end{proof}

\begin{example}
    \label{exp:extending-scalars-tensor-product}
    If \(R\) is ring and \(S\) is any subring of \(R\), then \(R\) has a structure
    of \((R, S)\)-bimodule and therefore for any given left \(S\)-module \(M\) we
    can obtain a left \(R\)-module \(R \otimes_S M\). That is, the procedure shown
    in \cref{prop:extending-scalars-tensor-prod} allows one to extend the scalars
    associated to a module.
\end{example}



%%% Local Variables:
%%% mode: latex
%%% TeX-master: "../../../deep-dive"
%%% End:

\begin{definition}[Model category]
\label{def:model-category}
Let \(\mcat E\) be a category. We say that \(\mcat E\) has a \emph{model
  structure} if their morphisms are classified into three classes:
\emph{fibrations} (denoted by the arrow \(\epi\)), \emph{cofibrations} (denoted
by \(\mono\)), and \emph{weak equivalences} (denoted by \(\isoto\)), which
satisfy the following properties:
\begin{enumerate}[(a)]\setlength\itemsep{0em}
\item The category \(\mcat E\) is \emph{complete} and
  \emph{cocomplete}\footnote{Has all \emph{small} limits and colimits by
    \cref{def:completeness-categories}.}.

\item Given a commutative diagram
  \[
  \begin{tikzcd}
  X \ar[rr, "f"] \ar[rd, "g"'] & &Z \\
  &Y \ar[ru, "h"']
  \end{tikzcd}
  \]
  in \(\mcat E\), if \emph{any two} of the maps \(f\), \(g\) and \(h\) are weak
  equivalences, then the \emph{third} is a weak equivalence.

\item If \(f\) is a retract---in the category \(\Mor(\mcat E)\)---of a morphism
  \(g\), and \(g\) is a weak equivalence, then \(f\) is also a weak
  equivalence. For a shorter punchline, the classes of fibrations and
  cofibrations is \emph{closed under retracts}.

\item Given a commutative square
  \[
  \begin{tikzcd}
  A \ar[r] \ar[d, tail, "j"'] &E \ar[d, two heads, "p"] \\
  X \ar[ur, dotted] \ar[r] &B
  \end{tikzcd}
  \]
  in \(\mcat E\) where \(j\) is a cofibration and \(p\) is a fibration, the
  dotted morphism \(X \to E\) \emph{exists} when either \(j\) or \(p\) is a
  \emph{weak equivalence}. Said differently, fibrations have the \emph{right
    lifting property} with respect to \emph{trivial cofibrations} (both a
  cofibration and a weak equivalence), while \emph{trivial fibrations} (both a
  fibration and a weak equivalence) have the \emph{right lifting property} with
  respect to cofibrations.

\item Every morphism \(f: X \to Y\) can be factored as
  \[
  \begin{tikzcd}
  X
  \ar[rr, tail, "j"]
  \ar[rrd, "f" description]
  \ar[d, tail, "i"', "\dis"]
  &&Y' \ar[d, two heads, "\dis"', "q"]
  \\
  X' \ar[rr, two heads, "p"']
  &&Y
  \end{tikzcd}
  \]
  where \(j\) is a \emph{cofibration} and \(q\) is a \emph{trivial fibration} ,
  and \(i\) is a \emph{trivial cofibration} and \(p\) is a \emph{fibration}.
\end{enumerate}
The category \(\mcat E\) together with its model structure is said to be a
\emph{model category}.
\end{definition}

%%% Local Variables:
%%% mode: latex
%%% TeX-master: "../../../deep-dive"
%%% End:

\section{Characterizations and Morphisms}

\begin{definition}[Category]\label{def: category}
  A category \(\cat C\) consists of the following data
  \begin{enumerate}[(C1)]
    \item A collection of objects. We say that \(X\) is an object of \(\cat C\)
      by writing \(X \in \cat C\) or \(X \in \Obj(\cat C)\).
    \item For every given pair of objects \(X, Y \in \cat C\) there exists a
      collection of morphisms \(\Hom_{\cat C}(X, Y)\) with source \(X\) and
      target \(Y\). The collection of morphisms between objects of \(\cat C\) is
      denoted \(\Mor(\cat C)\).
    \item For every object \(X \in \cat C\), there exists an identity morphism
      \(\Id_X \in \Hom_{\cat C}(X, X)\).
    \item For every triple of given objects \(X, Y, Z \in \cat C\), there exists
      a composition map
      \[
        \Hom_{\cat C}(X, Y) \times \Hom_{\cat C}(Y, Z) \to \Hom_{\cat C}(X, Z).
      \]
      So that for given morphisms \(f \in \Hom_{\cat C}(X, Y)\) and \(g \in
      \Hom_{\cat C}(Y, Z)\) there exists a uniquely defined map \(g  f \in
      \Hom_{\cat C}(X, Z)\) such that the following diagram commutes
      \[
        \begin{tikzcd}
          Y \ar[r, "g"]
            &Z \\
          X \ar[u, "f"] \ar[ru, dashed, swap, "g  f"]
        \end{tikzcd}
      \]
    \item For every morphism \(f: X \to Y\) we have that
      \[
        \begin{tikzcd}
          X \ar[r, "f"] \ar[loop left, "\Id_X"] &Y \ar[loop right, "\Id_Y"]
        \end{tikzcd}
      \]
      so that \(\Id_Y  f = f = f  \Id_X\).
    \item Given ojects \(W, X, Y, Z \in \cat C\), the following diagram commutes
      \[
        \begin{tikzcd}
          W
          \ar[r, "f"]
          \ar[rr, bend right, swap, "g  f"]
          \ar[rrr, bend right = 50, swap, "h  (g  f)"]
          \ar[rrr, bend left = 50, "(h  g)  f"]
            &X
            \ar[r, "g"] \ar[rr, bend left, "h  g"]
              &Y
              \ar[r, "h"]
                &Z
        \end{tikzcd}
      \]
      that is, \(h  (g  f) = (h  g)  f\).
  \end{enumerate}
\end{definition}

\begin{definition}[Small]\label{def: small cat}
  A category \(\cat C\) is said to be small if it is composed of a
  set's worth of morphisms.
\end{definition}

\begin{definition}[Locally small]
  A category \(\cat C\) is said to be locally small if for any objects \(A,B
  \cat C\) there exists a set's worth of morphisms \(A\) and \(B\).
\end{definition}

\begin{corollary}[Unique identity]\label{cor: unique identity}
  Given a category \(\cat C\) and an object \(c \in \cat C\), the identity
  \(\Id_c \in \Mor(\cat C)\) is unique.
\end{corollary}

\begin{proof}
  Let \(f: c \to c\) be an identity of \(c\), then \(f = f \Id_c = \Id_c\).
\end{proof}

\begin{definition}
  Given a category \(\cat{C}\) and objects  \(A, B \in \cat{C}\), we
  define a morphism \(f \in \Hom(A, B)\) to be an \emph{isomorphism} if
  and only if it has a both sided inverse, so that exists \(f^{-1} \in
  \Hom(B, A)\) such that \(f^{-1}f = \Id_A\) and  \(ff^{-1} =
  \Id_B\).
\end{definition}

\begin{proposition}\label{prop: iso unique inverse}
   Given an isomorphism \(f\), its inverse is unique.
\end{proposition}

\begin{proof}
   Suppose for instance that there are two such functions, \(g, h \in
   \Hom(B, A)\), that act as an inverse for \(f \in \Hom(A,
   B)\). Note that
   \[
      g = g \Id_B = g(f h) = (g f) h = \Id_A h = h
   \]
   Thus \(g = h\) and therefore the inverse is indeed unique.
\end{proof}

\begin{definition}[Monoid]\label{def: monoid}
  A monoid is a set \(M\) equipped with a binary operation \(\otimes: M \times M
  \to M\) and a neutral element \(e \in M\). The binary operation is associative
  and obeys the right and left unit laws, that is
  \[
    x \otimes (y \otimes z) = (x \otimes y) \otimes z \qquad \text{ and } \qquad
    e \otimes x = x = x \otimes e.
  \]
\end{definition}

\begin{example}
  A monoid \(M\) defines a 1-category object category \(\cat{BM}\) such that
  \(\Obj(\cat{BM}) = \{*\}\) and \(\Hom_{\cat{BM}}(*, *) = \{*\}\). Composition
  of morphisms \(f, g: * \to *\) is defined as \(g  f = g \otimes f\). The
  identity morphism is \(\Id_* = e\). Hence we have \(e * f = f = f * e\) for
  all morphisms \(f \in \Mor(\cat{BM})\).
\end{example}

\begin{definition}[Groupoids]\label{def: groupoids}
  A \emph{groupoid} is the name given to a category in which all of its
  morphisms are isomorphisms.
\end{definition}

\begin{definition}[Group]\label{def: group}
  A group is a groupoid with one object.
\end{definition}

\begin{definition}[Automorphism]
  Given a category \(\cat C\) and an object  \(A \in \cat C\), we define an
  automorphism of \(A\) to be an isomorphism \(A \to A\). The set consisting of
  all such automorphisms of this object is denoted \(\Aut(A)\), the
  automorphisms have properties:
  \begin{enumerate}[i.]
    \item The composition of two automorphisms is an automorphism.
    \item Composition is associative.
    \item \(\Id_A \in \Aut(A)\).
    \item Every automorphism \(f \in \Aut(A)\) has an inverse \(f^{-1}
       \in \Aut(A)\).
  \end{enumerate}
  With this, the structure \(\Aut(A)\) is a \emph{group}, for all choices of
  objects \(A \in \cat{C}\).
\end{definition}

\begin{definition}[Subcategory]\label{def: subcategory}
  Let \(\cat C\) be a category. We define \(\cat D \subseteq \cat C\) as a
  subcategory of \(\cat C\) if: \(\Obj(\cat D)\) is a restriction of \(\Obj(\cat
  C)\); for all \(A \in \Obj(\cat D)\) there exists \(\Id_A \in \Mor(\cat D)\);
  for any \(f \in \Mor(\cat D)\) there exists \(\dom(f), \codom(f) \in
  \Obj(\cat D)\); for any composable pair of morphisms \(f, g \in \Mor(\cat D)\)
  there exists \(f g \in \Mor(\cat D)\).
\end{definition}

\begin{lemma}
  Any category \(\cat C\) contains a subcategory containing all of the objects
  and whose morphisms are only the isomorphisms. Such subcategory is called a
  maximal groupoid.
\end{lemma}

\begin{proof}
  We are going to prove that the maximal groupoid, name it \(\cat G\) is indeed
  a subcategory of \(\cat C\). Notice that \(\Obj(\cat G) = \Obj(\cat C)\),
  moreover every identity is an isomorphism, then \(\Id_{*} \in \Mor(\cat G)\).
  Let \(f \in \Mor(\cat G)\) then in particular we have \(f \in \Mor(\cat C)\)
  and hence \(\operatorname{dom} f, \operatorname{codom} f \in \Obj(\cat C) =
  \Obj(\cat G)\). Consider \(f \in \Hom_{\cat G}(A, B)\) and \(g \in \Hom_{\cat
  G}(B, C)\), composable isomorphisms, and notice that \(f^{-1}  g^{-1}
  \in \Hom_{\cat C}(C, A)\) is such that \((f^{-1}  g^{-1})  (g
  f) = \Id_A\) and \((g  f)  (f^{-1}  g^{-1}) = \Id_C\), hence we
  conclude that \(g  f\) is an isomorphism and therefore \(g  f \in
  \Hom_{\cat G}(A, C)\).
\end{proof}

\begin{definition}[Full subcategory]\label{def: full subcategory}
  A subcategory \(\cat D\) of \(\cat C\) (see \cref{def: subcategory}) is said
  to be a full subcategory if for all \(x, y \in \cat D\) we have \(\cat D(x, y)
  = \cat C(x, y)\).
\end{definition}

\begin{proposition}[Slice category]\label{prop: slice cat}
  Given a category \(\cat C\) and an object \(c \in \cat C\). The following
  define categories:
  \begin{enumerate}[(SC1)]
    \item\label{prop: slice under}
      (Slice under \(c\)) A category \(c/\cat C\), called slice category of
      \(\cat C\) under \(c\), whose objects are morphisms \(f \in \Hom_{\cat
      C}(c, *)\). Given objects \(f, g \in c/\cat C\) such that \(f: c \to x\)
      and \(g: c \to y\), we define a morphism \(f \to g\) as a map \(h: x \to
      y\) such that the following diagram commutes
      \[
        \begin{tikzcd}
          &c \ar[dl, swap,"f"] \ar[dr, "g"]\\
          x \ar[rr, "h"] & &y
        \end{tikzcd}
      \]
      that is, \(g = h  f\).
    \item\label{prop: slice over}
      (Slice over \(c\)) A category \(\cat C/c\), called the slice category of
      \(\cat C\) over \(c\), whose objects are morphisms \(f \in \Hom_{\cat
      C}(*, c)\). Morphisms between objects \(f, g \in \cat C/c\) such that \(f:
      x \to c\) and \(g: y \to c\) are maps \(h: x \to y\) such that the
      following diagram commutes
      \[
        \begin{tikzcd}
          x \ar[rr, "h"]
            & &y \\
            &c \ar[ul, "f"] \ar[ur, swap, "g"]
        \end{tikzcd}
      \]
      so that \(f = g  h\).
  \end{enumerate}
\end{proposition}

\begin{proof}
  (SC1) Given objects \(f, g \in c/\cat C\) we have from construction that
  \[
    \Hom_{c/\cat C}(f, g) = \Hom_{\cat C}(\codom f, \codom g),
  \]
  hence we are
  ensured of the existence of such morphisms between objects. Given an object
  \(f: c \to x\), the morphism \(\Id_x \in \Mor(\cat C)\) is such that \(f =
  \Id_x  f\), so that \(\Id_x\) is the identity morphism for \(f\). Let
  \(f, g, u \in c/\cat C\) be objects such that \(f: c \to x,\ g: c \to y,\ u: c
  \to z\), then there exists morphisms \(h \in \Hom_{c/\cat C}(f, u)\) and
  \(\ell: \Hom_{c/\cat C}(u, g)\) so that the following diagram commutes
  \[
    \begin{tikzcd}
        &c \ar[dl, swap, "f"] \ar[d, "u"] \ar[dr, "g"] & \\
      x \ar[r, "h"]
        &z \ar[r, "\ell"]
          &y
    \end{tikzcd}
  \]
  so that we have a uniquely defined morphism \(\ell  h \in \Hom_{\cat
  C}(x, y)\), where \(\ell  h \in \Hom_{c/\cat C}(f, g)\) and
  thus this defines a map \(\Hom_{c/\cat C}(f, u) \times \Hom_{c/\cat C}(u, g)
  \to \Hom_{c/\cat C}(f, g)\). Since \(\Id_x, \Id_z \in \Mor(c/\cat C)\) for any
  \(x, y \in \cat C\), then given \(f, u \in c/\cat C\) just as above, the
  morphism \(h: x \to z\) is such that \(\Id_z  h = h = h  \Id_x\). In
  addition to the objects and morphisms named above, define \(v: c \to w\) and
  the corresponding morphism \(t \in \Hom_{c/\cat C}(g, v)\) so that \(t: y \to
  w\) and \(v = t  g\). From the fact that \(\cat C\) is a category, we
  find that \((t  \ell)  h = t  (\ell  h)\) and hence the
  same is true for \(c/\cat C\).

  (SC2) Let \(f, g \in \cat C/c\), then from definition we have \(\Hom_{\cat
  C/c}(f, g) = \Hom_{\cat C}(\dom f, \dom g)\), which is well defined on \(\cat
  C\). Given an object \(f: x \to c\), there exists \(\Id_x: x \to x\) so that
  \(f = f  \Id_x\) and hence \(\Id_x\) is the identity morphism for \(f\).
  Define objects \(f, g, u \in \cat C/c\) such that \(f: x \to c,\ g: y \to c,\
  u: c \to z\), and morphisms \(h \in \Hom_{\cat C/c}(u, f)\) and \(\ell \in
  \Hom_{\cat C/c}(g, u)\). Then, the following diagram commutes
   \[
    \begin{tikzcd}
      x
        &z \ar[l, swap, "h"]
          &y \ar[l, swap, "\ell"] \\
        &c \ar[lu, "f"] \ar[u, "u"] \ar[ru, swap, "g"]
    \end{tikzcd}
  \]
  hence the morphism \(h  \ell \in \Hom_{\cat C}(y, x)\) defines a morphism
  \(h  \ell \in \Hom_{\cat C/c}(g, f)\) so that we can construct a map well
  defined map \(\Hom_{\cat C/c}(g, u) \times \Hom_{\cat C/c}(u, f) \to
  \Hom_{\cat C/c}(g, f)\). Moreover, given \(f\) and \(u\) as above, we have
  that there exists \(\Id_x, \Id_z \in \Mor(\cat C/c)\) so that from the
  category \(\cat C\) it follows that \(\Id_x  h = h  \Id_z\). In
  addition to the above, define \(v: w \to c\) and the morphism \(t \in
  \Hom_{\cat C/c}(v, g)\) so that \(t: w \to y\). Then we have \(g = t  v\)
  and since \(\cat C\) is a category, we find that \((h  \ell)  t = h
   (\ell  t)\).
\end{proof}

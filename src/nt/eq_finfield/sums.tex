\section{Gau{\ss} and Jacobi Sums over Finite Fields}

\begin{definition}[Trace]
   Let \(k\) be a finite field with \(p^n\) elements. If  \(\alpha \in k\) we
   define the trace of \(\alpha\) to be equal to
   \[
      \tr(\alpha) = \sum_{i=0}^{n-1} \alpha^{p^{i}}
   \] 
\end{definition}

\begin{proposition}
   Let \(k\) be a field with  \(p^n\) elements and let \(\alpha, \beta, \in k\)
   and \(\ell \in \F_p\), then
   \begin{enumerate}[I.]
      \item \(\tr(\alpha) \in \F_p\).
      \item \(\tr(\alpha + \beta) = \tr(\alpha) + \tr(\beta)\).
      \item \(\tr(\ell\alpha) = \ell \tr(\alpha)\).
      \item The trace is a surjective function, that is \( \tr : k
         \twoheadrightarrow \F_p \) 
   \end{enumerate}
\end{proposition}

\begin{proof}
   We prove the last statement. Let the polynomial \(\sum_{i=0}^{n-1} X^{p^i}\),
   then we know that in \(k\) it has at most \(p^{n-1}\) roots, but also  \(|k|
   = p^n\) and thus we can ensure the existence of an element \(\alpha \in
   k\) such that \(\tr(\alpha) \neq 0\), that is, is not root of the said
   polynomial. From this we conclude that, given any element \(\ell \in \F_p\)
   then, since \(\tr(\alpha) \in \F_p^\ast\) we have that \(\ell/\tr(\alpha) \in
   \F_p\) and also \(\tr\left( \frac{\alpha\ell}{\tr(\alpha)} \right) =
   \frac{\ell}{\tr(\alpha)}\tr(\alpha) = \ell \in \F_p \) and thus we conclude
   the surjectivity of the function.
\end{proof}

\begin{definition}
   We define the function \(\psi\) for a general finite field \(k\) as
    \[
       \psi:k \to \mathbb{C}\ \text{ such that }\ \psi(\alpha) :=
       \zeta_p^{\tr(\alpha)},\  \forall \alpha \in k.
   \] 
\end{definition}

\begin{proposition}
   The \(\psi\) function has the properties:
   \begin{enumerate}[I.]
      \item \(\psi(\alpha + \beta) = \psi(\alpha)\psi(\beta)\).
      \item \(\exists \alpha \in k\) such that \(\varphi(\alpha) \neq 1\).
      \item \(\sum_{\alpha \in k} \psi(\alpha) = 0\).
   \end{enumerate}
\end{proposition}

\begin{proof}
   We prove the last two items. For II we have ensured the existence of an
   element \(\alpha \in k\) such that \(\tr(\alpha) \neq  0\) and therefore
   \(\psi(\alpha) \neq  1\). For III define \(S := \sum_{\alpha \in
   k}\psi(\alpha)\) and let any \(\beta \in k\) such that \(\varphi(\beta) \neq
   1\), then surely \(\psi(\beta) S = \sum_{\alpha \in k} \psi(\alpha + \beta) =
   S \Rightarrow S = 0\).
\end{proof}

\begin{proposition}
   Let \(\alpha,\beta,\gamma \in k\) and \(|k| = q\), then
   \[
      \frac{1}{q} \sum_{\alpha \in k} \psi(\alpha(\beta - \gamma)) =
      \delta(\beta, \gamma)
   \] 
   where \(\delta\) is the Kronecker function.
\end{proposition}

\begin{proof}
   Trivially, for \(\beta = \gamma\) then we have a sum of \(\psi(0) = 1\) which
   yield  \(\sum_{\alpha \in k} \psi(0) = q\) and therefore the result follows.
   For the case \(\beta\neq\gamma\) then we have \(\alpha(\beta - \gamma)\) 
   ranging in all of \(k\), thus we can see the sum as simply  \(\sum_{x \in k}
   \psi(x) = 0\) where \(x := \alpha((\beta - \gamma)\) and the claim follows.
\end{proof}

\begin{definition}
   Let \(\chi\) be a character of \(k\) and let \(\alpha \in k^\ast\). We define
   the Gau{\ss} sum on \(k\) belonging to \(\chi\) to be 
   \[
      g_\alpha(\chi) = \sum_{t \in k} \chi(t) \psi(\alpha t).
   \] 
\end{definition}

\begin{theorem}
   Let \(k\) be a finite field with \(|k| = q \) and let \(q \equiv 1
   \pmod{m}\). The homogeneous polynomial \(f(x) := \sum_{i=0}^{n} a_i x_i^{m}
   \in k^\ast[x]\) defines projective hypersurface on \(\PP^{n}(k)\). We have
   the following properties:
   \begin{enumerate}[I.]
      \item The hypersurface defined by \(f(x) = 0\) on the projective space is
         such that the number of point is
         \[
            \sum_{i=0}^{n-1} q^i + \frac{1}{q-1} \sum_{\chi_0,\dots,\chi_n}
            \left( J_0(\chi_0,\dots,\chi_n)\prod_{j=0}^{n} \chi_j(a_j^{-1})
            \right) 
         \] 
         where we define the characters to be such that \(\chi_j^m =
         \varepsilon,\ \chi_j \neq  \varepsilon\) and \(\prod_{j=1}^n \chi_j =
         \varepsilon\).
      \item Under the same conditions specified above, we have
         \[
            \frac{1}{q-1} J_0(\chi_0,\dots,\chi_n) = \frac{1}{q} \prod_{j=0}^{n}
            g(\chi_j).
         \] 
   \end{enumerate}
\end{theorem}

\begin{proof}
   Notice that the polynomial \(\sum_{i=0}^{n} a_ix_i^{m} = 0\) forms a
   hyperplane in \(\A^{n+1}(k)\) with 
   \[
      N = q^n + \sum_{\chi_1,\dots, \chi_n} \left( J_0(\chi_0,\dots,\chi_n)
      \prod_{i=1}^n \chi_i(a_i^{-1}) \right) 
   \] 
   from theorem \ref{thm: fermat generalization}. Now, in order to find the
   number of points for the projective plane \(\PP^n(k)\) we have
   \[
      \frac{N-1}{q-1} = \frac{q^n - 1}{q-1} + \frac{1}{q-1}\sum_{\chi_1,\dots,
      \chi_n} \left( J_0(\chi_0,\dots,\chi_n) \prod_{i=1}^n \chi_i(a_i^{-1})
      \right) 
   \] 
   which matches exactly with the statement when we expand the first term.
   Also, we know that \(J_0(\chi_0,\dots,\chi_n) =
   \chi_0(-1)(q-1)J(\chi_1,\dots,\chi_n)\) and also 
    \[
       J(\chi_1,\dots,\chi_n) = \frac{\prod_{i=1}^{n}
       g(\chi_i)}{g(\chi_1\cdots\chi_n)}
   \] 
   thus, multiplying and dividing the equation by \(g(\chi_0)\) we find
   \(g(\chi_0)g(\chi_1\cdots\chi_n) = g(\varepsilon) = \chi_0(-1)q\), then we
   find that
   \[
      J_0(\chi_0,\dots,\chi_n) = \frac{q-1}{q} \prod_{i=0}^{n} g(\chi_i)
   \] 
   which finishes the proof.
\end{proof}

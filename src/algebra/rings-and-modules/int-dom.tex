\begin{remark}[Rings are commutative]
\label{rem:commutative-rings-in-chapter-int-dom}
In the \emph{entirety} of this chapter, if \(R\) is a \emph{ring}, we assume it
is \emph{commutative} unless otherwise stated.
\end{remark}

\section{Noetherian Rings \& Modules}

\begin{proposition}
\label{prop:equiv-conditions-noetherian}
Let \(R\) be a commutative ring, and \(M\) be an \(R\)-module. The following are
equivalent propositions:
\begin{enumerate}[(a)]\setlength\itemsep{0em}
\item \(M\) is a \emph{Noetherian} module.

\item Every \emph{ascending chain} of submodules of \(M\) \emph{stabilizes}. In
  other words, if \((N_j)_{j \in \N}\) is a collection of submodules of \(M\)
  such that \(N_j \subseteq N_{j+1}\), then there exists an index \(j_0 \in \N\)
  such that \(N_j = N_{j+1}\) for all \(j \geq j_0\).

\item Every non-empty collection of submodules of \(M\) has a \emph{maximal}
  element with respect to inclusion.
\end{enumerate}
\end{proposition}

\begin{proof}
\begin{itemize}\setlength\itemsep{0em}
\item (a) \(\implies\) (b). Let \(M\) be Noetherian and define the module
  \(N \coloneq \bigcup_{j \in \N} N_j\), which is a submodule of \(M\). Since
  submodules of Noetherian rings are finitely generated, it follows that \(N\)
  is finitely generated. Let \(N = \langle n_1, \dots, n_k \rangle\) be its
  generating set. For all \(1 \leq i \leq k\), there must exist \(j_i \in \N\)
  such that \(n_i \in N_j\) for all \(j \geq j_i\). Taking the maximum
  \(j_0 \coloneq \max(j_1, \dots, j_k)\), one finds that \(n_i \in N_j\) for
  each \(1 \leq i \leq k\) and every \(j \geq j_0\). Therefore
  \(N \subseteq N_j\) for all \(j \geq j_0\), which implies that \(N_j = N\) for
  each of those indexes --- therefore the chain stabilizes.

\item (b) \(\implies\) (c). We prove the contrapositive. Suppose there exists a
  non-empty collection \(\mathcal{N}\) of submodules of \(M\) admitting no
  maximal element. Let \(N_0 \in \mathcal{N}\) be any element. Inductively, for
  all \(j \geq 1\), define \(N_j \in \mathcal{N}\) such that \(N_{j-1}
  \subsetneq N_j\), that is, \(N_{j-1}\) is a \emph{proper} subset of \(N_j\)
  --- this is possible since \(N_{j-1}\) isn't maximal. The collection
  \((N_j)_{j \in \N}\) forms an ascending chain of submodules, but by
  construction does not stabilize.

\item (c) \(\implies\) (a). Let \(N \subseteq M\) be any submodule. Since
  \((0) \subseteq N\) is a finitely generated submodule of \(N\), one can define
  a non-empty collection \(\mathcal{N}\) of finitely generated submodules of
  \(N\). From (c) one has that \(\mathcal{N}\) admits a maximum element, say
  \(W \coloneq \langle n_1, \dots, n_k \rangle\). Let \(n \in N\) be any element
  and consider the finitely generated submodule \(\langle n_1, \dots, n_k, n
  \rangle \in \mathcal{N}\). Since \(W\) is maximal, we have \(\langle n_1,
  \dots, n_k, n \rangle \subseteq W\) --- therefore \(n \in W\) and \(N
  \subseteq W\). Therefore \(N = W\) is finitely generated, which proves that
  \(M\) is Noetherian.
\end{itemize}
\end{proof}

\begin{lemma}[Quotient of Noetherian rings is Noetherian]
\label{lem:quotient-is-noetherian}
Let \(R\) be a Noetherian ring, and \(\ideal{a} \subseteq R\) be an ideal. Then
the quotient ring \(R/\ideal{a}\) is Noetherian.
\end{lemma}

\begin{proof}
From \cref{prop:image-of-noetherian-is-noetherian} we find that the canonical
projection \(R \epi R/\ideal{a}\) implies that \(R/\ideal{a}\) is Noetherian.
\end{proof}

\begin{theorem}[Generalized Hilbert's basis theorem]
\label{thm:general-hilbert-basis}
Let \(R\) be a ring. Then \(R\) is Noetherian if and only if the polynomial ring
\(R[x_1, \dots, x_n]\).
\end{theorem}

\todo[inline]{Prove generalized Hilbert's basis theorem}

\begin{corollary}
\label{cor:noetherian-quotient-poly-ring}
Let \(R\) be a Noetherian ring, and \(\ideal{a} \subseteq R[x_1, \dots, x_n]\)
be an ideal of the polynomial ring. Then the quotient ring
\(R[x_1, \dots, x_n]/\ideal{a}\) is Noetherian.
\end{corollary}

\begin{proof}
Since \(R[x_1, \dots, x_n]\) is Noetherian by \cref{thm:general-hilbert-basis},
applying \cref{lem:quotient-is-noetherian} we find that \(R[x_1, \dots,
x_n]/\ideal a\) is Noetherian.
\end{proof}

\begin{corollary}
\label{cor:finite-type-alg-noetherian}
Every \emph{finite-type} algebra over a Noetherian ring is \emph{Noetherian}.
\end{corollary}

\subsection{Existence of Maximal Ideals}

\begin{proposition}
\label{prop:commutative-rings-have-maximal-ideals}
Let \(R\) be a commutative ring. If \(\ideal a\) is any \emph{proper} ideal of
\(R\), then there exists a \emph{maximal ideal} \(\ideal m\) of \(R\) containing
\(\ideal a\).
\end{proposition}

\begin{proof}
Consider the collection \(I\) of proper ideals of \(R\) containing
\(\ideal a\)---which is ordered by inclusion. Using this ordering, let
\((\ideal a_j)_{j \in J}\) be the chain of all proper ideals with
\(\ideal a_j \in I\) and \(\ideal a_j \subseteq \ideal a_{j+1}\). Define the set
\(\ideal m \coloneq \bigcup_{j \in J} \ideal a_j\), which is again an ideal of
\(R\). Since every \(\ideal a_j\) contains \(\ideal a\) and does \emph{not}
contain \(1\), it follows that \(\ideal m\) contains \(\ideal a\) and also
doesn't contain \(1\)---hence \(\ideal m\) is a proper ideal. Therefore
\(\ideal m\) is a \emph{maximal} ideal of \(R\) containing \(\ideal a\), which
proves the statement.
\end{proof}

\section{Localization}

\begin{definition}[Multiplicative subset]
\label{def:multiplicative-subset}
Given a commutative ring \(R\), a subset \(S \subseteq R\) is said to be
\emph{multiplicative} if \(1_R \in S\) and \(S\) is closed under
multiplication---that is, given \(s, s' \in S\), we have \(s s' \in S\).
\end{definition}

\begin{definition}[Localization]
\label{def:localization-ring}
Let \(R\) be a commutative ring, and \(S \subseteq R\) be a \emph{multiplicative
  subset}. A \emph{localization} of \(R\) over the set \(S\) is a morphism of
\emph{commutative rings} \(L_S: R \to R[S^{-1}]\) such that:
\begin{enumerate}[(a)]\setlength\itemsep{0em}
\item For all \(s \in S\), the image \(L_S(s) \in R[S^{-1}]\) is a \emph{unit}.
\item For every morphism of rings \(f: R \to K\) satisfying property (a), there
  exists a \emph{unique} ring morphism \(\phi: R[S^{-1}] \to K\) such that the
  diagram
  \[
  \begin{tikzcd}
  R \ar[r, "L_S"] \ar[rd, bend right, "f"']
  &{R[S^{-1}]} \ar[d, "\phi", dashed] \\
  &K
  \end{tikzcd}
  \]
\end{enumerate}
Equivalently, consider the minimal equivalence relation \(\sim_{\text{frac}}\)
on the set \(R \times S\) defined by \((r, s) \sim_{\text{frac}} (r', s')\) if
and only if \(r s' - s r' = 0\). Then the localization of \(R\) under the
multiplicative subset \(S\) is simply
\[
R[S^{-1}] = (R \times S)/{\sim_{\text{frac}}}.
\]
\end{definition}

\begin{example}
\label{exp:Z[1/p]/Z-finite-submodules}
Let \(p \in \Z\) be a prime number, and consider the ring of fractions
\(\Z[p^{-1}]\). Then every proper submodule of the \(\Z\)-module given by the
quotient \(\Z[p^{-1}]/\Z\) is \emph{finite}.
\todo[inline]{How to solve this?}
\end{example}

\begin{definition}
\label{def:field-of-fractions}
Let \(R\) be an integral domain. We define the \emph{field of fractions} of
\(R\) to be the localization
\[
\Frac(R) \coloneq R[(R \setminus \{0\})^{-1}].
\]
\end{definition}

It is immediate that \(\Frac(R)\) is a field, since a field is simply an
integral domain whose non-zero elements are units---and this exactly what we did
with the above localization.

\section{Principal Ideal Domains}

\subsection{Chinese Remainder Theorem}

\begin{theorem}[Chinese remainder]
\label{thm:chinese-remainder-theorem}
Let \(R\) be a commutative ring, and \(\ideal{a}_1, \dots, \ideal{a}_k\)
be ideals of \(R\) such that \(\ideal{a}_i + \ideal{a}_j = R\) for all
\(i \neq j\). Then:
\begin{itemize}\setlength\itemsep{0em}
\item We have the equality \(\ideal{a}_1 \cap \dots \cap \ideal{a}_k =
  \ideal{a}_1 \cdot \ldots \cdot \ideal{a}_k\).
\item The natural projection
  \(R \epi R/\ideal{a}_1 \times \dots \times R/\ideal{a}_k\) is
  \emph{surjective}, and induces a natural \emph{isomorphism} of rings
  \[
  \frac{R}{\ideal{a}_1 \dots \ideal{a}_k}
  \iso (R/\ideal{a}_1) \times \dots \times (R/\ideal{a}_{k})
  \]
\end{itemize}
\end{theorem}

\begin{proof}
In particular, we have \(\ideal{a}_j + \ideal{a}_k = R\) for all \(1 \leq j \leq
k - 1\). Therefore, for all such indices there exists \(a_j \in \ideal{a}_k\)
such that \(1 - a_j \in \ideal a_j\), hence
\[
(1 - a_1) \dots (1 - a_{k-1}) \in
\ideal{a}_1 \cdot \ldots \cdot \ideal a_{k-1},
\]
and since \(a_j \in \ideal{a}_k\), then
\(1 - \prod_{j=1}^{k-1} (1 - a_j) \in \ideal{a}_k\). This shows that
\begin{equation}\label{eq:coprime-ideals-CRT}
(\ideal{a}_1 \cdot \ldots \cdot \ideal{a}_{k-1}) + \ideal{a}_k = R.
\end{equation}

Notice that
\(\ideal{a}_1 \cdot \ldots \cdot \ideal a_k \subseteq \ideal a_1 \cap \dots \cap
\ideal a_k\), thus it remains to prove the other side of the inclusion. From our
last paragraph, we know that
\(\ideal a_1 \cap \dots \cap \ideal a_k \subseteq \ideal{a}_1 \cdot \ldots \cdot
\ideal a_k \) for all \(k \geq 3\). Since \(k = 1\) is trivial, we just need to
prove the case for \(k = 2\). Let \(\ideal{b}, \ideal{c} \subseteq R\) be ideals
such that \(\ideal b + \ideal c = R\)---then there exists \(b_0 \in \ideal b\)
and \(c_0 \in \ideal c\) such that \(b_0 + c_0 = 1\). If \(x \in \ideal{b} \cap
\ideal c\), then \(x = b_0 x + c_0 x\), implying in \(x \in \ideal b \cdot
\ideal c\). Thus \(\ideal b \cap \ideal c \subseteq \ideal b \cdot \ideal c\).

We now prove the second assertion via induction on \(k\). For the base case
\(k = 1\), the statement follows trivially from the first isomorphism
theorem. Assume as the hypothesis of induction that the statement is true
\(k - 1 > 1\)---that is, we have an isomorphism
\(R/(\ideal a_1 \cdot \ldots \cdot \ideal a_{k-1}) \iso (R/\ideal a_1) \times
\dots \times (R/\ideal a_{k-1})\). Consider the natural map
\begin{equation}\label{eq:natural-projection-CRT}
\pi: R \longrightarrow \frac{R}{\ideal a_{1} \cdot \ldots \cdot \ideal a_{k-1}}
\times (R/\ideal a_k).
\end{equation}
From \cref{eq:coprime-ideals-CRT} the statement is reduced for the case of two
ideals \(\ideal b \coloneq \ideal a_1 \cdot \ldots \cdot a_{k-1}\) and
\(\ideal c \coloneq \ideal a_k\) such that \(\ideal b + \ideal c = R\). Let
\(b, c \in R\) be any two elements---we shall show that there exists \(r \in R\)
such that \(r - b \in \ideal b\) and \(r - c \in \ideal c\). Since \(\ideal b\)
and \(\ideal c\) are relatively prime, let as before \(b_0 + c_0 = 1\) and
define \(r \coloneq b_0 c + c_0 b\). Then, one has
\begin{align*}
  r &= b_0 c + (1 - b_0) b = b + b_0 (c - b) \equiv b \pmod{\ideal b} \\
  r &= (1 - c_0) c + c_0 b = c + c_0 (b - c) \equiv c \pmod{\ideal c}
\end{align*}
since \(b_0 \in \ideal b\) and \(c_0 \in \ideal c\). Therefore
\(r - c \in \ideal b\) and \(r - b \in \ideal c\) a wanted. This shows that the
natural morphism of rings \(\pi\) (see \cref{eq:natural-projection-CRT}) is
surjective. From our first considerations, we know that
\(\ker \pi = \ideal b \cap \ideal c = \ideal b \cdot \ideal c\), therefore the
first isomorphism theorem establishes that
\(R/(\ideal b \cdot \ideal c) \iso (R/\ideal b) \times (R/\ideal c)\).
\end{proof}

\begin{corollary}[Chinese remainder for PIDs]
\label{cor:chinese-remainder-for-PID}
Let \(R\) be a principal ideal domain, and \(a_1, \dots, a_k \in R\) be elements
such that \(\gcd(a_i, a_j) = 1\) for all pairs \(i \neq j\). Then the natural
map \(r + (a_1 \cdots a_k) \mapsto (r + (a_1), \dots, r + (a_k))\) establishes an
\emph{isomorphism} of rings
\[
R/(a_1 \cdots a_k) \iso \frac{R}{(a_{1})} \times \dots \times \frac{R}{(a_k)}.
\]
\end{corollary}

%%% Local Variables:
%%% mode: latex
%%% TeX-master: "../../deep-dive"
%%% End:

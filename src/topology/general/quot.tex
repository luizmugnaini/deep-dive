\section{Quotient Space}

The motivation for the construction of the quotient topology is the study of
surjective set-functions \(\pi: X \epi S\) between topological spaces \(X\) and
sets \(S\), which induce an equivalence relation on the initial topological
space by means of arranging the points of \(X\) in classes where \(x \sim y\)
for \(x, y \in X\) if and only if \(\pi(x) = \pi(y)\), that is, \(x\) and \(y\)
are common points of a fiber \(\pi^{-1}(s)\) for some \(s \in S\).

The geometrically appealing version of such construction would be the idea of
gluing every element \(x \in \pi^{-1}(s)\) into a unique point --- this way we
loose some initial information about the topological space \(X\). However, we
would like to make this gluing process compatible with the theory of topology so
far constructed. To achieve that, one may make the point that we should make
\(S\) into a topological space for which its topology makes \(\pi\) a continuous
map. The first idea that may come to mind is that we can simply force the
continuity of \(\pi\) by defining a topology \(\mathcal T\) on \(S\) so that \(U
\subseteq S\) is an open set if and only if \(\pi^{-1}(U)\) is open --- which is
exactly the definition of a continuous map. Lets give this a formal definition.

\begin{definition}\label{def:quotient-topology}
Let \(X\) be a topological space and \(S\) be any set. Let also \(\pi: X \epi S\)
be a set-function. The quotient topology on \(S\) induced by the map \(\pi\) is
the finest topology such that \(\pi\) is a continuous map.
\end{definition}

\begin{proposition}
The quotient topology is a topology.
\end{proposition}

\begin{proof}
Let \(X /{\sim}\) be a topological space with the quotient topology induced by
the map \(\pi: X \epi X /{\sim}\). Consider an arbitrary collection
\({\{U_{j}\}}_j\) of open sets of \(X /{\sim}\). Notice that since
\(\pi^{-1}(\bigcup_{j} U_{j}) = \bigcup_j \pi^{-1}(U_j)\) then since
\(\pi^{-1}(U_j)\) is open for all index \(j\), we conclude that \(\bigcup_j
\pi^{-1}(U_{j}) \subseteq X\) is open, hence the set \(\bigcup_j U_j \subseteq X
/{\sim}\) is necessarily open --- since \(\pi\) is continuous. Let now \(A, B
\subseteq X /{\sim}\) be any open sets, then \(\pi^{-1}(A \cap B) = \pi^{-1}(A)
\cap \pi^{-1}(B) \subseteq X\), which is open and therefore \(A \cap B\) is open
in \(X /{\sim}\). We conclude that the quotient topology indeed satisfies the
properties of a topology.
\end{proof}

In other terms, let \(T_S\) be the collection of all topologies on \(S\) such
that \(\pi\) is a continuous map. The above definition simply says that the
quotient topology \(\mathcal T\) on the set \(S\) is the intersection \(\mathcal
T = \bigcap_{T \in T_S}T\). Even better than that definition is the fact that we
can determine the quotient topology by the following universal property.

\begin{theorem}[Universal property of the quotient topology]
\label{thm:universal-property-quotient-topology}
Let \(X\) be a topological space and \(S\) be any set, together with a
surjective set-function \(\pi: X \epi S\). The quotient topology \(\mathcal T\)
on \(S\) induced by \(\pi\) is such that, for all topological spaces \(Z\) and
every morphism \(g: X \to Z\) for which all \(x, y \in X\) such that \(\pi(x) =
\pi(y)\) then \(g(x) = g(y)\) --- i.e. \(g\) is constant on the fibers of
\(\pi\) ---, there exists a unique morphism \(f: (S, \mathcal T) \to Z\) such
that the diagram
\[
  \begin{tikzcd}
    X \ar[r, "g"] \ar[d, two heads, swap, "\pi"] &Z\\
    S \ar[ur, swap, bend right, dashed, "f"]
  \end{tikzcd}
\]
commutes in \(\Top\). Moreover, if \(\mathcal T'\) is a topology on \(S\) such
that the diagram commutes, then necessarily \(\mathcal T' = \mathcal T\).
\end{theorem}

\begin{proof}
Since \(\pi\) is surjective, we can completely define a map \(f: S \to Z\)
sending \(s \mapsto g(x)\) such that \(x \in \pi^{-1}(s)\), which is well
defined because, for all \(x, y \in \pi^{-1}(s)\), we have \(g(x) = g(y)\), so
that the image of each \(s \in S\) under the map \(f\) is uniquely defined. We
now show that \(f\) is, in fact, continuous --- and hence a morphism. Let \(U
\subseteq Z\) be any open set of \(Z\), then, since \(g^{-1}(U)\) is open and
\((f \pi)^{-1}(U) = \pi^{-1} f^{-1}(U) = g^{-1}(U)\), \(f^{-1}(U)\) cannot be a
closed set --- in fact, it needs to be an open set, because
\(\pi^{-1}(f^{-1}(U)) \subseteq X\) must be open, since \(\pi\) is continuous.
It follows that \(f\) is continuous and thus the said morphism indeed exists.

For the uniqueness, let \(f\) and \(f'\) be two morphisms such that \(f\pi = g\)
and \(f'\pi = g\). Let \(s \in S\) be any point. Since \(\pi\) is surjective,
there exists \(x \in X\) for which \(x \in \pi^{-1}(s)\), therefore, \(f \pi(x)
= f(s) = f'\pi(x) = f'(s)\) for every element of their domain --- hence \(f =
f'\).

Suppose now that both \((S, \mathcal T)\) and \((S, \mathcal T')\) satisfy the
universal property, that is, the following diagrams commutes for unique
morphisms \(f\) and \(h\)
\[
  \begin{tikzcd}
    &Z & \\
    (S, \mathcal T) \ar[ur, dashed, bend left, "f"]
    &X \ar[l, two heads, "\pi"] \ar[r, swap, two heads, "\pi"] \ar[u, "g"]
    &(S, \mathcal T') \ar[ul, dashed, bend right, swap, "f'"]
  \end{tikzcd}
\]
Since the diagram commutes for all \(Z\), let \(Z = (S, \mathcal T)\), and
consider the map \(f' = \Id': (S, \mathcal T') \to (S, \mathcal T)\). Then,
given any \(U \in \mathcal T\) we find that since \(\Id'\) is continuous that
\(\Id'^{-1}(U) = U \subseteq (S, \mathcal T')\) is open, hence \(\mathcal T
\subseteq \mathcal T'\). Analogously, let \(Z = (S, \mathcal T')\) and consider
\(f = \Id: (S, \mathcal T) \to (S, \mathcal T')\). Let \(U' \in \mathcal T'\)
then from the continuity of \(\Id\) we find that \(\Id^{-1}(U') = U' \subseteq
(S, \mathcal T)\) is open, therefore \(\mathcal T' \subseteq \mathcal T\). Thus
indeed \(\mathcal T = \mathcal T'\) as wanted.
\end{proof}

Many important spaces can be obtained with the inclusion of the quotient
topology to our toolkit, I'll now briefly discuss some of those, which will most
probably come up further into this text.

\begin{example}[Projective space]\label{exp:real-projective-space}
Let \(\sim\) be the equivalence relation for which \(x \sim y\) if and only if \(x = \gamma
y\), where \(x, y \in \R^{n+1} \ \{0\}\) and \(\gamma \in \R\). We define the
\(n\)-dimensional real projective space as the quotient \((\R^{n+1} \setminus
\{0\})/{\sim}\), which is denoted by \(\Proj^{n}(\R)\).
\end{example}

\begin{example}[Cone]\label{exp:cone}
Let \(X\) be any topological space and \(I\) be the standard interval. The
topological space \(X \times I\) is known as the cylinder on \(X\). Via a quotienting
operation, we can collapse regions of this cylinder. For instance, we can create
a cone by collapsing one of the sides of the cylinder, such as \((X \times I) / (X \times
\{0\})\). Such object is denoted \(\Cone(X)\), the cone on \(X\).
\end{example}

\begin{example}
\label{exp:wedge-sum-space}
Let \(J\) be an indexing set and \(\{X_j\}_{j \in J}\) be a collection of
non-empty topological spaces. For each \(j \in J\), choose any \(p_j \in X_j\) as a
base point. We define the wedge sum of the collection \(\{X_{j}\}_{j \in J}\) with
respect to the base points \(\{p_{j}\}_{j \in J}\) as the topological space
\[
  \bigvee_{j \in J} X_j = \coprod_{j \in J} X_j / \{p_{j}\}_{j \in J}
\]
\end{example}

\begin{proposition}
\label{prop:hausdorff-wedge-sum}
Let \(\{X_{j}\}_{j \in J}\) be an indexed collection of Hausdorff topological
spaces. Then the wedge sum \(\bigvee_{j \in J} X_j\) with respect to any choice of base
points is Hausdorff.
\end{proposition}

\begin{proof}
Let \(\{p_j \in X_j\}_{j \in J}\) be any choice of base points for the given
collection. Let \(x, y \in \bigvee_{j \in J} X_j\) be any distinct points in the wedge sum
space. If \(x, y \in X_j\) for some \(j \in J\), then it is clear that there exists
non-intersecting neighbourhoods of \(x\) and \(y\) on \(X_j\) --- of which we can
take their intersection with the disjoint union \(\coprod_{j \in J} X_j\) and the
proposition will hold for \(\bigvee_{j \in J} X_j\). On the other hand, if \(i, j \in J\)
are distinct indices and \(x \in X_i\) while \(y \in X_j\), then there exists \(U_x
\subseteq X_i\) and \(U_y \subseteq X_j\) neighbours of \(x\) and \(y\), respectively, such that
\(U_x \cap X_j = \emptyset\) and \(U_y \cap X_i = \emptyset\), which in particular imply in \(U_x \cap
U_y= \emptyset\). For our end, we just need to consider the neighbourhoods \(U_x' = U_x
\cap \coprod_{j \in J} X_j\) and \(U_y' = U_y \coprod_{j \in J} X_j\) so that \(U_x' = U_y'\).
\end{proof}

\section{Testing}
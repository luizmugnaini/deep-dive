\section{Implict Map Theorem}

\begin{definition}[Level curve]
\label{def:level-curve}
Let \(f: \Omega \to \R\), where \(\Omega \subseteq \R^n\), be any map. For every \(c \in \R\) we
define the set \(N_f(c) = \{x \in \Omega \colon f(x) = c\}\) to be the \(c\)-level
curve of \(f\). Moreover, if \(n = 2\), we can call \(N_f(c)\) the \(c\)-contour
line of \(f\) with value \(c\).
\end{definition}

\begin{theorem}[Implicit Theorem]
\label{thm:implicit-map}
Let \(V\) and \(L\) be normed vector spaces and \(W\) be a Banach
space. Define \(\Omega \subseteq V \times W\) to be an open set and \((x_0, y_0) \in \Omega\). Consider
\(F: \Omega \to L\) to be a mapping such that
\begin{itemize}\setlength\itemsep{0em}
  \item \(F(x_0, y_0) = c\) for some \(c \in L\).
  \item \(F\) is continuous at \((x_0, y_0)\).
  \item \(F\) is differentiable and its differential \(\diff F: \Omega \to L\)
    is continuous at \((x_0, y_0)\).
  \item \(\partial_2F(x_0, y_0): W \to L\) is an isomorphism, that is, it is invertible.
\end{itemize}
Then there exists neighbourhoods \(U_{x_0} \subseteq V\) and \(U_{y_0} \subseteq W\) such that
\(U_{x_0} \times U_{y_0} \subseteq \Omega\), and a map \(f: U_{x_0} \to U_{y_0}\) such that
\begin{itemize}\setlength\itemsep{0em}
  \item \(f(x_0) = y_0\).
  \item \(f\) is continuous at \(x_0\).
  \item \(F(x, y) = 0\) if and only if \(f(x) = y\), for \(x, y \in U_{x_0} \times
  U_{y_0}\).
\end{itemize}
\end{theorem}

\begin{proof}
To ease our lives, lets assume that \(\Omega\) has the following form
\[
  \Omega = \{(x, y) \in V \times W \colon \norm{x - x_{0}}_{V} < \alpha
  \text{ and } \norm{y - y_0}_W < \beta\}.
\]
If that is not the case, since \(\Omega\) is open --- and hence \((x_{0}, y_0)\) is
an internal point --- we can merely choose an open set contained in \(\Omega\) that
satisfyies the above form.

Define a collection \(\{g_{x} \colon x \in B_{x_0}(\alpha)\}\) of maps
\[
  g_{x}(y) = y - [\partial_2F(x_0, y_0)]^{-1} (F(x, y))
\]
The domain of each \(g_x\) is defined to be the collection \(B_{y_0}(\beta) = \{y
\in W \colon \norm{y - y_0}_W < \beta\}\). The maps \(g_{x}\) are well defined since
\([\partial_2F(x_0, y_0)]^{-1}\) exists and is a continuous linear map --- moreover, the
domain of \(g_x\) is the normed vector space \(W\) --- that is \(g_x: B_{y_0}(\beta) \to
W\).

Suppose \(y_x\) is a fixed point of \(g_x\), then \(g_x(y_x) = y_x -
[\partial_2F(x_0, y_0)]^{-1}(F(x, y_x)) = y_{x}\) hence clearly \(y_x\) is indeed a
fixed point of \(g_x\) if and only if \([\partial_2F(x_0, y_0)]^{-1}(F(x, y_x)) =
0\), that is, \(F(x, y_x) \in \ker [\partial_2F(x_0, y_0)]^{-1}\) --- but \(\partial_2 F(x_0,
y_0)\) is an isomorphism, so clearly \(F(x, y_x) = 0\).

Let \(x \in B_{x_0}(\alpha)\) be any element and consider the map \(g_x\). Notice
that since \(F\) is differentiable and \(g_{x}\) is therefore a composition of
differentiable maps, it follows that \(g_x\) is differentiable and since \(\partial_2
F(x_0, y_0)\) is continuous and linear we get
\begin{align*}
  \partial_2 \left( [\partial_2F(x_0, y_0)]^{-1}(F(x, y)) \right)
  &= \lim_{t \to 0} \frac{[\partial_{2}F(x_0, y_0)]^{-1}(F(x, y + t)) - [\partial_2F(x_0,
    y_0)]^{-1}(F(x, y))}{t}
  \\
  &= \lim_{t \to 0} \frac{[\partial_2F(x_0, y_{0})]^{-1}(F(x, y + t) - F(x, y))}{t}
  \\
  &= \lim_{t \to 0} [\partial_2F(x_0, y_0)]^{-1}
    \left( \frac{F(x, y + t) - F(x, y)}{t} \right)
  \\
  &= [\partial_2F(x_0, y_0)]^{-1}
    \left( \lim_{t \to 0} \frac{F(x, y + t) - F(x, y)}{t} \right)
  \\
  &= [\partial_{2}F(x_0, y_0)]^{-1}(\partial_2 F(x, y))
\end{align*}
Therefore we can write the differential of \(g_{x}\) as
\[
  \diff g_x(y) = 1_W - \partial_{2}\left([\partial_2F(x_0, y_0)]^{-1}(F(x, y))\right)
  = [\partial_2F(x_0, y_0)]^{-1}(\partial_2F(x_0, y_0) - \partial_2F(x, y)).
\]
By the continuity of the map \(\partial_2F(x_0, y_0)\) at the point \((x_0, y_0)\),
we find that there exists \(0 < \gamma < \min(\alpha, \beta)\) such that in the neighbourhood
\(B_{x_0}(\gamma) \times B_{y_0}(\gamma) \subseteq \Omega\) we have
\[
  \norm{\diff g_x(y)} \leq \norm{[\partial_{2}F(x_0, y_0)]^{-1}} \cdot \norm{\partial_2F(x_0, y_0) -
    \partial_2F(x, y)} < \frac{1}{2}.
\]

Let \(x \in B_{x_0}(\gamma)\) be any element and take any two \(y_1, y_2 \in
B_{y_0}(\gamma)\). Then we have by the generalization of the mean value theorem
that
\[
  \norm{g_x(y_1) - g_x(y_{2})}_{W} \leq \sup_{t \in (y_1, y_2)} \norm{\diff g(t)}
  \norm{y_1 - y_2}_W < \frac{1}{2} \norm{y_1 - y_2}.
\]
That is, \(g_x\) is Lipschitz continuous.

From the definition of \(g_x\) and the fact that \(g_{x_0}(y_0) = y_0\) ---
since \(F(x_0, y_0) = 0\) --- we have
\begin{align*}
  \norm{g_x(y) - y_0}_W
  &= \norm{g_x(y) - g_{x_0}(y_0)}_W \\
  &\leq \norm{g_x(y) - g_{x}(y_0)}_W + \norm{g_x(y_0) - g_{x_0}(y_0)}_W \\
  &\leq \frac{1}{2} \norm{y - y_0}_W + \norm{[\partial_2F(x_0, y_0)]^{-1}(F(x, y_0) -
    F(x_0, y_0))}_W \\
  &= \frac{1}{2} \norm{y - y_0}_W
    + \norm{[\partial_{2}F(x_{0}, y_{0})]^{-1}(F(x, y_0))}_W
\end{align*}
Since \(F\) is continuous at \((x_{0}, y_0)\) --- so is the projection
map \(x \mapsto F(x, y_0)\) --- then for all \(\varepsilon' \in (0, \gamma)\) there
exists \(\delta \in (0, \gamma)\) such that \(\norm{x - x_0}_V < \delta\) implies \(\norm{F(x, y_0) -
  F(x_0, y_0)}_L = \norm{F(x, y_0)}_L < \varepsilon'\). In particular, choose \(\varepsilon' =
\frac{\varepsilon}{2 \norm{[\partial_{2}F(x_0, y_0)]^{-1}}}\) for any given \(\varepsilon > 0\), then for
all \(\norm{x - x_0}_V < \delta\) and for all \(\norm{y - y_0}_W \leq \varepsilon\) we have --- using
the property that the norm of linear maps is sub-multiplicative
\begin{align*}
  \norm{g_x(y) - y_0}_{W}
  &\leq \frac{1}{2} \norm{y - y_0}_{W} + \norm{[\partial_2F(x_0, y_0)]^{-1}(F(x, y_0))}_W
  \\
  &\leq \frac{1}{2} \varepsilon + \norm{[\partial_2F(x_0, y_0)]^{-1}} \cdot \norm{F(x, y_0)}_W
  \\
  &\leq \frac{1}{2}\varepsilon
    + \norm{[\partial_2F(x_0, y_0)]^{-1}} \frac{\varepsilon}{2 \norm{[\partial_2F(x_0, y_0)]^{-1}}}
  \\
  &= \frac{1}{2} \varepsilon + \frac{1}{2}\varepsilon = \varepsilon
\end{align*}
The relation above is equivalent to: for all \(\norm{x - x_0}_V < \delta\) we have
\[
  g_x(\overline{B}_{y_0}(\varepsilon)) \subseteq B_{y_0}(\varepsilon)
\]

Since \(B_{y_0}(\varepsilon) \subseteq W\) is a closed set, it follows from
\cref{thm: Banach fixed point} that there exists a unique fixed point \(y_x
\in \overline{B}_{y_0}(\varepsilon)\) of \(g_{x}\). Define now the map \(f: B_{x_0}(\delta) \to
B_{y_0}(\varepsilon)\) where \(f(x) = y_{x}\) and \(y_x\) is the corresponding fixed
point of each \(g_x\).

From construction \(B_{x_0}(\delta) \times B_{y_0}(\varepsilon) \subseteq \Omega\). Clearly, for all \(x \in
B_{x_0}(\delta)\) and \(y \in B_{y_0}(\varepsilon)\), \(F(x, y) = 0\) if and only if \(f(x) =
y\). Moreover, we have immediatly that \(f(x_0) = y_0\). For the continuity of
\(f\) at the point \(x_0\), we can observe that for all \(\varepsilon \in (0, \gamma)\) there
exists a \(\delta \in (0, \gamma)\) such that \(\norm{g_x(y_x) - y_0} = \norm{f(x) - y_0}
< \varepsilon\), thus we are finally done.
\end{proof}

We'll now extend the Implict Map Theorem for cases 3 where we have more
special conditions, allowing for additional properties for the implicit map
\(f\).

\begin{lemma}[Continuity of the implicit map]
\label{lem:continuity-implicit-map}
Let \(F\) satisfy the properties described in \cref{thm:implicit-map} and
additionally suppose that there exists a neighbourhood of \((x_0, y_0)\) where
the map \(\partial_2F(x_0, y_0): W \to L\) is continuous. Then the implicit map \(f: U
\to V\) is such that there exists a neighbourhood of \(x_0\) such that \(f\) is
continuous.
\end{lemma}

\begin{proof}

\end{proof}


\begin{lemma}[Differentiability of the implicit map]
\label{lem:differential-implicit-map}
Let \(F\) satisfy the properties described in \cref{thm:implicit-map} and
additionally suppose that the partial derivative \(\partial_1F(x, y): V \to L\)
exists in some neighbourhood of the point \((x_0, y_0)\) and is continuous at
\((x_0, y_0)\). Then the implicit map \(f\) is differentiable at \(x_0\) and
\[
  \diff f(x_0) = - [\partial_2F(x_0, y_0)]^{-1} \partial_1 F(x_{0}, y_0).
\]
\end{lemma}

\begin{proof}

\end{proof}

\begin{lemma}[Continuous differentiability of the implicit map]
\label{lem:continuous-diff-implicit-map}
Let \(F\) satisfy the properties described in \cref{thm:implicit-map} and
additionally suppose that the partial derivatives of \(F\) are continuous in
some neighbourhood of \((x_0, y_0)\). Then the map \(f\) is continuously
differentiable in some neighbourhood of \(x_0\) and its differential in this
neighbourhood is given by
\[
  \diff f(x) = - [\partial_1F(x, f(x))]^{-1} \partial_1F(x, f(x)).
\]
\end{lemma}

\begin{proof}

\end{proof}

\begin{lemma}[\(C^k\) implicit map]
\label{lem:Ck-implicit-map}
Let \(F\) satisfy the properties described in \cref{thm:implicit-map} and
additionally suppose that \(F \in C^k(\Omega, L)\). Then the implicit map \(f\) is
a member of the class \(C^k(U, W)\) in some neighbourhood \(U \subseteq V\) of \(x_0\).
\end{lemma}

\begin{proof}

\end{proof}

\todo[inline]{Write down the proof of the extensions of the basic Implicit Map
Theorem}

\subsection{Corollaries of the Implicit Map Theorem}

\subsection{Inverse Map Theorem}

\begin{definition}[Diffeomorphisms]
\label{def:diffeormorphism-on-R}
Let \(U\) and \(V\) be open subsets of \(\R^m\). A map \(f: U \to V\) is said to
be a isomorphism of manifolds of class \(C^p\) (or diffeomorphism) --- for \(p \in
\N \cup \{\infty\}\) --- if the following conditions are satisfied
\begin{itemize}\setlength\itemsep{0em}
  \item \(f \in C^p(U, V)\).
  \item \(f\) is bijective and \(f^{-1} \in C^p(V, U)\).
\end{itemize}
\end{definition}

\begin{theorem}[Inverse Map Theorem]
\label{thm:inverse-map}
Let \(V\) and \(W\) be Banach spaces and \(\Omega \subseteq V\) be an open
set. Consider a point \(x_0 \in \Omega\) and a map \(f: \Omega \to W\) such that the
following conditions are satisfied
\begin{itemize}\setlength\itemsep{0em}
  \item \(f \in C^1(\Omega, W)\).
  \item \(\diff f(x_0)\) is invertible and \([\diff f(x_{0})]^{-1}\) is a
  continuous map.
\end{itemize}
Then there exists an open neighbourhood \(X \subseteq \Omega\) of \(x_0\) and an open
neighbourhood \(Y \subseteq W\) of the point \(y_0 = f(x_0)\), for which the
restriction \(f: X \to Y\) is bijective. The inverse map \(f^{-1}: Y \to X\)
is differentiable and its differential is given by
\[
  \diff f^{-1}(y_0) = [\diff f(x_{0})]^{-1}.
\]
\end{theorem}

\begin{proof}
Let \(F: N \to W\) be a map defined on \(N \subseteq V \times W\) where \(N\) is a
neighbourhood of \((x_0, y_0)\) and let \(F(x, y) = f(x) - y\). Since \(F\) is
the composition of the restriction of maps that are continuously
differentiable, it follows that \(F\) is continuously differentiable, moreover,
\(\partial_1F(x_0, y_0) = \diff f(x_0)\).  Moreover, since \(\diff f(x_0)\) is
invertible then \(\partial_1F(x_0, y_0)\) is also invertible. We have \(F(x_0, y_0) =
0\) by construction, since \(f(x_{0}) = y_0\).

We can now see that \(F\) satisfies the requirements for the Implicit Map
Theorem, thus there exists a neighbourhood \(Y\) of \(y_0\) and a
continuously differentiable map \(g: Y \to V\) (where we use extension
\cref{lem:Ck-implicit-map}) for which \(g(Y)\) is contained in a
neighbourhood \(X' \subseteq V\) of \(x_{0}\). Moreover, \(F(x, y) = 0\) if and only
if \(g(y) = x\), that is, \(F(g(y), y) = 0\) and therefore \(f g(y) = y\) for
any \(y \in Y\), that is, \(g\) is injective on \(Y\) --- also \(g(y_0) =
x_0\). The map \(g\) has a differential given by (using the extension
\cref{lem:differential-implicit-map})
\[
  \diff g(y) = [\partial_{1} F(x, y)]^{-1} [\partial_2 F(x, y)] \text{, for all } (x, y) \in
  X' \times Y.
\]
From the definition of \(F\) we find that
\[
  \diff g(y) = [\diff f(x)]^{-1} \text{, for all } (x, y) \in X' \times Y.
\]

Lets consider the restriction \(f: g(Y) \to W\). Since \(g\) is injective, the
restriction \(g: Y \to g(Y)\) is a bijection. Since \(f\) is continuous and
\(Y\) is open, then \(f^{-1}(Y) = g(Y)\) is open. Define \(X = g(Y)\), so that
\(f: X \to Y\) is a bijection and clearly \(g = f^{-1}\) for such
restriction. Hence we conclude that
\[
  \diff f^{-1}(y) = [\diff f(x)]^{-1} \text{, for all } (x, y) \in X \times Y.
\]
\end{proof}

\begin{theorem}[Open Map Theorem]
\label{thm:open-map-theorem}
Let \(\Omega \subseteq \R^n\) be an open set and \(f: \Omega \to \R^{n}\) be a continuously
differentiable map. If \(\diff f(x)\) is invertible for all \(x \in \Omega\), then
the map \(f\) is an open mapping --- that is, maps open subsets of \(\Omega\) to
open subsets of \(\R^n\).
\end{theorem}

\begin{proof}
Let \(x \in \Omega\) be any point. From hypothesis, \(\diff f(x)\) is invertible,
hence we can apply \cref{thm:inverse-map} in order to obtain an open
neighbourhood \(V_{x} \subseteq \Omega\) of \(x\) and \(V_{f(x)} \subseteq \R^n\) such that the map
\(f: V_{x} \to V_{f(x)}\) is a local bijection and hence \(f(V_x)\) is
open. With this in our hands, we can create an open cover \(\mathcal U =
\{V_{x} \subseteq \Omega \colon x \in \Omega\}\) of such neighbourhoods --- that is, given any open set
\(U \subseteq \Omega\), there exists a collection of open sets \(\mathcal V \subseteq \mathcal U\)
such that \(U = \bigcup_{V \in \mathcal V} V\) and since \(f(U) = \bigcup_{V \in \mathcal V}
f(V)\) is the union of open sets, then \(f(U)\) is open.
\end{proof}

\begin{theorem}[Maximal Rank Theorem]
\label{thm:maximal-rank-theorem}
Let \(\Omega \subseteq \R^n\) be an open set and \(x_0 \in \Omega\). Let \(f: \Omega \to \R^m\) be a
continuously differentiable map. Define \(y_0 \in \R^m\) so that \(f(x_0) =
y_0\). The following holds
\begin{enumerate}[(a).]\setlength\itemsep{0em}
  \item Suppose \(n \leq m\) and that \(\diff f(x_0)\) has maximal
    \(\rank(\diff f(x_{0})) = n\). Then there exists open sets \(\Omega_{y_0} \subseteq
    \R^m\) and \(\Omega_{x_0} \subseteq \Omega \subseteq \R^n\), respectively neighbourhoods of the points
    \(y_0\) and \(x_0\) with \(f(\Omega_{x_0}) \subseteq \Omega_{y_0}\), and a differentiable
    map \(g: \Omega_{y_0} \to \R^m\) such that the following diagram commutes
    \[
      \begin{tikzcd}
        \Omega_{x_{0}} \ar[rr, "f"] \ar[dr, hook, swap, "\iota"]
        & &\Omega_{y_0} \ar[dl, "g"] \\
        &\R^m &
      \end{tikzcd}
    \]
    Where \(\iota: \R^n \emb \R^m\) is the canonical inclusion map.
  \item Suppose \(n \geq m\) and that \(\diff f(x_0)\) has maximal
    \(\rank(\diff f(x_0)) = m\). Then there exists \(\Omega_{x_0} \subseteq \Omega \subseteq \R^n\),
    neighbourhood of \(x_0\), and a differentiable map \(g: \Omega_{x_{0}} \to \Omega\)
    such that the following diagram commutes
    \[
      \begin{tikzcd}
        &\Omega_{x_0} \ar[dr, two heads, "\pi"] \ar[ld, swap, "g"] & \\
        \Omega \ar[rr, "f"] & &\R^m
      \end{tikzcd}
    \]
    Where \(\pi: \R^n \twoheadrightarrow \R^m\) is the canonical projection map.
\end{enumerate}
\end{theorem}

\begin{proof}
\begin{enumerate}\setlength\itemsep{0em}
  \item Since \(\rank(\diff f(x_0)) = n\), then, from the rank plus nullity
    theorem we find that \(\ker (\diff f(x_0)) = 0\) and therefore \(\diff
    f(x_{0})\) is injective. Consider the matrix representation \(f'(x_0)\)
    of the differential \(\diff f(x_0)\). From the injective property of
    \(f\), there must exist \(n\) linearly independent rows in
    \(f'(x_0)\). Let \(A\) be the \((n \times n)\)-matrix containing these linearly
    indepent rows and \(C\) the \(((m - n) \times n)\)-matrix containing the
    remaining rows of \(f'(x_0)\). Do row operations on \(f'(x_0)\) so that we
    see it as equivalent to the matrix
    \[
      \begin{bmatrix}
        A \\ C
      \end{bmatrix}
    \]
    Notice that the collection of rows of \(A\) form a basis for \(\R^n\),
    thus \(A\) is invertible and hence \(\det A \neq 0\). Define a map \(F: \Omega \times
    \R^{m-n} \to \R^m\) given by the mapping \((x, y) \mapsto f(x) + (0, y)\), then we
    obtain
    \[
      F'(x_0, 0) =
      \begin{bmatrix}
        A &0 \\ C &I_{m - n}
      \end{bmatrix}
    \]
    which in particular makes the, otherwise dependent, rows of \(C\) into a
    collection of linearly independent vectors, by attaching the canonical
    base of the space \(\R^{m-n}\) into each of them. This makes \(F'(x_0,
    0)\) an invertible matrix. By applying \cref{thm:inverse-map} we obtain a
    neighbourhood \(U \subseteq \Omega \times \R^{m-n}\) of \((x_0, 0)\) and a neighbourhood
    \(\Omega_{y_0} \subseteq \R^m\) of \(F(x_0, 0) = f(x_{0}) = y_0\) for which the
    restriction map \(F: U \to \Omega_{y_0}\) is an isomorphism of manifolds. Let \(g:
    \Omega_{y_0} \to U\) be the continuously differentiable inverse of \(F\), and
    define \(\Omega_{x_0} = f^{-1}(\Omega_{y_0}) \cap \Omega\), which is clearly a neighbourhood
    of \(x_0\). Notice that the composition \(g f(x) = g F (x, 0) = (x, 0) =
    \iota(x)\), thus we are done.

  \item Since \(\rank(\diff f(x_0))\) equals the dimension of its codomain, it
    follows that \(\diff f(x_0)\) is a surjective linear map. Since \(\diff
    f(x_0)\) has rank \(m\), then its matrix representation \(f'(x_0)\) has
    \(m\) linearly independent columns. Let \(D\) be the \((m \times m)\)-matrix
    whose columns
    are those of \(f'(x_0)\) that are linearly independent and \(C\) be the
    \((m \times (n - m))\)-matrix composed of the remaining columns of
    \(f'(x_0)\). Operate on the matrix \(f'(x_0)\) via column operations so
    that its final equivalent matrix is
    \[
      \begin{bmatrix}
        D & C
      \end{bmatrix}
    \]
    Since \(D\) is composed of linearly independent vectors, \(D\) is
    invertible. Define the map \(F: \Omega \to \R^m \times \R^{n-m}\) by \((x, y) \mapsto (f(x),
    y)\). Clearly, \(F\) is differentiable at \(x_0\) and its matrix
    representation is
    \[
      F'(x_0) =
      \begin{bmatrix}
        0 & I_{n - m}\\
        D & C
      \end{bmatrix}
    \]
    which is invertible since the attachment of the canonical basis of
    \(\R^{n-m}\) into the column vectors of \(C\) transforms the collection of
    the last \(n - m\) column vectors of \(F'(x_{0})\) into a linearly
    independent set. Applying \cref{thm:inverse-map} we are able to obtain a
    neighbourhood \(\Omega_{x_0} \subseteq \Omega\) such that the restriction map \(F: \Omega_{x_0} \to
    \R^m \times\R^{n-m}\) is an isomorphism of manifolds. Let \(g: \R^m \times \R^{n-m} \to
    \Omega_{x_0}\) be its continuously differentiable inverse map, then the
    composition \(f g(x, y) = x = \pi(x, y)\) is merely the canonical projection
    map, as we expected.
\end{enumerate}
\end{proof}

\section{Extrema With Constraints}

\begin{theorem}[Existence of Lagrange multipliers]
\label{thm:existence-lagrange-multipliers}
Let \(f: \Omega \to \R\) be a map, where \(\Omega \subseteq \R^d\), and a map \(F: \Omega \to \R^m\),
where \(m < d\) ---  which will be called constraint map. Suppose \(x_0\) is a
local extremum of \(f\) in the surface \(S \subseteq \Omega\) defined by the constraint \(S
= \{x \in \Omega \colon F(x) = 0\}\). Suppose additionally that \(\rank F'(x) = m\) for
every \(x \in \Omega\). Then, there exists a vector \(\lambda \in \R^m\) --- whose components
are called Lagrange multipliers --- such that
\[
  f'(x_0) = F'(x_0) \lambda.
\]
In the particular case where \(m = 1\), the constraint map generates a \(1\)
dimensional surface and therefore \(\grad f(x_0) = \lambda \grad F(x_0)\), where
\(\lambda \in \R\).
\end{theorem}

\begin{proof}
Since \(F'(x_0)\) has rank \(m\), let \(D\) be the \((m \times m)\)-matrix whose
columns are the \(m\) linearly independent columns of \(F'(x_0)\). Define
\(C\) to be the \((m \times d - m)\)-matrix whose columns are the remaining columns
of \(F'(x_0)\). By means of column operations, arrange \(F'(x_0)\) into an
equivalent matrix of the form
\[
  \begin{bmatrix}
    C & D
  \end{bmatrix}
\]
That is, the equivalent matrix has an invertible principal minor \(D\) defined
on its last \(m\) columns. If we now identify points \(x \in \R^d\) with points
\((x_1, x_2) = x \in \R^{d-m} \times \R^m\), we find that \(\partial_2 F(x_0^1, x_0^2)\) is an
invertible map (it corresponds to the matrix \(D\)).

We can now apply the Implicit Map Theorem to find neighbouhoods \(\Omega_{x_{0}^1}
\subseteq \R^{d-m}\) and \(\Omega_{x_0^2} \subseteq \R^m\) of \(x_0^1\) and \(x_0^2\), respectively,
and a map \(g: \Omega_{x_0^1} \to \Omega_{x_0^2}\) for which \(F(x_{1}, x_2) = 0\) if and
only if \(g(x_1) = x_2\). In particular, the mapping \(x_1 \mapsto F(x_1, g(x_1))\)
is identically zero and therefore \(\Omega_{x_0^1} \times g(\Omega_{x_0^1}) \subseteq S\). From the
fact that \(x_0 = (x_0^1, x_0^2) = (x_0^1, g(x_0^1))\) is a local extremum of
\(f\), defining \(\phi: \Omega_{x_0^{1}} \to \R\) by \(\phi(x_1) = f(x_1, g(x_1))\), we
conclude that \(x_0^1\) is a local maximum of \(\phi\). In particular, it is
necessary that \(\diff \phi(x_0^1) = 0\) and therefore
\[
  \diff \phi(x_0^{1}) = \sum_{j=1}^{d-m} \partial_j f(x_0^1, g(x_0^1)) + \sum_{i = d - m +
    1}^{m} \partial_i f(x_0^1, g(x_0^1)) \partial_i g(x_0^1) = 0.
\]
If we now regard \(f\) as map of the form \(\R^{d-m} \times \R^m \to \R^m\), then we
see that
\[
  \diff \phi(x_0^1) = \partial_1 f(x_0^1, g(x_0^1)) + \partial_{2} f(x_0^1, g(x_0^1)) \diff
  g(x_0^1) = 0.
\]

From the Implicit Map Theorem, \(g\) was constructed so that
\[
  \diff g(x_0^1) = - [\partial_2F(x_0^1, g(x_0^1))]^{-1} \partial_1 F(x_0^1, g(x_{0}^1)).
\]
Hence, substituting into the above equation we find
\[
  \partial_1 f(x_0^1, g(x_0^1)) - \partial_2 f(x_0^1, g(x_0^1)) [\partial_2 F(x_0^1,
  g(x_0^1))]^{-1} \partial_1 F(x_0^1, g(x_0^1)) = 0
\]
Now, if we define \(\lambda \in \R^m\) by \(\lambda = \partial_2f(x_0^{1}, g(x_{0}^{1})) [\partial_2
F(x_{0}^1, g(x_0^1))]^{-1}\), we find that
\begin{gather*}
  \partial_1 f(x_0^{1}, g(x_0^1)) = \partial_1 F(x_0^1, g(x_0^1)) \lambda \\
  \partial_2 f(x_0^{1}, g(x_0^1)) = \partial_2 F(x_0^1, g(x_0^1)) \lambda \\
\end{gather*}
So that we can conclude
\[
  \diff f(x_0) = \diff F(x_0) \lambda.
\]
\end{proof}

\begin{theorem}[Sufficient condition for a constraint extremum]
\label{thm:sufficient-extremum}
Let \(\Omega \subseteq \R^d\) and maps \(f: \Omega \to \R\), and  \(F: \Omega \to \R^m\), both of which
are \(C^2(\Omega, \R)\). Let \(S \subseteq \Omega\) be the surface defined by \(S = \{x \in \Omega \colon
F(x) = 0\}\). Consider \(x_0 \in S\) as a possible candidate of local extremum
of \(f\) in the surface \(S\). Suppose additionally that \(\rank F'(x) =
m\) for every \(x \in \Omega\). Define the lagrange multipliers \(\lambda \in \R^m\) so that
the map \(L: S \to \R\) given by
\[
  L(x) = f(x) - \left\langle \lambda, F(x) \right\rangle,
\]
satisfy \(\grad L(x) = 0\) for all \(x \in S\) and \(\partial_{\lambda} L(x) = 0\).

It's sufficient for \(x_0 \in S\) to be a local extremum in \(S\) if \(\Hess
L(x_0)\) is either positive definite or negative definite. If \(\Hess L(x_0)\)
is not definite, then \(x_0\) cannot be an extremum point. Moreover, if \(\Hess
L(x_0)\) is positive definite, then \(x_0\) is a local minimum on \(S\), on the
other hand, if \(\Hess L(x_0)\) is negative definite, then \(x_0\) is a local
maximum on \(S\).
\end{theorem}

\begin{proof}
Since \(x_0 \in S\), then in particular \(\grad L(x_0) = 0\), thus, as \(S \ni x \to
x_0\) we have the polynomial approximation
\begin{align*}
  L(x) - L(x_0)
  &= \frac{1}{2!} \sum_{i, j = 1}^{d} \partial_{i j} L(x_0) (x_i - x_0^i) (x_j - x_0^j)
  + o(\norm{x - x_0}^2) \\
  &= \frac{1}{2!} \left\langle x - x_{0}, \Hess(L(x_0)) (x - x_0) \right\rangle
  + o(\norm{x - x_0}_{\R^d}^2).
\end{align*}

We'll assume that \(S\) --- which is a \((d - m)\)-dimensional surface, since
\(\rank F(x) = m\) --- can be parametrically defined in some neighbourhood of
\(x_0 \in S\) by a smooth map \(\R^{d - m} \ni t \mapsto x(t) \in \R^d\) such that \(x(0)
= x_0\) and that exists a neighbourhood of \(0 \in \R^{d-m}\) for which the
parametrization is bijective. Since the mapping is smooth, as \(t \to 0\) we
have
\[
  x(t) - x(0) = \diff x(0)(t) + o(\norm{t}_{\R^{d-m}}).
\]
Which implies that as \(t \to 0\) we have \(\norm{x(t) - x(0)}_{\R^d} =
O(\norm{t}_{\R^{d}})\).

We can now exploit the parametrization of the surface \(S\) so that, as \(t \to
0\) we have
\begin{align*}
  L(x(t)) - L(x_0)
  &= \frac{1}{2!} \left\langle t, \Hess(L(x(0))) t \right\rangle +
  o(O(\norm{t}_{\R^d})) \\
  &= \frac{1}{2!} \left\langle t, \Hess(L(x(0))) t \right\rangle +
  o(\norm{t}_{\R^d}).
\end{align*}
Where the entries of the Hessian are of the form \(\partial_{ij} L(x(0)) = \partial_{i j}
L(x(0)) \partial_i x(0) \partial_j x(0)\). Then, if \(\Hess L(x(0))\) is positive or negative
definite, by \cref{thm:classification-critical-points}, we obtain that \(t =
0\) is an extremum of \(L(x(t))\). On the other hand, since there exists a
neighbourhood of \(t = 0\) for which \(x(t)\) is a bijective parametrization,
it follows that \(L\) has an extremum at \(x_0\) --- and hence \(x_0\) is an
extremum of \(f\) in \(S\) and the classifications of maximum or minimum come
again from the same theorem. If \(\Hess L(x(0))\) is indefinite, by
\cref{thm:classification-critical-points} we conclude that \(L(x(t))\) has no
extremum at \(t = 0\) --- and with the same analogous arguments as before, we
argue that \(L\) has no extrema at \(x_0\) and neither does \(f\).
\end{proof}
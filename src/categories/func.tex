\section{Functors}

\begin{definition}[Covariant functor]\label{def: functor}
Let \(\cat C\) and \(\cat D\) be categories. A covariant functor \(F: \cat C \to
\cat D\) has the following data:
\begin{enumerate}[({DF}1)]
\item For all \(c \in \cat C\) exists a corresponding \(F c \in \cat
  D\)\footnote{When convenient, we may discard the use of parenthesis, but in
    occasions where the use of parenthesis brings more clarity to the
    situation, we shall use it.}.
\item For all \(f: c \to c' \in \Mor(\cat C)\) there exists a morphism \(F
  f: F c \to F c' \in \Mor(\cat D)\).
\end{enumerate}
Such data satisfies the two following axioms:
\begin{enumerate}[({AF}1)]
\item For all composable \(f, g \in \Mor(\cat C)\), we have \(F g \circ F f
  = F (g \circ f)\).
\item For all \(c \in \cat C\) we have \(F \Id_c = \Id_{F c} \in \Mor(\cat
  D)\).
\end{enumerate}
\end{definition}

\begin{definition}[Contravariant functor]
\label{def: contravariant functor}
A contravariant functor from categories \(\cat C\) to \(\cat D\) is a functor
\(F: \cat C^\op \to \cat D\) together with the following data:
\begin{enumerate}[({DCF}1)]
\item For all \(c \in \cat C\) exists \(F c \in \cat D\).
\item For all \(f: c \to c' \in \Mor(\cat C)\) we have \(F f: F c' \to F c
  \in \Mor(\cat D)\).
\end{enumerate}
Moreover, a contravariant functor satisfies the following axioms:
\begin{enumerate}[({ACF}1)]
\item For all composable \(f, g \in \Mor(\cat C)\) we have \(F f \circ F g =
  F (g \circ f)\).
\item For all \(c \in \cat C\) we have \(F \Id_c = \Id_{F c}\).
\end{enumerate}
This can all be comprised diagrammatically as:
\[
  \begin{tikzcd}
    \cat C^\op \ar[rr, "F"] & &\cat D
    \\
    c \ar[d, "f"] \ar[dd, bend right = 60, swap, "g f"]
    \ar[rr, mapsto]
    & &F c
    \\
    c' \ar[d, "g"] \ar[rr, mapsto]
    & &F c' \ar[u, "F f"]
    \\
    c'' \ar[rr, mapsto]
    & & F c''
    \ar[u, "F g"]
    \ar[uu, bend right = 60, swap, "F f \circ F g = F (gf)"]
  \end{tikzcd}
\]
\end{definition}

\begin{definition}[Composition of functors]
\label{def:composition-functors}
Let \(F: \cat C \to \cat D\) and \(G: \cat D \to \cat B\) be functors. We define
their composition \(G F: \cat C \to \cat B\) by \((GF)(x) = G(F(x))\), for all
\(x \in \cat C\), and \((G F)(f) = G(F(f))\) for all morphism \(f \in \Mor(\cat
C)\).
\end{definition}

\begin{example}
\label{exp:opposite-category}
An important example of contravariant functor is \(\op: \cat C \to \cat
C^{\op}\), where \(\cat C\) is some category, defined by the identity on objects
and morphisms.
\end{example}

\begin{definition}[Forgetful functor]
A functor is said to be forgetful if the functor ``forgets'' some object,
structure or property of its domain category.
\end{definition}

\begin{example}
We have some classical examples of forgetful functors, for instance, the
following are functors that forget the structure of their domain categories:
\begin{itemize}
\item The functor \(G: \Grp \to \Set\) mapping groups to its corresponding
  underlying set.
\item The functor \(T: \Top \to \Set\) maps any topological space
  to its corresponding set of points.
\item The functor \(V, E: \Graph \to \Set\) maps the vertices and
  edges of a graph to the set of such vertices an edges.
\end{itemize}
\end{example}

\begin{example}[\(\Top^\op \to \Rng\)]
Let \(C: \Top^\op \to \Rng\) be a contravariant functor such that
for all \(X \in \Top\), let \(C X\) be the ring of continuous functions
\(X \to \R\). The ring operations on \(C X\) are defined pointwise,
that is, given \(p, q: X \to \R \in C X\) we have \((p \cdot q) (x) =
p(x) \cdot q(x)\) and \((p + q)(x) = p(x) + q(x)\) for all \(x \in X\).
Moreover, given a morphism \(f: X \to Y \in \Mor(\Top)\) we define \(C f:
C Y \to C X\) as the composition \((C f)(q) = q f \in \Mor(C X)\) for all \(q
\in C Y\), that is
\[
  \begin{tikzcd}
    X \ar[r, "f"] \ar[rr, bend right = 40, swap, "C f(q) = q f"]
    & Y \ar[r, "q"] & \R
  \end{tikzcd}
\]

We now show that the axioms for the contravariant functor are satisfied by
\(C\). Let \(f: X \to Y\) and \(g: Y \to Z\), then given any \(p \in C Z\) we
have
\[
  \begin{tikzcd}
    X \ar[r, "f"]
    \ar[rrr, bend left = 60, "C f(C g (p))"]
    \ar[rrr, bend right, swap, "C (g f)(p)"]
    & Y \ar[r, "g"]
    \ar[rr, bend left = 35, "C g(p)"]
    & Z \ar[r, "p"] & \R
  \end{tikzcd}
\]
hence \(C (f) C (g) = C (g f)\). Moreover, given any \(X \in \Top^\op\)
we find that \(C \Id_X: C X \to C X\) is such that for all \(q \in C X\), \(C
\Id_X (q) = q \Id_X = q\) hence \(C \Id_X = \Id_{C X}\). This finishes the
proof that \(C: \Top^\op \to \Rng\) is a contravariant functor.
\end{example}

\begin{definition}[Presheaf]\label{def: presheaf}
Let \(\cat C\) be a \(\mathcal U\)-small category. A contravariant functor
\(\cat C^\op \to \Set\) is called a presheaf on \(\cat C\).
\end{definition}

\begin{example}[\(\mathcal O(X)^\op \to \Set\)]
Let \(X \in \Top\) we define \(\mathcal O(X)\) to be the poset category
whose objects are open sets of \(X\). That is, for sets \(U, U' \in \mathcal
O(X)\), if \(U \subseteq U'\), then there exists a morphism \(U \to U'\) in
\(\Mor(\mathcal O(X))\). A presheaf on the category \(\mathcal O(X)\) is a
functor \(F: \mathcal O(X)^\op \to \Set\) that assigns \(F U = \{f: U \to
\R: f \text{ continuous}\}\) for all \(U \in \mathcal O(X)\).
Moreover, for maps \(g: U \to U'\) (that is \(U \subseteq U'\)) we have \(F g:
F U' \to F U\) such that \(F g(f) = f|_U: U \to \R\) for all \(f: U'
\to \R\) continuous. Since the restriction of a continuous map is
continuous, then \(f|_U \in F U\)
\end{example}

\begin{example}[Simplex category]
\label{exp:simplex-category}
The simplex category \(\Delta\) comprises objects that are finite non-empty
ordinals and order-preserving morphisms. Simplicial sets are defined as
presheaves \(\Delta^{\op} \to \Set\).
\end{example}

\begin{lemma}\label{lem: functor preserve iso}
Functors preserve isomorphisms. Let \(\cat C\) and \(\cat D\) be categories
and \(F: \cat C \to \cat D\) be a functor. Given an isomorphism \(f: c \isoto
c' \in \Mor(\cat C)\), we have that \(F f : F c \isoto c'\) is an isomorphism.
\end{lemma}

\begin{proof}
Denote by \(f^{-1}: c' \isoto c\) the inverse of \(f\). By the composition
axiom we have
\begin{gather*}
  F (f^{-1}) F (f) = F (f^{-1} f) = F \Id_c    = \Id_{F c}, \\
  F (f) F (f^{-1}) = F (f f^{-1}) = F \Id_{c'} = \Id_{F c'}.
\end{gather*}
This shows that \(F f^{-1}\) is the right and left inverse of \(F f\), hence
\(F f: F c \isoto F c'\) is indeed an isomorphism.
\end{proof}

\begin{example}[Group action]\label{exp:grp-action}
Let \(G\) be any group and consider the category \(\cat{B}G\) generated by \(G\)
--- that is, \(\cat BG\) consists of a unique object \(*\) and the morphisms of
the category are automorphisms \(* \to *\) given by the elements of \(G\). Given
a category \(\cat C\), a functor \(X: \cat BG \to \cat C\) --- given by mapping
\(X* \coloneq X \in \cat C\) and each object \(g \in G\) to an endomorphism \(Xg
\coloneq g_{*}: X \to X\) --- is said to define a left group action on the
object \(X \in \cat C\). Moreover, the functor \(X\) has to obey
\begin{itemize}\setlength\itemsep{0em}
\item Composition preserving: for any \(h, g \in G\), we have \((h g)_{*} =
  h_{*} g_{*}\).
\item Identity: \(e_{*} = \Id_X\).
\end{itemize}
Since functors preserve isomorphisms and every morphism of \(\cat BG\) is an
automorphism, it follows that, for all \(g \in G\), the map \(g_{*}: X \to X\)
is an automorphism --- in particular, this implies that \((g^{-1})_{*} =
g_{*}^{-1}\).

Some particular cases of interest are the following:
\begin{itemize}\setlength\itemsep{0em}
\item If \(\cat C = \Set\), then the set \(X\) together with the actions
  \(\{g_{*} : g \in G\}\) is called a \(G\)-set.
\item If \(\cat C = \Vect_k\), then the \(k\)-vector space \(X\) together with
  the actions generated by \(G\) is said to be a \(G\)-representation.
\item If \(\cat C = \Top\), then the topological space \(X\) endowed with the
  actions generated by \(G\) is called a \(G\)-space.
\end{itemize}

A right group action is nothing more than a contravariant functor \(X: \cat
BG^{\op} \to \cat C\) such that \(X* \coloneq X\) and \(Xg \coloneq g^{*}: X \to
X\) are endomorphisms. The rules for such a functor are the contravariant
preservation of compositions, that is, \((h g)_{*} = g_{*} h_{*}\), and that
\(e^{*} = \Id_X\) as before.
\end{example}

\begin{lemma}\label{lem: func-preserve-split}
Functors preserve split monomorphisms and split epimorphisms.
\end{lemma}

\begin{proof}
Let \(\cat C\) and \(\cat D\) be categories and consider a functor \(F: \cat C
\to \cat D\). Define morphisms \(x \xrightarrow s y \xrightarrow r x\) in
\(\Mor(\cat C)\) such that \(r s = \Id_x\), that is, \(s\) is a split
monomorphism and \(r\) is a split epimorphism. Consider the morphisms \(F s: F
x \to F y\) and \(F r: F y \to F x\) in \(\Mor(\cat D)\). Notice that \(F (s)
F(r) = F(s r) = F(\Id_x) = \Id_{F x}\). Hence \(F s\) is a split monomorphism
and \(F r\) is a split epimorphism.
\end{proof}

\begin{definition}[\(\Hom\) functors]\label{def:hom-functors}
Let \(\cat C\) be a \(\mathcal U\)-category. Given any \(c \in \cat C\), there
exists a pair of covariant and contravariant functors, \(\Hom(c, -)\) and
\(\Hom(-, c)\), respectively --- represented by the object \(c\). That is:
\[
  \begin{tikzcd}
    \cat C \ar[rr, "\Hom{(c, -)}"] & & \Set
    \\
    x \ar[rr, maps to] \ar[d, swap, "f"]
    & & \Hom(c, x) \ar[d, "f_*"]
    \\
    y \ar[rr, maps to] & & \Hom(c, y)
  \end{tikzcd}
  \qquad
  \begin{tikzcd}
    \cat C^\op \ar[rr, "\Hom{(-, c)}"] & & \Set
    \\
    x \ar[rr, maps to] \ar[d, swap, "f"]
    & & \Hom(x, c)
    \\
    y \ar[rr, maps to] & & \cat \Hom(y, c) \ar[u, swap, "f^*"]
  \end{tikzcd}
\]
\end{definition}

We now prove that such definition indeed satisfies the axioms for covariant and
contravariant functors. Given morphisms \(f: x \to y\) and \(g: y \to z\) in
\(\Mor(\cat C)\), we see that \(g_* f_* = (g f)_*\), moreover \(f^* g^* = (f
g)^*\). Let \(x \in \cat C\) be any object, then \(\Id_{x *} = \Id_{\Hom(c, x)}
= \Id_x^*\). This proves that \(\Hom(c, -)\) is covariant and \(\Hom(-, c)\) is
contravariant.

\begin{definition}[Faithful, full \& its friends]
\label{def:faithful-full-fully-faithful-essencially-surjective-conservative}
Let \(\cat C\) and \(\cat D\) be categories. A functor \(F: \cat C \to \cat D\)
is said to be
\begin{enumerate}[(a)]\setlength\itemsep{0em}
\item \emph{Faithful} if for all \(x, y \in \cat C\) the map \(\Hom_{\cat C}(x,
  y) \mono \Hom_{\cat D}(F x, F y)\) is injective.
\item \emph{Full} if for all \(x, y \in \cat C\) the map \(\Hom_{\cat C}(x, y)
  \epi \Hom_{\cat D}(F x, F y)\) is surjective.
\item \emph{Fully faithful} if for all \(x, y \in \cat C\) the map \(\Hom_{\cat
    C}(x, y) \isoto \Hom_{\cat D}(F x, F y)\) is a bijection.
\item \emph{Essencially surjective} if for each \(y \in \cat D\) there exists
  \(x \in \cat C\) and an isomorphism \(F x \isoto y\) in \(\cat D\).
\item \emph{Conservative} if, given a morphism \(f\) in \(\cat C\), if \(F f\)
  is an isomorphism in \(\cat D\), then \(f\) is an isomorphism in \(\cat C\).
\end{enumerate}
\end{definition}

\begin{definition}[Product \& disjoint union categories]
\label{def:product-disjoint-categories}
Let \(I\) be an indexing set and \(\{\cat C_{i}\}_{i \in I}\) be a collection of
categories associated with \(I\). We define the following categories:
\begin{enumerate}[(a)]\setlength\itemsep{0em}
\item The \emph{product} category \(\prod_{i \in I} \cat C_i\) consists of objects
  \(\Obj(\prod_{i \in I} \cat C_i) \coloneq \prod_{i \in I} \Obj(\cat C_i)\),
  and morphisms \(\Hom_{\prod_{i \in I} \cat C_i}((x_i)_{i \in I}, (y_i)_{i \in
    I}) \coloneq \prod_{i \in I} \Hom_{\cat C_i}(x_i, y_i)\) between any two
  objects \((x_i)_{i \in I}\) and \((y_i)_{i \in I}\) in the category. Composable
  morphisms \((f_i)_{i \in I}\) and \((g_i)_{i \in I}\) have composition defined
  componentwise --- that is, \((f_i)_{i \in I} (g_i)_{i \in I} \coloneq
  (f_i g_i)_{i \in I}\).
\item The \emph{disjoint union} category \(\coprod_{i \in I} \cat C_i\) consists
  of objects
  \[
    \Obj\bigg(\coprod_{i \in I} \cat C_i\bigg) \coloneq
    \coprod_{i \in I} \{(x, i) : i \in I \text{ and } x \in \cat C_i\},
  \]
  and morphisms
  \[
    \Hom_{\coprod_{i \in I} \cat C_i}((x, i), (y, j)) \coloneq
    \begin{cases}
      \Hom_{\cat C_i}(x, y), &\text{if } i = j \\
      \emptyset, &\text{otherwise}
    \end{cases}
  \]
  for any objects \((x, i)\) and \((y, j)\) in the category.
\end{enumerate}
Moreover, if \(\{\cat D_{i}\}_{i \in I}\) is another collection of categories,
and \(\{F_{i}: \cat C_i \to \cat D_i\}_{i \in I}\) is a collection of functors,
we associate to each of the above categories the functors \(\prod_{i \in I}
F_i\) and \(\coprod_{i \in I} F_i\).
\end{definition}

\begin{definition}[Two sided represented functor]
\label{def:two-sided-represented-functor}
Let \(\cat C\) be a \(\mathcal U\)-category, then there is a functor \(\Hom(-,
-): \cat C^{\op} \times \cat C \to \Set\) defined by mapping objects \((x, y)
\mapsto \Hom(x, y)\) and morphisms \((f, g): (a, y) \to (x, b)\) to a
set-function \((f^{*}, g_{*}): \Hom(x, y) \to \Hom(a, b)\) given by the mapping
\(g \mapsto h g f\).
\end{definition}

\begin{example}[Orbit category]
\label{exp:orbit-category}
Let \(G\) be a group. We define the orbit category associated to \(G\) as
\(\mathcal{O}_G\) whose objects are subgroups \(H \subseteq G\), identified by
the left \(G\)-set \(G/H\) of left cosets of \(H\). The morphisms \(\phi: G/H \to
G/Q\) are maps commuting with the left \(G\)-action, that is, \(\phi(g_{*}h) =
g_{*} \phi(h)\) --- such maps are called \(G\)-equivariant.
\end{example}

\begin{proposition}[Bifunctor]\label{def:bifunctor}
Let \(\cat A, \cat B\) and \(\cat C\) be any categories. A functor
\[
  F: \cat A \times \cat B \longrightarrow \cat C
\]
is called a \emph{bifunctor}. The functor \(F\) is defined so that, given any
objects \(x \in \cat A\) and \(y \in \cat B\),
\[
  F(x, -): \cat B \longrightarrow \cat C\ \text{ and }\
  F(-, y): \cat A \longrightarrow \cat C
\]
are both functors. Moreover, given any morphisms \(f: x \to y\) in \(\cat A\)
and \(g: x' \to y'\) in \(\cat B\), the following diagram commutes
\[
  \begin{tikzcd}
    F(x, x') \ar[rr, "F{(x, g)}"] \ar[d, swap, "F{(f, x')}"]
      & &F(x, y') \ar[d, "F{(f, y')}"] \\
    F(y, x') \ar[rr, "F{(y, g)}"]
      & &F(y, y')
  \end{tikzcd}
\]
\end{proposition}

\begin{definition}
Let \(F: \cat C \to \cat D\) be a functor between categories \(\cat C\) and
\(\cat D\), and let \(y \in \cat D\). We define the following two categories:
\begin{enumerate}[(a)]\setlength\itemsep{0em}
\item The category \(\cat C_y\) is defined to consist of objects
  \[
    \Obj(\cat C_y) \coloneq
    \{(x, s) : x \in \cat C \text{ and } s: F x \to y \text{ in } \cat D\},
  \]
  and morphisms between any objects \((a, s), (b, t) \in \cat C_{y}\) are
  defined to be
  \[
    \Hom_{\cat C_y}\left( (a, s), (b, t) \right) \coloneq
    \{f \in \Hom_{\cat C}(a, b) : s = t F(f) \text{ in } \cat D\},
  \]
  that is, a morphism \(f: (a, s) \to (b, t)\) makes to following diagram
  commute
  \[
    \begin{tikzcd}
      F a \ar[r, "s"] \ar[dr, swap, "Ff"] &y \\
      &F b \ar[u, swap, "t"]
    \end{tikzcd}
  \]
  Together with such category, we define a faithful functor \(j_y: \cat C_y \to
  \cat C\) by \(j_y(x, s) \coloneq x\), acting as a projection.
\item The category \(\cat C^y\) is defined to consist of objects
  \[
    \Obj(\cat C^y) \coloneq
    \{(x, s) : x \in \cat C \text{ and } s: y \to F x \text{ in } \cat D\},
  \]
  and morphisms between any objects \((a, s), (b, t) \in \cat C^{y}\) are
  defined to be
  \[
    \Hom_{\cat C^y}\left( (a, s), (b, t) \right) \coloneq
    \{f \in \Hom_{\cat C}(a, b) : t = F(f) s \text{ in } \cat D\},
  \]
  that is, a morphism \(f: (a, s) \to (b, t)\) makes to following diagram
  commute
  \[
    \begin{tikzcd}
      y \ar[r, "s"] \ar[d, swap, "t"] &Fa \ar[dl, "F f"] \\
      F b &
    \end{tikzcd}
  \]
  Together with such category, we define a faithful functor \(j^y: \cat C^y \to
  \cat C\) by \(j^y(x, s) \coloneq x\), which acts as a projection.
\end{enumerate}
\end{definition}

\begin{definition}[Equivalence classes]
\label{def:equivalence-class-category}
Let \(\cat C\) be a category and \(\sim\) denote an equivalence relation on the
objects of \(\cat C\) --- where \(x \sim y\) if \(\Hom_{\cat C}(x, y) \neq
\emptyset\). We denote the collection of all equivalence classes on the objects
of \(\cat C\) by \(\pi_0(\cat C)\).
\end{definition}

\begin{corollary}
\label{cor:connected-iff-class-point}
A category \(\cat C\) is \emph{connected} if and only if \(\pi_0(\cat C)\)
consists of a single element.
\end{corollary}

\begin{proof}
If \(\pi_0(\cat C)\) is a single object, every equivalence class on \(\cat C\)
is such that every element is equivalent to each other, which implies that the
collection of morphisms \(\Hom_{\cat C}(x, y)\), between any two elements \(x, y
\in \cat C\), is non-empty --- thus clearly \(\cat C\) is connected.
\end{proof}

\begin{definition}[Isomorphisms of monomorphisms \& epimorphisms]
\label{def:iso-mono-epi}
Let \(\cat C\) be a category and \(x, y, z \in \cat C\) be any objects. We
define the following concepts:
\begin{enumerate}[(a)]\setlength\itemsep{0em}
\item Two monomorphisms \(f: x \mono z\) and \(g: y \mono z\) in \(\cat C\) are
  said to be \emph{isomorphic} in \(\cat C\) if there exists an isomorphism \(h:
  x \isoto y\) for which the following diagram commutes in
  \[
    \begin{tikzcd}
      &z & \\
      x \ar[rr, "h", "\iso"'] \ar[ur, tail, "f"] &
      &y \ar[ul, tail, swap, "g"]
    \end{tikzcd}
  \]
  This is equivalent to \(f\) and \(g\) being isomorphic in \(\cat C_z\).
\item Two epimorphisms \(f: x \epi z\) and \(g: y \epi z\) in \(\cat C\) are
  said to be \emph{isomorphic} in \(\cat C\) if there exists an isomorphism \(h:
  x \isoto y\) for which the following diagram commutes in
  \[
    \begin{tikzcd}
      x \ar[rr, "h", "\iso"'] \ar[dr, two heads, swap, "f"] &
      &y \ar[dl, two heads, "g"] \\
      &z &
    \end{tikzcd}
  \]
  This is equivalent to \(f\) and \(g\) being isomorphic in \(\cat C^z\).
\end{enumerate}
\end{definition}

\begin{definition}
\label{def:subobject-quotient}
Let \(\cat C\) be a category and \(x \in \cat C\) be any object. We define the
following:
\begin{enumerate}[(a)]\setlength\itemsep{0em}
\item An isomorphism class of a monomorphism with \emph{target} \(x\) is called a
  \emph{subobject} of \(x\).
\item An isomorphism class of an epimorphim with \emph{source} \(x\) is called a
  \emph{quotient} of \(x\).
\end{enumerate}
These isomorphism classes are given in the sense of \cref{def:iso-mono-epi}.
\end{definition}

\begin{example}[Ordering subobjects]
\label{exp:order-subobject}
One can \emph{order} the collection of subobjects of a given object \(x \in \cat
C\) by defining a relation \([f: y \mono x] \leq [g: z \mono x]\) if there
exists \(h: y \to z\) such that
\[
  \begin{tikzcd}
    y \ar[r, tail, "f"] \ar[rd, dashed, swap, "h"] &x \\
    &z \ar[u, tail, swap, "g"]
  \end{tikzcd}
\]
Moreover, \emph{if} \(h\) exists, then it's unique.
\end{example}

\begin{definition}[Comma category]
\label{def:comma-category}
\todo[inline]{Comma category}
\end{definition}

\section{Natural Transformations}

\begin{definition}[Natural transformation]
\label{def:natural-transformation}
Let \(\cat C\) and \(\cat D\) be categories, and consider functors \(F, G:
\cat C \rightrightarrows \cat D\). A \emph{natural transformation} \(\alpha: F
\nat G\) consists of morphisms \(\alpha_x: F x \to G x\), for all \(x \in \cat
C\), such that, for any morphism \(f: x \to y\) in \(\cat C\), the following
diagram commutes
\[
  \begin{tikzcd}
    F x \ar[r, "\alpha_x"] \ar[d, swap, "F f"]
    &G x \ar[d, "G f"] \\
    F y \ar[r, "\alpha_y"] &Gy
  \end{tikzcd}
\]
Moreover, if \(L: \cat C \to \cat D\) is another functor, and \(\beta: G \nat
L\) is a natural transformation, we define the \emph{composition} of natural
transformations \(\alpha\) and \(\beta\) as the map \(\beta \alpha: F \nat L\)
such that \((\beta \alpha)_x \coloneq \beta_x \alpha_x\) for every \(x \in \cat
C\).
\end{definition}

\begin{definition}[Functor category]
\label{def:functor-category}
Let \(\cat C\) and \(\cat D\) be categories. We define a category \(\Fct(\cat C,
\cat D)\) whose objects are functors \(\cat C \to \cat D\), and morphisms are
natural transformations between functors.
\end{definition}

\begin{definition}[Horizontal composition]
\label{def:horizontal-composition-natural-transformation}
Let \(\cat A, \cat B\) and \(\cat C\) be categories. Consider functors \(F, F':
\cat A \rightrightarrows \cat B\), and \(G, G': \cat B \rightrightarrows \cat
C\).  Let \(\alpha: F \nat F'\) and \(\beta: G \nat G'\) be natural
transformations --- that is,
\[
  \begin{tikzcd}
    \cat A \ar[r, bend left, "F"{name=F}]
    \ar[r, bend right, "{F'}"'{name=FF}]
    &\cat B \ar[r, bend left, "G"{name=G}]
    \ar[r, bend right, "{G'}"'{name=GG}]
    & \cat C
    \ar[Rightarrow, from=F, to=FF, "\alpha"]
    \ar[Rightarrow, from=G, to=GG, "\beta"]
  \end{tikzcd}
\]
We define the \emph{horizontal composition} of \(\beta\) with \(\alpha\) as the
natural transformation \(\beta \alpha: G F \nat G' F'\) so that
\[
  \begin{tikzcd}
    \cat A \ar[r, bend left=40, "G F"{name=s}]
    \ar[r, bend right=40, swap, "{G' F'}"{name=t}]
    &\cat C
    \ar[Rightarrow, from=s, to=t, "\beta \alpha"]
  \end{tikzcd}
\]
\end{definition}

\begin{definition}[Vertical composition]
\label{def:vertical-compostion-natural-transformation}
Let \(\cat C\) and \(\cat D\) be categories, and consider functors and natural
transformations given in the following diagram
\[
  \begin{tikzcd}
  \cat C \ar[rr, bend left=60, "F"{name=F}]
  \ar[rr, "H"{near start, description}, ""'{name=H}, ""{name=HH}]
  \ar[rr, bend right=60, swap, "G"{name=G}]
  & &\cat D
  \ar[Rightarrow, "\alpha", from=F, to=H]
  \ar[Rightarrow, "\beta", from=HH, to=G]
  \end{tikzcd}
\]
We define the \emph{vertical compostion} of \(\beta\) with \(\alpha\) as the
natural transformation \(\beta \alpha: F \nat G\) --- diagrammatically,
\[
  \begin{tikzcd}
  \cat C \ar[r, bend left=50, "F"{name=F}]
  \ar[r, bend right=50, swap, "G"{name=G}]
  &\cat D
  \ar[Rightarrow, "\beta\alpha", from=F, to=G]
  \end{tikzcd}
\]
\end{definition}

Notice that, given a functor \(\phi: \cat C \to \cat D\) between categories
\(\cat C\) and \(\cat D\), for every category \(\cat A\), there arises a natural
functor
\[
  \phi^{*}: \Fct(\cat A, \cat C) \longrightarrow \Fct(\cat I, \cat D)\text{,}
  \ \text{ mapping }\
  F \longmapsto \phi F.
\]

\begin{lemma}
\label{lem:faithful-pushforward-functor}
If \(\phi\) is a faithful functor (respectively, fully faithful), then so is the
functor \(\phi^{*}\) for any category \(\cat A\).
\end{lemma}

\begin{proof}
Given any two functors \(F, G: \cat A \rightrightarrows \cat C\), let \(\eta: F
\nat G\) be any natural transformation. If we apply \(\phi^{*}\), we get the
following commutative diagram --- for every pair \(x, y \in \cat A\) and every
morphism \(f: x \to y\) in \(\cat A\),
\[
  \begin{tikzcd}
    \phi F x \ar[r, "\phi \eta_x"] \ar[d, swap, "\phi F(f)"]
    &\phi G x \ar[d, "\phi G(f)"] \\
    \phi F y \ar[r, "\phi \eta_y"]
    &\phi G y
  \end{tikzcd}
\]
Since \(\phi\) is faithful (or fully faithful), the mappings \(\eta_x \mapsto
\phi \eta_x\) and \(\eta_y \mapsto \phi \eta_y\) are both injective (or
bijective), thus the natural map
\[
  \Hom_{\Fct(\cat A, \cat C)}(F, G) \longrightarrow
  \Hom_{\Fct(\cat A, \cat D)}(\phi F, \phi G)\text{,}
  \ \text{ mapping }\ \eta \mapsto \phi^{*} \eta,
\]
is ensured to be injective (or bijective).
\end{proof}

We consider now the category consisting of \(\mathcal U\)-small categories and
the morphisms are functors between them, we denote this category by
\(\UCat\). Notice that, given any two \(\mathcal U\)-small categories \(\cat C\)
and \(\cat D\), the collection of functors between them also forms a category
\(\Hom_{\UCat}(\cat C, \cat D) = \Fct(\cat C, \cat D)\) --- this emergent
structure gives birth to the concept of a \(2\)-category.

\begin{definition}[Isomorphism of categories]
\label{def:isomorphism-categories}
Let \(\cat C\) and \(\cat D\) be categories. We say that \(\cat C\) is
isomorphic to \(\cat D\) if there are morphisms \(F: \cat C \to \cat D\) and
\(G: \cat D \to \cat C\) such that \(GF = \Id_{\cat C}\), and \(FG = \Id_{\cat
D}\).
\end{definition}

A weaker and even more important concept it that of an equivalence between
categories.

\begin{definition}[Equivalence of categories]
\label{def:equivalence-categories}
Let \(\cat C\) and \(\cat D\) be categories. We say that a functor \(F: \cat C
\isoto \cat D\) is an \emph{equivalence} of the categories \(\cat C\) and \(\cat
D\) if there exists a functor \(G: \cat D \to \cat C\), and two natural
isomorphisms \(\alpha: G F \isonat \Id_{\cat C}\) and \(\beta: F G \isonat
\Id_{\cat D}\). If this is the case, we say that \(F\) and \(G\) are
\emph{quasi-inverses} of each other.
\end{definition}

\begin{lemma}
\label{lem:quasi-inverse}
Let \(F: \cat C \isoto \cat D\) and \(G: \cat D \isoto \cat C\) be equivalences
of given categories \(\cat C\) and \(\cat D\), and suppose that \(F\) and \(G\)
are quasi-inverses. Then there are natural isomorphisms \(\alpha: G F \isonat
\Id_{\cat C}\) and \(\beta: F G \isonat \Id_{\cat D}\) for which
\[
  F \alpha = \beta F\quad \text{ and } \quad \alpha G = G \beta.
\]
\end{lemma}

\begin{proof}
Let \(x \in \cat C\) be any object. Notice that, since \(\Id_{\cat C} x = x\)
and \(\Id_{\cat D} F x = F x\), it follows that
\[
  \begin{tikzcd}
    G F x \ar[r, "\alpha_x", "\dis"'] \ar[d, swap, "F"] &x \ar[d, "F"] \\
    FG(Fx) \ar[r, "\beta_{Fx}"', "\dis"] &Fx
  \end{tikzcd}
\]
is commutative --- thus indeed \(F \alpha = \beta F\). Now if we let \(y \in
\cat D\) be any other object, since \(\Id_{\cat D} y = y\) and \(\Id_{\cat C} G
y = G y\), we get the following commutative diagram
\[
  \begin{tikzcd}
    G F (G y) \ar[r, "\alpha_{G_y}", "\dis"'] &G y \\
    F G y \ar[u, "G"] \ar[r, "\beta_y"', "\dis"] &y \ar[u, swap, "G"]
  \end{tikzcd}
\]
therefore \(\alpha G = G \beta\) as wanted.
\end{proof}

\begin{lemma}
\label{lem:full-subcategory-functor-correspondence}
Let \(F: \cat C \to \cat D\) be a functor, and \(\cat D_0\) be a full
subcategory of \(\cat D\) such that, for all \(x \in \cat C\), there exists \(y
\in \cat D_0\) and an isomorphism \(Fx \iso y\).

Denote by \(\iota: \cat D_0 \emb \cat D\) the canonical embedding functor. Then
there exists a functor \(F_0: \cat C \to \cat D_0\) and a natural isomorphism
\(\alpha: F \isonat \iota F_0\).  Moreover, \(F_0\) is unique up to
unique isomorphism\footnote{We say that \(F_0\) is \emph{unique up to unique
isomorphism} when, given another functor \(G: \cat C \to \cat D_0\) and natural
isomorphism \(\beta: F \isonat \iota G\), there exists a \emph{unique} natural
isomorphism \(\eta: G \isonat F_0\) for which \(\alpha = \iota \eta
\beta\).}. This can be diagrammatically expressed by the
following quasi-commutative diagram\footnote{A diagram whose nodes are
categories an arrows are morphisms is said to be \emph{quasi-commutative} if it
commutes up to natural isomorphism of functors.}
\[
  \begin{tikzcd}
    \cat C \ar[r, "F"] \ar[dr, swap, dashed, bend right, "F_0"] &\cat D \\
    &\cat D_0 \ar[u, hook, swap, "\iota"]
  \end{tikzcd}
\]
\end{lemma}

\begin{proof}
We first build the functor \(F_0: \cat C \to \cat D_0\):
\begin{itemize}\setlength\itemsep{0em}
\item Let \(x \in \cat C\) be any object. By means of Zorn's Lemma (see
  \cref{lem:zorn}), we choose \(a \in \cat \cat D_0\) for which exists an
  isomorphism \(\phi_x: a \isoto F(x)\) in \(\cat D\) --- and consequently we let
  \(F_0 x \coloneq a\).
\item Given any morphism \(f: x \to y\) in \(\cat C\), we know from construction
  that the objects \(F_0 x \coloneq a\) and \(F_0 y \coloneq b\) are defined so
  that there exists isomorphisms \(\phi_x: a \isoto F x\) and \(\phi_y: b \isoto
  F y\) in the category \(\cat D\). This allows us to define the morphism \(F_0
  f: F_0 x \to F_0 y\) in \(\cat D\) as the mapping \(F f \coloneq \phi_y^{-1} (F
  f) \phi_x\) --- that is, so that the following diagram commutes
  \[
    \begin{tikzcd}
      F_0 x \ar[r, "F_0 f"] \ar[d, "\dis", "\phi_x"']
      &F_0 y \ar[d, "\dis"', "\phi_y"] \\
      F x \ar[r, swap, "F f"] &F y
    \end{tikzcd}
  \]
  Notice that from this definition we find naturally that the composition
  condition is met --- given any other morphism \(g: y \to z\) in \(\cat C\) we
  have that \(F_0(g f) = (F_0 g)(F_0 f)\).
\end{itemize}
This proves the existence of \(F_0\) as a functor. For the isomorphism
\(\alpha\), we can define for each pair \(x, y \in \cat C\) the morphisms
\(\alpha_x \coloneq \phi_x^{-1}\) and \(\alpha_y \coloneq \phi_y^{-1}\), so that
the following diagram commutes
\[
  \begin{tikzcd}
    F x \ar[rr, "\alpha_x", "\dis"'] \ar[d, swap, "F f"]
    & &\iota F_0(x) = a \ar[d, "\iota F_0 (f)"] \\
    F y \ar[rr, "\alpha_y"', "\dis"]
    & &\iota F_0(y) = b
  \end{tikzcd}
\]
For the uniqueness of \(F_0\) up to unique isomorphism, let \(G: \cat C \to \cat
D_0\) be another functor, together with a natural isomorphism \(\beta: F \isonat
\iota G\). Define \(\eta: G \isonat F_0\) so that, for each \(x \in \cat C\), we
have \(\eta_x \coloneq \alpha_x \beta_{x}^{-1}\) --- then, \(\eta\) is clearly an
isomorphism and also uniquely defined, thus the proposition follows.
\end{proof}

\documentclass[12pt, a4paper, reqno, openany]{memoir}
\usepackage[english]{babel}

\usepackage{layout}
\usepackage{afterpage}
\usepackage[
  asymmetric,
  textheight     = 673pt,
  marginparsep   = 7pt,
  footskip       = 27pt,
  hoffset        = 0pt,
  paperwidth     = 597pt,
  textwidth      = 452pt,
  marginparwidth = 101pt,
  voffset        = 0pt,
  paperheight    = 845pt,
]{geometry}

\newcommand{\changegeometry}{\newgeometry{includehead,headheight=89pt}%
  \afterpage{\aftergroup\restoregeometry}%
}

% Math stuff: do not mess with the ordering!
\usepackage{mathtools}
\usepackage{amsthm}
\usepackage{amssymb}
\usepackage{stmaryrd}
\usepackage{tikz-cd}

% Font
\usepackage[no-math]{newpxtext}
\usepackage{newpxmath}
% Set arrow tip to that of newpxmath
\tikzset{>=Straight Barb, commutative diagrams/arrow style=tikz}

% Utilities
\usepackage{enumerate}
\usepackage{todonotes}
\usepackage{graphicx}

% Color
\usepackage{xcolor}
\definecolor{brightmaroon}{rgb}{0.76, 0.13, 0.28}

% References
\usepackage{hyperref}
\hypersetup{
  colorlinks = true,
  allcolors  = brightmaroon,
}
\usepackage[capitalize,nameinlink]{cleveref}
\usepackage[
  backend = biber,
  style   = alphabetic,
]{biblatex}
\addbibresource{src/bibliography.bib}

\linespread{1.05}
\vfuzz=14pt % No more vbox errors all over the place

%%%%%%%%%%%%%%%%%%%%%%%%%%%%%%%%%%%%%%%%%%%%%%%%%%%%%%%%%%%%%%%%%%%%%%%%%%%%%%%
% ** Environments **

\theoremstyle{definition}
\newtheorem{theorem}{Theorem}[section]
\newtheorem{proposition}[theorem]{Proposition}
\newtheorem{lemma}[theorem]{Lemma}
\newtheorem{corollary}[theorem]{Corollary}
\newtheorem{axiom}[theorem]{Axiom}
\newtheorem{definition}[theorem]{Definition}
\newtheorem{remark}[theorem]{Remark}
\newtheorem{example}[theorem]{Example}
\newtheorem{notation}[theorem]{Notation}

%%%%%%%%%%%%%%%%%%%%%%%%%%%%%%%%%%%%%%%%%%%%%%%%%%%%%%%%%%%%%%%%%%%%%%%%%%%%%%%
% ** symbols **

\renewcommand{\qedsymbol}{\(\natural\)}
\renewcommand{\leq}{\leqslant}
\renewcommand{\geq}{\geqslant}
\renewcommand{\setminus}{\smallsetminus}

% ':' for maps and '\colon' for relations on collections
\DeclareMathSymbol{:}{\mathpunct}{operators}{"3A}
\let\colon\relax
\DeclareMathSymbol{\colon}{\mathrel}{operators}{"3A}

%%%%%%%%%%%%%%%%%%%%%%%%%%%%%%%%%%%%%%%%%%%%%%%%%%%%%%%%%%%%%%%%%%%%%%%%%%%%%%%
% ** arrows **

% Alias for Rightarrow
\newcommand{\To}{\Rightarrow}

% Monomorphism arrow
\newcommand{\mono}{\rightarrowtail}

% Epimorphism arrow
\newcommand{\epi}{\twoheadrightarrow}

% Unique morphism
\newcommand{\unique}{\dashrightarrow}
\newcommand{\xdashrightarrow}[2][]{\ext@arrow 0359\rightarrowfill@@{#1}{#2}}

% Isomorphism symbol
\newcommand{\iso}{\simeq}

% Isomorphism arrow
\newcommand{\isoto}{\xrightarrow{\raisebox{-.6ex}[0ex][0ex]{\(\sim\)}}}

% diagram isomorphism
\newcommand{\dis}{\iso}

\newcommand{\isounique}{%
  \xdashrightarrow{\raisebox{-.6ex}[0ex][0ex]{\(\sim\)}}
}%

% Natural transformation arrow
\newcommand{\nat}{\Rightarrow}

% Natural isomorphism
\newcommand{\isonat}{\xRightarrow{\raisebox{-.8ex}[0ex][0ex]{\(\sim\)}}}

% Embedding arrow
\newcommand{\emb}{\hookrightarrow}

% Implication
\renewcommand{\implies}{\Rightarrow}

% Limits
\renewcommand{\projlim}{\varprojlim}
\newcommand{\colim}{\varinjlim}


%%%%%%%%%%%%%%%%%%%%%%%%%%%%%%%%%%%%%%%%%%%%%%%%%%%%%%%%%%%%%%%%%%%%%%%%%%%%%%%
% ** Common collections **

\newcommand{\Z}{\mathbf{Z}}
\newcommand{\N}{\mathbf{N}}
\newcommand{\Q}{\mathbf{Q}}
\newcommand{\CC}{\mathbf{C}}
\newcommand{\R}{\mathbf{R}}

\renewcommand{\emptyset}{\varnothing}

\newcommand{\Uhs}{\mathbf{H}}  % Upper half space
\newcommand{\Proj}{\mathbf{P}} % Projective space

%%%%%%%%%%%%%%%%%%%%%%%%%%%%%%%%%%%%%%%%%%%%%%%%%%%%%%%%%%%%%%%%%%%%%%%%%%%%%%%
% ** Categories **

% Font for categories
\newcommand{\cat}{\mathcal}

% Opposite category
\newcommand{\op}{\mathrm{op}}

% Common categories
\newcommand{\Set}{{\textbf{Set}}}          % Sets
\newcommand{\FinSet}{{\textbf{FinSet}}}    % Finite sets
\newcommand{\pSet}{{\textbf{pSet}}}        % Pointed sets
\newcommand{\sSet}{{\textbf{sSet}}}        % Simplicial sets
\newcommand{\Vect}{{\textbf{Vect}}}        % Vector spaces
\newcommand{\FinVect}{{\textbf{FinVect}}}  % Finite vector spaces
\newcommand{\TVect}{{\textbf{TVect}}}      % Topological vector spaces
\newcommand{\Ban}{{\textbf{Ban}}}          % Banach spaces
\newcommand{\Man}{{\textbf{Man}}}          % Manifolds
\let\Top\relax
\newcommand{\Top}{{\textbf{Top}}}          % Topological spaces
\newcommand{\HoTop}{{\textbf{Ho}(\textbf{Top})}} % Homotopy space
\newcommand{\Grp}{{\textbf{Grp}}}          % Groups
\newcommand{\Ab}{{\textbf{Ab}}}            % Abelian groups
\newcommand{\Graph}{{\textbf{Graph}}}      % Graphs
\newcommand{\SimpGraph}{{\textbf{sGraph}}} % Simple graphs
\newcommand{\Rng}{{\textbf{Ring}}}         % Rings
\newcommand{\Cat}{{\textbf{Cat}}}          % Small categories
\newcommand{\CAT}{{\textbf{CAT}}}          % Big categories
\newcommand{\UCat}{{\mathcal{U}\text{-}\textbf{Cat}}} % U-Categories

% Operators
\DeclareMathOperator{\Hom}{Mor}
\DeclareMathOperator{\Fct}{Fct}
\DeclareMathOperator{\Obj}{Obj}
\DeclareMathOperator{\Mor}{Mor}
\DeclareMathOperator{\End}{End}
\DeclareMathOperator{\Aut}{Aut}
\DeclareMathOperator{\Id}{id}
\DeclareMathOperator{\im}{im}     % Image
\DeclareMathOperator{\dom}{dom}   % Domain
\DeclareMathOperator{\codom}{cod} % Codomain
\DeclareMathOperator{\supp}{supp} % Support

% Yoneda embedding
\newcommand{\yo}{\text{\usefont{U}{min}{m}{n}\symbol{'210}}}

\DeclareFontFamily{U}{min}{}
\DeclareFontShape{U}{min}{m}{n}{<-> udmj30}{}

%%%%%%%%%%%%%%%%%%%%%%%%%%%%%%%%%%%%%%%%%%%%%%%%%%%%%%%%%%%%%%%%%%%%%%%%%%%%%%%
% ** algebra **
\DeclareMathOperator{\rank}{rank}
\DeclareMathOperator{\coker}{coker}
\DeclareMathOperator{\Tr}{tr}   % Trace
\DeclareMathOperator{\Sym}{Sym} % Symmetric space
\DeclareMathOperator{\Alt}{Alt} % Alternating space
\DeclareMathOperator{\Char}{char}
\DeclareMathOperator{\Span}{span}
\DeclareMathOperator{\Inn}{Inn} % Inner automorphisms
\newcommand{\lie}[1]{\mathfrak{#1}} % Font for Lie structures

\DeclareMathOperator{\eval}{eval}
\DeclareMathOperator{\sign}{sign}

% Matrices
\DeclareMathOperator{\Mat}{Mat}
\DeclareMathOperator{\GL}{GL}
\DeclareMathOperator{\SL}{SL}
\DeclareMathOperator{\PSL}{PSL}
\DeclareMathOperator{\SO}{SO}
\DeclareMathOperator{\SU}{SU}
\DeclareMathOperator{\Unit}{U}
\DeclareMathOperator{\Orth}{O}

% Symbol for the group of units --- for instance, the group of units of a ring
% \(R\) will be denoted by \(R^{\unit}\).
\newcommand{\unit}{*}

% Ring ideals font
\newcommand{\ideal}[1]{\mathfrak{#1}}

%%%%%%%%%%%%%%%%%%%%%%%%%%%%%%%%%%%%%%%%%%%%%%%%%%%%%%%%%%%%%%%%%%%%%%%%%%%%%%%
% ** Topology **

% attaching spaces
\newcommand{\att}{\sqcup}       % Disjoint union
\newcommand{\bigatt}{\bigsqcup} % Indexed disjoint union

% Operators
\DeclareMathOperator{\Cone}{C}
\DeclareMathOperator{\Int}{Int}
\DeclareMathOperator{\Ext}{Ext}

%%%%%%%%%%%%%%%%%%%%%%%%%%%%%%%%%%%%%%%%%%%%%%%%%%%%%%%%%%%%%%%%%%%%%%%%%%%%%%%
% ** Differentiable structures **

% Norm
\DeclarePairedDelimiter{\norm}{\lVert}{\rVert}

\newcommand{\diff}{\mathrm{d}} % Differential 'd' symbol
\newcommand{\Diff}{\mathrm{D}} % Differential 'D' symbol

% Operators
\DeclareMathOperator{\Vol}{vol}   % Volume
\DeclareMathOperator{\Mesh}{mesh} % Mesh
\DeclareMathOperator{\grad}{grad} % Gradient
\DeclareMathOperator{\Hess}{Hess} % Hessian
\DeclareMathOperator{\Jac}{Jac}   % Jacobian
\DeclareMathOperator{\Curl}{Curl} % Curl

%%%%%%%%%%%%%%%%%%%%%%%%%%%%%%%%%%%%%%%%%%%%%%%%%%%%%%%%%%%%%%%%%%%%%%%%%%%%%%%
% ** Graphs **

\DeclareMathOperator{\Circ}{circ}
\DeclareMathOperator{\diam}{diam}
\newcommand{\emptygraph}{\varnothing}
\DeclarePairedDelimiterX{\size}[1]{\lVert}{\rVert}{#1}

%%%%%%%%%%%%%%%%%%%%%%%%%%%%%%%%%%%%%%%%%%%%%%%%%%%%%%%%%%%%%%%%%%%%%%%%%%%%%%%
% ** MACROS END HERE **
%%%%%%%%%%%%%%%%%%%%%%%%%%%%%%%%%%%%%%%%%%%%%%%%%%%%%%%%%%%%%%%%%%%%%%%%%%%%%%%

\author{Luiz G. Mugnaini A.}
\date{Last modification: \today}
\title{Deep Dive}

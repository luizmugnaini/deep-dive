\subsection{Gauss Sum and The Quadratic Reciprocity}

Throughout this subsection we'll define \(\zeta := e^{\frac{2\pi i}{p}}\), the
\(p\)th root of unity, where \(p > 2\) is prime.

\begin{definition}
    We define \[g_a := \sum_{0 \leqslant t \leqslant p-1} \left( \frac{t}{p}
        \right) \zeta^{at}\] as the \emph{quadratic Gau{\ss} sum}.
\end{definition}

\begin{proposition}
    The equality \(g_a = \left( \frac{a}{p} \right) g\) holds, where \(g:=g_1\).
\end{proposition}

\begin{proof}
    For the case \(a \equiv 0 \pmod{p}\) we have that \(\zeta^{at} = 1^t = 1\)
    and therefore \(g_a = \sum_{1 \leqslant t \leqslant p-1} \left( \frac{t}{p}
    \right) \zeta^{at} = \sum_{1 \leqslant t \leqslant p-1} \left( \frac{t}{p}
    \right) = 0\) and thus the equality holds.

    Lets consider the case \(a \not\equiv 0 \pmod{p}\), in this case we can
    analyse the product
    \[
        \left( \frac{a}{p} \right) g_a = \left( \frac{a}{p} \right) \sum_{1
            \leqslant t \leqslant p-1} \left( \frac{t}{p} \right) \zeta^{at} =
        \sum_{1 \leqslant t \leqslant p-1} \left( \frac{at}{p} \right) \zeta^{at}
    \]
    but notice that \(\{\overline{0}, \overline{1}, \dots, \overline{p-1}\} =
    \{\overline{0}, \overline{a}, \dots, \overline{(p-1)a}\}\) therefore we have
    that
    \[
        \left( \frac{a}{p} \right) g_a = \sum_{1 \leqslant x \leqslant p-1} \left(
        \frac{x}{p}\right) \zeta^x = g
    \]
    then, since \(\left( \frac{a}{p} \right) = \pm 1\), we can simply multiply
    both sides by \(\left( \frac{a}{p} \right)\) to get the desired equation.
\end{proof}

\begin{proposition}
    \(g^2 = (-1)^{\frac{p-1}{2}} p\).
\end{proposition}

\begin{proof}
    Consider the two following convergent ideas:
    \begin{enumerate}[i.]
        \item If \(a \not\equiv 0 \pmod{p}\) then we have that
              \[
                  g_{a}g_{-a} = \left( \frac{a}{p} \right) g \left( \frac{-a}{p}
                  \right) g
                  = \left( \frac{-1}{p} \right) \left( \frac{a^2}{p} \right) g^2
                  = \left( \frac{-1}{p} \right)g^2
              \]
              Therefore, if we proceed by taking the sum, we end up getting:
              \[
                  \sum_{1 \leqslant a \leqslant p-1} g_a g_{-a} = \sum_{1 \leqslant a
                      \leqslant p-1} \left( \frac{-1}{p} \right) g^2 = (p-1) \left(
                  \frac{-1}{p} \right) g^2.
              \]
        \item Now we take another approach:
              \[
                  g_a g_{-a} = \sum_{1 \leqslant x \leqslant p-1} \left( \frac{x}{p}
                  \right) \zeta^{xa} \sum_{1 \leqslant y \leqslant p-1} \left(
                  \frac{y}{p} \right) \zeta^{-ya}
                  =
                  \sum_{\substack{1 \leqslant x \leqslant p-1\\ 1 \leqslant y
                          \leqslant p-1}} \left( \frac{x}{p} \right)  \left( \frac{y}{p}
                  \right) \zeta^{a(x-y)}
              \]
              Now, denoting \(\delta(x, y) = 1\) if  \(x \equiv y \pmod{p}\) and
              \(\delta(x,y)= 0\) if  \(x \not\equiv y\pmod{p}\) then we get
              \begin{align*}
                  \sum_{1 \leqslant a \leqslant p-1} g_a g_{-a}
                   & =
                  \sum_{1 \leqslant a \leqslant p-1} \sum_{\substack{1 \leqslant x
                  \leqslant p-1                                                   \\ 1 \leqslant y \leqslant p-1}} \left( \frac{x}{p}
                  \right) \left( \frac{y}{p} \right) \zeta^{a(x-y)}               \\
                   & = \sum_{\substack{1 \leqslant x \leqslant p-1                \\ 1 \leqslant y
                          \leqslant p-1}} \left( \frac{x}{p} \right) \left( \frac{y}{p}
                  \right) (\delta(x, y)p)                                         \\
                   & = \sum_{1 \leqslant x = y \leqslant p-1} \left( \frac{xy}{p}
                  \right) p                                                       \\
                   & = (p-1)p
              \end{align*}
              Since the two sums must be equal to each other, we conclude that
              \[
                  \left( \frac{-1}{p} \right) g^2 = p \Rightarrow g^2 = \left(
                  \frac{-1}{p} \right) p = (-1)^{\frac{p-1}{2}} p.
              \]
    \end{enumerate}
\end{proof}


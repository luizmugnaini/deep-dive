\section{Torsion \& Torsion-Free Groups}

\subsection{Torsion}

\begin{definition}[Torsion subgroup]
    \label{def:torsion-subgroup}
    Given an abelian group \(G\), we define the \emph{torsion subgroup} of \(G\) to be
    \[
        \torsion G = \{g \in G \colon g \text{ has finite order}\}.
    \]
    The group \(G\) is said to be of \emph{torsion} if \(\torsion G = G\), while
    \(G\) is \emph{torsion-free} if \(\torsion G = 0\).
\end{definition}

\begin{corollary}
    \label{cor:torsion-subgrp-first-properties}
    Let \(G\) and \(H\) be abelian groups, then one has:
    \begin{enumerate}[(a)]\setlength\itemsep{0em}
        \item The quotient \(G/\torsion G\) is torsion-free.
        \item If \(G \iso H\), then \(\torsion G \iso \torsion H\) and
              \(G/\torsion G \iso H/\torsion H\).
    \end{enumerate}
\end{corollary}

\begin{proof}
    \begin{enumerate}[(a)]\setlength\itemsep{0em}
        \item Let \(x + \torsion G \in \torsion(G/\torsion G)\) be any element, then
              there exists \(n \in \Z_{>0}\) such that
              \[
                  \torsion G = n(x + \torsion G) = n x + \torsion G,
              \]
              therefore \(n x \in \torsion G\). The latter conclusion implies in the
              existence of \(m \in \Z_{>0}\) such that \(0 = m (n x) = (m n) x\). But then
              \(x\) is an element of finite order, thus \(x \in \torsion G\) and hence
              \(\torsion(G/\torsion G) = \torsion G\), proving the statement.

        \item Let \(\phi: G \isoto H\) be an isomorphism. If \(x \in \torsion G\) then
              there exists \(n \in \Z_{>0}\) such that \(n x = 0\) and hence \(\phi(n x) = n
              \phi x = 0_H\)---thus \(\phi x \in \torsion H\). Furthermore, if \(y \in
              \torsion H\) with \(m y = 0\) for some \(m \in \Z_{> 0}\), and \(\phi x = y\),
              then
              \[
                  0_H = m y = m \phi x = \phi(m x)
              \]
              but since \(\phi\) is injective then \(m x = 0_G\)---thus
              \(x \in \torsion G\). With this we have shown that
              \[
                  \phi(\torsion G) = \torsion H.
              \]
              Now by \cref{prop:iso-mod-to-iso-quotient-of-mod} we know that the induced map
              \(\phi_{*}: G/\torsion G \isoto H/\torsion H\) is an isomorphism. Moreover,
              the restriction \(\phi|_{\torsion G}: \torsion G \isoto \torsion H\) is also
              an isomorphism, finishing the proof of item (b).
    \end{enumerate}
\end{proof}

\begin{theorem}
    \label{thm:finitely-generated-torsion-free-abelian-grp-is-free}
    Every finitely generated torsion-free abelian group is free.
\end{theorem}

\begin{proof}
    Let \(G\) be a finitely generated torsion-free abelian group with \(n\)
    generators. We proceed by induction on \(n\). For the case \(n = 1\) we know
    that \(G\) is cyclic, moreover, since \(G\) is also torsion-free then
    \(G \iso \Z\). Assume, for the inductive hypothesis, that the statement is true
    for some \(n > 1\).

    Suppose now that \(G = \langle g_1, \dots, g_{n+1} \rangle\) is abelian and
    torsion-free, and define a non-zero subgroup
    \[
        U \coloneq
        \{x \in G \colon \exists m \in \Z \setminus 0
        \text{ with } m x \in \langle g_{n+1} \rangle\}.
    \]
    Notice that the quotient \(G/U\) is torsion-free, since: if
    \(g + U \in \torsion(G/U)\), then there exists a non-zero \(q \in \Z\) such that
    \(U = q(g + U) = q g + U\), thus \(q g \in U\). Therefore there exists
    \(q' \in \Z\) such that \(g_{n+1} = q'(q g) = (q' q) g\), then
    \(g \in U\)---thus \(\torsion(G/U) = U\). By the hypothesis of induction, since
    \(G/U = \langle g_1 + U, \dots, g_n + U \rangle\) then \(G/U\) is a free
    group. From freeness we know that the short exact sequence
    \(0 \to U \emb G \epi G/U \to 0\) is split and therefore
    \[
        G \iso U \oplus \frac{G}{U}.
    \]

    Now we shall prove that \(U\) is free. Define a map \(\phi: U \to \Q\) as
    follows: if \(x \in U\) let \(r \in \Z\) be a non-zero
    integer such that there exists \(a \in \Z\) for which \(r x = a g_{n+1}\)---then
    define \(\phi x \coloneq r/a\). To see that \(\phi\) is well defined, suppose we
    also have \(s x = b g_{n + 1}\), then \(s a g_{n+1} = r b g_{n+1}\)---however,
    since \(g_{n+1}\) has infinite order, it must be the case that \(s a = r b\)
    thus \(r/a = s/b\). Notice that if \(y \in \ker \phi\) then it must be the case
    that \(y = 0\), thus \(\phi\) is injective and therefore \(U \iso \im
    \phi\). Since \(U\) is a direct summand of the finitely generated group \(G\),
    then \(U\) is finitely generated (and so is \(\im \phi\)).

    We'll show that \(\im \phi\) is a cyclic subgroup of \(\Q\). For that, we prove
    a stronger statement: if
    \(D = \langle b_1/c_1, \dots, b_m/c_m \rangle \subseteq \Q\) is a finitely
    generated subgroup, then \(D\) is cyclic. Define
    \(c \coloneq \prod_{j=1}^m c_j\), and a map \(\psi: D \to \Z\) given by
    \(d \mapsto c d\). Clearly, \(\Q\) is torsion-free and so is \(D\), hence
    \(\psi\) is an injective morphism of groups. Therefore there is an isomorphism
    \(D \iso \im \psi \subseteq \Z\)---then \(D\) has the form of an ideal of
    \(\Z\), which is principal, thus \(D\) is cyclic. It follows that \(D \iso \Z\),
    hence either \(U \iso \im \phi \iso \Z\) or \(U = 0\). This shows that \(U\) is
    free and hence so is \(G\).
\end{proof}

\begin{lemma}
    \label{lem:subgroup-fg-free-ab-grp-is-free}
    Every \emph{subgroup} \(S\) of a finitely generated free abelian group \(F\) is
    \emph{free}, and \(\rank S \leq \rank F\)\footnote{In fact, even if \(F\) isn't
        finitely generated, the last rank property holds}.
\end{lemma}

\begin{proof}
    Let \(F\) be generated by \(n\) elements---we do induction on \(n\). If
    \(n = 1\) then \(F\) is a cyclic free group, and since \(F\) is torsion-free,
    \(F \iso \Z\). Therefore, if \(S \subseteq F\) is a subgroup, either \(S\) is
    zero or an ideal of \(\Z\), in which case one has \(S \iso \Z\). For the
    hypothesis of induction, assume the proposition holds for some \(n > 1\).

    Suppose now that \(F = \langle x_1, \dots, x_{n+1} \rangle\) and consider the
    short exact sequence
    \[
        \begin{tikzcd}
            0 \ar[r]
            &S \cap \langle x_1, \dots, x_n \rangle \ar[r, hook]
            &S \ar[r, two heads]
            &\frac{S}{S \cap \langle x_1, \dots, x_n \rangle} \ar[r]
            &0
        \end{tikzcd}
    \]
    Using the inductive hypothesis, \(S \cap \langle x_1, \dots, x_n \rangle\) is a
    free group with \(\rank S \leq n\). By the second isomorphism theorem (see
    \cref{cor:grp-intersection-coset-isomorphism}) one has the isomorphism
    \[
        \frac{S}{S \cap \langle x_1, \dots, x_n \rangle}
        \iso
        \frac{S + \langle x_1, \dots, x_n \rangle}{\langle x_1, \dots, x_n \rangle}.
    \]
    Notice that \(S/(S \cap \langle x_1, \dots, x_n \rangle)\) is a subgroup of the
    cyclic group \(\langle x_{n+1} + \langle x_1, \dots, x_n \rangle \rangle\), thus
    generated by a single element. Therefore, by means of
    \cref{prop:short-exact-finitely-generated} we conclude that \(S\) is finitely
    generated with \(\rank S \leq n + 1\).
\end{proof}

\begin{corollary}
    \label{cor:subgroup-fg-is-fg}
    If an abelian group \(G\) is generated by \(n\) elements, every subgroup
    \(S \subseteq G\) can be generated by at most \(n\) elements.
\end{corollary}

\begin{proof}
    Let \(G = \langle g_1, \dots, g_n \rangle\), and consider a free group \(F\)
    with basis \(\{x_1, \dots, x_n\}\). Define a surjective group morphism
    \(\phi: F \to G\) mapping \(x_j \mapsto g_j\) for each \(1 \leq j \leq n\). Now
    if \(S \subseteq \phi F'\) for a subgroup \(F' \subseteq F\), then by the
    isomorphism theorem we have \(S \iso F'/\ker \phi\). From
    \cref{lem:subgroup-fg-free-ab-grp-is-free} we know that \(F'\) is free and has
    rank at most \(n\), therefore \(S\) shall be generated by at most \(n\)
    elements.
\end{proof}

\begin{corollary}
    \label{cor:quotient-by-torsion-is-free}
    If \(G\) is a finitely generated abelian group, then the quotient
    \(G/\torsion G\) is a free abelian group of finite rank.
\end{corollary}

\begin{proof}
    Torsion-freeness of \(G/\torsion G\) is already known from
    \cref{cor:torsion-subgrp-first-properties}, moreover by
    \cref{cor:subgroup-fg-is-fg} we find that \(G/\torsion G\) is finitely
    generated. Therefore by
    \cref{thm:finitely-generated-torsion-free-abelian-grp-is-free} we conclude that
    \(G/\torsion G\) is free.
\end{proof}

\begin{corollary}
    \label{cor:fg-ab-grp-direct-sum-torsion-and-free-grp}
    Given any finitely generated abelian group \(G\) there exists a finitely
    generated free abelian group \(F\) for which
    \[
        G = \torsion G \oplus F.
    \]
\end{corollary}

\begin{proof}
    Recall that \(G \iso H \oplus (G/H)\) for a normal subgroup \(H \subseteq G\),
    therefore since \(G/\torsion G\) is free (see
    \cref{cor:quotient-by-torsion-is-free}) then
    \(G \iso \torsion G \oplus (G/\torsion G)\) proves the statement.
\end{proof}

\begin{corollary}
    \label{cor:iso-fg-grps-iff-torsions-and-rank}
    Given a pair of finitely generated abelian groups \(G\) and \(H\), one has
    \(G \iso H\) if and only if \(\torsion G \iso \torsion H\) and
    \(\rank(G/\torsion G) = \rank(H/\torsion H)\).
\end{corollary}

\begin{proof}
    From \cref{cor:torsion-subgrp-first-properties} we know that if \(G \iso H\)
    then \(\torsion G \iso \torsion H\) and \(G/\torsion G \iso H/\torsion
    H\). Moreover, since both quotients are free (see
    \cref{cor:quotient-by-torsion-is-free}) then their rank must be the same.

    For the converse, suppose the latter hypothesis and notice that the rank
    hypothesis implies in \(G/\torsion G \iso H/\torsion H\). Using
    \cref{cor:fg-ab-grp-direct-sum-torsion-and-free-grp} we find that
    \(G \iso \torsion G \oplus (G/\torsion G)\) and
    \(H \iso \torsion H \oplus (H/\torsion H)\). Therefore we may rightly conclude
    that \(G \iso H\).
\end{proof}

\subsection{Basis Theorem}

\begin{definition}[\(p\)-group]
    \label{def:p-group}
    Let \(G\) be a group and \(p\) be a prime number. We say that \(G\) is a
    \emph{\(p\)-group} if for every \(g \in G\) there exists \(n \in \Z\) such that
    \(|g| = p^n\) is the order of \(g\) in \(G\).

    Analogously, if \(A\) is an \emph{abelian} group, we shall say that \(A\) is
    \emph{\(p\)-primary} if for every \(a \in A\) there exists \(k \geq 1\) such
    that \(p^k a = 0\). We shall define the notation
    \[
        A_p = \{a \in A \colon p^k a = 0 \text{ for some } k \geq 1\}
    \]
    for the \(p\)-primary subgroup, or component, of \(A\).
\end{definition}

\begin{theorem}[Primary decomposition]
    \label{thm:primary-decomposition-torsion-abelian-grp}
    Let \(G\) and \(H\) be torsion abelian groups, then:
    \begin{enumerate}[(a)]\setlength\itemsep{0em}
        \item The group \(G\) is the direct sum of its \(p\)-primary components:
              \[
                  G = \bigoplus_{p \text{ prime}} G_p.
              \]

        \item The groups \(G\) and \(H\) are isomorphic if and only if \(G_p \iso H_p\)
              for each prime \(p\).
    \end{enumerate}
\end{theorem}

\begin{proof}
    \begin{enumerate}[(a)]\setlength\itemsep{0em}
        \item Let \(x \in G\) be any element, and assume \(|x| \coloneq d > 1\). Let
              \(d = \prod_{j=1}^t p_j^{k_j}\) be the prime decomposition of \(d\), and define
              for each \(1 \leq j \leq t\) the integer \(r_j \coloneq d/p_j^{k_j}\). Since
              \(d x = 0\), then \(r_j x \in G_{p_j}\) for each \(j\). Since the collection
              \(\{r_1, \dots, r_t\}\) share no common divisors other than \(1\), then
              \(\gcd_{1 \leq j \leq t} r_j = 1\) and hence one can find a collection
              \((a_j)_{j=1}^t\) of integers such that \(\sum_{j=1}^t a_j r_j = 1\). In
              particular, one has \(x = \sum_j a_j (r_j x) \in \sum_j G_{p_j}\). To prove
              that this sum of abelian groups is direct, we must show that for each
              \(1 \leq j \leq t\) the group \(H_j \coloneq \sum_{i \neq j} G_{p_i}\) has a
              trivial intersection with \(G_{p_j}\).

              To obtain this intersection, let \(y \in G_{p_j} \cap H_j\) be any element and
              take \(s \geq 0\) such that \(p_j^s y = 0\), and a collection
              \((s_i)_{i \neq j}\) of non-negative integers such that
              \(u x \coloneq \prod_{i \neq j} p_i^{s_i} x = 0\). Since \(p_j\) does not
              appear in the prime decomposition of \(u\), it follows that
              \(\gcd(p_j^s, u) = 1\). Therefore we may find \(a, b \in \Z\) such that
              \(a p_j^s + b u = 1\). Hence one has
              \[
                  x = (a p_j^s + b u) x = a p_j^s x + b u x = 0,
              \]
              therefore \(G_{p_j} \cap H_j = 0\) as wanted.

        \item Suppose that there exists an isomorphism \(\phi: G \isoto H\). Notice that
              if \(x \in G_p\) then \(p^k x = 0\) for some \(k \geq 0\) and hence
              \(0 = \phi(p^k x) = p^k \phi x\) shows that \(\phi x \in H_p\)---thus \(\phi
              G_p \subseteq H_p\). Moreover, since \(\phi\) is an isomorphism, given any
              element \(y \in H_p\), let \(\phi x = y\) and \(\ell \geq 0\) such that
              \(p^{\ell} y = 0\), then we have
              \[
                  0 = p^{\ell} y = p^{\ell} \phi x = \phi(p^{\ell} x)
              \]
              hence \(p^{\ell} x \in \ker \phi = 0\) since \(\phi\) is injective and thus
              \(x \in G_p\). This shows that \(\phi G_p = H_p\), therefore the induced map
              \(\phi|_{G_p}: G_p \isoto H_p\) is an isomorphism.

              For the contrary, let \((f_p: G_p \isoto H_p)_{p \text{ prime}}\) be a
              collection of isomorphisms. From the coproduct universal property we know that
              there exists a uniquely iduced group morphism \(f: G \to H\) such that \(f_p =
              f \iota_p\) for each prime \(p\)---where \(\iota_p: G_p \emb G\) is the
              canonical inclusion. Since each \(f_p\) is an isomorphism, so is \(f\) and
              hence \(G \iso H\).
    \end{enumerate}
\end{proof}

\begin{lemma}
    \label{lem:grp-independence}
    Let \(G = \langle x_1, \dots, x_n \rangle\) be a finitely generated abelian
    group. Then \(G = \bigoplus_{j=1}^n \langle x_j \rangle\) if and only if
    \(\sum_{j=1}^n a_j x_j = 0\) implies in \(a_j x_j = 0\) for all
    \(1 \leq j \leq n\), where \(a_j \in \Z\).
\end{lemma}

\begin{proof}
    We shall make use of \cref{prop:internal-sum-modules}. Let \(1 \leq j \leq n\)
    be any index, and consider any element
    \(g \in \langle x_j \rangle \cap \langle x_1, \dots, \widehat x_j, \dots, x_n
    \rangle\). Then there exists \(a \in \Z\) and a collection \((a_i)_{i \neq j}\)
    of integers such that
    \[
        a x_j = g = \sum_{i \neq j} a_i x_i,
    \]
    so that \(\big(\sum_{i \neq j} a_i x_i \big) - a x_j = 0\). From hypothesis this
    last equation implies in \(a_i x_i = 0\) for each \(i \neq j\) and
    \(g = a x_j = 0\), thus
    \[
        \langle x_j \rangle
        \cap \langle x_1, \dots, \widehat x_j, \dots, x_n \rangle
        = 0.
    \]

    For the converse, assume \(G = \bigoplus_{j=1}^n \langle x_j \rangle\) and
    suppose \(\sum_j b_j x_j = 0\) with \(b_j \in \Z\). Again from
    \cref{prop:internal-sum-modules} the uniqueness of each \(b_j x_j\) shows that
    it must be the case that \(b_j x_j = 0\).
\end{proof}

\begin{proposition}
    \label{prop:torsion-grp-iso-iff-primary-iso}
    Two torsion abelian groups \(G\) and \(H\) are isomorphic if and only if
    \(G_p \iso H_p\) for each prime \(p\).
\end{proposition}

\begin{proof}
    If \(G \iso H\) then certainly the restriction to each \(G_p\) is an isomorphism
    \(G_p \iso H_p\). For the converse, if there exists a collection of isomorphisms
    \(G_p \iso H_p\) for each prime \(p\), then by the universap property of
    coproducts there exists a unique induced isomorphism
    \(\bigoplus_p G_p \iso \bigoplus_p H_p\)---proving that \(G\) and \(H\) are
    isomorphic.
\end{proof}

\begin{definition}
    \label{def:pure-subgroup}
    Let \(G\) be a \(p\)-primary abelian group for a given prime \(p\). A subgroup
    \(P \subseteq G\) is said to be \emph{pure} if for all \(n \geq 0\) one has
    \(P \cap p^n G = p^n P\).

    In general, given any abelian group \(H\), a \emph{pure subgroup}
    \(S \subseteq H\) is a subgroup such that \(S \cap m H = m S\) for each
    \(m \in \Z\).
\end{definition}

\begin{lemma}
    \label{lem:finite-p-primary-has-pure-cyclic-subgrp}
    Let \(G\) be a \emph{finite} \(p\)-primary abelian group, then \(G\) has a
    \emph{non-zero pure cyclic subgroup}.
\end{lemma}

\begin{proof}
    Since \(G\) is finite, let \(y \in G\) be the element for the maximal order
    \(|y| \coloneq p^{\ell}\), and define \(S \coloneq \langle y \rangle\). Consider
    any non-zero element \(s \coloneq m p^t y \in S\) with \(t \geq 0\) and
    \(p \nmid m\). If there exists \(g \in G\) and \(n \geq 0\) with \(s = p^n g\),
    then in particular it must be the case that \(n < \ell\) since \(p^{\ell}\) is
    the biggest order in the group and \(s\) is non-zero.

    Suppose, for the sake of contradiction, that \(t < n\), then
    \[
        p^\ell g = (p^{\ell - n} p^n) g = p^{\ell - n} s = p^{\ell - n} (m p^t y) = m
        p^{\ell - n + t} y.
    \]
    Notice however that \(p \nmid m\) and \(\ell - n + t < \ell\), therefore
    contradicting the fact that \(p^{\ell} g = 0\). Thus we may assume that
    \(t \geq n\). In such case, defining \(s' \coloneq m p^{t-n} y \in S\) we obtain
    \[
        p^n s' = m p^t y = s,
    \]
    which is a solution for \(s\) in the equation \(p^n S = S \cap p^n G\).
\end{proof}

\begin{definition}
    \label{def:finite-p-primary-quotiented-is-Fp-vector-space}
    Given a finite \(p\)-primary abelian group \(G\), the quotient \(G/pG\) is an
    \emph{\(\FF_p\)-vector space} with dimension
    \[
        \delta G \coloneq \dim_{\FF_p}(G/pG).
    \]
\end{definition}

Also, if \(G\) and \(H\) are both finite \(p\)-primary groups, then since
\[
    \frac{G \oplus H}{p(G \oplus H)} = \frac{G \oplus H}{p G \oplus p H}
    \iso \frac{G}{p G} \oplus \frac{H}{p H}
\]
it follows that \(\delta(G \oplus H) = \delta G + \delta H\)---thus \(\delta\)
is additive over direct sums.

\begin{corollary}
    \label{cor:finite-p-primary-is-trivial-iff-delta-is-zero}
    If \(G\) is a finite \(p\)-primary abelian group, then \(G = 0\) if and only if
    \(\delta G = 0\).
\end{corollary}

\begin{definition}
    \label{def:prufer-grp}
    We define the \emph{Pr\"{u}fer group} \(\Z(p^{\infty})\) to be the group
    \[
        \Z(p^{\infty}) \coloneq \langle e^{2 \pi \img / p^\ell} \colon \ell \geq 0 \rangle.
    \]
\end{definition}

\begin{lemma}
    \label{lem:G=pG-for-p-primary-iff-zero-or-infinite}
    Let \(G\) be a \(p\)-primary abelian group. If \(G = p G\) then either \(G = 0\)
    or \(G\) is infinite.
\end{lemma}

\begin{proof}
    Suppose \(G\) is non-zero and satisfies \(G = p G\). Let \(g \in G\) be a
    non-zero element with \(p^k g = 0\) for a minimal \(k \geq 1\), then there
    exists \(h \in G\) with \(h \neq g\) for which \(g = p h\)---then
    \(p^{k+1} h = 0\). This shows that the order of the elements of \(G\) isn't
    bounded, and thus \(G\) must contain an infinite number of elements.
\end{proof}

\begin{lemma}
    \label{lem:finite-p-primary-is-cyclic-iff-delta-is-one}
    A finite \(p\)-primary abelian group \(G\) is a \emph{non-zero cyclic group} if
    and only if \(\delta G = 1\).
\end{lemma}

\begin{proof}
    Suppose \(G\) is a non-zero cyclic group, then \(G/pG\) will be a non-zero
    cyclic group with \(\dim_{\FF_p}(G/pG) = 1\), then \(G/pG \iso \FF_p\).

    For the converse, assume that \(\delta G = 1\), so that \(G/pG \iso \Z/p\Z\) is
    cyclic. Let \(g + p G\) be the generator of \(G/pG\) and assume, for the sake of
    contradiction, that \(G\) is \emph{not} generated by \(g\), that is,
    \(\langle g \rangle \subsetneq G\) is a \emph{proper} subgroup. Since \(\Z/p\Z\)
    is a simple group, it follows that \(p G \subseteq G\) is a \emph{maximal}
    subgroup. Suppose that \(L \subseteq G\) is another maximal subgroup of \(G\),
    so that \(G/L\) is a simple abelian group---and since \(G\) is \(p\)-primary,
    then \(G/L\) is a \(p\)-group and thus \(G/L \iso \Z/p\Z\).
\end{proof}

%%% Local Variables:
%%% mode: latex
%%% TeX-master: "../../../deep-dive"
%%% End:

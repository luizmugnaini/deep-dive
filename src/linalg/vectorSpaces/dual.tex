\section{Dual Vector Spaces}

\begin{definition}
  Let \(V \in \Obj(\Vect_k)\), then, we define the dual vector space of
  \(V\) to be
  \[
    V^\ast = \Hom_{\Vect_k} (V, k)
  \]
  and \(f \in V^\ast\) is called a linear functional on \(V\).
\end{definition}

\begin{proposition}
  For all sets \(S\) there exists a natural isomorphism between the dual free
  vector space generated by \(S\) and the function space \(S \to k\), that is
  \[
    (k^{\oplus S})^\ast \iso k^S.
  \]
\end{proposition}

\begin{proof}
  %Let the morphism \(\delta : k^S \to (k^{\oplus S})^*\) be defined as mapping \(f
  %\xmapsto \delta f^*\). We now show that such morphism is indeed an
  %isomorphims.
  This result comes directly from the free vector space universal property, that
  is
  \[
    \begin{tikzcd}
      S \ar[r, "f"] \ar[d, hook] & k \\
      k^{\oplus S} \ar[ur, dashed, swap, "\ell"]
    \end{tikzcd}
  \]
  so that we can assign for each function \(f \in k^S\) the corresponding unique
  \(k\)-linear morphism \(\ell \in (k^{\oplus S})^\ast\).
\end{proof}

\begin{proposition}\label{prop: finite dim vs iso dual vs}
  Let \(V\) be finite dimensional, then there exists a non-canonical isomorphism
  \[
    V \iso V^\ast
  \]
\end{proposition}

\begin{proof}
  The proof here is rather arbitrary, we'll just construct explicitly an
  isomorphism by defining an ordered basis \(B = (v_1, \dots, v_n)\) of \(V\),
  but we could build other isomorphisms (this is why the proposition states
  ``non-canonical''). Define the \(k\)-linear morphism \(\varphi(v_i) = e_i\)
  where \(e_i\) is the \(i\)-th standard basis on \(k^n\), then \(\varphi\) is
  an isomorphims \(V \iso k^n\). Moreover, from the standard inner product, we
  have that \(k^n \iso (k^n)^\ast\), thus we are done, since
  \[
    \begin{tikzcd}
      V \ar[r, "\varphi"] & k^n \iso (k^n)^\ast \ar[r, "f \mapsto f \circ
      \varphi"] &V^\ast
    \end{tikzcd}
  \]
  thus indeed \(V \iso V^\ast\).
\end{proof}

\begin{proposition}
  If \(V\) is finite-dimensional, then \(V^\ast\) is also finite dimensional and
  \[
    \dim_k(V) = \dim_k (V^\ast).
  \]
\end{proposition}

\begin{proposition}
  Given \(k\)-vector spaces \(V\) and \(W\), there exists a natural isomorphism
  \[
    (V \oplus W)^\ast \iso V^\ast \oplus W^\ast.
  \]
\end{proposition}

\begin{proof}
  Consider the mapping \(\Phi: (V \oplus W)^\ast \to V^\ast \oplus W^\ast\)
  defined by \(\Phi(f^\ast) = (f^\ast_V, f^\ast_W)\), where \(f^\ast_V =
  f^\ast|_{V \oplus \{0\}} \) and \(f^\ast_W = f^\ast|_{\{0\} \oplus W}\). We
  now show that \(\Phi\) is an isomorphism. (Injective) Suppose \(g^\ast \in
  \ker \Phi\), then necessarily \(g^\ast|_{V\oplus \{0\}} = 0\) and
  \(g^\ast|_{\{0\} \oplus W} = 0\), but since \(g^\ast = g^\ast|_{V\oplus
  \{0\}} + g^\ast|_{\{0\} \oplus W}\) we conclude that \(g^\ast = 0\).
  (Surjective) Let any \((\ell^\ast, t^\ast) \in V^\ast \oplus W^\ast\), then
  define the function \(f^\ast \in (V \oplus W)^\ast\) such that \(f^\ast(v, w)
  = \ell^\ast(v) + t^\ast(w)\), then \(f^\ast|_{V \oplus \{0\}} = \ell^\ast\)
  and \(f^\ast|_{\{0\} \oplus W} = t^\ast\), therefore \(\Phi(f^\ast) =
  (\ell^\ast, t^\ast)\).
\end{proof}

\begin{definition}
  Let \(f : V \to W\) be a \(k\)-linear morphism. We define the dual (or
  transpose) \(k\)-linear morphism of \(f\) as \(f^\ast = f^T : W^\ast \to
  V^\ast\). This is a particular case of the induced \(\Hom\) \(k\)-linear
  morphism, by taking the \(k\)-vector space \(k\), so that
  \[
    f^\ast : \Hom(W, k) \to \Hom(V, k),\ \alpha \mapsto \alpha \circ f
  \]
\end{definition}

\begin{proposition}
  Let \(f : V \to W\) be a \(k\)-linear morphism and consider its dual \(f^\ast
  : W^\ast \to V^\ast\). If \(f\) is injective, then \(f^\ast\) is surjective,
  on the other hand, if \(f\) is surjective, then \(f^\ast\) is injective.
  Hence, if \(f\) is an isomorphism, then \(f^*\) is an isomorphism.
\end{proposition}

\begin{proposition}
  Let \(V_1 \to V_2 \to V_3\) be an exact sequence, then \(V_3^\ast \to
  V_2^\ast \to V_1^\ast\) is also exact.
\end{proposition}

\begin{proof}
  Let \(f: V_1 \to V_2\) and \(g: V_2 \to V_3\) be such that \(\im(f) =
  \ker(g)\). Then, given any \(\alpha \in V_3^\ast\) and consider
  \(g^\ast(\alpha) = \alpha \circ g\), we then have \(f^\ast(\alpha \circ g) =
  (\alpha \circ g) \circ f = \alpha \circ (g \circ f) = 0\) from the
  construction of \(f, g\), thus \(\im(g^\ast) \subseteq \ker(f^\ast)\). On the
  other hand, let \(\beta \in \ker(f^\ast) \subseteq V_2^\ast\) so that
  \(f^\ast(\beta) = \beta \circ f = 0\), which is the same thing as requiring
  \(\beta|_{\im(f)} = 0\); this way we can see that for any element \(\alpha \in
  V_3^\ast\) we have \(g^\ast(\alpha) = \alpha \circ g\) be such that \(\alpha
  \circ g|_{\im(f)} = 0\) since \(\ker(g) = \im(f)\) and thus already satisfies
  as an element of the kernel of \(f^\ast\), therefore \(\ker(f^\ast) \subseteq
  \im(g^\ast)\).
\end{proof}

\begin{proposition}
  Let \(V_1 \to V_2 \to V_3\) be exact sequence of \(k\)-vector spaces, then for
  any given \(k\)-vector space \(W\) we have that
  \[
    \Hom(V_3, W) \longrightarrow \Hom(V_2, W) \longrightarrow \Hom(V_1, W)
  \]
  is exact (contravariant \(\Hom\)).
\end{proposition}

\begin{proof}
  The proof is almost identical to the latter.
\end{proof}

\begin{proposition}
  Let \(V_1 \to V_2 \to V_3\) be exact sequence of \(k\)-vector spaces, then for
  any given \(k\)-vector space \(W\) we have that
  \[
    \Hom(W, V_1) \longrightarrow \Hom(W, V_2) \longrightarrow \Hom(V_3, W)
  \]
  is exact (covariant \(\Hom\)).
\end{proposition}

\begin{proof}
  The proof only diverges from the above because we have the mappings
  \(g_\ast(\alpha) = g \circ \alpha\) and \(f_\ast(\beta) = f \circ \beta\).
\end{proof}

\begin{proposition}
  Consider a \(k\)-linear morphism \(f : V \to W\) such that \(\mathrm{rank}(f)\) is
  finite (that is, \(\dim_k(\im(f))\) exists). Then \(\mathrm{rank}(f) =
  \mathrm{rank}(f^\ast)\).
\end{proposition}

\begin{proof}
  Notice that if \(f : V \twoheadrightarrow U \hookrightarrow W\) is the
  decomposition of \(f\), then we can consider the dual \(f^\ast: W^\ast \to
  V^\ast\) as having a decomposition \(W^\ast \twoheadrightarrow U^\ast
  \hookrightarrow V^\ast\), since the dual of an injective map is surjective and
  the dual of a surjective map is injective. Therefore, given a basis \(B\) for
  \(U\), since \(\forall b \in B\) there exists \(v \in V\) for which \(f(v) =
  b\) we can say that \(b \in \im(f)\) and, moreover, \(\rank(f) \geq
  \dim_k(U)\). Also, since \(\im(f) \subseteq U\), then we have directly that
  \(\rank(f) \leq \dim_k(U)\), thus \(\rank(f) = \dim_k(U)\). The same shows
  that \(\rank(f^\ast) = \dim_k(U^\ast)\) and since \(U \iso U^\ast\) we find
  that \(\dim_k(U) = \dim_k(U^\ast)\) and hence \(\rank(f) = \rank(f^\ast)\).
\end{proof}

\begin{proposition}\label{prop: dual matrix equals transpose}
  Let \(\ell \in \Hom(V, W)\), where \(V \iso k^n\) and \(W \iso k^m\), so that
  \(\ell\) can be written as a matrix \(L : k^n \to k^m\). Then the dual
  \(\ell^\ast\) is represented by the matrix \(L^T\).
\end{proposition}

\begin{proof}
  Define \((v_j)_{j=1}^n\) an ordered basis for \(V\) and \((v_i^\ast)_{i=1}^n\)
  be its dual ordered basis; define also \((w_i)_{i=1}^m\) to be an ordered
  basis for \(W\) and \((w_j^\ast)_{j=1}^m\) be its dual ordered basis. Notice
  that the dual of the matrix, that is, \(L^\ast : (k^m)^\ast \to (k^n)^\ast\)
  can be written as \(L^\ast : k^m \to k^n\), since \((k^m)^\ast \iso k^m\) and
  \((k^n)^\ast \iso k^n\). Define the representations
  \[
    L =
    \begin{bmatrix}
      t_{1, 1} & \dots   & t_{1, n} \\
      \vdots   & \ddots  & \vdots   \\
      t_{m, 1} & \dots   & t_{m, n}
    \end{bmatrix}
    \ \text{ and define }\
    L^\ast =
    \begin{bmatrix}
      d_{1, 1} & \dots   & d_{1, m} \\
      \vdots   & \ddots  & \vdots   \\
      d_{n, 1} & \dots   & d_{n, m}
    \end{bmatrix}
  \]
  By definition of the dual linear morphism, we have \(L^\ast(w_j^\ast) =
  w_j^\ast \circ L\). When this transformation is thought of as a \(k^n \to
  k^m\) matrix, the definition of a matrix for a linear tranformation with
  respect to the given dual basis for \(W^\ast\) and \(V^\ast\) is defined such
  that \(L^\ast(w_j^\ast) = \sum_{r = 1}^n d_{r, j} v_r^\ast\). Hence, given any
  element \(v_k \in (v_j)_{j=1}^n \subseteq V\), from the first definition:
  \begin{equation}
    w_j^\ast \circ L (v_k)
    = w_j^\ast \left( \sum_{r=1}^m t_{r, k} w_r \right)
    = \sum_{r=1}^m t_{r, k} w_j^\ast(w_r)
    = t_{j, k}
  \end{equation}
  from the fact that we defined \(w_j^\ast\) so that \(w_j^\ast(w_j) = 1\) and
  \(w_j^\ast(w_i) = 0\) for \(i \neq j\). Now, from the second definition that
  we discussed, we have (applying the same \(v_k\)):
  \begin{equation}
    \sum_{r=1}^n d_{r,j} v_r^\ast(v_k)
    = d_{k, j}
  \end{equation}
  from the same fact. Thus we conclude that \(d_{k,j} = t_{j,k}\) for all \(1
  \leq k \leq n\) and \(1 \leq j \leq m\), thus indeed \(L^\ast = L^T\)
\end{proof}

\begin{definition}[Column and Row Rank]\label{def: column and row rank}
  Let \(L: k^n \to k^m\) be a matrix. Define \((v_j)_{j=1}^n\) to be the vectors
  whose components are given by the ordered \(j\)-th column elements of the
  matrix representation of \(L\). Analogously, define \((w_i)_{i=1}^m\) to be
  the vectors whose components are the ordered \(i\)-th row elements of the
  matrix representation of \(L\). We define the column rank of \(L\) to be
  \(\dim_k(\mathrm{span}(v_j)_{j=1}^n)\) and the row rank of \(L\) to be
  \(\dim_k(\mathrm{span}(w_i)_{i=1}^m)\).
\end{definition}

\begin{proposition}
  \label{prop: column and row rank equal the rank}
  Let \(L: V \to W\) be a linear morphism with \(V \iso k^n\) and \(W \iso
  k^m\), its row and column rank of its matrix representation \(k^n \to k^m\)
  are both equal to the rank of \(L\).
\end{proposition}

\begin{proof}
  Let \((v_j)_{j=1}^n\) and  \((w_i)_{i=1}^m\) be the column and row vectors of
  the matrix representation of \(L\).
  (Rank equals column rank) Let \(B_V\) and \(S_W\) be ordered basis for \(V\)
  and \(W\) respectively and any element \(w \in \im(L) = \mathrm{span}(Lb_k :
  b_k \in B_V)\) notice that since \(Lb_k = \sum_{i=1}^m t_{i, k} s_i\), where
  \(t_{i,j}\) are the elements of the matrix representation of \(L\), then we
  see that \(Lb_k\) is a linear combination of the \(k\)-th column vector of
  \(L\), hence \(Lb_k \in \mathrm{span}(v_j)_{j=1}^n\) for all \(1 \leq k \leq
  m\). Therefore we conclude that \(w \in \mathrm{span}(v_i)_{i=1}^n\).
  Moreover, we have \(\mathrm{span}(v_i)_{i=1}^n \subseteq \im(L)\), thus
  \(\mathrm{span}(v_i)_{i=1}^n = \im(L)\), hence we conclude finally that
  \(\dim_k(\im(L)) = \rank(L) = \dim_k(\mathrm{span}(v_i)_{i=1}^n)\) which
  states that the rank of \(L\) equals the column rank.
  (Rank equals column rank) Notice that \(L^\ast : (k^m)^\ast \to (k^n)^\ast\)
  is isomorphic to the matrix \(k^m \to k^n\) and the matrix representation of
  the dual of \(L\) is equal to its tranpose by \ref{prop: dual matrix equals
  transpose}, moreover we proved lastly that the column rank of a matrix equals
  its rank, thus it should be true that \(\rank(L^\ast) =
  \dim_k(\mathrm{span}(w_i)_{i=1}^m)\). Since \(L^\ast \iso L\), it follows that
  \(\rank(L) = \dim_k(\mathrm{span}(w_i)_{i=1}^m)\). Hence we conclude that
  \[
    \rank(L) = \rank(\text{column}) = \rank(\text{row}).
  \]
\end{proof}

\begin{definition}[Bilinear map]\label{def: bilinear map}
  A map \(V^\ast \times V \to k\) that has the mapping \((\alpha, v) \mapsto
  \alpha(v)\) is called a bilinear map. This map is linear only when we fix the
  second component to some \(v \in V\) so that \(f_v: V^\ast \to k\) with the
  mapping \(\alpha \mapsto \alpha(v)\) is a \(k\)-linear morphism.
\end{definition}

This defines a rather interesting canonical \(k\)-linear morphism
\[
  \Psi_V : V \to (V^\ast)^\ast,\
  \text{ mapping }\ v \mapsto (f_v = (\alpha \mapsto \alpha(v))).
\]

\begin{proposition}
  The morphism \(\Psi_V\) is an injection for every given vector space \(V\).
\end{proposition}

\begin{proposition}
  A finite dimensional space \(V\) is isomorphic to its double dual
  \(V^{\ast\ast}\). The same can't be said if \(V\) is infinite dimensional.
\end{proposition}

\begin{proof}
  Let \(\{v_i\}_{i=1}^n\) be a basis for \(V\), \(\{v_i^\ast\}_{i=1}^n\) be a
  basis for \(V^\ast\) dual to the first given basis, and
  \(\{v_i^{\ast\ast})_{i=1}^n\}\) be a basis for \(V^{\ast\ast}\) dual to the
  second given basis. We'll show that \(\Psi_V(v_i) = v_i^{\ast\ast}\). Let
  \(v_i\) be an element of the given basis of \(V\), then, we know from the
  definition of the dual of the basis that \(v_k^\ast(v_i) = \delta_{i, k}\) and
  \(v_t^{\ast\ast}(v_k^\ast) = \delta_{k, t}\).  Moreover, since \(\Psi_V\) is
  an injection, we see that if \(\Psi_V(v_i) \neq 0\), hence \(v_i \mapsto
  v_i^{\ast\ast}\), which proves the statement.
\end{proof}

\begin{definition}[Annihilator]\label{def: annihilator}
  Let \(V\) be a finite dimensional \(k\)-vector space and \(W\) be a subspace
  of \(V\). The annihilator of \(W\) is defined as the subspace of \(V^\ast\)
  given by
  \[
    W^0 = \{\alpha \in V^\ast : \alpha(W) = 0\}.
  \]
\end{definition}

\begin{proposition}
  Let \(V\) be a finite dimensional space and \(W \subseteq V\) a subspace. Then
  \[
    \dim_k(V) = \dim_k(W) + \dim_k(W^0).
  \]
\end{proposition}

\begin{proof}
  Let the inclusion \(\iota : W \hookrightarrow V\), then its dual \(\iota^\ast :
  V^\ast \twoheadrightarrow W^\ast\) is surjective. Moreover, let \(\alpha \in
  W^0\) then, \(\iota^\ast(\alpha) = \alpha \circ \iota\), since \(\im(\iota)
  \subseteq W\) we conclude that \(\iota^\ast(\alpha) = 0\) and thus \(\alpha
  \in \ker(\iota^\ast)\), so that \(W^0 \subseteq \ker(\iota^\ast)\). Moreover,
  if \(\alpha \in \ker(\iota^\ast)\) then, in particular, \(\alpha(W) = 0\)
  hence \(\alpha \in W^0\). Then we find that \(W^0 = \ker(\iota^\ast)\). We now
  use the theorem \ref{thm: rank plus nullity} on the morphism \(\iota^\ast\) so
  that \(\dim_k(V^\ast) = \dim_k(\ker(\iota^\ast)) + \rank(\iota^\ast)\). Since
  \(\iota^\ast\) is surjective, then \(\im(\iota^\ast) = W^\ast\) and therefore
  \(\rank(\iota^\ast) = \dim_k(W^\ast)\); moreover, \(\dim_k(\ker(\iota^\ast)) =
  \dim_k(W^0)\); and finally \(\dim_k(V^\ast) = \dim_k(V)\), and
  \(\dim_k(W^\ast) = \dim_k(W)\). Thus indeed \(\dim_k(V) = \dim_k(W) +
  \dim_k(W^0)\).
\end{proof}

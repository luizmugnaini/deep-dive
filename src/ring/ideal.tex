\section{Ideals \& Quotients of Rings}

\subsection{Ideals}

\begin{definition}[Ideal]
\label{def:ring-ideal}
Let \(R\) be a ring. A subgroup \(\ideal{a}\) of \((R, +)\) is said to be a
\emph{left-ideal} of \(R\) if for all \(r \in R\) we have
\(r \ideal{a} \subseteq \ideal{a}\). On the other hand, \(\ideal{a}\) is a
\emph{right-ideal} if for all \(r \in R\) we have
\(\ideal{a} r \subseteq \ideal{a}\). Furthermore, if \(\ideal{a}\) is both a
left and right ideal, we say that \(\ideal{a}\) is a \emph{two-sided-ideal} ---
or simply an \emph{ideal}, without qualifiers, which will be the preferred
nomenclature.
\end{definition}

\begin{remark}[Ideals and \(1\)]
\label{rem:ideals-and-unity}
The only ideal of a ring \(R\) containing the unity \(1_R\) is \(R\) itself ---
thus ideals need not be subrings.
\end{remark}

\begin{corollary}
\label{cor:image-is-ideal-then-surjective}
If \(\phi: R \to S\) is a ring morphism such that \(\im \phi\) is an ideal of
\(S\), then \(\phi\) is surjective.
\end{corollary}

\begin{proof}
Since \(\im \phi\) is a subring, it contains \(1_S\) --- by
\cref{rem:ideals-and-unity} we have \(\im \phi = S\).
\end{proof}

\begin{corollary}
\label{cor:ideals-addition-intersection}
The collection of ideals of a ring is closed under addition and intersection.
\end{corollary}

\begin{proof}
Let \(R\) be a ring and \(\ideal{a}, \ideal{b} \subseteq R\) be ideals. Let
\(a, b \in R\) be any two elements, then
\((a + b)(\ideal{a} + \ideal{b}) = (a + b) \ideal{a} + (a + b) \ideal{b}\) but
since \((a + b) \ideal{a}, (a + b) \ideal{b} \subseteq \ideal{a} + \ideal{b}\),
then \((a + b) (\ideal{a} + \ideal{b}) \subseteq \ideal{a} + \ideal{b}\) --- the
same analogous arguments can be used to show that
\((\ideal{a} + \ideal{b})(a + b) \subseteq \ideal{a} + \ideal{b}\). If
\(a, b \in \ideal{a} \cap \ideal{b}\), then
\((ab) \ideal{a} \cap \ideal{b}, \ideal{a} \cap \ideal{b} (a b) \subseteq
\ideal{a}, \ideal{b}\) thus also contained in \(\ideal{a} \cap \ideal{b}\).
\end{proof}

\begin{corollary}[Kernel is an ideal]
\label{cor:kernel-is-ideal}
Given a ring morphism \(\phi: R \to S\), then \(\ker \phi\) is a ring ideal of
\(R\).
\end{corollary}

\begin{proof}
From group theoretic considerations, we already know that \(\ker \phi\) is a
subring of \(R\). On the other hand, let \(r \in R\) be any element and \(a \in
\ker \phi\), then \(\phi(r a) = \phi(r) \phi(a) = 0\) and \(\phi(a r) = \phi(a)
\phi(r) = 0\), thus indeed both \(r a, a r \in \ker \phi\).
\end{proof}

\begin{corollary}[Preimage of ideal is an ideal]
\label{cor:inverse-image-of-ideal-is-ideal}
Let \(\phi: R \to S\) be a ring morphism and \(\ideal{a}\) be an ideal of \(S\),
then the preimage \(\phi^{-1}(\ideal{a})\) is an ideal of \(R\).
\end{corollary}

\begin{proof}
Let \(r \in R\) and \(a \in \phi^{-1}(\ideal{a})\) be any two elements. Notice
that \(\phi(r a) = \phi(r) \phi(a)\) and \(\phi(a r) = \phi(a) \phi(r)\), and
since \(\phi(a) \in \ideal{a}\) then \(\phi(r a), \phi(a r) \in \ideal{a}\) ---
therefore \(r a, a r \in \phi^{-1}(\ideal{a})\).
\end{proof}

\begin{remark}[Image of ideals]
\label{rem:image-of-ideal-need-not-be-ideal}
Although the preimage of ideals is an ideal of the morphism's source, the
\emph{image} of a given ideal \emph{need not be an ideal}. For instance, let
\(\ideal{a} \subseteq R\) be an ideal of a ring \(R\) and \(\phi: R \to S\) be a
\emph{non-surjective} morphism, then there exists \(s \in S\) whose preimage is
the empty set, therefore \(s \phi(\ideal{a})\) is not contained in
\(\phi(\ideal{a})\).
\end{remark}

\begin{corollary}[Surjective morphisms preserve ideals]
\label{cor:surjective-preserve-ideals}
If \(\phi: R \epi S\) is a surjective ring morphism, then the image of any ideal
\(\ideal{a} \subseteq R\) is an ideal of \(S\).
\end{corollary}

\begin{proof}
This is immediate from \cref{rem:image-of-ideal-need-not-be-ideal}, for all \(s
\in S\) we have both \(s \phi(\ideal{a}), \phi(\ideal{a}) s \subseteq
\phi(\ideal{a})\) because there will exist \(r \in R\) whose image is \(s\).
\end{proof}

\begin{example}
\label{exp:ideal-out-of-an-element}
For every element \(r \in R\) of a ring \(R\), the objects \(r R\) and \(R r\)
are, respectively, \emph{right} and \emph{left} ideals of \(R\). Indeed, given
any \(a \in R\), we have \((r R) a = r (R a) \subseteq r R\), the analogous
being true for the left ideal. If \(R\) is commutative, we have \(r R = R r\)
and such ideal is commonly denoted by \((r)\) --- the \emph{principal ideal}
generated by \(r\).
\end{example}

\begin{proposition}
\label{prop:direct-sum-of-ideals-is-ideal}
Let \(R\) be a ring and \((\ideal{a}_j)_{j \in J}\) be a collection of ideals of
\(R\), then the direct sum \(\bigoplus_{j \in J} \ideal{a}_j\) is an ideal of
\(R\).
\end{proposition}

\begin{proof}
Let \(r \in R\) be any element and consider any formal sum
\(\sum_{j \in J} a_j \in \bigoplus_{j \in J} \ideal{a}_j\) where \(a_j \neq 0\)
for only finitely many \(j \in J\), then
\(r (\sum_{j \in J} a_j) = \sum_{j \in J} r a_j\) bus since
\(r a_j \in \ideal{a}_j\) for each \(j \in J\), then
\(\sum_{j \in J} r a_j \in \bigoplus_{j \in J} \ideal{a}_j\) --- the same
argument can be used for right-multiplication.
\end{proof}

\begin{lemma}
\label{lem:direct-sum-smallest-ideal}
Given a collection \((\ideal{a}_j)_{j \in J}\) of ideals of a ring \(R\), the
ideal \(\bigoplus_{j \in J} \ideal{a}_j\) is the smallest ideal of \(R\)
containing each ideal \(\ideal{a}_j\) for \(j \in J\).
\end{lemma}

\begin{proof}
Let \(\ideal{b}\) be an ideal of \(R\) containing every ideal \(\ideal{a}_j\)
for \(j \in J\). Then in particular, given any element
\(\sum_{j \in J} a_j \in \bigoplus_{j \in J} \ideal{a}_j\), since
\(a_j \in \ideal{b}\) for every \(j \in J\) only finitely many such \(a_j\) are
non-zero, we find that the sum \(\sum_{j \in J} a_j \in \ideal{b}\) --- thus
indeed \(\bigoplus_{j \in J} \ideal{a} \subseteq \ideal{b}\).
\end{proof}

\begin{definition}[Finitely generated ideals]
\label{def:finitely-generated-ideal}
An ideal \(\ideal{a}\) of a \emph{commutative} ring \(R\) is said to be
\emph{finitely generated} if there exists a finite set of elements
\(A \subseteq R\) such that \(\ideal{a} = (A)\).
\end{definition}

\subsubsection{Division Rings and its Ideals}

\begin{proposition}[Units and its ideals]
\label{prop:unit-iff-R=aR}
Let \(R\) be a ring and \(a \in R\). The element \(a\) is a \emph{left-unit} (or
\emph{right-unit}) of \(R\) if and only if \(R = a R\) (or \(R = R a\))
\end{proposition}

\begin{proof}
We prove the proposition only for left-units, for right-units the proof is
completely analogous. Suppose \(a\) is a left-unit of \(R\) and let \(u \in R\)
be such that \(a u = 1\), then \(aR\) contains \(1\) which implies in \(a R =
R\). Now if \(R = a R\), then \(1 \in a R\), which means that there must exist
\(u \in R\) for which \(a u = 1\), thus \(a\) is a left-unit.
\end{proof}

\begin{proposition}[Division ring ideals]
\label{prop:division-ring-ideals-are-0-or-ring}
Let \(R\) be a ring. Then \(R\) is a \emph{division ring} if and only if its
\emph{only} left-ideals and right-ideals are \(\{0\}\) and \(R\).
\end{proposition}

\begin{proof}
Let \(R\) be a division ring. Suppose \(\ideal{a}\) is a left-ideal (or
right-ideal) of \(R\), then given any \(r \in R\) and \(a \in \ideal{a}\) we
have \(r a \in \ideal{a}\) (or \(a r \in \ideal{a}\)) --- in particular, if
\(a \neq 0\), then \(a^{-1} \in R\) thus \(a^{-1} a = 1 \in \ideal{a}\) (or
\(a a^{-1} = 1 \in \ideal{a}\)), thus \(\ideal{a} = R\) or \(\ideal{a} = 0\).

Suppose \(0\) and \(R\) are the only left and right ideals. Given any non-zero
element \(r \in R\), the ideal \(r R\) (or \(R r\)) is non-zero, thus must be
equal to \(R\), which is equivalent of \(1 \in r R\) (or \(1 \in R r\)) ---
therefore \(r\) must be a unit of \(R\), proving that \(R\) is a division ring.
\end{proof}

\begin{corollary}[Field ideals]
\label{cor:field-ideals}
The only left or right ideals of a field \(k\) are \(0\) and \(k\).
\end{corollary}

\begin{proposition}
\label{prop:morphism-field-to-ring-is-injective}
Let \(k\) be a field and \(R\) be a ring, then any ring morphism \(k \to R\) is
injective.
\end{proposition}

\begin{proof}
Let \(\phi: k \to R\) be a ring morphism and let \(a, b \in k\) be any two
elements such that \(\phi(a) = \phi(b)\), then \(a - b \in \ker \phi\). Remember
that \(\ker \phi\) is an ideal of \(k\) but this means that \(\ker \phi\) is
either \(0\) or \(k\) --- by
\cref{prop:division-ring-ideals-are-0-or-ring}. Since \(\phi(1_k) = 1_R\) for
\(\phi\) to be a ring morphism, then \(\ker \phi \neq k\) and we are left with
\(\ker \phi = 0\) --- thus \(a = b\).
\end{proof}

\subsubsection{Nilpotent Ideal}

\begin{definition}[Nilpotent ideal]
\label{def:nilpotent-ideal}
Let \(R\) be a ring. We say that an ideal \(\ideal a \subseteq R\) is a \emph{nilpotent
  ideal} if there exists \(k \in \Z_{> 0}\) such that the product of any \(k\)
elements of \(\ideal a\) equals zero.
\end{definition}

\subsubsection{Nilradicals}

\begin{definition}[Nilradical]
\label{def:nilradical}
Given a ring \(R\), we define the \emph{nilradical} of \(R\)
to be the object \(N\) composed of every nilpotent element of \(R\).
\end{definition}

\begin{corollary}
\label{cor:nilradical-is-ideal}
The nilradical \(N\) of a \emph{commutative ring} \(R\) is an \emph{ideal} of
\(R\).
\end{corollary}

\begin{proof}
Let \(a \in R\) and \(x \in N\) be any two elements --- suppose \(n \in
\Z_{>0}\) is such that \(x^n = 0\). Since \(R\) is commutative then \((a x)^n =
a^n x^n = a^n \cdot 0 = 0\), thus \(a x \in N\).
\end{proof}

\begin{definition}[Reduced ring]
\label{def:reduced-ring}
A ring \(R\) is said to be reduced if it contains no non-zero nilpotent elements
\end{definition}

\begin{corollary}
\label{cor:reducing-with-nilradical}
Let \(R\) be a \emph{commutative ring} and \(N\) be its nilradical. The quotient
ring \(R/N\) is \emph{reduced}.
\end{corollary}

\begin{proof}
Suppose \(a + N \in R/N\) is a nilpotent element and \(a^n + N = N\) for some
\(n \in \Z_{>0}\) --- which implies in \(a^n \in N\) thus there must exist
\(m \in \Z_{>0}\) such that \((a^n)^m = a^{nm} = 0\), which implies in
\(a \in N\) and hence \(a + N = N\).
\end{proof}

\begin{lemma}
\label{lem:reduced-iff-r2=0-implies-r=0}
A ring \(R\) is \emph{reduced} if and only if for all \(r \in R\) such that
\(r^2 = 0\) implies \(r = 0\).
\end{lemma}

\begin{proof}
If \(R\) is reduced then clearly \(r^2 = 0\) implies \(r = 0\). Let \(r \in R\)
be any element such that \(r^n = 0\) for some \(n \in \Z_{>0}\), then proceed
by the following algorithm --- start with \(r^n\), then:
\begin{itemize}\setlength\itemsep{0em}
\item If \(n = 1\), return \(r\) --- since \(r = 0\).
\item If \(n\) is odd, \(r^{n + 1} = 0\) and thus
  \(r^{(n+1)/2} r^{(n+1)/2} = 0\) which by hypothesis implies \(r^{(n+1)/2} =
  0\). Continue the algorithm for \(r^{(n+1)/2}\).
\item If \(n\) is even, \(r^n = r^{n/2} r^{n/2} = 0\) implies \(r^{n/2} =
  0\). Continue the algorithm for \(r^{n/2} = 0\).
\end{itemize}
Such algorithm is ensured to terminate and will always result in \(r = 0\),
which implies in \(R\) being a reduced ring.
\end{proof}

\subsection{Quotient Ring}

We define now, for every ideal \(\ideal{a}\) of a given ring \(R\), a ring
\(R/\ideal{a}\) together with an additive and multiplicative structure: for
every \(a + \ideal{a}, b + \ideal{a} \in R/\ideal{a}\), we define
\begin{gather*}
(a + \ideal{a}) + (b + \ideal{a}) \coloneq (a + b) + \ideal{a}, \\
(a + \ideal{a}) \cdot (b + \ideal{a}) \coloneq a b + \ideal{a}.
\end{gather*}

Let's show that both operations are well defined. Addition is clearly well
defined. For the multiplication, suppose \(a_1 + \ideal{a} = a_2 + \ideal{a}\)
and \(b_1 + \ideal{a} = b_2 + \ideal{a}\) --- that is, both differences
\(a_1 - a_2\) and \(b_1 - b_2 \in \ideal{a}\) are elements of
\(\ideal{a}\). Notice that
\begin{align*}
a_1 b_1 - a_2 b_2
&= a_1 b_1 - a_2 b_2 + (a_1 b_2 - a_1 b_2) \\
&= (a_1 b_1 - a_1 b_2) + (a_1 b_2 - a_2 b_2) \\
&= a_1 (b_1 - b_2) + (a_1 - a_2) b_2
\end{align*}
and therefore \(a_1 b_1 - a_2 b_2 \in \ideal{a}\), implying in
\(a_1 b_1 + \ideal{a} = a_2 b_2 + \ideal{a}\) --- thus multiplication is well
defined.

\begin{proposition}[Universal property of quotient rings]
\label{prop:universal-property-quotienting-rings}
Let \(R\) be a ring. For every ring \(S\) together with a ring morphism \(\psi:
R \to S\) and ideal \(\ideal{a} \subseteq \ker \psi\), there exists a
\emph{unique} ring morphism \(\phi: R/\ideal{a} \unique S\) such that the
following diagram commutes
\[
\begin{tikzcd}
R \ar[r, "\psi"] \ar[d, swap, "\pi", two heads] &S \\
R/\ideal{a} \ar[ru, bend right, swap, "\phi", dashed] &
\end{tikzcd}
\]
\end{proposition}

\begin{proof}
Let \(S\) be any ring together with a ring morphism \(\psi: R \to S\). Let
\(\pi: R \epi R/\ideal{a}\) be the canonical projection morphism. Define a map
\(\phi: R/\ideal{a} \to S\) by sending \(r + \ideal{a} \mapsto \psi(r)\). We
show now that \(\phi\) is indeed well defined, consider elements
\(x, y \in R/\ideal{a}\) to be such that \(\phi(x) = \phi(y)\), then
\(\psi(x) = \psi(y)\) which is the same as
\(\psi(x) - \psi(y) = \psi(x - y) = 0\), therefore \(x - y \in \ker \psi\), then
\(x = y\) in \(R/\ideal{a}\). Moreover, \(\phi\) inherits from \(\psi\) the
preservation of both additive and multiplicative structures, thus \(\phi\) is a
morphism of rings and \(\phi \pi = \psi\). If \(\phi'\) is another morphism such
that \(\phi' \pi = \psi\), since \(\pi\) is surjective then \(\phi \pi = \phi'
\pi\) implies in \(\phi = \phi'\), thus \(\phi\) is unique.
\end{proof}

\begin{corollary}[Every ideal is a kernel]
\label{cor:ideal-is-kernel}
Given a ring \(R\), for every ideal \(\ideal{a} \subseteq R\) is the kernel of
some ring morphism \(R \to S\). Therefore an additive subgroup of a ring \(R\)
is an ideal if and only if it is a kernel of some ring morphism.
\end{corollary}

\begin{proof}
Simply let \(S = R/\ideal{a}\) and consider the canonical projection \(\pi: R
\to R/\ideal{a}\), whose kernel is clearly \(\ideal{a}\).
\end{proof}

\begin{definition}[Characteristic]
\label{def:ring-characteristic}
Let \(R\) be a ring and \(\phi: \Z \unique R\) be the unique ring morphism
mapping \(r \mapsto r \cdot 1_R\). We define the \emph{characteristic} of \(R\)
to be the non-negative integer \(n \in \Z_{> 0}\) such that \(\ker \phi = n \Z\)
--- we denote such a property by \(\Char R = n\).
\end{definition}

\begin{proposition}[Characteristic of integral domains]
\label{prop:char-integral-domain-is-0-or-p}
The characteristic of an \emph{integral domain} is either \emph{zero} or a
\emph{prime number}.
\end{proposition}

\begin{proof}
Let \(R\) be an integral domain and let \(\Char R = n\). Suppose that \(n\) is
non-zero and that there exists an integer \(m < n\) dividing \(n\) --- that is,
for some integer \(q\) we have \(n = q m\). Then in particular
\(n \cdot 1_R = (q \cdot 1_R) (m \cdot 1_R)\) but \(n \cdot 1_R = 0\) and since
\(q, m < n\) it follows that \(q \cdot 1_R, m \cdot 1_R \neq 0\) in \(R\) ---
thus we obtained a contradiction since \(R\) is said to be an integral domain,
hence \(\Char R\) is either prime or zero.
\end{proof}

\begin{definition}[Boolean ring]
\label{def:boolean-ring}
A ring \(R\) is said to be \emph{boolean} if for all \(r \in R\) we have
\(r^2 = r\).
\end{definition}

\begin{corollary}
\label{cor:boolean-is-commutative-and-char-2}
A non-zero \emph{boolean} ring is \emph{commutative} and has characteristic
\(2\).
\end{corollary}

\begin{proof}
Let \(x \in R\) be any element of a boolean ring \(R\). Then we have
\(x + x = (x + x)^2 = 4 x^2\) and \(x + x = x^2 + x^2\), therefore,
\(2 x^2 = 4 x^2\) which implies in \(x^2 + x^2 = 0\) but since
\(x^2 + x^2 = x + x\) then we conclude that \(x + x = 0\) and thus
\(\Char R = 2\).

If \(x, y \in R\) are any two elements, then consider the sum
\[
x + y
= (x + y)^2
= x^2 + x y + y x + y^2
= x + x y + y x + y,
\]
hence cancelling the common terms we obtain \(x y + y x = 0\), now since
\(\Char R = 2\) one can use the fact that \(y x + yx = 0\) to make
\[
0 = x y + y x = x y + y x - (y x + y x) = x y - y x,
\]
thus \(x y = y x\).
\end{proof}

\begin{corollary}
\label{cor:boolean-integral-domain-is-Z/2Z}
If \(R\) is a boolean ring and also an integral domain, then there exists a
canonical isomorphism of rings \(R \iso \Z/2\Z\).
\end{corollary}

\begin{proof}
Let \(r in R\) be any element, since \(r^2 = r\) then
\(r^2 - r = r(1_R - r) = 0_R\) but since \(R\) is an integral domain, either
\(r = 0_R\) or \(1_R - r = 0_R\), that is, \(r = 1_R\). Thus the map
\(R \to \Z/2\Z\) sending \(0_R \mapsto [0]_2\) and \(1_R \mapsto [1]_2\) is an
isomorphism of rings.
\end{proof}

\subsection{Decompositions}

\begin{theorem}[First isomorphism]
\label{thm:ring-first-isomorphism}
Every ring morphism \(\phi: R \to S\) can be decomposed into the commutative
diagram
\[
\begin{tikzcd}
R \ar[r, two heads] \ar[rrr, bend left, "\phi"]
&R/{\ker \phi} \ar[r, "\dis", "\overline{\phi}"']
&\im \phi \ar[r, hook]
&S
\end{tikzcd}
\]
where \(\overline{\phi}: R/\ker \phi \isoto \im \phi\) is the natural ring
morphism induced by \(\phi\).
\end{theorem}

\begin{proof}
The morphism \(\overline{\phi}\) is obtained by the quotient ring universal
property --- that is, \(\overline{\phi}(r) = \phi(r) + \ker \phi\), which is a
morphism \(R/{\ker \phi} \to S\). Restricting the codomain of
\(\overline{\phi}\) we obtain the claimed isomorphism. The rest follows
trivially.
\end{proof}

\begin{corollary}
\label{cor:isomorphism-quotient-by-kernel-of-surjective-morphism}
Let \(\phi: R \epi S\) be a \emph{surjective} ring morphism, then there exists
a natural isomorphism
\[
S \iso R/{\ker \phi}
\]
\end{corollary}

\begin{proof}
Indeed, if \(\phi\) is surjective, then \(\im \phi = S\) and by the first
isomorphism theorem we obtain the natural isomorphism \(S \iso R/{\ker \phi}\).
\end{proof}

\begin{proposition}[Ideal of a quotient]
\label{prop:ideal-of-a-quotient}
Let \(R\) be a ring and \(\ideal{a} \subseteq R\) be an ideal. If \(\ideal{b}\)
is an ideal of \(R\) and \(\ideal{a} \subseteq \ideal{b}\), then
\(\ideal{b}/\ideal{a}\) is an ideal of \(R/\ideal{a}\) and there exists a
natural isomorphism
\[
\frac{R/\ideal{a}}{\ideal{b}/\ideal{a}} \iso R/\ideal{b}.
\]
\end{proposition}

\begin{proof}
By \cref{prop:universal-property-quotienting-rings}, let
\(\phi: R/\ideal{a} \unique R/\ideal{b}\) be the morphism making the following
diagram commute
\[
\begin{tikzcd}
R \ar[r, two heads] \ar[d, two heads] & R/\ideal{b} \\
R/\ideal{a} \ar[ru, dashed, bend right, "\phi"'] &
\end{tikzcd}
\]
That is, \(\phi\) is defined by mapping \(r + \ideal{a} \mapsto r +
\ideal{b}\). Notice that the kernel of \(\phi\) is composed of those elements
\(r + \ideal{a} \in R/\ideal{a}\) for which \(\phi(r + \ideal{a}) = \ideal{b}\),
that is, \(r \in \ideal{b}\) for \(r + \ideal{b} = \ideal{b}\) --- thus
\(\ker \phi = \ideal{b}/\ideal{a}\), which is ensured to be an ideal. Since
\(\phi\) is surjective, by means of
\cref{cor:isomorphism-quotient-by-kernel-of-surjective-morphism} we obtain a
natural isomorphism \(R/\ideal{b} \iso \frac{R/\ideal{a}}{\ideal{b}/\ideal{a}}\).
\end{proof}

\begin{corollary}
\label{cor:surjective-is-projection-into-quotient}
Any surjective morphism \(\phi: R \epi S\) can be bijectively identified as a
canonical projection \(R \epi R/\ker \phi\).
\end{corollary}

\begin{example}
\label{exp:principal-ideals-and-ideals-of-a-quotient}
If \(R\) is a \emph{commutative} ring and we consider principal ideals \((a)\)
and \((b)\) of \(R\). If \([b] \in R/(a)\) is the class of \(b\), then we have
\(([b]) = (a, b) / (a)\). By \cref{prop:ideal-of-a-quotient} we find a
canonical isomorphism
\[
\frac{R/(a)}{([b])} \iso R/(a, b).
\]
\end{example}

\subsection{Generation of Ideals}

\begin{definition}[Noetherian ring]
\label{def:noetherian-ring}
A \emph{commutative} ring \(R\) is said to be \emph{Noetherian} if \emph{every}
ideal of \(R\) is \emph{finitely} generated.
\end{definition}

\begin{proposition}[Image of Noetherian ring]
\label{prop:image-of-noetherian-is-noetherian}
Let \(R\) be a Noetherian ring and \(S\) be a ring. If there exists a
\emph{surjective} ring morphism \(R \epi S\) then \(S\) is Noetherian.
\end{proposition}

\begin{proof}
Suppose there exists \(\phi: R \epi S\), a surjective ring morphism. Let
\(\ideal{s}\) be any ideal of \(S\), since the preimage of an ideal is an ideal,
then \(\phi^{-1}(\ideal{s})\) is an ideal of \(R\) --- thus finitely
generated. Now, since \(\phi\) is surjective, \(\phi^{-1}(\ideal{s})\) is
non-empty and there exists a finite set \(A \subseteq \phi^{-1}(\ideal{s})\)
such that \(\phi^{-1}(\ideal{s}) = (A)\), since \(R\) is Noetherian. Therefore
\(\phi(A) = B \subseteq \ideal{s}\) is a finite set, hence
\(\phi((A)) = (B) = \ideal{s}\) --- proving that \(\ideal{s}\) is finitely
generated.
\end{proof}

\begin{definition}[Principal ideal domain]
\label{def:PID}
An \emph{integral domain} \(R\) is said to be a \emph{principal ideal domain}
(which we'll shortly name PID) if \emph{every} ideal of \(R\) is
\emph{principal}.
\end{definition}

\begin{example}
\label{exp:integers-are-PID}
The ring of integers \(\Z\) is a PID. Indeed, if \(\ideal{a}\) is an ideal of
\(\Z\), then there exists \(n \in \Z\) for which \(\ideal{a}\) is a subgroup of
\(n \Z = (n)\), thus \(\ideal{a}\) itself is principal.

Notice also that, given any two integers \(m, n \in \Z\), if
\(d \coloneq \gcd(m, n)\) then \(m, n \in (d)\), which implies that
\((m, n) \subseteq (d)\). Moreover, from Bézout's identity\footnote{Using the
  well order on \((m, n)\), let \(\ell
  \coloneq a_0 m + b_0 n\) be the smallest element of \((m, n)\). Notice that
  given any other \(u = a m + b n \in (m, n)\), by the euclidean division
  algorithm there exists two integers \(q, r \in \Z\) for which \(u = q \ell +
  r\) and \(0 \leq r < \ell\). Therefore, one can write
  \[
  r = u - q \ell
  = (a m + b n) - q (a_0 m + b_0 n) = (a - q a_0) m + (b - q b_0) n
  \]
  so that \(r \in (m, n)\) --- since \(\ell\) is the smallest element, then \(r
  = 0\) and thus \(\ell\) divides \(u\).

  In general, we have shown that \(\ell\) divides every element of \((m, n)\),
  and in particular \(\ell\) also divides both \(m\) and \(n\). Now, if \(c \in
  \Z\) is any common divisor of \(m\) and \(n\) then in particular \(c\) divides
  \(a_0 m + b_0 n = \ell\) --- showing that \(c \leq \ell\) and hence \(\ell =
  \gcd(m, n)\).
},
we have the existence of \(a, b \in \Z\) for which \(a m + b n = d\), thus
\(d \in (m, n)\) --- proving that \((m, n) = (d)\).
\end{example}

\begin{example}
\label{exp:Z[x]-isnt-PID}
The ring \(\Z[x]\) is \emph{not} a PID.

We consider the ideal \((2, x)\) and show that it isn't principal. Suppose there
exists \(f(x) \in \Z[x]\) for which \((f(x)) = (2, x)\), so that there exists
\(q(x) \in \Z[x]\) such that \(q(x) f(x) = 2\). Notice however that since \(\Z\)
is an integral domain, the product of the leading term coefficients of \(f(x)\)
and \(q(x)\) is necessarily non-zero (for non-zero polynomials), thus
\(\deg f(x) g(x) = \deg f(x) + \deg g(x)\). Notice that \(\deg q(x) f(x) = 0\)
thus \(\deg f(x) = 0\) and this can't be the case since we would have
\(x \notin (f(x))\) --- this is a contradiction, such polynomial \(f(x)\) cannot
exist and therefore \((2, x)\) isn't principal.
\end{example}

\begin{proposition}
\label{prop:k[x]-is-PID}
Given a field \(k\), the ring of polynomials \(k[x]\) is a PID.
\end{proposition}

\begin{proof}
The zero ideal is obviously principal, thus let \(\ideal{a}\) be any non-zero
proper ideal of \(k[x]\) and, by the well-ordering of the positive degree
polynomials in \(k[x]\), let \(f(x) \in \ideal{a}\) be a non-zero monic
polynomial with least degree.  We can indeed be certain that we can choose an
\(f(x)\), because every non-zero coefficient of a polynomial of \(k[x]\) is a
unity --- since \(\ideal{a}\) is non-zero, we are ensured that there will exist
a polynomial with at least one non-zero coefficient.

Let \(g(x) \in k[x]\) be any polynomial and, since \(k\) is a field, let
\(q(x), r(x) \in k[x]\) be polynomials such that \(g(x) = q(x) f(x) + r(x)\),
where \(\deg r < \deg f(x)\). Notice then that the last condition implies in
\(\deg r(x) \leq 0\), otherwise \(f(x)\) wouldn't be the monic polynomial with
least degree among polynomials of positive degree in \(k[x]\). Now, if
\(r(x) = 0\), then \(g(x) = q(x) f(x)\) and \(g(x) \in (f(x))\) --- on the other
hand, if \(r(x) = a\) for some \(a \in k\), then \(a \in \ideal{a}\) but since
\(k\) is a field, this implies in \(1 \in \ideal{a}\) thus \(\ideal{a} = k[x]\)
and hence a contradiction to the hypothesis that \(\ideal{a}\) is
proper. Assuming the latter does not hold, we find that \((f(x)) = \ideal{a}\)
and hence \(k[x]\) is a PID.
\end{proof}

\begin{definition}[Product of ideals]
\label{def:product-ideal}
Given a ring \(R\) and ideals \(\ideal{a}\) and \(\ideal{b}\) of \(R\), we
denote by \(\ideal{a b}\) the ideal generated by all products \(a b\) for \(a
\in \ideal{a}\) and \(b \in \ideal{b}\).
\end{definition}

\begin{lemma}
\label{lem:}
Let \(\ideal{a}\) and \(\ideal{b}\) be ideals of a \emph{commutative} ring
\(R\). If either one of the following properties is true:
\begin{enumerate}[(a)]\setlength\itemsep{0em}
\item The ideals \(\ideal{a}\) and \(\ideal{b}\) are \emph{comaximal}, that is,
  \(\ideal{a} + \ideal{b} = R\).

\item The quotient \(R/(\ideal{a b})\) is a \emph{reduced} ring.
\end{enumerate}
Then it follows that
\[
\ideal{a b} = \ideal{a} \cap \ideal{b}.
\]
\end{lemma}

\begin{proof}
If \(a b \in \ideal{a b}\) is any element, then
\(a, b \in \ideal{a} \cap \ideal{b}\) and hence
\(a b \in \ideal{a} \cap \ideal{b}\) --- thus in either cases we have
\(\ideal{a b} \subseteq \ideal{a} \cap \ideal{b}\). We now prove the other side
of the inclusion for each of the properties --- let
\(\ell \in \ideal{a} \cap \ideal{b}\) be any element:
\begin{enumerate}[(a)]\setlength\itemsep{0em}
\item Since \(\ideal{a} + \ideal{b} = R\), then there are \(a \in \ideal{a}\)
  and \(b \in \ideal{b}\) such that \(a + b = 1\). Moreover, since
  \(\ell \in \ideal{a} \cap \ideal{b}\), it follows that
  \(a \ell + b \ell = \ell\) is an element of \(\ideal{a b}\), thus
  \(\ideal{a} \cap \ideal{b} \subseteq \ideal{a b}\).

\item Notice that we have \(\ell^2 \in \ideal{a b}\), thus
  \(\ell^2 + \ideal{a b} = \ideal{a b}\). Since \(R/(\ideal{a b})\) is reduced,
  it follows that \(\ell + \ideal{a b} = \ideal{a b}\) (see
  \cref{lem:reduced-iff-r2=0-implies-r=0}) --- thus \(\ell \in \ideal{a b}\) and
  \(\ideal{a} \cap \ideal{b} \subseteq \ideal{a b}\).
\end{enumerate}
\end{proof}

\begin{lemma}
\label{lem:monomial-non-zero-divisor}
Let \(R\) be a ring and \(f(x) \in R[x]\) be a \emph{monomial}. Then \(f(x)\) is
\emph{not} a left or right zero-divisor.
\end{lemma}

\begin{proof}
Since \(f(x)\) is is a monomial and there is no element \(r \in R\) other than
zero for which \(r \cdot 1_R = 0\) (where \(1_R\) is the coefficient of the
leading term of \(f(x)\)), then it follows that the only polynomial
\(z(x) \in R[x]\) for which \(f(x) z(x) = 0\) (or \(z(x) f(x) = 0\)) is the zero
polynomial \(z(x) = 0\).
\end{proof}


\begin{lemma}[Degree of the product of polynomials]
\label{lem:degree-product-of-polynomials}
If \(f(x) \in R[x]\) is \emph{monic} a polynomial and \(R\) is a ring, then for
every \(g(x) \in R[x]\) we have
\[
\deg (f(x) g(x)) = \deg f(x) + \deg g(x).
\]
\end{lemma}

One can extend the euclidean division algorithm to the ring of polynomials,
where we'll be able to divide polynomials by monomials.

\begin{lemma}[Division of polynomials]
\label{lem:division-of-polynomials}
Given a \emph{monic} polynomial \(f(x) \in R[x]\), for some ring \(R\), then for
all \(g(x) \in R[x]\) there exists two polynomials \(q(x), r(x) \in R[x]\) for
which
\[
g(x) = f(x) q(x) + r(x),
\]
and \(\deg r(x) < \deg f(x)\). Moreover, such polynomials \(q(x)\) and \(r(x)\)
are unique.
\end{lemma}

\begin{proof}
Indeed, if \(q'(x), r'(x) \in R[x]\) are polynomials --- and, for the sake of
contradiction, \emph{distinct} from the respective \(q(x)\) and \(r(x)\) --- for
which \(g(x) = f(x) q'(x) + r'(x)\), and \(\deg r'(x) < \deg f(x)\), we find
that \(f(x) q(x) + r(x) = f(x) q'(x) + r'(x)\) and thus
\begin{equation}\label{eq:polynomial-divison-unique-poly}
r(x) - r'(x) = f(x) (q'(x) - q(x)).
\end{equation}
Notice however that by hypothesis \(\deg r(x), \deg r'(x) < \deg f(x)\), thus
\(\deg(r(x) - r'(x)) \leq \max(\deg r(x), \deg r'(x)) < \deg f(x)\) --- which is
in contradiction with \cref{eq:polynomial-divison-unique-poly} since for such
equation to be true we should have
\(\deg(r(x) - r'(x)) = \deg(f(x) (q'(x) - q(x))) = \deg f(x) + \deg(q'(x) -
q(x))\). Therefore \(r(x) - r'(x)\) is necessarily the zero polynomial and
\(f(x) q(x) = f(x) q'(x)\), now since \(f(x)\) is a monomial and by
\cref{lem:monomial-non-zero-divisor} is not a zero-divisor, we use
\cref{prop:no-zero-divisor-injective-multiplication} to conclude that
\(q(x) = q'(x)\).
\end{proof}

In what follows, one should regard \(R^{\oplus d}\), for a ring \(R\), to be the
ring of polynomials with degree less than or equal to \(d\) --- this observation
is based on the fact that one can map injectively
\begin{equation}\label{eq:embedding-Rd-in-R[x]}
\Psi: R^{\oplus d} \mono R[x]\ \text{ sending }\
(r_0, \dots, r_{d-1}) \mapsto \sum_{j=0}^{d-1} r_j x^j,
\end{equation}
which is a morphism of \emph{abelian groups}. Therefore, the restriction of the
codomain to the image of such map induces an \emph{isomorphism of abelian
groups}
\begin{equation}\label{eq:iso-Rd-in-R[x]}
R^{\oplus} \iso \im \Psi \subseteq R[x].
\end{equation}

\begin{proposition}
\label{prop:isomorphism-ring-quotient-by-monic-degree-d}
Let \(R\) be a \emph{commutative} ring, and \(f(x) \in R[x]\) a \emph{monic}
polynomial with degree \(d \in \Z_{> 0}\). Define a map
\(\phi: R[x] \to R^{\oplus d}\) sending \(g(x) \mapsto (r_0, \dots, r_{d-1})\),
where \(r(x) \coloneq \Psi(r_0, \dots, r_{d-1}) \in R[x]\) is the
\emph{remainder} of the division of \(g(x)\) by \(f(x)\). Such map \(\phi\)
induces a natural isomorphism of \emph{abelian groups}
\[
R[x]/(f(x)) \iso R^{\oplus d}.
\]
\end{proposition}

\begin{proof}
Notice that every polynomial \(g(x)\) can be divided by a monic polynomial
\(f(x)\) and the remainder \(r(x)\) of such division is unique by
\cref{lem:division-of-polynomials}. Moreover, the isomorphism in
\cref{eq:iso-Rd-in-R[x]} is a right inverse of \(\phi\), thus \(\phi\) is
surjective. Moreover, \(\ker \phi = (f(x))\) since every polynomial in the
principal ideal \((f(x))\) is divisible by \(f(x)\) and hence has a zero
remainder when divided by \(f(x)\).

Let's check that \(\phi\) is indeed a morphism of abelian groups. Let \(g(x),
g'(x) \in R[x]\) be two polynomials and let \(q(x), r(x) \in R[x]\) and \(q'(x),
r'(x) \in R[x]\) be their respective pair of quotient and remainder in the
division by \(f(x)\) --- where \(\deg r(x), \deg r'(x) < \deg f(x)\). Therefore,
the sum of \(g(x)\) with \(g'(x)\) can be written as
\[
g(x) + g'(x) = f(x) (q(x) + q'(x)) + (r(x) + r'(x)).
\]
Since \(\deg (r(x) + r'(x)) \leq \max(\deg r(x), \deg r'(x)) < \deg f(x)\), then
\(r(x) + r'(x)\) is the remainder of the division of \(g(x) + g'(x)\) by
\(f(x)\). With this in our hands we can rightly see that
\begin{align*}
\phi(g(x) + g'(x))
&= (r_0 + r_0', \dots, r_{d-1} + r_{d-1}') \\
&= (r_0, \dots, r_{d-1}) + (r_0', \dots, r_{d-1}') \\
&=\phi(g(x)) + \phi(g'(x)),
\end{align*}
where \(\Psi(r_0, \dots, r_{d-1}) \coloneq r(x)\) and
\(\Psi(r_0', \dots, r_{d-1}') \coloneq r'(x)\) --- thus \(\phi\) is a morphism
of abelian groups. By \cref{prop:universal-property-quotients-grp} we find
\[
R[x]/{\ker \phi} = R[x]/(f(x)) \iso \im \phi = R^{\oplus d}.
\]
\end{proof}

\begin{example}[Evaluation]
\label{exp:evaluation-induces-isomorphism}
Let \(R\) be a commutative ring. The evaluation morphism \(\eval_a: R[x] \to R\)
(see \cref{prop:universal-property-polynomial-rings}), mapping
\(f(x) \mapsto f(a)\), induces a natural isomorphism of \emph{abelian groups}
\[
R[x]/(x - a) \iso R.
\]
Such isomorphism, name it \(\phi\), is given by the map
\(g(x) + (x - a) \mapsto r\), where \(r \in R\) is the remainder of the division
of \(g(x)\) by \(x - a\).

First of all, if \(f(x) \in R[x]\) then the division by \(x - a\) yields
\(q(x) \in R[x]\) and \(r \in R\) such that \(f(x) = (x - a) q(x) + r\) since
the degree of the remainder should be less than \(\deg(x - a) = 1\). Therefore,
evaluating \(\eval_{a}(f(x)) = (a - a) q(x) + r = r\) implies in \(f(a) = r\)
--- thus \(f(x) \in \ker \phi\) if and only if \(f(x) \in (x - a)\), hence
\(\ker \phi = (x - a)\). Moreover, we can now rethink of the \(\phi\) as the
mapping \(f(x) + (x - a) \mapsto f(a)\).

If \(u(x) = u(a) + (x - a)\) and \(v(x) = v(a) + (x - a)\) are any two
polynomials in \(R[x]/(x - a)\), one finds that
\(u(x) + v(x) = (u(a) + v(a)) + (x - a)\) and therefore \(\phi(u(x) + v(x)) =
u(a) + v(a) = \phi(u(x)) + \phi(v(x))\), thus \(\phi\) is a morphism of abelian
groups. Therefore, by means of
\cref{cor:isomorphism-quotient-by-kernel-of-surjective-morphism} we find that
\(\phi\) is an isomorphism.
\end{example}

\begin{example}[Constructing \(\CC\)]
\label{exp:complex-numbers-from-R[x]}
Consider, as in \cref{prop:isomorphism-ring-quotient-by-monic-degree-d}, the map
\(\phi: \R[x] \to \R \oplus \R\) sending \(f(x) \mapsto (r_0, r_1)\), where
\(r(x) = r_0 + r_1 x\) is the remainder of the division of \(f(x)\) by the
polynomial \(x^2 + 1 \in \R[x]\).

For any element \(f(x) \coloneq a_0 + a_1 x \in \R[x]\) we have
\(\deg f(x) < \deg(x^2 + 1) = 2\), then \(\phi(f(x)) = (a_0, a_1)\). Hence,
given another \(g(x) = b_0 + b_1 x \in \R[x]\), one has that
\begin{align*}
  f(x) g(x)
  &= (a_0 + a_1 x) (b_0 + b_1 x) \\
  &= a_0 b_0 + (a_0 b_1 + a_1 b_0) x + a_1 b_1 x^2 \\
  &= (x^2 + 1) a_1 b_1 + ((a_0 b_0 - a_1 b_1) + (a_0 b_1 + a_1 b_0)x).
\end{align*}
Thus \(\phi(f(x) g(x)) = (a_0 b_0 - a_1 b_1, a_0 b_1 + a_1 b_0)\), this induces
a multiplicative structure \(\cdot: (\R \oplus \R)^2 \to \R \oplus \R\) defined
by
\[
(a_0, a_1) \cdot (b_0, b_1) \coloneq (a_0 b_0 - a_1 b_1, a_0 b_1 + a_1 b_0).
\]

Now, since \((\R \oplus \R, \cdot) \iso \CC\), by identifying \((a, b) \mapsto a
+ b i\), one obtains the following natural isomorphism
\[
\R[x]/(x^2 + 1) \iso \CC,
\]
hence we've constructed the complex numbers out of the ring \(\R[x]\).
\end{example}

\begin{example}
\label{exp:multiple-var-eval-isomorphism}
Let \(R\) be a \emph{commutative} ring and elements \(a_1, \dots, a_n \in
R\). Then there exists a canonical isomorphism of \emph{abelian groups}
\[
\frac{R[x_1, \dots, x_n]}{(x_1 - a_1, \dots, x_n - a_n)} \iso R.
\]

Denote \(S \coloneq \frac{R[x_1, \dots, x_n]}{(x_1 - a_1, \dots, x_n -
  a_n)}\). Consider the evaluation map \(\eval: S \to R\) given by
\([p(x_1, \dots, x_n)] \mapsto p(a_1, \dots, a_n)\), which is clearly a group
morphism. Moreover, the map \(\rho: R \to S\) mapping \(r \mapsto [r]\) is both
a group morphism and right inverse of the evaluation map,
\(\eval \circ \rho = \Id_R\). Since for all \(1 \leq j \leq n\) we have
\([x_j] = [a_j]\) as classes in the quotient ring \(S\), we find that, for any
given \(p(x_1,\dots, x_n) \in R[x]\)
\[
[p(x_1, \dots, x_n)] = [p(a_1, \dots, a_n)],
\]
implying in \(\rho \circ \eval = \Id_S\).
\end{example}

\subsection{Prime \& Maximal Ideals}

\begin{definition}[Prime and maximal ideal]
\label{def:prime-maximal-ideal}
Let \(R\) be a \emph{commutative} ring and \(\ideal{a}\) be a \emph{proper}
ideal of \(R\). We define the following:
\begin{enumerate}[(a)]\setlength\itemsep{0em}
\item If \(R/\ideal{a}\) is an \emph{integral domain}, we call \(\ideal{a}\) a
  \emph{prime ideal}.

\item If \(R/\ideal{a}\) is a \emph{field}, we call \(\ideal{a}\) a
  \emph{maximal ideal}.
\end{enumerate}
\end{definition}

\begin{example}
\label{exp:prime-maximal-principal-ideal-of-R[x]}
Given a commutative ring \(R\) and any monomial \(x - a \in R[x]\), then:
\begin{itemize}\setlength\itemsep{0em}
\item The ideal \((x - a)\) is prime if and only if \(R\) is an integral
  domain.
\item The ideal \((x - a)\) is maximal if and only if \(R\) is a field.
\end{itemize}
Such propositions are direct implications of the isomorphism
\(R[x]/(x - a) \iso R\) given by
\cref{prop:isomorphism-ring-quotient-by-monic-degree-d}.
\end{example}

Some may want to avoid the machinery of quotient rings when defining prime and
maximal ideals. Such ambition is palpable, the following proposition exposes
this old fashioned alternative.

\begin{proposition}
\label{prop:equivalent-prime-maximal-ideals}
Let \(\ideal{a}\) be a \emph{proper} ideal of a \emph{commutative} ring
\(R\) --- then:
\begin{enumerate}[(a)]\setlength\itemsep{0em}
\item The ideal \(\ideal{a}\) is \emph{prime} if and only if for every given
  \(a, b \in R\) such that \(a b \in \ideal{a}\), we have either \(a \in
  \ideal{a}\) or \(b \in \ideal{a}\).

\item The ideal \(\ideal{a}\) is \emph{maximal} if and only the only ideals
  that contain \(\ideal{a}\) is either \(\ideal{a}\) itself or \(R\).
\end{enumerate}
\end{proposition}

\begin{proof}
\begin{enumerate}\setlength\itemsep{0em}
\item The quotient ring \(R/\ideal{a}\) is an integral domain if and only if for
  all \(a + \ideal{a}, b + \ideal{a} \in R/\ideal{a}\) such that
  \((a + \ideal{a})(b + \ideal{a}) = \ideal{a}\) we have
  \(a + \ideal{a} = \ideal{a}\) or \(b + \ideal{a} = \ideal{a}\) --- which is
  equivalent to the requirement that either \(a \in \ideal{a}\) or
  \(b \in \ideal{b}\).

\item The quotient ring \(R/\ideal{a}\) is a field if and only if \(\ideal{a}\)
  and \(R/\ideal{a}\) are the only ideals of \(R/\ideal{a}\) --- that is, if
  \(\ideal{b}\) is an ideal of \(R\) with \(\ideal{a} \subseteq \ideal{b}\),
  then \(\ideal{b}\) is also an ideal of \(R/\ideal{a}\) (see
  \cref{prop:ideal-of-a-quotient}), thus either \(\ideal{b} = \ideal{a}\) or
  \(\ideal{b} = R/\ideal{a}\).
\end{enumerate}
\end{proof}

\begin{corollary}
\label{cor:maximal-implies-prime}
Maximal ideals are prime ideals.
\end{corollary}

\begin{proof}
Let \(R\) be a commutative ring and \(\ideal{m}\) be maximal. Let \(a, b \in R\)
be any elements such that \(a b \in \ideal{m}\), then consider the principal
ideals \((a)\) and \((b)\). Since \(\ideal{m}\) is maximal, such principal
ideals can be either \(R\) or \(\ideal{m}\) --- for the former case, one of them
equals \(1_R\), thus the other is contained in \(\ideal{m}\), on the other hand,
for the latter case the element whose ideal equal to \(\ideal{m}\) belongs to
\(\ideal{m}\). This shows that either \(a \in \ideal{m}\) or
\(b \in \ideal{m}\).
\end{proof}

\begin{proposition}
\label{prop:finite-quotient-prime-iff-maximal}
Let \(\ideal{a}\) be an ideal of a \emph{commutative} ring \(R\). If
\(R/\ideal{a}\) is a \emph{finite} ring, then \(\ideal{a}\) is \emph{prime} if
and only if it is \emph{maximal}.
\end{proposition}

\begin{proof}
Since a finite commutative ring is an integral domain if and only if it is a
field, the proposition follows --- see
\cref{prop:commutative-field-iff-integral-domain}.
\end{proof}

\begin{example}
\label{exp:Z/nZ-finite-quotient-prime-iff-maximal-iff-prime-number}
A good example of \cref{prop:finite-quotient-prime-iff-maximal} in action is the
case of the ring \(\Z\) and the principal ideals \((n)\) for
\(n \in \Z_{> 0}\). Since \(\Z/(n)\) is always finite, \((n)\) is a prime ideal
if and only if \((n)\) is maximal --- on the other hand, \((n)\) is maximal if
and only if in the case where \(n\) is a prime number.
\end{example}

\begin{definition}[Ring spectrum]
\label{def:ring-spectrum}
Given a \emph{commutative} ring \(R\), we define the following:
\begin{enumerate}[(a)]\setlength\itemsep{0em}
\item The collection of all \emph{prime ideals} of \(R\) is called the
  \emph{prime spectrum} of \(R\) and is denoted \(\Spec R\).

\item The collection of all \emph{maximal ideals} of \(R\) is called the
  \emph{maximal spectrum} of \(R\) and is denoted \(\Specm R\).
\end{enumerate}
\end{definition}

\begin{proposition}[Prime \& maximal ideals in PIDs]
\label{prop:PID-prime-iff-maximal}
Let \(R\) be a \emph{PID}, and let \(\ideal{a}\) be a \emph{non-zero} ideal of
\(R\). Then the ideal \(\ideal{a}\) is \emph{prime} if and only if it is
\emph{maximal}.
\end{proposition}

\begin{proof}
Suppose \(\ideal{a}\) is prime. Since \(R\) is PID, we can be certain that there
exists a non-zero element \(a \in R\) such that \(\ideal{a} = (a)\). Now,
consider \(\ideal{b} \coloneq (b)\) to be any ideal of \(R\) containing
\(\ideal{b}\), then in particular there exists \(q \in \R\) such that
\(a = b q\) --- since \(\ideal{a}\) is prime and \(bq \in \ideal{a}\) then
either \(b \in \ideal{a}\) or \(q \in \ideal{a}\):
\begin{itemize}\setlength\itemsep{0em}
\item For the case where \(b \in \ideal{a}\) then
  \(\ideal{b} \subseteq \ideal{a}\) --- thus from hypothesis that \(\ideal{b}\)
  contained \(\ideal{a}\) we conclude that \(\ideal{b} = \ideal{a}\).

\item If \(q \in \ideal{a}\), then there exists \(d \in R\) such that
  \(q = a d\), therefore \(a = b q = b (a d) = b (d a) = (b d) a\) --- since
  a PID is an integral domain and \(a \neq 0\), one can cancel \(a\) from both
  sides to obtain \(b d = 1\), thus \(b\) is a unit element of \(R\). In
  particular \(b d = 1 \in \ideal{b}\), thus \(\ideal{b} = R\).
\end{itemize}
By \cref{prop:equivalent-prime-maximal-ideals} we conclude that \(\ideal{a}\) is
maximal.
\end{proof}

\begin{example}[Prime ideals of \(k{[x]}\) are maximal]
\label{exp:prime-ideals-of-k[x]-are-maximal}
Let \(k\) be a field. Then non-zero prime ideals of \(k[x]\) are
maximal. Indeed, from \cref{prop:k[x]-is-PID} we know that \(k[x]\) is PID, thus
this is all but a consequence of \cref{prop:PID-prime-iff-maximal}.
\end{example}

\begin{proposition}[Maximal ideals in algebraically closed fields]
\label{prop:alg-closed-field-maximal-iff-(x-a)}
Let \(k\) be an \emph{algebraically closed field} and \(\ideal{a}\) be an ideal
of \(k[x]\). Then \(\ideal{a}\) is maximal if and only if there exists some
\(a \in k\) for which \(\ideal{a} = (x - a)\).
\end{proposition}

\begin{proof}
From \cref{exp:prime-maximal-principal-ideal-of-R[x]} we know that \((x - a)\)
is a maximal ideal of \(k[x]\), since \(k\) is a field.

Suppose \(\ideal{a}\) is maximal, since \(k[x]\) is a PID, it follows that there
exists an \(f(x) \in k[x]\) for which \(\ideal{a} = (f(x))\). Moreover, since
\(k\) is algebraically closed, there exists \(a \in R\) for which \(f(a) = 0\),
therefore \(f(x) \in (x - a)\) and then \(\ideal{a} \subseteq (x - a)\). From
the hypothesis that \(\ideal{a}\) is maximal, we conclude that either
\((x - a) = \ideal{a}\) or \((x - a) = k[x]\) --- notice that the latter cannot
be the case since the collection of constant polynomials is not contained in
\((x - a)\), thus \(\ideal{a} = (x - a)\).
\end{proof}

\begin{example}
\label{exp:bijection-compact-specm}
Let \(K\) be a \emph{compact} topological space and \(C(K)\) be the ring of
continuous maps \(K \to \R\), with addition and multiplication defined
point-wise. We have the following facts:
\begin{enumerate}[(a)]\setlength\itemsep{0em}
\item For every \(p \in K\), the collection
  \(\ideal{m}_p \coloneq \{f \in C(K) \colon f(p) = 0\}\) is a \emph{maximal
    ideal} of \(R\).

\item If \(f_1, \dots, f_n \in C(K)\) have no common zeros, then
  \((f_1, \dots, f_n) = R\).

\item For every maximal ideal \(\ideal{m} \in \Specm C(K)\), there exists a
  point \(p \in K\) for which \(\ideal{m} = \ideal{m}_p\).

\item If in addition \(K\) is Hausdorff, then there exists a \emph{bijective
    set-function} \(K \isoto \Specm C(K)\) given by \(p \mapsto \ideal{m}_p\).
\end{enumerate}
\end{example}

\begin{proof}
We now prove each one of these statements.
\begin{enumerate}[(a)]\setlength\itemsep{0em}
\item Consider any map \(f + \ideal{m}_p \in C(K)/\ideal{m}_p\) such that
  \(f(p) \neq a\) for some \(a \neq 0\) and therefore \(f \notin
  \ideal{m}_p\). Notice that the unit of \(C(K)/\ideal{m}_p\) is composed of
  every element \(g \in C(K)\) such that \(g(p) = 1\). In particular, since
  \(a\) is non-zero, then there exists an inverse \(a^{-1} \in \R\). If
  \(g \in C(K)\) is any map assuming \(g(p) = a^{-1}\), then
  \((f + \ideal{m}_p) (g + \ideal{m}_p) = 1 + \ideal{m}_p\).

\item Since \(f_1, \dots, f_n\) share no zeros, the function
  \(f \coloneq \sum_{j=1}^n f_j^2\) must be strictly positive, therefore one can
  define a continuous map \(g \in C(K)\) as \(g(x) \coloneq \frac{1}{f(x)}\) for
  all \(x \in K\). Therefore \(g f = 1\), which implies in
  \(1 \in (f_1, \dots, f_n)\) --- where we denote by \(1\) the constant map
  assuming value \(1\).

\item Let \(\ideal{m}\) be maximal and suppose, for the sake of contradiction,
  that for all \(p \in K\) we have \(\ideal{m} \neq \ideal{m}_p\). Fix any
  \(p \in K\) and let \(f_p \in \ideal{m}\) be such that \(f_p(p) \neq 0\). By
  the continuity of \(f_p\) there exists a neighbourhood \(U_p \subseteq K\) of
  \(p\) such that \(f(x) \neq 0\) for all \(x \in U_p\). Let
  \(\mathcal{U} \coloneq \{ U_{p}\}_{p \in K}\) be an open cover of \(K\), where
  each neighbourhood \(U_p\) is associated with a map \(f_p\) as described
  above. Since \(K\) is compact, there exists a finite collection
  \(U_{p_1}, \dots, U_{p_n} \in \mathcal{U}\) covering \(K\). From construction,
  we obtain \((f_{p_1}, \dots, f_{p_n}) \subseteq \ideal{m}\). Notice that the
  maps \(f_{p_j}\) share no common zero --- if \(q \in K\) is such that
  \(f_{p_j}(q) = 0\) for each \(1 \leq j \leq n\), then
  \(q \notin \bigcup_{j=1}^n U_{p_j}\), which is a contradiction --- therefore
  \(1 \in \ideal{m}\) and \(\ideal{m} = C(K)\). We conclude that there must
  exist \(p \in K\) such that \(\ideal{m} = \ideal{m}_p\).
\todo[inline]{Prove last item, requires Urysohn's lemma}
\end{enumerate}
\end{proof}

\begin{definition}[Krull dimension]
\label{def:krull-dimension}
The \emph{Krull dimension} of a commutative ring \(R\) is defined as the length
of the longest chain of prime ideals in \(R\).
\end{definition}

%%% Local Variables:
%%% mode: latex
%%% TeX-master: "../../deep-dive"
%%% End:

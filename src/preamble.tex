\documentclass[12pt, reqno, a4]{memoir}
\usepackage{subfiles}

\usepackage[english]{babel}

\usepackage{layout}
\usepackage{afterpage}
\usepackage[
    asymmetric,
    textheight     = 673pt,
    marginparsep   = 7pt,
    footskip       = 27pt,
    hoffset        = 0pt,
    paperwidth     = 597pt,
    textwidth      = 452pt,
    marginparwidth = 101pt,
    voffset        = 0pt,
    paperheight    = 845pt,
]{geometry}

\newcommand{\changegeometry}{\newgeometry{includehead,headheight=89pt}%
    \afterpage{\aftergroup\restoregeometry}%
}

% Math stuff: do not mess with the ordering!
\usepackage{mathtools}
\usepackage{amsthm}
\usepackage{amssymb}
\usepackage{stmaryrd}
\usepackage{tikz-cd}

% Font
\usepackage[no-math]{newpxtext}
\usepackage{newpxmath}

% Utilities
\usepackage{enumerate}
\usepackage{todonotes}
\usepackage{graphicx}

% Color
\usepackage{xcolor}
\definecolor{brightmaroon}{rgb}{0.76, 0.13, 0.28}

% References
\usepackage{hyperref}
\hypersetup{
    colorlinks = true,
    allcolors  = brightmaroon,
}
\usepackage[capitalize,nameinlink]{cleveref}
\usepackage[
    backend = biber,
    style   = alphabetic,
]{biblatex}
\addbibresource{src/bibliography.bib}

% Table of contents: show subsections
\setcounter{tocdepth}{2}

\linespread{1.05}
\vfuzz=14pt % No more vbox errors all over the place

%%%%%%%%%%%%%%%%%%%%%%%%%%%%%%%%%%%%%%%%%%%%%%%%%%%%%%%%%%%%%%%%%%%%%%%%%%%%%%%
% ** Environments **

\theoremstyle{definition} % Plain text font for all envs
\newtheorem{theorem}{Theorem}[section]
\newtheorem{proposition}[theorem]{Proposition}
\newtheorem{lemma}[theorem]{Lemma}
\newtheorem{corollary}[theorem]{Corollary}
\newtheorem{axiom}[theorem]{Axiom}
\newtheorem{definitionImpl}[theorem]{Definition}
\newtheorem{remarkImpl}[theorem]{Remark}
\newtheorem{exampleImpl}[theorem]{Example}
\newtheorem{notationImpl}[theorem]{Notation}

\newenvironment{definition}
  {\begin{definitionImpl}}
  {\par\noindent\hfill\(\clubsuit\)\end{definitionImpl}}
\newenvironment{remark}
  {\begin{remarkImpl}}
  {\par\noindent\hfill\(\spadesuit\)\end{remarkImpl}}
\newenvironment{example}
  {\begin{exampleImpl}}
  {\par\noindent\hfill\(\spadesuit\)\end{exampleImpl}}
\newenvironment{notation}
  {\begin{notationImpl}}
  {\par\noindent\hfill\(\spadesuit\)\end{notationImpl}}


%%%%%%%%%%%%%%%%%%%%%%%%%%%%%%%%%%%%%%%%%%%%%%%%%%%%%%%%%%%%%%%%%%%%%%%%%%%%%%%
% ** symbols **

\renewcommand{\qedsymbol}{\(\natural\)}  % QED symbol at the end of proofs
\renewcommand{\leq}{\leqslant}           % Less than or equal to
\renewcommand{\geq}{\geqslant}           % Greater than or equal to
\renewcommand{\setminus}{\smallsetminus} % Negation operation for sets
\renewcommand{\preceq}{\preccurlyeq}     % Partial order 'less than or equal to'

% ':' for maps and '\colon' for relations on collections
\DeclareMathSymbol{:}{\mathpunct}{operators}{"3A}
\let\colon\relax
\DeclareMathSymbol{\colon}{\mathrel}{operators}{"3A}

% Disjoint unions over sets
\newcommand{\disj}{\amalg}     % Disjoint union
\newcommand{\bigdisj}{\coprod} % Indexed disjoint union

% Complex stuff
\newcommand{\img}{\text{i}}     % Imaginary unit
\newcommand{\cconj}{\overline}  % Complex conjugate
\DeclareMathOperator{\Log}{Log} % Complex logarithm
\DeclareMathOperator{\Real}{Re} % Real part
\DeclareMathOperator{\Img}{Img} % Imaginary part

% Common maps throughout
\DeclareMathOperator{\const}{cons} % Constant map
\DeclareMathOperator{\mul}{mul}    % Multiplication map

% Implication
\renewcommand{\implies}{\Rightarrow}
\renewcommand{\impliedby}{\Leftarrow}

%%%%%%%%%%%%%%%%%%%%%%%%%%%%%%%%%%%%%%%%%%%%%%%%%%%%%%%%%%%%%%%%%%%%%%%%%%%%%%%
% ** Category-theoretical utilities **

% Set arrow tip to that of newpxmath
\tikzset{>=Straight Barb, commutative diagrams/arrow style=tikz}

% Isomorphism symbol, i.e. \(C \iso D\)
\newcommand{\iso}{\simeq}

% Arrow holding a symbol above it
\newcommand{\xdashrightarrow}[2][]{\ext@arrow 0359\rightarrowfill@@{#1}{#2}}

\newcommand{\To}{\Rightarrow}         % Alias for Rightarrow
\newcommand{\mono}{\rightarrowtail}   % Monomorphism
\newcommand{\epi}{\twoheadrightarrow} % Epimorphism
\newcommand{\unique}{\to}             % Unique morphism
\newcommand{\arrowiso}{\iso}          % Isomorphism
\newcommand{\dis}{\iso}               % Isomorphism in tikzcd diagrams
\newcommand{\nat}{\Rightarrow}        % Natural transformation
\newcommand{\emb}{\hookrightarrow}    % Embedding
\newcommand{\adj}{\rightleftarrows}   % Adjunction pair i.e. \(F: C \adj D :G\)
\newcommand{\Adj}{\dashv}             % Adjunction pair i.e. \(F \Adj G\)
\newcommand{\para}{\rightrightarrows} % Parallel arrows

% Isomorphism arrows
\newcommand{\isoto}{\xrightarrow{\raisebox{-.6ex}[0ex][0ex]{\(\arrowiso\)}}}         % Isomorphism arrow
\newcommand{\isounique}{\xdashrightarrow{\raisebox{-.6ex}[0ex][0ex]{\(\arrowiso\)}}} % Unique isomorphism arrow
\newcommand{\isonat}{\xRightarrow{\raisebox{-.6ex}[0ex][0ex]{\(\arrowiso\)}}}        % Natural (functorial) isomorphism

% Limits
\DeclareMathOperator{\Lim}{lim}     % Limit
\DeclareMathOperator{\Colim}{colim} % Colimit
\DeclareMathOperator{\Eq}{eq}       % Equalizer
\DeclareMathOperator{\Coeq}{coeq}   % Coequalizer

%%%%%%%%%%%%%%%%%%%%%%%%%%%%%%%%%%%%%%%%%%%%%%%%%%%%%%%%%%%%%%%%%%%%%%%%%%%%%%%
% ** Common collections **

\newcommand{\Z}{\mathbf{Z}}  % Integers
\newcommand{\N}{\mathbf{N}}  % Naturals (with 0, obviously)
\newcommand{\Q}{\mathbf{Q}}  % Rationals
\newcommand{\CC}{\mathbf{C}} % Complex
\newcommand{\R}{\mathbf{R}}  % Real

\newcommand{\FF}{\mathbf{F}}

\renewcommand{\emptyset}{\varnothing} % Empty set

% Distinct spaces that don't appear often in the notes (yet?)
\newcommand{\Uhs}{\mathbf{H}}  % Upper half space
\newcommand{\Proj}{\mathbf{P}} % Projective space

%%%%%%%%%%%%%%%%%%%%%%%%%%%%%%%%%%%%%%%%%%%%%%%%%%%%%%%%%%%%%%%%%%%%%%%%%%%%%%%
% ** Categories **

% Font for categories
\newcommand{\cat}{\texttt}
\newcommand{\catfont}{\texttt}
\newcommand{\mcat}{\mathcal} % Model category font

% Opposite category
\newcommand{\op}{\mathrm{op}}

% Operators
\DeclareMathOperator{\Hom}{Mor}   % Morphisms
\DeclareMathOperator{\Mono}{Mono} % Monomorphisms
\DeclareMathOperator{\Epi}{Epi}   % Epimorphisms
\DeclareMathOperator{\Fct}{Fct}   % Functors
\DeclareMathOperator{\Obj}{Obj}   % Objects
\DeclareMathOperator{\Mor}{Mor}   % Morphisms, again
\DeclareMathOperator{\End}{End}   % Endomorphisms
\DeclareMathOperator{\Aut}{Aut}   % Automorphisms
\DeclareMathOperator{\Id}{id}     % Identity
\DeclareMathOperator{\im}{im}     % Image
\DeclareMathOperator{\coim}{coim} % Coimage
\DeclareMathOperator{\dom}{dom}   % Domain
\DeclareMathOperator{\codom}{cod} % Codomain
\DeclareMathOperator{\supp}{supp} % Support

% Yoneda embedding
\font\maljapanese=dmjhira at 2ex
\newcommand{\yo}{\maljapanese \char"48}

% %%%%%%%%%%%%%%%%%
% Common categories
% %%%%%%%%%%%%%%%%%

\newcommand{\Set}{{\catfont{Set}}}                     % Sets
\newcommand{\FinSet}{{\catfont{FinSet}}}               % Finite sets
\newcommand{\pSet}{{\catfont{pSet}}}                   % Pointed sets
\newcommand{\tOrd}{{\catfont{tOrd}}}                   % Totally ordered sets
\newcommand{\Vect}{{\catfont{Vect}}}                   % Vector spaces
\newcommand{\FinVect}{{\catfont{FinVect}}}             % Finite vector spaces
\newcommand{\TVect}{{\catfont{TVect}}}                 % Topological vector spaces
\newcommand{\Ban}{{\catfont{Ban}}}                     % Banach spaces
\newcommand{\Grp}{{\catfont{Grp}}}                     % Groups
\newcommand{\Ab}{{\catfont{Ab}}}                       % Abelian groups
\newcommand{\GSet}[1]{{{#1}\text{-}\Set}}              % G-sets
\newcommand{\Grpd}{{\catfont{Grpd}}}                   % Groupoids
\newcommand{\Mon}{{\catfont{Mon}}}                     % Monoidal category
\newcommand{\coMon}{{\catfont{coMon}}}                 % coMonoidal category
\newcommand{\rActMon}{{\catfont{rActMon}}}             % right action category
\newcommand{\lActMon}{{\catfont{lActMon}}}             % left action category
\newcommand{\BrMonCat}{{\catfont{BrMonCat}}}           % Cat of braided monoidal cats
\newcommand{\Graph}{{\catfont{Graph}}}                 % Graphs
\newcommand{\SimpGraph}{{\catfont{sGraph}}}            % Simple graphs
\newcommand{\ProfCol}{{\catfont{Prof}(\Col)}}          % C-profile category
\newcommand{\Rng}{{\catfont{Ring}}}                    % Rings
\newcommand{\cRng}{{\catfont{CRing}}}                  % Commutative rings
\newcommand{\rMod}[1]{{\texttt{Mod}_{#1}}}             % Right modules
\newcommand{\lMod}[1]{{{}_{#1}\catfont{Mod}}}          % Left modules
\newcommand{\Mod}[1]{{#1\text{-}\catfont{Mod}}}        % Modules over comm. ring
\newcommand{\Alg}[1]{{#1\text{-}\catfont{Alg}}}        % Algebras
\newcommand{\cAlg}[1]{{#1\text{-}\catfont{CAlg}}}      % Commutative algebras
\newcommand{\Cat}{{\catfont{Cat}}}                     % Small categories
\newcommand{\CAT}{{\catfont{CAT}}}                     % Big categories
\newcommand{\UCat}{{\mathcal{U}\text{-}\catfont{Cat}}} % U-Categories
\newcommand{\Psh}[1]{{\catfont{Psh}({#1})}}            % Category of presheaves
\newcommand{\comma}{\downarrow}                        % Comma category separator
\DeclareMathOperator{\El}{El}                          % Category of elements

%%%%%%%%%%%%%%%%%%%%%%%%%%%%%%%%%%%%%%%%%%%%%%%%%%%%%%%%%%%%%%%%%%%%%%%%%%%%%%%
% ** algebra **

% Fonts
\newcommand{\lie}[1]{\mathfrak{#1}}   % Lie structures
\newcommand{\ideal}[1]{\mathfrak{#1}} % Ring ideal

% Operators
\DeclareMathOperator{\rank}{rank}            % Rank
\DeclareMathOperator{\coker}{coker}          % Cokernel
\DeclareMathOperator{\codim}{codim}          % Codimension
\DeclareMathOperator{\Tr}{tr}                % Trace
\DeclareMathOperator{\Sym}{Sym}              % Symmetric space
\DeclareMathOperator{\Alt}{Alt}              % Alternating space
\DeclareMathOperator{\Ann}{Ann}              % Annihilator
\DeclareMathOperator{\Char}{char}            % Characteristic of a ring
\DeclareMathOperator{\Span}{span}            % Linear span
\DeclareMathOperator{\Inn}{Inn}              % Inner automorphisms
\DeclareMathOperator{\Spec}{Spec}            % Prime spectrum
\DeclareMathOperator{\Specm}{Spec_m}         % Maximal spectrum
\DeclareMathOperator{\Rees}{Rees}            % Rees algebra
\DeclareMathOperator{\Frac}{Frac}            % Field of fractions
\DeclareMathOperator{\torsion}{\texttt{tor}} % Torsion
\DeclareMathOperator{\Centre}{Z}             % Ring centre
\DeclareMathOperator{\eval}{eval}            % Evaluation
\DeclareMathOperator{\sign}{sign}            % Sign of value
\DeclareMathOperator{\Orb}{Orb}              % Group orbit
\DeclareMathOperator{\Stab}{Stab}            % Group stabilizer
\newcommand{\laction}{\circlearrowright}     % Left group action
\newcommand{\raction}{\circlearrowleft}      % Right group action

% Special linear spaces
\DeclareMathOperator{\Mat}{Mat}
\DeclareMathOperator{\GL}{GL}
\DeclareMathOperator{\SL}{SL}
\DeclareMathOperator{\PSL}{PSL}
\DeclareMathOperator{\SO}{SO}
\DeclareMathOperator{\SU}{SU}
\DeclareMathOperator{\Unit}{U}
\DeclareMathOperator{\Orth}{O}

% Postfix symbols
\DeclareMathOperator{\Transp}{\top}  % Matrix transposition, i.e. \(M^{\Transp}\)
\newcommand{\unit}{\times}           % Group of units, i.e. \(R^{\unit}\)

%%%%%%%%%%%%%%%%%%%%%%%%%%%%%%%%%%%%%%%%%%%%%%%%%%%%%%%%%%%%%%%%%%%%%%%%%%%%%%%
% ** Topology **

% Distinct structures
\newcommand{\splx}{\Delta}                  % Standard simplex
\newcommand{\splxtop}{\Delta_{\text{top}}}  % Standard topological simplex

% Homotopy equivalence relations
\newcommand{\simht}{\sim_{\text{h}}}                    % Homotopy between maps
\newcommand{\simhtrel}[1]{\sim_{\text{rel }{#1}}}       % Relative htpy
\newcommand{\isoht}{\iso_{\text{h}}}                    % Homotopy equivalence
\newcommand{\isotoht}{\isoto_{\text{h}}}                % Homotopy equivalence map
\newcommand{\htpy}{\Rightarrow}                         % Htpy arrow
\newcommand{\htpyrel}[1]{\Rightarrow_{\text{rel }{#1}}} % Relative htpy arrow

% %%%%%%%%%%
% Categories
% %%%%%%%%%%

\let\Top\relax
\newcommand{\Top}{{\catfont{Top}}}                       % Topological spaces
\newcommand{\wHTop}{{\catfont{wH}\text{-}\catfont{Top}}} % Weak Hausdorff
\newcommand{\kTop}{{k\text{-}\catfont{Top}}}             % k-spaces
\newcommand{\cgTop}{{\catfont{cgTop}}}                   % Compactly generated
\newcommand{\pTop}{{\catfont{Top}^{*}}}                  % Pointed top spaces
\newcommand{\bpTop}{{\catfont{Top}^{*/}}}                % Base point top spaces
\newcommand{\Ho}[1]{{\catfont{Ho}(#1)}}                  % Homotopy category
\newcommand{\HoTop}{{\catfont{Ho}(\catfont{Top})}}       % Classical Homotopy cat
\newcommand{\Splx}{{\mathbf{\Delta}}}                    % Simplex category
\newcommand{\sSet}{{\catfont{sSet}}}                     % Simplicial sets
\newcommand{\Simp}[1]{{\catfont{Simp}(#1)}}              % Simplicial category
\newcommand{\CoSimp}[1]{{\catfont{CoSimp}(#1)}}          % Cosimplicial category

% %%%%%%%%%
% Operators
% %%%%%%%%%

% Function-like
\DeclareMathOperator{\Sing}{Sing}           % Singular complex functor
\DeclareMathOperator{\Nondeg}{nd}           % Non-degenerate simplices
\newcommand{\disc}{\text{disc}}             % discrete
\DeclareMathOperator{\face}{\partial}       % Face map and boundary operator
\DeclareMathOperator{\Cycle}{Z}             % Cycles in singular chains
\DeclareMathOperator{\Bdry}{B}              % Boundaries in singular chains
\DeclareMathOperator{\Hmlg}{H}              % Homology
\DeclareMathOperator{\rHmlg}{\widetilde{H}} % Reduced homology
\DeclareMathOperator{\Sk}{sk}               % Skeleton
\DeclareMathOperator{\Cell}{Cell}           % Collection of cells
\DeclareMathOperator{\Sub}{Sub}             % Collection of substructures of an object

% Set operators
\DeclareMathOperator{\Cl}{Cl}               % Closure
\DeclareMathOperator{\Bd}{\partial}         % Boundary
\DeclareMathOperator{\Int}{Int}             % Interior
\DeclareMathOperator{\Ext}{Ext}             % Exterior
\newcommand{\Compl}[1]{{#1}^{\text{c}}}     % Complement

% Prefix operators
\newcommand{\lift}{\widehat}                % Lifting of path and stuff

% Operators for attaching spaces
\newcommand{\att}{\amalg}                   % Disjoint union
\newcommand{\bigatt}{\coprod}               % Indexed disjoint union

% Functors on topological spaces
\DeclareMathOperator{\Disc}{Disc}           % Discrete topology: Set -> Top
\DeclareMathOperator{\Cone}{Cone}           % Cone
\DeclareMathOperator{\Cyl}{Cyl}             % Cylinder
\DeclareMathOperator{\Susp}{S}              % Suspension
\DeclareMathOperator{\rSusp}{\Sigma}        % Reduced suspension
\DeclareMathOperator{\Path}{Path}           % Path object
\DeclareMathOperator{\Loop}{\Omega}         % Loop space
\DeclareMathOperator{\Eval}{eval}           % Evaluation map
\DeclareMathOperator{\curry}{curry}         % Currying a map
\DeclareMathOperator{\uncurry}{uncurry}     % Currying a map
\DeclareMathOperator{\co}{co}               % Compact open co(K, U)

%%%%%%%%%%%%%%%%%%%%%%%%%%%%%%%%%%%%%%%%%%%%%%%%%%%%%%%%%%%%%%%%%%%%%%%%%%%%%%%
% ** Differentiable structures **

% Categories
\newcommand{\Bun}{{\catfont{Bun}}}       % Bundle category
\newcommand{\VecBun}{{\catfont{VecBun}}} % Vector Bundle category
\newcommand{\Man}{{\catfont{Man}}}       % Manifolds

% Operators
\newcommand{\Diff}{\mathrm{D}}
\newcommand{\diff}{\mathrm{d}}                 % Differential
\DeclarePairedDelimiter{\norm}{\lVert}{\rVert} % Norm of a vector
\DeclareMathOperator{\grad}{grad}              % Gradient
\DeclareMathOperator{\Hess}{Hess}              % Hessian
\DeclareMathOperator{\Jac}{Jac}                % Jacobian
\DeclareMathOperator{\Curl}{Curl}              % Curl
\DeclareMathOperator{\VecField}{\mathfrak{X}}  % Space of all vector fields
\newcommand{\trans}{\pitchfork}                % Transversality
\DeclareMathOperator{\Vol}{vol}                % Set volume
\DeclareMathOperator{\Mesh}{mesh}              % Set mesh
\DeclareMathOperator{\Grass}{Grass}            % Grassmann variety
\DeclareMathOperator{\Stie}{Stie}              % Stiefel variety

% Distinct structures
\DeclareMathOperator{\zerosec}{Zero} % Vector bundle zero section

%%%%%%%%%%%%%%%%%%%%%%%%%%%%%%%%%%%%%%%%%%%%%%%%%%%%%%%%%%%%%%%%%%%%%%%%%%%%%%%
% ** Graphs **

% Fonts
\newcommand{\Col}{\mathfrak{C}}       % Wheel graph colors
\newcommand{\prof}[1]{\underline{#1}} % Wheel graph prof

% Operators
\DeclareMathOperator{\Edge}{Edge}                      % Set of edges
\DeclareMathOperator{\Vertex}{Vert}                    % Set of vertices
\DeclareMathOperator{\Circ}{circ}                      % Circumference
\DeclareMathOperator{\diam}{diam}                      % Diameter
\newcommand{\emptygraph}{\varnothing}                  % Empty graph
\DeclarePairedDelimiterX{\size}[1]{\lVert}{\rVert}{#1} % Size of a graph

%%%%%%%%%%%%%%%%%%%%%%%%%%%%%%%%%%%%%%%%%%%%%%%%%%%%%%%%%%%%%%%%%%%%%%%%%%%%%%%
% ** machine learning **

% Operators
\DeclareMathOperator{\Prob}{\mathbb{P}}   % Probability
\DeclareMathOperator{\Expect}{\mathbb{E}} % Expectation
\newcommand{\Mean}{\mu}                   % Mean
\DeclareMathOperator{\Cov}{Cov}           % Covariance
\DeclareMathOperator{\Var}{Var}           % Variance
\DeclareMathOperator{\Correlation}{Corr}  % Correlation
\DeclareMathOperator{\stdev}{\sigma}      % standard deviation
\DeclareMathOperator{\ndist}{\mathcal{N}} % Normal distribution
\DeclareMathOperator{\precision}{\beta}   % Precision term
\DeclareMathOperator*{\argmax}{arg\,max}  % Argmax
\DeclareMathOperator*{\argmin}{arg\,min}  % Argmin
\DeclareMathOperator{\vcdim}{VCdim}       % VC-dimension
\DeclareMathOperator{\Growth}{Growth}     % Growth function
\DeclareMathOperator{\Quantile}{Quant}    % Quantile

%%%%%%%%%%%%%%%%%%%%%%%%%%%%%%%%%%%%%%%%%%%%%%%%%%%%%%%%%%%%%%%%%%%%%%%%%%%%%%%
% ** MACROS END HERE **
%%%%%%%%%%%%%%%%%%%%%%%%%%%%%%%%%%%%%%%%%%%%%%%%%%%%%%%%%%%%%%%%%%%%%%%%%%%%%%%

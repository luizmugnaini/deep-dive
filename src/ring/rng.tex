\section{Introduction}

%
\begin{definition}[Ring]
\label{def:ring}
A ring \(R\) is a set endowed with an additive structure \(+: R \times R \to R\)
such that \((R, +)\) is an \emph{abelian group}, moreover, \(R\) is also endowed
with a second multiplicative structure \(\cdot: R \times R \to R\), making
\((R, \cdot)\) into a monoid --- that is, it is both associative and has a
two-sided identity:
\begin{itemize}\setlength\itemsep{0em}
\item Given \(r, s, t \in R\), we have \(r (s t) = (r s) t\).

\item There exists a unitary element \(1_R \in R\) for which all \(r \in R\) is
  such that
  \[
    r 1_R = 1_R r = r.
  \]
\end{itemize}
Furthermore, we impose that the product structure is distributive over the
additive structure, that is, given any three \(r, t, s \in R\), one has
\[
  (r + t) s = r s + t s \text{ and } r (t + s) = r t + r s.
\]
\end{definition}
%

%
\begin{remark}
\label{rem:def-ring}
Notice that we impose the following in the definition of a ring --- which may
vary from author to author --- the ring is always associative and has a unity.
\end{remark}
%

%
\begin{definition}
\label{def:commutative-ring}
A ring \(R\) is said to be \emph{commutative} if for all \(r, s \in R\) we have
\(r s = s r\).
\end{definition}
%

%
\begin{corollary}
\label{cor:ring-unique-identity}
Given a ring \(R\), its identity is unique.
\end{corollary}
%

%
\begin{proof}
Let \(1, 1' \in R\) be both identities of \(R\), then \(1 1' = 1\) but \(1 1' =
1'\), thus \(1 = 1'\).
\end{proof}
%

%
\begin{corollary}
\label{cor:zero-element-ring}
Given a ring \(R\), we have for all \(r \in R\) that \(0_R r = r 0_R = 0_R\).
\end{corollary}
%

%
\begin{proof}
Note that for both cases, since \(0 = 0 + 0\), we are able to write the
following
\begin{align*}
  r \cdot 0 &= r \cdot (0 + 0) = r \cdot 0 + r \cdot 0, \\
  0 \cdot r &= (0 + 0) \cdot r = 0 \cdot r + 0 \cdot r.
\end{align*}
Further, since \((R, +)\) is a group, we evoke the cancellation to obtain
\(r \cdot 0 = 0\) and \(0 \cdot r = 0\).
\end{proof}
%

%
\begin{corollary}
\label{cor:ring-additive-inverse}
Let \(R\), then \((-1) \cdot r = r \cdot (-1) = -r\).
\end{corollary}
%

%
\begin{proof}
Notice that since \(1 \cdot r = r \cdot 1 = r\) we may write
\begin{align*}
  r + (-1) \cdot r &= (1 - 1) \cdot r = 0 \cdot r = 0, \\
  r + r \cdot (-1) &= r \cdot (1 - 1) = r \cdot 0 = 0.
\end{align*}
Thus we arrive at the desired conclusion.
\end{proof}
%

\subsection{Examples of Rings}

%
\begin{example}[Trivial ring]
\label{exp:trivial-ring}
The trivial group \(*\) can be endowed with a ring structure by imposing that
\(* \cdot * = *\) and \(* + * = *\), so that \((*, +, \cdot)\) is a ring --- we
call such ring \emph{trivial} or \emph{zero-ring}. Moreover, in such ring we
have the equality \(0 = 1\).
\end{example}
%

%
\begin{corollary}
\label{cor:zero-ring-condition}
A ring \(R\) is a zero-ring if and only if \(0 = 1\).
\end{corollary}
%

%
\begin{proof}
If \(R\) is a zero-ring, it's obvious that \(0 = 1\). On the other hand, assume
that \(R\) is a ring such that \(0 = 1\), then if \(r \in R\) is any element, we
have \(r = 1 \cdot r = 0 \cdot r = 0\), thus \(R\) is indeed a zero ring.
\end{proof}
%

%
\begin{example}[Power set ring]
\label{exp:power-set-ring}
Let \(S\) be any set and \(2^S\) be its power set. If we define additive and
multiplicative structures in \(2^S\) given by
\(A + B \coloneq (A \cup B) \setminus (A \cap B)\) and
\(A \cdot B \coloneq A \cap B\), for all \(A, B \in 2^S\), then
\((2^S, +, \cdot)\) is a ring.

In order to prove that, we first show that \((2^S, +)\) forms an abelian
group. Since \(A \cup B = B \cup A\) and \(A \cap B = B \cap A\), then clearly
\(A + B = B + A\). Moreover, the empty set \(\emptyset\) is the additive
identity, since \(A \cup \emptyset = A\) and \(A \cap \emptyset = \emptyset\)
and \(A \setminus \emptyset = A\) --- thus \(A + \emptyset = A\). Every set is
its own inverse: notice that \(A \cup A = A\) and \(A \cup A = A\), hence
\(A + A = A \setminus A = \emptyset\). We therefore conclude that \((2^S, +)\)
is an abelian group.

Now we show that \((2^S, \cdot)\) is a monoid with distributivity over
addition. Let \(A, B, C \in 2^S\) be any three sets. For associativity we have
\[
  (A \cdot B) \cdot C
  = (A \cap B) \cap C
  = A \cap (B \cap C)
  = A \cdot (B \cdot C).
\]
The unitary element is \(S\) itself, since \(A \cap S = S \cap A = A\) for all
\(A \in 2^S\). Moreover,
\begin{align*}
  (A + B) \cdot C
  &= [(A \cup B) \setminus (A \cap B)] \cap C \\
  &= [(A \cap C) \cup (B \cap C)] \setminus [(A \cap C) \cap (B \cap C)] \\
  &= A \cdot C + B \cdot C.
\end{align*}
We can analogously show the same for
\(A \cdot (B + C) = A \cdot B + A \cdot C\).
\end{example}
%

%
\begin{example}
\label{exp:function-ring}
Let \(R\) be a ring and \(S\) be a set. The set of set-functions \(R^S\), of the
form \(S \to R\), together with the pointwise addition and multiplication make
\(R^S\) into a ring.
\end{example}
%

%
\begin{example}
\label{exp:square-matrices-ring}
Let \(M_n(R)\) denote the collection of all \(n \times n\) matrices with entries
in the ring \(R\). If we endow \(M_n(R)\) with component-wise addition and matrix
multiplication, then \((M_n(R), +, \cdot)\) is a ring.
\end{example}
%

\subsection{Zero-Divisors}

%
\begin{definition}[Zero-divisor]
\label{def:zero-divisor}
Let \(R\) be a ring. We say that \(a \in R\) is a \emph{left-zero-divisor} if
there exists \(b \neq 0\) in \(R\) for which \(a b = 0\). On the other hand,
\(a\) is said to be a \emph{right-zero-divisor} if \(b a = 0\).
\end{definition}
%

%
\begin{remark}
\label{rem:zero-ring-only-no-zero-divisors}
The zero-ring is the \emph{only} ring with \emph{no} zero-divisors.

If \(R\) is a non-zero ring, then \(0 \in R\) and since \(0\) is clearly a
zero-divisor, \(R\) has at least one zero-divisor. On the other hand, if \(*\)
is a zero-ring, then zero is the only element of \(*\) --- thus we have no
element different than zero for there to be a zero-divisor.
\end{remark}
%

%
\begin{proposition}
\label{prop:no-zero-divisor-injective-multiplication}
Let \(R\) be a ring. An element \(a \in R\) is \emph{not} a left-zero-divisor
(or right-zero-divisor) if and only if the map \(R \to R\) given by the
\emph{left-multiplication} by \(a\) (or right-multiplication) is
\emph{injective}.
\end{proposition}
%

%
\begin{proof}
We only concern ourselves with the left case, the right case is clearly
analogous. Let \(a \in R\) be a non-left-zero-divisor, then if \(a b = a c\) for
some \(b, c \in R\), then \(a b - a c = a (b - c) = 0\) --- and, since \(a\) is
not a zero-divisor, then necessarily \(b - c = 0\), that is \(b = c\).

On the other hand, if \(a\) is a left-zero-divisor then let \(b \neq 0\) be any
element of \(R\) such that \(a b = 0\). Notice then that both \(b\) and zero
have the same image under the multiplication map, thus \(R \to R\) is definitely
not injective.
\end{proof}
%

%
\begin{remark}
\label{rem:no-zero-divisor-is-cancellation-in-disguise}
Notice that \cref{prop:no-zero-divisor-injective-multiplication} simply states
that for an element to be a zero-divisor the cancellation law must be true,
that is, \(a \in R\) is not a zero-divisor if \(a b = a c\) implies \(b = c\)
for every \(b, c \in R\).
\end{remark}
%

%
\begin{definition}[Domain]
\label{def:domain}
An object \(R\) is said to be a \emph{domain} if \(R\) is non-zero ring and for
every two elements \(a, b \in R\) such that \(a b = 0\) we must have either
\(a = 0\) or \(b = 0\).
\end{definition}
%

%
\begin{definition}[Integral domain]
\label{def:integral-domain}
An object \(R\) is said to be an \emph{integral domain} if \(R\) is a non-zero
commutative ring such that, for all \(a, b \in R\) such that \(a b = 0\), then
either \(a = 0\) or \(b = 0\).
\end{definition}
%

That is, in an integral domain \emph{every non-zero} element is a
non-zero-divisor. Classic examples of integral domains are \(\Z, \Q, \R\) and
\(\CC\).

%
\begin{corollary}
\label{cor:cancellation-integral-domain}
In an integral domain \(R\), if \(r, a, b \in R\) are such that \(r a = r b\)
and \(r \neq 0\), then \(a = b\) --- that is, cancellation by non-zero elements
holds.
\end{corollary}
%

%
\begin{proof}
This is immediate from \cref{prop:no-zero-divisor-injective-multiplication}.
\end{proof}
%


%
\begin{corollary}
\label{cor:idempotent-in-integral-domain}
If \(R\) is an integral domain, an element \(x \in R\) is such that \(x^2 = 1\),
if and only if \(x \in \{-1, 1\}\).
\end{corollary}
%

%
\begin{proof}
If \(x \in \{-1, 1\}\), clearly \(x^2 = 1\), since this is true for any given
ring. For the other implication, suppose now that \(x^2 = 1\), then \(x^2 - 1 =
(x + 1) (x - 1) = 0\) but since \(R\) is an integral domain, either \(x = 1\) or
\(x = -1\).
\end{proof}
%

\subsubsection{Nilpotent Elements}

%
\begin{definition}[Nilpotent]
\label{def:nilpotent}
Let \(R\) be a ring. An element \(a \in R\) is said to be \emph{nilpotent} if
there exists some \(n \in \Z\) such that \(a^n = 0\).
\end{definition}
%

%
\begin{lemma}
\label{lem:nilpotent-sum}
Let \(R\) be a ring. If \(a, b \in R\) are nilpotent elements with \(a b = b
a\), then \(a + b\) is nilpotent.
\end{lemma}
%

%
\begin{proof}
Suppose \(n, m \in \Z\) are such that \(a^n = b^m = 0\). Since \(a\) and \(b\)
commute, we can write the binomial equation
\((a + b)^{n m} = \sum_{j=0}^{n m} \binom{nm}{j} a^{n m - j} b^j\) --- for
\(j < m\) we have \(n m - j > nm - m = n (m - 1)\), greater than a
multiple of \(n\), so that \(a^{n m - j} = 0\); for \(j \geq m\), we have
\(b^j = 0\). Thus we arrive at \((a + b)^{n m} = 0\).
\end{proof}
%

%
\begin{lemma}
\label{lem:nilpotent-in-Z/nZ}
A class \([m]_n\) is nilpotent in \(\Z/n\Z\) if and only if \(m\) shares all
prime factors of \(n\).
\end{lemma}
%

%
\begin{proof}
Suppose that there exists a prime \(p\) in the prime factorization of \(n\) such
that \(p\) does not divide \(m\). If \(\ell \in \Z\) is any exponent,
\(m^{\ell}\) won't be divisible by \(n\) since it lacks the \(p\) factor ---
hence \(m\) fails to be nilpotent.

If \(m\) shares every prime factor of \(n\), let \(k\) be defined to be the
maximum exponent in the prime factorization of \(n\) (which is ensured to exist
since the collection is finite). Then clearly \([m]_n^k = [m^k]_n = 0\) since
\(m^k\) is divisible by \(n\).
\end{proof}
%

%
\begin{lemma}
\label{lem:properties-nilpotent-element-in-commutative-ring}
Let \(R\) be a \emph{commutative} ring and \(x \in R\) be a nilpotent
element. The following statements hold:
\begin{enumerate}[(a)]\setlength\itemsep{0em}
\item The element \(x\) is either zero or zero-divisor.

\item For every \(r \in R\), the product \(r x\) is nilpotent.

\item For every invertible element \(u \in R\), the sum \(x + u\) is invertible.
\end{enumerate}
\end{lemma}
%

%
\begin{proof}
\begin{enumerate}[(a)]\setlength\itemsep{0em}
\item If \(x\) is non-zero, both the left and right multiplications of \(x\)
  have the same image under the distinct elements \(x^{n-1}\) and \(0\) ---
  hence \(x\) is a zero-divisor.

\item From commutativity \((r x)^n = r^n x^n = r^n \cdot 0 = 0\).

\item First we deal with the case where \(u = 1\). Remember that given any
  \(a \in R\) and \(m \in \N\) we have
  \[
    (1 + a) (1 - a + a^2 - \dots + (-1)^m a^m) = 1 + (-1)^m a^{m + 1}.
  \]
  Therefore, for every \(m > n\), we have
  \((1 + x) \sum_{j=0}^{m} (-1)^j a^j = 1\) --- which shows that \(1 + x\) is
  invertible. For the general case, if \(v\) is the inverse of \(u\), one sees
  that for any \(m > n\) we have
  \[
    (u + x) \sum_{j=0}^m (-1)^j v x^j = 1 + (-1)^m x^{m + 1} = 1.
  \]
\end{enumerate}
\end{proof}
%

\subsection{Units}

%
\begin{definition}[Unit]
\label{def:unit-element-ring}
Let \(R\) be a ring. An element \(u \in R\) is said to be a \emph{left-unit} (or
\emph{right-unit}) if there exists \(a \in R\) for which \(u a = 1\) (or
\(a u = 1\)). An element is said to be a \emph{unit} if it is both a left and
right unit.
\end{definition}
%

%
\begin{proposition}
\label{prop:unit-properties-ring}
Let \(R\) be a ring and \(u \in R\) be an element. The following propositions
hold.
\begin{enumerate}[(a)]\setlength\itemsep{0em}
\item The element \(u\) is a left-unit (or right-unit) if and only if the
  left-multiplication (or right-multiplication) map \(R \to R\) is
  \emph{surjective}.

\item If \(u\) is a \emph{left}-unit (or right-unit), then \(u\) is a
  non-\emph{right}-zero-divisor (or non-\emph{left}-zero-divisor).

\item The \emph{inverse} of a unit is \emph{unique}.

\item Units form a \emph{group} under multiplication --- such group is denoted
  by \(R^{\unit}\).
\end{enumerate}
\end{proposition}
%

%
\begin{proof}
\begin{enumerate}[(a)]\setlength\itemsep{0em}
\item Suppose first that \(u\) is a left-unit. Let \(a \in R\) be any element
  and suppose that \(v \in R\) is such that \(u v = R\) (which is ensured to
  exist) --- then \(u (v a) = (u v) a = a\), thus left-multiplication is
  surjective.

  On the other hand, if the left-multiplication by \(u\) is surjective, we can
  conclude that there must exist \(v \in R\) for which \(u v = 1\) --- hence
  \(u\) is a left-unit of \(R\).

  The proof for the right-unit case is completely analogous and won't be
  included.

\item If \(u\) is a left-unit, assume \(v \in R\) is such that
  \(u v = 1\). Notice that the \emph{right}-multiplication maps
  \(f_u, f_v: R \rightrightarrows R\) by \(u\) and \(v\), respectively, are such
  that --- for all \(a \in R\),
  \[
    f_v f_u(a) = f_v(a u) = (a u) v = a (u v) = a \cdot 1 = a.
  \]
  Therefore, \(f_v\) is the left-inverse of \(f_u\) and thus \(f_u\) is
  injective. The equivalent proof can be written for the right-unit.

\item Let \(u\) be a unit and \(v_1, v_2 \in R\) be such that \(v_1 u = 1\) and
  \(u v_2 = 1\) --- notice that
  \[
    v_1 = v_1 \cdot 1 = v_1 (u v_2) = (v_1 u) v_2 = 1 \cdot v_2 = v_2.
  \]


\item Let \(U\) denote the collection of units of \(R\) (such set is ensured to
  be non-empty since \(1\) is a unit) --- zero-rings are not a problem since
  zero is a unit in such case. If \(a, b \in U\) are units, then there are
  \(x, y \in U\) such that \(a x = x a = 1\) and \(b y = y b = 1\) (from the
  last item, inverses are unique) --- therefore,
  \((a b) (y x) = a (b y) x = a x = 1\), that is, \(a b \in U\) and \(y x\)
  its inverse.
\end{enumerate}
\end{proof}
%

%
\begin{lemma}
\label{lem:unit-two-or-more-inverses}
Let \(R\) be a ring and \(a \in R\) be an element. If \(a\) is a right-unit (or
left-unit) and has two or more left-inverses (or right-inverses), then
\begin{itemize}\setlength\itemsep{0em}
\item The element \(a\) is \emph{not} a left-zero-divisor (or
  right-zero-divisor).
\item The element \(a\) \emph{is} a right-zero-divisor (or left-zero-divisor).
\end{itemize}
\end{lemma}
%

%
\begin{proof}
We shall restrict ourselves with the right-unit case --- the other is completely
analogous. Since \(a\) is a right-unit, there must exist \(u \in R\) such that
\(u a = 1\), then if \(x, y \in R\) are such that \(a x = a y\), we can multiply
both left-sides by \(u\) and obtain \((u a) x = x = y = (u a) y\) --- thus
left-multiplication by \(a\) is injective and \(a\) is not a left-zero-divisor.

For the other item, notice that if \(u\) and \(v\) are distinct left-inverses of
\(a\), then \(u a = v a = 1\) have the same image under right-multiplication by
\(a\) --- hence \(a\) is a right-zero-divisor.
\end{proof}
%

\subsubsection{Division Rings \& Fields}

%
\begin{definition}[Division ring]
\label{def:division-ring}
A ring \(R\) is said to be a \emph{division ring} if \emph{every} non-zero
element is a \emph{unit}.
\end{definition}
%

%
\begin{definition}[Field]
\label{def:field}
A non-zero ring \(R\) is said to be a field if \(R\) is a \emph{commutative}
division ring.
\end{definition}
%

%
\begin{remark}
\label{rem:field-and-integral-domain}
By the item (b) of \cref{prop:unit-properties-ring}, we see that a unit is a
non-zero-divisor, thus, if every non-zero element is a unit, then every non-zero
element is also a non-zero-divisor. This implies directly that \emph{every field
is an integral domain}. On the other hand, \(\Z\) is an integral domain but fails
to be a field, hence \emph{not every integral domain is a field}. The concepts
can however coincide in a special case, as we see in
\cref{prop:commutative-field-iff-integral-domain}.
\end{remark}
%

%
\begin{proposition}
\label{prop:subring-field-is-integral-domain}
Any subring of a field is an integral domain.
\end{proposition}
%

%
\begin{proof}
Let \(k\) be a field and \(R \subseteq k\) be a non-zero subring. Certainly
\(R\) is commutative. Moreover if \(a, b \in R\) are such that \(a b = 0\),
suppose, for the sake of contradiction, that \(a\) and \(b\) are both
non-zero. From the field properties, there are inverses \(a^{-1}, b^{-1} \in k\)
so that \(a^{-1}(a b) b^{-1} = (a^{-1} a) (b b^{-1}) = 1 \cdot 1 = 1\), but \(a
b = 0\) --- this implies that \(1 = 0\), which is a contradiction, thus \(a\) or
\(b\) are zero.
\end{proof}
%

%
\begin{proposition}
\label{prop:commutative-field-iff-integral-domain}
A finite and commutative ring \(R\) is a field if and only if \(R\) is an
integral domain.
\end{proposition}
%

%
\begin{proof}
If \(R\) is a finite commutative integral domain, let \(a \in R\) be a
non-zero-divisor of \(R\) --- so that multiplication by \(a\) is an injective
map. Since the ring is finite, the multiplication by \(a\) is also surjective
--- thus by \cref{prop:unit-properties-ring} we find that \(a\) is a unit.
\end{proof}
%

%
\begin{proposition}[Particular case of Weddenburn's theorem]
\label{prop:division-ring-p2-elements-commutative}
Let \(R\) be a division ring with \(p^2\) elements and \(p\) be a prime,
then \(R\) is a field.
\end{proposition}
%

%
\begin{proof}
For the proof, we shall use the concepts of centre and centralizer, developed in
\cref{def:ring-center} and \cref{def:ring-centralizer}. If we assume that \(R\)
is non-commutative, the centre \(Z(R)\) will be a proper subring of
\(R\). Moreover, since \(Z(R)\) is also a subgroup of \(R\) under
multiplication, by \cref{cor:order-subgroup-divides-order-group} we conclude
that \(|Z(R)| = p\). Let now \(r \in R \setminus Z(R)\), clearly the centralizer
must contain both the centre of the ring --- since by definition it must commute
with \(r\) --- and by symmetry, \(r \in Z(r)\), thus
\(\{r\} \cup Z(R) \subseteq Z(r)\). Since the centralizer is a subgroup of \(R\)
and \(|Z(r)| > p\), one concludes by
\cref{cor:order-subgroup-divides-order-group} that \(|Z(r)| = p^2\) ---
therefore \(Z(r) = R\). That, however gives us a contradiction because if the
centralizer of \(r\) is the whole ring, then \(r\) commutes with every element
and thus \(r \in Z(R)\) --- this cannot be the case by hypothesis, thus one
concludes that \(R\) must be commutative and therefore a field.
\end{proof}
%

\subsection{Monoid Ring}

%
\begin{definition}[Monoid ring]
\label{def:monoid-ring}
Given a monoid \((M, \cdot)\) and a ring \(R\), we define a new ring called
\emph{monoid ring} \(R[M] \coloneq R^{\oplus M}\) --- that is, the ring whose
elements are formal linear combinations \(\sum_{m \in M} a_m \cdot m\), where
the coefficients \(a_m \in R\) are non-zero for at most finitely many
\(m \in M\).

The unitary element of \(R[M]\) is defined to be \(1_R \cdot 1_{M}\). The
additive and multiplicative structures are defined as follows
\begin{align*}
  \sum_{m \in M} a_m \cdot m + \sum_{m \in M} b_m \cdot m
  &\coloneq \sum_{m \in M} (a_m + b_m) \cdot m, \\
  \sum_{m \in M} a_m \cdot m + \sum_{m \in M} b_m \cdot m
  &\coloneq \sum_{m \in M} \sum_{m_1 \cdot m_2 = m} (a_{m_1} b_{m_2}) \cdot m.
\end{align*}
\end{definition}
%

\subsection{Polynomial Ring}

%
\begin{definition}[Polynomial]
\label{def:polynomial}
Let \(R\) be a ring. A \emph{polynomial} is a map \(f: R \to R\) given by a
\emph{finite} linear combination of non-negative powers of the indeterminate
variable with coefficients in \(R\)
\[
  f(x) = \sum_{j \geq 0} a_j x^j, \text{ where } a_j = 0 \text{ for } j \gg 0.
\]
The collection of all polynomials over \(R\) with variable \(x\) is denoted by
\(R[x]\) --- this collection forms a ring under point-wise addition and product,
that is, given another element \(g(x) \coloneq \sum_{j \geq 0} b_j x^j\), we
define
\begin{align*}
  f(x) + g(x) &\coloneq \sum_{j \geq 0} (a_j + b_j) x^j,\\
  f(x) \cdot g(x) &\coloneq \sum_{k \geq 0} \sum_{i + j = k} a_i b_j x^{i + j}.
\end{align*}
It should be noted immediately that \(R[x] \iso R[\N]\), where \(\N\) is viewed
as a monoid under addition.
\end{definition}
%

Two polynomials are said to be equal if the sequence of coefficients match ---
that is, if \(f(x) \coloneq \sum_{j \geq 0} a_j x^j\) and
\(g(x) \coloneq \sum_{j \geq 0} b_j x^j\), then \(f = g\) if and only if
\(a_j = b_j\) for all \(j \geq 0\) --- since the collection of non-zero
coefficients is necessarily finite, this equality relation is well defined.

%
\begin{corollary}
\label{cor:ring-R[x]-inherits-commutativity}
If \(R\) is a commutative ring, then \(R[x]\) is commutative.
\end{corollary}
%

%
\begin{lemma}
\label{lem:ring-R[x]-inherits-integral-domain}
A ring \(R\) is an integral domain if and only if \(R[x]\) is an integral
domain.
\end{lemma}
%

%
\begin{proof}
If \(R\) is an integral domain, then clearly \(R[x]\) is both non-zero and
commutative. Moreover, suppose that \(f(x) \coloneq \sum_{j \geq 0} a_j x^j\)
and \(g(x) \coloneq \sum_{j \geq 0} b_j x^j\) are elements of \(R[x]\) such that
\(f \cdot g = 0\) --- notice that from definition we have
\[
  f(x) g(x) = \sum_{k \geq 0} \sum_{i + j = k} a_i b_j x^{i + j},
\]
which is merely a finite combination with coefficients, so that if
\(f \cdot g = 0\), we necessarily have \(a_j = 0\) for all \(j \geq 0\) or
\(b_j = 0\) for all \(j \geq 0\) since \(R\) is an integral domain --- thus
either \(f = 0\) or \(g = 0\).

On the other hand, if \(R[x]\) is an integral domain, let \(a, b \in R\) be
elements such that \(a b = 0\) in \(R\), then we can consider the constant
polynomials \(f(x) \coloneq a\) and \(g(x) \coloneq b\) to conclude that, since
\(f \cdot g = 0\) then \(a = 0\) or \(b = 0\).
\end{proof}
%

%
\begin{definition}
\label{def:degree-polynomial}
The \emph{degree} of a \emph{non-zero} polynomial
\(f(x) \coloneq \sum_{j \geq 0} a_j x^j\) is defined as the maximum index for
which the corresponding coefficient is non-zero, that is
\[
  \deg f \coloneq \max \{j \colon a_j \neq 0\}.
\]
This is well defined because \(a_j \neq 0\) for only finitely many indices.
By convention, the zero-polynomial has degree \(-\infty\).
\end{definition}
%

Polynomials of multiple variables are obtained simply as an iteration of the
process of construction of \(R[x]\), that is,
\[
  R[x_1, \dots, x_n] \coloneq R[x_1] \dots [x_n].
\]
The ordering of the list \(x_1, \dots, x_n\) is irrelevant in the construction
of the ring, if we swap any elements of the list we still end up with isomorphic
rings.

One can consider a ring of polynomials over infinitely many variables, say
\(R[x_1, x_2, \dots]\), but still every polynomial of such ring consists of only
finitely many terms --- as we impose that the coefficient indexing set is
finite.

%
\begin{definition}[Power series ring]
\label{def:power-series-ring}
Let \(R\) be a ring. The ring of \emph{power series} with variable \(x\) and
coefficients in \(R\), denoted by \(R[\![x]\!]\), is defined to be the ring
whose elements are formal sums \(\sum_{j=0}^{\infty} a_j x^j\).
\end{definition}
%

\section{The Category of Rings}

%
\begin{definition}
\label{def:ring-morphism}
Let \(R\) and \(S\) be rings. We define a morphism of rings \(\phi: R \to S\) to
be a map such that, for all \(a, b \in R\), we have \(\phi(a + b) = \phi(a) +
\phi(b)\) and \(\phi(a b) = \phi(a) \phi(b)\). Moreover, we also impose that
morphisms are unitary, that is, \(\phi(1_R) = 1_S\).
\end{definition}
%

%
\begin{definition}
\label{def:ring-category}
The category of rings is defined to consist of rings and morphisms between them
--- such category will be denoted by \(\Rng\).
\end{definition}
%

%
\begin{remark}
\label{rem:zero-ring-not-zero-object-in-Ring}
Curiously, the zero-ring is \emph{not} a zero-object in the category of rings,
it \emph{is} a \emph{final} object but \emph{fails} to be \emph{initial} ---
this is due to the fact that we imposed morphisms to be unitary, hence for a
non-zero ring \(R\), there exists no ring morphism from the zero-ring to \(R\).

One should, however, not be afraid, because \(\Rng\) does have initial objects.
In fact, \(\Z\) is initial in \(\Rng\) --- for any ring \(R\) we can
define a unique ring morphism \(\Z \unique R\) mapping \(n \mapsto n \cdot 1_R\)
for any \(n \in \Z\).
\end{remark}
%

%
\begin{corollary}[Unit preservation]
\label{cor:ring-morphisms-preserve-units}
If \(\phi: R \to S\) is a ring morphism and \(u \in R\) is a left-unit (or
right-unit), then \(\phi(u) \in S\) is again a left-unit (or right-unit).
\end{corollary}
%

%
\begin{proof}
Let \(v \in R\) be such that \(v u = 1_R\), then \(\phi(v u) = \phi(v) \phi(u)\)
but \(\phi(v u) = \phi(1_R) = 1_S\) thus \(\phi(v) \phi(u) = 1_S\) and
\(\phi(u)\) is indeed a left-unit in \(S\).
\end{proof}
%

%
\begin{proposition}[Image of morphism is a subring]
\label{prop:image-of-morphism-is-subring}
The image of a ring morphism \(R \to S\) is a subring of \(S\).
\end{proposition}
%

%
\begin{proof}
Let \(\phi: R \to S\) be any ring morphism. Since it is a ring morphism,
\(\phi(1_R) = 1_S \in \im \phi\), moreover, if \(a, b \in R\) then \(\phi(a + b)
= \phi(a) + \phi(b) \in \im \phi\) and \(\phi(a b) = \phi(a) \phi(b) \in \im
\phi\) --- thus \(\im \phi\) is a subring of \(S\).
\end{proof}
%

%
\begin{remark}
\label{rem:non-zero-divisors-may-not-be-preserved}
The image of non-zero-divisors may be a zero-divisor in the new ring --- for
instance, \(3 \in Z\) is a non-zero-divisor but the ring morphism
\(Z \epi \Z/6\Z\) (canonical projection) maps \(3 \mapsto [3]_6\), which is a
zero divisor since \([2]_6 [3]_6 = [0]_6\).
\end{remark}
%

%
\begin{proposition}
\label{prop:sufficient-conditions-morphism-of-rings}
Let \(\phi: R \to S\) be a set-function between rings \(R\) and \(S\) such that
the additive and multiplicative structures are preserved. If either one of the
following propositions hold, then \(\phi\) is a morphism of rings:
\begin{enumerate}[(a)]\setlength\itemsep{0em}
\item The set-function \(\phi\) is surjective.

\item The set-function \(\phi \neq 0\) and \(S\) is an integral domain.
\end{enumerate}
\end{proposition}
%

%
\begin{proof}
We simply need to show that each of the propositions yield \(\phi(1_R) = 1_S\)
--- which is the only condition left for \(\phi\) to be a morphism of rings.
\begin{enumerate}[(a)]\setlength\itemsep{0em}
\item If \(\phi\) is surjective, then there exists \(r \in R\) such that
  \(\phi(a) = 1_S\) but then
  \[
  1_S = \phi(a) = \phi(a \cdot 1_R) = \phi(a) \phi(1_R)
  = 1_S \cdot \phi(1_R) = \phi(1_R).
  \]

\item If \(\phi(1_R)\) where to be zero, the whole image of \(\phi\) would also
  evaluate to zero, thus \(1_R\) has a non-zero image under \(\phi\). Let
  \(r \in R\) be any element such that \(\phi(r) \neq 0\) --- then
  \(\phi(r) = \phi(r \cdot 1_R) = \phi(r) \phi(1_R)\) thus by means of
  \cref{cor:cancellation-integral-domain} we can conclude that
  \(\phi(1_R) = 1_S\).
\end{enumerate}
\end{proof}
%

\subsection{Universal Property of Polynomial Rings}

Let \(A \coloneq \{a_1, \dots, a_n\}\) be a set and define a category
\(\cat R_A\) consisting of pairs \((\alpha, R)\) --- where \(\alpha: A \to \R\)
is a set-function and \(R\) is a ring such that \(\alpha(a_i)\) commutes with
every element of \(R\) for each \(1 \leq i \leq n\) (one may restrict this
further by imposing that \(R\) is commutative). Intuitively, morphisms between
two objects \((\alpha, R) \to (\beta, S)\) are ring morphisms \(\phi: R \to S\)
such that the following diagram commutes in the category of sets:
\[
\begin{tikzcd}
R \ar[rr, "\phi"] & &S \\
&A \ar[lu, "\alpha"] \ar[ru, swap, "\beta"] &
\end{tikzcd}
\]

In the same context of what is explained above, we now state a universal
property concerning the ring \(\Z[x_1, \dots, x_n]\) in the category
\(\cat R_A\).

%
\begin{proposition}
\label{prop:Z[several]-initial-polynomial-rings}
Let \(R\) be a ring and define a set-function \(\alpha: A \to R\) mapping
\(a_j \mapsto x_j\). The object \((\alpha, \Z[x_1, \dots, x_n])\) is initial in
\(\cat R_A\).
\end{proposition}
%

%
\begin{proof}
Let \((\beta, R)\) be any object in \(\cat R_A\). We now construct a morphism
\[
\phi: (\alpha, \Z[x_1, \dots, x_n]) \to (\beta, R).
\]
We can impose that \(\phi(x_j) \coloneq \beta(a_j)\) for every
\(1 \leq j \leq n\). For \(\phi\) to be a ring morphism, one must also impose
that is preserves both the additive and multiplicative structures of the
rings. Since \(\Z\) is initial in \(\Rng\), we can uniquely determine that
\(\phi(r) = \psi(r)\) --- where \(\psi: \Z \unique \R\) is a unique
morphism. That is, we defined morphism \(\phi\) given by
\begin{align*}
  \phi\bigg( \sum m_{i_1 \dots i_n} x_1^{i_1} \dots x_n^{i_n} \bigg)
  &= \sum \phi(m_{i_1 \dots i_n}) \phi(x_1^{i_1}) \dots \phi(x_n^{i_n}) \\
  &= \sum \psi(m_{i_1 \dots i_n}) \beta(a_1)^{i_1} \dots \beta(a_n)^{i_n},
\end{align*}
which is surely both a ring morphism and unique --- thus \((\alpha, \Z[x_1,
\dots, x_n])\) is initial, and
\[
\begin{tikzcd}
\Z[x_1, \dots, x_n] \ar[rr, dashed, "\phi"] & &R \\
&A \ar[lu, "\alpha"] \ar[ru, swap, "\beta"]   &
\end{tikzcd}
\]
\end{proof}
%

%
\begin{proposition}[Universal property of polynomial rings]
\label{prop:universal-property-polynomial-rings}
Let \(f: R \to S\) be any ring morphism and \(s_0 \in S\) be a fixed element
commuting with \(f(r)\) for all \(r \in R\). There exists a \emph{unique} ring
morphism \(\phi: R[x] \unique S\) which \emph{extends} \(f\) and maps
\(x \mapsto s_0\).
\end{proposition}
%

%
\begin{proof}
In the construction of \(\phi\) we first impose that \(\phi(x) \coloneq
s_0\). Moreover, for \(\phi\) to extend \(f\) one has to define \(\phi(r)
\coloneq f(r)\) for all given \(r \in R\). In order for \(\phi\) to be a ring
morphism it also needs to preserve the additive and multiplicative structures of
the rings --- hence we end up with
\[
\phi\bigg( \sum_{j \geq 0} r_j x^j \bigg)
= \sum_{j \geq 0} \phi(r_j) \phi(x)^j
= \sum_{j \geq 0} f(r_j) s_0^j.
\]
Since \(\phi\) is completely defined by the image under constant polynomials and
\(x\), we find that \(\phi\) is the unique ring morphism sending \(x\) to
\(s_0\) for which the following diagram commutes
\[
\begin{tikzcd}
R \ar[d, hook] \ar[r, "f"]         & S \\
R[x] \ar[ru, dashed, swap, "\phi"] &
\end{tikzcd}
\]
\end{proof}
%

In fact, in the case where \(R\) is \emph{commutative} and \(f: R \to R\) is an
\emph{endomorphism}, the unique ring morphism \(\phi: R[x] \to R\) defined above
is called the \emph{evaluation map} at the given fixed point. We normally denote
such map by \(\eval_r\) --- where \(r \in R\) is the chosen fixed point for
evaluation.

\subsection{Monomorphisms}

%
\begin{definition}[Kernel]
\label{def:ring-morphism-kernel}
The kernel of a ring morphism \(\phi: R \to S\) is the set
\[
\ker \phi \coloneq \{r \in R \colon \phi(r) = 0_S\}.
\]
\end{definition}
%

%
\begin{proposition}[Ring monomorphisms]
\label{prop:ring-kernel-monomorphism-injective}
Let \(\phi: R \to S\) be a ring morphism. The following properties are
equivalent:
\begin{enumerate}[(a)]\setlength\itemsep{0em}
\item The ring morphism \(\phi\) is a monomorphism.

\item The kernel of \(\phi\) is trivial --- that is, \(\ker \phi = \{0_R\}\).

\item The set-function \(\phi\) injective.
\end{enumerate}
\end{proposition}
%

%
\begin{proof}
(a) \(\implies\) (b): Suppose \(\phi\) is monic. Let \(r \in \ker \phi\) be any
element and consider the uniquely defined morphisms
\(\eval_r, \eval_{0_R}: \Z[x] \rightrightarrows R\) provided by
\cref{prop:Z[several]-initial-polynomial-rings} by fixing \(\eval_r(x) = r\) and
\(\eval_{0_R}(x) = 0_S\). Notice that
\(\phi \eval_r(n) = \phi \eval_{0_R}(n) = \phi(n)\) for any \(n \in \Z\), while
\(\phi \eval_r(x) = \phi(r) = 0_S\) and
\(\phi \eval_{0_R}(x) = \phi(0_R) = 0_S\) --- thus \(\eval_r = \eval_{0_S}\),
which in turn implies that \(r = \eval_r(x) = \eval_{0_R}(x) = 0_S\). Therefore
indeed \(\ker \phi = \{0_R\}\).

(b) \(\implies\) (c): Suppose \(\phi\) has trivial kernel and let \(r, t \in R\)
be such that \(\phi(r) = \phi(t)\), then \(\phi(r) - \phi(t) = \phi(r - t) = 0\)
and \(r - t \in \ker \phi\), which implies in \(r = t\) --- that is, \(\phi\) is
injective.

(c) \(\implies\) (a): Suppose \(\phi\) is an injective set-function. Since
injections are monomorphisms in \(\Set\), we conclude that the set-function
\(\phi: R \to S\) is a monomorphism in \(\Set\) when \(R\) and \(S\) are viewed
as sets. If we now endows \(R\) and \(S\) with their respective ring structures,
we obtain that \(\phi\) is a monic in \(\Rng\).
\end{proof}
%

\subsection{Subrings}

%
\begin{definition}[Subring]
\label{def:subring}
Let \(R\) be a ring. A subring \(S\) of \(R\) is a ring whose elements are
contained in \(R\) and the canonical inclusion map \(S \emb R\) is a ring
morphism.
\end{definition}
%

%
\begin{remark}
\label{rem:subring-same-unity}
It should be noted that \(S\) is a ring with unity where \(1_S = 1_R\).
\end{remark}
%

%
\begin{proposition}[Intersection of subrings]
\label{prop:intersection-subring-is-subring}
Let \(R\) be a ring and \(\{S_{j}\}_{j \in J}\) be a collection of subrings of
\(R\). The intersection \(\bigcap_{j \in J} S_j\) is also a subring of \(R\).
\end{proposition}
%

%
\begin{proof}
Since each \(S_j\) is a subring, if \(a, b \in \bigcap_{j \in J} S_j\) then \(a,
b \in S_j\) for all \(j \in J\), and \(a + b, a b \in S_j\). Since every subring
contains the unity, so does the intersection --- hence \(\bigcap_{j \in J} S_j\)
is a subring of \(R\).
\end{proof}
%


\subsubsection{Center}

%
\begin{definition}[Center]
\label{def:ring-center}
Given a ring \(R\), we define its centre to be
\[
Z(R) \coloneq \{r \in R \colon r x = x r \text{ for all } x \in R\}.
\]
\end{definition}
%

%
\begin{corollary}
\label{cor:center-is-subring}
The centre of a ring is a subring.
\end{corollary}
%

%
\begin{proof}
Let \(R\) be a ring. Clearly, \(1 \in Z(R)\). Moreover, if \(r, s \in Z(R)\)
then for any \(x \in R\) we have
\begin{gather*}
(r + s) x = r x + s x = x r + x s = x (r + s), \\
(r s) x = r(s x) = (s x) r = (x s) r = x (s r) = x (r s).
\end{gather*}
Therefore \(Z(R)\) is indeed a subring of \(R\).
\end{proof}
%

%
\begin{corollary}
\label{cor:division-ring-center-is-field}
If \(R\) is a division ring, then its centre \(Z(R)\) is a field.
\end{corollary}
%

%
\begin{proof}
Since \(Z(R)\) inherits the division ring structure and every element of
\(Z(R)\) commutes, it is indeed a field.
\end{proof}
%

%
\begin{definition}[Centralizer]
\label{def:ring-centralizer}
Given a ring \(R\) and an element \(r \in R\), the centralizer of \(r\) is
defined to be the collection of elements \(x \in R\) such that \(x r = r x\) ---
we shall denote the centralizer of \(r\) as \(Z(r)\).
\end{definition}
%

%
\begin{corollary}
\label{cor:centralizer-is-subring}
The centralizer of an element is a subring.
\end{corollary}
%

%
\begin{proof}
This is simply a straightforward particular case of
\cref{cor:center-is-subring}.
\end{proof}
%

%
\begin{corollary}
\label{cor:division-ring-centralizer-is-division-ring}
In a division ring a centralizer is also a division ring.
\end{corollary}
%

%
\begin{proof}
Let \(R\) be a division ring and \(r \in R\) any element. If \(x \in Z(r)\) is
any element, then since \(r x = x r\), we have \((r^{-1} x^{-1}) r x = (r^{-1}
x^{-1}) x r = 1\) thus \(r^{-1} x^{-1} r\) is a left-inverse of \(r\) ---
moreover, we equivalently see that \(r x^{-1} r^{-1}\) is a right-inverse of
\(r\). Since \(Z(r)\) is a subring, such inverses are contained in \(Z(r)\) and
thus \(Z(r)\) is a division ring.

Notice however that the centralizer may not be a field since there can be
non-commuting elements in \(Z(r)\).
\end{proof}
%

%
\begin{corollary}
\label{cor:center-is-intersection-of-all-centralizers}
The centre of a ring is the intersection of all centralizers of the ring.
\end{corollary}
%

%
\begin{proof}
Let \(R\) be a ring. Certainly, if \(x \in Z(R)\) then it commutes with every
element of \(R\) --- which is equivalent to \(x \in \bigcap_{r \in R}
Z(r)\). Moreover, if an element belongs to the intersection of all centralizers,
every element of the ring commutes with it and thus such element is also present
in the centre of the ring. Therefore \(Z(R) = \bigcap_{r \in R} Z(r)\).
\end{proof}
%

\subsection{Epimorphisms}

%
\begin{remark}[Epimorphisms and surjection in \(\Rng\)]
\label{rem:ring-epimorphisms-and-surjection}
In \(\Rng\), epimorphisms are \emph{not necessarily} surjective set-functions.

A classical counterexample is the ring morphism given by the canonical inclusion
\(\iota: \Z \emb \Q\). Let \(R\) be any ring and consider ring morphisms
\(f, g: \Q \rightrightarrows R\) such that the following diagram commutes
\[
\begin{tikzcd}
\Z \ar[r, hook, "\iota"]
&\Q \ar[r, shift left, "f"] \ar[r, shift right, swap, "g"]
&R
\end{tikzcd}
\]
Since \(f|_{\Z} = g|_{\Z}\) we have, for any \(p/q \in \Q\), that
\(f(p/q) = f(p) f(q)^{-1}\) and \(g(p/q) = g(p) g(q)^{-1}\), which implies in
\(f(p/q) = g(p/q)\) --- that is, \(f = g\) in general, which implies that
\(\iota\) is an epimorphism in \(\Rng\). For the shock of the reader, the same
is obviously not the case in \(\Set\). If we take the rings \(\Z\) and \(\Q\) as
abelian groups, one sees that \(\coker \phi\) is non-trivial and by
\cref{prop:epic-in-Ab} we arrive at the fact that \(\phi\) is not an epimorphism
in \(\Ab\) neither.
\end{remark}
%

%
\begin{remark}
\label{rem:isomorphisms-in-ring}
With the caution given by \cref{rem:ring-epimorphisms-and-surjection} one can
rightly observe that in the category of rings a morphism may be both monic and
epic but yet lack the conditions for being an isomorphism.
\end{remark}
%

\begin{remark}
\label{rem:cokernel-in-ring}
Cokernels in \(\Rng\) are not what one would normally expect of a good category,
notice that given a ring morphism \(\phi: R \to S\) and, if \(\alpha: S \to Q\)
is any ring morphism such that \(\alpha \phi = 0\) --- as is required by the
universal property of cokernels --- then \(\alpha\phi(1) = \alpha(1) = 0\),
which can only be the case for \(Q = 0\), the zero-ring. We therefore conclude
that \(\coker \phi\) must be the zero-ring.
\end{remark}

\subsection{Products}

%
\begin{proposition}[Product]
\label{prop:product-ring}
Products exist in the category of rings.
\end{proposition}
%

%
\begin{proof}
Let \(R\) and \(S\) be rings. We'll define on \(R \times S\) additive and
multiplicative structures naturally as follows:
\begin{gather*}
(x, a) +_{R \times S} (y, b)     \coloneq (x +_R y, a +_S b), \\
(x, a) \cdot_{R \times S} (y, b) \coloneq (x \cdot_R y, a \cdot_S b),
\end{gather*}
for any \(x, y \in R\) and \(a, b \in S\). We now check that \(R \times S\) is a
product in \(\Rng\).

Let \(Z\) be any ring and let \(\phi: Z \to R\) and \(\psi: Z \to S\) be any two
ring morphisms. Considering the canonical projections \(\pi_R: R \times S \epi
R\) and \(\pi_S: R \times S \epi S\), we can construct a map \(\ell: Z \to R
\times S\) sending \(z \mapsto (\phi(z), \psi(z))\). Such a map inherits the
preservation of both the multiplicative and additive structures of \(Z\) and \(R
\times S\) since \(\phi\) and \(\psi\) do so --- therefore \(\ell\) is a
morphism of rings. Moreover, \(\ell\) is completely determined by the image of
both \(\phi\) and \(\psi\), hence \(\ell\) is the unique morphism of rings such
that the following diagram commutes
\[
\begin{tikzcd}
&Z \ar[ddl, bend right, swap, "\phi"]
\ar[ddr, bend left, "\psi"]
\ar[d, dashed, "\ell"] & \\
&R \times S \ar[ld, two heads, "\pi_R"]
\ar[rd, two heads, swap, "\pi_S"] &\\
R & &S
\end{tikzcd}
\]
\end{proof}
%

\begin{remark}
\label{rem:coproduct-Z[x,y]}
Coproducts on the other hand, although present in \(\Rng\), are not as easy to
construct. Lets work out a special case: consider the commutative ring
\(\Z[x, y]\) \emph{in} the \emph{category of commutative rings}. We'll show that
\(\Z[x, y]\) is the coproduct of two copies of \(\Z[x]\). Let \(R\) be any
commutative ring together with morphisms of rings \(f, g: \Z[x]
\rightrightarrows R\). We define a map \(\phi: \Z[x, y] \to R\) for which \(x
\mapsto f(x)\), while \(y \mapsto g(x)\) and finally \(n \mapsto n\) for every
\(n \in \Z\). One immediately sees that \(\phi\) indeed satisfies every
condition of being a ring morphism. Moreover, since the image of \(x, y\) and
\(\Z\) completely defines \(\phi\), we conclude that \(\phi\) is the unique
morphism of rings such that the following diagram commutes
\[
\begin{tikzcd}
\Z[x] \ar[dr, hook] \ar[rdd, bend right, swap, "f"]
& &\Z[x] \ar[dl, hook'] \ar[dld, bend left, "g"] \\
&\Z[x, y] \ar[d, dashed, "\phi"] & \\
&R &
\end{tikzcd}
\]
thus \(\Z[x, y]\) is a coproduct of two copies of \(\Z[x]\) in the category of
commutative rings.
\end{remark}

\subsection{The Ring \texorpdfstring{\(\End_{\Ab}(G)\)}{End\_Ab(G)}}

The ring of endomorphisms of an abelian group \(\End_{\Ab}(G)\) is a structure
that pops up in plenty of situations --- module theory will explore this ring
structure vastly --- so in this subsection we take some time to consider a small
collection of interesting and useful facts about it. Such ring, has its additive
structure defined by point-wise addition and the multiplicative structure is
given by composition of morphisms.

%
\begin{proposition}
\label{prop:Z-isomorphic-End(Z)-in-Ring}
In the category of rings, there exists a natural isomorphism
\[
\Z \iso \End_{\Ab}(\Z).
\]
\end{proposition}
%

%
\begin{proof}
Let \(\phi: \End_{\Ab}(\Z) \to \Z\) sending \(f \mapsto f(1)\) it is clear that
\[
\phi(f + g) = (f + g)(1) = f(1) + g(1) = \phi(f) + \phi(g),
\]
moreover the multiplicative structure is also preserved:
\[
\phi(f g) = (f g)(1) = f(g(1)) = f(1 \cdot g(1)) = f(1) g(1) = \phi(f) \phi(g).
\]
Also, \(\phi(\Id) = \Id(1) = 1\), therefore \(\phi\) is a ring morphism. Since
the image of the identity element over a group morphism completely determines
the map, one can be certain that \(\phi\) is a bijection --- thus an isomorphism
of rings.
\end{proof}
%

Lets agree for the time being that, given a ring \(R\) and an element
\(r \in R\), the morphisms of rings \({}_rm, m_r: R \rightrightarrows R\) are
the left and right, respectively, multiplication of elements of \(R\) by \(r\).

%
\begin{proposition}
\label{prop:left-multiplication-injective-ring-morphism}
Let \(R\) be a ring. The map \(m: R \mono \End_{\Ab}(R)\) sending
\(r \mapsto {}_rm\) is an injective ring morphism.
\end{proposition}
%

%
\begin{proof}
We first show that \(m\) is a ring morphism: let \(a, b \in R\) be any three
elements, then for any \(r \in R\) we have
\[
m(a + b)(r) = {}_{a + b}m(r) = (a + b)r = a r + b r = m(a)(r) + m(b)(r).
\]
On the other hand, multiplication yields composition, as expected
\[
m(a b)(r) = {}_{a b}m(r) = (a b) r = a (b r) = {}_am({}_bm(r)) = m(a)(m(b)(r)).
\]
Also, \(m(1)(r) = 1 \cdot r = r = \Id_R(r)\) and hence \(m(1) = \Id_R\). We
conclude that \(m\) is indeed a ring morphism. The injectivity comes from the
fact that the only element that yields a zero-map in \(\End_{\Rng}(R)\) is zero
--- thus the kernel is trivial.
\end{proof}
%

%
\begin{remark}
\label{rem:ring-right-multiplication-lacks-conditions}
Notice that one cannot further extend
\cref{prop:left-multiplication-injective-ring-morphism} for
right-multiplications, this comes from the fact that
\[
m_{a b}(r) = r (a b) = (r a) b = m_b(m_a(r)).
\]
That is, the multiplicative structure has its order reversed when transitioning
to the compositional structure.
\end{remark}
%

%
\begin{proposition}
\label{prop:unique-ring-structure-Z}
Up to isomorphism, there exists a unique ring (with identity) structure whose
underlying group is \((\Z, +)\).
\end{proposition}
%

%
\begin{proof}
Let \(R\) be any ring with underlying group \(\Z\), and fix any \(r \in
R\). Consider the ring morphism \(m: R \to \End_{\Ab}(R)\) as defined in
\cref{prop:left-multiplication-injective-ring-morphism}. Let
\(f \in \End_{\Ab}(R)\) is completely defined by \(f(1_R)\), thus
\({}_{f(1_r)}m = f\). Therefore \(m\) is surjective and hence a bijection ---
\(m\) thus establishes an isomorphism \(R \iso \End_{\Ab}(R)\). Moreover, by
\cref{prop:Z-isomorphic-End(Z)-in-Ring} we have \(\End_{\Ab}(R) \iso \Z\), thus
\[
R \iso \End_{\Ab}(R) \iso \Z,
\]
which proves the statement.
\end{proof}
%

%
\begin{proposition}
\label{prop:center-ring-iso-center-End(R)}
Let \(R\) be a ring and. There is a subring \(S \subseteq Z(R)\) of the
centre of \(R\) such that there exists a \emph{ring} isomorphism
\[
S \iso Z(\End_{\Ab}(R)).
\]
\end{proposition}
%

%
\begin{proof}
Let \(f \in Z(\End_{\Ab}(R))\) be any group morphism. Since \(f\) commutes with
every group endomorphism on \(R\), in particular \(f\) commutes with every
right-multiplication by an element of \(R\) --- that is, given \(r \in R\), we
have \(f(x r) = f(x) r\), which can only be the case if \(f\) was a
left-multiplication by an element of \(R\). Moreover, by
\cref{prop:left-multiplication-injective-ring-morphism} the map
\(m: R \to \End_{\Ab}(R)\) is injective --- restricting \(m\) to the subring of
\(Z(R)\) given by
\(S \coloneq \{r \in Z(R) \colon {}_rm \in Z(\End_{\Ab}(R))\}\) makes \(m|_S\)
into a surjective morphism. Therefore the morphism
\(\overline{m}: S \iso Z(\End_{\Ab}(R))\), where \(\overline{m}(r) = m(r)\) for
all \(r \in S\), is an isomorphism of rings.
\end{proof}
%

%
\begin{corollary}
\label{cor:Z/nZ-iso-End(Z/nZ)}
Let \(n \in \Z_{> 0}\) be a positive integer. There exists a \emph{ring}
isomorphism
\[
\Z/n\Z \iso \End_{\Ab}(\Z/n\Z).
\]
\end{corollary}
%

%%% Local Variables:
%%% mode: latex
%%% TeX-master: "../../deep-dive"
%%% End:

\section{Homotopy}

\begin{definition}[Left homotopy in \(\Top\)]
\label{def:left-homotopy-Top}
Let \(f, g: X \para Y\) be parallel topological morphisms. We define a
\emph{left homotopy} \(\eta: f \nat g\) between \(f\) and \(g\) to be a morphism
\(\eta: X \times I \to Y\) such that the following diagram commutes
\[
\begin{tikzcd}
X \ar[d, rd, "f"'] \ar[r, "i_0"]
& X \times I \ar[d, "\eta"]
& X \ar[l, "i_1"'] \ar[ld, "g"]
\\
& Y
\end{tikzcd}
\]
where \(i_0, i_1: X \to X \times I\) are parallel morphisms with \(x
\xmapsto{i_{0}} (x, 0)\) and \(x \xmapsto{i_{1}} (x, 1)\).
\end{definition}

To put concretely, left homotopy between \(f\) and \(g\) is a continuous map
\(\eta\) such that \(\eta(-, 0) = f\) and \(\eta(-, 1) = g\) --- which can be
visually thought of as a deformation of the morphism \(f\) to \(g\). If \(f\)
and \(g\) are indeed homotopic, we denote this by \(f \simht g\).

\begin{corollary}[Equivalence relation]
\label{cor:htpy-equivalence-rel}
Left homotopy induces an \emph{equivalence relation} \(\simht\) on the
collection of topological morphisms. Moreover, homotopy respects composition of
morphisms.
\end{corollary}

\begin{proof}
First we prove that \(\simht\) is an equivalence relation:
\begin{itemize}\setlength\itemsep{0em}
\item (Reflexive) Given a continuous map \(f: X \to Y\), we explicitly define a
  homotopy \(f \nat f\) by mapping \((x, t) \mapsto x\) for all
  \((x, t) \in X \times I\).

\item (Symmetric) Let \(g: X \to Y\) be another continous map and suppose that
  there exists a homotopy \(\eta: f \nat g\). We can define a homotopy \(\eta':
  g \nat f\) by mapping \((x, t) \mapsto \eta(x, 1 - t)\) --- so that \(\eta'(-,
  0) = \eta(-, 1) = g\) and \(\eta'(-, 1) = \eta(-, 0) = f\).

\item (Transitive) Consider yet another morphism \(h: X \to Y\) and suppose that
  there exists a homotopy \(\sigma: g \nat h\). In order to construct a homotopy
  \(\delta: f \nat h\), we define a \(X\)-parametrized path by concatenating
  \(\eta\) and \(\sigma\) appropriately
  \[
  \delta(x, t) \coloneq
  \begin{cases}
    \eta(x, 2 t), & t \in [0, 1/2], \\
    \sigma(x, 2 t - 1), & t \in [1/2, 1].
  \end{cases}
  \]
  Indeed, \(\delta(-, 0) = \eta(-, 0) = f\) while
  \(\delta(-, 1) = \sigma(-, 1) = h\). It remains to prove that \(\delta\) is
  continuous. Notice that the sets \(A \coloneq X \times [0, 1/2]\) and
  \(B \coloneq X \times [1/2, 1]\) are both closed in the product topology and
  cover the whole space \(X \times I\). By
  \cref{prop:continuous-from-covering-subspaces}, since \(\delta\)
  is continuous on both \(A\) and \(B\) then \(\delta\) is continuous on
  \(A \cup B = X \times I\).
\end{itemize}
\end{proof}

Given any two topological spaces \(X\) and \(Y\), we denote the
collection of continuous maps \(X \to Y\) \emph{up to homotopy equivalence} by
\[
[X, Y] \coloneq \Hom_{\Top}(X, Y)/{\simht}.
\]
We call a morphism \([f] \in [X, Y]\) a \emph{homotopy class}. Moreover, for any
three spaces \(X, Y, Z \in \Top\), there exists a \emph{unique} compositional
map
\[
[X, Y] \times [Y, Z] \to [X, Z]
\]
such that the following diagram commutes
\[
\begin{tikzcd}
\Hom_{\Top}(X, Y) \times \Hom_{\Top}(Y, Z)
\ar[rr] \ar[dd, two heads]
&&\Hom_{\Top}(X, Z) \ar[dd, two heads]
\\ && \\
{[X, Y] \times [Y, Z]} \ar[rr, dashed]
&& {[X, Z]}
\end{tikzcd}
\]
That is, given morphisms \(X \xrightarrow{f} Y \xrightarrow{g} Z\), we have a
composition of \([f]\) with \([g]\) uniquely defined as
\[
[g] \circ [f] \coloneq [g f].
\]

%%% Local Variables:
%%% mode: latex
%%% TeX-master: "../../deep-dive"
%%% End:

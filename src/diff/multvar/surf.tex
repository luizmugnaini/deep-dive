\section{An Elementary Construction of Differential Forms}

\begin{definition}[\(k\)-surface]
\label{def:k-surface-Rn}
Let \(E \subseteq \R^n\) be an open set. We define a \emph{\(k\)-surface in}
\(E\) to be a \(C^1\) \emph{map} \(\Phi: D \to E\), where \(D \subseteq \R^k\)
is a \emph{compact} set --- this set is commonly referenced to as the parameter
domain of \(\Phi\).
\end{definition}

\begin{definition}[Differential \(k\)-form]
Let \(E \subseteq \R^n\) be an open set. A \emph{differential \(k\)-form} in
\(E\), for \(k > 0\), is a multilinear map \(\omega\) given by
\[
  \omega \coloneq \sum f_{j_1 \dots j_k}\,
  \diff x_{j_1} \wedge \dots \wedge \diff x_{j_k},
\]
where the sum runs over the sequence of indices \((j_1, \dots, j_k)\), where \(1
\leq j_r \leq n\) for each \(1 \leq r \leq k\), and \(f_{j_1 \dots j_k}: E \to
\R\) are continuous maps --- that is, \(0\)-forms. A differential \(k\)-form is
said to be of class \(C^p\) if each map \(f_{j_1 \dots j_k}\) is of class
\(C^p\).

The form \(\omega\) is said to be a \emph{basic} \(k\)-form if we have the
ordering \(1 \leq j_1 < \dots < j_k \leq n\) for its indices --- in this case we
commonly denote each sequence of indexes as \(I \coloneq (j_1, \dots, j_k)\) and
write
\[
  \omega = \sum_I f_{I}\, \diff x_{I}.
\]
\end{definition}

\begin{definition}[Line integral]
\label{def:line-integral}
The integrals of \(1\)-forms are called \emph{line integrals}.
\end{definition}

\begin{example}
Lets calculate a line integral on the \(1\)-surface (\(C^1\) curve)
\(\gamma: [0, 2 \pi] \to \R^2\) defined by \(t \mapsto (a \cos(t), b \sin(t))\),
for constants \(a, b > 0\). Let \(\omega = x\, \diff y\) be a \(1\)-form on
\(\R^2\). Measuring this curve with respect to \(\omega\) yields
\[
  \int_{\gamma} \omega = \int_0^{2\pi} a \cos(t) [b \cos(t)]\, \diff t
  = ab \int_0^{2 \pi} \frac{\cos(2t) - 1}{2}\, \diff t = - ab \pi.
\]
\end{example}

\begin{definition}[Surface area]
\label{def:k-surface-area}
Let \(\Phi: D \to E\) be a \(k\)-surface on the open set \(E \subseteq \R^n\),
and \(\omega \coloneq \sum_I f_{i_1 \dots i_k}\, \diff x_{i_1 \dots i_k}\). We
define the area of the surface \(\Phi\) on \(\omega\) to be given by
\[
  \int_{\Phi} \omega
  \coloneq \int_D \sum_I f_{i_1 \dots i_k}(\Phi(t))
  \det {\left[ \partial_j \Phi_{i_r}(t) \right]}_{1 \leq r, j \leq k}
  \, \diff t
\]
\end{definition}

\begin{definition}[Piecewise smooth]
\label{def:piecewise-smooth-surface}
We define the concept of a piecewise smooth surface in an inductive manner. A
point is a zero dimensional surface of any smoothness class. A surface \(S
\subseteq \R^n\) of dimension \(k\) is piecewise smooth if, after a countable
collection of at most \((k-1)\)-dimensional piecewise smooth surfaces can be
removed from \(S\), the resulting surface can be decomposed into a countable
collection of \(k\)-dimensional smooth surfaces.
\end{definition}

\begin{theorem}[A zero form has null coefficients]
\label{thm:zero-form-standard-representation}
Let \(\omega \coloneq \sum_J f_J\, \diff x_J\) be a \(k\)-form in an open set
\(E \subseteq \R^{n}\), in its standard representation. If \(\omega = 0\) in
\(E\) (that is \(\omega(\Phi) = 0\) for all \(k\)-surface \(\Phi\) in \(E\)),
then \(f_J(x) = 0\) for all \(x \in E\).
\end{theorem}

\begin{proof}
Suppose, for the sake of contradiction, that there exists an index sequence \(S
\coloneq (s_1, \dots, s_{k})\) such that \(f_{S}(y) \neq 0\) for some \(y \in
E\). Since \(f_{S}\) is continuous, let \(\delta > 0\) be such that \(f_J(x) >
0\) whenever \(\norm{x_i - y_i} \leq \delta\), for all \(1 \leq i \leq n\). Let
\(D \subseteq \R^k\) be the compact set given by
\[
  D \coloneq \{t \in \R^{k} \colon \norm{t_j} \leq \delta
  \text{ for all } 1 \leq j \leq k\}.
\]
Define a \(k\)-surface \(\Phi: D \to E\) to be given by
\[
  \Phi(t) \coloneq v + \sum_{j=1}^k t_j e_{s_j}\text{, for all } t \in D.
\]
Notice that \(D\) was chosen so that \(f_S(\Phi(t)) > 0\) for all \(t \in
D\). Notice that
\[
  \int_{\Phi} \omega = \int_D \sum_J f_J(\Phi(t))
  \det[\partial_i \Phi_{j_r}(t)]_{1 \leq i, r \leq k}\, \diff t
  = \int_D f_S(\Phi(t)) \det[\partial_i \Phi_{s_r}]_{1 \leq i, r \leq k}\, \diff t
\]
since for all indexing sequence \(J \coloneq (j_1, \dots, j_k) \neq S\)
the matrix \([\partial_i \Phi_{j_r}(t)]_{1 \leq i, r \leq k}\) has at least one
column equal to zero --- since there must exists at least one \(j_{r_0} \in J
\setminus S\), thus \(\Phi_{j_0}(t) = y_{j_0}\) for all \(t \in D\) and thus the
column \([\partial_i \Phi_{j_0}(t)]_{1 \leq i \leq k} = 0\), which implies in
\(\det [\partial_i \Phi_{j_r}(t)]_{1 \leq i, r \leq k} = 0\). On the other hand,
notice that \(\partial_i \Phi_{s_r}(t) = \delta_{i r}\) for all \(1 \leq i, r
\leq k\), thus \(\det [\partial_i \Phi_{s_r}(t)]_{1 \leq i, r \leq k} = 1\). We
conclude that
\[
  \int_{\Phi} \omega = \int_D f_S(t)\, \diff t,
\]
which is strictly positive since \(f_S(y) > 0\). Therefore \(\omega(\Phi) \neq
0\), which is a contradiction --- thus there must exist no indexing sequence
\(S\) and therefore every coefficient is the zero-map.
\end{proof}

\subsection{Differential Operator}

\begin{definition}
\label{def:differential-operator-form}
Let \(f: E \to \R\) be a map of class \(C^1\), we define an operator \(\diff\)
which transforms any \(0\)-form \(f\) into
\[
  \diff f \coloneq \sum_{j=1}^n \partial_j f\, \diff x_j.
\]
Now, for any \(k\)-form \(\omega \coloneq \sum_J f_J\, \diff x_J\), where \(k
\geq 1\) and \(f_J: E \to \R\) is again a map of class \(C^1\) (a \(0\)-form) we
associate the \((k+1)\)-form \(\diff \omega\) --- which is defined by
\[
  \diff \omega \coloneq \sum_J \diff f_J \wedge \diff x_{J}.
\]
\end{definition}

\begin{example}
\label{exp:curve-integral}
Let \(E \subseteq \R^n\) be an open set, and \(\gamma: [0, 1] \to E\) be a
\(1\)-surface (that is, a continuous differentiable curve). If we let \(f: E \to
\R\) be a \(C^1\) map, we have from the definition that the integral over the
curve \(\gamma\) of \(\diff f\) is given by --- recalling \cref{thm:
multi-comp-diff},
\[
  \int_{\gamma} \diff f =
  \int_0^1 \sum_{j=1}^n \partial_j f (\gamma(t)) \gamma_j'(t)\, \diff t
  = \int_0^1 \diff (f \gamma)(t)\, \diff t
  = f \gamma(1) - f \gamma(0).
\]
since \(f \gamma: [0, 1] \to \R\) is a continuously differentiable map.
\end{example}

\begin{theorem}
\label{thm:differential-operator-forms-properties}
The following are properties of the differential operator on forms. Let \(E
\subseteq \R^n\) be some open set.
\begin{enumerate}[(a)]\setlength\itemsep{0em}
\item (Skew-product rule) Let \(\omega\) be a \(k\)-form and \(\gamma\) be a
  \(m\)-form, both of class \(C^1(E)\). Then
  \[
    \diff (\omega \wedge \gamma)
    = (\diff \omega) \wedge \gamma + (-1)^k \omega \wedge \diff \gamma.
  \]
\item If \(\omega\) is a \(k\)-form of class \(C^2(E)\), then \(\diff(\diff
  \omega) = 0\).
\end{enumerate}
\end{theorem}

\begin{proof}
\begin{enumerate}[(a)]\setlength\itemsep{0em}
\item Let \(\omega \coloneq \sum_I f_{I} \diff x_I\) and \(\gamma(x)
  \coloneq \sum_J g_{J} \diff x_{J}\) for \(C^1\) coefficients \(f_I, g_J: E
  \rightrightarrows \R\) --- if \(k\) or \(m\) are zero, we just omit the
  \(1\)-forms from the definitions. From the wedge product we have
  \[
    \diff(\omega \wedge \gamma)
    = \sum_{I, J} \diff\left( f_I \cdot g_J\, \diff x_I \wedge \diff x_J \right)
    = \sum_{I, J} \diff(f_I \cdot g_J) \wedge \diff x_I \wedge \diff x_J.
  \]
  Define, for each pair \(I\) and \(J\), the indexing sequence \((\!(I, J)\!)\)
  consisting of the increasing ordered union of the sequences \(I\) and
  \(J\). Moreover, for each \((\!(I, J)\!)\), there will be an associated sign
  \begin{equation}
    \label{eq:sequence-index-sign}
    \sign(I, J) \coloneq |\{j - i \colon j - i < 0 \text{ for }
    (i, j) \in I \times J\}|,
  \end{equation}
  that is, the number of times the indices of \(J\) is greater than the ones
  from \(I\). From the skew-commutativity property
  (\cref{prop:exterior-algebra-associative-skew-commutative}),
  \begin{align*}
    \diff (\omega \wedge \gamma)
    &= \sum_{I, J} \sign(I, J)\,
      \diff(f_{I} \cdot g_{J}) \wedge \diff x_{(\!(I, J)\!)} \\
    &= \sum_{(I, J)} \sign(I, J)
    \left( \diff f_I \cdot g_J + f_I \cdot \diff g_J \right)
    \wedge \diff x_{(\!(I, J)\!)} \\
    &= \sum_{I, J} \left( \diff f_I \cdot g_J + f_I \cdot \diff g_J \right)
    \wedge \diff x_I \wedge \diff x_J.
  \end{align*}
  From the distributive property, associativity and skew-commutativity we get
  \begin{align*}
    \diff(\omega \wedge \gamma)
    &= \sum_{I, J} (g_J\, \diff f_I \wedge \diff x_I \wedge \diff x_J
    +  f_I\, \diff g_J \wedge \diff x_I \wedge \diff x_J) \\
    &= \sum_{I, J} (\diff f_I \wedge \diff x_I) \wedge (g_J\, \diff x_J)
      + (-1)^k (f_I\, \diff x_I) \wedge (\diff g_J \wedge \diff x_J) \\
    &= \sum_{I, J} (\diff f_I \wedge \diff x_I) \wedge (g_J\, \diff x_J)
      + (-1)^k \sum_{I, J} (f_I\, \diff x_I) \wedge (\diff g_J \wedge \diff x_J)
    \\
    &= \bigg( \sum_I \diff f_I \wedge \diff x_I\bigg) \wedge
      \bigg( \sum_J g_J\, \diff x_J \bigg)
      + (-1)^k \bigg( \sum_I f_I\, \diff x_I \bigg) \wedge
      \bigg( \sum_J \diff g_J \wedge \diff x_J \bigg) \\
    &= \diff \omega \wedge \lambda + (-1)^k \omega \wedge \diff \lambda.
  \end{align*}
\item For the case of a zero form \(f: E \to \R\), of class \(C^2\), we have
  \[
    \diff(\diff f) =
    \diff \bigg( \sum_{j=1}^n \partial_j f\, \diff x_j \bigg)
    = \sum_{j = 1}^n \diff (\partial_j f)\, \diff x_j
    = \sum_{i, j = 1}^n \partial_{i j} f\, \diff x_i \wedge \diff x_j
  \]
  Notice however that \(\partial_{i j} f = \partial_{j i} f\) for any \(1 \leq
  i, j \leq n\), thus each pair \((i, j)\) and \((j, i)\) of the sum cancel with
  each other --- since \(\diff x_i \wedge \diff x_j = - \diff x_j \wedge \diff
  x_i\) --- which implies in \(\diff(\diff f) = 0\).

  For the general case, if \(\omega = \sum_I f_I\, \diff x_I\) is any \(C^2\)
  \(k\)-form, since \(\diff^2 \omega = \sum_I \diff^2 f_I\, \diff x_I\) and
  \(\diff^2 f_I = 0\), for all \(I\), it follows immediately that \(\diff^2
  \omega = 0\).
\end{enumerate}
\end{proof}

\subsection{Change of Variables --- The Pullback Operation}

Let \(\omega \coloneq \sum_I f_I\, \diff x_I\) be a \(k\)-form in an open set
\(E \subseteq \R^n\) and \(\phi: V \to E\) be a \(C^1\) map from another open
set \(V \subseteq \R^m\). Notice that there arises a \emph{natural pullback}
operation \(\phi^{*}\) that allows for a \emph{change of variables}
\begin{equation}
\label{eq:change-of-variables-forms}
  \phi^{*}(\omega) = \sum_I f_I \phi\, \diff v_I,
\end{equation}
where \(v_I\) represents the coordinates coming from \(V\).

\begin{proposition}[Pullback properties]
\label{prop:algebraic-properties-pullback-forms}
Let \(\omega\) and \(\lambda\) be, respectively, a \(k\)-form and an \(m\)-form
in the open set \(E \subseteq \R^n\). Let \(\phi: V \to E\) be a \(C^{1}\)
map. The following properties hold:
\begin{enumerate}[(a)]\setlength\itemsep{0em}
\item If \(k = m\), then \(\phi^{*}(\omega + \lambda) = \phi^{*}(\omega) +
  \phi^{*}(\lambda)\).
\item \(\phi^{*}(\omega \wedge \lambda) = \phi^{*}(\omega) \wedge \phi^{*}
  (\lambda)\).
\item If \(\omega\) is \(C^{1}\) and \(\phi\) is of class \(C^2\), then
  \(\diff(\phi^{*}(\omega)) = \phi^{*}(\diff \omega)\).
\end{enumerate}
\end{proposition}

\begin{proof}
Let \(\omega \coloneq \sum f_I\, \diff x_I\) and \(\lambda \coloneq \sum g_J\,
\diff x_J\).
\begin{enumerate}[(a)]\setlength\itemsep{0em}
\item If \(k = m\), then \(\omega + \lambda\) is a \(k\)-form in \(E\), and we
  have
  \[
    \phi^{*}(\omega + \lambda)
    = \phi^{*} \left( \sum f_I\, \diff x_I + \sum g_J\, \diff x_J \right)
    = \sum f_I \phi\, \diff v_I + \sum g_J \phi\, \diff v_J
    = \phi^{*}(\omega) + \phi^{*}(\lambda).
  \]
\item Notice that
  \begin{align*}
    \phi^{*}(\omega \wedge \lambda)
    &= \phi^{*} \bigg(
      \sum_{I, J} f_I \cdot g_J\, \diff x_I \wedge \diff x_J
    \bigg)
    = \sum_{I, J} (f_I \cdot g_J) \phi\, \diff v_I \wedge \diff v_J
    = \sum_{I, J} (f_I \phi) \cdot (g_J \phi)\, \diff v_I \wedge \diff v_J \\
    &= \sum_{I, J} (f_I \phi\, \diff v_I) \wedge (g_J \phi\, \diff v_J)
    = \phi^{*}(\omega) \wedge \phi^{*}(\lambda).
  \end{align*}
\item We first prove the equality for the base case of a \(0\)-form. Let \(h: E
  \to \R\) be a \(0\)-form, then
  \begin{align*}
    \phi^{*} (\diff h)
    &= \phi^{*} \bigg( \sum_{i=1}^n \partial_i h\, \diff x_i \bigg)
    = \sum_{i=1}^n (\partial_i h) \phi\, \diff v_i
    = \sum_{i=1}^n (\partial_i h) \phi
      \bigg( \sum_{j=1}^{n} \partial_j v_i\, \diff x_j \bigg)
    = \sum_{j=1}^n \partial_j (h \phi)\, \diff x_j \\
    &= \diff (\phi^{*}(h))
  \end{align*}
  We now turn to the general case. Notice that \(\phi^{*}(\omega) = \sum_I
  \phi^{*}(f_I)\, \diff v_I = \sum_I \phi^{*}(f_I)\, \phi^{*}(\diff x_{I})\),
  since \(\phi^{*} (\diff x_I) = \diff v_I\). Therefore, assuming \(\phi\) is of
  class \(C^2\), we have
  \begin{align*}
    \diff (\phi^{*} \omega)
    &= \sum_I \diff(\phi^{*} f_{I}) \wedge \diff v_{I}
    = \sum_I \phi^{*}(\diff f_I) \wedge \phi^{*} (\diff x_{I})
    = \sum_I \phi^{*}(\diff f_I \wedge \diff x_I)
    = \phi^{*} \bigg( \sum_I \diff f_I \wedge \diff x_I \bigg) \\
    &= \phi^{*}(\diff \omega).
  \end{align*}
  The need for \(\phi\) to be \(C^2\) comes from the fact that the first
  equality --- by \cref{thm:differential-operator-forms-properties} --- is
  obtained by noting that
  \[
    \diff (\phi^{*}(f)\, \diff v_I)
    = \diff (\phi^{*}(f)) \wedge \diff v_I
      + \phi^{*}(f) \wedge \underbrace{\diff(\diff v_I)}_{0}
    = \diff (\phi^{*}f) \wedge \diff v_I.
  \]
\end{enumerate}
\end{proof}

\begin{lemma}[Pullback composition]
\label{lem:pullback-composition-forms}
Let \(\omega\) be a \(k\)-form in an open set \(W \subseteq \R^p\). Consider two
composable \(C^1\) maps \(\phi: E \to V\) and \(\psi: V \to W\), where \(U
\subseteq \R^n\) and \(V \subseteq \R^m\). Then the following equality holds
\[
  \phi^{*}(\psi^{*}(\omega)) = {(\psi \phi)}^{*}(\omega).
\]
Moreover, these are \(k\)-forms in \(U\).
\end{lemma}

\begin{proof}
Define \(\omega \coloneq \sum_I f_I\, \diff w_I\). Thus
\begin{align*}
  \phi^{*}(\psi^{*}(\omega))
  = \phi^{*} \bigg( \sum_I f_I \psi\, \diff v_I \bigg)
  = \sum_I (f_I \psi) \phi\, \diff u_I
  = \sum_I f_I (\psi \phi)\, \diff u_I
  = (\psi \phi)^{*} (\omega).
\end{align*}
\end{proof}

\begin{lemma}
\label{lem:integral-on-k-surface}
Let \(\omega\) be a \(k\)-form in an open set \(E \subseteq \R^n\). We consider
two \(k\)-surfaces \(\Phi: D \to E\) and \(\Id_D: D \to \R^k\), with \(\Id_D(t)
\coloneq t\) --- where \(D \subseteq \R^k\) is a compact set. Then
\[
  \int_{\Phi} \omega = \int_{\Id_D} \Phi^{*}(\omega).
\]
\end{lemma}

\begin{proof}
Define \(\omega \coloneq \sum_I f_I\, \diff x_I\), and \(\Phi_j \coloneq \pi_j
\Phi\) as the \(j\)-th projection of the surface \(\Phi\) --- that is
\[
  \Phi^{*}(\omega) = \sum_I f_I \Phi\, \diff \Phi_I.
\]
Defining \(I \coloneq (i_p)_{p=1}^k\), and since \(\diff \Phi_{i_p} =
\sum_{q=1}^k \partial_q \Phi_{i_p}\, \diff t_{q}\) for all \(1 \leq p \leq k\),
we see that
\begin{align*}
  \diff \Phi_{i_1} \wedge \dots \wedge \diff \Phi_{i_k}
  &= \bigg( \sum_{q=1}^k \partial_q \Phi_{i_1}\, \diff t_{q} \bigg)
    \wedge \dots \wedge \bigg( \sum_{q=1}^k \partial_q \Phi_{i_k} \diff t_q \bigg)
  \\
  &= \sum_{1 \leq i_1, \dots, i_k \leq k} \left[
    \prod_{q=1}^k \partial_q \Phi_{i_q}
  \right]\, \diff t_{i_1} \wedge \dots \wedge \diff t_{i_k} \\
  &= \sum_{1 \leq i_1, \dots, i_k \leq k} \left[
    \prod_{q=1}^k \partial_q \Phi_{i_q}
  \right] \left(
    \sign(i_1, \dots, i_k)\, \diff t_{1} \wedge \dots \wedge \diff t_{k}
  \right) \\
  &= \det [\partial_q \Phi_{i_p}]_{p, q = 1}^{n}\,
    \diff t_1 \wedge \dots \wedge \diff t_k \\
  &= \det(\Jac \Phi)\, \diff t_1 \wedge \dots \wedge \diff t_k.
\end{align*}
Where the \(\sign\) map is defined as in
\cref{eq:sequence-index-sign}. Therefore, combining equations we find
\[
  \Phi^{*}(\omega) = \sum_I f_I \Phi\, \det(\Jac \Phi)\,
  \diff t_1 \wedge \dots \wedge \diff t_k,
\]
thus indeed
\begin{align*}
  \int_{\Id_D} \Phi^{*}(\omega)
  &= \int_{\Id_D} \sum_I
    f_I \Phi(t)\, \det(\Jac \Phi(t))\, \diff t_1 \wedge \dots \wedge \diff t_k
  \\
  &= \int_D \sum_I f_I \Phi(t)\, \det(\Jac \Phi(t))\, \diff t \\
  &= \int_{\Phi} \omega.
\end{align*}
\end{proof}

\begin{theorem}
\label{thm:change-of-variable-integral-over-surface-forms}
Let \(\psi: V \to E\) be a map of class \(C^1\) on open sets \(V \subseteq
\R^m\), and \(E \subseteq \R^n\). Consider a \(k\)-surface \(\Phi: D \to V\),
where \(D \subseteq \R^k\) is compact, and a \(k\)-form \(\omega\) in
\(E\). Then we have the following equality
\[
  \int_{\psi \Phi} \omega = \int_{\Phi} \psi^{*}(\omega).
\]
\end{theorem}

\begin{proof}
The theorem is a direct consequence of the preceding lemmas --- notice that
\[
  \int_{\psi \Phi} \omega
  = \int_{\Id_D} (\psi \Phi)^{*} \omega
  = \int_{\Id_D} \Phi^{*}(\psi^{*}(\omega))
  = \int_{\Phi} \psi^{*}(\omega).
\]
\end{proof}

\subsection{Simplexes and Chains}

\begin{notation}
For the purposes of this section, we are going to define the collection
\[
  \Delta_{\text{c}}^k \coloneq \left\{
    (t_{1}, \dots, t_k) \in \R^k \colon
    \sum_{j=1}^k t_j \leq 1 \text{, and }
    t_j \geq 0 \text{ for all } 1 \leq j \leq k
  \right\}.
\]
which represents the \emph{\(k\)-simplex} obtained by the corner of a
\(k\)-dimensional unit cube.
\end{notation}

\begin{definition}[Affine map]
\label{def:affine-map}
Given vector spaces \(V\) and \(L\), we say that \(f: V \to L\) is a affine map
if \(f(x) = \ell + \phi(x)\) for all \(x \in V\), where \(\phi: V \to L\) is a
\(k\)-linear map and \(\ell = f(0) \in L\).
\end{definition}

\begin{definition}[Oriented affine \(k\)-simplex]
\label{def:oriented-affine-k-simplex}
Let \(p_0, \dots, p_k \in \R^n\) be any points. We define the oriented affine
\(k\)-simplex induced by the points \(p_0, \dots, p_k\) in \(\R^n\) as an affine
map \(\sigma: \Delta_{\text{c}}^k \to \R^n\) given by
\[
  \sigma(\alpha) \coloneq p_0 + \sum_{j=1}^k \alpha_j (p_j - p_0)
  \text{, for all } \alpha \in \Delta_{\text{c}}^k.
\]
We can also denote \(\sigma\) by \([p_0, \dots, p_k]\) as in
\cref{def:orientation}. If we define the \(\R\)-linear map \(A: \R^k \to \R^n\)
to be given by \(A(e_j) \coloneq p_j - p_0\), then we can write \(\sigma(\alpha)
= p_0 + A(\alpha)\).

If \(\tau \in S_{k+1}\) is a permutation on \(k+1\) elements, and \(\mu \coloneq
[p_{\tau(0)}, p_{\tau(1)}, \dots, p_{\tau(k)}]\) is an oriented affine
\(k\)-simplex, also induced by the points \(p_0, \dots, p_k\), we say that
\[
  \tau = \sign(\tau) \sigma.
\]
Moreover, \(\mu\) is said to have the \emph{same orientation} of \(\sigma\) if
\(\sign(\tau) > 0\), otherwise, if \(\sign(\tau) < 0\), then \(\mu\) is said to
have the \emph{opposite orientation} of \(\sigma\).

In the special case where \(k = n\) and the collection \(\{p_{j} -
p_0\}_{j=1}^k\) is linearly independent, we say that:
\begin{itemize}\setlength\itemsep{0em}
\item \(\sigma\) is \emph{positively} oriented if \(\det A > 0\).
\item \(\sigma\) is \emph{negatively} oriented if \(\det A < 0\).
\end{itemize}
In particular, the simplex \(\Id_{\R^k} \coloneq [0, e_1, \dots, e_k]\) in
\(\R^k\) is positively oriented.
\end{definition}

\begin{example}
\label{exp:oriented-0-simplex}
A special example of an oriented affine \(k\)-simplex occurs for the case \(k =
0\), where we get a simplex induced by a single point \(p_0 \in \R^k\) --- in
such case, two \(0\)-simplexes are conceivable, \(\sigma = p_0\) or \(\sigma =
-p_0\).
\end{example}

\begin{definition}
\label{def:oriented-0-simplex-integral-0-form}
Let \(\sigma = \varepsilon p_0\) be an oriented affine \(0\)-simplex --- where
\(\varepsilon \in \{-1, 1\}\) and \(p_0 \in \R^k\). If \(f: E \to \R\) is a
\(0\)-form in the open set \(E \subseteq \R^n\), we define its integral over the
\(0\)-simplex \(\sigma\) as
\[
  \int_{\sigma} f \coloneq \varepsilon f(p_0).
\]
\end{definition}

\begin{proposition}\label{prop:integral-simplex-change-orientation}
Let \(\sigma: \Delta_{\text{c}}^k \to E\) be an oriented affine \(k\)-simplex in
an open set \(E \subseteq \R^n\), and \(\omega\) be any \(k\)-form in \(E\). If
\(\mu = \varepsilon \sigma\) where \(\varepsilon \in \{-1, 1\}\)\footnote{If
\(\sigma = [p_0, \dots, p_k]\), we say that \(\mu = \sign(\tau) \sigma\) if
\(\mu = [p_{\tau(0)}, \dots, p_{\tau(k)}]\) --- where \(\tau \in S_{k+1}\) is a
permutation on \(k+1\) elements.}, then
\[
  \int_{\mu} \omega = \varepsilon \int_{\sigma} \omega.
\]
\end{proposition}

\begin{proof}
For the base case \(k=0\), we have
\cref{def:oriented-0-simplex-integral-0-form}. Now, let \(k \geq 1\) and define,
for each \(1 \leq j_0 \leq k\), the transposition \(\tau_{j_0} \in S_{k+1}\)
given by \(\tau_{j_0}(0) = j_0\) and \(\tau_{j_0}(j_0) = 0\) --- while
\(\tau_{j_0}(i) = i\) for all \(i \neq 0, j_0\). Suppose \(\sigma \coloneq [p_0,
\dots, p_{k}]\), then since \(\sign(\tau_{j_0}) = -1\), the oriented affine
\(k\)-simplex \(\mu \coloneq [p_{\tau_{j_{0}}}(0), \dots, p_{\tau_{j_0}(k)}]\) is
such that \(\mu = \sign(\tau_{j_0}) \sigma = -\sigma\) --- moreover,
\[
  \mu(u) = p_{j_0} + \sum_{j=1}^k u_i (p_{\tau_{j_0}(i)} - p_{j_0})
  \coloneq p_{j_0} + B(u)
  \text{, for all } u \in \Delta_{\text{c}}^k,
\]
where \(B: \R^k \to \R^n\) is a linear map with matrix representation \(B =
[p_{\tau_{j_0}(j)} - p_{j_0}]_{j=1}^k\) (notice we are disregarding the column
with \(j = 0\)), where each \(p_{\tau_{j_0}(j)} - p_{j_0} = [\partial_j
\mu_i]_{i=1}^n\) is the \(j\)-th column of \(B\). Therefore, if \(\omega
\coloneq \sum_I f_I\, \diff x_{I}\) and \(I \coloneq (i_p)_{p=1}^k\) for each
\(I\), we obtain
\[
  \int_{\mu} \omega
  = \int_{\Delta_{\text{c}}^k}
    \sum_I f_I \mu(t) \det(B)\, \diff t.
\]

If we let \(A: \R^k \to \R^n\) be the linear map defined by \(A(u) \coloneq
\sum_{i=1}^k u_i (p_i - p_0)\) for all \(u \in \R^k\), we can write \(\sigma(u)
= p_0 + A(u)\). Notice however that, for each \(1 \leq i \leq k\) with \(i \neq
j_0\), we have the relation \(B(e_i) = A(e_i) - A(e_{j_0})\) (which does not
affect the value of the determinant) --- while \(B(e_{j_0}) = - A(e_{j_0})\)
(which multiplies the determinant by \(-1\)). Therefore \(\det B = - \det
A\). Hence,
\begin{equation}
\label{eq:dets-change-sign-int-form}
  \int_{\mu} \omega
  = \int_{\Delta_{\text{c}}^k} \sum_I f_I \mu(t) (- \det A)\, \diff t
  = \int_{\Delta_{\text{c}}^k} \sum_{I} f \sigma(t) \det(A)\, \diff t
  = \int_{\sigma} \omega.
\end{equation}

For the case where we transpose a pair of indices \(i\) and \(j\), with \(0 < i
< j \leq k\), then \(\mu(u) = p_0 + C(u)\) --- where \(C\) has all columns
matching to those of \(A\), except the interchange between the \(i\)-th and
\(j\)-th columns, which implies in \(\det C = - \det A\). Therefore we get again
the same as in \cref{eq:dets-change-sign-int-form}, which finishes the proof.
\end{proof}

\begin{definition}[Affine \(k\)-chain]
\label{def:affine-k-chain}
Let \(E \subseteq \R^n\) be an open set. A finite collection of oriented affine
\(k\)-simplexes in \(E\) is said to be an \emph{affine \(k\)-chain} in \(E\). It
is to be noted that the finite collection may contain some \emph{multiplicities}
--- that is, some of the \(k\)-simplexes can coincide.

If \(\Gamma \coloneq (\sigma_j)_{j=1}^m\) is an affine \(k\)-chain in \(E\), we
define
\[
  \int_{\Gamma} \omega \coloneq \sum_{j=1}^m \int_{\sigma_j} \omega.
\]
We'll usually denote \(\Gamma\) by the \emph{formal sum} \(\sum_{j=1}^m \sigma_j\)
\emph{mapping} \(\omega \xmapsto{\Gamma} \sum_j \int_{\sigma_j} \omega\).
\end{definition}

\begin{definition}[Boundary]
\label{def:affine-k-simplex-boundary}
Let \(k \geq 1\). Given an oriented affine \(k\)-simplex \(\sigma \coloneq [p_0,
\dots, p_k]\), we define the \emph{boundary} of \(\sigma\) to be an affine
\((k-1)\)-chain defined by
\[
  \partial \sigma \coloneq
  \sum_{j=0}^k (-1)^j [p_0, \dots, p_{j-1}, p_{j+1}, \dots, p_k].
\]
\end{definition}

\begin{definition}[Differentiable simplexes and chains]
\label{def:differentiable-simplexes-and-chains}
Let \(E \subseteq \R^n\) and \(V \subseteq \R^m\) be open sets. If \(\phi: E \to
V\) is a map of class \(C^2\), and \(\sigma\) is an oriented affine
\(k\)-simplex in \(E\), then the induced map
\[
  \Phi \coloneq \phi \sigma: \Delta_{\text{c}}^k \longrightarrow V
\]
is a \(k\)-surface in \(V\). The surface \(\Phi\) is said to be an
\emph{oriented \(k\)-simplex of class \(C^2\)} in \(V\). Moreover, \(\Phi\) has
a boundary defined by
\[
  \partial \Phi \coloneq \phi(\partial \sigma),
\]
therefore, \(\partial \Phi\) is a \((k-1)\)-chain of class \(C^2\) in \(V\).

A finite collection \(\Psi \coloneq (\Phi_j)_{j=1}^r\) of oriented
\(k\)-simplexes of class \(C^2\) in \(V\) is said to be a \emph{\(k\)-chain of
class \(C^2\) in \(V\)}. Given a \(k\)-form \(\omega\) in \(V\), we define
\[
  \int_{\Psi} \omega \coloneq \sum_{j=1}^r \int_{\Phi_j} \omega.
\]
As expected, the boundary of \(\Psi\) is defined to be the \((k-1)\)-chain of
class \(C^2\) in \(V\) given by \(\partial \Psi \coloneq \sum_{j=1}^r \partial
\Phi_j\).
\end{definition}

In the context of the last definition, given an affine \(k\)-chain \(\Gamma
\coloneq \sum_{j=1}^r \sigma_j\) in \(E\), the map \(\phi\) induces a
corresponding \(k\)-chain of class \(C^2\) in \(V\), namely, \(\sum_{j=1}^r \phi
\sigma_j\).

\subsection{Oriented Boundaries on Sets}

\begin{definition}[Positively oriented boundaries on a set]
\label{def:positively-oriented-boundary-on-sets}
Let \(\phi: \Delta_{\text{c}}^n \mono \R^n\) be an injective map of class
\(C^2\), whose Jacobian determinant is positive in the interior of
\(\Delta_{\text{c}}^k\). Let \(E \coloneq \phi(\Delta_{\text{c}}^n)\) --- then,
by the inverse map theorem (see \cref{thm:inverse-map}), \(E\) is the closure of
an open set of \(\R^n\). We define the \emph{positively oriented boundary of
  the set} \(E\) to be the \((n-1)\)-chain
\[
  \partial E \coloneq \partial \phi = \phi(\partial \Id_{\R^n}),
\]
where \(\Id_{\R^n} \coloneq [0, e_1, \dots, e_n]\) --- that is to say that, for
any \((n-1)\)-form \(\omega\) in \(E\), we have
\[
  \int_{\partial E} \omega \coloneq \int_{\partial \phi} \omega.
\]

If \(\{E_{j}\}_{j=1}^r\) is a collection of subsets of \(\R^n\) with disjoint
\emph{interior}, let \(\{\phi_{j}: \Delta_{\text{c}}^n \mono \R^n\}_{j=1}^r\) be
an associated collection of injective \(C^2\)-maps with positive Jacobian
determinant in the interior of \(\Delta_{\text{c}}^n\), and such that
\(\phi_j(\Delta_{\text{c}}^n) = E_j\). We define the positively oriented
boundary of the set \(\Omega \coloneq E_1 \cup \dots \cup E_r\) as the
\((n-1)\)-chain
\[
  \partial \Omega \coloneq \partial \phi_1 + \dots + \partial \phi_{r}
  \text{,}\ \text{ that is }
  \int_{\partial \Omega} \omega \coloneq \sum_{j=1}^r \int_{\partial \phi_j} \omega,
\]
where \(\omega\) is any \((n-1)\)-form in \(\Omega\).
\end{definition}

\begin{proposition}
\label{prop:same-boundary-same-integral-over-form}
Let \(\phi, \psi: \Delta_{\text{c}}^n \rightrightarrows \R^n\) be injective
\(C^2\)-maps with positive Jacobian determinant in the interior of
\(\Delta_{\text{c}}^n\). If \(\phi(\Delta_{\text{c}}^n) =
\psi(\Delta_{\text{c}}^n)\), then \(\partial \phi = \partial \psi\) --- that is,
for every \((n-1)\)-form \(\omega\) in the image of the maps, we have
\[
  \int_{\partial \phi} \omega = \int_{\partial \psi} \omega.
\]
\end{proposition}

\begin{proof}
\todo[inline]{Prove}
\end{proof}

\begin{example}
Let \(S: [0, \pi] \times [0, 2\pi] \to \R^3\) be a \(2\)-surface in \(\R^3\),
defined by
\[
  S(u, v) \coloneq (\sin(u) \cos(v), \sin(u) \sin(v), \cos(u)).
\]
Notice that the positively oriented boundary of \(S\) corresponds to the \(4\)
distinct curves induced by \(S\) under the positively oriented boundary of the
rectangle \([0, \pi] \times [0, 2 \pi]\) --- that is, \(\partial S =
\sum_{j=1}^4 \gamma_j\) where we have
\begin{align*}
  \gamma_1:& [0, \pi] \longrightarrow \R^3
  \text{ mapping } u \longmapsto S(u, 0) = (\sin(u), 0, \cos(u)); \\
  \gamma_2:& [0, 2 \pi] \longrightarrow \R^3
  \text{ mapping } v \longmapsto S(\pi, v) = (0, 0, -1); \\
  \gamma_3:& [0, \pi]\longrightarrow \R^3
  \text{ mapping } u \longmapsto S(\pi - u, 2 \pi) = (\sin(u), 0, -\cos(u)); \\
  \gamma_4:& [0, 2 \pi] \longrightarrow \R^3
  \text{ mapping } v \longmapsto S(0, 2\pi - v) = (0, 0, 1). \\
\end{align*}
Now, given any \(1\)-form \(\omega \coloneq f_1\, \diff x_1 + f_2\, \diff x_2 +
f_3\, \diff x_3\) in \(\R^3\), we have
\begin{align*}
  \int_{\partial S} \omega
  &= \sum_{j=1}^4 \int_{\gamma_j} \omega \\
  &= \int_0^{\pi} \sum_{i=1}^3
    f_i \gamma_1(x)\, \underbrace{\partial \gamma_1^{(i)}(x)}_{0}\, \diff x
  + \int_{\gamma_2} \omega
  + \int_{\gamma_3} \omega
  + \int_0^{2 \pi} \sum_{i=1}^3
    f_i \gamma_4(x)\, \underbrace{\partial \gamma_4^{(i)}(x)}_{0}\, \diff x \\
  &= \int_{\gamma_2} \omega + \int_{\gamma_3} \omega.
\end{align*}
Notice however that if we make the change of variables from \(x\) to \(\pi - x\)
in the integral \(\int_{\gamma_3} \omega\), we obtain
\begin{align*}
  \int_{\gamma_3} \omega
  &= \int_0^{\pi} - f_1 \gamma_3(x) \cos(x) + f_3 \gamma_3(x) \sin(x)\, \diff x
  \\
  &= - \int_{\pi}^0 - f_1\gamma_3(\pi - x) \cos(x) + f_{3} \gamma_{3}(\pi - x)
  \sin(x)\, \diff x
  \\
  &= \int_0^{\pi} -f_1\gamma_2(x) \cos(x) + f_3\gamma_2(x) \sin(x) \, \diff x
  \\
  &= - \int_{\gamma_2} \omega
\end{align*}
Where the third equality may be obtained by noting that \(\gamma_3(\pi - x) =
\gamma_2(x)\). We conclude that
\[
  \int_{\partial S} \omega = 0,
\]
and, more generally, \(\partial S = 0\).
\end{example}

\subsection{Stoke's Theorem}

\begin{theorem}[Stoke's]
\label{thm:stokes}
Let \(\Psi\) be a \(k\)-chain of class \(C^2\) in an open set \(V \subseteq
\R^n\), and let \(\omega\) be a \((k-1)\)-form of class \(C^1\) in \(V\). Then
the following equality holds
\[
  \int_{\Psi} \diff \omega = \int_{\partial \Psi} \omega.
\]
\end{theorem}

\begin{proof}
Since \(\Psi = \sum_{j=1}^r \Phi_j\) for some finite collection
\((\Phi_j)_{j=1}^r\) of oriented \(k\)-simplexes of class \(C^2\), it suffices
to prove that the theorem holds for any oriented \(k\)-simplex of class \(C^2\),
since \(\int_{\Psi} \gamma = \sum_{j=1}^r \int_{\Phi_j} \gamma\) for any
\((k-1)\)-form \(\gamma\) in \(V\).

Let \(\Phi\) be any oriented \(k\)-simplex of class \(C^2\) in \(V\), and
consider the positively oriented affine \(k\)-simplex \(\Id_{\R^k} \coloneq [0,
e_1, \dots, e_k]: \Delta_{\text{c}}^k \to \R^k\). From
\cref{def:differentiable-simplexes-and-chains}, there must exist an open set \(E
\subseteq \R^k\) with \(\Delta_{\text{c}}^k \subseteq E\), and a \(C^2\)-map
\(\phi: E \to V\) such that \(\Phi = \phi \Id_{\R^k}\). Therefore, given any
\((k-1)\)-form \(\omega\) in \(V\), we have
\[
  \int_{\Phi} \diff \omega
  = \int_{\phi \Id_{\R^k}} \diff \omega
  = \int_{\Id_{\R^k}} \phi^{*}(\diff \omega)
  = \int_{\Id_{\R^k}} \diff (\phi^{*} (\omega)).
\]
Moreover, we can also consider the positively oriented boundary of \(\Phi\),
which yields
\[
  \int_{\partial \Phi} \omega
  = \int_{\partial(\phi \Id_{\R^k})} \omega
  = \int_{\phi(\partial \Id_{\R^k})} \omega
  = \int_{\partial \Id_{\R^k}} \phi^{*}(\omega).
\]
Hence, our goal will be to prove that the theorem is true for any \((k-1)\)-form
of the type \(\phi^{*}(\omega)\), that is, any \((k-1)\)-form \(\gamma\) in
\(E\) it should be true that
\begin{equation}
\label{eq:goal-stokes}
  \int_{\Id_{\R^k}} \diff \gamma = \int_{\partial \Id_{\R^k}} \gamma.
\end{equation}

Let's first deal with the base case, \(k = 1\). Given any \(0\)-form \(\gamma: E
\to \R\) of class \(C^1\) in \(E \subseteq \R\), we have --- by the fundamental
theorem of calculus,
\[
  \int_{\Id_{\R}} \diff \gamma
  = \int_0^1 (\diff \gamma(x))\, \diff x
  = \gamma(1) - \gamma(0)
  = \int_{\partial \Id_{\R}} \gamma
\]

We now assume that \(k > 1\). Since any \((k-1)\)-form \(\gamma\) in \(E\) can
be written as a sum
\[
  \gamma =
  \sum_{r=1}^k f_r\, \diff x_1 \wedge \dots \wedge \diff x_{r-1} \wedge
    \diff x_{r+1} \wedge \dots \wedge \diff x_k
  \coloneq \sum_{r=1}^k \gamma_r,
\]
we may as well simply prove \cref{eq:goal-stokes} for every \(\gamma_r\), with
\(1 \leq r \leq k\) --- this is now what we'll do, for each index \(r\). The
boundary of \(\Id_{\R^k}\) is given by
\begin{equation}
\label{eq:stokes-standard-boundary}
  \partial \Id_{\R^k} = [e_1, \dots, e_k] + \sum_{j=1}^k (-1)^j \sigma_j,
\end{equation}
where we define each \(\sigma_j\) to be the oriented \((k-1)\)-simplex
\(\sigma_j \coloneq [0, e_1, \dots, e_{j-1}, e_{j+1}, \dots, e_k]\) --- that is,
removing the \(j\)-th component of \(\Id_{\R^k}\), for each \(1 \leq j \leq
k\). Moreover, to facilitate our analysis, we'll define another oriented
\((k-1)\)-simplex, \(\sigma_0\), defined by transposing the \(r\)-th component of
\(\Id_{\R^k}\) an amount of \(r-1\) times to the left, so that
\[
  \sigma_0 \coloneq [e_r, e_1, \dots, e_{r-1}, e_{r+1}, \dots, e_k]
  = (-1)^{r-1} [e_1, \dots, e_k].
\]
Therefore, we can rewrite \cref{eq:stokes-standard-boundary} as
\[
  \partial \Id_{\R^k}
  = (-1)^{r-1} \sigma_0 + \sum_{j=1}^k (-1)^{j} \sigma_j.
\]

We can analyse each of the \((k-1)\)-simplexes \(\sigma_j: \Delta_{\text{c}}^{k-1}
\to \R^k\), for \(0 \leq j \leq k\):
\begin{itemize}\setlength\itemsep{0em}
\item For the special case \(j = 0\), given any \(t \in
  \Delta_{\text{c}}^{k-1}\), if \(x \coloneq \sigma_0(t) \in \R^k\), then
  \begin{equation}
    \label{eq:stokes-sigma-0}
    x_i =
    \begin{cases}
      t_i,                    &1 \leq i < r \\
      1 - \sum_{i=1}^{k-1} t_i, &i = r \\
      t_{i-1},                 &r < i \leq k
    \end{cases}
  \end{equation}
  Therefore, the matrix \([\partial_q \sigma_0^{(p)}(t)]_{i, j}\), where \(1
  \leq p, q \leq k - 1\) and \(p \neq r\), corresponds to the identity matrix on
  \(\R^{k-1}\) restricted to \(\Delta_{\text{c}}^k\), therefore \(\det [\partial_q
  \sigma_0^{(p)}] = 1\) and thus
  \begin{equation}\label{eq:sigma-0-stokes}
    \int_{\sigma_0} \gamma_r
    = \int_{\Delta_{\text{c}}^{k-1}} f_r\sigma_r(t)\, \diff t.
  \end{equation}
\item For any \(j \neq 0\), given any \(t \in \Delta_{\text{c}}^k\), if \(x
  \coloneq \sigma_j(t) \in \R^k\), then
  \begin{equation}
    \label{eq:stokes-sigma-j}
    x_i =
    \begin{cases}
      t_i,    &1 \leq i < j \\
      0,      &i = j \\
      t_{i-1}, &j < i \leq k
    \end{cases}
  \end{equation}
  Moreover, since
  \[
    \int_{\sigma_j} \gamma_r
    = \int_{\Delta_{\text{c}}^k} f_r\sigma_j(t)
    \det{\left[
        \partial_q \sigma_j^{(p)}(t)
      \right]}_{\substack{p, q = 1 \\ p \neq r}}^{k-1}\,
    \diff t,
  \]
  there are two possible cases, for any \(t \in \Delta_{\text{c}}^k\): if \(j \neq
  r\), then the \(j\)-th row of the Jacobian matrix is entirely composed of
  zeros, hence \(\det[\partial_q \sigma_j^{(p)}(t)]_{p, q} = 0\), otherwise, if
  \(j = r\), the matrix corresponds exactly to the identity matrix of
  \(\R^{k-1}\) restricted to \(\Delta_{\text{c}}^{k-1}\) --- therefore
  \(\det[\partial_q \sigma_j^{(p)}(t)]_{p, q} = 1\) for any \(t \in
  \Delta_{\text{c}}^{k-1}\), which implies in
  \begin{equation}\label{eq:sigma-js-stokes}
    \int_{\sigma_j} \gamma_r =
    \begin{cases}
      0,                                               &j \neq r \\
      \int_{\Delta_{\text{c}}^{k-1}} f_r\sigma_r(t)\, \diff t, &j = r
    \end{cases}
  \end{equation}
\end{itemize}
With \cref{eq:sigma-0-stokes} and \cref{eq:sigma-js-stokes} at hand, we obtain
\begin{align*}
  \int_{\partial \Id_{\R^k}} \gamma_r
  &= (-1)^{r-1} \int_{\sigma_0} \gamma_r + \sum_{j=1}^r(-1)^j \int_{\sigma_j} \gamma_r
  \\
  &= (-1)^{r-1} \int_{\sigma_0} \gamma_r + (-1)^r \int_{\sigma_r} \gamma_r
  \\
  &= (-1)^{r-1} \int_{\Delta_{\text{c}}^{k-1}} f_r\sigma_0(t) - f_r\sigma_r(t)\,
    \diff t
\end{align*}

Notice, on the other hand, that
\begin{align*}
  \diff \gamma_r
  &= \diff f_r \wedge \diff x_1 \wedge \dots \wedge \diff x_{r-1}
    \wedge \diff x_{r+1} \wedge \dots \wedge \diff x_k \\
  &= \bigg(\sum_{j=1}^k \partial_j f_r\, \diff x_j\bigg) \wedge \diff x_1 \wedge
    \dots \wedge \diff x_{r-1} \wedge \diff x_{r+1} \wedge \dots \wedge \diff x_k
  \\
  &= \partial_r f_r\, \diff x_r \wedge \diff x_1 \wedge \dots \wedge \diff x_{r-1}
    \wedge \diff x_{r+1} \wedge \dots \wedge \diff x_k \\
  &= (-1)^{r-1} \partial_r f_r\, \diff x_1 \wedge \dots \wedge \diff x_k.
\end{align*}
Therefore, by Fubini's theorem (see \cref{thm:fubini}),
\begin{align*}
  \int_{\Id_{\R^k}} \diff \gamma
  &= (-1)^{r-1} \int_{\Delta_{\text{c}}^k} \partial_r f_r(x)\, \diff x \\
  &= (-1)^{r-1} \int_{\Delta_{\text{c}}^{k-1}} \left[
      \int_0^{1 - x_1 + \dots + x_{r-1} + x_{r+1} + x_{k-1}}
      \partial_r f_r(x)\, \diff x_r
    \right]\,
  \diff (x_1, \dots, x_{r-1}, x_{r+1}, \dots, x_k) \\
  &= (-1)^{r-1} \int_{\Delta_{\text{c}}^{k-1}} f_r\sigma_0(t) - f_r
    \sigma_r(t)\,
    \diff t.
\end{align*}
Where the last equality was obtained by means of \cref{eq:stokes-sigma-0} and
\cref{eq:stokes-sigma-j} since the variables are independent of each other. That
is, we just proved that
\[
  \int_{\Id_{\R^k}} \diff \gamma_r = \int_{\partial \Id_{\R^k}} \gamma_r,
\]
which implies in \cref{eq:goal-stokes} --- thus the proof is complete!
\end{proof}

%%% Local Variables:
%%% mode: latex
%%% TeX-master: "../../../deep-dive"
%%% End:

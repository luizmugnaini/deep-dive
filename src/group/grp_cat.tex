\section{\texorpdfstring{\(\Grp\)}{Grp} Category}

\begin{definition}[Group morphism]\label{def: grp-morphism}
Let \((G, \circledast)\) and \((H, \circledcirc)\) be groups together with
their binary operation. A group morphism --- also called homomorphism --- is a
map \(\varphi: (G, \circledast) \to (H, \circledcirc)\) such that the that the
following diagram commutes
\[
  \begin{tikzcd}
    G \times G \ar[d, swap, "\circledast"] \ar[r, "\varphi \times \varphi"]
    &H \times H \ar[d, "\circledcirc"] \\
    G \ar[r, "\varphi"] &H
  \end{tikzcd}
\]
Where \(\varphi \times \varphi\) is uniquely defined in \(\Set\) by
\cref{prop: product-morphism} --- mapping \((g, \ell) \xmapsto{\varphi \times
\varphi} (\varphi(g), \varphi(\ell))\). The commutativity of such diagram can
be viewed as the requirement that \(\varphi\) preserves the structure coming
from the binary operations --- that is, for any \(g, \ell \in G\)
\[
  \varphi(g \circledast \ell) = \varphi(g) \circledcirc \varphi(\ell).
\]
\end{definition}

\begin{definition}[Category of groups]\label{def: grp}
The category of groups \(\Grp\) consists of the collection of objects ---
called groups --- and group morphisms between them.
\end{definition}

\begin{proposition}
\(\Grp\) is a category.
\end{proposition}

\begin{proof}
Let \((G, \circledast)\), \((H, \circledcirc)\) and \((K, \star)\) be any
groups. The identity \(\Id_G: G \to G\) is a group morphism since \(\Id_G(g
\circledast \ell) = g \circledast \ell\) for any \(g, \ell \in G\). Moreover,
we can define a map
\[
  f: \Hom_{\Grp}(G, H) \times \Hom_{\Grp}(H, K) \to \Hom_{\Grp}(G, K)
\]
with the mapping \((\psi, \varphi) \xmapsto f \psi \varphi\) --- since
the following diagram commutes
\[
  \begin{tikzcd}
    G \times G \ar[r, "\varphi \times \varphi"]
    \ar[d, swap, "\circledast"]
    \ar[rr, bend left, "(\psi \varphi) \times (\psi \varphi)"]
    &H \times H \ar[r, "\psi \times \psi"]
    \ar[d, "\circledcirc"]
    &K \times K \ar[d, "\star"]
    \\
    G \ar[r, "\varphi"]
    \ar[rr, bend right, swap, "\psi \varphi"]
    &H \ar[r, "\psi"] &K
  \end{tikzcd}
\]
In other words, for any \(g, \ell \in G\) we have
\[
  \psi \varphi (g \circledast \ell)
  = \psi(\varphi(g \circledast \ell))
  = \psi(\varphi(g) \circledcirc \varphi(\ell))
  = \psi(\varphi(g)) \star \psi(\varphi(\ell))
  = \psi \varphi(g) \star \psi \varphi(\ell).
\]
Therefore \(\psi \varphi \in \Hom_\Grp(G, K)\). For the other part of the
diagram, we have
\[
  (\psi \varphi) \times (\psi \varphi) (g, \ell)
  = (\psi \varphi(g), \psi \varphi(\ell))
  = \psi \times \psi(\varphi(g), \varphi(\ell))
  =(\psi \times \psi) (\varphi \times \varphi) (g, \ell)
\]

Since group morphisms are maps in \(\Set\), we have that associativity is
inherited.
\end{proof}

\begin{proposition}\label{prop: forgetful-func-grp-set}
There exists a covariant forgetful functor \(F: \Grp \to \Set\).
\end{proposition}

\begin{proof}
For objects, define \(F\) as \(F(G, \circledast) = G\) --- where we denoted
\(G\) together with its binary operation only to express that the
multiplicative structure is lost in the process. Let \(\varphi: (G,
\circledast) \to (H, \circledcirc)\) be a group morphism, denote by
\(\overline \varphi \in \Mor(\Set)\) the function \(\overline\varphi: G \to
H\) such that \(\overline\varphi(g) = \varphi(g)\) for all \(g \in G\). For
such morphisms we define \(F\) as \(F\varphi = \overline\varphi: F(G,
\circledast) \to F(H, \circledcirc)\).

Let \(\psi \in \Hom_\Grp((H, \circledcirc), (K, \star))\), then we have
\(\overline{\psi \varphi} = \overline \psi \overline \varphi: H \to K\). Thus
\[
  F(\psi \varphi) = \overline{\psi \varphi}
  = \overline \psi \overline \varphi = F\psi F\varphi.
\]
This shows that \(F\) is a covariant forgetful functor.
\end{proof}

\begin{proposition}
The trivial group \(* \in \Grp\) is the initial and final object of \(\Grp\).
That is, for any \(G \in \Grp\) the diagram
\[
  \begin{tikzcd}
    * \ar[r, dashed, bend right, swap, "\varphi"]
    &G \ar[l, dashed, bend right, swap, "\psi"]
  \end{tikzcd}
\]
commutes for uniquely defined group morphisms \(\varphi\) and \(\psi\).
\end{proposition}

\begin{proof}
Let \(G \in \Grp\) be any group. We define maps \(\varphi: * \to G\) mapping
\(e \xmapsto\varphi e_G\), where \(e\) is the only element of \(*\) --- being
unique possible map \(* \to G\) that preserves the group structure. Clearly,
\(\varphi\) is a group morphism since \(\varphi(e e) = \varphi(e) = e_G =
\varphi(e) \varphi(e)\) --- this shows that \(*\) is the initial object of
\(\Grp\). Let \(\psi: G \to *\) be a map defined by \(g \xmapsto\psi e\) ---
which is clearly unique. Then \(\psi\) is a morphism of groups, because
\(\psi(g h) = e = \psi(g) \psi(h)\) --- showing that \(*\) is the final object
of \(\Grp\).
\end{proof}

\subsection{Properties of Morphisms}

\begin{proposition}[Monomorphisms, epimorphisms and isomorphisms]
\label{prop:mono-epi-iso-in-grp}
Let \(\phi\) be a morphism of groups. Then
\begin{enumerate}[(a)]\setlength\itemsep{0em}
\item \(\phi\) is a monomorphism if and only if \(\phi\) is injective.
\item \(\phi\) is an epimorphism if and only if \(\phi\) is surjective.
\item \(\phi\) is an isomorphism if and only if \(\phi\) is bijective.
\end{enumerate}
\end{proposition}

\begin{proof}
\todo[inline]{Prove}
\end{proof}

\begin{proposition}[Commuting on inverses]
\label{prop:grp-morphism-commute-inverse}
Let \((G, \circledast), (H, \circledcirc)\) be groups and consider \(\varphi \in
\Hom_\Grp(G, H)\). Define \(\operatorname{inv}_G: G \isoto G\) and
\(\operatorname{inv}_H: H \isoto H\) as the maps \(g \xmapsto
{\operatorname{inv}_G} g^{-1}\) and \(h \xmapsto {\operatorname{inv}_H}
h^{-1}\). Then the following diagram commutes
\[
  \begin{tikzcd}
    G \ar[r, "\varphi"]
    \ar[d, swap, "\operatorname{inv}_G"]
    &H \ar[d, "\operatorname{inv}_H"] \\
    G \ar[r, "\varphi"] &H
  \end{tikzcd}
\]
That is, \(\varphi(g^{-1}) = \varphi(g)^{-1}\) for every \(g \in G\).
\end{proposition}

\begin{proof}
Let \(g \in G\) be any element, then
\[
  \varphi(g^{-1}) = \varphi(g^{-1} e_G) = \varphi(g^{-1} \circledast g
  \circledast g^{-1}) = \varphi(g^{-1}) \circledcirc \varphi(g) \circledcirc
  \varphi(g^{-1}),
\]
applying cancellation law on the equation above we find
\[
  e_H = \varphi(g) \circledcirc \varphi(g^{-1}).
\]
Hence \(\varphi(g^{-1}) = {\varphi(g)}^{-1}\). Moreover, this implies that \(e_G
\xmapsto \varphi e_H\) since
\[
  \varphi(e_G) = \varphi(g \circledast g^{-1}) = \varphi(g) \circledcirc
  \varphi(g^{-1}) = \varphi(g) \circledcirc {\varphi(g)}^{-1} = e_H.
\]
\end{proof}

\begin{proposition}[Generators and unique extension]
\label{prop:generator-unique-extension}
Let \(G\) be a group and \(S\) be a generator set for \(G\). Let \(H\) be any
group and \(f: S \to H\) be a set-function. If there exists a morphism \(\phi: G
\to H\) such that \(\phi|_S = f\), then \(\phi\) is unique.
\end{proposition}

\begin{proof}
Let \(\psi: G \to H\) be another morphism satisfiying the condition specified
above --- then clearly \(\psi|_S = f = \phi|_{S}\), that is, \(\phi\) and
\(\psi\) agree on \(S\). Since \(S\) generates \(G\), every element \(g \in G\)
can be written as a finite product \(g = \prod_{j} s_j \in \langle S \rangle\)
thus
\[
  \psi(g) = \psi\Big( \prod_j s_j \Big)
  = \prod_j \psi(s_j) = \prod_j \phi(s_j)
  = \phi\Big( \prod_j s_j \Big) = \phi(g),
\]
which implies in \(\psi = \phi\).
\end{proof}

\begin{proposition}[Image subgroup]
\label{prop:morphism-image-subgroup}
Let \(\phi: G \to H\) be a morphism of groups, then \(\im \phi \subseteq H\) is
a subgroup of \(H\).
\end{proposition}

\begin{proof}
Let \(h \in \im \phi\) be any element and consider \(g \in \phi^{-1}(h)\), from
\cref{prop:grp-morphism-commute-inverse} we see that \(\phi(g^{-1}) =
\phi(g)^{-1} = h^{-1} \in \im \phi\), thus \(\im \phi\) is closed under
inverses. Moreover, given another \(h' \in \im \phi\), there exists \(g' \in
\phi^{-1}(h')\) and \(\phi(g g') = \phi(g) \phi(g') = h h' \in \im \phi\) ---
hence \(\phi\) is closed under products.
\end{proof}

\begin{definition}[Kernel]
We define the kernel of a morphism of groups \(\phi \in \Hom_\Grp(G, H)\) as
the collection \(\ker \phi = \{g \in G: \phi(g) = e_H\}\).
\end{definition}

\begin{lemma}[Kernel subgroup]
\label{lem:kernel-subgroup}
Let \(\phi: G \to H\) be a group morphism. The kernel \(\ker \phi \subseteq G\)
is a subgroup of \(G\).
\end{lemma}

\begin{proof}
Let \(g \in \ker \phi\) be any element, then since \(\phi(g^{-1}) = \phi(g)^{-1}
= e_H^{-1} = e_H\), then \(g^{-1} \in \ker \phi\). Also, if \(u \in \ker \phi\)
is another element, then \(\phi(g u) = \phi(g) \phi(u) = e_H e_H = e_H\) and
hence \(g u \in \ker \phi\) --- thus \(\ker \phi\) is a subgroup of \(G\).
\end{proof}

\begin{proposition}[Injectivity of morphisms]\label{prop: ker-trivial-inj}
A morphism of groups is injective if and only if its kernel is trivial.
\end{proposition}

\begin{proof}
Let \(\phi: G \to H\) be any morphism. If \(\phi\) is injective then clearly
\(\ker\phi = \{e_G\}\). On the other hand, if \(\ker\phi = \{e_G\}\), then,
given elements \(g, g' \in G\) such that \(\phi(g) = \phi(g')\), then,
multiplying both sides by \(\phi{(g')}^{-1}\) we get
\[
  \phi(g) {\phi(g')}^{-1} = \phi(g) \phi(g'^{-1}) = \phi(gg'^{-1}) = e_G.
\]
That is, \(gg'^{-1} \in \ker\phi\), but since \(\ker\phi\) is trivial, then
\(gg'^{-1} = e_G\) and hence \(g = g'\) --- in other words, \(\phi\) is
injective.
\end{proof}

The following proposition is a trivial one, but it is a really useful tool to
prove the non-existence of non-trivial morphisms between certain groups --- it
relies on arguments based on the order of elements of each group.

\begin{proposition}[Morphisms and orders]\label{prop: grp-morph-order}
Let \(G \in \Grp\) be a group admitting an element \(g \in G\) of finite order
\(|g| \in \N\). Let \(H\) be any group and consider the morphism \(\phi: G \to
H\). We have that \(|\phi(g)|\) divides the order of \(|g|\).
\end{proposition}

\begin{proof}
Notice that \({\phi(g)}^{|g|} = \phi(g^{|g|}) = \phi(e_G) = e_H\), hence \(|g|\)
is a multiple of the order of \(\phi(g) \in H\).
\end{proof}

\begin{example}
Consider for example the collection of morphisms \(\Hom_\Grp(C_7, C_{15})\).
Let \(\phi\) be any such morphism. Consider any element \(g \in C_7\) and
recall that \(|g|\) must be a divisor of \(7\) --- see \cref{prop:
ord-cyclic-elem}. On the other hand, if \(h \in C_{15}\) then \(|h|\) divides
\(15\). From \cref{prop: grp-morph-order} we see that if \(\phi(g) = h\), then
\(|h|\) must divide \(|g|\), but \(\gcd(7, 15) = 1\), hence
\(|h| = 1\) --- that is \(h = e_H\) and \(\phi\) is the trivial morphism
\(\phi(g) = e_H\) for all \(g \in C_7\). This shows that there is no
non-trivial morphism between \(C_7\) and \(C_{15}\).
\end{example}

\subsubsection{Isomorphism of Groups}

\begin{proposition}[Isomorphisms are bijections]\label{prop: grp-iso-bij}
Let \(\phi \in \Mor(\Grp)\). Then \(\phi\) is an isomorphism if and only if it
is a bijection.
\end{proposition}

\begin{proof}
Consider an isomorphism \(\phi: G \to H\). Using the forgetful functor \(F:
\Grp \to \Set\) we see that \(F\phi\) is a bijection of sets in \(\Set\) ---
recall \cref{lem: functor preserve iso} ---, thus \(\phi\) defines a bijection
between the elements of \(G\) and \(H\).

On the other hand, if \(\phi\) is a bijection, consider its set-function
inverse \({(F\phi)}^{-1}: H \to G\). We now show that \({(F\phi)}^{-1}\) preserves
the structures of groups. Since \(\phi(e_G) = e_H\), then \({(F\phi)}^{-1}(e_H)
= e_G\). Moreover, for any \(h, h' \in H\) --- since \(\phi\) is surjective
---, take elements \(g, g' \in G\) such that \(\phi(g) = h\) and \(\phi(g') =
h'\), then we find that \({(F\phi)}^{-1}(hh') = g g' = {(F\phi)}^{-1}(h) \cdot
{(F\phi)}^{-1}(h')\). This implies in the existence of a naturally induced
morphism of groups \(\phi^{-1}: H \to G\) defined by \(\phi^{-1}(h) =
{(F\phi)}^{-1}(h)\). It is clear that \(\phi^{-1}\) is the right and left
inverse of \(\phi\), thus \(\phi^{-1}\) is the inverse of \(\phi\) in \(\Grp\)
and \(\phi\) is an isomorphism.
\end{proof}

\begin{definition}[Embedding]
\label{def:grp-embedding}
Let \(\phi: G \isoto H\) be an isomorphism of groups. We define the group \(\im
\phi \subseteq H\) as an embedding of \(G\) on \(H\).
\end{definition}

\begin{proposition}\label{prop: iso-order-com}
Let \(\phi: G \isoto H\) be an isomorphism of groups. Then:
\begin{itemize}
  \setlength\itemsep{0em}
  \item For all \(g \in G\), we have \(|g| = |\phi(g)|\).
  \item \(G\) is commutative if and only if \(H\) is commutative.
\end{itemize}
\end{proposition}

\begin{proof}
Let \(g \in G\) be any element, then from \cref{prop: grp-morph-order} we have
that \(|\phi(g)|\) divides \(|g|\). Since \(\phi^{-1}\) exists and is a
morphism of groups, it also follows that \(|g|\) divides \(|\phi(g)|\) ---
hence \(|g| = |\phi(g)|\).

Let \(G\) be a commutative group and \(G \iso H\). Let \(h, h' \in H\) be any
elements and consider \(\phi^{-1}(h) = g\) and \(\phi^{-1}(h') = g'\). From
the structure preserving property of \(\phi\) we have
\[
  h h' = \phi(g) \phi(g') = \phi(gg') = \phi(g'g) = \phi(g')\phi(g) = h' h.
\]
That is, \(H\) is commutative. The counter-implication is equivalent and will
be omitted.
\end{proof}

\begin{lemma}[Inner automorphism]\label{lem: grp-inner-aut}
Let \(G \in \Grp\). For each \(g \in G\), the map \(\gamma_g: G \to G\) given
by \(\gamma_g(a) = gag^{-1}\) is an automorphism --- called inner
automorphism of \(G\).
\end{lemma}

\begin{proof}
Let \(g \in G\) be any element. Suppose \(a \in \ker\gamma_g\), then
\(\gamma_g(a) = gag^{-1} = e_G\), hence, \(a = g^{-1} e_G g = g^{-1}g = e_G\),
that is \(\ker\gamma_g = e_G\) --- \(\gamma_g\) is injective. Let \(g' \in G\)
be any element, then, \(\gamma_g(g^{-1} g' g) = g (g^{-1} g' g) g^{-1} = g'\),
that is, \(\gamma_g\) is surjective. We conclude that \(\gamma_g\) is a
bijection --- hence an isomorphism, so \(\gamma_g \in \Aut_\Grp(G)\).
\end{proof}

\begin{lemma}[Inner automorphism correspondence]\label{lem: grp-inner-aut-cor}
Let \(G \in \Grp\). The map \(\phi: G \to \Aut_\Grp(G)\) defined by the
mapping \(\phi(g) = \gamma_g\) (where \(\gamma_g\) is defined in \cref{lem:
grp-inner-aut}) is a morphism of groups.
\end{lemma}

\begin{proof}
Let \(g, g' \in G\) be any elements, then
\[
  \gamma_{gg'}(a) = (gg')a{(gg')}^{-1} = (g g') a (g'^{-1} g^{-1})
  = g(g' a g'^{-1}) g^{-1} = \gamma_g \gamma_{g'}(a).
\]
This being said, its easy to see that \(\phi\) preserves the group structure:
\(\phi(gg') = \gamma_{gg'} = \gamma_g \gamma_{g'} = \phi(g) \phi(g')\). Thus
\(\phi\) is a morphism of groups.
\end{proof}

\subsubsection{More Thoughts On Cyclic Groups}

We can now state the definition of a cyclic group in a formal manner, it goes as
follows:

\begin{definition}[Cyclic group]\label{def: cyclic-grp}
A group \(G\) is said to be cyclic if \(G \iso \Z\) or \(G \iso \Z/n\Z\) for
some \(n \in \N\).
\end{definition}

\begin{proposition}
A finite group of order \(n \in \N\) is cyclic if and only if it contains an
element of order \(n\).
\end{proposition}

\begin{proof}
Let \(G\) be a cyclic group of order \(n\). Since \(G\) is finite, then there
exists an isomorphism \(\phi: G \isoto \Z/n\Z\). Consider the element \(g =
\phi^{-1}({[1]}_n) \in G\), from \cref{prop: iso-order-com} we see that \(|g| =
n\).

Let \(G\) be a finite group of order \(n\) and \(x \in G\) be such that \(|x|
= n\). Let \(\phi: G \to \Z/n\Z\) be any morphism of groups sending \(x
\mapsto {[1]}_n\). Consider the collection \(G' = \{e_G, x, x^2, \dots,
x^{n-1}\} \subseteq G\), and notice that, together with the binary operation
of \(G\), \(G'\) becomes a group of \(n\) elements --- that is, \(G' = G\) and
every element \(g \in G\) can be written as \(g = x^k\) for some \(1 \leq k
\leq n\). This implies in \(\phi\) injective --- thus a bijection. From
\cref{prop: grp-iso-bij} we see that \(\phi\) is an isomorphism \(G \iso
\Z/n\Z\) --- \(G\) is a cyclic group, which finishes our proof.
\end{proof}

\begin{proposition}
\label{prop:order-cyclic-totient}
The order of the cyclic group \(C_n\) is equal to \(\phi(n)\), where \(\phi\) is
the Euler totient function --- that is, the number of positive integers less
than \(n\) that are relatively prime to \(n\).
\end{proposition}

\begin{proof}
Let \(x \in C_n\) be a generator of the group --- that is, \(x^n = e\) --- then
for all \(d < n\) such that \(\gcd(d, n) = 1\) we have \(x^d \neq x\) and
\((x^d)^n = e\) thus \(x^d\) is a generator of \(C_n\). Therefore the number
of distict elements of \(C_n\) is the same as the number of positive integers
coprimes of \(n\).
\end{proof}

\subsection{Group Products}

Let \((G, \circledast), (H, \circledcirc) \in \Grp\) be any objects. We define a
binary operation \(\cdot: (G \times H)^2 \to G \times H\) as the mapping
\begin{equation}\label{eq: grp-prod-bin}
(g, h) \cdot (g', h') = (g \circledast g', h \circledcirc h').
\end{equation}
Such binary operation defines a group structure on \(G \times H\). Notice that,
given an element \((g, h) \in G \times H\), there exists an element \((g^{-1},
h^{-1}) \in G \times H\) such that \((g, h) \cdot (g^{-1}, h^{-1}) = (e_G,
e_H)\). Moreover, clearly \((e_G, e_H) \in G \times H\) is the identity element
of the structure. Hence \((G \times H, \cdot) \in \Grp\).

Also, the natural projections \(\pi_G: G \times H \to G\) and \(\pi_H: G \times
H \to H\) define morphisms of groups.

\begin{definition}[Direct product]
Let \(\{G_{j}\}_{j \in J}\) be a collection of groups. We define the direct
product of this family as the group \(\prod_{j \in J} G_j\) given by elements
\((x_j)_{j \in J}\) such that \(x_j \in G_j\). The composition of elements of
the direct product is defined componentwise, that is, if \((x_j)_{j \in J},
(y_j)_{j \in J} \in \prod_{j \in J} G_j\), then \((x_j)_{j \in J} (y_j)_{j \in
J} \coloneq (x_j y_j)_{j \in J}\). Moreover, inverses are also defined
componentwise, \((x_j)_{j \in J}^{-1} \coloneq (x_j^{-1})_{j \in J}\).
\end{definition}

\begin{proposition}
The direct products of groups are products on the category of groups,
\(\Grp\). That is, for all group \(W\) and group morphisms \(f \in
\Hom_\Grp(W, G)\) and \(g \in \Hom_\Grp(W, H)\), there exists a unique
morphism \(\varphi \in \Hom_\Grp(W, G \times H)\) such that the following
diagram commutes
\[
  \begin{tikzcd}
    &W
    \ar[d, dashed, "\varphi"]
    \ar[ddr, bend left, "g"]
    \ar[ddl, swap, bend right, "f"]
    & \\
    &G \times H \ar[dr, swap, "\pi_H"] \ar[dl, "\pi_G"] & \\
    G & &H
  \end{tikzcd}
\]
\end{proposition}

\begin{proof}
We just take \(\varphi: W \to G \times H\) as the mapping \(w \xmapsto \varphi
(f(w), g(w))\). We show that \(\varphi\) exists in \(\Grp\): let \(x, y \in
W\) be any elements, then, since \(f\) and \(g\) are group morphisms, we find
that
\[
  \varphi(xy) = (f(xy), g(xy)) = (f(x) f(y), g(x) g(y))
  = (f(x), g(x)) (f(y), g(y)) = \varphi(x) \varphi(y).
\]
That is, \(\varphi\) is a group morphism. The uniqueness comes from the
covariant functor \(F: \Grp \to \Set\), since \(\Set\) allows for products and
hence the set-function \(F \varphi\) is unique.
\end{proof}

\begin{remark}
We now show that if \(G, H \in \Grp\) are such that \(G \iso H \times G\),
\emph{it does not follow that} \(H\) is trivial.
\todo[inline]{Find counterexample}
\end{remark}

\begin{proposition}
\label{prop:product-subgroups-isomorphism}
Let \(G\) be a group and \(H, Q \subseteq G\) be subgroups for which \(H \cap Q
= e\) and \(H Q = G\) --- that is, for every \(g \in G\), there exists \(h \in
H\) and \(q \in Q\) such that \(g = h q\) ---, we also impose that \(h q = q h\)
for every \(h \in H\) and \(q \in Q\). Then, the morphism of groups \(H \times Q
\isoto G\) defined by the mapping \((h, q) \mapsto hq\) is an isomorphism.
\end{proposition}

\begin{proof}
Notice that \((h, q)(h', q') = (h h', q q') \mapsto (hh')(qq') = (hq)(h'q')\)
thus the map is indeed a morphism of groups. Moreover, since every element of
\(G\) can be written as a product \(HQ\), it follows that the map is
surjective. Now, let \((h, q)\) be in the kernel of the morphism, then \(hq =
e\) which implies in \(h = q^{-1}\) but then \(h \in H \cap Q\) and by
hypothesis \(h = e\) --- thus the morphism is injective.
\end{proof}

\subsection{Cosets}

\begin{definition}[Coset]
\label{def:coset}
Let \(G\) be a group and \(H\) be a subgroup of \(G\). Given any \(g \in G\), a
left coset of \(H\) in \(G\) induced by \(g\) and denoted by \(g H\) is a set
whose elements have the form \(g h\) for each \(h \in H\). A right coset of
\(H\) in \(G\) induced by \(g\) is denoted \(H g\) and is a set consisting of
elements of the form \(h g\) for each \(h \in H\). An element of a coset is
commonly called \emph{coset representative}.
\end{definition}

\begin{corollary}
Let \(G\) be a group and \(H\) be subgroup of \(G\), then, for every \(h \in
H\), we have
\[
    h H = H h = H.
\]
\end{corollary}

\begin{proof}
Notice that, given \(x \in H\), the element \(h (h^{-1} x) = x \in h H\), thus
\(H \subseteq h H\), on the other hand, it's clear that \(hH \subseteq H\),
since \(hH\) is composed of product of elements of \(H\), which itself is closed
under products. The same analogous proof goes for \(Hh\) so I won't bother to
write it down.
\end{proof}

\begin{corollary}[Equal cosets]
\label{cor:equal-cosets}
Let \(G\) be a group and \(H \subseteq G\) be a subgroup. Given \(x, y \in G\),
if the cosets \(x H\) and \(y H\) share any common element, then \(x H = y H\).
\end{corollary}

\begin{proof}
Let \(g \in xH \cap yHk\), then there exists \(x h \in xH\) and \(y h' \in yH\)
such that \(x h = g = y h'\), then, in particular, \(x = y h' h^{-1}\) moreover,
since \(H\) is a subgroup, it is clear that \(h' h^{-1} \in H\) then \(x H = (y
h' h^{-1})H = y(h' h^{-1})H = y H\).
\end{proof}

\begin{definition}[Index]
\label{def:grp-index}
Let \(G\) be a group and \(H \subseteq G\) be a subgroup. The number of left
cosets of \(H\) in \(G\) is denoted by \((G : H)\), which will be commonly
refered to as the \emph{index} of \(H\) in \(G\).
\end{definition}

\begin{corollary}
\label{cor:order-as-index}
If we denote by \(*\) the trivial group, the \emph{order} of a group
\(G\) is the same as \((G : *)\) --- that is, \(|G| = (G : *)\).
\end{corollary}

\begin{proof}
One can view the trivial group \(*\) as a subgroup of \(G\) containing only the
identity. Notice that the number of left cosets of \(*\) will be exactly the
number of elements of \(G\), that is \((G : *) = |G|\).
\end{proof}

\begin{proposition}
\label{prop:grp-index-subgroup}
Let \(G\) be a group, and \(H \subseteq G\) be a subgroup, and \(Q \subseteq H\)
be a subgroup. Then, if any two of the quantities \(\{(G : H), (H : Q), (G :
Q)\}\) is finite, the third is also finite and the following equality holds
\[
  (G : H) (H : Q) = (G : Q).
\]
\end{proposition}

\begin{proof}
Let \(\{x_{i}\}_{i \in I} \subseteq H\) be coset representatives of \(Q\), and
\(\{y_{j}\}_{j \in J} \subseteq G\) be coset representatives of \(H\) --- that
is, each one of the collections \(\{x_i Q\}_{i \in I}\) and \(\{y_{j} H\}_{j \in
J}\) have pairwise disjoint elements, and \(H = \bigcup_{i \in I} x_i Q\), and
\(G = \bigcup_{j \in J} y_j H\). Then we have that \(G = \bigcup_{(i, j) \in I
\times J} y_j x_i Q\), and our goal will be to prove that \(y_j x_i Q \cap
y_{i'} x_{j'} Q = \emptyset\). Suppose on the contrary that their intersection
is non-empty, which by \cref{cor:equal-cosets} implies \(y_j x_i Q = y_{j'}
x_{i'} Q\). Since \(x_j, x_{j'} \in H\), we have \(y_j x_i Q H = y_j x_i H = y_j
H\) and analogously \(y_{j'} x_{i'} Q H = y_{j'} H\) --- thus \(y_j H = y_{j'}
H\), which implies in \(y_j = y_{j'}\). This shows that the collection \(\{y_{j}
x_i\}_{(i, j) \in I \times J} \subseteq G\) are coset representatives for \(Q\)
and therefore \((G : H) (H : Q) = (G : Q)\).
\end{proof}

\begin{corollary}
\label{cor:order-subgroup-divides-order-group}
Let \(G\) be a finite group, then the order of any subgroup \(H\) of \(G\)
divides the order \(|G|\).
\end{corollary}

\begin{proof}
Since \(|G| = (G : *)\) is finite, any subgroup \(H\) of \(G\) is also finite
and therefore \((G : H) (H : *) = (G : *)\), which is exactly the same as \((G :
H) |H| = |G|\).
\end{proof}

\begin{example}[Prime order]
\label{exp:grp-prime-order-cyclic}
Let \(G\) be a group with order \(|G| \coloneq p\) prime. Choose any \(g \in G\)
with \(g \neq e\), and consider the subgroup \(H \coloneq \langle g
\rangle\). From \cref{prop:grp-index-subgroup} we find that \((G : H) |H| = p\),
hence \(|H|\) divides \(p\), but since \(|H| \leq p\), then \(|H| = p\) and
therefore \(H = G\). This implies that any non-identity element of \(G\)
generates the whole group, which is the same as to say that \(G\) is cyclic.
\end{example}

\subsection{Normal Subgroups}

\begin{definition}[Normal subgroup]
\label{def:normal-subgroup}
Let \(G\) be a group. We define a \emph{normal subgroup} to be the kernel of
some morphism of groups in \(\Hom_{\Grp}(G, -)\)\footnote{In this case, we are
using \(\Hom_{\Grp}(G, -)\) to denote the same as the collection of all group
morphisms whose source is \(G\).}. In other words, a subgroup \(N \subseteq G\)
is normal if there exists a morphism of groups \(\phi: G \to H\), for some group
\(H\), for which \(\ker \phi = N\).
\end{definition}

\begin{definition}[Factor group]
\label{def:factor-group}
Let \(G\) be a group and \(N\) be a normal subgroup of \(G\). We denote by
\(G/N\) the collection of all left cosets of \(N\) in \(G\), on the other hand,
\(G \backslash N\) denotes the collection of all right cosets of \(N\) in
\(G\). Moreover, we view \(G/N\) (and \(G \backslash N\)) as groups where:
\begin{itemize}\setlength\itemsep{0em}
\item The product of two cosets \(x N\) and \(y N\) (or, respectively, \(N
  x\) and \(N y\)) is given by \((x N) (y N) \coloneq (x y) N\) which is again a
  left coset in \(G/N\) (conversely, \((N x) (N y) \coloneq N (x y) \in G
  \backslash N\)).
\item Given any coset \(x N\) (respectively, \(N x\)), its inverse is given by
  \(x^{-1} N\) (respectively, \(N x^{-1}\)).
\item The identity of the group is \(N\).
\end{itemize}
The group \(G/N\) is commonly refered to as the \emph{factor group} of \(G\) by
\(H\).
\end{definition}

\begin{proposition}
\label{prop:normal-subgroup-equivalence}
Let \(G\) be a group. A subgroup \(N \subseteq G\) is normal if and only if, for
every \(g \in G\), we have \(gNg^{-1} = N\).
\end{proposition}

\begin{proof}
First, suppose that \(N\) is a normal group and \(\phi: G \to H\) is a morphism
such that \(\ker \phi = N\), then if \(g \in G\) is any element, we see that for
any \(n \in N\) we have \(\phi(g n) = \phi(g) \phi(n) = \phi(g)\) and
analogously \(\phi(n g) = \phi(n) \phi(g) = \phi(g)\), thus in general \(g N = N
g = \phi^{-1}(\phi(g))\). Note that if we multiply both groups on the right by
\(g^{-1}\) we get \(g N g^{-1} = N\), as wanted.

On the other hand, let \(N \subseteq G\) be a subgroup such that \(g N g^{-1} =
N\) for every \(g \in G\), then in particular left cosets are equal to right
cosets because, multiplying on the right by \(g\) we obtain \(g N = N
g\). Consider the group of left cosets \(G/N\) (since right and left cosets are
equivalent in this specific case, we could also have considered \(G \backslash
N\)) and define the morphism of groups \(\pi: G \to G/N\) by the mapping
\(\pi(g) = g N\). Notice that if \(g \in \ker \pi\), then \(g N = N\), which
implies that \(g \in N\), moreover, if \(n \in N\) is any element, then
\(\pi(n) = n N = N\) --- thus \(\ker \pi = N\).
\end{proof}

In fact the morphism \(\pi: G \epi G/N\), defined above by the map \(g \xmapsto
\psi g N\), is so important we are even going to distinguishly call it the
\emph{canonical projection map} of \(G\) onto the factor group \(G/N\). It is
trivial that such canonical projection \(\pi\) is surjective.

\begin{corollary}[Intersection of normal subgroups is normal]
\label{cor:intersection-normal-subgroups}
Let \(\{N_{j}\}_{j \in J}\) be any collection of normal subgroups of a given
group \(G\). Then \(N \coloneq \bigcap_{j \in J} N_j\) is a normal subgroup of
\(G\).
\end{corollary}

\begin{proof}
Let \(n \in N\) and \(g \in G\) be any two elements, then \(n \in N_j\) for all
\(j \in J\) and from the normal condition we obtain that \(g n g^{-1} \in N_j\)
for all \(j \in J\) as well --- which implies in \(g n g^{-1} \in N\).
\end{proof}

\begin{definition}[Centralizers and normalizers]
\label{def:normalizer-centralizer}
Let \(G\) be a group and \(S \subseteq G\) be any set of elements. We define the
following groups:
\begin{enumerate}[(a)]\setlength\itemsep{0em}
\item The \emph{normalizer} of \(S\) is defined as the group \(N(S) \coloneq \{g
  \in G : g S g^{-1} = S\}\).
\item A \emph{centralizer} of \(S\) is defined as the group \(Z(S) \coloneq \{g
  \in G : g s g^{-1} = g \text{, for all } s \in S\}\). The centralizer \(Z(G)\)
  is commonly called the \emph{center} of \(G\).
\end{enumerate}
\end{definition}

The normalizer and centralizer are indeed groups, notice that if \(g \in N(S)\),
then \(g^{-1} \in N(S)\) since \(g S g^{-1} = S\) it follows, by multiplying on
the left by \(g^{-1}\) and on the right by \(g\), that \(S = g^{-1} S
g\). Moreover, given any two \(g, h \in N(S)\), we have
\[
  (g h)^{-1} N (g h) = (h^{-1} g ^{-1}) N (g h) = (h^{-1}N h) (g^{-1} N g) = S.
\]
For the case of the centralizer the proof is analogous as the one just made for
normalizers, where instead of \(S\) we would be considering any element \(s \in
S\).

\todo[inline]{Continue: Lang page 30}

\section{\texorpdfstring{\(\Ab\)}{Ab} --- Category of Abelian Groups}

\begin{definition}[Category of abelian groups]
We define the category of abelian groups, denoted \(\Ab\), to be the category
whose objects are abelian groups and group morphisms between them.
\end{definition}

In the following propositions we are going to describe ways of telling if the
group you might be interested is abelian or not.

\begin{proposition}
A group \(G\) is abelian if and only either one of the following conditions
are satisfied:
\begin{itemize}
  \setlength\itemsep{0em}
  \item The map \(G \to G\) mapping \(g \mapsto g^{-1}\) is a morphism of
    groups.
  \item The map \(G \to G\) mapping \(g \mapsto g^2\) is a morphism of groups.
\end{itemize}
\end{proposition}

\begin{proof}
Denote the first map by \(\phi\) and the second by \(\psi\). If \(G\) is
abelian, then for any \(g, g' \in G\) we have:
\begin{gather*}
  \phi(gg') = (gg')^{-1} = g'^{-1} g^{-1} = g^{-1}g'^{-1} = \phi(g) \phi(g'),
  \\
  \psi(gg') = (gg')^2 = (gg')(gg') = g^2g'^2 = \psi(g)\psi(g').
\end{gather*}
That is, \(\phi\) and \(\psi\) are morphisms. Now suppose \(\phi\) and
\(\psi\) are morphisms and consider any elements \(g, g' \in G\) --- we get
the following relations from each of the morphisms:
\begin{gather*}
  gg' = \phi(g^{-1})\phi(g'^{-1}) = \phi(g^{-1}g'^{-1})
  = \phi((g'g)^{-1}) = g'g,
  \\
  (gg')^2 = \phi(gg') = \phi(g) \phi(g') = g^2 g'^2.
\end{gather*}
Using cancellation law for the second relation we find \(g'g = gg'\). This
shows that \(G\) is abelian.
\end{proof}

\begin{proposition}
Let \(\phi \in \Hom_\Grp(G, \Aut_\Grp(G))\) be the morphism of groups mapping
\(g \xmapsto \phi \gamma_g\) --- as defined in \cref{lem: grp-inner-aut-cor}.
\(G\) is an abelian group if and only if \(\phi\) is trivial.
\end{proposition}

\begin{proof}
Let \(G\) be an abelian group, then for all \(g \in G\) the corresponding inner
automorphism \(\gamma_g\) maps \(a \xmapsto{\gamma_g} g a g^{-1} = a (g g^{-1}) = a\)
hence \(\gamma_g = \Id_G\), thus \(\phi\) is indeed trivial. Let \(\phi\) be trivial,
then for all \(g \in G\) the map \(\gamma_g = \Id_G\) and therefore for all \(a \in G\)
we have \(g a g^{-1} = a\), which implies in \(g a g^{-1} = a g g^{-1} = a
g^{-1} g = g g^{-1} a = g^{-1} g a\) --- that is, the group \(G\) is abelian.
\end{proof}

Now that we know some ways of identifying abelian groups, we dive deep again
into the categorical foundations of the category of abelian groups \(\Ab\).

\begin{proposition}[\(\Hom_\Ab\) abelian group]\label{prop: hom-ab-grp}
Let \(G, H \in \Ab\) be any two commutative groups. The collection of
morphisms \(\Hom_\Ab(G, H)\) forms an abelian group with a binary operation
defined by \((\phi + \psi)(g) = \phi(g) +_H \psi(g)\) --- where \(+_H\) is the
binary operation of \(H\).
\end{proposition}

\begin{proof}
Let \(\phi, \psi \in \Hom_\Ab(G, H)\) be any morphisms and consider elements
\(g, g' \in G\). From the commutativity of \(H\) it follows that
\begin{align*}
  (\phi + \psi)(g +_G g')
  &= \phi(g +_G g') +_H \psi(g +_G g')
  \\
  &= \left(\phi(g) +_H \phi(g')\right) +_H \left(\psi(g) +_H \psi(g')\right)
  \\
  &= \left(\phi(g) +_H \psi(g)\right) +_H \left(\phi(g') +_H \psi(g')\right)
  \\
  &= (\phi + \psi)(g) +_H (\phi + \psi)(g').
\end{align*}
That is, \(\phi + \psi \in \Hom_\Ab(G, H)\). Moreover, given a morphism \(f
\in \Hom_\Ab(G, H)\), define the map \(k: G \to H\) mapping \(k(g) =
f(g)^{-1}\). Notice that \(k\) is a morphism:
\begin{align*}
  k(g +_G g')
  &= f(g +_G g')^{-1} \\
  &= (f(g) +_H f(g'))^{-1} \\
  &= f(g')^{-1} +_H f(g)^{-1} \\
  &= f(g)^{-1} +_H f(g')^{-1} \\
  &= k(g) +_H k(g').
\end{align*}
Moreover, \((f + k)(g) = f(g) +_H k(g) = f(g) +_H f(g)^{-1} = e_H\) --- that
is, \(f + k\) is the trivial morphism \(g \mapsto e_H\), i.e. \(k\) is the
inverse of \(f\). This finishes the proof that \(\Hom_\Ab(G, H)\) is a group.
For the commutativity, it follows directly from the commutativity of \(H\):
for all \(\phi, \psi \in \Hom_\Ab(G, H)\) we have
\[
  (\phi + \psi)(g) = \phi(g) +_H \psi(g) = \psi(g) +_H \phi(g) = (\psi +
  \phi)(g).
\]
\end{proof}

The following corollaries follow immediately from the construction of the
category.

\begin{corollary}
Let \(G \in \Grp\) and \(H \in \Ab\). The collection of morphisms
\(\Hom_\Grp(G, H)\) forms a group under the binary operation defined above.
\end{corollary}

\begin{corollary}
Let \(A \in \Set\) and \(H \in \Ab\). The collection of morphisms
\(\Hom_\Set(A, FH)\) forms a group under the binary operation defined above
--- where \(F: \Grp \to \Set\) is a forgetful functor.
\end{corollary}

\subsection{Coproduct and Fiber Product}

\begin{proposition}[Coproduct in \(\Ab\)]\label{prop: coprod-ab}
The direct product of abelian groups is a coproduct in \(\Ab\). That is, for
any \(G, H, W \in \Ab\) and morphisms \(f \in \Hom_\Ab(W, G)\) and \(k \in
\Hom_\Ab(W, H)\), there exists a unique morphism \(\varphi \in \Hom_\Ab(W, G
\times H)\) such that the following diagram commutes
\[
  \begin{tikzcd}
    G \ar[dr, "\iota_G"] \ar[ddr, bend right, swap, "f"] &
    &H \ar[dl, swap, "\iota_H"] \ar[ddl, bend left, "k"] \\
    &G \times H \ar[d, dashed, "\varphi"]  & \\
    &W &
  \end{tikzcd}
\]
Where we define inclusion morphisms \(g \xmapsto{\iota_G} (g, e_H)\) and \(h
\xmapsto{\iota_H} (e_G, h)\).
\end{proposition}

\begin{proof}
Since \(\Ab \subset \Grp\) --- that is, the category of abelian groups is a
subcategory of \(\Grp\) --- then, from \cref{prop: forgetful-func-grp-set}
there exists a functor \(\Ab \to \Set\). Since coproducts exists in \(\Set\)
and are unique, a set-function \(\varphi\) exists and is unique, commuting the
diagram in \(\Set\). We now show that we can extend such set-function into a
morphism of groups. Let \(\varphi: G \times H \to W\) be the mapping \((g, h)
\xmapsto \varphi f(g) k(h)\). Notice that
\begin{align*}
  \varphi((g, h)(g', h'))
  = \varphi(gg', hh')
  = f(g g') k(hh')
  = f(g) f(g') k(h) k(h')
  &= f(g) k(h) f(g') k(h') \\
  &= \varphi(g, h) \varphi(g', h')
\end{align*}
that is, \(\varphi\) is a morphism of groups.
\end{proof}

\begin{remark}[Coproducts in \(\Grp\)]\label{rem:coprod-grp}
\cref{prop: coprod-ab} is not at all true for the category of
groups. Consider for instance the cyclic groups \(C_2 = \{e_x, x\}\) and \(C_3
= \{e_y, y, y^2\}\). Let \(\sigma_k \in S_3\), for \(0 \leq k \leq 2\) be the
rotation of \(\{1, 2, 3\}\) by \(k\), that is, the permutations represented by
\[
  M_{\sigma_0} =
  \begin{bmatrix}
    1 &0 &0 \\ 0 &1 &0 \\ 0 &0 &1
  \end{bmatrix}
  \qquad
  M_{\sigma_1} =
  \begin{bmatrix}
    0 &0 &1 \\ 1 &0 &0 \\ 0 &1 &0
  \end{bmatrix}
  \qquad
  M_{\sigma_2} =
  \begin{bmatrix}
    0 &1 &0 \\ 0 &0 &1 \\ 1 &0 &0
  \end{bmatrix}
\]
Consider the embeddings \(f: C_2 \mono S_3\) mapping \(x^k \xmapsto f
\sigma_k\) for \(k \in \{0, 1\}\), and \(g: C_3 \mono S_3\) mapping \(y^k
\xmapsto g \sigma_k\) for \(k \in \{0, 1, 2\}\).

Suppose, for the sake of contradiction, that \(C_2 \times C_3\) is a coproduct
in \(\Grp\), that is, exists a unique morphism \(\varphi: C_2 \times C_3 \to
S_3\) such that \(f = \varphi \iota_{C_2}\) and \(g = \varphi \iota_{C_3}\).
Since \(\varphi\) is supposedly a morphism of groups,
\begin{gather*}
  \varphi(x, y) = \varphi(x, e_y) \varphi(e_x, y) = \sigma_1 \sigma_1 =
  \sigma_2 \\
  \varphi(x, y^2) = \varphi(x, e_y) \varphi(e_x, y^2) = \sigma_1 \sigma_2
  = \sigma_0
\end{gather*}
However, \(\varphi(e_x, e_y) = \sigma_0\) and on the other hand we have
\(\varphi(x, y) \varphi(x, y^2) = \sigma_2\), which contradicts the properties
of a group morphism. This shows that there exists no such \(\varphi\) in
\(\Grp\) and hence \(C_2 \times C_3\) is not a coproduct in \(\Grp\).

Although \(C_2 \times C_3\) is not a coproduct in \(\Grp\), that doesn't mean
that \(\Grp\) has no coproducts, they just behave differently when compared
with \(\Ab\). For instance, let \(C_2 * C_3 \in \Grp\) be defined to be the
group generated by elements \(x\) and \(y\), such that \(x^2 = e\) and \(y^3 =
e\). We'll now show that \(C_2 * C_3\) is a coproduct of \(C_2\) and \(C_3\)
in \(\Grp\). Let \(G\) be any group and consider morphisms \(f: C_2 \to G\)
and \(k: C_3 \to G\). The inclusions \(\iota_{C_2}: C_2 \to C_2 * C_3\) and
\(\iota_{C_3}: C_3 \to C_2 * C_3\) will be naturally given maps by taking each
element to itself.

Let \(q \in C_2 * C_3\) be any element. We know that there exists a finite
collection of coefficients \(I = \{(a, b) \in \Z^2\}\) such that \(q =
\prod_{(a, b) \in I} x^a y^b\). Define \(\phi: C_2 * C_3 \to G\) as the
mapping
\[
  \phi(q) = \phi\left(\prod_{(a, b) \in I} x^a y^b\right)
  = \prod_{(a, b) \in I} f(x^a) k(y^b)
  = \prod_{(a, b) \in I} f(x)^a k(y)^b
\]
It should be clear that this definition implies \(\phi \iota_{C_2} = f\) and
\(\phi \iota_{C_3} = g\). Notice that \(\phi(e) = f(e) k(e) = e_G\) and for
all \(q, p \in C_2 * C_3\) --- with respective coefficients \(I = \{(a, b) \in
\Z^2\}\) and \(J = \{(c,d) \in \Z^2\}\) --- we have
\begin{align*}
  \phi(q) \phi(p)
  = \phi\left[ \prod_{(a, b) \in I} x^a y^b \right]
  \phi\left[ \prod_{(c, d) \in J} x^c y^d \right]
  &= \prod_{(a, b) \in I} f(x^a) k(y^b)
  \prod_{(c, d) \in J} f(x^c) k(y^d)
  \\
  &= \prod_{(\alpha, \beta) \in A} f(x^\alpha) g(y^\beta)
  \\
  &= \phi\left(\prod_{(\alpha, \beta) \in A} f(x^\alpha) g(y^\beta) \right)
  \\
  &= \phi\left(\prod_{(a, b) \in I} x^a y^b
    \prod_{(c, d) \in J} x^c y^d\right)
  = \phi(qp)
\end{align*}
that is, \(\phi\) is a morphism of groups --- where we define \(A\) as the
concatenation of the coefficients \(I\) and \(J\). We have shown that the
following diagram commutes
\[
  \begin{tikzcd}
    C_2 \ar[dr, "\iota_{C_2}"] \ar[ddr, swap, bend right, "f"]
    & &C_3 \ar[dl, swap, "\iota_{C_3}"] \ar[ddl, bend left, "g"] \\
    &C_2 * C_3 \ar[d, dashed, "\phi"] & \\
    &G &
  \end{tikzcd}
\]
We can then conclude that \(C_2 * C_3\), as defined above, is the coproduct of
\(C_2\) and \(C_3\) in \(\Grp\).
\end{remark}

\begin{proposition}[Fiber products]
Fiber products exist in \(\Ab\). That is, given abelian groups \(G, H, W \in
\Ab\) and group morphisms \(\phi \in \Hom_\Ab(G, W)\) and \(\psi \in
\Hom_\Ab(H, W)\). Let
\[
  G \times_W H \in \Ab_W
\]
be the fiber product of \(\phi\) and \(\psi\) in the category \(\Ab_W
\subseteq \Ab\)\footnote{As a reminder: \(\Ab_W\) is the category whose
objects are morphisms \(f \in \Hom_\Ab(-, W)\). If \(f: G \to W\) and \(g: H
\to W\) are objects of \(\Ab_W\), a morphism \(h \in \Hom_{\Ab_W}(f, g)\) is
such that \(hg = f\).} for which exists natural projections \(\pi_G\) and
\(\pi_H\). Let \(Q \in \Ab\) be any abelian group and consider any morphisms \(f \in
\Hom_\Ab(Q, G)\) and \(k \in \Hom_\Ab(Q, H)\). Then there exists a unique
morphism \(\varphi \in \Hom_\Ab(Q, P)\) such that the following diagram commutes
\[
  \begin{tikzcd}
    Q \ar[dr, dashed, "\varphi"] \ar[ddr, bend right, swap, "f"]
    \ar[drr, bend left, "k"]
    & & \\
    &G \times_W H \ar[r, "\pi_H"] \ar[d, swap, "\pi_G"] &H \ar[d, "\psi"] \\
    &G \ar[r, swap, "\phi"] &W
  \end{tikzcd}
\]
\end{proposition}

\begin{proof}
Define \(G \times_W H = \{(g, h) \in G \times H : \phi(g) = \psi(h)\}\). Since
\(\phi\) and \(\psi\) are group morphisms, given \((g, h) \in G \times_W H\),
we have
\[
  \phi(g^{-1}) = \phi(g)^{-1} = \psi(h)^{-1} = \psi(h^{-1}),
\]
that is, \((g^{-1}, h^{-1}) \in G \times_W H\) exists and is the inverse of
\((g, h)\). Moreover, given any elements \((g, h), (g', h') \in G \times_W H\), the
product \((g, h)(g', h') = (gg', hh')\) is such that
\[
  \phi(gg') = \phi(g) \phi(g') = \psi(h) \psi(h') = \psi(hh'),
\]
thus \((gg', hh') \in G \times_W H\). This shows that \(G \times_W H\) is
indeed a group and since \(G \times H\) is abelian, so is the fiber product
defined above.

From the forgetful functor \(\Ab \to \Set\) we know that there exists a unique
set-function \(\varphi\) such that the diagram commutes in \(\Set\). If we
define \(\varphi\) as the mapping \(q \overset \varphi \longmapsto (f(q),
k(q))\) we can see that it preserves the group structure, since
\[
  \varphi(qq') = (f(qq'), k(qq')) = (f(q)f(q'), k(q)k(q'))
  = (f(q), k(q))(f(q'), k(q')) = \varphi(q) \varphi(q'),
\]
that is, \(\varphi \in \Mor(\Ab)\).
\end{proof}

\subsection{Direct Sums}

\begin{definition}[Direct sum of abelian groups]
\label{def:Ab-direct-sum}
Let \(\{G_j\}_{j \in J}\) be an indexed collection of abelian groups. We define
their direct sum as the collection of tuples \((g_j)_{j \in J}\) such that \(g_j
\neq e_{G_j}\) for only finitely many indexes \(j \in J\) --- that is, set of
tuples with finite support. We denote the direct sum of \(\{G_{j}\}_{j \in J}\)
as \(\bigoplus_{j \in J} G_j\) and the group structure of the direct sum is
defined naturally as \((x_j)_{j \in J} + (y_j)_{j \in J} = (x_j +_{G_{j}}
y_j)_{j \in J}\).
\end{definition}

\begin{proposition}[Direct sum universal property]
\label{prop:Ab-direct-sum-universal-property}
The direct sum defined in \cref{def:Ab-direct-sum} satisfies the universal
property of direct sums. In other words, let \(\{G_{j}\}_{j \in J}\) be a
collection of abelian groups and \(H\) be any abelian group, in addition,
consider the collection of morphisms \(\{\phi_j \in \Hom_{\Grp}(G_j,
H)\}\). There exists a unique morphism \(\psi: \bigoplus_{j \in J} G_j \to H\)
such that the following diagram commutes
\[
  \begin{tikzcd}
    G_j \ar[rd, bend left, "\phi_j"] \ar[d, hook, "\iota"] &\\
    \bigoplus_{j \in J} G_j \ar[r, dashed, "\psi"] &H
  \end{tikzcd}
\]
for every \(j \in J\) --- where \(\iota_{j}: G_j \to \bigoplus_{j \in J} G_j\)
is the natural inclusion, mapping \(x_{j_0} \xmapsto{\iota_{j}} (x_j)_{j \in
J}\) such that \(x_j = x_{j_0}\) for \(j = j_0\) and \(x_j = e_{G_j}\) for \(j
\neq j_0\).
\end{proposition}

\begin{proof}
Let \(\psi: \bigoplus_{j \in J} G_j \to H\) be the map defined by \(\psi(x) =
\prod_{j \in J} \phi_{j}(x_j)\), where \(x = (x_j)_{j \in J} \in \bigoplus_{j
\in J} G_j\) is any element. Notice that \(\psi(x)\) is therefore a finite
product of elements of \(H\) and from the group structure of the direct sum, we
find that \(\psi\) is clearly a group morphism. Moreover, \(\psi\) satisfies the
commutativity \(\psi \iota_j = \phi_j\) for each \(j \in J\). Since such
definition of \(\psi\) defines the image of every element of its domain,
\(\psi\) is the unique morphism making the diagram commute.
\end{proof}

If \(G\) is an abelian group and \(A, B \subseteq G\) are subgroups such that
\(A \cap B = 0\) and \(A + B = G\) --- that is, every \(g \in G\) can be
written as \(g = a + b\) for some \(a \in A\) and \(b \in B\) --- then, in the
context of abelian groups, we'll denote this fact shortly by \(G = A \oplus B\).

Just like a vector space, we can define a basis of an abelian group by means of
the ring \(\Z\). Moreover, if an abelian group has a basis, then we say that it
is free.

\begin{definition}[Basis]
\label{def:Ab-basis}
Let \(G\) be an abelian group. We say tha a non-empty collection \(\{g_j\}_{j
\in J}\) is a basis for \(G\) if, given an element \(g \in G\), there exists a
unique tuple of coefficients \((a_j)_{j \in J} \in \bigoplus_{j \in J} \Z\) such
that \(g = \sum_{j \in J} a_j g_j\).
\end{definition}

Therefore the existence of a basis allow us to couple the coefficients coming
from \(\bigoplus_{j \in J} \Z\) to any element of \(G\) in a unique way, which
induces a natural isomorphism
\[
  G \iso \bigoplus_{j \in J} \Z.
\]

\begin{definition}[Abelian free group]\label{def:Ab-free}
An abelian group is said to be free if it allows a basis.
\end{definition}

Equivalently as we did with free vector spaces, we can build free abelian groups
out of sets, say \(S\), by analysing maps \(S \to \Z\) with finite support ---
the collection of those will be likewise denoted by \(\Z^{\oplus S}\), which is
a group under pointwise addition. For each \(s \in S\) we define a map
\(\mathbf{s} \in \Z^{\oplus}\) by the mapping
\[
  \mathbf{s}(x) \coloneq
  \begin{cases}
    1, &\text{for } x = s \\
    0, &\text{otherwise}
  \end{cases}
\]
Then, any element \(\phi \in \Z^{\oplus S}\) can be written as a linear
combination of finitely many maps \(\mathbf{s}\) (which is possible because of
the finite support of \(\phi\)) that is, for some \(a_j \in \Z\) for each \(1
\leq j \leq n\), we have
\[
  \phi = \sum_{j=1}^n a_j \mathbf{s}_j,\ \text{ mapping }\
  s \xmapsto \phi
  \begin{cases}
    a_j, &\text{ if } s = s_j \text{ for some } 1 \leq j \leq n \\
    0, &\text{otherwise}
  \end{cases}
\]
Moreover, such choice of coefficients \(a_j \in \Z\) is unique. Let
\(\{b_{j}\}_{j=1}^n \subseteq \Z\) be another set of coefficients with the same
property, then in particular \(\sum_{j=1}^n (a_j - b_j) \mathbf{s}_j = 0\) and
since \(\mathbf{s}_j\) are all non-zero maps, we find that \(b_j = a_j\). We
also define a map \(\iota_S: S \mono \Z^{\oplus S}\) by pairing \(s \mapsto
\mathbf{s}\).

\begin{proposition}[Free \(\Ab\) universal property]
\label{prop:free-abelian-group-universal-property}
Let \(S\) be a set. Consider \(G\) to be any abelian group and a set-function
\(g: S \to G\), then there exists a unique morphism of groups \(g_{*}:
\Z^{\oplus S} \to G\) such that
\[
  \begin{tikzcd}
    S \ar[d, hook, "\iota_S"] \ar[r, "g"] &G \\
    \Z^{\oplus S} \ar[ru, bend right, dashed, "g_{*}"']  &
  \end{tikzcd}
\]
is a commutative diagram.
\end{proposition}

\begin{proof}
Define \(g_{*}\) by \(g_{*}(\sum_{s \in S} a_s \mathbf{s}) \coloneq \sum_{s \in
S} a_s g(s)\). Then \(g_{*}\iota_S(s) = g_{*}(\mathbf{s}) = g(s)\). Moreover, it
follows from the construction that \(g_{*}(\sum_{s \in S} a_s \mathbf{s}) =
\sum_{s \in S} a_s g_{*}(\mathbf{s})\) thus \(g_{*}\) is a group morphism. If we
let \(f: \Z^{\oplus S} \to G\) be a morphism satisfying such commutativity,
we'll see that, since \(f_{*}(\mathbf{s}) = g(s) = g_{*}(\mathbf{s})\), since
any element of \(\Z^{\oplus S}\) can be written uniquely as a linear combination
of the basis \(\{\mathbf{s}\}_{s \in S}\), then \(f_{*}\) and \(g_{*}\) agree in
every point of the domain --- hence \(f_{*} = g_{*}\).
\end{proof}

\begin{corollary}
Let \(f: X \to Y\) be a set-function between sets \(X\) and \(Y\). Then, there
exists a unique morphism \(\overline{f}: \Z^{\oplus X} \to \Z^{\oplus Y}\) such
that the following diagram commutes
\[
  \begin{tikzcd}
    X \ar[d, "f"'] \ar[r, hook, "\iota_X"]
    &\Z^{\oplus X} \ar[d, dashed, swap, "\overline f"] \\
    Y \ar[r, hook, "\iota_Y"] &\Z^{\oplus Y}
  \end{tikzcd}
\]
\end{corollary}

\begin{proof}
Let \(\overline f\) be defined by \(\overline f(\sum a_x \mathbf{x}) \coloneq
\sum a_x \iota_Y f(x)\) --- so that, in particular, \(\overline f(\mathbf{x}) =
\iota_Y f(x)\), that is, the diagram commutes. Let \(\overline g: \Z^{\oplus X}
\to \Z^{\oplus Y}\) be a morphism such that the diagram commutes, then
necessarily \(\overline g(\mathbf{x}) = \iota_Y f(x) = \overline
f(\mathbf{x})\), that is, \(\overline g\) and \(\overline f\) agree on the basis
of \(\Z^{\oplus X}\), thus \(\overline g = \overline f\).
\end{proof}

\begin{notation}
When it's not confusing, we can even drop the notation \(\mathbf{s}\) and
instead identify the elements as \(s \in \Z^{\oplus S}\), so that \(\sum a_s s
\coloneq \sum a_s \mathbf{s}\). Moreover, for now on, we'll refer to the
\emph{free abelian group generated by} \(S\), that is, \(\Z^{\oplus S}\), as
\(F_{\Ab}(S)\) --- this is motivated by the fact that the notation \(\Z^{\oplus
S}\), although cool, may be a rather obscure way of talking about a group. The
elements of \(S\), in particular, will be refered to as the \emph{free
generators}.
\end{notation}

\begin{proposition}
\label{prop:Ab-isomorphic-factor-of-free}
Let \(G\) be any abelian group
\todo[inline]{Continue: mini exercises on free ab grp}
\end{proposition}

\todo[inline]{Continue direct products}

%%% Local Variables:
%%% mode: latex
%%% TeX-master: "../../deep-dive"
%%% End:

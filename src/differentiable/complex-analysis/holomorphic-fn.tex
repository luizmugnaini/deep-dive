\begin{definition}[Holomorphic function]
\label{def:holomorphic-function}
    A function is said to be \emph{holomorphic} (also called \emph{analytic})
    if it is a differentiable complex-valued function with a complex variable.
\end{definition}

Let \(f: \CC \to \CC\) be a holomorphic function such that \(f = u + \img v\) where 
\(u, v: \CC \para \R\). Let's write the variables of \(f\) as \(z = x + \img y\) so that \(f\) may
be seen as a function of two variables \(\R^2 \to \R^2\) given by \((x, y) \mapsto (u(x), v(y))\)
from the bijection \(\CC \iso \R^2\). Since \(f\) is differentiable, then so are \(u\) and \(v\).
Let's consider the definition of the derivative of \(f\) at a point \(z \in \CC\):
\begin{equation}\label{eq:derivative-complex-function}
    f'(z) = \lim_{\R \ni t \to 0} \frac{f(z + t) - f(z)}{t} 
    = \lim_{\R \ni t \to 0} \frac{f(z + \img t) - f(z)}{\img t}.
\end{equation}

If we consider the real case where we approach from the purely real values we get
\begin{align*}
    \lim_{\R \ni t \to 0} \frac{f(z + t) - f(z)}{t}
    &= \lim_{t \to 0} \frac{u(x + t, y) + \img v(x + t, y) - (u(x, y) + \img v(x, y))}{t} \\
    &= \lim_{t \to 0} \frac{u(x + t, y) - u(x, y)}{t} 
       + \img \lim_{t \to 0} \frac{v(x + t, y) - v(x, y)}{t} \\
    &= \frac{\partial u}{\partial x} + \img \frac{\partial v}{\partial x}.
\end{align*}
On the other hand, if we restrict the limit by approaching from strictly imaginary values, we obtain:
\begin{align*}
    \lim_{\R \ni t \to 0} \frac{f(z + \img t) - f(z)}{\img t}
    &= \lim_{t \to 0} \frac{u(x + \img t, y) + \img v(x + \img t, y) - (u(x, y) + \img v(x, y))}{\img t} \\
    &= -\img \Big(
        \lim_{t \to 0} \frac{u(x + t, y) - u(x, y)}{t} 
       + \img \lim_{t \to 0} \frac{v(x + t, y) - v(x, y)}{t}
    \Big) \\
    &= -\img \frac{\partial u}{\partial x} + \frac{\partial v}{\partial x}.
\end{align*}
Since both limits are equal, as seen in \cref{eq:derivative-complex-function}, we obtain what we
call the \emph{Cauchy-Riemann equation} for holomorphic functions:

\begin{theorem}[Cauchy-Riemann]
\label{thm:cauchy-riemann-complex-function}
    Let \(D \subseteq \CC\) be an open set, and \(f = u + \img v: D \to \CC\)
    be any function --- where \(u, v: \R^2 \para \R\) are differentiable functions. The function
    \(f\) is \emph{holomorphic} if and only if the following two equations are satisfied:
    \[
        \frac{\partial u}{\partial x} = \frac{\partial v}{\partial y},
        \quad\text{ and }\quad
        \frac{\partial v}{\partial x} = -\frac{\partial u}{\partial y}.
    \]
\end{theorem}

\begin{corollary}
    The complex derivative of the holomorphic function \(f\) satisfies the following equalities:
    \[ 
        f'(z) = \frac{\partial u}{\partial x} + \img \frac{\partial v}{\partial x}
              = \frac{\partial v}{\partial y} - \img \frac{\partial u}{\partial y}.
    \]
\end{corollary}

\begin{corollary}
    The squared absolute value of the first order derivative of \(f\) is equal to the Jacobian of
    the \(\R\)-vector field \(F: D \to \R^2\) given by \((x, y) \mapsto (u(x, y), v(x, y))\):
    \begin{align*}
        |f'(z)|^2 &= \Jac F(z) \\
                  &= \frac{\partial u}{\partial x} \frac{\partial v}{\partial y}   
                    - \frac{\partial u}{\partial y} \frac{\partial v}{\partial x} \\
                  &= \Big( \frac{\partial u}{\partial x} \Big)^2
                    + \Big( \frac{\partial v}{\partial x} \Big)^2 \\
                  &= \Big( \frac{\partial v}{\partial y} \Big)^2
                    + \Big( \frac{\partial u}{\partial y} \Big)^2
    \end{align*}
\end{corollary}

\section{Angles \& Holomorphic Maps}

Let \(D \subseteq \CC\) be an open set and let \(\gamma = x + \img y: [a, b] \to D\) be a
differentiable curve parametrized by the real valued interval \([a, b] \subseteq \R\).
Consider a holomorphic function \(f: D \to \CC\).

Let's take a closer look at the composition \(f \gamma: [a, b] \to \CC\). We can interpret
\(\gamma'(t)\) as defining a direction at the tangent space \(T_{\gamma(t)}D\) whenever 
\(\gamma'(t) \neq 0\). Consider \(z_0 \coloneq \gamma(t)\) and let \(\eta: [a, b] \to D\) be
another differentiable curve such that \(\eta(h) = z_0\) for some \(h \in [a, b]\).

\begin{theorem}[Holomorphic functions are conformal]
\label{thm:holomorphic-fn-is-conformal}
    If \(f'(z_0) \neq 0\) then the angle between the directional vectors defined by \(\gamma'(t)\)
    and \(\eta'(h)\) at \(T_{z_0}D\) is the same as the angle between the directional vectors
    defined by \((f \gamma)'(t)\) and \((f \eta)'(h)\).
\end{theorem}

\begin{proof}
   Define an inner product \(\langle -, - \rangle: \CC \times \CC \to \R\) given by 
   \(\langle z, w \rangle \coloneq \Real(z \cconj w)\). If \(z = a + \img b\) and
   \(w = c + \img d\) then 
   \[
       \langle z, w \rangle = \Real(a c + b d + \img (b c - a d))
                            = a c + b d
                            = \langle (a, b), (c, d) \rangle_{\R^2}
   \]
   where \(\langle -, - \rangle_{\R^2}\) is the canonical inner product of the real vector space
   \(\R^2\).

   Let \(\theta(z, w)\) be the angle between \(z\) and \(w\), then
   \[
       \cos(\theta(z, w)) = \frac{\langle z, w \rangle}{|z| |w|}
   \]
   and since \(\sin(t) = \cos(t - \pi/2)\) it follows that
   \[
       \sin(\theta(z, w)) = \frac{\langle z, -\img w \rangle}{|z| |w|}.
   \]
   \todo[inline]{continue proof}
\end{proof}

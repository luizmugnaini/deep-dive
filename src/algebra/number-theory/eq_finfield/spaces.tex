\section{Affine space, projective space, and polynomials}

\begin{definition}[Affine space]
   Let \(k\) be a field and \(\A^n(k)\) be the collection of objects \(a =
   (a_i)_{i=1}^{n}\) with \(a_i \in k\). We define in \(\A^n(k)\) the notion of
   addition and multiplication as usual, so that we can view it as a vector
   space. We call \(\A^n(k)\) the affine \(n\) dimensional \(k\)-space. The
   point \(0 := (0, \dots, 0)\) is called the origin.
\end{definition}

\begin{definition}[Projective space]
   Let \(k\) be a field, we define the projective  \(n\) dimensional
   \(k\)-space to be
   \[
      \PP^n(k) := (\A^{n+1}(k)\setminus\{0\})/\sim
   \] 
   where \(\sim\) is an equivalence relation between objects \(a, b \in
   \A^{n+1}(k) \setminus \{0\} \) defined as
   \[
      (a_0, a_1, \dots, a_n) \sim (b_0, b_1, \dots, b_n) \Leftrightarrow \exists
      \gamma \in k^\ast \text{ such that } a_i = \gamma b_i.
   \] 
   Objects in the projective space are equivalence classes containing
   representative points of the affine \(n+1\) dimensional \(k\)-space.
   Geometrically we see these equivalence classes as injective correspondences
   with lines in \(\A^{n+1}(k)\) that pass through the origin.
\end{definition}

Let \(k\) be a finite field with  \(|k| = q \), then the number of objects
composing the affine space \(\A^n(k)\) is exactly  \(q^n\). Now, in order to
determine the number of elements of the projective space, we can see that
\(\A^{n+1}(k) \setminus \{0\} \) has a total of \(q^{n+1} - 1\) elements, and
given any representative \(a \in \A^{n+1}(k) \setminus \{0\} \) we have that the
class \([a] \in \PP^n(k)\) has a total of \(q-1\) distinct elements (since
\(\gamma \in k^\ast\) has a total of \(q-1\) possibilities). Therefore we
conclude that the number of objects in \(\PP^{n}(k)\) is
 \[
    \frac{q^{n+1}-1}{q-1} = q^n + q^{n-1} + \cdots + q + 1.
\] 

\begin{lemma}
   Let \(\overline{H}\) be the set whose elements are classes \([x] \in \PP^n(k)
   : x_0 = 0\). Then the map 
   \[
      \varphi: \PP^{n}(k) \to \A^n(k) \text{ defined as } \varphi([x]) :=
      (x_1/x_0, x_2/x_0, \dots, x_n/x_0)
   \] 
   establishes an isomorphism \(\PP^{n}(k) \iso \A^{n}(k)\).
\end{lemma}

\begin{proof}
   (Monomorphism) Suppose that \(\varphi([x]) = \varphi([y])\), then from the
   definition of the affine space, we have that for all index \(i\) we have
   \(x_i/x_0 = y_i/y_0 \Rightarrow \frac{x_0}{y_0}x_i = y_i\) if we take
   \(\gamma := \frac{x_0}{y_0} \in k^\ast\) we see that in fact 
    \[
       (x_1/x_0,\dots,x_n/x_0) \sim (y_1/y_0,\dots,y_n/y_0) \Leftrightarrow [x]
       = [y]
   \] 
   thus \(\varphi\) is a monomorphism.
   (Epimorphism) Let \((\ell_1, \dots, \ell_n) \in \A^n(k)\), then we have that
   \[\varphi\left(\left[(1, \ell_1, \dots, \ell_n)\right]\right) =
   (\ell_1,\dots,\ell_n)\] which shows, that  \(\varphi\) is an epimorphism.
\end{proof}

\begin{definition}[Hyperplane]
   We define the set \(\overline{H}\) to be the hyperplane at infinity of
   \(\PP^n(k)\). This set has the structure of \(\PP^{n-1}(k)\), since we are
   confined to define \(x_0 = 0\). Thus 
   \[
      \overline{H} \iso \PP^{n-1}(k)
   \] 
   will be interpreted as the points at infinity.
\end{definition}

\begin{definition}
   We define \(k[x_1, \dots, x_n]\) to be the ring of polynomials of \(n\)
   variables over \(k\). If  \(f \in k[x_1, \dots, x_n]\) is such a polynomial,
   then it has the form
   \[
      f(x_1, \dots, x_n) = \sum_{(i_k)_{k=1}^n} \bigg(a_{(i_k)} \prod_{k = 1}^n
      x_k^{i_k}\bigg)
   \] 
   where each \(i_k \in \mathbb{Z}_{\geqslant 0}\) and \(a_{(i_k)} \neq 0\).

   A monomial has the form \(\prod_{k=1}^n x_k^{i_k}\) and it's total degree is
   defined to be the sum \(\sum_{k=1}^n i_k\). The total degree of \(f\),
   denoted by \(\deg(f)\), is defined to be the maximum of the degrees of the
   monomials that occur in \(f\). We also extend this notion for each variable,
   thus we may say that the degree of the variable \(x_m\), which we denote by
   \(\deg_m(f)\), is the maximum of it's degrees over all monomials of \(f\).
\end{definition}

\begin{corollary}
   Given \(f, g \in k[x_1, \dots, x_n]\) then
   \begin{enumerate}[I.]
      \item \(\deg(fg) = \deg(f) + \deg(g)\).
      \item  \(\deg_m(fg) = \deg_m(f) + \deg_m(g)\).
   \end{enumerate}
\end{corollary}

\begin{definition}[Homogeneous polynomial]
   We say that a polynomial \(f\) is homogeneous of degree \(\ell\) if every
   monomial of \(f\) has degree \(\ell\). 
\end{definition}

Let \(K \supseteq k\) be a field. Then, if \(f(x_1, \dots, x_n) \in k[x_1, \dots,
x_n]\) we can map \(\A^n(K) \xrightarrow{f} K\) by sending \(a \mapsto f(a)\).
We call \(a \in \A^n(K) : f(a) = 0\) a zero of \(f\).

\begin{definition}[Hypersurface]
   Let \(f \in k[x_1, \dots, x_n]\) be a non-zero polynomial and \(K \supseteq
   k\) be fields. We define the object \(H_f(K) := \{a \in \A^n(K) : f(a) = 0\}
   \) to be the hypersurface defined by \(f\) in \(\A^n(K)\).
\end{definition}

\begin{definition}[Projective hypersurface]
   Let \(K \supseteq k\) be fields and \(h \in k[x_0, \dots, x_n]\) be a non-zero
   homogeneous polynomial of degree \(\ell\). Since \(\forall \gamma \in K\) we
   have \(h(\gamma x) = \gamma^\ell h(x)\) then if \(a \in \A^{n+1}(K)\) is a
   zero of \(h\), it follows that \(h(\gamma a) = \gamma^\ell h(a) = 0\). We
   define the set
    \[
       \overline{H}_h(K) = \{[a] \in \PP^n(K) : h(a) = 0\} 
   \] 
   to be the projective hypersurface defined by \(h\) on \(\PP^n(K)\).
\end{definition}

\begin{definition}[Algebraic variety]
   Let a collection \(f_1, \dots, f_m \in k[x_1, \dots, x_n]\). If the ideal
   \((f_1, \dots, f_m)\) is prime, then
    \[
       V = \{(a_1, \dots, a_n) \in \A^n(k) :f_j(a_1, \dots, a_n) = 0\} 
   \] 
   is an algebraic variety.
\end{definition}

\begin{definition}[Projective algebraic set]
   Let \(h_1, \dots, h_m \in k[x_0, x_1, \dots, x_n]\) be a collection of
   homogeneous polynomials. If the ideal \((h_1,\dots,h_m)\) is a prime ideal,
   then
    \[
       V = \{[a] \in \PP^{n}(k) : h_i(a) = 0\} 
   \] 
   is called a projective algebraic set.
\end{definition}

\begin{proposition}[Homogenization]
   Let \(f \in k[x_1, \dots, x_n]\) and define the polynomial \(\overline{f} \in
   k[y_0, y_1,\dots,y_n]\) such that
   \[
      \overline{f}(y_0, \dots, y_n) := y_0^{\deg(f)} f\left( y_1/y_0, \dots,
      y_n/y_0 \right).
   \] 
   Then \(\overline{f}\) is a homogeneous polynomial with degree equal to
   \(\deg(f)\) and moreover \[\overline{f}(1, y_1, \dots, y_n) = f(y_1, \dots,
   y_n).\]
\end{proposition}

\begin{proof}
   Let \(\prod_{k=1}^n x_k^{i_k}\) be any monomial of \(f\), with degree
   \(\sum_k i_k = \ell \leqslant \deg(f)\). Then, the respective monomial in
   \(\overline{f}\) will have the form \[y_0^{\deg(f)}\prod_{k=1}^n
   y_k^{i_k}/y_0 = y_0^{\deg(f) - \ell} \prod_{k=1}^n y_k^{i_k}\] which
   has degree \(\deg(f)\) and therefore all monomials of \(\overline{f}\) have
   degree \(\deg(f)\) and thus it is homogeneous.
\end{proof}

\begin{definition}[Hypersurface projective closure]
   Let \(H_f(K) \subseteq \A^n(K)\) be the hypersurface generated by \(f \in
   k[x_1, \dots, x_n]\).  Since  \(\overline{f} \in k[y_0, \dots, y_n]\) and is
   homogeneous, then it defines a projective hypersurface
   \(\overline{H}_{\overline{f}} \subseteq \PP^n(K)\). We call
   \(\overline{H}_{\overline{f}}(K)\) the projective closure of \(H_f(K)\) in
   \(\PP^n(K)\).
\end{definition}

\begin{definition}
   We now define another mapping, namely
   \[
      \lambda : \A^n(K) \to \PP^n(K) \text{ such that } (a_1, \dots, a_n)
      \mapsto [1 : a_1 : \cdots : a_n].
   \] 
   Which is clearly injective and also \(\mathrm{im}\big(
   \lambda\big|_{H_f(K)}\big) \subseteq \overline{H}_{\overline{f}}(K)\). Also
   importantly we have 
   \[
      \mathrm{im}(\lambda) \iso \A^n(K)
   \] 
   will be viewed as the finite points.
\end{definition}

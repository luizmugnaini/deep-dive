\section{Group definitions}

Lets recall \cref{def: group} and demystify it in the following definition.

\begin{definition}[Group]
  A group \(G\) is a groupoid \(\cat G\) with one object \(*\). The elements of
  the group are the morphisms \(\Aut_{\cat G}(*)\). From axioms of \cref{def:
  category}:
  \begin{enumerate}[(G1)]
    \item An identity element \(e = \Id_*\).
    \item An associative binary operation \(G \times G \to G\), commonly denoted
      by juxtaposition.
    \item For all \(g \in G\) there exists an inverse element \(g^{-1} \in G\)
      such that \(g g^{-1} = e = g^{-1} g\).
  \end{enumerate}
\end{definition}

\begin{proposition}
  The identity element of a group is unique.
\end{proposition}

\begin{proof}
  See \cref{cor: unique identity}.
\end{proof}

\begin{proposition}
  The inverse of an element of a group is unique.
\end{proposition}

\begin{proof}
  See \cref{prop: iso unique inverse}.
\end{proof}

\begin{proposition}
  Let \(G\) be a group and \(g, h \in G\), then \((g h)^{-1} = h^{-1} g^{-1}\).
\end{proposition}

\begin{proof}
  \((h^{-1} g^{-1})(g h) = h^{-1} (g^{-1} g) h = h^{-1} e h = h^{-1} h = e\)
  hence \(h^{-1} g^{-1} = (g h)^{-1}\).
\end{proof}

\begin{proposition}[Cancellation]
  Let \(G\) be a group and elements \(a, b, c \in G\). If \(a c = b c\) then
  \(a = b\).
\end{proposition}

\begin{proof}
  Notice that \(a = a e = (a c) c^{-1} = (b c) c^{-1} = b e = b\).
\end{proof}

\begin{proposition}
  Let \(G\) be a group. If \(g \in G\), then the collection \(\{g h: h \in G\}\)
  is equal to \(G\).
\end{proposition}

\begin{proof}
  Denote \(G' := \{g h: h \in G\}\). Clearly we have \(G' \subseteq G\). On the
  other hand, if \(\ell \in G\), the element \(g (g^{-1} \ell) = e \ell = \ell
  \in G'\), hence \(G \subseteq G'\). Thus \(G = G'\).
\end{proof}

\begin{definition}[Commutative group]
  A group \(G\) is said to be commutative (or abelian) if for all \(g, h \in G\)
  we have \(g h = h g\).
\end{definition}

\begin{corollary}
  Let \(G\) be a group such that for all \(g \in G\), \(g^2 = e\). Then \(G\) is
  a commutative group.
\end{corollary}

\begin{proof}
  Let \(g, h \in G\) be any elements, then \((g h) (h g) = g h^2 g = g e g = g^2
  = e = (g h) (g h)\) and from cancellation law we find that \(g h = h g\).
\end{proof}

\subsection{Orders}

\begin{definition}[Order of an element]\label{def: group elem order}
  Let \(G\) be a group and \(g \in G\) be any element. We say that \(g\) has
  finite order if there exists \(n \in \Z_{> 0}\) such that \(g^n = e\).
  The order \(|g|\) of the element \(g\) is defined as the smallest such
  positive integer. If \(g\) does not have a finite order, it is common to write
   \(|g| = \infty\).
\end{definition}

\begin{lemma}\label{lem: order and multiples}
  Let \(G\) be a group and \(g \in G\) be an element with finite order. Then
  \(g^n = e\) for some \(n \in \Z_{> 0}\) if and only if \(|g|\) divides
  \(n\).
\end{lemma}

\begin{proof}
  Since \(|g| \leq n\), define \(m \in \Z_{> 0}\) such that \(n - m |g|
  \geq 0\) and \(n - (m + 1) |g| < 0\). Define \(r = n - m |g|\) to be the
  remainder, hence \(r < |g|\). Our goal is to show that \(r = 0\). Notice that
  \(g^r = g^n g^{-|g| m} = e e^{-m} = e\), which can only be the case for \(r =
  0\), since \(|g|\) is defined to be the least positive integer such that
  \(g^{|g|} = e\). This proves that \(m |g| = n\).

  For the second part, suppose \(n\) is a multiple of \(|g|\) and denote it by
  \(m|g| = n\). Then \(g^{m |g|} = e^m = e\).
\end{proof}

\begin{definition}[Order of a group]
  Let \(G\) group of finite number of elements. We define the order of \(G\) to
  be the number of its elements and denote it by \(|G|\). If \(G\) is an
  infinite group, then \(|G| = \infty\).
\end{definition}

\subsubsection{Order of Products}

\begin{proposition}[Order of the power]\label{prop: order of the power}
  Let \(G\) be a group and \(g \in G\) be an element with finite order. Then for
  all \(m \in \Z_{\geq 0}\) the element \(g^m\) has finite order. For
  \(m = 0\) we have \(|g^m| = 1\), for \(m > 0\) we have
  \[
    |g^m| = \frac{\operatorname{lcm}(m, |g|)}{m} =
    \frac{|g|}{\operatorname{gcd}(m, |g|)}.
  \]
\end{proposition}

\begin{proof}
  From divisibility arguments, we have \(\operatorname{lcm}(a, b)
  \operatorname{gcd}(a, b) = ab\) for integers \(a\) and \(b\), hence the second
  equality is justified. We prove the equality \(|g^m| =
  \frac{\operatorname{lcm}(m, |g|)} m\). For the sake of notation, let \(d :=
  |g^m|\). Notice that \(g^{m d} = e\) and hence \(|g|\) divides \(m d\). Since
  \(d\) is the least positive integer with such property, it follows that \(m
  d\) is the least common multiple of \(m\) and \(|g|\). Therefore \(m |g^m| =
  \operatorname{lcm}(m, |g|)\), which proves the equation.
\end{proof}

\begin{proposition}\label{prop: order-prod-commutes}
  Let \(G\) be a group, then for any \(g, h \in G\) we have \(|gh| = |hg|\).
\end{proposition}

\begin{proof}
  First, let \(x, y \in G\) be any elements, we prove that \(|y x y^{-1}| =
  |x|\). Notice that for any \(n \geq 1\) we have \((y x y^{-1})^n = y x^n
  y^{-1}\), hence the least element that annihilates the product \(y x y^{-1}\)
  is the order of \(x\) --- that is, \(|y x y^{-1}| = x\). In particular, \(hg =
  g^{-1} gh g\), hence \(|hg| = |g^{-1} (gh) g| = |gh|\).
\end{proof}

\begin{proposition}\label{prop: commutative-order-of-prod}
  Let \(G\) be a group and \(g, h \in G\) be such that \(g h = h g\). Then \(|g
  h|\) divides \(\operatorname{lcm}(|g|, |h|)\).
\end{proposition}

\begin{proof}
  Let \(n\) be a common multiple of \(|g|\) and \(|h|\), then \(g^n = h^n = e\)
  from \cref{lem: order and multiples}. Notice that the commutative property \(g
  h = h g\) allow us to permute the terms of \(g^n h^n = e\) in order to obtain
  \(g^n h^n = (g h)^n = e\). In particular, since \(\operatorname{lcm}(|g|,
  |h|)\) is a common multiple of \(|g|\) and \(|h|\), we find that \((g
  h)^{\operatorname{lcm}(|g|, |h|)} = e\) and again from \cref{lem: order and
  multiples} we find that \(|g h|\) divides \(\operatorname{lcm}(|g|, |h|)\)
\end{proof}

\begin{lemma}\label{lem: ord-prod-rel-prime}
  Let \(G\) be a group and \(g, h \in G\) commute --- \(gh = hg\). If
  \(\operatorname{gcd}(|g|, |h|) = 1\), then \(|gh| = |g| |h|\).
\end{lemma}

\begin{proof}
  Let \(|gh| = \ell, |g| = m\) and \(|h| = n\). From \cref{prop:
  commutative-order-of-prod}, we have that \(\ell \mid \operatorname{lcm}(m,
  n)\), and since \(m n = \operatorname{lcm}(m, n) \operatorname{gcd}(m, n) =
  \operatorname{lcm}(m, n)\), then \(\ell \mid mn\), which implies that \(\ell
  \leq mn\). Moreover, since the elements commute, \((g h)^\ell = g^\ell h^\ell
  = e\) hence \(g^\ell = (h^\ell)^{-1}\). From \cref{prop: order of the power},
  \(|g^\ell| = \frac{|g|}{\operatorname{gcd}(\ell, |g|)} = \frac m m = 1 \) and
  equivalently for \(h\) we have \(|h^\ell| = 1\). This shows that \(g^\ell =
  h^\ell = e\) and therefore both \(m\) and \(n\) divide \(\ell\), hence so does
  the product \(mn\), thus \(mn \leq \ell\). This completes the proof that
  \(\ell = mn\).
\end{proof}

\begin{definition}[Maximal finite order]\label{def: maximal-finite-order}
  Let \(G\) be a group. An element \(g \in G\) is said to be of maximal finite
  order if its order is finite and for all \(h \in G\) with finite order, we
  have \(|h| \leq |g|\).
\end{definition}

\begin{proposition}
  Let \(G\) be a commutative group and \(g \in G\) be of maximal finite order.
  If \(h \in G\) has finite order, then \(|h|\) divides \(|g|\).
\end{proposition}

\begin{proof}
  Define the notation \(|g| = m\) and \(|h| = n\). Let \(P = (p_j)_j\) be a
  finite sequence containing all primes such that less than or equal to \(m\)
  and define a finite sequence of integers of same length \(A = (a_j)_j\) such
  that \(m = \prod_j p_j^{a_j}\). Since \(n \leq m\) it follows that there also
  exists a finite sequence of integers \(B = (b_j)_j\) such that \(n = \prod_j
  p_j^{b_j}\). Suppose for the sake of contradiction that \(n\) doesn't divide
  \(m\), so that there exists an index \(k\) such that \(a_{k} < b_{k}\)
  --- from the fact that \(\frac m n = \prod_j p_j^{a_j - b_j}\). Consider now
  the order --- following from \cref{lem: ord-prod-rel-prime}:
  \[
    |g^{\left( p_k^{a_k} \right)} h^{\left(n / p_k^{b_k}\right)}|
    = |g^{\left( p_k^{a_k} \right)}| |h^{\left(n / p_k^{b_k}\right)}|
    = \frac m {\operatorname{gcd}(m, p_k^{a_k})}
    \frac n {\operatorname{gcd}(n, n / p_k^{b_k})}
    = \frac m {p_k^{a_k}} \frac n {n / p_k^{b_k}}
    = m p_k^{b_k - a_k}
  \]
  and since \(b_k > a_k\), it follows that \(m p_k^{b_k - a_k} > m\) and hence
  there is a contradiction since we assumed that \(m\) was the the maximal
  finite order of the group. We conclude that there does not exist \(k\) for
  which \(b_k\) is less than \(a_k\) --- thus \(n\) divides \(m\).
\end{proof}

\subsubsection{Finite Groups and Elements of Order 2}

\begin{lemma}[Order 2 elements implies commutative]\label{lem: order2-commutes}
  Let \(G\) be a group such that all non-identity elements have order \(2\).
  Then \(G\) is commutative.
\end{lemma}

\begin{proof}
  Let \(g, h \in G\), notice that \(|gh| = 2\) from \cref{prop:
  commutative-order-of-prod} hence \(gh \in G\), then \((gh)^2 = e\) and
  therefore \(gh = g^{-1} h^{-1} = g^{-1}(ghgh)h^{-1} = hg\) commutes.
\end{proof}

\begin{lemma}
  Let \(G\) be a finite group such that any element has order at most \(2\). Let
  \(H\) be any subgroup of \(G\), for any \(t \in G \setminus H\) consider the
  collection \(T = H \cup \{h t : h \in H\}\), then \(T\) is a subgroup of \(G\)
  with \(|T| = 2|H|\).
\end{lemma}

\begin{proof}
  Since \(t \not\in H\), any element \(ht \in T\) with \(h \in H\) is such that
  \(ht \not\in H\). The number of distinct elements of the form is \(|H|\) ---
  one for each \(h \in H\) ---, hence \(|T| = |H| + |H| = 2|H|\).
\end{proof}

\begin{lemma}[Order \(2^n\)]\label{lem: order-2n}
  Let \(G\) be a finite group such that any element has order at most \(2\). The
  order of \(G\) is of the form \(2^n\) for some \(n \geq 0\). Moreover, if
  \(|G| > 1\), then there exists a subgroup \(H\) of \(G\) with order \(|H| =
  2^{n-1}\).
\end{lemma}

\begin{proof}
  We create a recursive algorithm to find the collection of elements of \(G\)
  that generate any other element contained in \(G\):
  \begin{enumerate}
    \item (Base case) If \(G = H_j\), return \(H_j\).
    \item (Recursion) Let \(g \in G \setminus H_j\) and construct \(H_{j + 1} =
      H_j \cup \{h g: h \in H_j\}\), recursively call the algorithm with \(H_{j
      + 1}\).
  \end{enumerate}
  Such algorithm is certain to terminate since \(|G|\) is finite. Notice that
  the second step always doubles the order of the set \(H_j\), so that --- if
  \(H_n\) is the result of the algorithm --- then \(|H_n| = 2^n\). This shows
  that \(|G| = |H_n| = 2^n\). The second part of the statement follows
  immediately from the construction of the algorithm.
\end{proof}

\begin{proposition}
  Let \(G\) be a commutative group, if there exists exactly one element of order
  \(2\) --- say \(f \in G\) --- then the product of all elements of the group is
  \[
    \prod_{g \in G} g = f.
  \]
  Otherwise, we have \(\prod_{g \in G} g = e\).
\end{proposition}

\begin{proof}
  Since \(G\) is finite, let \(G = \{e, g_1, g_1^{-1}, \dots, g_n, g_n^{-1}\}\).
  Suppose there exists one and only one element of order \(2\) and denote it by
  \(f\) --- that is, \(f = f^{-1}\). Since the group is commutative, we can
  rearrange the product of elements so that we can take the pairwise product of
  each element and its inverse --- this being possible only in the case where
  \(g \neq g^{-1}\) ---, taking the product we find
  \[
    \prod_{g \in G} g = e (g_1 g_1^{-1}) \dots (g_j g_j^{-1}) f \dots (g_{j + 2}
    g_{j + 2}^{-1}) (g_n g_n^{-1}) = e^{j + 1} f e^{n - (j + 1)} = f
  \]
  since \(f\) has no inverse in \(G \setminus \{f\}\).

  Let there be no element with order \(2\) in \(G\), then the rearrangement of
  pairwise element and respective inverse is possible with each \(g \in G\),
  making
  \[
    \prod_{g \in G} g = e (g_1 g_1^{-1}) \dots (g_j g_j^{-1}) \dots (g_n
    g_n^{-1}) = e^{n + 1} = e.
  \]

  Suppose there exists \(m > 1\) distinct elements in \(G\) with order \(2\) and
  define \(T = \{f \in G: |f| \leq 2\}\) and \(S = G \setminus T\). From the
  last case we know that \(\prod_{s \in S} s = e\). Note that \(T\) forms a
  subgroup of \(G\):
  \begin{itemize}
    \item \(e \in T\).
    \item If \(g, h \in T\), then from \cref{prop: commutative-order-of-prod}
      \(|gh|\) divides \(\operatorname{lcm}(|g|, |h|) \leq 2\) hence \(|gh| \leq
      2\) and \(gh \in T\).
    \item Since \(g \in T\) implies \(g g = e\) then \(g = g^{-1}\) and
      \(|g^{-1}| = 2\), so \(g^{-1} \in T\).
  \end{itemize}
  Moreover, the order of the group \(T\) has to be of the form \(2^k\) for some
  \(k \geq 2\), from \cref{lem: order-2n}. From the same lemma, choose a
  subgroup \(H\) with order \(2^{k-1}\) and take \(u \in T \setminus H\), so
  that \(T = H \cup \{hu: h \in H\}\) --- such \(u\) exists from the algorithm
  constructed in the lemma. Since \(T\) is commutative (see \cref{lem:
  order2-commutes}), we can write
  \[
    \prod_{t \in T} t = \prod_{h \in H} h \prod_{h \in H} hu = \prod_{h \in H} u
    h^2 = \prod_{h \in H} u = u^{2^{k-1}} = (u^2)^{2^{k-2}} = e.
  \]
  Therefore we can finally conclude that
  \[
    \prod_{g \in G} g = \prod_{s \in S} s \prod_{t \in T} t = e.
  \]
\end{proof}

\section{Examples of Groups}

\subsection{Symmetry Group}

\begin{definition}[Symmetric groups]\label{def: sym-group}
  Let \(A \in \Set\). The symmetric group of \(A\) --- also referred to as the
  permutation group of \(A\) ---, \(\Aut_\Set(A)\), is denoted by \(\mathcal
  S_A\). The symmetric group of the range set \(\{1, \dots, n\}\) is denoted
  \(\mathcal S_n\).
\end{definition}

\begin{notation}[Permutations]
  A permutation \(\sigma \in \mathcal S_n\) is denoted by a table of the form
  \[
    \sigma =
    \begin{pmatrix}
      1 &2 &\dots &n-1 &n \\
      \sigma(1) &\sigma(2) &\dots &\sigma(n-1) &\sigma(n)
    \end{pmatrix}
  \]
\end{notation}

\begin{remark}[Convention]\label{rem: convention-perm}
  If \(\sigma, \tau \in \mathcal S_n\), we write the composition \(\sigma \tau\)
  to denote the permutation
  \[
    i \xmapsto{\sigma\tau} \sigma(\tau(i)).
  \]
  That is, it follows the same order of composition for maps.
\end{remark}

\begin{proposition}
  There exists an injection \(f: \mathcal S_n \mono M_{n \times n}(\{0, 1\})\)
  --- where \(M_{n \times n}(\{0, 1\})\) is the collection of \(n \times n\)
  matrices with entries assuming the value of \(1\) or \(0\). Moreover, if
  \(f(\sigma) = M_\sigma\) and \(f(\tau) = M_\tau\) then \(f(\sigma \tau) =
  M_\sigma M_\tau\).
\end{proposition}

\begin{proof}
  Let \(\sigma \in \mathcal S_n\) be any permutation. We define \(f\) as the
  mapping
  \[
    \sigma \overset f \longmapsto M_\sigma =
    \begin{bmatrix}
      \delta_{1 \sigma(1)} &\dots &\delta_{1 \sigma(n)} \\
      \vdots &\ddots &\vdots \\
      \delta_{n \sigma(1)} &\dots &\delta_{n \sigma(n)}
    \end{bmatrix}
  \]
  It should be noted that since \(\sigma\) is injective, \(\delta_{i
  \sigma(j)}\) assumes the value \(1\) only once in the \(j\)-th column. If
  \(\sigma = \tau\) then clearly \(\delta_{i \sigma(j)} = \delta_{i \tau(j)}\)
  for every \(1 \leq i, j \leq n\), thus \(f(\sigma) = f(\tau)\) --- \(f\) is
  injective.

  Consider now the composition of permutations \(\sigma \tau \in \mathcal S_n\).
  Suppose we want to check where the \(j\)-th element goes to from the action of
  \(\sigma \tau\). Notice that for all possible new positions \(1 \leq i \leq
  n\) we have \(\delta_{i \sigma\tau(j)} = \sum_{k = 1}^n \delta_{i \sigma(k)}
  \delta_{k \tau(j)}\) --- that is, \(j\) belongs to \(i\) after \(\sigma \tau\)
  if and only if \(\tau(j) = k\) and \(\sigma(k) = i\) for some \(1 \leq k \leq
  n\). We have
  \[
    f(\sigma \tau) = M_{\sigma \tau} =
    \begin{bmatrix}
      \delta_{1 \sigma\tau(1)} &\dots &\delta_{1 \sigma\tau(n)} \\
      \vdots &\ddots &\vdots \\
      \delta_{n \sigma\tau(1)} &\dots &\delta_{n \sigma\tau(n)}
    \end{bmatrix}
    =
    \begin{bmatrix}
      \delta_{1 \sigma(1)} &\dots &\delta_{1 \sigma(n)} \\
      \vdots &\ddots &\vdots \\
      \delta_{n \sigma(1)} &\dots &\delta_{n \sigma(n)}
    \end{bmatrix}
    \begin{bmatrix}
      \delta_{1 \tau(1)} &\dots &\delta_{1 \tau(n)} \\
      \vdots &\ddots &\vdots \\
      \delta_{n \tau(1)} &\dots &\delta_{n \tau(n)}
    \end{bmatrix}
    = M_\sigma M_\tau.
  \]
\end{proof}

\begin{proposition}\label{prop: sym-n-all-orders}
  The symmetry group \(\mathcal S_n\) contains elements of all orders \(d\) for
  \(1 \leq d \leq n\).
\end{proposition}

\begin{proof}
  Let \(d\) be any integer in the range \(1 \leq d \leq n\).
  \todo[inline]{Write proof}
\end{proof}

\begin{corollary}
  For every \(n \in \N\), there exists an element \(x \in \mathcal S_\N\) with
  order \(|x| = n\).
\end{corollary}

\begin{proof}
  Let any \(n \in \N\) and choose \(m \in \N\) with \(m \geq n\). From
  \cref{prop: sym-n-all-orders} we find that the group \(\mathcal S_m\) contains
  an element \(\sigma \in \mathcal S_m\) with order \(|\sigma| = n\). We can now
  construct an element \(\tau \in \mathcal S_\N\) --- defined by
  \[
    \tau(a) =
    \begin{cases}
      \sigma(a), &a \leq m \\
      a, &a > m
    \end{cases}
  \]
  So clearly \(|\tau| = n\).
\end{proof}

\subsubsection{Permutations, Transpositions and Sign}

\begin{definition}[Transposition]
  \label{def: transposition}
  We define a transposition on a collection \(\{1, \dots, n\}\) to be a map
  \(\tau \in \mathcal S_n\) such that exists indices \(1 \leq i < j \leq n\)
  for which \(\tau(i) = j\) and \(\tau(j) = i\), and \(\tau(k) = k\) for all
  \(k \neq i, j\).
\end{definition}

\begin{proposition}\label{prop: permutations to transpositions}
  Every permutation can be written as a composition of finitely many
  transpositions.
\end{proposition}

\begin{proof}
  We proceed via induction on the number of points of \(\{1, \dots, n\}\). For
  the base case \(n = 2\) the composition is trivial. Assume as the induction
  hypothesis that for \(n - 1 > 2\) the statement is true. Now, consider
  \(\sigma \in \mathcal S_n\) and \(i \in \{1, \dots, n\}\) be any element in
  the ordered collection \(I_n := \{1, \dots, n\}\). Denote \(\tau_{i, j}\) the
  transposition that changes \(i\) with \(j\) and maintains unchanged the
  remainder of the points. Assume that \(\sigma(i) = j\), then clearly
  \(\tau_{i, j} \sigma(i) = \tau_{i, j}(j) = i\). Notice that since \(\tau_{i,
  j} \sigma\) maintains \(i\) unchanged, we can see it as a permutation of \(n
  - 1\) points (by simply ignoring the point \(i\)), hence it can be written as
  a composition of finitely many transpositions by the induction hypothesis.
  Notice that since \(\sigma = \tau_{i, j} (\tau_{i, j} \sigma)\) we find that
  \(\sigma\) can also be written as a composition of finitely many
  transpositions.
\end{proof}

\begin{definition}[Elementary transpositions]
  Let \(\tau \in \mathcal S_n\) be a transposition. We say that \(\tau\) is an
  elementary transposition if exists \(i \in \{1, \dots, n\}\) for which
  \(\tau(i) = i + 1\) and \(\tau(i + 1) = i\), and for all \(j \neq i\) we have
  \(\tau(j) = j\).
\end{definition}

\begin{proposition}
  Every transposition can be written as a composition of finitely many
  elementary transpositions.
\end{proposition}

\begin{proof}
  Let \(\sigma \in \mathcal S_n\) be any transposition. Denote by \(\tau_k\) the
  elementary transposition \(\tau_k(k) = k + 1\) and \(\tau(k + 1) k\). Let
  \(x\) be the transposed element of \(\sigma\), and \(\sigma(x) = y\). Without
  loss of generality we can assume that \(y > x\). Now we write \(\sigma\) as
  the composition --- beware of \cref{rem: convention-perm}
  \[
    \sigma = \left(\tau_x \tau_{x + 1} \cdots \tau_{y - 3} \tau_{y - 2} \right)
    \left( \tau_{y - 1} \tau_{y - 1} \cdots \tau_{x + 1} \tau_{x} \right)
    = \left( \prod_{k = x}^{y-2} \tau_k \right)
    \left( \prod_{k = 0}^{(y - 1) - x} \tau_{(y-1) - k} \right)
  \]
  Which follows from the fact that
  \begin{gather*}
    \left( \prod_{k = 0}^{(y - 1) - x} \tau_{(y-1) - k} \right)(i) =
    \begin{cases}
      y, &i = x \\
      i - 1, &x < i \leq y \\
      i, &i < x \text{ or } y < i
    \end{cases}
    \\
    \left( \prod_{k = x}^{y-2} \tau_k \right)(i) =
    \begin{cases}
      x, &i = y \\
      i + 1, &x \leq i < y \\
      i, &i < x \text{ or } y < i
    \end{cases}
  \end{gather*}
  Hence the composition of both gives
  \[
    \left( \prod_{k = x}^{y-2} \tau_k \right)
    \left( \prod_{k = 0}^{(y - 1) - x} \tau_{(y-1) - k} \right)(i) =
    \begin{cases}
      y, &i = x \\
      x, &i = y \\
      i, &i \not\in \{x, y\}
    \end{cases}
  \]
  which is equivalent to the transposition \(\sigma\).
\end{proof}

\begin{corollary}
  Every permutation can be written as a composition of finitely many elementary
  transpositions.
\end{corollary}

\begin{definition}[Sign]
  \label{def: sign}
  Let \(\sigma \in \mathcal S_n\) be a permutation on \(n\) objects. We define
  the sign of \(\sigma\) as a map \(\sign: \mathcal S_n \to
  \{1, -1\}\) such that
  \[
    \sign(\sigma) = (-1)^m, \text{ where }
    m := |\{(i, j) : 1 \leq i < j \leq n,\ \sigma(i) > \sigma(j)\}|.
  \]
  Equivalently, if \(x_1, \dots, x_n \in k\) then we can define the
  sign of \(\sigma\) as
  \[
    \sign(\sigma) = \prod_{i < j} \frac{x_{\sigma(i)} -
    x_{\sigma(j)}}{x_i - x_j}.
  \]
\end{definition}

\begin{corollary}
  If \(\sigma\) can be written as an odd number of transpositions, then
  \(\sign(\sigma) = -1\). Otherwise, if it is an even number of
  transpositions, we have \(\sign(\sigma) = 1\).
\end{corollary}

\begin{proposition}\label{prop: sign is a group homomorphism}
  Let \(\sigma, \tau \in \mathcal S_n\), then
  \[
    \sign(\sigma \tau) = \sign(\sigma)
    \sign(\tau).
  \]
  This implies that \(\sign: \mathcal S_n \to \{1, -1\}\) is a
  group homomorphism.
\end{proposition}

\begin{proof}
  Consider the compostion \(\sigma \tau\), then we have
  \begin{equation}\label{eq: sign sigma tau}
     \sign(\sigma \tau) = \prod_{i < j} \frac{x_{\sigma\tau(i)} -
     x_{\sigma\tau(j)}}{x_i - x_j}.
  \end{equation}
  Define the following: when \(\tau(i) < \tau(j)\) then \(\tau(i) := p\) and
  \(\tau(j) := q\); on the other hand, when \(\tau(j) < \tau(i)\) let \(\tau(i)
  := q\) and \(\tau(j) := p\). This way we find that
  \begin{equation}\label{eq: sigma term}
    \frac{x_{\sigma\tau(i)} - x_{\sigma\tau(j)}}{x_{\tau(i)} - x_{\tau(j)}}
    = \frac{x_{\sigma(p)} - x_{\sigma(q)}}{x_p - x_q}.
  \end{equation}
  From construction of the above, \cref{eq: sigma term} yields
  \begin{equation}\label{eq: sign sigma}
    \sign(\sigma) = \prod_{p < q} \frac{x_{\sigma(p)} -
    x_{\sigma(q)}}{x_p - x_q}.
  \end{equation}
  Notice that we can write \cref{eq: sign sigma tau} (using \cref{eq: sign
  sigma}) as the product
  \[
    \sign(\sigma \tau) = \prod_{i < j}
    \frac{x_{\sigma \tau(i)} - x_{\sigma \tau(j)}}{x_{\tau(i)} - x_{\tau(j)}}
    \frac{x_{\tau(i)} - x_{\tau(j)}}{x_i - x_j}
    = \sign(\sigma) \sign(\tau)
  \]
  as wanted.
\end{proof}

\subsection{Dihedral Group}

\begin{definition}[Dihedral group]\label{def: dihedral}
  A dihedral group is defined as the group of isometric symmetries of regular
  polygons --- which are rotations and reflections about a line. Given a
  \(n\)-sided regular polygon, its group of symmetries have \(2n\) elements ---
  \(n\) rotations and \(n\) reflections --- and we denote it by \(D_{2n}\).
\end{definition}

\begin{proposition}\label{prop: dihetral-to-sym}
  There exists an injection \(D_{2n} \mono \mathcal S_n\).
\end{proposition}

\begin{proof}
  Label the vertices of the \(n\)-gon by \(V = \{[1]_n, \dots, [n]_n\} =
  \Z/n\Z\). Notice that any element \(x \in D_{2n}\) can be described as \(x \in
  \Aut_\Set(V)\) --- where the automorphism is restricted to the adjacency of
  the vertices, that is, if \(x([i]_n) = [k]_n\), then \(x([i - 1]_n), x([i +
  1]_n) \in \{[k - 1]_n, [k + 1]_n\}\). This shows the existence of the
  inclusion \(D_{2n} \mono \Aut_\Set(V) = \mathcal S_V = \mathcal S_n\).
\end{proof}

\begin{proposition}
  Any symmetry \(x \in D_{2n}\) can be written as \(y^a z^b\) --- where we
  choose any \(y, z \in D_{2n}\) which are, respectively, a rotation and a
  reflection about a line --- with \(0 \leq a < 2\) and \(0 \leq b < n\).
\end{proposition}

\begin{proof}
  Notice that if \(y\) is any rotation, then \(|y| = n\) and if \(z\) is any
  reflection, then \(|z| = 2\).

  We first show that \(y\) and \(z\) are independent. We'll work with the
  injection \(D_{2n} \mono \Aut_\Set(V)\), where we have the collection of
  vertices \(V = \Z/n\Z\). Let \(j \in V\) be any vertex and suppose \(z(j) =
  k\), then by the adjecency of the vertices are maintained, implying in \(z(j -
  1) = k - 1\) and \(z(j + 1) = k + 1\). On the other hand, if \(y(j) = k\),
  then the adjacency of the vertices is inverted, that is, \(y(j - 1) = k + 1\)
  and \(y(j + 1) = k - 1\). Hence clearly reflections and rotations cannot be
  dependent.

  The only possible symmetries of \(D_{2n}\) involve the maintainence or the
  inversion of the adjacency of each vertex --- we cannot break nor deform the
  edges that connect each of the vertices ---, thus if \(x \in D_{2n}\), then
  \(x = \prod_{(\alpha, \beta) \in I} z^\alpha y^\beta\) for some finite set
  \(I\) with \(0 \leq \alpha < 2\) and \(0 \leq \beta < n\). We now show that such
  product can be reduced. Let \(j \in V\) be any vertex and assume
  \[
    z(j) = k
    \quad\text{ and }\quad
    y(k) = \ell,
  \]
  We now analyse the symmetries \(y z\) and \(z y^m\) --- for \(0 \leq m < n\).
  \begin{itemize}
    \item The adjacent vertices of \(j\) --- when subjected to the symmetry \(y
      z\) --- are obtained as
      \[
        y z(j - 1) = y(k + 1) = \ell + 1
        \quad\text{ and }\quad
        y z(j + 1) = y(k - 1) = \ell - 1
      \]
    \item Since \(y(k) = \ell\), \(y\) is the action rotating the vertex \(k\)
      an amount of \(\ell - k\) times  --- where we'll adopt the convention that
      if \(\ell - k < 0\), the rotation is counter-clockwise, if \(\ell + k >
      0\) the rotation is clockwise. Therefore
      \[
        y(j) = j + (\ell - k).
      \]
      Notice that this implies in \(y^m(j) = j + m(\ell - k)\) --- we now define
      \(t = y^m(j)\). Since \(z(j) = k\), then \(z(t) = k - m(\ell - k)\).
      Therefore, adjacent vertices of \(j\) --- when subjected to the symmetry
      \(z y^m\) --- are given by
      \begin{gather*}
        \text{Vertex } j - 1: \quad
        z y^m(j - 1) = z(j + m (\ell - k) - 1) = k - m(\ell - k) + 1 \\
        \text{Vertex } j + 1: \quad
        z y^m(j + 1) = z(j + m (\ell - k) + 1) = k - m (\ell - k) - 1
      \end{gather*}
      In particular, for the case \(m = n - 1\) --- beware of the modularity of
      the vertices, they lie on \(\Z/n\Z\) --- we get
      \begin{align*}
        &\text{Vertex } j - 1\text{:}\
        zy^{n-1}(j - 1)
        = k - (n - 1) (\ell - k) + 1
        = k n - n \ell + \ell + 1
        = \ell + 1
        \\
        &\text{Vertex } j\text{:} \qquad
        zy^{n-1}(j)
        = k - (n - 1) (\ell - k)
        = kn - n \ell + \ell
        = \ell
        \\
        &\text{Vertex } j + 1\text{:}\
        zy^{n-1}(j + 1)
        = k - (n - 1) (\ell - k) - 1
        = k n - n \ell + \ell - 1
        = \ell + 1
      \end{align*}
  \end{itemize}
  From this analysis we conclude the general relation \(y z = z y^{n-1}\).

  Lets analyse the product \((z^a y^b)(z^c y^d)\) for the different cases of \(0
  \leq a, c < 2\) and \(0 \leq b, d < n\).
  \begin{itemize}
    \item If \(a = 1\) and \(c = 0\) then \((z^a y^b)(z^c y^d) = z y^{b + d}\).
    \item If \(a, c = 0\) then
      \[
        (z^a y^b)(z^c y^d) = y^{b + d}.
      \]
    \item If \(a = 0\) and \(c = 1\) then \((z^a y^b)(z^c y^d) = y^b z^c y^d =
      y^{b-1} z y^{d + (n - 1)}\), by recurrence we find find finally that
      \[
        (z^a y^b)(z^c y^d) = z y^{d + b(n - 1)}.
      \]
    \item If \(a, c = 1\) then \((z^a y^b)(z^c y^d) = z y^b z y^d = z y^{b-1} z
      y^{d + (n - 1)}\) and, by recurrence
      \[
        (z^a y^b)(z^c y^d) = z^2 y^{d + b(n - 1)} = y^{d + b(n - 1)}.
      \]
  \end{itemize}
  This shows that the finite product \(x = \prod_{(\alpha, \beta) \in I}
  z^\alpha y^\beta\) can be reduced --- for some \(0 \leq a < 2\) and \(0 \leq b
  < n\) --- to \(x = z^a y^b\).
\end{proof}

\begin{corollary}
  Let \(y, z \in D_{2n}\) be, respectively, a rotation and a reflection. The
  following relation are satisfied:
  \[
    |y| = n \text{, } |z| = 2 \text{, and}\ yz = z y^{n-1}.
  \]
  The set \(\{y, z\}\) generates any element of \(D_{2n}\).
\end{corollary}

\subsection{Cyclic Group}

\begin{definition}[Cyclic group]\label{def: cyclic-grp}
  A cyclic group is defined as a group generated by one element \(x\) with a
  relation \(x^n = e\) for some \(n \in \N\). We'll denote such groups by
  \(C_n\).
\end{definition}

\begin{example}
  An example of a cyclic group is \((\Z/n\Z, +)\), where \([1]_n \in \Z/n\Z\) is
  the element generating any elements of \(\Z/n\Z\).
\end{example}

\begin{proposition}
  Given an element \(y \in C_n\) --- where \(x\) is the generating element of
  \(C_n\) ---, assume \(y = x^m\). We have that
  \[
    |y| = \frac n {\operatorname{gcd}(m, n)}.
  \]
\end{proposition}

\begin{proof}
  Notice that \(|y| = |x^m| = \frac{|x|}{\operatorname{gcd}(m, |x|)} = \frac n
  {\operatorname{gcd}(m, n)}\).
\end{proof}

\begin{corollary}
  The element \(x \in C_n\) is the generator of the cyclic group if and only if
  \(\operatorname{gcd}(|x|, n) = 1\).
\end{corollary}

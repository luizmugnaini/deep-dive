\section{Projective Modules}

\begin{theorem}[Free modules have the lifting property]
\label{thm:free-modules-are-projective}
Let \(R\) be a ring and \(F\) be a free \(R\)-module. For every surjective
morphism of \(R\)-modules \(p: M \to N\) and morphism \(h: F \to N\), there
\emph{exists a unique morphism} \(\ell: F \to M\), called \emph{lifting of
  \(h\)}, such that the diagram
\[
\begin{tikzcd}
&F \ar[d, "h"] \ar[dl, bend right, dashed, "\ell"'] & & \\
M \ar[r, two heads, "p"'] &N \ar[r] &0
\end{tikzcd}
\]
commutes in \(\Mod{R}\).
\end{theorem}

\begin{proof}
Since \(F\) is free, let \(B \coloneq (b_j)_{j \in J}\) be a basis for
\(F\). From the surjectivity of \(p\), for every \(j \in J\) there exists
\(m_j \in M\) such that \(p(m_j) = h(b_j)\). From the free module universal
property there exists a unique \(\ell: F \to M\) such that \(\ell(b_j) = m_j\)
for each \(j \in J\). From construction we have
\(p \ell(b_j) = p(m_j) = h(b_j)\), therefore \(p \ell = h\), since \(B\)
generates \(F\), and the diagram commutes.
\end{proof}

\begin{remark}[Uniqueness of the lift]
\label{rem:uniqueness-of-lifting}
The lift \(\ell\) of \(h\) \emph{need not be unique} in case \(F\) isn't free!
\end{remark}

\begin{definition}[Projective module]
\label{def:projective-module}
Let \(R\) be a ring and \(P\) be an \(R\)-module. We say that \(P\) is a
\emph{projective \(R\)-module} if
\[
\Hom_{\Mod{R}}(P, -): \Mod{R} \longrightarrow \Ab
\]
is an \emph{exact covariant functor}.
\end{definition}


\begin{proposition}[Equivalences for projective modules]
\label{prop:equivalences-projective-module}
Let \(R\) be a ring and \(P\) be an \(R\)-module. The following properties are equivalent:
\begin{enumerate}[(a)]\setlength\itemsep{0em}
\item The module \(P\) is \emph{projective}.

\item For every exact sequence of \(R\)-modules \(M \overset{g}\epi N \to 0\)
  the sequence of abelian groups
  \[
  \begin{tikzcd}
  \Hom_{\Mod{R}}(P, M) \ar[r, "g_{*}", two heads]
  &\Hom_{\Mod{R}}(P, N) \ar[r] &0
  \end{tikzcd}
  \]
  is \emph{exact}. Equivalently, for every \(h: P \to N\), there \emph{exists a
    lifting} \(\ell: P \to N\) of \(h\)---\emph{not necessarily unique}---such
  that the diagram
  \[
  \begin{tikzcd}
  &P \ar[d, "h"] \ar[dl, bend right, "\ell"'] & & \\
  M \ar[r, two heads, "p"'] &N \ar[r] &0
  \end{tikzcd}
  \]
  is commutative in \(\Mod{R}\).

\item Every short exact sequence of \(R\)-modules of the form
  \[
  \begin{tikzcd}
  0 \ar[r] &L \ar[r, tail] &M \ar[r, two heads] &P \ar[r] &0
  \end{tikzcd}
  \]
  is a \emph{split} sequence.

\item The module \(P\) is a \emph{direct summand} of a \emph{free} \(R\)-module.

\item There exists a collection \((x_j)_{j \in J}\) of elements \(x_j \in P\),
  and a collection of associated morphisms of \(R\)-modules
  \((\phi_j: P \to R)_{j \in J}\)\footnote{The collection of pairs
    \((x_j, \phi_j: P \to R)_{j \in J}\) is sometimes referred to as the
    \emph{dual ``basis''}, but it should be noted right away that such family
    \emph{may not form a basis} for the module \(P\)---\emph{not every
      projective module is free}!} such that, for all \(x \in P\) we have:
\begin{itemize}\setlength\itemsep{0em}
\item The elements \(\phi_j(x) \in R\) are \emph{non-zero for only finitely many
  \(j \in J\)}.
\item The element \(x\) can be written as \(x = \sum_{j \in J} x_j \phi_j(x)\).
\end{itemize}
\end{enumerate}
\end{proposition}

\begin{proof}
From the definition, the equivalence of (a) and (b) is immediate. We prove the
following:
\begin{itemize}\setlength\itemsep{0em}
\item (b) \(\implies\) (c). Let \(g: M \epi P\) be the epimorphism depicted in
  the sequence. Consider the identity morphism \(\Id_P: P \to P\) and apply (b)
  to obtain \(\rho: P \to M\) such that
  \[
  \begin{tikzcd}
  &P \ar[d, "\Id_P"] \ar[ld, "\rho"', bend right] & \\
  M \ar[r, "g"', two heads] &P \ar[r] &0
  \end{tikzcd}
  \]
  is a commutative diagram of \(R\)-modules. Notice that \(g \rho = \Id_P\),
  therefore \(\rho\) is a section of \(g\)---thus the sequence splits.

\item (c) \(\implies\) (d). Via
  \cref{thm:any-module-is-quotient-of-free-module} let \(p: F \epi M\) be a
  surjective morphism of \(R\)-modules, where \(F\) is free. Thus we have a
  short exact sequence
  \[
  \begin{tikzcd}
  0 \ar[r] &\ker p \ar[r, hook] &F \ar[r, "p", two heads] &P \ar[r] &0.
  \end{tikzcd}
  \]
  By item (c) we find that the above sequence is split, therefore if \(\iota: P
  \mono F\) is a section of \(p\), then
  \[
  F = \ker p \oplus \im \iota \iso \ker p \oplus P.
  \]

\item (d) \(\implies\) (b). Let \(f: M \epi N\) be a surjective morphism of
  \(R\)-modules, and \(\psi: P \to N\) be any morphism. By item (d), let \(F\)
  be a free module with \(F \iso P \oplus P'\), where \(P'\) is the complement
  \(R\)-module of \(P\) with respect to \(F\)---also, let \(B\) be a basis of
  \(F\). Considering the natural projection \(\pi_P: F \epi P\), define a map
  \(\phi': F \to M\) as follows: given \(b \in B\), by the surjectivity of
  \(f\), there exists \(m \in M\) such that \(f(m) = \psi \pi_P(b)\)---we shall
  define \(\phi'(b) \coloneq m\). It is easily seen that \(\phi'\) is
  \(R\)-linear, and that \(f \phi' = \psi \pi_P\). Moreover, the surjectivity of
  \(f\) implies that \(\phi'\) is the unique morphism of modules with such
  property. Considering the natural inclusion \(\iota_P: P \emb F\)---which is a
  section of \(\pi_P\)---define \(\phi \coloneq \phi' \iota_P: P \to M\) and
  notice that
  \[
  f \phi
  = f (\phi' \iota_P)
  = (f \phi') \iota_P
  = (\psi \pi_P) \iota_P
  = \psi (\pi_P \iota_P)
  = \psi.
  \]
\end{itemize}
This finishes the proof of the equivalence of the items (a), (b), (c), and
(d). For item (e), we shall prove its equivalence with (d).
\begin{itemize}\setlength\itemsep{0em}
\item (d) \(\implies\) (e). Via item (d), there exists an indexing set \(J\) and
  an isomorphism \(\psi: \bigoplus_{j \in J} R \isoto P \oplus P'\), where
  \(P'\) is the complement of \(P\). If \(\pi_P: P \oplus P' \epi P\) denotes
  the canonical projection, define a collection \((x_j)_{j \in J}\) by
  \(x_j \coloneq \pi_P \psi(e_j)\)---where
  \(e_j \coloneq (\delta_{ij})_{i \in J}\). Let \(\iota_P: P \emb P \oplus P'\)
  be the canonical inclusion of \(P\), and
  \(\pi_j: \bigoplus_{j \in J} R \epi R\) be the canonical projection of the
  \(j\)-th coordinate. Define a collection of morphisms \((\phi_j)_{j \in J}\)
  by \(\phi_j \coloneq \pi_j \psi^{-1} \iota_P: P \to R\), so that---since
  \(\psi^{-1} \iota_P(x) \in \bigoplus_{j \in J} R\) has finitely many non-zero
  components---there are finitely many \(j \in J\) such that
  \(\pi_j \psi^{-1} \iota_P(x) \in R\) is non-zero. For the last condition of
  item (e), if \(x \in P\) is any element, we have
  \begin{align*}
  x &= \pi_P \iota_P(x)
  = \pi_P(\psi \psi^{-1}) \iota_P(x)
  = \pi_P \psi (\psi^{-1} \iota_P)(x)
  = \pi_P \psi (\phi_j(x))_{j \in J} \\
  &= \pi_P \psi \bigg( \sum_{j \in J} e_j \phi_j(x) \bigg)
  = \sum_{j \in J} \pi_P \psi(e_j \phi_j(x))
  = \sum_{j \in J} \pi_P \psi(e_j) \phi_j(x) \\
  &= \sum_{j \in J} x_j \phi_j(x).
  \end{align*}

\item (e) \(\implies\) (d). The collection \((\phi_j)_{j \in J}\) induces, by
  the universal property of the product, a unique morphism of \(R\)-modules
  \(\phi: P \to \prod_{j \in J} R\) mapping \(x \mapsto (\phi_j(x))_{j \in
    J}\). For any \(x \in P\) we know from hypothesis that the collection
  \((\phi_j(x))_{j \in J}\) has finitely many non-zero elements, therefore
  \(\im \phi \subseteq \bigoplus_{j \in J} R\). Then we may naturally restrict
  the codomain of \(\phi\), obtaining a morphism
  \(\phi: P \to \bigoplus_{j \in J} R\). Define a collection \((e_j)_{j \in J}\)
  where \(e_j \coloneq (\delta_{ij})_{i \in J}\), and consider an \(R\)-linear
  map \(\lambda: \bigoplus_{j \in J} R \to P\) defined by sending
  \(e_j \mapsto x_j\). By hypothesis, one has
  \[
  \lambda \phi(x)
  = \lambda(\phi_j(x))_{j \in J}
  = \sum_{j \in J} x_j \phi_j(x)
  = x,
  \]
  therefore \(\phi\) is a section of \(\lambda\), showing that \(\lambda\) is a
  split epimorphism. Thus \(P\) is a direct summand of the free module
  \(\bigoplus_{j \in J} R\).
\end{itemize}
\end{proof}

\begin{example}
\label{exp:free-mod-is-projective}
From \cref{thm:free-modules-are-projective} we find that every free module is a
projective module.
\end{example}

\begin{example}
\label{exp:idempotent-ring-projective}
Let \(R\) be a ring. If there exists an idempotent element \(e \in R\), then we
have a decomposition \(R = e R \oplus (1 - e) R\). Therefore \(e R\) is
projective.
\end{example}

\begin{proposition}
\label{prop:direct-sum-projective}
Let \((P_j)_{j \in J}\) be a family of \(R\)-modules. Then the module
\(P \coloneq \bigoplus_{j \in J} P_j\) is projective if and only if \(P_j\) is
projective for each \(j \in J\).
\end{proposition}

\begin{proof}
Suppose that \(P\) is projective, then we let \(F\) be a free module of which
\(P\) is a direct summand, say \(F = P \oplus P'\). If \(\sigma: J \to J\) is
any permutation, we know that \(\bigoplus_{j \in J} P_j \iso
\bigoplus_{j \in J} P_{\sigma(j)}\), therefore for all \(i \in J\) we have
\[
F = P \oplus P' = \Big( \bigoplus_{j \in J} P_j \Big) \oplus P'
\iso \Big( P_i \oplus \bigoplus_{j \in J \setminus i} P_j \Big) \oplus P'
= P_i \oplus \bigg(\Big(\bigoplus_{j \in J \setminus i} P_j \Big) \oplus P'\bigg)
\]
Since \(P_i\) is a direct summand of a free module, it is a projective module.

For the converse, suppose that \(P_j\) is projective for all \(j \in J\). Let
\(g: M \epi N\) be a surjective morphism of \(R\)-modules, and \(\phi: P \to N\)
be any morphism. Since \(P_j\) is projective, if \(\iota_j: P_j \emb P\) is the
canonical inclusion, there exists a morphism \(\psi_j: P_j \to M\) making the
diagram
\[
\begin{tikzcd}
&P_j \ar[d, "\phi \iota_j"] \ar[dl, bend right, "\psi_j"'] &\\
M \ar[r, two heads, "g"'] &N \ar[r] &0
\end{tikzcd}
\]
commute in \(\Mod{R}\). By the universal property of the coproduct \(P\), the
collection of morphisms \((\psi_j)_{j \in J}\) induce a unique morphism \(\psi:
P \to M\) such that \(\psi \iota_j = \psi_j\) for all \(j \in J\). To show that
\(g \psi = \phi\), notice that, given any \((x_j)_{j \in J} \in P\), one has
\[
g \psi(x_j)_{j \in J} = g \Big(\sum_{j \in J} \psi_j(x_j)\Big)
= \sum_{j \in J} g \psi_j(x_j)
= \sum_{j \in J} \phi \iota_j(x_j)
= \phi(x_j)_{j \in J}.
\]
Therefore, the following diagram commutes in \(\Mod{R}\):
\[
\begin{tikzcd}
&P \ar[d, "\phi"] \ar[dl, bend right, "\psi"'] &\\
M \ar[r, two heads, "g"'] &N \ar[r] &0
\end{tikzcd}
\]
which shows that \(P\) is a projective module.
\end{proof}

\begin{proposition}
\label{prop:fg-projective-then-dual-module-is-projective}
Let \(P\) be a finitely generated projective right-\(R\)-module (or left). Then
the dual module \(P^{*}\) is a projective left-\(R\)-module (or right).
\end{proposition}

\begin{proof}
If \(P\) is finitely generated and projective we can choose a finitely generated
free right-\(R\)-module \(F\) of which \(P\) is a direct summand---say, \(F = P
\oplus P'\). Since \(F\) is finitely generated, there exists \(n \in \Z_{> 0}\)
such that \(F \iso \bigoplus_{j=1}^n R\). We know that there exists a natural
isomorphism of \emph{left}-\(R\)-modules
\[
F^{*} = \Hom_{\rMod{R}}(P \oplus P', R)
\iso \Hom_{\rMod{R}}(P, R) \oplus \Hom_{\rMod{R}}(P', R)
= P^{*} \oplus P'^{*}.
\]
Moreover, we can rewrite the morphism set using the fact that \(F\) is finitely
generated:
\[
F^{*} \iso \Hom_{\rMod{R}}\Big(\bigoplus_{j=1}^n R, R\Big)
\iso \bigoplus_{j=1}^n \Hom_{\rMod{R}}(R, R)
\iso \bigoplus_{j=1}^n R,
\]
therefore \(F^{*}\) is a finitely generated free left-\(R\)-module. Since
\(P^{*}\) is a direct summand of \(F^{*}\), then \(P^{*}\) is a finitely
generated \emph{projective} left-\(R\)-module.
\end{proof}

\begin{proposition}
\label{prop:projective-module-evaluation-morphism-is-injective}
Let \(P\) be a right-\(R\)-module (or left). The \emph{double dual} \(P^{* *}\)
is a \emph{right}-\(R\)-module (or left), and the natural evaluation map
\[
\eval: P \longrightarrow P^{* *}
\]
sending \(x \mapsto \eval_x: P^{*} \to R\)---where \(\eval_x(f) = f(x)\)---is an
\emph{injective morphism of right-\(R\)-modules}.
\end{proposition}

\begin{proof}
Let \((x_j, \phi_j)_{j \in J}\) be the dual ``basis'' of \(P\), where
\(x_j \in P\) and \(\phi_j \in P^{*}\). If \(x \in \ker \eval\), then
\(f(x) = 0\) for all \(f \in P^{*}\)---thus in particular \(x = \sum_{j \in J}
x_j \phi_j(x) = 0\). This shows that \(\ker \eval = 0\), thus the evaluation
morphism is injective.
\end{proof}

\begin{proposition}
\label{prop:properties-of-fg-proj-modules-and-duals}
Let \(P\) be a finitely generated projective right-\(R\)-module (or left), with
a dual ``basis'' \((x_j, \phi_j)_{j=1}^n\). The following properties hold:
\begin{enumerate}[(a)]\setlength\itemsep{0em}
\item The finite collection \((\phi_j, \eval_{x_j})_{j=1}^n\) forms a \emph{dual
  ``basis''} for the dual module \(P^{*}\).
\item The dual \(P^{*}\) is a \emph{finitely generated projective
    left-\(R\)-module} (or right), with a generating set \((\phi_j)_{j=1}^n\).
\item The double dual \(P^{* *}\) is a \emph{finitely generated projective
    right-\(R\)-module} (or left), with a generating set
  \((\eval_{x_j})_{j=1}^n\).
\item The natural evaluation morphism \(\eval: P \mono P^{* *}\) is an
  \emph{isomorphism} of right-\(R\)-modules (or left).
\end{enumerate}
\end{proposition}

\begin{proof}
\begin{enumerate}[(a)]\setlength\itemsep{0em}
\item Let \(f \in P^{*}\) be any functional, then given any \(x \in P\) one has
  \[
  \sum_{j=1}^n \eval_{x_j}(f) \phi_j(x)
  = \sum_{j=1}^{n} f(x_j \phi_j(x))
  = f\Big( \sum_{j=1}^{n} x_j \phi_j(x) \Big)
  = f(x),
  \]
  therefore \(f = \sum_{j=1}^n \eval_{x_j}(f) \phi_j\)---which proves that
  \((\phi_j, \eval_{x_j})_j\) is a dual basis for \(P^{*}\).

\item It is immediate from the last item's proof that \((\phi_j)_{j=1}^n\) is a
  generating set for \(P^{*}\). Moreover, since \(P^{*}\) admits a dual basis it
  follows that \(P^{*}\) is projective.

\item Define a map \(\eval^{*}: P^{*} \to (P^{* *})^{*}\) given by \(f \mapsto
  \eval_f^{*} \in \Hom_{\rMod{R}}(P^{* *}, R)\), where \(\eval_f^{*}(\Phi)
  \coloneq \Phi(f)\) for any \(\Phi \in \Hom_{\lMod{R}}(P^{*}, R)\). Notice
  that, for any \(f \in P^{*}\) one has
  \[
  \sum_{j=1}^n \eval_{x_j}(f) \eval_{\phi_j}^{*}(\Phi)
  = \sum_{j=1}^n f(x_j) \Phi(\phi_j)
  = \sum_{j=1}^n \Phi(f(x_j) \phi_j)
  = \Phi\Big( \sum_{j=1}^n f(x_j) \phi_j \Big)
  = \Phi(f),
  \]
  that is, \(\Phi = \sum_{j=1}^n \eval_{x_j} \cdot \eval_{\phi_j}^{*}
  (\Phi)\). Therefore \((\eval_{x_j})_{j=1}^n\) is a generating set for
  \(P^{* *}\), and \((\eval_{\phi_j}^{*}, \eval_{x_j})_{j=1}^n\) is a dual
  ``basis'' for \(P^{* *}\). Therefore \(P^{* *}\) is a finitely generated
  projective right-\(R\)-module.

\item Given an element \(\Phi \in P^{* *}\), we can rewrite it as
  \(\Phi = \sum_{j=1}^n \eval_{x_j} \cdot \eval_{\phi_j}^{*}(\Phi)\), therefore,
  if we take \(\sum_{j=1}^n x_j \Phi(\phi_j) \in P\), one obtains
  \begin{align*}
  \eval\Big(\sum_{j=1}^n x_j \Phi(\phi_j)\Big)
  &= \sum_{j=1}^n \eval(x_j \Phi(\phi_j))
  = \sum_{j=1}^n \eval(x_j) \Phi(\phi_j) \\
  &= \sum_{j=1}^n \eval_{x_j} \cdot \Phi(\phi_j)
  = \sum_{j=1}^n \eval_{x_j} \cdot \eval_{\phi_j}^{*}(\Phi) \\
  &= \Phi.
  \end{align*}
  This shows that \(\eval\) is also a surjective morphism. Therefore \(\eval\)
  is an isomorphism of right-\(R\)-modules
  \[
  P \iso P^{* *}.
  \]
\end{enumerate}
\end{proof}

%%% Local Variables:
%%% mode: latex
%%% TeX-master: "../../deep-dive"
%%% End:

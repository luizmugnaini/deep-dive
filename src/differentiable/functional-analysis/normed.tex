\begin{remark}
\label{rem:k-is-C-or-R}
Throughout this whole chapter we shall denote by \(k\) a field that is either
\(\CC\) or \(\R\) --- we'll adopt the use of \(|-|: k \to \R_{\geq 0}\) for the
standard norm of the underlying field in order to distinguish that from the norm
of the vector spaces.
\end{remark}

\section{Norms on Spaces}

\begin{definition}[(Pre)Norm and (pre)normed vector spaces]
\label{def:prenorm-and-norm-normed-vect-space}
Let \(E\) be a \(k\)-vector space. We say that a map \(\|-\|: E \to
\R_{\geq 0}\) is a \emph{pre-norm} in \(E\) if for all \(x, y \in E\) and
\(\lambda \in k\) the map satisfies the following properties
\begin{enumerate}[(a)]\setlength\itemsep{0em}
\item Product by scalar: \(\|\lambda x\| = |\lambda|\, \|x\|\).
\item Triangle inequality: \(\| x + y \| \leq \| x \| + \| y \|\).
\end{enumerate}
Moreover, if \(\| - \|\) satisfies the following additional condition, it is
called a \emph{norm}.
\begin{enumerate}[(a)]\setlength\itemsep{0em}\setcounter{enumi}{2}
\item If \(\| x \| = 0\) then \(x = 0\).
\end{enumerate}
The vector space \(E\) endowed with the (pre)norm \(\| - \|\) is called a
\emph{(pre)normed vector space}.
\end{definition}

It should be noted immediately that the first condition for a pre-norm implies in
\(\| 0 \| = 0\), hence the last condition for \(\| - \|\) to be a norm can be
substituted equivalently by ``\(\| x \| = 0\) if and only if \(x =
0\)''. Moreover, in a prenormed space, it's not possible to assert that a
sequence has a unique limit, property which is only ensured by the last
condition.

A simple property that can be extracted from
\cref{def:prenorm-and-norm-normed-vect-space}, is that any subspace of a
(pre)normed \(k\)-vector space is itself (pre)normed with the naturally
inherited norm.

\begin{example}[Norms from maps]
\label{exp:norm-from-functional}
Let \(E\) be a \(k\)-vector space and let's consider any functional \(f \in
E^{*}\). We can build a \emph{pre-norm} from the linear map \(f\) by defining a
map \(\| - \|_f: E \to \R_{\geq 0}\) given by \(\| x \|_f \coloneq |f(x)|\) ---
which clearly satisfies all of the required conditions for a norm. Note however
that for \(\| - \|_f\) to satisfy the last condition, it must be true that
\(\ker f = 0\), that is, \(\dim_{k} E = 1\) necessarily for \(\| - \|_f\) to be
a \emph{norm}.
\end{example}

\begin{example}[Metrics from norms]
\label{exp:metric-from-norm}
Given a (pre)normed \(k\)-vector space \((E, \| - \|)\), we can naturally
define a metric in \(E\) to be a map \(d: E \times E \to \R_{\geq 0}\) given by
\(d(x, y) \coloneq \| x - y \|\) --- which makes \(E\) into a (pre)metric
space (refer to \cref{def:premetric-and-metric-space}). We now verify each of
the conditions for a pre-metric:
\begin{enumerate}[(a)]\setlength\itemsep{0em}
\item If \(x, y \in E\) are any elements, then
\[
  d(x, y) = \| x - y \| = |-1|\, \| x - y \| = \| y - x \| = d(y, x).
\]
\item If \(z \in E\) is another element, then
\[
  d(x, z) = \| x - z \| \leq \| x - y \| + \| y - z \| = d(x, y) + d(y, z).
\]
\end{enumerate}
Moreover, if \(\| - \|\) is a norm, then we are also able to satisfy:
\begin{enumerate}[(a)]\setlength\itemsep{0em}\setcounter{enumi}{2}
\item Given \(x, y \in E\) such that \(d(x, y) = 0\), then \(\| x - y \| = 0\)
and since \(\| - \|\) is a norm, we obtain \(x - y = 0\), thus \(x = y\) as
wanted.
\end{enumerate}
\end{example}

\begin{proposition}
\label{prop:hausdorff-space-norm}
Let \((E, \| - \|)\) be a (pre)normed \(k\)-vector space, and \(d: E \times E
\to \R_{\geq 0}\) be the (pre)metric induced by \(\| - \|\). Then, \(E\) is a
\emph{Hausdorff} space if and only if \(\| - \|\) is a \emph{norm}.
\end{proposition}

\begin{proof}
Suppose \(E\) is a Hausdorff space and let \(x, y \in E\) be such that \(\| x -
y \| = 0\), then in particular \(d(x, y) = 0\) and therefore any open ball
\(B_x(r)\) centred at \(x\), for \(r > 0\), does also contain \(y\) --- but
since \(E\) is Hausdorff, this can only be true if \(x = y\), thus \(x - y = 0\)
and we obtain that \(\| - \|\) is a norm, and \(d\) is a metric.

Now, assume that \(\| - \|\) is a norm, thus \(d\) is a metric and since any
metric space is Hausdorff (see \cref{prop: metric space T2}), in particular,
\(E\) is Hausdorff.
\end{proof}

\begin{lemma}
\label{lem:norm-is-continuous}
Let \((E, \| - \|)\) be a normed \(k\)-vector space, then the following is true
for any given \(x, y, z \in E\):
\begin{enumerate}[(a)]\setlength\itemsep{0em}
\item If \(d: E \times E \to \R_{\geq 0}\) is the metric induced by \(\| - \|\),
then \(d(x + y, y + z) = d(x, z)\).

\item The inequality \(\big| \| x \| - \| y \| \big| \leq \| x - y \|\) holds.
\end{enumerate}
\end{lemma}

\begin{proof}
\begin{enumerate}[(a)]\setlength\itemsep{0em}
\item \(d(x + y, y + z) = \| (x + y) - (y + z)\| = \| x - z \| = d(x, z)\).

\item Notice that \(\| x \| = \| (x - y) + y \| \leq \| x - y \| + \| y \|\),
therefore \(\| x \| - \| y \| \leq \| x - y \|\). Moreover, symmetrically we
have that \(\| y \| - \| x \| \leq \| y - x \| = \| x - y \|\), therefore,
\[
- \| x - y \| \leq \| x \| - \| y \| \leq \| x - y \|
\]
as wanted.
\end{enumerate}
\end{proof}

\begin{proposition}
\label{prop:norm-is-continuous}
Any norm in an \(\R\)-vector space is \emph{continuous}.
\end{proposition}

\begin{proof}
This follows directly from the inequality obtained in item (b) of
\cref{lem:norm-is-continuous}.
\end{proof}

\begin{proposition}
\label{prop:continuity-sum-multiplication-scalar}
Given a (pre)normed \(k\)-vector space \(E\), if we endow the products \(E
\times E\) and \(k \times E\) with the product topology, \emph{addition} of
vectors and \emph{multiplication} of vectors by scalars are both
\emph{continuous} maps.
\end{proposition}

\begin{proof}
Given sequences \(x_j \to x\) and \(y_j \to y\) for any \(x, y \in E\), then
\(x_j + y_j \to x + y\), let \(\varepsilon > 0\) be any bound, then choose a
common \(n \in \N\) for which \(\| x - x_j \| < \frac{\varepsilon}{2}\) and
\(\| y - y_{j} \| < \frac{\varepsilon}{2}\) for all \(j \geq n\). Therefore, by
the triangle inequality
\[
\| (x + y) - (x_j - y_j)\| \leq  \| x - x_j \| - \| y - y_j \| < \varepsilon
\text{ for all } j > n,
\]
where we thus conclude that \(x_j + y_j \to x + y\) --- all limits are well
defined since \(E\) is Hausdorff.

Let \(\lambda_j \to \lambda\) be a convergent sequence in \(k\) for any given
\(\lambda \in k\). Since convergent sequences are bounded, let \(M \in k\) be a
bound for \((x_j)_{j \in \N}\), that is, \(|x_j| < M\) --- also,
define \(M' \coloneq \max (M, |\lambda|)\). Given any \(\varepsilon > 0\) we
choose a common \(n \in \N\) for which \(|\lambda - \lambda_j| <
\frac{\varepsilon}{2 M'}\) and \(\| x - x_{j} \| < \frac{\varepsilon}{2 M'}\)
for all \(j \geq n\). With this we can use the triangle inequality to find
\begin{align*}
  \| \lambda x - \lambda_j x_j \|
  &= \| (\lambda - \lambda_j) x_j + \lambda (x - x_j)\| \\
  &\leq \| (\lambda - \lambda_{j}) x_j \| + \| \lambda (x - x_j) \| \\
  &= |\lambda - \lambda_j|\, \| x_j \| + |\lambda|\, \| x - x_j \| \\
  &= \frac{\varepsilon}{2 M'} \| x_{j} \| + |\lambda| \frac{\varepsilon}{2 M'}
  \\
  &< \frac{\varepsilon}{2 M'} M' + M' \frac{\varepsilon}{2 M'} \\
  &= \varepsilon,
\end{align*}
that is, \(\lambda_j x_j \to \lambda x\) as wanted.
\end{proof}

\begin{definition}[Equivalence of norms]
\label{def:equivalence-of-norms}
Given vector space \(E\) and two norms, \(\| - \|_1\) and \(\| - \|_2\), on
\(E\), we say that such norms are equivalent if there exists \(a, b > 0\) such
that for all \(v \in E\) we have the inequalities
\[
  a \| v \|_1 \leq \| v \|_2 \leq b \| v \|_1.
\]
\end{definition}

\begin{proposition}
\label{prop:equivalent-norms-same-topology}
Two norms are equivalent if and only if they induce the same topology.
\end{proposition}

\begin{proof}
Let \(E\) be a \(k\)-vector space and both \(\| - \|_1\) and \(\| - \|_2\) be
norms on \(E\). Let \(U \subseteq (E, \| - \|_1)\) be open and, for any given
\(x \in U\), let \(B_x(r) \subseteq (E, \| - \|_1)\) be an open ball centred at
\(x\). Notice that if \(b > 0\) is the right constant in
\cref{def:equivalence-of-norms}, then by choosing the open ball \(B_x(r/b)
\subseteq (E, \| - \|_2)\), we are able to obtain that for all \(y \in
B_x(r/b)\) we have
\[
\| x - y \|_1 \leq b \| x - y \|_2 < r
\]
thus \(y \in B_x(r)\) in \(\| - \|_1\) --- this implies in \(B_x(c r) \subseteq
B_x(r)\) in \(\| - \|_2\) and therefore we conclude that \(B_x(r)\) is open in
\((E, \| - \|_2)\). Notice that the proof can be mirrored for the other case,
thus we are done.
\end{proof}

\subsection{Quotient Space}

\begin{example}[Pre-norm on quotients of normed spaces]
\label{exp:prenorm-quotient-space}
Let \((E, \| - \|)\) be a \emph{pre-normed} \(k\)-vector space and \(F
\subseteq E\) a subspace. We can define a \emph{pre-norm} \(\| - \|_{\sim}: E/F
\to \R_{\geq 0}\) on the quotient space \(E/F\) as
\[
  \| [v] \|_{\sim} \coloneq \inf_{u \in [v]} \| u \|,
\]
that is, the infimum of the norm of the representatives of the class. This
indeed defines a pre-norm since, if \([v], [w] \in E/F\), then
\begin{align}\label{eq:quotient-norm-triang-ineq}
  \nonumber
  \| [v] + [w] \|_{\sim}
  &= \inf \{\|x + y\| \colon x + y \in [v] + [w]\} \\
  &\leq \inf \{\| x \| + \| y \| \colon x \in [v] \text{ and } y \in [w]\}
\end{align}
We claim that \(\inf \{\| x \| + \| y \|\} \leq \inf \| x \| + \inf \| y
\|\). To prove that, assume that both \(\inf \| x \|\) and \(\inf \| y \|\) are
finite --- otherwise, if one of them is infinite, the inequality is trivially
true. For any \(\varepsilon > 0\) we can find \(x' \in [v]\) and \(y' \in
[w]\) for which
\begin{gather*}
\inf_{x \in [v]} \| x \|
\leq \| x' \|
\leq \inf_{x \in [v]} \| x \| + \varepsilon,
\\
\inf_{y \in [v]} \| y \|
\leq \| y' \|
\leq \inf_{y \in [v]} \| y \| + \varepsilon.
\end{gather*}
Therefore we find that \(\inf \{\| x \| + \| y \|\} \leq \| x' \| + \| y' \|\)
therefore \(\inf \{\| x \| + \| y \|\} \leq \inf \| x \| + \inf \| y \| + 2
\varepsilon\), but since \(\varepsilon\) may be indefinitely little, we find
that indeed \(\inf \{\| x \| + \| y \|\} \leq \inf \| x \| + \inf \| y
\|\). Hence, we can turn to \cref{eq:quotient-norm-triang-ineq} and conclude
that
\begin{align*}
\| [v] + [w] \|_{\sim}
&\leq \inf \{\| x \| + \| y \| \colon x \in [v] \text{ and } y \in [w]\} \\
&\leq \inf_{x \in [v]} \| x \| + \inf_{y \in [w]} \| y \| \\
&= \| [v] \|_{\sim} + \| [w] \|_{\sim},
\end{align*}
thus satisfying the triangle inequality. The condition that \(\| \lambda v
\|_{\sim} = |\lambda|\, \| v \|_{\sim}\) is trivially obtained --- hence \(\| -
\|_{\sim}\) is a pre-norm in \(E/F\).
\end{example}

\begin{proposition}[Quotient norm]
\label{prop:quotient-norm}
The pre-norm \(\| - \|_{\sim}\) described in \cref{exp:prenorm-quotient-space} is
a \emph{norm} if and only if the subspace \(F\) is closed in \(E\).
\end{proposition}

\begin{proof}
Let \([v] \in E/F\) be any class such that \(\| [v] \|_{\sim} = 0\), therefore,
we must be able to find a sequence \((v_j)_{j \in \N}\) of elements \(v_j \in
[v]\) such that \(\| v_{j} \| \to 0\) as \(j \to \infty\). Moreover, choosing
\(v \in [v]\) to be any representative, since \(v_j - v \in F\) and \(F\) is
closed normed space, the convergence of the norm to zero implies that \(v_j \to
0\) for \(j \to \infty\) in \(F\) --- therefore, we conclude that \(F \ni v_j -
v \to -v\) and thus \(-v \in F\) from the closeness property, in particular, we
find that \(v \in F\).

We now claim that since \(v \in F\), then \([v] = [0]\). Let \(u \in [v]\) be
any representative, then \(u - v \in F\) and since \(v \in F\) by assumption,
then in particular \(u \in F\), hence \(u \in [0]\) --- that is, \([v] \subseteq
[0]\). Now, if \(w \in [0]\), from definition we obtain \(w \in F\), but since
\(v \in F\) then in particular \(w - v \in F\) hence \(w \in [v]\) --- which
implies that \([0] \subseteq [v]\) and thus \([v] = [0]\) for all \(v \in
F\). Therefore, from this claim we obtain that \(\| [v] \|_{\sim} = 0\) implies
\([v] = [0] = F\), the zero element of \(E/F\).
\end{proof}

\begin{proposition}[Norm out of pre-norm]
\label{prop:norm-out-of-prenorm}
Let \((E, \| - \|)\) be a \emph{pre-normed} \(k\)-vector space and \(E_0 \coloneq
\{x \in E \colon \| x \| = 0\}\). Then, the map \(\| - \|_{\sim}: E / E_0 \to
\R_{\geq 0}\) defined by \(\| [x] \|_{\sim} \coloneq \| x \|\) is well defined
and is a \emph{norm} for the space \(E / E_0\).
\end{proposition}

\begin{proof}
Let \([x] \in E / E_0\) be a class such that \(\| [x] \|_{\sim} = 0\). Choose
any representative \(x \in [x]\) and notice that since \(\| x \| = 0\) then \(x
\in E_0\), that is, \([x] \subseteq E_0\) --- moreover, if \(y \in E_0\), then
surely \(y \in [x]\), that is, \(E_0 \subseteq [x]\). This shows that \([x] =
E_0\), where \(E_0 = [0] \in E / E_0\).
\end{proof}

\subsection{Examples of Normed Spaces}

The following is an immediate proposition, so we won't bother to write down
the proof.

\begin{lemma}[Complex conjugate space norm]
\label{lem:prenorm-complex-conjugate-space}
Let \((E, \| - \|)\) be a pre-normed \(\CC\)-vector space. Then the naturally
induced map \(\| - \|: \overline{E} \to \R_{\geq 0}\) is a pre-norm for the
complex conjugate space \(\overline{E}\).
\end{lemma}

\begin{proposition}
Let \(E\) be a pre-normed \(k\)-vector space, \(F \subseteq E\) be a closed
subspace, and \(x \in E \setminus F\). Then there exists a scalar \(C > 0\)
for which, for every given scalar \(\lambda \in k\) and vector \(y \in F\), we
have
\[
|\lambda| \leq C \| \lambda x + y \|.
\]
\end{proposition}

\begin{proof}
If \(\lambda = 0\) then the proposition follows trivially. Otherwise, let
\(\lambda \neq 0\) and notice that since \(F\) is closed, there must exist
\(\theta > 0\) for which \(\| x - y \| \geq \theta\) for every given \(y \in F\)
--- since \(x\) lies outside of \(F\). In particular, since \(-\frac{1}{\lambda}
y \in F\), then \(\| x - (-\frac{1}{\lambda} y) \| \geq \theta\) --- notice that
such choice of vector was not made arbitrarily since
\[
\| \lambda x + y \|
= |\lambda|\, \| x - (- y / \lambda) \| \geq |\lambda| \theta,
\]
therefore, if we choose \(C \coloneq \frac{1}{\theta}\), we obtain the desired
inequality.
\end{proof}

\begin{example}[\(p\)-norms]
\label{exp:p-norms}
The following are recurrent norms on two of the most relevant spaces to our
analytical study of normed \(k\)-vector spaces:
\begin{enumerate}[(a)]\setlength\itemsep{0em}
\item For every integer \(1 \leq p < \infty\) we define a norm \(\| - \|_p: k^n
\to \R_{\geq 0}\) defined by, for all \(x \in k^n\),
\[
\| x \|_p \coloneq \bigg( \sum_{j=1}^n |x_j|^p \bigg)^{1/p}.
\]

\item The infinite case for the space \(k^n\) is defined by a map
\(\| - \|_{\infty}: k^n \to \R_{\geq 0}\) given by
\[
\| x \|_{\infty} \coloneq \max_{1 \leq j \leq n} |x_j|.
\]

\item Let \(\ell^p(J) \subseteq k^J\) be the \(k\)-vector subspace consisting of
maps \(f: J \to k\) with \emph{countable support} --- that is, \(\sum_{t \in J}
|f(t)|^p < \infty\). For each integer \(1 \leq p < \infty\), we define a norm
\(\| - \|_p: \ell^p(J) \to \R_{\geq 0}\) given by
\[
\| f \|_p \coloneq \bigg( \sum_{t \in J} |f(t)|^p \bigg)^{1/p}.
\]

\item For the infinite case, we define \(\ell^{\infty}(J) \subseteq k^{J}\) to
be the \(k\)-vector subspace consisting of maps \(f: J \to k\) such that
\(\sup_{t \in J} |f(t)| < \infty\). We define the norm \(\| - \|_{\infty}:
\ell^{\infty}(J) \to \R_{\geq 0}\) by
\[
\| f \|_{\infty} \coloneq \sup_{t \in J} |f(t)|.
\]
\end{enumerate}
In fact, each one of the above spaces is Banach, but we'll prove this later.
We need now to prove that these are indeed normed vector spaces.
\end{example}

\begin{proof}
\begin{enumerate}[(a)]\setlength\itemsep{0em}
\item Let \(x \in k^n\) be any vector and \(\lambda \in k\) any scalar, then
\[
\| \lambda x \|_p
= \bigg( \sum_j | \lambda x_j |^p \bigg)^{1/p}
= \bigg( |\lambda|^p \sum_j | x_j |^p \bigg)^{1/p}
= |\lambda|\, \| x \|_p.
\]
Moreover, if \(y \in k^n\) is another vector, then since \(p \geq 1\) we may
just use Minkowski's inequalities (see \cref{prop: minkowski-ineq}) in order to
obtain the triangle inequality --- it should be noted that \(p < 1\) does not
yield a valid triangle inequality, hence justifying the restriction \(p \geq
1\). Also, if \(\| z \| = 0\) for some \(z \in k^n\), then \(z_j = 0\) for all
\(1 \leq j \leq n\) --- since \(|z_j| = 0\) --- and thus \(z = 0\).

\item The map \(\| - \|_{\infty}: k^n \to \R_{\geq 0}\) is clearly a norm.

\item

\item Let \(\lambda \in k\) and \(f, g \in \ell^{\infty}(J)\) be any
elements. Since \(\sup_t |f(t)| < \infty\), we conclude that \(\| \lambda f \| =
\sup_t |\lambda f(t)| = \sup_t |\lambda|\, |f(t)| = |\lambda| \sup_t |f(t)|\),
thus \(\| \lambda f \| = |\lambda|\, \| f \|\). Moreover, we have \(\sup_t |f(t)
+ g(t)| \leq \sup_t (|f(t)| + |g(t)|) \leq \sup_t |f(t)| + \sup_t |g(t)| <
\infty\), therefore \(\| f + g \| \leq \| f \| + \| g \|\). Also, if \(h \in
\ell^{\infty}(J)\) is a map such that \(\| h \| = 0\), then \(\sup_t |h(t)| =
0\), which by the definition of the supremum of a set and since \(|h(t)| \geq
0\), we conclude that \(h(t) = 0\) for every \(t \in J\) --- that is, \(h = 0\).
\end{enumerate}
\todo[inline]{Prove that the maps define norms for \(\ell^p\) and
\(\ell^{\infty}\), moreover, prove that each of the mentioned spaces is
Banach.}
\end{proof}

\begin{lemma}
\label{lem:p-norms-equivalent-finite-case}
Let \(\| - \|_p, \| - \|_{\infty}: k^n \rightrightarrows \R_{\geq 0}\) be the
norms defined in \cref{exp:p-norms}, then the following inequality holds for all
\(1 \leq p < \infty\) and all \(x \in k^n\):
\[
\| x \|_{\infty} \leq \| x \|_p \leq n^{1/p} \| x \|_{\infty}.
\]
Therefore \(\| - \|_p\) and \(\| - \|_{\infty}\) are equivalent norms.
\end{lemma}

\begin{proof}
Since \((\sum_j |x_j|^p)^{1/p} \leq (\sum_j (\max_j |x_j|)^p)^{1/p} = n^{1/p}
\max_j |x_j|\), we obtain \(\| x \|_p \leq n^{1/p} \| x \|_{\infty}\). Moreover,
clearly \(\| x \|_{\infty} \leq \| x \|_p\), thus the proposition follows.
\end{proof}

\begin{example}[\(\ell^{\infty}\) space of maps]
\label{exp:ell-infty-space-maps}
Let \(X\) be any set and define \(\ell^{\infty}(X)\) to be a \(k\)-vector
subspace of \(k^X\), consisting of all bounded maps. We define a norm
\(\| - \|_{\infty}: \ell^{\infty}(X) \to \R_{\geq 0}\) to be the map given by
\[
\| f \|_{\infty} \coloneq \sup_{t \in X} |f(t)|.
\]
Such norm is called the \emph{uniform norm} on the function space.
\end{example}

\begin{example}[Uniform convergence norm]
\label{exp:uniform-convergence-norm}
Let \(X\) be a \emph{Hausdorff} space and \(\Omega \subseteq X\) be a
\emph{compact} set. If \(C(\Omega, k)\) denotes the \(k\)-vector space of
continuous functionals \(\Omega \to k\), the map \(\| - \|_{\infty}: C(\Omega,
k) \to \R_{\geq 0}\) given by
\[
\| f \|_{\infty} \coloneq \sup_{t \in \Omega} |f(t)| = \max_{t \in \Omega} |f(t)|
\]
defines a \emph{norm} in \(C(\Omega, k)\). Moreover, the normed \(k\)-vector
space \((C(\Omega, k), \| - \|_{\infty})\) is \emph{Banach}.
\end{example}

\begin{proof}
\todo[inline]{Prove}
\end{proof}

% \begin{definition}[Total subset]
% \label{def:total-subset}
% Given a normed \(k\)-vector space \(V\), a subset \(A \subseteq V\) is said to
% be \emph{total} if \(\Span(A)\) is dense in \(V\).
% \end{definition}

\section{Properties of Normed Vector Spaces}

\subsection{Finite Dimensional}

\begin{lemma}
\label{lem:finite-normed-scalars-inequality}
Let \(E\) be an \(n\)-dimensional normed \(k\)-vector space and \(\{x_1, \dots,
x_n\}\) be a basis of \(E\). There exists a scalar \(C > 0\) such that, for
every choice of scalars \(\{\lambda_1, \dots, \lambda_n\} \subseteq k\), we have
\[
\bigg\| \sum_{j=1}^n \lambda_j x_j \bigg\| \geq C \sum_{j=1}^n |\lambda_j|.
\]
\end{lemma}

\begin{proof}
For the sake of brevity, denote \(S \coloneq \sum_{j=1}^{n} |\lambda_j|\). If
\(S = 0\) then the lemma follows trivially. Suppose that \(S > 0\) and, for the
sake of contradiction, that there exists no scalar \(C > 0\) such that \(\|
\sum_j | \lambda_j x_j | \| \geq C \sum_j |\lambda_j|\) --- this implies that
for all integer \(p \geq 1\) there exists a point \(y_p \coloneq \sum_{j=1}^n
\lambda_j(p) x_j\) such that
\begin{equation}\label{eq:sum-coeff-inequality-normed-fin-dim}
\sum_{j=1}^n |\lambda_j(p)| \geq p \| y_p \|.
\end{equation}
Moreover, for each \(p \geq 1\), we may as well construct \(z_p \coloneq
\sum_{j=1}^n \alpha_j(p) x_j\) --- where \(\alpha_j: \N \to k\) for all \(1 \leq
j \leq n\) --- for which
\[
\alpha_j(p) \coloneq \frac{\lambda_j(p)}{\sum_{j=1}^n |\lambda_j(p)|}.
\]
This ensures us that \(\sum_j \alpha_j(p) = 1\) --- hence \(|\alpha_j(p)| \leq
1\). Dividing \cref{eq:sum-coeff-inequality-normed-fin-dim} by \(\sum_j
|\lambda_j(p)|\) we find
\begin{equation}\label{eq:yp-converges-zero-inequality-normed-fin-dim}
\frac 1 p \geq \| y_p \|.
\end{equation}

For every fixed \(1 \leq j \leq n\), the sequence \((\alpha_j(p))_p\) is bounded
and since we are working either with the complex or real numbers as the
underlying field, we can conclude from Bolzano-Weirstra{\ss} theorem (see
\cref{thm:bolzano-weierstrass}) that from the sequence of bounded scalars one
can extract a convergent subsequence \((\alpha_j(p'))_{p'}\) --- assume that
\(\alpha_j(p') \to \beta_j\) for some \(\beta_j \in k\). This induces a
subsequence of \((y_p)_{p}\), given by \((y_{p'})_{p'}\), so that \(y_{p'} \to
\sum_j \beta_j x_j\) --- moreover, \(\sum_j |\beta_j| = 1\). However, from
\cref{eq:yp-converges-zero-inequality-normed-fin-dim} we find that \(y_p \to
0\), thus also \(y_{p'} \to 0\). Since \(E\) is Hausdorff, the limit of the
sequence \((y_{p'})_{p'}\) must be unique, hence \(\sum_j \beta_j x_j = 0\)
--- which contradicts the initial hypothesis that the set \(\{x_1, \dots,
x_n\}\) was linearly independent in \(E\).

This shows one cannot build a sequence \((y_p)_p\) for which
\cref{eq:sum-coeff-inequality-normed-fin-dim} is satisfied, hence the
proposition follows.
\end{proof}

\begin{corollary}
\label{cor:unit-ball}
The unit ball is compact in a finite dimensional normed vector space.
\end{corollary}

\begin{proof}
Let \(E\) be an \(n\)-dimensional \(k\)-vector space and \(\{e_1, \dots, e_n\}\)
be a basis for \(E\). Let \((x_p)_{p \in \N}\) be a sequence of points in the
unit ball \(B_0(1)\) (that is, \(\| x_{p} \| \leq 1\)) --- we'll define for
each \(p \in \N\) that \(x_p \coloneq \sum_{j=1}^{n} \alpha_j(p) e_j\), where
\(\alpha_j: \N \to k\) for all \(1 \leq j \leq n\). One concludes that, since
\(\| \alpha_j(p) \| \leq 1\) for each \(1 \leq j \leq n\), we can use the
Bolzano-Weierstra{\ss} theorem (see \cref{thm:bolzano-weierstrass}) to conclude
that we can extract a convergent subsequence \((\alpha_j(p'))_{p'}\) from
\((\alpha_j(p))_{p \in \N}\) --- for instance, assume that \(\alpha_j(p) \to
\beta_j\) for some \(\beta_j \in k\). Then the induced subsequence
\((x_{p'})_{p'}\) is such that \(x_{p'} \to \sum_j \beta_j e_j \coloneq
x\). Moreover, for each \(p'\) we have the inequality \(\| x \| \leq \| x -
x_{p'} \| + \| x_{p} \| \leq \| x - x_{p'} \| + 1\), thus as \(p' \to \infty\)
we have \(\| x - x_{p'} \| \to 0\) and therefore \(\| x \| \leq 1\) --- that
is, every sequence in \(B_0(1)\) has a convergent subsequence in \(B_0(1)\),
which by \cref{thm:metric-2d-ctbl-hausdorff-equiv-compactness} implies that
\(B_0(1)\) is compact in \(E\).
\end{proof}

\begin{lemma}
\label{lem:finite-dim-equivalent-norms}
If \(E\) is a \emph{finite} dimensional \(k\)-vector space, then \emph{all}
norms in \(E\) are \emph{equivalent}.
\end{lemma}

\begin{proof}
Assume that \(\{e_1, \dots, e_n\}\) is a basis for \(E\). We'll show that every
norm is equivalent to \(\| - \|: E \to \R_{\geq 0}\) given by \(\| \sum_{j=1}^n
\lambda_j e_j \| \coloneq \sum_{j=1}^n |\lambda_j|\). Let \(\| - \|': E \to
\R_{\geq 0}\) be any other norm --- then from the triangle inequality we have,
for any point \(x \coloneq \sum_j \lambda_j e_j\) in \(E\):
\[
\| x \|'
\leq \sum_{j=1}^n \| \lambda_j e_j \|
\leq \max_{1 \leq j \leq n} \| e_j \| \sum_{j=1}^n |\lambda_j|
= \max_{1 \leq j \leq n} \| e_j \|\, \| x \|.
\]
Defining \(C \coloneq \max_j \| e_j \|\), we have shown that \(\| x \|' \leq C
\| x \|\).

For the last part, we must prove the existence of a scalar \(B > 0\) such that
\(\| x \| \leq B \| x \|'\). We proceed by contradiction, that is, assuming we
can choose a sequence of points \((x_p)_{p \in \N}\) such that
\begin{equation}\label{eq:all-norms-equivalent-ineq}
\| x_p \| \geq p \| x_p \|'
\end{equation}
for all \(p \in \N\). Let \((y_p)_{p \in \N}\) be the sequence defined by \(y_p
\coloneq x_p/\| x_p \|\), so that \(\| y_p \| = 1\) --- thus, by
\cref{eq:all-norms-equivalent-ineq} we obtain the bound \(\| y_p \|' \leq 1/p\),
thus \(y_p \to 0\). If we let \(y_p \coloneq \sum_j \alpha_j(p) e_j\), we obtain
that \(\sum_j |\alpha_j(p)| = 1\) implies \(|\alpha_j(p)| \leq 1\) --- hence,
for every fixed \(1 \leq j \leq n\), the sequence \((\alpha_j(p))_{p \in \N}\)
contains a convergent subsequence \((\alpha_j(p'))_{p'}\) such that
\(\alpha_{j}(p') \to \beta_j\) for some \(\beta_j \in k\). Moreover, the limits
are such that \(\sum_j |\beta_j| = 1\). Therefore, the induced subsequence
\((y_{p'})_{p'}\) is such that \(y_{p'} \to \sum_j \beta_j e_j\) --- however,
since \(E\) is Hausdorff, the sequence must have a unique limit, hence \(\sum_j
\beta_j e_j = 0\), which is only possible if \(\beta_j = 0\) for all \(1 \leq j
\leq n\) --- this contradicts the fact that \(\sum_j |\beta_j| = 1\). We
conclude that a sequence \((x_p)_{p \in \N}\) satisfying
\cref{eq:all-norms-equivalent-ineq} must not exist, thus implying that there
exists \(B > 0\) such that \(\| x \| \leq B \| x \|'\) for all \(x \in E\).
\end{proof}

\begin{corollary}
\label{cor:finite-dimensional-is-banach}
Every \emph{finite} dimensional normed \(k\)-vector space is Banach.
\end{corollary}

\begin{proof}
We'll work on a \(n\)-dimensional \(k\)-vector space \((E, \| - \|)\) --- where
we let \(\{e_1, \dots, e_n\}\) be any basis and \(\| - \|'\) be the norm \(\|
\sum_j \lambda_j e_j \|' \coloneq \sum_j |\lambda_j|\). Let \((x_p)_{p \in \N}\)
be any Cauchy sequence with respect to \(\| - \|\), and define \(x_p \coloneq
\sum_j \lambda_j(p) e_j\) for some map \(\lambda_j: \N \to k\). From
\cref{lem:finite-dim-equivalent-norms} we find that there exists \(C > 0\) for
which \(\| x \|' \leq C \| x \|\) for every \(x \in E\) --- therefore \((x_p)_{p
\in \N}\) is also Cauchy with respect to \(\| - \|'\). Hence, for all
\(\varepsilon > 0\) there exists \(N \in \N\) such that for all \(p, q \geq N\)
we have \(\| x_p - x_q \|' = \sum_j |\lambda_j(p) - \lambda_j(q)| <
C \varepsilon\) --- therefore for every \(1 \leq j \leq n\) we have
\(|\lambda_j(p) - \lambda_j(q)| < C \varepsilon\), which implies that
\((\lambda_j(p))_{p \in \N}\) is Cauchy in \(k\). Let for instance
\(\lambda_j(p) \to \alpha_j\) for some \(\alpha_j \in k\). Moreover, notice that
\(\| x_p - \sum_j \alpha_j e_j\|' = \sum_j |\lambda_j(p) - \alpha_j|\), which
converges to zero as \(p \to \infty\) --- thus \(x_p \to \sum_j \alpha_j e_j\),
which proves that \((x_p)_{p \in \N}\) converges in \(E\).
\end{proof}

\begin{proposition}[Compact sets]
\label{prop:compact-on-normed-space}
If \(E\) is a \emph{finite} dimensional normed \(k\)-vector space, a subset
\(\Omega \subseteq E\) is \emph{compact} if and only if \(\Omega\) is
\emph{bounded and closed}.
\end{proposition}

\begin{proof}
The first part comes from
\cref{prop:metric-space-compact-implies-bounded-closed}. For the second, let
\(\Omega\) be bounded and closed. If \(\dim E = n\), we let
\(\{e_1, \dots, e_n\}\) be a basis for \(E\). Let \((x_p)_{p \in \N}\) be any
sequence of points in \(\Omega\) and assume that each \(x_p\) has the form
\(x_j\coloneq \sum_{i=1}^n \lambda_i(p) e_i\) for scalars
\(\lambda_i(p) \in k\). From the boundness of \(\Omega\) one can find \(M > 0\)
such that \(\| x_p \| \leq C\) for every index \(p \in \N\). Evoking
\cref{lem:finite-normed-scalars-inequality} for each index \(p \in \N\), we find
that
\[
M \geq \| x_p \| \geq C \sum_{i=1}^n |\lambda_i(p)|,
\]
for some \(C > 0\). Fixing any \(1 \leq i \leq n\), one concludes that
\(|\lambda_i(p)| \leq M\) and therefore the sequence \((\lambda_i(p))_{p \in
\N}\) is bounded. Using Bolzano-Weierstra{\ss} theorem (see
\cref{thm:bolzano-weierstrass}) there exists a convergent subsequence
\((\lambda_i(p'))_{p'}\) --- assume for instance that \(\lambda_i(p') \to
\lambda_i\) for some \(\lambda_i \in k\). Such subsequence induces another
subsequence \((x_{p'})_{p'}\), for which \(x_{p'} \to \sum_i \lambda_i e_i
\coloneq x\) --- but since \(\Omega\) is closed, we find \(x \in \Omega\). Thus
any sequence in \(\Omega\) has a convergent subsequence in \(\Omega\) --- this
shows that \(\Omega\) is compact (see
\cref{thm:metric-2d-ctbl-hausdorff-equiv-compactness}).
\end{proof}

\subsection{%
  \texorpdfstring{\(\ell^{\infty}(\N)\)}{l infinity} and
  \texorpdfstring{\(\ell^p(\N)\)}{lp} are Banach Spaces
}

\begin{lemma}
\label{lem:cauchy-bounded}
Every Cauchy sequence in a normed \(k\)-vector space is bounded.
\end{lemma}

\begin{proof}
Let \((x_j)_{j \in \N}\) be a Cauchy sequence in a normed \(k\)-vector space
\(E\). If \(\varepsilon = 1\), then there exists \(N \in \N\) for which \(i, j
\geq N\) implies \(\| x_j - x_i \| < 1\) and therefore \(\| x_j \| \leq \| x_j -
x_N \| + \| x_N \| < 1 + \| x_N \|\). We conclude that the sequence is bounded:
\[
\| x_j \| \leq 1 + \max_{0 \leq i \leq N} \| x_i \|.
\]
\end{proof}

\begin{proposition}
\label{prop:ell-infty-is-banach}
The space \(\ell^{\infty}(\N)\) is Banach.
\end{proposition}

\begin{proof}
Let \(x \coloneq (x^p)_{p \in \N}\) denote any Cauchy sequence of points \(x^p
\in \ell^{\infty}(\N)\) --- that is, \(x^p \coloneq (x_j(p))_{j \in \N}\) is
itself a sequence --- therefore for all \(\varepsilon > 0\) there exists \(N \in
\N\) such that for all \(p, q \geq N\) we have
\begin{equation}\label{eq:ell-infty-is-banach}
\| x^p - x^q \| = \sup_{j \in \N} |x_j(p) - x_j(q)| < \varepsilon
\end{equation}

We first construct a candidate for the limit of \(x\). Let \(j_0 \in \N\) be any
fixed index --- we'll show that the sequence \((x_{j_0}(p))_{p \in \N}\) forms a
Cauchy sequence in \(k\). Notice that from \cref{eq:ell-infty-is-banach} it is
clear that \(\sup_{j \in \N} |x_j(p) - x_j(q)| < \varepsilon\) implies in
\(|x_{j_0}(p) - x_{j_0}(q)| < \varepsilon\) for all \(p, q > N\) --- therefore
\((x_{j_0}(p))_{p \in \N}\) is indeed Cauchy, and since \(k\) (either \(\R\) or
\(\CC\)) is complete, there exists \(y_{j_0} \in k\) for which \(x_{j_0}(p) \to
y_{j_0}\). Our candidate sequence will thus be formed by \(y \coloneq (y_j)_{j
\in \N}\) --- where each \(y_j\) is constructed just as above. Indeed, \(y \in
\ell^{\infty}(\N)\), since for every \(j \in \N\) we have \(|y_j| = \lim_{p \to
\infty} |x_j(p)| \leq \lim_{p \to \infty} \| x^p \|_{\infty}\) and since
\(x\) is Cauchy, from \cref{lem:cauchy-bounded}, we find that there exists \(C >
0\) such that \(|y_j| \leq \| x^{p} \|_{\infty} < C\) --- hence \(\| y
\|_{\infty} = \sup_{j \in \N} |y_j| < \infty\).

We now show that \((y_j)_{j \in \N}\) is the limit of \((x^p)_{p \in \N}\). Let
\(\varepsilon > 0\) be any bound and \(N \in \N\) be such that \(p, q > N\)
implies \(\| x^p - x^q \|_{\infty} < \varepsilon\) --- that is, for any fixed
\(j_0 \in \N\) we have \(|x_{j_0}^p - x_{j_0}^q| < \varepsilon\). Moreover, if
we let \(q \to \infty\) we'll find that \(|x_{j_0}^p - x_{j_0}^q| \to |x_{j_0}^p
- y_{j_0}| < \varepsilon\) --- hence, since this must be true for any \(j_0 \in
\N\), we obtain that \(\| x \|_{\infty} = \sup_{j \in \N} |x_j^q - y_j| <
\varepsilon\). This proves that \(x_j(p) \to y_j\) as \(p \to \infty\) and
therefore \(x^p \to y\) as wanted. Thus any Cauchy sequence converges and the
limit is given by the sequence of the limit of the components.
\end{proof}

\begin{proposition}
\label{prop:ell-p-is-banach}
The space \(\ell^p(\N)\) is Banach for all \(1 \leq p < \infty\).
\end{proposition}

\todo[inline]{prove}

\subsection{Common Properties Disregarding the Dimension of the Space}

\begin{proposition}
\label{prop:continous-linear-thus-bounded}
Let \(f: X \to Y\) be an \(\R\)-linear map between normed \(\R\)-vector spaces
\(X\) and \(Y\). Then, \(f\) is contiguous if and only if there exists a scalar
\(C > 0\), called \emph{bound}, for which \(\norm{f(x)}_Y \leq C \norm{x}_{X}\)
for all \(x \in X\).
\end{proposition}

\begin{proof}
If \(f\) is bounded by \(C\), let \(B\) be a basis for \(X\) and consider any
element \(x \coloneq \sum_{v \in B} a_{v} v\). From linearity we have \(f(x) =
\sum_{v \in B} a_v f(v)\), thus
\[
  \| f(x) \|_Y = \bigg\| \sum_{v \in B} a_v f(v) \bigg\|_{Y}
  \leq \sum_{v \in B} \| a_v \|_{\R} \| f(v) \|_{Y}
  \leq C \|x\|_{X} \sum_{v \in B} \|a_v\|_{\R}.
\]
This boils down to \(f = O(\Id_{X})\) --- which implies in \(f(x - x_0) = f(x) -
f(x_0) \to 0\) as \(x \to x_0\), where \(x_0 \in X\) is any point, that is,
\(f\) is continuous at any point of \(X\). Even better than that, we can show
that \(f\) is uniformly continuous (I won't carry it out since it's equivalent to
what we wrote in \cref{prop: linear-continuous}).

For the opposite, suppose \(f\) is continuous at \(0\), then there will surely
exist \(\delta > 0\) for which \(\|x\|_X < \delta\) implies in
\(\|f(x)\|_Y < 1\). Therefore, for any choice of non-zero \(x \in X\), we
find
\[
  \bigg\| f \left( \frac{\delta}{\norm{x}_X} x \right) \bigg\|_Y
  = \frac{\delta}{\| x \|_X} \| f(x) \|_{Y} < 1,
\]
therefore \(\| f(x) \| < \frac{\| x \|_{X}}{\delta}\), thus \(f\) is indeed
bounded.
\end{proof}

\begin{proposition}
\label{prop:composition-banach-morphisms}
Let \(E, F\) and \(G\) be normed vector spaces, and let \(u: E \to F\) and \(v:
F \to G\) be continuous linear maps. Then, \(v u: E \to G\) is also a continuous
linear map, and
\[
  \|v u\| \leq \|v\|\, \|u\|,
\]
where \(\|f\| \coloneq \sup_{x \in X} \|f(x)\|\) for a continuous linear
map \(f: X \to Y\) between normed vector spaces\footnote{Beware! This is not the
norm we shall adopt for our studies on banachable topological vector spaces. For
the latter, see \cref{def:norm-morphism-TopVect}}.
\end{proposition}

\begin{proof}
The first assertion is trivial, since the composition of continuous maps is
continuous, and the same is true for linear maps. Let \(x \in E\) be any element,
notice that
\[
  \|vu(x)\|_G
  \leq \|v\|\, \|u(x)\|_F
  \leq \|v\|\, \|u\|\, \|x\|_{E},
\]
thus the inequality holds.
\end{proof}

\begin{proposition}
\label{prop:multilinear-continous-iff-bounded}
A multilinear map \(\phi: \prod_{j=1}^n E_j \to F\) between normed vector spaces
\(E_1, \dots, E_n\), and \(F\), is continuous if and only if there exists a bound
\(C > 0\) such that, for every \(x \in \prod_{j = 1}^n E_j\),
\[
  \|\phi(x)\|_{F} \leq C \prod_{j=1}^n \|x_j\|_{E_j}.
\]
\end{proposition}

\begin{proposition}
\label{prop:canonical-iso-banach-multilinear}
Let \(E_1, \dots, E_r\), and \(F\) be normed vector spaces. There exists a
canonical map from repeated continuous linear maps to the continuous multilinear
maps, which is a continuous linear isomorphism, and is norm-preserving --- that
is, the canonical map
\[
  \Phi: L(E_1, L(E_2, \dots, L(E_r, F), \dots)) \isoto L^n(E_1, \dots, E_n; F)
\]
is a \emph{Banach isomorphism}.
\end{proposition}

\begin{proof}
We define \(\Phi\) by the following: if \(\lambda \in L(E_1, L(E_2, \dots,
L(E_n, F) \dots))\) is given by
\[
  \lambda(x_1) = \lambda_2,\ \text{ where }\ \lambda_2(x_2) = \lambda_3, \
  \dots, \ \lambda_n(x_n) = y \in F,
\]
we define \(\Phi(\lambda) \coloneq \overline{\lambda} \in L(E_1, \dots, E_n;
F)\) by the mapping
\[
  \overline\lambda(x_1, \dots, x_n) \coloneq \lambda(x_1)(x_2) \dots (x_n),
\]
where \(\lambda_j(x_j)(x_{j+1}) \dots (x_n) \coloneq \lambda_{j-1}(x_{j-1})
\dots (x_n)\) for every \(1 \leq j \leq n\) --- where \(\lambda_1 \coloneq
\lambda\).

Given \(\lambda \in L(E_1, L(E_2, \dots, L(E_n, F), \dots))\), the map
\(\overline\lambda\) is surely multilinear since each of the recursive arguments
are linear. Moreover, notice that, for any \(x \in \prod_{j=1}^n E_j\), we have
\[
  \|\overline\lambda(x)\|_{F}
  \leq \|\lambda(x_1) (x_2) \dots (x_n)\|_F
  \leq \|\lambda\| \prod_{j=1}^n \|x_{j}\|_{E_j},
\]
thus \(\|\overline\lambda\| \leq \|\lambda\|\).

On the other hand, given \(\overline\phi \in L(E_1, \dots, E_n; F)\), define the
map \(\phi = \Phi^{-1}(\overline\phi)\) by
\[
  \phi(x_1)(x_2) \dots (x_n) \coloneq \overline\phi(x_1, \dots, x_n).
\]
Therefore
\[
  \|\phi(x_1)(x_2) \dots (x_n)\|_{F}
  \leq \|\overline\phi\| \prod_{j=1}^n \|x_j\|_{E_j},
\]
which shows that \(\|\phi\| \leq \|\overline\phi\|\). We conclude that
\(\Phi(\lambda) = \overline\lambda\) for all repeating map \(\lambda\).
\end{proof}

\begin{theorem}[Hahn-Banach]
\label{thm:Hahn-Banach}
Let \(E\) be a normed \(\R\)-vector space, and \(F \subseteq E\) be a
subspace. Let \(\lambda \in F^{*}\) be a functional with bound \(C > 0\). Then
there exist an extension of \(\lambda\) to a functional on \(E\) with the same
bound \(C\) --- that is, a map \(\overline \lambda: E \to \R\) such that
\(\overline\lambda|_F = \lambda\) and \(\|\overline\lambda(x)\|_\R \leq C
\|x\|_{E}\) for all \(x \in E\).
\end{theorem}

\begin{corollary}[Hahn-Banach]
\label{cor:Hahn-Banach}
Let \(E\) be a Banach space and \(x \in E\) be a non-zero element. There exists
a continuous linear map \(\phi \in E^{*}\) such that \(\phi(x) \neq 0\).
\end{corollary}

\subsection{Properties of Banach Spaces}

\begin{definition}
\label{def:banach-isomorphism}
We define a \emph{Banach isomorphism} to be a continuous linear map \(u: E \to
F\), between Banach spaces \(E\) and \(F\), that is both invertible (there
exists a continuous linear map \(u^{-2}: F \to E\) that is the two-sided inverse
of \(u\)), and norm preserving --- that is, given any \(x \in E\), we have
\(\|u(x)\|_F = \|x\|_E\). Banach isomorphisms may also be referenced to
isometries in the literature.
\end{definition}

\begin{proposition}[Bijections are isomorphisms]
\label{prop:continuous-bijective-linear-is-isomorphism}
Every continuous bijective \(\R\)-linear map between topological vector spaces
is an isomorphism.
\end{proposition}

\begin{proposition}[Splitting]
\label{prop:banach-split}
Let \(E\) be a Banach space, and \(F\) and \(G\) be complementary closed
subspaces of \(E\) --- that is, \(E = F + G\) and \(F \cap G = 0\). Then the
morphism \(F \times G \to E\) given by \((f, g) \mapsto f + g\) is a continuous
linear isomorphism.
\end{proposition}

\section{Topological Vector Spaces}

\begin{definition}
\label{def:topological-vector-space}
A topological vector space is a \(k\)-vector space together with a topology
such that addition of vectors and the product by scalars are both continuous
\(k\)-linear maps.

We denote by \(\TVect_{\R}\) the category consisting of topological
\(\R\)-vector spaces together with morphisms, which are continuous \(\R\)-linear
maps (which may also be referenced to by the term ``top-linear'').

Let \(E\) be a topological vector space. The continuous \(\R\)-linear maps
corresponding to the dual space \(E^{*} = \Hom_{\TVect_{\R}}(E, \R)\), of a
topological \(\R\)-vector space \(E\), are called \(\R\) \emph{forms}. The
collection of forms of the form \(E \to \R\) will be conveniently separated in
classes and denoted:
\begin{itemize}\setlength\itemsep{0em}
\item \(L(E)\): the collection of continuous linear maps \(E \to \R\).
\item \(L^r(E)\): the collection of continuous \(r\)-multilinear maps \(E^r \to \R\).
\item \(L^r_{\Sym}(E)\): the collection of continuous \(r\)-multilinear
  symmetric maps \(E^r \to \R\).
\item \(L^r_{\Alt}(E)\): the collection of continuous \(r\)-multilinear
  alternating maps \(E^r \to \R\).
\end{itemize}
\end{definition}

\begin{definition}[Locally convex]
\label{def:locally-convex}
A topological vector space \(E\) is said to be locally convex if, for every open
set \(U \subseteq E\), any pair of points \(x, y \in U\) are such that \(t x +
(1 - t) y \in U\) for all \(t \in [0, 1]\).
\end{definition}

\begin{definition}[Banachable]
\label{def:banachable}
A topological \(\R\)-vector space \(E\) is said to be banachable if \(E\) is
complete and its topology can be defined by a norm.
\end{definition}

As a point of order, \emph{every time} we mention a topological \(\R\)-vector
space in the course of this chapter, we shall mean a \emph{banachable space}.

\begin{definition}[Norm of a morphism]
\label{def:norm-morphism-TopVect}
Let \(E\) and \(F\) be topological \(\R\)-vector spaces. In order to make
\(\Hom_{\TVect_{\R}}(E, F)\) into a topological \(\R\)-vector space, we can
construct a norm for which, given a morphism \(A: E \to F\), define \(K \coloneq
\{k \in \R \colon \norm{Ax}_{F} \leq k \norm{x}_E \text{, for all } x \in E\}\), the
norm of \(A\) is
\[
  \norm{A} \coloneq \sup_{k \in K} k.
\]

If \(\Hom_{\TVect_{\R}}(E_1, \dots, E_n; F)\) is the collection of continuous
\(\R\)-multilinear maps, then we define similarly the norm of a continuous
multilinear map \(B: \prod_{j=1}^n E_j \to F\) as
\[
  \norm{B} \coloneq \sup_{m \in M} m,
\]
where \(M \coloneq \{m \in \R \colon \norm{Bx}_F \leq m \prod_{j=1}^n
\norm{x_j}_{E_j} \text{, for all } x \in E\}\).
\end{definition}

\begin{remark}
\label{rm:Cp-morphism}
From now on, \emph{\(C^p\)-morphism} will refer to a map \(f: U \to V\) between
open subsets of Banach spaces such that \(f\) is a continuous map of class
\(C^p\), where \(p \leq \infty\).
\end{remark}

%%% Local Variables:
%%% mode: latex
%%% TeX-master: "../../../deep-dive"
%%% End:

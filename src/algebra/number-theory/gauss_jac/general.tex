\section{Generalized Jacobi Sum}

\begin{definition}[Generalized Jacobi sum]
    Let \(\chi_1, \dots, \chi_n\) be characters on \(\F_p\). We define the Jacobi
    sum of these characters as
    \[
        J(\chi_1, \dots, \chi_n) = \sum_{t_1+ \dots+t_n = 1}\chi_1(t_1) \cdots
        \chi_n(t_n)
    \]
    where \(t_i \in \F_p\). Another fruitful formula is
    \[
        J_0(\chi_1,\dots,\chi_n) = \sum_{t_1+\dots+t_n = 0}
        \chi_1(t_1)\cdots\chi_n(t_n)
    \]
\end{definition}

\begin{proposition}
    The following properties hold
    \begin{enumerate}[I.]
        \item \(J_0(\varepsilon, \dots, \varepsilon) = J(\varepsilon, \dots,
              \varepsilon) = p^{n-1}\).
        \item If not all \(\chi_i\) are trivial but some are, then
              \[
                  J_0(\chi_1,\dots,\chi_n) = J(\chi_1,\dots,\chi_n) = 0.
              \]
        \item Let \(\chi_n \neq  \varepsilon\), then
              \[
                  J_0(\chi_1,\dots,\chi_n) =
                  \begin{cases}
                      0,                                         & \text{ if } \prod_{i=1}^n \chi_i \neq  \varepsilon, \\
                      (p-1)\chi_n(-1)J(\chi_1,\dots,\chi_{n-1}), & \text{ otherwise. }
                  \end{cases}
              \]
    \end{enumerate}
\end{proposition}

\begin{proof}
    We prove each assertion
    \begin{enumerate}[i.]
        \item Notice that if we make \( \sum_{i=1}^{n}t_i = 0 \Rightarrow t_n =
              -\sum_{i=1}^{n-1} t_i\) is completely determined by the choices for the
              \(1 \leqslant i \leqslant n-1\) indexes and therefore they are the only
              ones that really vary through all of \(\F_p\), thus, every one of those
              has \(p\) possible choices and thus \(J(\varepsilon, \dots, \varepsilon) =
              \sum_{t_1+\dots+t_n = 0} \varepsilon(t_1)\dots \varepsilon(t_n) =
              \sum_{t_1 + \dots + t_n = 0} 1 = p^{n-1}\). The exact same argument
              structure can be used to show the same for \(J(\varepsilon, \dots,
              \varepsilon)\).
        \item Let the indexes \(1 \leqslant i \leqslant s\) be the ones with non
              trivial character and \(s+1 \leqslant i \leqslant n\) the ones with
              trivial character, then
              \[
                  \sum_{t_1+\dots+t_n=0} \chi_1(t_1) \cdots \chi_n(t_n)
                  = \sum_{t_1,\dots,t_{n-1} \in \F_p} \chi_1(t_1) \cdots \chi_s(t_s)
                  = p^{(n-s) - 1} \sum_{i=1}^{s} \left( \sum_{t_i \in \F_p}
                  \chi_i(t_i) \right) = 0
              \]
              The same can be argued for \(J(\chi_1, \dots, \chi_n)\).
        \item Note that rearranging terms we can write
              \[
                  J_0(\chi_1, \dots, \chi_n) = \sum_{t_1+\dots+t_n=0} \chi_1(t_1)
                  \cdots \chi_n(t_n) = \sum_{t_n \in \F_p} \left( \sum_{t_1 + \dots +
                      t_{n-1} = -t_n} \chi_1(t_1)\cdots\chi_{n-1}(t_{n-1})\right)
                  \chi_n(t_n)
              \]
              from hypothesis we have \(\chi_n \neq \varepsilon\) which implies that
              \(\chi_n(0) = 0\) and therefore, the case \(t_n=0\) does not contribute
              to the sum, thereof  we may consider only \(t_n \in \F_n^\ast\). Since
              we are assuming now that \(t_n\) is invertible, simply define for every
              index \(t_i := -t_n t_i'\), which yields on the inner sum of the
              equation above, the following
              \begin{align*}
                  \sum_{t_1 + \dots + t_{n-1} = -t_n}
                  \chi_1(t_1)\cdots\chi_{n-1}(t_{n-1})
                   & = \sum_{t_1' + \dots + t_{n-1}' = 1} \chi(-t_nt_1') \cdots
                  \chi(-t_nt_{n-1}')                                                   \\
                   & = \prod_{i = 1}^{n-1} \chi_i(-t_n) \sum_{t_1' + \dots + t_n' = 1}
                  \chi_1(t_1') \cdots \chi_n(t_n')                                     \\
                   & = \prod_{i=1}^{n-1} \chi_i(-t_n) J(\chi_1, \dots, \chi_n)
              \end{align*}
              therefore we find
              \begin{align*}
                  J_0(\chi_1, \dots, \chi_n)
                   & = \sum_{t_n \in \F_p^{\ast}} \left( \prod_{i=1}^{n-1} \chi_i(-t_n)
                  J(\chi_1,\dots, \chi_n)\right) \chi_n(t_n)                             \\
                   & = \prod_{i=1}^{n-1} \chi_i(-1) J(\chi_1,\dots,\chi_{n-1}) \sum_{t_n
                      \in \F_p^\ast}\left( \prod_{i=1}^{n} \chi_i(t_n) \right)
              \end{align*}
              Now we apply the two distinct cases. For \(\prod_{i=1}^{n} \chi_i =
              \varepsilon\) we have that
              \[
                  J_0(\chi_1,\dots,\chi_n) = \left( \chi_n(-1) \right)^{-1} J(\chi_1,
                  \dots, \chi_n) \sum_{t_n \in \F_p^\ast} \varepsilon(t_n) = (p-1)
                  \chi_n(-1) J(\chi_1,\dots,\chi_n)
              \]
              and for the last case if \(\prod_{i = 1}^n \chi_i \neq  \varepsilon\)
              we have that \(\sum_{t_n \in \F_p^\ast} \prod_{i=1}^n \chi_i(t_n) =
              0\), which concludes the argument since \(J_0(\chi_1, \dots, \chi_n)=
              0\) in this case.
    \end{enumerate}
\end{proof}

\begin{theorem}
    Let characters \(\chi_1, \dots, \chi_n\) be non-trivial and  \(\prod_{i =
        1}^{n} \chi_i \neq  \varepsilon\) then
    \[
        \prod_{i=1}^{n} g(\chi_i) = J(\chi_1, \dots, \chi_n) g(\chi_1 \cdots
        \chi_n).
    \]
\end{theorem}

\begin{proof}
    \todo[inline]{Maybe some day write the proof of this theorem about Jacobi
        Generalized, it's really the same thing as in the \(n=2\) case \dots}
\end{proof}

\begin{corollary}
    Let \(\chi_1,\dots,\chi_n\) be non-trivial and \(\prod_{i=1}^n \chi_i =
    \varepsilon\) then
    \[
        \prod_{i=1}^n g(\chi_i) = p\chi_n(-1) J(\chi_1,\dots,\chi_{n-1}).
    \]
\end{corollary}

\begin{corollary}
    Let \(\chi_1,\dots,\chi_n\) be non-trivial and \(\prod_{i=1}^n \chi_i =
    \varepsilon\) then
    \[
        J(\chi_1, \dots, \chi_n) = -\chi_n(-1) J(\chi_1,\dots,\chi_{n-1})
    \]
\end{corollary}

\begin{theorem}
    Let \(\chi_1, \dots, \chi_n\) be all non-trivial. Then
    \begin{enumerate}[I.]
        \item If \(\prod_{i=1}^n \chi_i \neq  \varepsilon\), then
              \[
                  |J(\chi_1, \dots, \chi_n)| = p^{\frac{n-1}{2}}.
              \]
        \item If \(\prod_{i=1}^n \chi_i = \varepsilon\), then
              \[
                  |J_0(\chi_1, \dots, \chi_n)| = (p-1)p^{\frac{n}{2} - 1}\ \text{ and
                  }\ |J(\chi_1,\dots, \chi_n)| = p^{\frac{n}{2} - 1}.
              \]
    \end{enumerate}
\end{theorem}

\begin{proof}
    If \(\prod_i \chi_i \neq  \varepsilon\) then \(|g(\chi_i)| = \sqrt{p}  \) and
    therefore
    \[
        p^{\frac{n}{2}} = |J(\chi_1, \dots, \chi_n)| p^{\frac{1}{2}}
        \Rightarrow  |J(\chi_1, \dots, \chi_n)| = p^{\frac{n-1}{2}}.
    \]
    On the contrary suppose the product is trivial and then we know that
    \[
        J_0(\chi_1,\dots,\chi_n) = \chi_n(-1) (p-1) J(\chi_1, \dots, \chi_{n-1})
    \]
    and therefore, since \(J(\chi_1, \dots, \chi_n) = -\chi_n(-1)
    J(\chi_1,\dots,\chi_{n-1})\) implies
    \[
        J_0(\chi_1,\dots, \chi_n) = - J(\chi_1, \dots, \chi_n) (p-1) = -
        \frac{\prod_{i=1}^{n}g(\chi_i)}{p \chi_n(-1)} (p-1)
    \]
    from this we conclude the assertion. More directly, the second equality can
    be obtained by using
    \[
        J(\chi_1,\dots,\chi_n) = \frac{\prod_{i=1}^{n+1}
            g(\chi_i)}{p\chi_{n+1}(-1)}.
    \]
\end{proof}

\begin{theorem}\label{thm: fermat generalization}
    Let the polynomial \(\sum_{i=1}^{n} a_i x_i^{\ell_i} = b \) where we have
    \(a_i \in \F_p^\ast\) and \(b \in \F_p\) also define \(N\) to be the number
    of solutions to such polynomial. Then
    \begin{enumerate}[I.]
        \item If \(b = 0\) then
              \[
                  N = p^{n-1} + \sum_{\chi_1,\dots,\chi_n} \left(
                  J_0(\chi_1,\dots,\chi_n) \prod_{i=1}^{n} \chi_i(a_i^{-1}) \right)
              \]
              where we define \(\chi_i^{\ell_i} = \varepsilon,\ \chi_i \neq
              \varepsilon\) and \(\prod_{i=1}^{n} \chi_i = \varepsilon\). Let \(M\)
              be the number of \(n\)-tuples \((\chi_1,\dots,\chi_n)\) that satisfy
              such properties, then
              \[
                  |N - p^{n-1}| \leqslant M(p-1) p^{\frac{n}{2} - 1}.
              \]
        \item If \(b \neq 0\) then
              \[
                  N = p^{n-1} + \sum_{\chi_1, \dots, \chi_n} \left(
                  \chi_1\cdots\chi_n(b) J(\chi_1,\dots,\chi_n) \prod_{i =1}^n
                  \chi_i(a_i^{-1}) \right)
              \]
              were, not exactly as above, we define \(x_i \neq  \varepsilon\) and
              \(\chi_i^{\ell_i} = \varepsilon\) only. Let \(M_0\) be the number of
              \(n\)-tuples \((\chi_1,\dots,\chi_n)\) satisfying such conditions and
              also \(\prod_{i=1}^{n}\chi_i = \varepsilon\), let also \(M_1\) be the
              same but now over the tuples such that \(\prod_{i=1}^n \chi_i \neq
              \varepsilon\), then
              \[
                  |N-p^{n-1}| \leqslant M_0 p^{\frac{n}{2} -1}  + M_1
                  p^{\frac{n-1}{2}}.
              \]
    \end{enumerate}
\end{theorem}

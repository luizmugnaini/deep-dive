\section{\texorpdfstring{\(\Grp\)}{Grp} Category}

\begin{definition}[Group morphism]\label{def: grp-morphism}
  Let \((G, \otimes)\) and \((H, \oplus)\) be groups together with their binary
  operation. A group morphism --- also called homomorphism --- is a map
  \(\varphi: (G, \otimes) \to (H, \oplus)\) such that the
  that the following diagram commutes
  \[
    \begin{tikzcd}
      G \times G \ar[d, swap, "\otimes"] \ar[r, "\varphi \times \varphi"]
      &H \times H \ar[d, "\oplus"] \\
      G \ar[r, "\varphi"] &H
    \end{tikzcd}
  \]
  Where \(\varphi \times \varphi\) is uniquely defined in \(\Set\) by
  \cref{prop: product-morphism} --- mapping \((g, \ell) \xmapsto{\varphi \times
  \varphi} (\varphi(g), \varphi(\ell))\). The commutativity of such diagram can
  be viewed as the requirement that \(\varphi\) preserves the structure coming
  from the binary operations --- that is, for any \(g, \ell \in G\)
  \[
    \varphi(g \otimes \ell) = \varphi(g) \oplus \varphi(\ell).
  \]
\end{definition}

\begin{definition}[Category of groups]\label{def: grp}
  The category of groups \(\Grp\) consists of the collection of objects ---
  called groups --- and group morphisms between them.
\end{definition}

\begin{proposition}
  \(\Grp\) is a category.
\end{proposition}

\begin{proof}
  Let \((G, \otimes)\), \((H, \oplus)\) and \((K, \star)\) be any groups. The
  identity \(\Id_G: G \to G\) is a group morphism since \(\Id_G(g \otimes \ell)
  = g \otimes \ell\) for any \(g, \ell \in G\). Moreover, we can define a map
  \[
    f: \Hom_{\Grp}(G, H) \times \Hom_{\Grp}(H, K) \to \Hom_{\Grp}(G, K)
  \]
  with the mapping \((\psi, \varphi) \xmapsto f \psi \varphi\) --- since
  the following diagram commutes
  \[
    \begin{tikzcd}
      G \times G \ar[r, "\varphi \times \varphi"]
      \ar[d, swap, "\otimes"]
      \ar[rr, bend left, "(\psi \varphi) \times (\psi \varphi)"]
      &H \times H \ar[r, "\psi \times \psi"]
      \ar[d, "\oplus"]
      &K \times K \ar[d, "\star"]
      \\
      G \ar[r, "\varphi"]
      \ar[rr, bend right, swap, "\psi \varphi"]
      &H \ar[r, "\psi"] &K
    \end{tikzcd}
  \]
  In other words, for any \(g, \ell \in G\) we have
  \[
    \psi \varphi (g \otimes \ell)
    = \psi(\varphi(g \otimes \ell))
    = \psi(\varphi(g) \oplus \varphi(\ell))
    = \psi(\varphi(g)) \star \psi(\varphi(\ell))
    = \psi \varphi(g) \star \psi \varphi(\ell).
  \]
  Therefore \(\psi \varphi \in \Hom_\Grp(G, K)\). For the other part of the
  diagram, we have
  \[
    (\psi \varphi) \times (\psi \varphi) (g, \ell)
    = (\psi \varphi(g), \psi \varphi(\ell))
    = \psi \times \psi(\varphi(g), \varphi(\ell))
    =(\psi \times \psi) (\varphi \times \varphi) (g, \ell)
  \]

  Since group morphisms are maps in \(\Set\), we have that associativity is
  inherited.
\end{proof}

\begin{proposition}\label{prop: forgetful-func-grp-set}
  There exists a covariant forgetful functor \(F: \Grp \to \Set\).
\end{proposition}

\begin{proof}
  For objects, define \(F\) as \(F(G, \otimes) = G\) --- where we denoted \(G\)
  together with its binary operation only to express that the multiplicative
  structure is lost in the process. Let \(\varphi: (G, \otimes) \to (H,
  \oplus)\) be a group morphism, denote by \(\overline \varphi \in \Mor(\Set)\)
  the function \(\overline\varphi: G \to H\) such that \(\overline\varphi(g) =
  \varphi(g)\) for all \(g \in G\). For such morphisms we define \(F\) as
  \(F\varphi = \overline\varphi: F(G, \otimes) \to F(H, \oplus)\).

  Let \(\psi \in \Hom_\Grp((H, \oplus), (K, \star))\), then we have
  \(\overline{\psi \varphi} = \overline \psi \overline \varphi: H \to K\). Thus
  \[
    F(\psi \varphi) = \overline{\psi \varphi}
    = \overline \psi \overline \varphi = F\psi F\varphi.
  \]
  This shows that \(F\) is a covariant forgetful functor.
\end{proof}

\begin{proposition}
  The trivial group \(* \in \Grp\) is the initial and final object of \(\Grp\).
  That is, for any \(G \in \Grp\) the diagram
  \[
    \begin{tikzcd}
      * \ar[r, dashed, bend right, swap, "\varphi"]
      &G \ar[l, dashed, bend right, swap, "\psi"]
    \end{tikzcd}
  \]
  commutes for uniquely defined group morphisms \(\varphi\) and \(\psi\).
\end{proposition}

\begin{proof}
  Let \(G \in \Grp\) be any group. We define maps \(\varphi: * \to G\) mapping
  \(e \xmapsto\varphi e_G\), where \(e\) is the only element of \(*\) --- being
  unique possible map \(* \to G\) that preserves the group structure. Clearly,
  \(\varphi\) is a group morphism since \(\varphi(e e) = \varphi(e) = e_G =
  \varphi(e) \varphi(e)\) --- this shows that \(*\) is the initial object of
  \(\Grp\). Let \(\psi: G \to *\) be a map defined by \(g \xmapsto\psi e\) ---
  which is clearly unique. Then \(\psi\) is a morphism of groups, because
  \(\psi(g h) = e = \psi(g) \psi(h)\) --- showing that \(*\) is the final object
  of \(\Grp\).
\end{proof}

\begin{proposition}[Group morphism properties]
  Let \((G, \otimes), (H, \oplus) \in \Grp\) and \(\varphi \in \Hom_\Grp(G,
  H)\). Define \(\operatorname{inv}_G: G \isoto G\) and \(\operatorname{inv}_H:
  H \isoto H\) as the maps \(g \xmapsto {\operatorname{inv}_G} g^{-1}\) and \(h
  \xmapsto {\operatorname{inv}_H} h^{-1}\). Then the following diagram commutes
  \[
    \begin{tikzcd}
      G \ar[r, "\varphi"]
      \ar[d, swap, "\operatorname{inv}_G"]
      &H \ar[d, "\operatorname{inv}_H"] \\
      G \ar[r, "\varphi"] &H
    \end{tikzcd}
  \]
\end{proposition}

\begin{proof}
  Let \(g \in G\) be any element, then
  \[
    \varphi(g^{-1}) = \varphi(g^{-1} e_G) = \varphi(g^{-1} \otimes g \otimes
    g^{-1}) = \varphi(g^{-1}) \oplus \varphi(g) \oplus \varphi(g^{-1}),
  \]
  applying cancellation law on the equation above we find
  \[
    e_H = \varphi(g) \oplus \varphi(g^{-1}).
  \]
  Hence \(\varphi(g^{-1}) = \varphi(g)^{-1}\). Moreover, this implies that \(e_G
  \xmapsto \varphi e_H\) since
  \[
    \varphi(e_G) = \varphi(g \otimes g^{-1}) = \varphi(g) \oplus \varphi(g^{-1})
    = \varphi(g) \oplus \varphi(g)^{-1} = e_H.
  \]
\end{proof}

\section{Group Products}

Let \((G, \otimes), (H, \oplus) \in \Grp\) be any objects. We define a binary
operation \(\cdot: (G \times H)^2 \to G \times H\) as the mapping
\begin{equation}\label{eq: grp-prod-bin}
  (g, h) \cdot (g', h') = (g \otimes g', h \oplus h').
\end{equation}
Such binary operation defines a group structure on \(G \times H\). Notice that,
given an element \((g, h) \in G \times H\), there exists an element \((g^{-1},
h^{-1}) \in G \times H\) such that \((g, h) \cdot (g^{-1}, h^{-1}) = (e_G,
e_H)\). Moreover, clearly \((e_G, e_H) \in G \times H\) is the identity element
of the structure. Hence \((G \times H, \cdot) \in \Grp\).

Also, the natural projections \(\pi_G: G \times H \to G\) and \(\pi_H: G \times
H \to H\) define morphisms of groups.

\begin{definition}[Direct product]
  The group \(G \times H\) is called direct product group of \(G\) and \(H\).
\end{definition}

\begin{proposition}
  The direct products of groups are products on the category of groups,
  \(\Grp\). That is, for all group \(W\) and group morphisms \(f \in
  \Hom_\Grp(W, G)\) and \(g \in \Hom_\Grp(W, H)\), there exists a unique
  morphism \(\varphi \in \Hom_\Grp(W, G \times H)\) such that the following
  diagram commutes
  \[
    \begin{tikzcd}
      &W
      \ar[d, dashed, "\varphi"]
      \ar[ddr, bend left, "g"]
      \ar[ddl, swap, bend right, "f"]
      & \\
      &G \times H \ar[dr, swap, "\pi_H"] \ar[dl, "\pi_G"] & \\
      G & &H
    \end{tikzcd}
  \]
\end{proposition}

\begin{proof}
  We just take \(\varphi: W \to G \times H\) as the mapping \(w \xmapsto \varphi
  (f(w), g(w))\). We show that \(\varphi\) exists in \(\Grp\): let \(x, y \in
  W\) be any elements, then, since \(f\) and \(g\) are group morphisms, we find
  that
  \[
    \varphi(xy) = (f(xy), g(xy)) = (f(x) f(y), g(x) g(y))
    = (f(x), g(x)) (f(y), g(y)) = \varphi(x) \varphi(y).
  \]
  That is, \(\varphi\) is a group morphism. The uniqueness comes from the
  covariant functor \(F: \Grp \to \Set\), since \(\Set\) allows for products and
  hence the set-function \(F \varphi\) is unique.
\end{proof}

\begin{remark}
  We now show that if \(G, H \in \Grp\) are such that \(G \iso H \times G\),
  \emph{it does not follow that} \(H\) is trivial.
  \todo[inline]{Find counterexample}
\end{remark}

\section{\texorpdfstring{\(\Ab\)}{Ab} --- Category of Abelian Groups}

\begin{definition}[Category of abelian groups]
  We define the category of abelian groups, denoted \(\Ab\), to be the category
  whose objects are abelian groups and group morphisms between them.
\end{definition}

\begin{proposition}[Coproduct in \(\Ab\)]\label{prop: coprod-ab}
  The direct product of abelian groups is a coproduct in \(\Ab\). That is, for
  any \(G, H, W \in \Ab\) and morphisms \(f \in \Hom_\Ab(W, G)\) and \(k \in
  \Hom_\Ab(W, H)\), there exists a unique morphism \(\varphi \in \Hom_\Ab(W, G
  \times H)\) such that the following diagram commutes
  \[
    \begin{tikzcd}
      G \ar[dr, "\iota_G"] \ar[ddr, bend right, swap, "f"] &
      &H \ar[dl, swap, "\iota_H"] \ar[ddl, bend left, "k"] \\
      &G \times H \ar[d, dashed, "\varphi"]  & \\
      &W &
    \end{tikzcd}
  \]
  Where we define inclusion morphisms \(g \xmapsto{\iota_G} (g, e_H)\) and \(h
  \xmapsto{\iota_H} (e_G, h)\).
\end{proposition}

\begin{proof}
  Since \(\Ab \subset \Grp\) --- that is, the category of abelian groups is a
  subcategory of \(\Grp\) --- then, from \cref{prop: forgetful-func-grp-set}
  there exists a functor \(\Ab \to \Set\). Since coproducts exists in \(\Set\)
  and are unique, a set-function \(\varphi\) exists and is unique, commuting the
  diagram in \(\Set\). We now show that we can extend such set-function into a
  morphism of groups. Let \(\varphi: G \times H \to W\) be the mapping \((g, h)
  \xmapsto \varphi f(g) k(h)\). Notice that
  \begin{align*}
    \varphi((g, h)(g', h'))
    = \varphi(gg', hh')
    = f(g g') k(hh')
    = f(g) f(g') k(h) k(h')
    &= f(g) k(h) f(g') k(h') \\
    &= \varphi(g, h) \varphi(g', h')
  \end{align*}
  that is, \(\varphi\) is a morphism of groups.
\end{proof}

\begin{remark}[Coproducts in \(\Grp\)]
  Proposition \ref{prop: coprod-ab} is not at all true for the category of
  groups. Consider for instance the cyclic groups \(C_2 = \{e_x, x\}\) and \(C_3
  = \{e_y, y, y^2\}\). Let \(\sigma_k \in S_3\), for \(0 \leq k \leq 2\) be the
  rotation of \(\{1, 2, 3\}\) by \(k\), that is, the permutations represented by
  \[
    M_{\sigma_0} =
    \begin{bmatrix}
      1 &0 &0 \\ 0 &1 &0 \\ 0 &0 &1
    \end{bmatrix}
    \qquad
    M_{\sigma_1} =
    \begin{bmatrix}
      0 &0 &1 \\ 1 &0 &0 \\ 0 &1 &0
    \end{bmatrix}
    \qquad
    M_{\sigma_2} =
    \begin{bmatrix}
      0 &1 &0 \\ 0 &0 &1 \\ 1 &0 &0
    \end{bmatrix}
  \]
  Consider the embeddings \(f: C_2 \mono S_3\) mapping \(x^k \xmapsto f
  \sigma_k\) for \(k \in \{0, 1\}\), and \(g: C_3 \mono S_3\) mapping \(y^k
  \xmapsto g \sigma_k\) for \(k \in \{0, 1, 2\}\).

  Suppose, for the sake of contradiction, that \(C_2 \times C_3\) is a coproduct
  in \(\Grp\), that is, exists a unique morphism \(\varphi: C_2 \times C_3 \to
  S_3\) such that \(f = \varphi \iota_{C_2}\) and \(g = \varphi \iota_{C_3}\).
  Since \(\varphi\) is supposedly a morphism of groups,
  \begin{gather*}
    \varphi(x, y) = \varphi(x, e_y) \varphi(e_x, y) = \sigma_1 \sigma_1 =
    \sigma_2 \\
    \varphi(x, y^2) = \varphi(x, e_y) \varphi(e_x, y^2) = \sigma_1 \sigma_2
    = \sigma_0
  \end{gather*}
  However, \(\varphi(e_x, e_y) = \sigma_0\) and on the other hand we have
  \(\varphi(x, y) \varphi(x, y^2) = \sigma_2\), which contradicts the properties
  of a group morphism. This shows that there exists no such \(\varphi\) in
  \(\Grp\) and hence \(C_2 \times C_3\) is not a coproduct in \(\Grp\).

  Although \(C_2 \times C_3\) is not a coproduct in \(\Grp\), that doesn't mean
  that \(\Grp\) has no coproducts, they just behave differently when compared
  with \(\Ab\). For instance, let \(C_2 * C_3 \in \Grp\) be defined to be the
  group generated by elements \(x\) and \(y\), such that \(x^2 = e\) and \(y^3 =
  e\). We'll now show that \(C_2 * C_3\) is a coproduct of \(C_2\) and \(C_3\)
  in \(\Grp\). Let \(G\) be any group and consider morphisms \(f: C_2 \to G\)
  and \(k: C_3 \to G\). The inclusions \(\iota_{C_2}: C_2 \to C_2 * C_3\) and
  \(\iota_{C_3}: C_3 \to C_2 * C_3\) will be naturally given maps by taking each
  element to itself.

  Let \(q \in C_2 * C_3\) be any element. We know that there exists a finite
  collection of coefficients \(I = \{(a, b) \in \Z^2\}\) such that \(q =
  \prod_{(a, b) \in I} x^a y^z\). Define \(\phi: C_2 * C_3 \to G\) as the
  mapping
  \[
    \phi(q) = \phi\left(\prod_{(a, b) \in I} x^a y^z\right)
    = f\left(x^{\sum_{(a,b) \in I} a}\right)
    k\left(y^{\sum_{(a,b) \in I} b}\right).
  \]
  Notice that \(\phi(e) = f(e) k(e) = e_G\) and for all \(q, p \in C_2 * C_3\)
  --- with respective coefficients \(I = \{(a, b) \in \Z^2\}\) and \(J = \{(c,d)
  \in \Z^2\}\) --- we have
  \begin{align*}
    \phi(qp)
    &= \phi\left(\prod_{(a, b) \in I} x^a y^b
      \prod_{(c, d) \in J} x^c y^d\right)
    \\
    &= f \left( x^{\sum_{(\alpha, \beta) \in I \cup J} \alpha} \right)
    k\left( y^{\sum_{(\alpha, \beta) \in I \cup J} \beta} \right)
    %\\
    %&= f \left( x^ \right)
  \end{align*}
  \todo[inline]{finish}
\end{remark}

\begin{proposition}[Fiber products]
  Fiber products exist in \(\Ab\). That is, given abelian groups \(G, H, W \in
  \Ab\) and group morphisms \(\phi \in \Hom_\Ab(G, W)\) and \(\psi \in
  \Hom_\Ab(H, W)\). Let
  \[
    G \times_W H \in \Ab_W
  \]
  be the fiber product of \(\phi\) and \(\psi\) in the category \(\Ab_W
  \subseteq \Ab\)\footnote{As a reminder: \(\Ab_W\) is the category whose
  objects are morphisms \(f \in \Hom_\Ab(-, W)\). If \(f: G \to W\) and \(g: H
  \to W\) are objects of \(\Ab_W\), a morphism \(h \in \Hom_{\Ab_W}(f, g)\) is
  such that \(hg = f\).}
  --- for which exists natural projections \(\pi_G\) and \(\pi_H\). Let \(Q \in
  \Ab\) be any abelian group and consider any morphisms \(f \in \Hom_\Ab(Q, G)\)
  and \(k \in \Hom_\Ab(Q, H)\). Then there exists a unique morphism \(\varphi
  \in \Hom_\Ab(Q, P)\) such that the following diagram commutes
  \[
    \begin{tikzcd}
      Q \ar[dr, dashed, "\varphi"] \ar[ddr, bend right, swap, "f"]
      \ar[drr, bend left, "k"]
      & & \\
      &G \times_W H \ar[r, "\pi_H"] \ar[d, swap, "\pi_G"] &H \ar[d, "\psi"] \\
      &G \ar[r, swap, "\phi"] &W
    \end{tikzcd}
  \]
\end{proposition}

\begin{proof}
  Define \(G \times_W H = \{(g, h) \in G \times H : \phi(g) = \psi(h)\}\). Since
  \(\phi\) and \(\psi\) are group morphisms, given \((g, h) \in G \times_W H\),
  we have
  \[
    \phi(g^{-1}) = \phi(g)^{-1} = \psi(h)^{-1} = \psi(h^{-1}),
  \]
  that is, \((g^{-1}, h^{-1}) \in G \times_W H\) exists and is the inverse of
  \((g, h)\). Moreover, given any \((g, h), (g', h') \in G \times_W H\), the
  product \((g, h)(g', h') = (gg', hh')\) is such that
  \[
    \phi(gg') = \phi(g) \phi(g') = \psi(h) \psi(h') = \psi(hh'),
  \]
  thus \((gg', hh') \in G \times_W H\). This shows that \(G \times_W H\) is
  indeed a group and since \(G \times H\) is abelian, so is the fiber product
  defined above.

  From the forgetful functor \(\Ab \to \Set\) we know that there exists a unique
  set-function \(\varphi\) such that the diagram commutes in \(\Set\). If we
  define \(\varphi\) as the mapping \(q \overset \varphi \longmapsto (f(q),
  k(q))\). Notice that
  \[
    \varphi(qq') = (f(qq'), k(qq')) = (f(q)f(q'), k(q)k(q'))
    = (f(q), k(q))(f(q'), k(q')) = \varphi(q) \varphi(q'),
  \]
  so \(\varphi \in \Mor(\Ab)\).
\end{proof}

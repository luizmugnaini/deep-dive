\section{Compactly Generated Spaces}

\subsection{Weak-Hausdorff Spaces}

\begin{definition}[Weak-Hausdorff space]
\label{def:weak-hausdorff}
A space \(X\) is said to be \emph{weak-Hausdorff} if for every compact Hausdorff
space \(K\) and continuous map \(g: K \to X\) one has that \(g(K)\) is
\emph{closed} in \(X\).
\end{definition}

\begin{corollary}
\label{cor:weak-hausdorff-is-T1}
A weak-Hausdorff space is T\(_1\).
\end{corollary}

\begin{proof}
Let \(X\) be a weak-Hausdorff space and \(x \in X\), we shall take advantage of
the fact that \(X\) is T\(_1\) if and only if \(\{x\}\) is closed in \(X\). Let
\(K\) be a compact set and \(\const_x: K \to X\) be the constant map \(k \mapsto
x\), which is continuous. Since \(g K = \{x\}\), and \(g K\) is closed by
hypothesis, the property follows.
\end{proof}

\begin{lemma}
\label{lem:weak-haudorff-has-compact-hausdorff-subspace}
If \(X\) is weak-Hausdorff, \(K\) is a compact Hausdorff space, and
\(g: K \to X\) is a morphism, then \(g K\) is a \emph{compact Hausdorff subspace
  of \(X\)}.
\end{lemma}

\begin{proof}
We already know from \cref{prop:image-of-compact-is-compact} that \(g K\) is
compact, thus we shall only prove that \(g K\) is Hausdorff. Let
\(x, y \in g K\) be any two distinct points. Using
\cref{cor:weak-hausdorff-is-T1} and \cref{prop:compact-hausdorff-is-normal} we
can find \emph{disjoint open sets} \(U, V \subseteq K\) such that
\(g^{-1}x \subseteq U\) and \(g^{-1}y \subseteq V\). Then the sets
\(K \setminus U\) and \(K \setminus V\) are \emph{closed} in \(K\), thus
\emph{compact} (see \cref{prop:closed-subset-compact}). Therefore the sets
\(g(K \setminus U)\) and \(g(K \setminus V)\) are both \emph{closed in
  \(X\)}. From this we know that \(g K \setminus g(K \setminus U)\) and
\(g K \setminus g(K \setminus V)\) are both \emph{open in \(g K\)}, both of
which are \emph{disjoint} and contain \(x\) and \(y\), respectively---which
shows that \(X\) is Hausdorff.
\end{proof}

\begin{definition}[Compactly closed]
\label{def:compactly-closed}
A subset \(A\) of \(X\) is said to be \emph{compactly closed} if, for every
compact space \(K\) and morphism \(g: K \to X\), the preimage \(g^{-1} A\) is
closed in \(K\).
\end{definition}

\begin{proposition}
\label{prop:compactly-closed-in-weak-hausdorff-space}
If \(X\) is a weak-Hausdorff space, a subset \(A \subseteq X\) is compactly
closed if and only if the intersection of \(A\) with each compact subset of
\(X\) is closed.
\end{proposition}

\begin{proof}
\begin{itemize}\setlength\itemsep{0em}
\item (\(\implies\)) Suppose \(A\) is compactly closed, and let
  \(K \subseteq X\) be any compact subspace of \(X\). If we consider the
  inclusion morphism \(\iota: K \emb X\), we have that
  \(\iota^{-1} A = K \cap A\) is closed.

\item (\(\impliedby\)) Assume the latter property, and let \(K\) be a compact
  space together with a morphism \(g: K \to X\). Since \(g K\) is compact in
  \(X\) it follows from assumption that \(A \cap g K\) is closed in \(X\). Since
  \(g\) is continuous, then \(g^{-1}(A \cap g K) = g^{-1} A\) is closed in
  \(K\).
\end{itemize}
\end{proof}

\subsection{\texorpdfstring{\(k\)}{k}-Spaces}

\begin{definition}[\(k\)-space]
\label{def:k-space}
A space \(X\) is said to be a \emph{\(k\)-space} if every compactly closed
subspace of \(X\) is closed. The full-subcategory of \(\Top\) whose objects are
\(k\)-spaces will be denoted by \(\kTop\).
\end{definition}

\begin{lemma}[\(k\)-ification]
\label{lem:k-ification}
Given a topological space \((X, \tau)\), we can transform \(X\) into a
\(k\)-space by creating a topology \(\tau_k\) where \(C\) is closed in
\((X, \tau_k)\) if and only if \(C\) is compactly closed in \((X, \tau)\). We
shall shortly denote the \(k\)-space \((X, \tau_k)\) by \(kX\).
\end{lemma}

\begin{definition}[\(k\)-ification functor]
\label{def:k-ification-functor}
We define the \(k\)-ification functor
\[
k: \Top \longrightarrow \kTop
\]
to be the functor mapping topological spaces \(X\) to its \(k\)-ified space
\(k X\), and \(k f \coloneq f\) for every topological morphism \(f\).
\end{definition}

\begin{lemma}
\label{lem:weak-hausdorff-into-compactly-generated-space}
If \(X\) is a weak-Hausdorff space, then \(kX\) is also a weak-Hausdorff
\(k\)-space.
\end{lemma}

\begin{notation}[Products]
\label{not:cartesian-and-k-products}
In what follows, we shall denote by \(X \times_{\text{c}} Y\) the
\emph{cartesian product} of spaces \(X\) and \(Y\), which shall be endowed with
the usual product topology. Moreover, from now on we shall reserve the notation
\[
X \times Y \coloneq k(X \times_{\text{c}} Y),
\]
which may seem odd, but is a convention used throughout the literature and we'll
adopt here.
\end{notation}

\begin{proposition}[Quotients]
\label{prop:quotient-k-space-is-k-space}
The quotient of a \(k\)-space is a \(k\)-space.
\end{proposition}

\begin{proof}
Let \(X\) be a \(k\)-space and \(q: X \epi Y\) be a quotient map---we want to
show that \(Y\) is a \(k\)-space. Let \(A \subseteq Y\) be a compactly closed
subset. We'll prove that \(q^{-1} A\) is compactly closed in \(X\), thus closed,
yielding the conclusion that \(A\) is closed in \(Y\). Given any compact space
\(K\) and a morphism \(\phi: K \to X\). Then the map \(q \phi: K \to Y\) is such
that \((q \phi)^{-1} A = \phi^{-1}(q^{-1} A)\) is closed in \(K\) since \(A\) is
compactly generated---the result follows.
\end{proof}

\begin{proposition}
\label{prop:k-space-product-of-quotient-maps-is-quotient-map}
Let \(X\) and \(Y\) be \(k\)-spaces, and consider quotient maps \(q: X \epi X'\)
and \(p: Y \epi Y'\). Then the product
\[
q \times p: X \times Y \epi X' \times Y'
\]
is also a quotient map.
\end{proposition}

\begin{proposition}
\label{prop:k-space-is-weak-hausdorff-iff-diagonal-closed}
A \(k\)-space \(X\) is \emph{weak-Hausdorff} if and only if the \emph{diagonal}
\(\Delta X\) is \emph{closed} in \(X \times X\).
\end{proposition}

\begin{proof}
\begin{itemize}\setlength\itemsep{0em}
\item (\(\impliedby\)) Assume that \(\Delta X\) is closed in \(X \times X\). Let
  \(K\) be a compact space and \(\phi: K \to X\) a continuous map. Since \(X\)
  is a \(k\)-space, we may simply show that \(\phi K\) is compactly closed. To
  that end, let \(C\) be another compact space together with a morphism
  \(\psi: C \to X\), then
  \[
  \psi^{-1}(\phi K) = \pi_2 (\phi \times \psi)^{-1}(\Delta X)
  \]
  is a closed subset of \(C\)---where \(\pi_2: K \times C \epi C\) is the
  canonical second projection. This proves that \(\phi K\) is compactly closed.

\item (\(\implies\)) Suppose \(X\) is a weak-Hausdorff \(k\)-space. It suffices
  to show that \(\Delta X\) is compactly closed in \(X \times_{\text{c}} X\), so
  that \(\Delta X\) is closed in the \(k\)-space \(X \times X\). Let \(K\) be
  any compact set and \(\phi: K \to X \times_{\text{c}} X\) be a
  morphism. Considering the canonical projections
  \(\pi_1, \pi_2: X \times_{\text{c}} X \para X\), define the set
  \[
  A \coloneq \pi_1(\phi K) \cup \pi_2(\phi K)
  \]
  of \(X\). The set \(A\) is constructed so that one has
  \(\phi K \subseteq A \times_{\text{c}} A\) and hence
  \(\phi^{-1}(\Delta X) = \phi^{-1}(\Delta A)\). Since \(X\) is weak-Hausdorff,
  if we consider the maps \(\pi_1 \phi, \pi_2 \phi: K \para X\), we find that
  \(\pi_1\phi K\) and \(\pi_2 \phi K\) are both compact Hausdorff subspaces of
  \(X\)---hence \(A\) is a compact Hausdorff space. Since \(A\) is Hausdorff, it
  follows that \(\Delta A\) is closed in \(A \times_{\text{c}} A\), therefore by
  continuity \(\phi^{-1}(\Delta A) = \phi^{-1}(\Delta X)\) is closed in \(K\).
\end{itemize}
\end{proof}

\subsection{CG Spaces}

\begin{definition}[Compactly generated space]
\label{def:compactly-generated}
A space \(X\) is said to be \emph{compactly generated} (or, shortly, CG space)
if \(X\) is a \emph{weak-Hausdorff \(k\)-space}. The full-subcategory of
\(\Top\) composed of compactly generated spaces will be denoted by \(\cgTop\).
\end{definition}

The \(k\)-ification functor \(k: \Top \to \kTop\) can also act on the category
of weak-Hausdorff spaces \(\wHTop\), producing compactly generated spaces:
\[
k: \wHTop \longrightarrow \cgTop.
\]

\begin{lemma}
\label{lem:cgTop-and-wHTop}
The canonical forgetful functor \(j: \cgTop \to \wHTop\) embedding CG
spaces as weak-Hausdorff spaces is such that there exists a \emph{bijection}
\[
\Hom_{\cgTop}(X, k Y) \iso \Hom_{\wHTop}(j X, Y)
\]
for every \(X \in \cgTop\) and \(Y \in \wHTop\). Then \(k\) is
\emph{right-adjoint} to \(j\):
\[
\begin{tikzcd}
\cgTop \ar[r, shift left, "j"]
&\wHTop \ar[l, shift left, "k"]
\end{tikzcd}
\]
\end{lemma}

\begin{example}
\label{exp:compactly-generated-fst-ctbl-lcly-cpct-and-wk-haus}
As examples of compactly generated spaces we have:
\begin{enumerate}[(a)]\setlength\itemsep{0em}
\item If \(X\) is \emph{locally compact}, then it is a compactly generated
  space.

\item If \(X\) is a \emph{first-countable weak-Hausdorff} space, then it is
  compactly generated.
\end{enumerate}
\end{example}

\begin{lemma}
\label{lem:relations-cartesian-and-k-product}
Let \(X\) and \(Y\) be topological spaces, then
\begin{enumerate}[(a)]\setlength\itemsep{0em}
\item If \(X\) is locally compact and \(Y\) is compactly generated, then
  \[
  X \times Y = X \times_{\text{c}} Y.
  \]
\item If both \(X\) and \(Y\) are weak-Hausdorff, then
  \[
  X \times Y = k X \times k Y.
  \]
\item If both \(X\) and \(Y\) are compactly generated, then \(X \times Y\) is a
  \emph{product} in the category \(\cgTop\).
\end{enumerate}
\end{lemma}

\begin{lemma}
\label{lem:compactly-generated-cont-map-iff-restriction-to-cpct}
Let \(X\) be a compactly generated space. A set-function \(f: X \to Y\) is
\emph{continuous} if and only if the restriction \(f|_K\) is continuous for
every compact subspace \(K \subseteq X\).
\end{lemma}

\begin{proposition}[Quotients of CG spaces]
\label{prop:quotient-of-compactly-generated-space}
Let \(X\) be a compactly generated space, and \(q: X \epi Y\) be a quotient
map. Then \(Y\) is \emph{compactly generated} if and only if the preimage of the
diagonal of \(Y\),
\[
(q \times q)^{-1}(\Delta Y),
\]
is \emph{closed} in \(X \times X\).
\end{proposition}

\begin{proof}
\begin{itemize}\setlength\itemsep{0em}
\item (\(\implies\)) Suppose \(Y\) is compactly generated. From
  \cref{prop:k-space-is-weak-hausdorff-iff-diagonal-closed} we know that
  \(\Delta Y\) is a closed subspace of \(Y \times Y\), therefore by continuity
  of \(q \times q\) with respect to the product topology
  \(X \times_{\text{c}} X\) (see
  \cref{prop:product-of-continuous-maps-is-continuous}) we obtain that
  \((q \times q)^{-1}(\Delta Y)\) is a closed subspace of
  \(X \times_{\text{c}} X\), it follows\footnote{ Given any space \(Z\), if
    \(C \subseteq Z\) is a closed set then, given any compact set \(K\) and
    continuous map \(g: K \to Z\), we are always ensured that \(g^{-1} C\) is
    closed---this merely follows from the continuity of \(g\). Therefore \(k Z\)
    preserves the closed sets of \(Z\), in the sense that if \(C\) is closed in
    \(Z\) then \(C\) is also closed in \(k Z\) } that
  \((q \times q)^{-1}(\Delta Y)\) is closed in \(X \times X\).

\item (\(\impliedby\)) Suppose that \((q \times q)^{-1}(\Delta Y)\) is closed in
  \(X \times X\). Since \(q\) is a quotient map, then \(Y\) is a quotient
  space of a \(k\)-space \(X\)---therefore \(Y\) itself is a \(k\)-space.

  Since \(q\) is a quotient map and \(X\) is in particular a \(k\)-space, from
  \cref{prop:k-space-product-of-quotient-maps-is-quotient-map} we know that
  \(q \times q: X \times X \epi Y \times Y\) is a quotient map. Since
  \((q \times q)^{-1}(\Delta Y)\) is closed in \(X \times X\) then \(\Delta Y\)
  is closed in \(Y \times Y\). Therefore by
  \cref{prop:k-space-is-weak-hausdorff-iff-diagonal-closed} we conclude that
  \(Y\) is compactly generated.
\end{itemize}
\end{proof}

\begin{proposition}
\label{prop:attaching-space-compactly-generated}
Let \(X\) and \(Y\) be compactly generated spaces, and \(A \subseteq X\) be a
\emph{closed subspace}. Then for every morphism \(f: A \to Y\) the attaching
space \(Y \cup_f X\) is \emph{compactly generated}.
\end{proposition}

\begin{definition}[Weak topology]
\label{def:weak-topology}
Let \((X_j)_{j \in J}\) be a collection of topological spaces together with
inclusions \(X_j \emb X_{j+1}\). We define the \emph{weak topology} on the union
set \(X \coloneq \bigcup_{j \in J} X_j\) as follows: a set \(U \subseteq X\) is
\emph{open} if and only if \(U \cap X_j\) is open in \(X_j\) for all
\(j \in J\).
\end{definition}

\begin{proposition}[Colimits]
\label{prop:compactly-generated-passage-colimit}
Let \((X_j)_{j \in J}\) be a collection of compactly generated spaces together
with inclusions \(X_j \emb X_{j+1}\) with closed images. Then the \emph{colimit}
\[
\Colim_{j \in J} X_j = \bigcup_{j \in J} X_j
\]
is \emph{compactly generated}\footnote{
  The colimit of the sequence is endowed with the weak topology (see
  \cref{def:weak-topology}).
}.
\end{proposition}

\section{Construction of CW-Complexes}

\begin{definition}[Attaching cells to a space]
\label{def:attaching-cells}
Let \(f: \coprod_{j \in J} S_j^{n-1} \to X\) be a morphism of topological
spaces, where \(J\) is a set and \(S_j^{n-1}\) is an indexed copy of the
\((n-1)\)-sphere. We shall consider the attaching space \(Y\) given by the
\emph{pushout}
\[
\begin{tikzcd}
\coprod_{j \in J} S_j^{n-1} \ar[r, "f"] \ar[d, hook]
\ar[rd, phantom, "\ulcorner", very near end]
&X \ar[d]
\\
\coprod_{j \in J} D_j^n \ar[r]
&X \cup_f \Big( \coprod_{j \in J} D_j^n \Big) \eqqcolon Y
\end{tikzcd}
\]
where \(D_j^n \hookleftarrow S^{n-1}\) denotes an indexed copy of the
\(n\)-disk. We define the following notions concerning the triple \((Y, X, f)\):
\begin{itemize}\setlength\itemsep{0em}
\item An \emph{\(n\)-cell} in \(Y\) is defined to be the image \(e_j^n\) of a
  disk \(D_j^n\) in \(Y\), so that by construction we have
  \(Y = X \cup \big( \bigcup_{j \in J} e_j^n \big)\).

\item One can decompose \(f\) into a collection of maps
  \((f_j^n: D_j^n \to Y)_{j \in J}\), each of these is called a
  \emph{characteristic map}.
\end{itemize}
\end{definition}

\begin{definition}[Filtered space]
\label{def:filtered-space}
A \emph{filtered topological space} \(X\) is a space together with an increasing
sequence \((X_n)_{n \in \N}\) of closed subspaces, with inclusions
\(X_n \emb X_{n+1}\) for all \(n \in \N\), and
\[
X = \Colim_{n \in \N} X_n = \bigcup_{n \in \N} X_n.
\]
Where \(X\) is endowed with the \emph{weak topology}.
\end{definition}

\begin{definition}[Relative cell complex]
\label{def:relative-cell-complex}
A \emph{relative cell complex} is a pair \((X, A)\) where \(A\) is a
\(k\)-space, and \(X\) is a filtered space with:
\begin{itemize}\setlength\itemsep{0em}
\item The initial space of the sequence \((X_n)_{n \in \N}\) associated to \(X\)
  is
  \[
  X_0 = A \disj \Big( \coprod_{j \in J_0} D_j^0 \Big),
  \]
  where \(D_j^0\) is a copy of the \(0\)-disk indexed by a set \(J_0\).

\item For each \(n \in \N\) we have an associated attaching map
  \(f_n: \coprod_{j \in J_{n+1}} S_j^n \to X_n\) such that
  \[
  X_{n+1} = X_n \cup_{f_n} \Big( \coprod_{j \in J_{n+1}} D_j^{n+1} \Big)
  \]
  where \(S_j^n\) and \(D_j^{n+1}\) are copies of the \(n\)-sphere and
  \((n+1)\)-disk indexed by a set \(J_{n+1}\), respectively. In other words, the
  following diagram is a \emph{pushout} in \(\Top\):
  \[
  \begin{tikzcd}
  \coprod_{j \in J_{n+1}} S_j^{n-1} \ar[r, "f_n"] \ar[d, hook]
  \ar[rd, phantom, "\ulcorner", very near end]
  &X_n \ar[d, hook]
  \\
  \coprod_{j \in J_{n+1}} D_j^n \ar[r] &X_{n+1}
  \end{tikzcd}
  \]
  Each space \(X_n\) is called the \emph{\(n\)-skeleton} of \((X, A)\).
\end{itemize}
We say that \((X, A)\) is a \emph{finite} relative cell complex if \(X\) has
finitely many \(n\)-cells for each \(n \in \N\). Moreover, if \(X = X_n\) we say
that \((X, A)\) is an \emph{\(n\)-dimensional} relative cell complex.

Together with relative cell complexes we define a \emph{cellular map}
\(\phi: (X, A) \to (Y, B)\) between relative cell complexes to be a continuous
map \(\phi(X_n) \subseteq Y_n\) for each \(n \in \N\). We say that a pair
\((Y, A)\) is a \emph{subcomplex} of \((X, A)\) if \(Y\) is a subspace of \(X\)
that is given by the union of \(A\) with cells of \(X\).

A \emph{CW-complex} is a relative cell complex \((X, \emptyset)\), that is, each
skeleton \(X_n\) is formed exclusively by attaching \(n\)-cells on
\(X_{n-1}\).
\end{definition}

%%% Local Variables:
%%% mode: latex
%%% TeX-master: "../../../deep-dive"
%%% End:

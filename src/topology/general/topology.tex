\section{Topology}

\begin{definition}[Topology]
Let \(X\) be a set and \(\mathcal T \subseteq 2^X\). We say that \(\mathcal T\) is a
topology for \(X\) if
\begin{enumerate}[(T1)]\setlength\itemsep{0em}
  \item \(X, \emptyset \in \mathcal T\).
  \item The arbitrary union of elements of \(\mathcal T\) is an element of
    \(\mathcal T\).
  \item The finite intersection of elements of \(\mathcal T\) is an element of
    \(\mathcal T\).
\end{enumerate}
The elements of \(\mathcal T\) are called open sets of \(X\).
\end{definition}

\begin{example}\label{exp:some-topologies}
We proceed by listing some examples of topologies that are somewhat interesting,
they are included here in order to familiarize the reader with the possible
contructions for the topology on a given set.
\begin{itemize}\setlength\itemsep{0em}
\item The collection \(\mathcal T_1 = \{U \subseteq X : U = \emptyset \text{ or } X \setminus U \text{
is finite}\}\) is the cofinite topology on \(X\).

\item The collection \(\mathcal T_2 = \{U \subseteq X : U = \emptyset \text{ or } X \setminus U \text{
is countable}\}\) is the cocountable topology on \(X\).

\item Let \(p \in X\), then \(\mathcal T_3 = \{U \subseteq X : U = \emptyset \text{ or } p \in U\}\)
is the particular point topology on \(X\).

\item Let \(p \in X\), then \(\mathcal T_4 = \{U \subseteq X : U = X \text{ or } p \not\in
U\}\) is the excluded point topology on \(X\).
\end{itemize}
\end{example}

\subsection{Closed and Open Sets}

The notion of a closed set and an open set are closedly related --- pardon for the
pun. They have somewhat of a dual relationship, allowing for us to define the
notion of topology via either of them. Lets first define the closed set.

\begin{definition}[Closed set]
Let \(X\) be a topological space. We define a set \(A \subseteq X\) to be
closed if \(X \setminus A\) is open.
\end{definition}

The following proposition realizes the idea that the duality of open and closed
sets allow us to work with topological spaces by analysing both open and closed
elements of the space of interest.

\begin{proposition}
Let \(X\) be a topological space, then
\begin{enumerate}\setlength\itemsep{0em}
  \item The sets \(X\) and \(\emptyset\) are closed.
  \item The finite union of closed sets is closed.
  \item The arbitrary intersection of closed sets is closed.
\end{enumerate}
\end{proposition}

\begin{definition}[Miscelaneous]
Let \(X\) be a topological space and \(A \subseteq X\) be a set. We define
\begin{enumerate}\setlength\itemsep{0em}
  \item\label{def: closure}
    The closure of \(A\) is the least closed set, \(\overline A\), that
    contains \(A\). This can be equivalently described as
    \[
      \overline A = \bigcap \{F \subseteq X : A \subseteq F \text{ and } F
      \text{ is closed}\}.
    \]
  \item\label{def: interior}
    The interior of \(A\) is the biggest open set \(\Int(A)\) contained in
    \(A\). That is
      \[
        \Int(A) = \bigcup \{U \subseteq X : U \subseteq A, U \text{ is open}\}.
    \]
  \item\label{def: exterior}
    The exterior of \(A\) in \(X\) is defined as
    \[
      \Ext(A) = X \setminus \overline A.
    \]
  \item\label{def: boundary}
    The boundary of \(A\) in \(X\) is defined as
    \[
      \partial A = X \setminus (\Int(A) \cup \Ext(A)).
    \]
\end{enumerate}
\end{definition}

\begin{proposition}\label{prop: closure equivalent prop}
Let \(A \subseteq X\) be a subset of the topological space \(X\). The
following propositions are equivalent
\begin{enumerate}[(i)]
  \item \(x \in \overline A\).
  \item Every neighbourhood \(U \subseteq X\) of \(x\) is such that \(U
    \cap A \neq \emptyset\).
  \item There exists a base \(\mathcal B_x\) at the point \(x\) (see
    \cref{def: basis at a point}) such that for all \(U \in \mathcal B_x\) we
    have \(U \cap A = \emptyset\).
\end{enumerate}
\end{proposition}

\begin{proof}
(i \(\Rightarrow\) ii) Let \(x \in X\) and suppose that ii is false for \(x\),
so that \(\exists U \subseteq X\) neighbourhood of \(x\) such that \(U \cap A
= \emptyset\). Then \(A \subseteq X \setminus U\), that is, \(A\) is a subset
of the complement of \(U\). Since \(U\) is open, then \(X \setminus U \in C_A
:= \{F \subseteq X : A \subseteq F,\ F \text{ is closed}\}\). From the
definition of closure, we have that \(\overline A \subseteq X \setminus U\)
and therefore \(x \not\in \overline A\). (ii \(\Rightarrow\) iii) From the
definition of a base at the point \(x\), we know that every \(U \in \mathcal
B_x\) is a neighbourhood of \(x\), hence if ii is true for \(x\), proposition
iii follows immediately. (\(\text{iii} \Rightarrow \text i\)) Let \(x \in X\)
such that \(x \not\in \overline A\), so that proposition i is false for \(x\).
From the definition of closure, there exists \(F \subseteq C_A\) such that \(x
\not\in F\). Consider the open complement \(V = X \setminus F\) so that \(x
\in V\) and \(V \cap A = \emptyset\). Hence, given any base at \(x\) there
exists a neighbourhood of \(x\), say \(U\), such that \(U \subseteq V\) and
hence \(U \cap A = \emptyset\), which implies that proposition iii is false
for \(x\).
\end{proof}

\begin{corollary}\label{cor: disjoint closure persistence}
If \(U\) is an open set and \(U \cap A = \emptyset\), then \(U \cap \overline
A = \emptyset\). Also, if \(U\) and \(V\) are disjoint sets, then \(U \cap
\overline V = \overline U \cap V = \emptyset\).
\end{corollary}

\begin{proof}
Suppose \(U \cap A = \emptyset\) and that there exists \(x \in U \cap
\overline A\), so that \(x \in \overline A\). Since \(U\) is open, it
is a neighbourhood of \(x\), hence from \cref{prop: closure equivalent prop}
we find that \(U \cap A \neq \emptyset\), which is false, thus \(U \cap
\overline A = \emptyset\).
\end{proof}

\begin{proposition}\label{prop: open set nbhd criterion}
Let be \(X\) a topological space and \(A \subseteq X\) be a subset. A point
\(x \in \Int(A)\) if and only if there exists a neighbourhood \(U \subseteq
X\) of \(x\) such that \(U \subseteq A\).
\end{proposition}

\begin{proof}
Let \(x \in A\) be any point. Suppose \(\exists U \subseteq X\) neighbourhood
of \(x\) such that \(U \subseteq A\), then from the fact that \(U\) is open we
conclude that \(U \subseteq \Int(A)\) from the definition of the interior
operator. Suppose \(x \in \Int A = \bigcup \{U \subseteq A : U \text{ is
open}\}\) then there exists \(U \subseteq A\) open such that \(x \in U\).
\end{proof}

\begin{proposition}
Let \(X\) be a topological space and \(A \subseteq X\). Then \(\overline{X
\setminus A} = X \setminus \Int(A)\) and also \(\Int(X \setminus A) =
X \setminus \overline A\).
\end{proposition}

\begin{proof}
We prove the first equality. Let \(\mathcal A := \{U \subseteq A : U \text{ is
open}\}\), from definition we have that \(\Int(A) = \bigcup_{U \in \mathcal A} U\),
moreover, \(X \setminus \Int(A) = X \setminus \bigcup_{U \in \mathcal A} U = \bigcap_{U \in \mathcal A} X \setminus U\)
from de Morgan's Laws. Notice that obviously \(X \setminus U \subseteq X\) and moreover since
\(U\) is open, then the complement \(X \setminus U\) is closed. This makes \(X \setminus \Int(A)
= \bigcap_{U \in \mathcal A} X \setminus U = \overline{X \setminus A}\). Now we show the second
equality. Define \(\mathcal{\widetilde A} := \{U \subseteq A : U \text{ is closed}\}\),
then \(\overline A = \bigcap_{U \in \mathcal{\widetilde A}} U\) and hence \(X \setminus
\overline A = X \setminus \bigcap_{U \in \mathcal{\widetilde A}} U = \bigcup_{U \in \widetilde{\mathcal{A}}} X \setminus
U\). Notice that \(X\setminus U \subseteq X \setminus A\) and since \(U\) is closed, then \(X \setminus U\) is
open. From this we conclude that \(X \setminus \overline A = \bigcup_{U \in \mathcal{\widetilde
A}} X \setminus U = \Int(X \setminus A)\).
\end{proof}

\begin{proposition}\label{prop: finite union of closures}
Let a finite collection of subsets \(\{A_i\}_{i = 1}^n \subseteq 2^X\), where
\(X\) is a topological space, then we have that
\[
  \overline{\bigcup_{i = 0}^n A_i} = \bigcup_{i = 0}^n \overline{A_i}.
\]
\end{proposition}

\begin{proof}
It suffices to prove that for \(A, B \subseteq X\) we have \(\overline{A \cup
B} = \overline A \cup \overline B\). Notice that \(\overline A,
\overline B \subseteq \overline{A \cup B}\), hence \(\overline A \cup
\overline B \subseteq \overline{A \cup B}\). Now, since \(A \subseteq
\overline A\) and \(B \subseteq \overline B\), we find \(A \cup B \subseteq
\overline A \cup \overline B\), then \(\overline{A \cup B} \subseteq
\overline{\overline A \cup \overline B} = \overline A \cup \overline B\).
\end{proof}

\begin{definition}[Locally finite family]
A collection of subsets \(\{A_i\}_{i \in I} \subseteq 2^X\) of a topological
space \(X\) is said to be a locally finite family if for every point \(x\)
there exists a neighbourhood \(U \subseteq X\) for which the collection \(\{i
\in I: U \cap A_i \neq \emptyset\}\) is finite.
\end{definition}

\begin{definition}[Discrete family]
A collection of subsets \(\{A_i\}_{i \in I} \subseteq 2^X\) of a topological
space \(X\) is said to be a discrete family if for all \(x \in X\) there
exists a neighbourhood \(U \subseteq X\) such that there exists at most one
\(A_i\) in the family such that \(U \cap A \neq \emptyset\). A discrete family
is also a locally finite family.
\end{definition}

\begin{proposition}
Given a locally finite family of sets \(\{A_i\}_{i \in I} \subseteq 2^X\),
where \(X\) is a topological space, we have that
\[
  \overline{\bigcup_{i \in  I} A_i} = \bigcup_{i \in  I} \overline{A_i}.
\]
\end{proposition}

\begin{proof}
Notice that clearly \(A_i \subseteq \overline{\bigcup_{i \in  I} A_i}\) for
all \(i \in I\), hence \(\bigcup_{i \in  I} \overline{A_i} \subseteq
\overline{\bigcup_{i \in I} A_i}\). Moreover, we have from hypothesis that
\(\{A_i\}_{i \in I}\) is a locally finite family, thus given \(x \in
\overline{\bigcup_{i \in  I} A_i}\) we can find a neighbourhood of \(x\), say
\(U \subset X\), such that \(I_0 := \{i \in I : U \cap A_i \neq \emptyset\}\)
is finite. Notice that from \cref{prop: closure equivalent prop} we have that
\(x\) cannot be a limit point of any of the sets with index \(i \in I
\setminus I_0\), hence \(x \not\in \overline{\bigcup_{i \in  I \setminus I_0}
A_i}\). On the other hand, we have \(x \in \overline{\bigcup_{i \in  I} A_i} =
\overline{\bigcup_{i \in  I_0} A_i} \cup \overline{\bigcup_{i \in  I \setminus
I_0} A_i}\), which implies that \(x \in \overline{\bigcup_{i \in  I_0} A_i} =
\bigcup_{i \in  I_0} \overline{A_i} \subseteq \bigcup_{i \in  I}
\overline{A_i}\) (the equality comes from the fact that \(I_0\) is finite and
hence \cref{prop: finite union of closures} hold).
\end{proof}

\begin{proposition}
If \(\{A_i\}_{i \in I}\) is a locally finite (resp. discrete) family, then the
family \(\{\overline{A_i}\}_{i \in I}\) is locally finite (resp. discrete).
\end{proposition}

\begin{proof}
Let the locally finite (resp. discrete) family \(\{A_i\}_{i \in I}\). Given
any element \(x \in X\) and a neighbourhood \(U\) of \(x\) such that the
indexing set \(I_0 := \{i \in I: U \cap A_i \neq \emptyset\}\) is finite
(resp. is either empty or a singleton). Notice that since \(U \cap A_i
\subseteq U \cap \overline{A_i}\), we find that \(\forall i \in I_0, U \cap
\overline{A_i} \neq \emptyset\), so that \(I_0 = I_0' := \{i \in I: U \cap
\overline{A_i} \neq \emptyset\}\). Since \(U\) is an open set, from \cref{cor:
disjoint closure persistence} we find that \(U \cap \overline{A_i} = U \cap
A_i = \emptyset\) for all \(i \in I \setminus I_0\). Which finishes the proof.
\end{proof}

\subsection{Dense, Derived, and Borel Sets}

\todo[inline]{Write}

\section{\texorpdfstring{\(\Ab\)}{Ab} --- Category of Abelian Groups}

\begin{definition}[Category of abelian groups]
We define the category of abelian groups, denoted \(\Ab\), to be the category
whose objects are abelian groups and group morphisms between them.
\end{definition}

In the following propositions we are going to describe ways of telling if the
group you might be interested is abelian or not.

\begin{proposition}
A group \(G\) is abelian if and only either one of the following conditions
are satisfied:
\begin{itemize}
  \setlength\itemsep{0em}
  \item The map \(\phi: G \to G\) sending \(g \mapsto g^{-1}\) is a morphism of
    groups.
  \item The map \(\psi: G \to G\) sending \(g \mapsto g^2\) is a morphism of
    groups.
\end{itemize}
\end{proposition}

\begin{proof}
If \(G\) is abelian, then for any \(g, g' \in G\) we have:
\begin{gather*}
  \phi(gg') = (gg')^{-1} = g'^{-1} g^{-1} = g^{-1}g'^{-1} = \phi(g) \phi(g'),
  \\
  \psi(gg') = (gg')^2 = (gg')(gg') = g^2g'^2 = \psi(g)\psi(g').
\end{gather*}
That is, \(\phi\) and \(\psi\) are morphisms. Now suppose \(\phi\) and
\(\psi\) are morphisms and consider any elements \(g, g' \in G\) --- we get
the following relations from each of the morphisms:
\begin{gather*}
  gg' = \phi(g^{-1})\phi(g'^{-1}) = \phi(g^{-1}g'^{-1})
  = \phi((g'g)^{-1}) = g'g,
  \\
  (gg')^2 = \phi(gg') = \phi(g) \phi(g') = g^2 g'^2.
\end{gather*}
Using cancellation law for the second relation we find \(g'g = gg'\). This
shows that \(G\) is abelian.
\end{proof}

\begin{proposition}
Let \(\phi \in \Hom_\Grp(G, \Aut_\Grp(G))\) be the morphism of groups mapping
\(g \xmapsto \phi \gamma_g\) --- as defined in \cref{lem: grp-inner-aut-cor}.
\(G\) is an abelian group if and only if \(\phi\) is trivial.
\end{proposition}

\begin{proof}
Let \(G\) be an abelian group, then for all \(g \in G\) the corresponding inner
automorphism \(\gamma_g\) maps
\(a \xmapsto{\gamma_g} g a g^{-1} = a (g g^{-1}) = a\) hence
\(\gamma_g = \Id_G\), thus \(\phi\) is indeed trivial. Let \(\phi\) be trivial,
then for all \(g \in G\) the map \(\gamma_g = \Id_G\) and therefore for all
\(a \in G\) we have \(g a g^{-1} = a\), which implies in
\(g a g^{-1} = a g g^{-1} = a g^{-1} g = g g^{-1} a = g^{-1} g a\) --- that is,
the group \(G\) is abelian.
\end{proof}

Now that we know some ways of identifying abelian groups, we dive deep again
into the categorical foundations of the category of abelian groups \(\Ab\).

\begin{proposition}[\(\Hom_\Ab\) abelian group]\label{prop: hom-ab-grp}
Let \(G, H \in \Ab\) be any two commutative groups. The collection of
morphisms \(\Hom_\Ab(G, H)\) forms an abelian group with a binary operation
defined by \((\phi + \psi)(g) = \phi(g) +_H \psi(g)\) --- where \(+_H\) is the
binary operation of \(H\).
\end{proposition}

\begin{proof}
Let \(\phi, \psi \in \Hom_\Ab(G, H)\) be any morphisms and consider elements
\(g, g' \in G\). From the commutativity of \(H\) it follows that
\begin{align*}
  (\phi + \psi)(g +_G g')
  &= \phi(g +_G g') +_H \psi(g +_G g')
  \\
  &= \left(\phi(g) +_H \phi(g')\right) +_H \left(\psi(g) +_H \psi(g')\right)
  \\
  &= \left(\phi(g) +_H \psi(g)\right) +_H \left(\phi(g') +_H \psi(g')\right)
  \\
  &= (\phi + \psi)(g) +_H (\phi + \psi)(g').
\end{align*}
That is, \(\phi + \psi \in \Hom_\Ab(G, H)\). Moreover, given a morphism
\(f \in \Hom_\Ab(G, H)\), define the map \(k: G \to H\) mapping
\(k(g) = f(g)^{-1}\). Notice that \(k\) is a morphism:
\begin{align*}
  k(g +_G g')
  &= f(g +_G g')^{-1} \\
  &= (f(g) +_H f(g'))^{-1} \\
  &= f(g')^{-1} +_H f(g)^{-1} \\
  &= f(g)^{-1} +_H f(g')^{-1} \\
  &= k(g) +_H k(g').
\end{align*}
Moreover, \((f + k)(g) = f(g) +_H k(g) = f(g) +_H f(g)^{-1} = e_H\) --- that
is, \(f + k\) is the trivial morphism \(g \mapsto e_H\), i.e. \(k\) is the
inverse of \(f\). This finishes the proof that \(\Hom_\Ab(G, H)\) is a group.
For the commutativity, it follows directly from the commutativity of \(H\):
for all \(\phi, \psi \in \Hom_\Ab(G, H)\) we have
\[
  (\phi + \psi)(g) = \phi(g) +_H \psi(g) = \psi(g) +_H \phi(g) = (\psi +
  \phi)(g).
\]
\end{proof}

The following corollaries follow immediately from the construction of the
category.

\begin{corollary}
Let \(G \in \Grp\) and \(H \in \Ab\). The collection of morphisms
\(\Hom_\Grp(G, H)\) forms a group under the binary operation defined above.
\end{corollary}

\begin{corollary}
Let \(A \in \Set\) and \(H \in \Ab\). The collection of morphisms
\(\Hom_\Set(A, FH)\) forms a group under the binary operation defined above
--- where \(F: \Grp \to \Set\) is a forgetful functor.
\end{corollary}

\begin{proposition}[Cokernel in \(\Ab\)]
\label{prop:Ab-has-cokernel}
The category of abelian groups has cokernels.
\end{proposition}

\begin{proof}
Let \(\phi: G \to H\) be a morphism of abelian groups and \(Q\) be any other
abelian group and \(\alpha: G \to Q\) be any morphism for which
\(\alpha \phi = 0\) --- the trivial morphism. Since \(G\) is abelian, it follows
that any subgroup of \(G\) is normal --- in particular, \(\im \phi\) is normal
in \(G\) and from \cref{cor:universal-property-quotients-grp} we see that
\[
\begin{tikzcd}
H \ar[r, "\phi"] \ar[rr, bend left, "0"]
&G \ar[r, "\alpha"] \ar[d, two heads, "\pi"]
&Q \\
&G/{\im \phi} \ar[ru, bend right, dashed]
\end{tikzcd}
\]
thus we conclude that
\[
G/{\im \phi} \iso \coker \phi.
\]
\end{proof}

A much more trivial proof of \cref{prop:epic-in-grp} can be achieved easily in
the category of abelian groups with the help of \cref{prop:Ab-has-cokernel} ---
for expository purposes we shall not use our results concerning the equivalence
of epimorphisms and surjection in the category of groups, as we proved above.

\begin{proposition}
\label{prop:epic-in-Ab}
Let \(\phi: G \to H\) be a morphism of \emph{abelian groups}. The following
propositions are equivalent in \(\Ab\):
\begin{enumerate}[(a)]\setlength\itemsep{0em}
\item The morphism \(\phi\) is an \emph{epimorphism}.

\item The \(\coker \phi\) is \emph{trivial}.

\item The set-function \(\phi\) is \emph{surjective}.
\end{enumerate}
\end{proposition}

\begin{proof}
\begin{itemize}\setlength\itemsep{0em}
\item (a) \(\implies\) (b): Let \(\phi\) be an epimorphism and consider both the
  canonical projection and the trivial morphism
  \(\pi, 0: G \rightrightarrows \coker \phi\). Notice that, from the definition
  of the cokernel, both maps are trivial when composed with \(\phi\), that is,
  \(\pi \phi = 0 \phi\) --- but since \(\phi\) is an epimorphism, then
  \(\pi = 0\) and such a thing can only occur when \(\coker \phi\) itself is
  trivial.

\item (b) \(\implies\) (c): Let \(\coker \phi\) be trivial, that is,
  \(G/{\im \phi} = \im \phi\), which implies that \(G = \im \phi\) and hence
  \(\phi\) is surjective.

\item (c) \(\implies\) (a): If \(\phi\) is surjective, then since \(\Ab\) is a
  concrete category, \(\phi\) is an epimorphism.
\end{itemize}
\end{proof}

\begin{proposition}[Coequalizers in \(\Ab\)]
\label{prop:coequalizer-Ab}
The category of abelian groups have coequalizers. Moreover, given group
morphisms \(f, g: A \para B\) between abelian groups, we have
\[
\Coeq(f, g) = B/\im(f - g).
\]
\end{proposition}

\begin{proof}
Given any abelian group \(G\) and a map \(m: B \to G\) such that \(m f = m g\),
we have that, for all \(a \in A\),
\[
m((f - g)(a)) = m(f(a) - g(a)) = m f(a) - m g(a) = 0
\]
therefore \(\im(f - g) \subseteq \ker m\). Thus using the universal property
\cref{cor:universal-property-quotients-grp} we obtain a unique morphism \(n:
B/\im(f - g) \to G\) such that the following diagram commutes
\[
\begin{tikzcd}
B/\im(f - g) \ar[d, dashed, "n"']
&B \ar[l, two heads, "\pi"] \ar[ld, bend left, "m"]
&A \ar[l, shift right, "f"'] \ar[l, shift left, "g"] \\
G & &
\end{tikzcd}
\]
Therefore \(B/\im(f - g) = \Coeq(f, g)\).
\end{proof}

\subsection{Coproduct and Fiber Product}

\begin{proposition}[Coproduct in \(\Ab\)]\label{prop: coprod-ab}
The direct product of abelian groups is a coproduct in \(\Ab\). That is, for any
\(G, H, W \in \Ab\) and morphisms \(f \in \Hom_\Ab(W, G)\) and
\(k \in \Hom_\Ab(W, H)\), there exists a unique morphism
\(\varphi \in \Hom_\Ab(W, G \times H)\) such that the following diagram commutes
\[
  \begin{tikzcd}
    G \ar[dr, "\iota_G"] \ar[ddr, bend right, swap, "f"] &
    &H \ar[dl, swap, "\iota_H"] \ar[ddl, bend left, "k"] \\
    &G \times H \ar[d, dashed, "\varphi"]  & \\
    &W &
  \end{tikzcd}
\]
Where we define inclusion morphisms \(g \xmapsto{\iota_G} (g, e_H)\) and \(h
\xmapsto{\iota_H} (e_G, h)\).
\end{proposition}

\begin{proof}
Since \(\Ab \subset \Grp\) --- that is, the category of abelian groups is a
subcategory of \(\Grp\) --- then, from \cref{prop: forgetful-func-grp-set}
there exists a functor \(\Ab \to \Set\). Since coproducts exists in \(\Set\)
and are unique, a set-function \(\varphi\) exists and is unique, commuting the
diagram in \(\Set\). We now show that we can extend such set-function into a
morphism of groups. Let \(\varphi: G \times H \to W\) be the mapping \((g, h)
\xmapsto \varphi f(g) k(h)\). Notice that
\begin{align*}
  \varphi((g, h)(g', h'))
  = \varphi(gg', hh')
  = f(g g') k(hh')
  = f(g) f(g') k(h) k(h')
  &= f(g) k(h) f(g') k(h') \\
  &= \varphi(g, h) \varphi(g', h')
\end{align*}
that is, \(\varphi\) is a morphism of groups.
\end{proof}

\begin{remark}[Coproducts in \(\Grp\)]\label{rem:coprod-grp}
\cref{prop: coprod-ab} is not at all true for the category of groups. Consider
for instance the cyclic groups \(C_2 = \{e_x, x\}\) and
\(C_3 = \{e_y, y, y^2\}\). Let \(\sigma_k \in S_3\), for \(0 \leq k \leq 2\) be
the rotation of \(\{1, 2, 3\}\) by \(k\), that is, the permutations represented
by
\[
  M_{\sigma_0} =
  \begin{bmatrix}
    1 &0 &0 \\ 0 &1 &0 \\ 0 &0 &1
  \end{bmatrix}
  \qquad
  M_{\sigma_1} =
  \begin{bmatrix}
    0 &0 &1 \\ 1 &0 &0 \\ 0 &1 &0
  \end{bmatrix}
  \qquad
  M_{\sigma_2} =
  \begin{bmatrix}
    0 &1 &0 \\ 0 &0 &1 \\ 1 &0 &0
  \end{bmatrix}
\]
Consider the embeddings \(f: C_2 \mono S_3\) mapping \(x^k \xmapsto f \sigma_k\)
for \(k \in \{0, 1\}\), and \(g: C_3 \mono S_3\) mapping
\(y^k \xmapsto g \sigma_k\) for \(k \in \{0, 1, 2\}\).

Suppose, for the sake of contradiction, that \(C_2 \times C_3\) is a coproduct
in \(\Grp\), that is, exists a unique morphism
\(\varphi: C_2 \times C_3 \to S_3\) such that \(f = \varphi \iota_{C_2}\) and
\(g = \varphi \iota_{C_3}\).  Since \(\varphi\) is supposedly a morphism of
groups,
\begin{gather*}
  \varphi(x, y) = \varphi(x, e_y) \varphi(e_x, y) = \sigma_1 \sigma_1 =
  \sigma_2 \\
  \varphi(x, y^2) = \varphi(x, e_y) \varphi(e_x, y^2) = \sigma_1 \sigma_2
  = \sigma_0
\end{gather*}
However, \(\varphi(e_x, e_y) = \sigma_0\) and on the other hand we have
\(\varphi(x, y) \varphi(x, y^2) = \sigma_2\), which contradicts the properties
of a group morphism. This shows that there exists no such \(\varphi\) in
\(\Grp\) and hence \(C_2 \times C_3\) is not a coproduct in \(\Grp\).

Although \(C_2 \times C_3\) is not a coproduct in \(\Grp\), that doesn't mean
that \(\Grp\) has no coproducts, they just behave differently when compared with
\(\Ab\). For instance, let \(C_2 * C_3 \in \Grp\) be defined to be the group
generated by elements \(x\) and \(y\), such that \(x^2 = e\) and \(y^3 =
e\). We'll now show that \(C_2 * C_3\) is a coproduct of \(C_2\) and \(C_3\) in
\(\Grp\). Let \(G\) be any group and consider morphisms \(f: C_2 \to G\) and
\(k: C_3 \to G\). The inclusions \(\iota_{C_2}: C_2 \to C_2 * C_3\) and
\(\iota_{C_3}: C_3 \to C_2 * C_3\) will be naturally given maps by taking each
element to itself.

Let \(q \in C_2 * C_3\) be any element. We know that there exists a finite
collection of coefficients \(I = \{(a, b) \in \Z^2\}\) such that
\(q = \prod_{(a, b) \in I} x^a y^b\). Define \(\phi: C_2 * C_3 \to G\) as the
mapping
\[
  \phi(q) = \phi\left(\prod_{(a, b) \in I} x^a y^b\right)
  = \prod_{(a, b) \in I} f(x^a) k(y^b)
  = \prod_{(a, b) \in I} f(x)^a k(y)^b
\]
It should be clear that this definition implies \(\phi \iota_{C_2} = f\) and
\(\phi \iota_{C_3} = g\). Notice that \(\phi(e) = f(e) k(e) = e_G\) and for all
\(q, p \in C_2 * C_3\) --- with respective coefficients
\(I = \{(a, b) \in \Z^2\}\) and \(J = \{(c,d) \in \Z^2\}\) --- we have
\begin{align*}
  \phi(q) \phi(p)
  = \phi\left[ \prod_{(a, b) \in I} x^a y^b \right]
  \phi\left[ \prod_{(c, d) \in J} x^c y^d \right]
  &= \prod_{(a, b) \in I} f(x^a) k(y^b)
  \prod_{(c, d) \in J} f(x^c) k(y^d)
  \\
  &= \prod_{(\alpha, \beta) \in A} f(x^\alpha) g(y^\beta)
  \\
  &= \phi\left(\prod_{(\alpha, \beta) \in A} f(x^\alpha) g(y^\beta) \right)
  \\
  &= \phi\left(\prod_{(a, b) \in I} x^a y^b
    \prod_{(c, d) \in J} x^c y^d\right)
  = \phi(qp)
\end{align*}
that is, \(\phi\) is a morphism of groups --- where we define \(A\) as the
concatenation of the coefficients \(I\) and \(J\). We have shown that the
following diagram commutes
\[
  \begin{tikzcd}
    C_2 \ar[dr, "\iota_{C_2}"] \ar[ddr, swap, bend right, "f"]
    & &C_3 \ar[dl, swap, "\iota_{C_3}"] \ar[ddl, bend left, "g"] \\
    &C_2 * C_3 \ar[d, dashed, "\phi"] & \\
    &G &
  \end{tikzcd}
\]
We can then conclude that \(C_2 * C_3\), as defined above, is the coproduct of
\(C_2\) and \(C_3\) in \(\Grp\).
\end{remark}

\begin{proposition}[Fiber products]
Fiber products exist in \(\Ab\). That is, given abelian groups
\(G, H, W \in \Ab\) and group morphisms \(\phi \in \Hom_\Ab(G, W)\) and
\(\psi \in \Hom_\Ab(H, W)\). Let
\[
  G \times_W H \in \Ab_W
\]
be the fiber product of \(\phi\) and \(\psi\) in the category
\(\Ab_W \subseteq \Ab\)\footnote{As a reminder: \(\Ab_W\) is the category whose
  objects are morphisms \(f \in \Hom_\Ab(-, W)\). If \(f: G \to W\) and
  \(g: H \to W\) are objects of \(\Ab_W\), a morphism
  \(h \in \Hom_{\Ab_W}(f, g)\) is such that \(hg = f\).}
for which exists natural projections \(\pi_G\) and \(\pi_H\). Let \(Q \in \Ab\)
be any abelian group and consider any morphisms \(f \in \Hom_\Ab(Q, G)\) and
\(k \in \Hom_\Ab(Q, H)\). Then there exists a unique morphism
\(\varphi \in \Hom_\Ab(Q, P)\) such that the following diagram commutes
\[
  \begin{tikzcd}
    Q \ar[dr, dashed, "\varphi"] \ar[ddr, bend right, swap, "f"]
    \ar[drr, bend left, "k"]
    & & \\
    &G \times_W H \ar[r, "\pi_H"] \ar[d, swap, "\pi_G"] &H \ar[d, "\psi"] \\
    &G \ar[r, swap, "\phi"] &W
  \end{tikzcd}
\]
\end{proposition}

\begin{proof}
Define \(G \times_W H = \{(g, h) \in G \times H \colon \phi(g) =
\psi(h)\}\). Since \(\phi\) and \(\psi\) are group morphisms, given
\((g, h) \in G \times_W H\), we have
\[
  \phi(g^{-1}) = \phi(g)^{-1} = \psi(h)^{-1} = \psi(h^{-1}),
\]
that is, \((g^{-1}, h^{-1}) \in G \times_W H\) exists and is the inverse of
\((g, h)\). Moreover, given any elements \((g, h), (g', h') \in G \times_W H\),
the product \((g, h)(g', h') = (gg', hh')\) is such that
\[
  \phi(gg') = \phi(g) \phi(g') = \psi(h) \psi(h') = \psi(hh'),
\]
thus \((gg', hh') \in G \times_W H\). This shows that \(G \times_W H\) is
indeed a group and since \(G \times H\) is abelian, so is the fiber product
defined above.

From the forgetful functor \(\Ab \to \Set\) we know that there exists a unique
set-function \(\varphi\) such that the diagram commutes in \(\Set\). If we
define \(\varphi\) as the mapping \(q \overset \varphi \longmapsto (f(q),
k(q))\) we can see that it preserves the group structure, since
\begin{align*}
  \varphi(qq')
  &= (f(qq'), k(qq')) \\
  &= (f(q)f(q'), k(q)k(q')) \\
  &= (f(q), k(q))(f(q'), k(q')) \\
  &= \varphi(q) \varphi(q'),
\end{align*}
that is, \(\varphi \in \Mor(\Ab)\).
\end{proof}

\subsection{Direct Sums and Free Groups}

\begin{definition}[Direct sum of abelian groups]
\label{def:Ab-direct-sum}
Let \(\{G_j\}_{j \in J}\) be an indexed collection of abelian groups. We define
their direct sum as the collection of tuples \((g_j)_{j \in J}\) such that
\(g_j \neq e_{G_j}\) for only finitely many indexes \(j \in J\) --- that is, set
of tuples with finite support. We denote the direct sum of
\(\{G_{j}\}_{j \in J}\) as \(\bigoplus_{j \in J} G_j\) and the group structure
of the direct sum is defined naturally as
\((x_j)_{j \in J} + (y_j)_{j \in J} = (x_j +_{G_{j}} y_j)_{j \in J}\).
\end{definition}

\begin{proposition}[Direct sum universal property]
\label{prop:Ab-direct-sum-universal-property}
The direct sum defined in \cref{def:Ab-direct-sum} satisfies the universal
property of direct sums. In other words, let \(\{G_{j}\}_{j \in J}\) be a
collection of abelian groups and \(H\) be any abelian group, in addition,
consider the collection of morphisms \(\{\phi_j \in \Hom_{\Grp}(G_j,
H)\}\). There exists a unique morphism \(\psi: \bigoplus_{j \in J} G_j \to H\)
such that the following diagram commutes
\[
  \begin{tikzcd}
    G_j \ar[rd, bend left, "\phi_j"] \ar[d, tail, "\iota"] &\\
    \bigoplus_{j \in J} G_j \ar[r, dashed, "\psi"] &H
  \end{tikzcd}
\]
for every \(j \in J\) --- where \(\iota_{j}: G_j \to \bigoplus_{j \in J} G_j\)
is the natural inclusion, mapping
\(x_{j_0} \xmapsto{\iota_{j}} (x_j)_{j \in J}\) such that \(x_j = x_{j_0}\) for
\(j = j_0\) and \(x_j = e_{G_j}\) for \(j \neq j_0\).
\end{proposition}

\begin{proof}
Let \(\psi: \bigoplus_{j \in J} G_j \to H\) be the map defined by
\(\psi(x) = \prod_{j \in J} \phi_{j}(x_j)\), where
\(x = (x_j)_{j \in J} \in \bigoplus_{j \in J} G_j\) is any element. Notice that
\(\psi(x)\) is therefore a finite product of elements of \(H\) and from the
group structure of the direct sum, we find that \(\psi\) is clearly a group
morphism. Moreover, \(\psi\) satisfies the commutativity
\(\psi \iota_j = \phi_j\) for each \(j \in J\). Since such definition of
\(\psi\) defines the image of every element of its domain, \(\psi\) is the
unique morphism making the diagram commute.
\end{proof}

If \(G\) is an abelian group and \(A, B \subseteq G\) are subgroups such that
\(A \cap B = 0\) and \(A + B = G\) --- that is, every \(g \in G\) can be
written as \(g = a + b\) for some \(a \in A\) and \(b \in B\) --- then, in the
context of abelian groups, we'll denote this fact shortly by \(G = A \oplus B\).

Just like a vector space, we can define a basis of an abelian group by means of
the ring \(\Z\). Moreover, if an abelian group has a basis, then we say that it
is free.

\begin{definition}[Basis]
\label{def:Ab-basis}
Let \(G\) be an abelian group. We say that a non-empty collection
\(\{g_j\}_{j \in J}\) is a basis for \(G\) if, given an element \(g \in G\),
there exists a unique tuple of coefficients
\((a_j)_{j \in J} \in \bigoplus_{j \in J} \Z\) such that
\(g = \sum_{j \in J} a_j g_j\).
\end{definition}

Therefore the existence of a basis allow us to couple the coefficients coming
from \(\bigoplus_{j \in J} \Z\) to any element of \(G\) in a unique way, which
induces a natural isomorphism
\[
  G \iso \bigoplus_{j \in J} \Z.
\]

\begin{definition}[Abelian free group]\label{def:Ab-free}
An abelian group is said to be free if it allows a basis.
\end{definition}

Equivalently as we did with free vector spaces, we can build free abelian groups
out of sets, say \(S\), by analysing maps \(S \to \Z\) with finite support ---
the collection of those will be likewise denoted by \(\Z^{\oplus S}\), which is
a group under pointwise addition. For each \(s \in S\) we define a map
\(\mathbf{s} \in \Z^{\oplus}\) by the mapping
\[
  \mathbf{s}(x) \coloneq
  \begin{cases}
    1, &\text{for } x = s \\
    0, &\text{otherwise}
  \end{cases}
\]
Then, any element \(\phi \in \Z^{\oplus S}\) can be written as a linear
combination of finitely many maps \(\mathbf{s}\) (which is possible because of
the finite support of \(\phi\)) that is, for some \(a_j \in \Z\) for each
\(1 \leq j \leq n\), we have
\[
  \phi = \sum_{j=1}^n a_j \mathbf{s}_j,\ \text{ mapping }\
  s \xmapsto \phi
  \begin{cases}
    a_j, &\text{ if } s = s_j \text{ for some } 1 \leq j \leq n \\
    0, &\text{otherwise}
  \end{cases}
\]
Moreover, such choice of coefficients \(a_j \in \Z\) is unique. Let
\(\{b_{j}\}_{j=1}^n \subseteq \Z\) be another set of coefficients with the same
property, then in particular \(\sum_{j=1}^n (a_j - b_j) \mathbf{s}_j = 0\) and
since \(\mathbf{s}_j\) are all non-zero maps, we find that \(b_j = a_j\). We
also define a map \(\iota_S: S \mono \Z^{\oplus S}\) by pairing
\(s \mapsto \mathbf{s}\).

\begin{proposition}[Free \(\Ab\) universal property]
\label{prop:free-abelian-group-universal-property}
Let \(S\) be a set. Consider \(G\) to be any abelian group and a set-function
\(g: S \to G\), then there exists a unique morphism of groups
\(g_{*}: \Z^{\oplus S} \to G\) such that
\[
  \begin{tikzcd}
    S \ar[d, hook, "\iota_S"] \ar[r, "g"] &G \\
    \Z^{\oplus S} \ar[ru, bend right, dashed, "g_{*}"']  &
  \end{tikzcd}
\]
is a commutative diagram.
\end{proposition}

\begin{proof}
Define \(g_{*}\) by
\(g_{*}(\sum_{s \in S} a_s \mathbf{s}) \coloneq \sum_{s \in S} a_s g(s)\). Then
\(g_{*}\iota_S(s) = g_{*}(\mathbf{s}) = g(s)\). Moreover, it follows from the
construction that
\(g_{*}(\sum_{s \in S} a_s \mathbf{s}) = \sum_{s \in S} a_s g_{*}(\mathbf{s})\)
thus \(g_{*}\) is a group morphism. If we let \(f: \Z^{\oplus S} \to G\) be a
morphism satisfying such commutativity, we'll see that, since
\(f_{*}(\mathbf{s}) = g(s) = g_{*}(\mathbf{s})\), since any element of
\(\Z^{\oplus S}\) can be written uniquely as a linear combination of the basis
\(\{\mathbf{s}\}_{s \in S}\), then \(f_{*}\) and \(g_{*}\) agree in every point
of the domain --- hence \(f_{*} = g_{*}\).
\end{proof}

\begin{corollary}
Let \(f: X \to Y\) be a set-function between sets \(X\) and \(Y\). Then, there
exists a unique morphism \(\overline{f}: \Z^{\oplus X} \to \Z^{\oplus Y}\) such
that the following diagram commutes
\[
  \begin{tikzcd}
    X \ar[d, "f"'] \ar[r, hook, "\iota_X"]
    &\Z^{\oplus X} \ar[d, dashed, swap, "\overline f"] \\
    Y \ar[r, hook, "\iota_Y"] &\Z^{\oplus Y}
  \end{tikzcd}
\]
\end{corollary}

\begin{proof}
Let \(\overline f\) be defined by
\(\overline f(\sum a_x \mathbf{x}) \coloneq \sum a_x \iota_Y f(x)\) --- so that,
in particular, \(\overline f(\mathbf{x}) = \iota_Y f(x)\), that is, the diagram
commutes. Let \(\overline g: \Z^{\oplus X} \to \Z^{\oplus Y}\) be a morphism
such that the diagram commutes, then necessarily
\(\overline g(\mathbf{x}) = \iota_Y f(x) = \overline f(\mathbf{x})\), that is,
\(\overline g\) and \(\overline f\) agree on the basis of \(\Z^{\oplus X}\),
thus \(\overline g = \overline f\).
\end{proof}

\begin{notation}
When it's not confusing, we can even drop the notation \(\mathbf{s}\) and
instead identify the elements as \(s \in \Z^{\oplus S}\), so that
\(\sum a_s s \coloneq \sum a_s \mathbf{s}\). Moreover, for now on, we'll refer
to the \emph{free abelian group generated by} \(S\), that is, \(\Z^{\oplus S}\),
as \(F_{\Ab}(S)\) --- this is motivated by the fact that the notation
\(\Z^{\oplus S}\), although cool, may be a rather obscure way of talking about a
group. The elements of \(S\), in particular, will be referred to as \emph{free
  generators}.
\end{notation}

\begin{proposition}
\label{prop:Ab-isomorphic-factor-of-free}
Let \(G\) be any abelian group
\todo[inline]{Continue: mini exercises on free ab grp}
\end{proposition}

\todo[inline]{Continue direct products}

%%% Local Variables:
%%% mode: latex
%%% TeX-master: "../../deep-dive"
%%% End:

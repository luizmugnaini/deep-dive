\section{Hausdorff Spaces}

\begin{definition}
Let \(X\) be a topological space. We define
\begin{enumerate}[(1)]
\item\label{def:T0-space-kolmogorov} \(X\) is T\(_0\) (or Kolmogorov) if and
  only if for every pair of distinct points \(x, y \in X\) there exists an open
  set containing one, but not both, of them.
\item\label{def:T1-space-frechet} \(X\) is T\(_1\) (or Fréchet) if and only if
  for every pair of distinct points \(x, y \in X\) there exists open sets
  \(U, V \subseteq X\) such that \(x \in U\) and \(y \in V\), but
  \(x \not\in V\) and \(y \not\in U\)
\end{enumerate}
\end{definition}

\begin{proposition}
A space \(X\) is T\(_1\) if and only if the constant sequence \((x)_{i \in
\N}\) converges to \(x\) and only to \(x\).
\end{proposition}

\begin{proof}
(\(\Rightarrow\)) Suppose \(X\) is T\(_1\), then certainly \(x \to x\). Let
\(y \in X\) be distinct of \(x\), then by the T\(_1\) property, there exists a
neighbourhood \(U \ni y\) such that \(x \not\in U\), hence \((x)_{i \in
\N}\) does not converge to \(y\). (\(\Leftarrow\)) Suppose \(X\) is
not T\(_1\), then choose \(y \in X\) for which every neighbourhood \(U \ni y\)
contains \(x\), then \(x \to y\).
\end{proof}

\begin{definition}[Hausdorff space]\label{def: Hausdorff space}
We say that a topological space \(X\) is Hausdorff (or T\(_2\)) if for all
pair of distinct points \(x, y \in X\), there exists neighbourhoods \(U_x\)
and \(U_y\), subsets of \(X\), such that \(U_x \cap U_y = \emptyset\).
\end{definition}

\begin{corollary}
Every open subset of a Hausdorff space is Hausdorff.
\end{corollary}

\begin{proof}
Let \(X\) be Hausdorff and \(U \subseteq X\) be an open. Choose any
distinct pair \(x, y \in U\) and let \(U_x, U_y \subseteq X\) be neighbourhoods
of \(x, y\) such that \(U_x \cap U_y = \emptyset\). Since \(U_x \cap U, U_y
\cap U \subseteq U\) are neighbourhoods of \(x\) and \(y\), respectively, then
\((U_x \cap U) \cap (U_y \cap U) = \emptyset\), hence \(U\) is Hausdorff.
\end{proof}

\begin{proposition}[Hausdorff properties]
\label{prop: Hausdorff properties}
Let \(X\) be a Hausdorff space. Then
\begin{enumerate}[(a)]
  \item Every finite subset of \(X\) is closed.
  \item If a sequence \((x_i) \subseteq X\) converges to \(x \in X\), then the
    limit is unique.
\end{enumerate}
\end{proposition}

\begin{proof}
(a) Let \(p_0 \in X\) be any point and \(p \in X\) be a distinct point. Since
\(X\) is Hausdorff, choose \(U_p, U_{p_0} \subseteq X\) such that \(U_p \cap
U_{p_0} = \emptyset\), hence  \(U_p \subseteq X \setminus \{p_0\}\). Since for all
\(x \in X \setminus \{p_0\}\) there exists a neighbourhood \(U_x
\subseteq X \setminus \{p\}\), hence \(\{p\}\) must be a closed set (every
point of \(X \setminus A\) has a neighbourhood in \(X \setminus A\) is
equivalent of saying that \(A\) is closed).

(b) Suppose that \(x_i \to x\) and \(x_i \to y\) are two limits of the
sequence. Since \(X\) is Hausdorff, if \(x \neq y\) implies that there exists
neighbourhoods \(U_x, U_y \subseteq X\) for which \(U_x \cap U_y =
\emptyset\). From the definition of convergence, there exists \(N \in
\N\) such that for all \(i \geq N, x_i \in U_x\) and exists \(M \in \N\)
for which every \(i \geq M\), we have \(x_i \in U_y\). Then, for \(i \geq
\max(N, M), x_i \in U_x \cap U_y\), which cannot happen if \(x \neq y\), hence
\(x = y\).
\end{proof}

\begin{proposition}[Neighbourhoods of limit points]
\label{prop:nbhd-limit-pt}
Let \(X\) be a Hausdorff space and \(A \subseteq X\). If \(p \in X\) is a
limit point of \(A\), then every neighbourhood of \(p\) contains infinitely
many points of \(A\).
\end{proposition}

\begin{proof}
Let \(p\) be a limit point of \(A\), then for any neighbourhood \(U_p
\subseteq X\) we must have \((U_p \setminus \{p\}) \cap A \neq \emptyset\).
For the sake of contradiction, suppose that there exists a finite number of
points, \(n > 1\), in the set \(U_p \cap A = \{x_i\}_{i=1}^n\), which implies
that \(U_p \cap A\) is closed. Consider now the non-empty closed set
\(\{x_i\}_{i=1}^n \setminus \{p\}\), then \(X \setminus (\{x_i\}_{i=1}^n
\setminus \{p\})\) is open and contains \(p\), thus is a neighbourhood of
\(p\). Notice that since the intersection of finitely many open sets is open,
then \(U_p \cap \left( X \setminus \left( \{x_i\}_{i=1}^n \setminus \{p\}
\right) \right)\) is also open and is a neighbourhood of \(p\). However, notice
that
\[
  A \cap U_p \cap (X \setminus (\{x_i\}_{i=1}^n \setminus \{p\}))
  = \{x_i\}_{i=1}^n \cap ((X \setminus \{x_i\}_{i=1}^n) \cup \{p\})
  = \{p\}
\]
which is a contradiction to the fact that \(p\) is a limit point of \(A\)
(because any neighbourhood of \(p\) should also contain a point of \(A\) other
than \(p\)). Hence we conclude that \(A \cap U_p\) must be infinite.
\end{proof}

\begin{proposition}
Let \(X\) be a Hausdorff space and \(A \subseteq X\). Then the set of limit
points of \(A\), denoted by \(A'\), is closed in \(X\).
\end{proposition}

\begin{proof}
Consider the complement set \(X \setminus A'\), we must show that it is open.
Consider \(x \in X \setminus A'\), so that \(x\) is not a limit point of \(A\)
and hence there exists \(U \subseteq X\) neighbourhood of \(x\) such that \((U
\setminus \{x\}) \cap A = \emptyset\). We now show that \(U \subseteq X
\setminus A'\). Let \(p \in U\) be any point, then the set \(U \setminus
\{x\}\) is a neighbourhood of \(p\) and is disjoint with \(A\), hence \(p\) is
not a limit point of \(A\), that is, \(p \not\in A'\), which proves that \(U
\subseteq X \setminus A'\). Moreover, since \(X\) is Hausdorff, the singleton
\(\{x\}\) is closed, hence \(U \setminus \{x\}\) is open
\end{proof}

\begin{proposition}
Let \(f,g : X \to Y\) be morphisms of topological spaces, and \(Y\) be a
Hausdorff space. Then the set \(\{x \in X \colon f(x) = g(x)\}\) is closed in
\(X\).
\end{proposition}

\begin{proof}
Consider the complement set \(A \coloneq X \setminus \{x \in X \colon f(x) = g(x)\}\).
Let any point \(x \in A\), since \(f(x) \neq g(x)\) and \(Y\) is Hausdorff,
there exists neighbourhoods \(U_1, U_2 \subseteq Y\) of \(f(x), g(x)\),
respectively, such that \(U_1 \cap U_2 = \emptyset\). Since \(U_1, U_2\) are
open, then \(f^{-1}(U_1), g^{-1}(U_2)\) are both open, hence \(f^{-1}(U_1)
\cap g^{-1}(U_2)\) is a neighbourhood of \(x\) such that \(f^{-1}(U_1) \cap
g^{-1}(U_2) \subseteq A\). From \cref{prop:classification-int-ext-boundary-points} we
find that \(A\) is open.
\end{proof}

\begin{proposition}\label{prop: metric space T2}
Every metric space is Hausdorff.
\end{proposition}

\begin{proof}
Let \((M, d)\) be a metric space and \(x, y \in M\) any distinct points. Let
\(r \coloneq d(x, y)\). The open balls \(B_{r/2}(x)\) and \(B_{r/2}(y)\) are
disjoint, thus \(M\) is Hausdorff.
\end{proof}

\begin{proposition}\label{prop: order top implies T2}
Every totally ordered set endowed with the order topology is a Hausdorff
space.
\end{proposition}

\begin{proof}
Let \(X\) be a space endowed with the order topology. Let \(x, y \in X\) be
distinct points. Suppose that \(x < y\). If there exists a point \(z \in X\)
such that  \(x < z < y\) then the neighbourhoods of \(x\) and \(y\),
respectively, \(U = \{u \in X \colon u < z\}\) and \(V = \{u \in X \colon u > z\}\) are
disjoint, that is \(U \cap V = \emptyset\). Suppose there is no such middle
element, then \(G = \{u \in X \colon u < y\} = \{u \in X \colon u \leq x\}\) and \(H =
\{u \in X \colon u > x\} = \{u \in X \colon u \geq y\}\) are neighbourhoods of \(x\) and
\(y\), respectively, and moreover \(G \cap V = \emptyset\). This shows that
\(X\) is Hausdorff.
\end{proof}

%%% Local Variables:
%%% TeX-master: "../../deep-dive"
%%% End:
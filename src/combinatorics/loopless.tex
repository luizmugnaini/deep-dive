\section{The Simple Loopless Graph}

Sometimes (and by that I mean almost always) graph theorists like the incidence
relation to be irreflexive---loops are therefore forbidden, that is,
\(\Edge(x, x)\) is always false for any \(x \in V\)---but this construction does
not get us a good category. We are going to reserve the word ``simple graph'' to
mean a simple graph with loops and the word ``graph'' for simple graphs with no
loops. The following is a formal definition of a loopless simple graph---a
``graph''.

\begin{definition}[Graph]\label{def: graph}
\(G = (V, E, d)\) is said to be a graph if
\[
  d: E \to \langle V \rangle^2 \text{, mapping } e \xmapsto d (x, y) = (y, x)
  \text{ and } x \neq y.
\]
That is, \(\Edge(x, y)\) if and only if \(\Edge(y, x)\), and \(\Edge(x, x)\) is
always false.
\end{definition}

From now on we are mostly going to assume the existence of the incidence
relation map \(d\) (symmetric and irreflexive) and simply denote a graph \(G\)
by \((V, E)\)---where we assume that \(G\) is a simple loopless graph.

\begin{definition}[Morphism of graphs]\label{def: graph-morph}
Let \(G = (V, E)\) and \(H = (V', E')\) be graphs. A morphism \(G \to H\) is a
map \(\varphi: V \to V'\) such that \(\Edge(x, y)\) implies
\(\Edge(\varphi(x), \varphi(y))\)---that is, adjacency of vertices need to be
preserved.
\end{definition}

\begin{definition}[Graph category]\label{def: graph-cat}
We define the category of simple loopless graphs, denoted by \(\Graph\), as
the category with graph objects and morphisms of graphs.
\end{definition}

\begin{lemma}[Stable preimage]\label{lem: stable-preimage}
Let \(\varphi: G \to H\) be a morphism of graphs, then, for any \(x' \in
\Vertex(H)\), the collection \(\varphi^{-1}(x') \subseteq \Vertex(G)\) is stable in \(G\).
\end{lemma}

\begin{proof}
Suppose, for the sake of contradiction, that there exists vertices \(x, y \in
\varphi^{-1}(x)\) such that \(\Edge(x, y)\) is true in \(G\). In particular, this
implies that \(\Edge(\varphi(x), \varphi(y)) = \Edge(x', x')\) is true in \(H\), which
is forbidden---hence there can be no such pair of vertices in
\(\varphi^{-1}(x')\).
\end{proof}

\begin{definition}[Graph property]\label{def: graph-property}
A class of graphs is said to be a property of graphs if it is closed up to
isomorphism.
\end{definition}

\begin{definition}[Graph invariant]\label{def: graph-invariant}
A map \(\phi: \Graph \to S\)---where \(S \in \Set\), possibly \(\R\) for
instance---is said to be a graph invariant if for all \(G, G' \in
\Graph\) such that \(G \iso G'\) then
\[
  \phi(G) = \phi(G').
\]
\end{definition}

\begin{definition}[Union and intersections]\label{def: union-intersection-gph}
Let \(G = (V, E), G' = (V', E') \in \Graph\). We define their union as the
graph
\[
  G \cup G' = (V \cup V', E \cup E').
\]
Analogously, their intersection is defined as
\[
  G \cap G' = (V \cap V', E \cap E').
\]
\end{definition}

\begin{definition}[Complete Graph]\label{def: complete-graph}
A graph \(G\) is said to be complete if for all pairs of vertices \(x, y \in
G\), \(\Edge(x, y)\) is true.
\end{definition}

\begin{notation}
We denote by \(K^n\) the complete graph on \(n\) vertices.
\end{notation}

\begin{definition}[Subgraph]
Let \(G = (V, E)\) and \(G' = (V', E')\) be graphs. \(G'\) is said to be a
\emph{subgraph} of \(G\)---denoted \(G' \subseteq G\)---if \(V' \subseteq
V\) and \(E' \subseteq E\). On the other hand, \(G\) is said to be a
\emph{supergraph} of \(G'\).
\end{definition}

\begin{definition}[Induced subgraph]
Let \(G' = (V', E')\) be a subgraph of \(G\). \(G'\) is said to be an
\emph{induced subgraph} of \(G\) if for all \(x, y \in V\), then \(\Edge_{G'}(x,
y)\) if \(\Edge_G(x, y)\). We denote \(G' = G[V']\). If \(G[V'] = G\), then we say
that \(G'\) is a \emph{spanning subgraph} of \(G\).
\end{definition}

\begin{definition}[Embedding of graphs]\label{def: graph-embedding}
Let \(G\) and \(H\) be graphs. A morphism of graphs \(\iota: H \emb G\) is
said to be an embedding of \(H\) into \(G\) if its underlying map \(\Vertex(H) \mono
\Vertex(G)\) is injective.
\end{definition}

\begin{definition}[Deletion vertices or edges]
\label{def: deletion-graph}
Let \(G = (V, E)\) be a graph and \(V' \subseteq V\) a subset of the
vertices of \(G\). The deletion of the parts of \(G\) associated with the
vertices \(V'\) is defined as the graph \(G[V \setminus V']\)---which we'll
shortly identify as \(G \setminus V'\). Equivalently we define the deletion
of edges \(E' \subseteq E\) as the graph \(G - E' \coloneq (V, E \setminus E')\).
\end{definition}

Notice that the non-connected vertices are not deleted from the resulting graph
\(G \setminus E'\), contrary to the case of the deletion of vertices \(G
\setminus V'\), where edges with endpoints out of \(V'\) where removed.

\begin{definition}[Edge-maximal]
\label{def: edge-maximal}
Given a graph property \(P\), a graph \(G = (V, E)\) is said to be
\emph{edge-maximal} with respect to \(P\) if \(G \in P\) and for all \(E'
\supsetneq E\) the graph \((V, E')\) does not have the property \(P\).
\end{definition}

\begin{definition}[Clique]\label{def: clique}
Let \(G\) be a graph and \(S \subseteq \Vertex(G)\) be a subset with \(k\) vertices.
If the graph \(G[S]\) is complete, we call it a \(k\)-clique. Trivially,
\(G[S] \iso K_k\).
\end{definition}

\begin{definition}[Complement graph]\label{def: complement-graph}
Given a graph \(G = (V, E)\), the complement of \(G\) is defined as the graph
\(\overline G = (V, \langle V \rangle^2 \setminus E)\).
\end{definition}

\begin{definition}[Line graph]\label{def: line-graph}
Given a graph \(G\), the line graph of \(G\), denoted by \(L(G)\), is the
graph such that, if \(e = (x, y), e' = (y, z) \in \Edge(G)\) are adjacent edges
then \(\Edge_{L(G)}(x, z)\)---that is \(L(G) = (\Vertex(G), E)\) where
\[
  E = \{(x, z) \in \langle V \rangle^2 \colon \Edge_G(x, y) \text{ and } \Edge_G(y, z)
  \text{ for some } y \in \Vertex(G)\}.
\]
\end{definition}

\begin{definition}[Join]\label{def: graph-join}
Let \(G = (V, E), G' = (V', E') \in \Graph\) be disjoint---\(G \cap G' =
\emptyset\). We define their join as the graph \(G * G'\) with vertices \(V
\cup V'\) and edges \(E \cup E' \cup E^*\), where \(E^*\) is defined as \(E^*
= \{(x, y) \colon x \in G, y \in G'\}\)---that is, the collection of all edges
connecting the vertices of \(G\) and \(G'\).
\end{definition}

The join definition fails to be a coproduct in the category of graphs exactly
because of the addition of the \(E^*\) edges---the edges between elements of
\(G\) with \(G'\) in \({G * G'}\) cannot be ensured to be preserved by a morphism
\(G * G' \to H\) as described in \cref{prop: coprod-graph}.

\subsection{Vertex Degree}

\begin{notation}[Neighbourhood]
Given a graph \(G\), we denote by \(N(v)\) the collection of all neighbours of
\(v \in G\).
\end{notation}

\begin{definition}[Degree]\label{def: degree}
Let \(G\) be a graph, we define the degree (or valency) of a vertex \(v \in
G\) as \(\deg v = |N(v)|\). If \(G\) is a finite graph,
\begin{itemize}
  \setlength\itemsep{0em}
  \item The minimum degree of \(G\) is defined as \(\delta(G) = \min_{v \in
    \Vertex(G)} \deg v\).
  \item The maximum degree of \(G\) is defined as \(\Delta(G) = \max_{v \in
    \Vertex(G)} \deg v\).
  \item The average degree of \(G\) is defined as
    \[
      \deg G = \frac 1 {|G|} \sum_{v \in \Vertex(G)} \deg v.
    \]
  \item The edge-vertex ratio of \(G\) is \(\epsilon(G) =
    \frac{\size{G}}{|G|}\). Since an edge is composed of two vertices, the sum
    \(\sum_{v \in \Vertex(G)} \deg v\) counts every edge exactly twice, so that
    \(\epsilon(G) = \frac 1 2 \sum_{v \in \Vertex(G)} \deg v\). Thus we arrive in
    the relation \(\epsilon(G) = \frac 1 2 \deg G\).
\end{itemize}
\end{definition}

\begin{lemma}\label{lem: handshaking}
The number of vertices of odd degree in a graph is even.
\end{lemma}

\begin{proof}
Let \(G\) be a graph. Since \(\size G \in \N\) and \(\size G = \frac 1 2
\sum_{v \in \Vertex(G)} \deg v\), then \(\sum_{v \in \Vertex(G)} \deg v\) is even. If the
number of odd degree vertices where odd, then the sum of their degrees would
also be odd. Since the sum of even degree vertices is always even, it is
necessary for the number of odd degree vertices to be even.
\end{proof}

\begin{proposition}[Edge-dense subgraph]\label{prop: edge-dense-subgraph}
Let \(G\) be a finite graph with size \(\size G \geq 1\). Then there exists a
subgraph \(H \subseteq G\) such that
\[
  \delta(H) > \epsilon(H) \geq \epsilon(G).
\]
\end{proposition}

\begin{proof}
We'll construct the subgraph \(H\) by means of a chain of consequent single
vertex deletion---as in \cref{def: deletion-graph}. Let \(H_0 = G\) Notice that,
if we want to have the edge-vertex ratio unaltered (or increased) by the
deletion, we got to choose a vertex \(v \in H_0\)---if existent---such that the
graph \(H_1 = H_0 \setminus v\) has \(\epsilon(H_1) \geq \epsilon(H_0)\), that is, \(\deg v\) has to
satisfy
\[
  \epsilon(H_1) = \frac{\size{H_0} - \deg v}{|H_0| - 1}
  \geq \frac{\size{H_0}}{|H_0|} = \epsilon(H_0).
\]
Solving the above equation for the degree of \(v\), we find that
\(\deg v \leq \epsilon(H_0)\). Since
\(\delta(H_0) \leq \deg H_0 \leq \Delta(H_0)\) and
\(\epsilon(H_0) \leq \deg H_0\), then \(\deg v \leq \Delta(H_0)\)---that is,
\(\delta(H_1) \leq \delta(H_0)\). If such vertex doesn't exists in
\(\Vertex(H_0)\), we terminate our algorithm and find \(H = H_0 =
G\). Otherwise, we continue recursively, generating a finite chain of
subgraphs---the finiteness of the chain comes from the fact that \(G\) itself is
finite
\[
  G = H_0 \supset H_1 \supset H_2 \supset \dots \supset H_n = H
\]
for some \(n \in \N\). Where \(\epsilon(H_{j+1}) \geq \epsilon(H_j)\) and \(\delta(H_{j+1}) \geq \delta(H_j)\)
for all \(0 \leq j < n\). At some point the chain will terminate into the subgraph
\(H\). Moreover, since the recursion terminates in \(H \neq \emptygraph\), then
necessarily \(\deg v > \epsilon(H)\) for all \(v \in H\)---in particular, this implies in
\(\delta(H) > \epsilon(H)\).
\end{proof}

\begin{definition}[Regular graph]\label{def: k-regular}
A graph \(G\) is said to be \(k\)-regular---for some \(k \in \N\)---if
\(\deg v = k\) for all \(v \in G\).
\end{definition}

\section{Path to Glory}

\begin{definition}[Walk]\label{def: walk}
A walk on a graph \(G\) is defined to be a subgraph \(W \subseteq G\)---assume \(|W| = n
+ 1\) for some \(n \in \N\)---such that there exists a \emph{surjective map} \(\ell:
[n] \epi \Vertex(W)\) satisfying the condition of \(\Edge_G(\ell(k), \ell(k + 1))\) for all \(0
\leq k < n\). We'll generally assume this underlying surjective vertex-labelling
map and denote \(W = (\ell(0), \dots, \ell(n)) = (v_0, \dots, v_n)\). The walk \(W\)
is said to be closed if \(v_0 = v_n\).
\end{definition}

\begin{definition}[Length]\label{def: walk-length}
We define the length of a walk \(W\) to be its size \(\size W\).
\end{definition}

\begin{definition}[Walk operations]\label{def: walk-operations}
Given a path \(W = (v_0, \dots, v_n)\) we can define the following:
\begin{itemize}
  \setlength\itemsep{0.0em}
  \item (Subwalk) Let \(v_j \in \Vertex(W)\) be any vertex. We define a subwalk of
    \(W\) restricted to \(v_j\) to be one of the following subgraphs of \(W\):
    \(v_jW = (v_j, v_{j+1}, \dots, v_n)\) and \(Wv_j = (v_0, \dots, v_{j-1},
    v_j)\). We can also define excluding subwalk induced by \(v_j\) to be
    \(\breve v_j W = (v_{j+1}, \dots, v_n)\) and \(W \breve v_j = (v_0, \dots,
    v_{j-1})\).
  \item (Inner walk) The inner walk of \(W\) is given by \(\breve W = (v_1,
    \dots, v_{n-1})\).
  \item (Walk concatenation) Let \(Q = (w_0, \dots, w_m)\) be a walk.
    If \(P\) and \(Q\) have coinciding endings, say \(v_n = w_0 = x\), we can
    define the concatenation of \(P\) with \(Q\) to be the walk \(PQ = (v_0,
    \dots, v_{n-1}, x, w_1, \dots, w_m)\).
\end{itemize}
\end{definition}

\begin{definition}[Path]\label{def: path}
We define a path to be walk with no repeating vertices. A path graph \(P_n\)
is a graph of order \(n + 1\) and size \(n\) such that there exists a
\emph{bijective map} \(\ell_n: [n] \isoto \Vertex(P_n)\) for which
\(\Edge_{P_n}(\ell_n(k), \ell_n(k+1))\) for all \(0 \leq k < n\). We'll usually
assume the existence of the underlying vertex-labelling bijection \(\ell_n\).
This way we can naturally write the path \(P_n\) as a unique sequence of
vertices \((v_0, v_1, \dots, v_n)\) \emph{up to isomorphism} of graphs.

Given a graph \(G\), we define a path of size \(n + 1\) \emph{on} \(G\) to be
the \emph{induced subgraph} of \(G\) given by \(G[\iota(P_n)]\), where
\(\iota: P_n \emb G\) is an \emph{embedding of graphs}---see \cref{def:
graph-embedding}. The underlying labelling is now given by the composition
\(\iota \ell_n: [n] \to \Vertex(G)\).
\end{definition}

Since a path is just a special type of walk, every operation described on
\cref{def: walk-operations} is carried over to the path graphs.

\begin{proposition}[Transforming paths into walks]
\label{prop: morphism-path-walk}
A map \(\phi: P_n \to G\) is a morphism of graphs if and only if the sequence
\((\phi(v_0), \dots, \phi(v_n))\) is a walk on \(G\).
\end{proposition}

\begin{proof}
Throughout, assume \(0 \leq j < n\). If \(\phi\) is a morphism of graphs then
\(\Edge_{P_n}(v_j, v_{j + 1})\) implies \(\Edge_G(\phi(v_j), \phi(v_{j+1}))\) thus
inducing a walk on \(G\). On the other hand, if \((\phi(v_0), \dots,
\phi(v_n))\) is a walk on \(G\), then since the only edges on \(P_n\) occur
between consequent vertices \(v_j\) and \(v_{j+1}\), we have that \(\phi\)
indeed preserves the adjacency structure of the path \(P_n\), that is,
\(\Edge_{P_n}(v_j, v_{j+1})\) implies \(\Edge_G(\phi(v_j), \phi(v_{j+1}))\)---hence
\(\phi\) is a morphism.
\end{proof}

\begin{lemma}[Paths on walks]\label{lem: paths-on-walks}
Let \(W\) be a walk. If \(x, y \in \Vertex(W)\), then there exists a path \(P
\subseteq W\) such that \(x, y \in \Vertex(P)\).
\end{lemma}

\begin{proof}
Assume \(W\) is a \(k\)-walk and contains \(x\) and \(y\) at its
end-vertices---if not, we would be analogously analysing the subwalk
\(xWy\). Let \(\ell: [k] \epi \Vertex(W)\) be a surjective labelling on
\(W\)---since \(x\) and \(y\) are end-vertices, we have
\begin{equation}\label{eq: paths-on-walks-1}
  \ell(0) = x\ \text{ and }\ \ell(k) = y.
\end{equation}
Let \(\mathcal C\) be the
collection of all cycles on \(W\)---refer to \cref{def: cycle}. Define the
equivalence relation \(\sim\) on the vertices \(\Vertex(W)\) to be so that \(v \sim
u\) if and only if \(v, u \in \Vertex(C)\) for some \(C \in \mathcal C\).

The quotient graph \(W/{\sim}\) (see \cref{def: quotient-graph}) is a path
joining the vertex classes \(\overline x\) and \(\overline y\). To see that, we
can first construct an equivalence relation \(\sim_\ell\) on the indexing set
\([k]\), defined as follows: \(i \sim_\ell j\) if and only if
\(\ell(i) \sim \ell(j)\)---clearly \([k]/{\sim_\ell}\) is an indexing set for
the vertices of the quotient graph. Lets assume that
\([k]/{\sim_\ell}\, = \{\overline 0, \dots, \overline m\}\). Let
\(\ell': [k]/{\sim_\ell}\, \to \Vertex(W/{\sim})\) be a labelling on
\(W/{\sim}\) defined as \(\ell'(\overline j) = \overline{\ell(j)}\). Using
\cref{eq: paths-on-walks-1}, the sequence of vertices
\((\ell'(\overline 0), \dots, \ell'(\overline m))\) is such that
\(\ell'(\overline 0) = \overline x\) and \(\ell'(\overline m) = \overline y\).
Also,
\(\Edge_{W/{\sim}}(\ell'(\overline j), \ell'(\overline{j + 1})) = \Edge_W(\ell(j),
\ell(j+1))\) is always true (because \(W\) is a walk). So far, we've shown that
\(W/{\sim}\) is a walk between \(\overline x\) and \(\overline y\)---we now
need to show that no vertex is repeated in the given labelled sequence, which is
actually trivial. Suppose that \(\ell'(\overline i) = \ell'(\overline j)\), then
from definition of \(\ell'\) we have
\(\overline{\ell(i)} = \overline{\ell(j)}\), which implies that
\(\ell(i) \sim \ell(j)\) and hence \(i \sim_\ell j\)---thus
\(\overline i = \overline j\). This shows that
\[
  W/{\sim}\ = (\overline x, \ell'(\overline 1), \dots, \ell'(\overline{m - 1}),
  \overline y)
\]
is indeed a path joining \(\overline x\) and \(\overline y\).

For the last part we just need to consider the embedding of graphs \(\iota: W/{\sim}\
\emb W\) such that \(\Edge_{W/{\sim}}(\overline v, \overline u)\) implies
\(\Edge_W(\iota(\overline v), \iota(\overline u))\). The sequence of vertices
\[
  \iota(W/{\sim}) \coloneqq (\iota(\overline x), \iota\ell'(\overline 1), \dots,
  \iota \ell'(\overline{m-1}), \iota(\overline y)) \subseteq W
\]
is clearly a path on \(W\) joining the vertices \(\iota(\overline x) = x\) and
\(\iota(\overline y) = y\). Hence, \(P = W[\im \iota]\) gives us the wanted
path---finally the proposition is proved.
\end{proof}

The last proof may have got really clumsy at some points, so here goes an
example of the basic operations we've developed and used throughout the
proof. Let \(W = (x, v_1, \dots, v_{12}, y)\) be a walk, visually given
by---where \(v_5 = v_2\) and \(v_{11} = v_6\)
\[
\begin{tikzcd}[%
  , cells={nodes={circle, draw, inner sep=1.5pt}}
  , every arrow/.append style={dash,thick}
  , row sep=tiny
  , column sep=tiny
  ]
  % 1 line
            &v_4 \ar[rr] \ar[dr] & &v_3 \ar[dl] & &
  \\
  % 2 line
  x \ar[dr] & &v_2 \ar[dl] \ar[dr] & &v_{12} \ar[dr] &
  \\
  % 3 line
            &v_1 & &v_6 \ar[ur] \ar[dl] \ar[dr] & &y
  \\
  % 4 line
            & &v_{10} \ar[d] & &v_7 \ar[d] &
  \\
  % 5 line
            & &v_9 \ar[rr] & &v_8 &
\end{tikzcd}
\]
Now, the process of taking the quotient of the graph amounts to the
identification of the cycles
\[
  \mathcal C = \{
    W[\{v_2, v_3, v_4, v_5\}],\,
    W[\{v_6, v_7, v_8, v_9, v_{10}, v_{11}\}]
  \}
\]
The cycles are respectively reduced to classes \(\overline v_2\) and \(\overline
v_3\), both elements of \(\Vertex(W)/{\sim}\). The quotient graph \(W/{\sim}\) can be visually
depicted as follows
\[
\begin{tikzcd}[%
  , cells={nodes={circle, draw, inner sep=1.5pt}}
  , every arrow/.append style={dash,thick}
  , row sep=tiny
  , column sep=tiny
  ]
  % 1 line
  \overline x \ar[dr]
  & &\overline v_2 \ar[dl] \ar[dr] & &\overline v_4 \ar[dl] \ar[dr] &
  \\
  &\overline v_1 & &\overline v_3 & &\overline y
\end{tikzcd}
\]
Now for the embedding of graphs \(\iota: W/{\sim}\, \emb W\), we can view
\(P = W[\im \iota]\) as the subgraph consisting of the red edges and their
end-vertices---that is, the collection of vertices
\(\{x, v_1, v_2, v_6, v_{12}, y\}\)---which can be visualised as follows
\[
\begin{tikzcd}[%
  , cells={nodes={circle, draw, inner sep=1.5pt}}
  , every arrow/.append style={dash,thick}
  , row sep=tiny
  , column sep=tiny
  ]
  % 1 line
            &v_4 \ar[rr] \ar[dr] & &v_3 \ar[dl] & &
  \\
  % 2 line
  x \ar[dr, color=red] & &v_2 \ar[dl, color=red] \ar[dr, color=red]
                        & &v_{12} \ar[dr, color=red] &
  \\
  % 3 line
            &v_1 & &v_6 \ar[ur, color=red] \ar[dl] \ar[dr] & &y
  \\
  % 4 line
            & &v_{10} \ar[d] & &v_7 \ar[d] &
  \\
  % 5 line
            & &v_9 \ar[rr] & &v_8 &
\end{tikzcd}
\]

\begin{corollary}[Distances \& morphisms of graphs]
\label{cor: morphism-cycle}
Consider graphs \(G\) and \(H\). If \(\phi: G \to H\) is a morphism of
graphs then, for all \(x, y \in G\), we have
\[
  d_H(\phi(x), \phi(y)) \leq d_G(x, y).
\]
\end{corollary}

\begin{proof}
Assume \(\phi\) is a morphism. Suppose \(x, y \in G\) are separated by a
finite distance---if not, the proposition follows trivially. Let \(P_k\) be
a minimal path on \(G\) joining \(x\) and \(y\). Since \(\phi\) is a morphism,
the image \(\phi(P_k) \subseteq H\) is a walk where \(\phi(x), \phi(y) \in
P_k\)---this follows directly from \cref{prop: morphism-path-walk}. Now,
using \cref{lem: paths-on-walks} on \(\phi(P_k)\) and the vertices \(\phi(x)\)
and \(\phi(y)\), it follows that there exists a path \(P \subseteq \phi(P_k)\)
joining \(\phi(x)\) and \(\phi(y)\). Moreover, clearly \(\size P \leq
\size{\phi(P_k)} \leq k\), thus \(d_H(\phi(x), \phi(y)) \leq d_G(x, y)\).
\end{proof}

\begin{definition}[Path between sets]\label{def: A-B-path}
Let \(A\) and \(B\) be disjoint sets of vertices. \(P = (v_0, \dots, v_n)\) is
said to be an \(A\)-\(B\) path if \(v_0 \in A\) and \(v_n \in B\).
\end{definition}

\begin{definition}[Independent paths]\label{def: independent-paths}
Two paths are said to be independent if their inner path share no vertex.
\end{definition}

\subsection{Cycles}

\begin{definition}\label{def: cycle}
A cycle is a closed walk with size at least \(3\). A \(k\)-cycle is a cycle
\(C^k\) whose length is \(k\).

A \(k\)-cycle \emph{on} a graph \(G\) is the \emph{induced subgraph} of \(G\)
given by \(G[\iota(C^k)]\) such that \(\iota: C^k \to G\) is an embedding of
graphs.
\end{definition}

\begin{definition}[Chords]\label{def: chord}
Let \(G\) be a graph and \(C\) be a cycle on \(G\). Given \(x, y \in C\), a
chord is an edge \((x, y) \in \Edge(G)\) such that \(\Edge_C(x, y)\) is false.
\end{definition}

\begin{definition}[Induced cycle]\label{def: induced-cycle}
An induced cycle on a graph is a cycle with no chords.
\end{definition}

\begin{definition}[Distance]\label{def: vertex-distance}
Let \(G\) be a graph and \(x, y \in \Vertex(G)\) be any vertices. Let \(\mathcal P\)
be the collection of all paths in \(G\) containing the vertices \(x\) and
\(y\). If \(\mathcal P\) is non-empty, we define the distance of \(x\) and
\(y\) on \(G\) to be the minimum length of the subpaths linking \(x\) to
\(y\), in other words
\[
  d_G(x, y) = \min_{P \in \mathcal P} \size{xPy}.
\]
On the other hand, if \(\mathcal P\) is empty, then \(d_G(x, y) = \infty\).
\end{definition}

\begin{definition}[Miscellaneous definitions]
\label{def: miscellaneous-graph-cycle}
We define the following:
\begin{enumerate}[(a).]
  \setlength\itemsep{0.0em}
  \item\label{def: girth}
    (Girth) The girth of a graph \(G\) is the minimum length of a cycle on
    \(G\). In other words, let \(\mathcal C\) be the collection of all cycles
    on \(G\), the girth of \(G\) is
    \[
      g(G) = \min_{C \in \mathcal C} \size C.
    \]
  \item\label{def: circ-graph}
    (Circumference) The circumference of a graph \(G\) is the maximal length
    of a cycle on \(G\). In other words, let \(\mathcal C\) be the collection
    of all cycles on \(G\). If \(\mathcal C\) is non-empty, we set
    \[
      \Circ(G) = \max_{C \in \mathcal C} \size C.
    \]
  \item\label{def: diameter}
    (Diameter) The diameter of a graph \(G\) is the maximal distance between
    two vertices of \(G\), that is
    \[
      \diam(G) = \max_{x, y \in \Vertex(G)} d_G(x, y).
    \]
\end{enumerate}
\end{definition}

\begin{proposition}\label{prop: path-cycle-len-delta}
Let \(G\) be a graph such that \(\delta(G) \geq 2\). Then there exists a path
on \(G\) with length \(\delta(G)\) and a cycle on \(G\) with length at least
\(\delta(G) + 1\).
\end{proposition}

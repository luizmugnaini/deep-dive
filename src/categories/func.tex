\section{Functors}

\begin{definition}[Functor]\label{def: functor}
  Let \(\cat C\) and \(\cat D\) be categories. A functor \(F: \cat C \to \cat
  D\) has the following data:
  \begin{enumerate}[({DF}1)]
    \item For all \(c \in \cat C\) exists a corresponding \(F c \in \cat D\).
    \item For all \(f: c \to c' \in \Mor(\cat C)\) there exists a morphism \(F
      f: F c \to F c' \in \Mor(\cat D)\).
  \end{enumerate}
  Such data satisfies the two following axioms:
  \begin{enumerate}[({AF}1)]
    \item For all composable \(f, g \in \Mor(\cat C)\), we have \(F g \cdot F f
      = F (g \cdot f)\).
    \item For all \(c \in \cat C\) we have \(F \Id_c = \Id_{F c} \in \Mor(\cat
      D)\).
  \end{enumerate}
  This definition is sometimes called covariant functor (in contrast with
  contravariant functor, see \cref{def: contravariant functor}).
\end{definition}

\begin{definition}[Forgetful functor]
  A functor is said to be forgetful if the functor ``forgets'' some object,
  structure or property of its domain category.
\end{definition}

\begin{example}
  We have some classical examples of forgetful functors, for instance, the
  following are functors that forget the structure of their domain categories:
  \begin{itemize}
    \item The functor \(G: \Grp \to \Set\) mapping groups to its
      corresponding underlying set.
    \item The functor \(T: \Top \to \Set\) maps any topological space
      to its corresponding set of points.
    \item The functor \(V, E: \Graph \to \Set\) maps the vertices and
      edges of a graph to the set of such vertices an edges.
  \end{itemize}
\end{example}

\begin{definition}[Contravariant functor]
  \label{def: contravariant functor}
  A contravariant functor from categories \(\cat C\) to \(\cat D\) is a functor
  \(F: \cat C^\op \to \cat D\) together with the following data:
  \begin{enumerate}[({DCF}1)]
    \item For all \(c \in \cat C\) exists \(F c \in \cat D\).
    \item For all \(f: c \to c' \in \Mor(\cat C)\) we have \(F f: F c' \to F c
      \in \Mor(\cat D)\).
  \end{enumerate}
  Moreover, a contravariant functor satisfies the following axioms:
  \begin{enumerate}[({ACF}1)]
    \item For all composable \(f, g \in \Mor(\cat C)\) we have \(F f \cdot F g =
      F (g \cdot f)\).
    \item For all \(c \in \cat C\) we have \(F \Id_c = \Id_{F c}\).
  \end{enumerate}
  This can all be comprised diagrammatically as:
  \[
    \begin{tikzcd}
      \cat C^\op \ar[rr, "F"] & &\cat D
      \\
      c \ar[d, "f"] \ar[dd, bend right = 60, swap, "g f"]
      \ar[rr, mapsto]
      & &F c
      \\
      c' \ar[d, "g"] \ar[rr, mapsto]
      & &F c' \ar[u, "F f"]
      \\
      c'' \ar[rr, mapsto]
      & & F c''
      \ar[u, "F g"]
      \ar[uu, bend right = 60, swap, "F f \cdot F g = F (g \cdot f)"]
    \end{tikzcd}
  \]
\end{definition}

\begin{example}[\(\Top^\op \to \Rng\)]
  Let \(C: \Top^\op \to \Rng\) be a contravariant functor such that
  for all \(X \in \Top\), let \(C X\) be the ring of continuous functions
  \(X \to \R\). The ring operations on \(C X\) are defined pointwise,
  that is, given \(p, q: X \to \R \in C X\) we have \((p \cdot q) (x) =
  p(x) \cdot q(x)\) and \((p + q)(x) = p(x) + q(x)\) for all \(x \in X\).
  Moreover, given a morphism \(f: X \to Y \in \Mor(\Top)\) we define \(C f:
  C Y \to C X\) as the composition \((C f)(q) = q f \in \Mor(C X)\) for all \(q
  \in C Y\), that is
  \[
    \begin{tikzcd}
      X \ar[r, "f"] \ar[rr, bend right = 40, swap, "C f(q) = q f"]
      & Y \ar[r, "q"] & \R
    \end{tikzcd}
  \]

  We now show that the axioms for the contravariant functor are satisfied by
  \(C\). Let \(f: X \to Y\) and \(g: Y \to Z\), then given any \(p \in C Z\) we
  have
  \[
    \begin{tikzcd}
      X \ar[r, "f"]
      \ar[rrr, bend left = 60, "C f(C g (p))"]
      \ar[rrr, bend right, swap, "C (g f)(p)"]
      & Y \ar[r, "g"]
      \ar[rr, bend left = 35, "C g(p)"]
      & Z \ar[r, "p"] & \R
    \end{tikzcd}
  \]
  hence \(C (f) C (g) = C (g f)\). Moreover, given any \(X \in \Top^\op\)
  we find that \(C \Id_X: C X \to C X\) is such that for all \(q \in C X\), \(C
  \Id_X (q) = q \Id_X = q\) hence \(C \Id_X = \Id_{C X}\). This finishes the
  proof that \(C: \Top^\op \to \Rng\) is a contravariant functor.
\end{example}

\begin{definition}[Presheaf]\label{def: presheaf}
  Let \(\cat C\) be a small category. A contravariant functor \(\cat C^\op \to
  \Set\) is called a presheaf on \(\cat C\).
\end{definition}

\begin{example}[\(\mathcal O(X)^\op \to \Set\)]
  Let \(X \in \Top\) we define \(\mathcal O(X)\) to be the poset category
  whose objects are open sets of \(X\). That is, for sets \(U, U' \in \mathcal
  O(X)\), if \(U \subseteq U'\), then there exists a morphism \(U \to U'\) in
  \(\Mor(\mathcal O(X))\). A presheaf on the category \(\mathcal O(X)\) is a
  functor \(F: \mathcal O(X)^\op \to \Set\) that assigns \(F U = \{f: U \to
  \R: f \text{ continuous}\}\) for all \(U \in \mathcal O(X)\).
  Moreover, for maps \(g: U \to U'\) (that is \(U \subseteq U'\)) we have \(F g:
  F U' \to F U\) such that \(F g(f) = f|_U: U \to \R\) for all \(f: U'
  \to \R\) continuous. Since the restriction of a continuous map is
  continuous, then \(f|_U \in F U\)
\end{example}

\begin{lemma}\label{lem: functor preserve iso}
  Functors preserve isomorphisms. Let \(\cat C\) and \(\cat D\) be categories
  and \(F: \cat C \to \cat D\) be a functor. Given an isomorphism \(f: c \isoto
  c' \in \Mor(\cat C)\), we have that \(F f : F c \isoto c'\) is an isomorphism.
\end{lemma}

\begin{proof}
  Denote by \(f^{-1}: c' \isoto c\) the inverse of \(f\). By the composition
  axiom we have
  \begin{gather*}
    F (f^{-1}) F (f) = F (f^{-1} f) = F \Id_c    = \Id_{F c}, \\
    F (f) F (f^{-1}) = F (f f^{-1}) = F \Id_{c'} = \Id_{F c'}.
  \end{gather*}
  This shows that \(F f^{-1}\) is the right and left inverse of \(F f\), hence
  \(F f: F c \isoto F c'\) is indeed an isomorphism.
\end{proof}

\begin{lemma}
  Functors preserve split monomorphisms and split epimorphisms.
\end{lemma}

\begin{proof}
  Let \(\cat C\) and \(\cat D\) be categories and consider a functor \(F: \cat C
  \to \cat D\). Define morphisms \(x \xrightarrow s y \xrightarrow r x\) in
  \(\Mor(\cat C)\) such that \(r s = \Id_x\), that is, \(s\) is a split
  monomorphism and \(r\) is a split epimorphism. Consider the morphisms \(F s: F
  x \to F y\) and \(F r: F y \to F x\) in \(\Mor(\cat D)\). Notice that \(F (s)
  F(r) = F(s r) = F(\Id_x) = \Id_{F x}\). Hence \(F s\) is a split monomorphism
and \(F r\) is a split epimorphism.
\end{proof}

\begin{definition}
  Let \(\cat C\) be a locally small category. Given any \(c \in \cat C\), there
  exists a pair of covariant and contravariant functors, \(\cat C(c, -)\) and
  \(\cat C(-, c)\) respectively, represented by the object \(c\). That is:
  \[
    \begin{tikzcd}
      \cat C \ar[rr, "\cat C{(c, -)}"] & & \Set
      \\
      x \ar[rr, maps to] \ar[d, swap, "f"]
      & & \cat C(c, x) \ar[d, "f_*"]
      \\
      y \ar[rr, maps to] & & \cat C(c, y)
    \end{tikzcd}
    \qquad
    \begin{tikzcd}
      \cat C^\op \ar[rr, "\cat C{(-, c)}"] & & \Set
      \\
      x \ar[rr, maps to] \ar[d, swap, "f"]
      & & \cat C(x, c)
      \\
      y \ar[rr, maps to] & & \cat C(y, c) \ar[u, swap, "f^*"]
    \end{tikzcd}
  \]
\end{definition}

We now prove that such definition indeed satisfies the axioms for covariant and
contravariant functors. Given morphisms \(f: x \to y\) and \(g: y \to z\) in
\(\Mor(\cat C)\), we see that \(g_* f_* = (g f)_*\), moreover \(f^* g^* = (f
g)^*\). Let \(x \in \cat C\) be any object, then \(\Id_{x *} = \Id_{\cat C(c,
x)} = \Id_x^*\). This proves that \(\cat C(c, -)\) is covariant and \(\cat C(-,
c)\) is contravariant.

\begin{definition}[Faithful]\label{def: faithful}
  Let \(\cat C\) and \(\cat D\) be categories. A functor \(F: \cat C \to \cat
  D\) is said to be faithful if for all \(x, y \in \cat C\) the map \(\cat C(x,
  y) \to \cat D(F x, F y)\) is injective.
\end{definition}

\begin{definition}[Full]\label{def: full}
  Let \(\cat C\) and \(\cat D\) be categories. A functor \(F: \cat C \to \cat
  D\) is said to be full if for all \(x, y \in \cat C\) the map \(\cat C(x,
  y) \to \cat D(F x, F y)\) is surjective.
\end{definition}

\begin{definition}[Product category]
  Given categories \(\cat C\) and \(\cat D\), we define their product \(\cat C
  \times \cat D\) to be a category whose objects are ordered pairs \((c, d)\)
  such that \(c \in \cat C\) and \(d \in \cat D\), and morphisms are ordered
  pairs \((f, g): (c, d) \to (c', d')\), where \(f: c \to c' \in \Mor(\cat C)\)
  and \(g: d \to d' \in \Mor(\cat D)\).
\end{definition}

%\begin{proposition}[Bifunctor]
  %Let categories \(\cat A, \cat B\) and \(\cat C\). There exists a uniquely
  %defined functor \(F: \cat A \times \cat B \to \cat C\).
%\end{proposition}

\todo[inline]{Bifunctor and two sided represented functor!}

\todo[inline]{After solving some problems on functors, next topic: natural
transformations}

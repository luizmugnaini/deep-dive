\section{Existence of Finite fields}

\begin{remark}
   Our main goal of this section is to prove that, given a number \(p^n\), where
   \(p\) is prime, there exists a finite field with cardinality \(p^n\).
\end{remark}

\begin{proposition}
   Let \(f \in k[x]\) be a non-zero irreducible polynomial. There exists a field \(K
   \supseteq k\) and an element \(\alpha \in K\) such that \(f(\alpha) = 0\).
\end{proposition}

\begin{proof}
   Since \(k[x]\) is a principle ideal domain and \(f\) is irreducible and
   non-zero we conclude that the prime ideal \((f(x))\) is a maximal ideal of
   \(k[x]\). From this latter fact we conclude that \(k[x]/(f(x))\) is a
   field (analogously on \(\mathbb{F}_p\)). Define \(\varphi : k[x] \to
   k[x]/(f(x))\) be the homomorphism mapping \(g \mapsto g k[x]/(f(x))\) the
   coset of \(g\) modulo \((f(x))\). Then \(\varphi(k) \subseteq k[x]/(f(x))\)
   is a subfield, we want to show that \(\varphi(k) \simeq k\).
   \todo[inline]{Finish proof, needs some understanding of algebra that I
   currently do not have \dots}
\end{proof}

\begin{proposition}\label{prop: basis fini}
   The set \(\{1, \alpha, \alpha^2, \dots, \alpha^n\} \) form a basis for
   \(k(\alpha)\) as a linear \(k\)-space, where \(n := \deg(f)\).
\end{proposition}

\begin{proposition}
   Let \(F_d(x)\) be the product of all monic irreducible polynomials of
   \(\mathbb{F}_p[x]\) with degree \(d\). Then 
   \[
      x^{p^n} - x = \prod_{d \mid n} F_d(x)
   \] 
\end{proposition}

\begin{proof}
   Suppose firstly that \(f \in k[x]\) is such that \(f(x) \mid x^{p^n} - x\),
   we ought to show that \(f^2(x) \nmid x^{p^n} - x\). Notice that if  \(f^2(x)
   \mid x^{p^n} - x\) then  \(\exists g \in k[x]\) such that \(f^2(x)g(x) =
   x^{p^n} - x\) and therefore 
   \[
       f(x)\left( 2f'(x)g(x)\right) + f(x) \left( f(x) g'(x) \right) = p^n
       x^{p^n - 1} - 1 = -1 \Rightarrow f(x) \mid 1
   \] 
   which is a contradiction.

   Let \(\alpha\) be a root of \(f\) and define \(\deg(f) := d\). Then, from
   proposition \ref{prop: basis fini} we have that \(\mathbb{F}_p(\alpha)\) has
   as a \(\mathbb{F}_p\)-space has dimension \(d\) and thus
   \(|\mathbb{F}_p(\alpha)| = p^d\). Then, all \(\omega \in
   \mathbb{F}_p(\alpha)\) satisfy \(x^{p^d} - x = 0\), we want to show that
   these elements are roots of \(x^{p^n} - x\) either. 

   Let again \(f(x) \mid x^{p^n} - x\) and suppose that  \(f(x)g(x) = x^{p^n} -
   x\), then, since \(f(\alpha) = 0\) we need \(\alpha^{p^n} - \alpha = 0\).
   Since \(\{1, \alpha, \alpha^2, \dots, \alpha^{d-1}\}\) is a basis for
   \(\mathbb{F}_p(\alpha)\), let \(\omega = \sum_{j=0}^{d-1} b_j \alpha^j \in
   \mathbb{F}_p(\alpha)\) be any element, where \(b_j \in \mathbb{F}_p\), we
   want \(\omega^{p^n} - \omega = 0\), but notice that this is pretty direct
   now, since
    \[
       (\omega)^{p^n} = \sum_{0 \leqslant j \leqslant d-1} b_j \alpha^{j(p^n)}
   \] 
   from the Frobenius endomorphism and also, since \(\alpha^{p^n} = \alpha\) we
   have 
    \[
       (\omega)^{p^n} = \sum_{0 \leqslant j \leqslant d-1} b_j \alpha^j = \omega
   \] 
   as wanted, thus \(\forall \omega \in \mathbb{F}_p(\alpha)\) we have
   \(\omega^{p^d} - \omega = 0\) and also  \(\omega^{p^n} - \omega = 0\), which
   implies that  
   \[
      x^{p^d} - x \mid x^{p^n} - x \Rightarrow x^{p^d - 1} - 1 \mid
      x^{p^n - 1} - 1 \Rightarrow p^d - 1 \mid p^n - 1
      \Rightarrow d \mid n.
   \]

   Let on the opposite side \(d \mid n\). Since \(f\) is the minimal polynomial
   of \(\alpha\) and \(f(x) \mid x^{p^d} - x\) (because \(\alpha^{p^d} -
   \alpha = 0\)) and also \(x^{p^d} - x \mid x^{p^n} - x\) (because \(d \mid
   n\)), then \(f(x) \mid x^{p^n} - x\), which means for every \(d \mid n\) we
   have monic irreducible polynomials of degree \(d\) appearing in the
   decomposition of \(x^{p^n} - x\) and also (from the first part of the proof)
   each of these appear only once in the factorization.
\end{proof}

\begin{corollary}
   Let \(N_d\) be the number of polynomials \(f \in \mathbb{F}_p[x]\) such that
   \(\deg(f) = d\). Equating the degrees of the polynomials, we have
   \[
      p^n = \sum_{d \mid n} d N_d.
   \] 
\end{corollary}

\begin{corollary}
   From Möbius inversion formula we have
   \[
      N_n = \sum_{d \mid n} \mu(n/d) \frac{p^d}{n}.
   \] 
\end{corollary}

\begin{corollary}
   Let \(n \geqslant  1\). There exists an irreducible polynomial of degree
   \(n\) in \(\mathbb{F}_p[x]\).
\end{corollary}

\begin{theorem}
   For every \(n \geqslant 1\) and \(p\) a prime. Then there exists a finite
   field \(\mathbb{F}_p(\alpha)\) with  \(p^n\) elements, where \(\alpha\) has
   minimal polynomial \(f \in \mathbb{F}_p[x]\) with degree \(n\).
\end{theorem}

\section{Duality}

\begin{definition}[Opposite category]\label{def: opposite cat}
  Let \(\cat C\) be a category. We define the opposite category of \(\cat C\) as
  \(\cat C^\op\), such that
  \begin{enumerate}[(COP1)]\setlength\itemsep{0em}
    \item \(\Obj(\cat C^\op) = \Obj(\cat C)\).
    \item Given a morphism \(f: x \to y\) in  \(\cat C\), there exists a
      corresponding morphism \(f^\op: y \to x\) in the \(\cat C^\op\). That is
      \(\dom f = \codom f^\op\) and \(\codom f = \dom f^\op\), these form all of
      the morphisms in the category \(\cat C^\op\).
    \item For all \(A \in \cat C^\op\), there exists an identity morphism
      \(\Id_A^\op \in \End_{\cat C^\op}(A)\).
    \item A pair of morphisms \(f^\op, g^\op \in \cat C^\op\) is composable, so
      that \(\dom g^\op = \codom f^\op\), if and only if the pair \(g, f \in
      \cat C\) is composable, that is \(\dom f = \codom g\). Moreover, we
      define their composition as \(g^\op f^\op = (f g)^\op\).
  \end{enumerate}
\end{definition}

\begin{lemma}\label{lem:f-iso-iff-f*-iso}
  The following propositions are equivalent
  \begin{enumerate}[(a)]\setlength\itemsep{0em}
    \item \(f : x \isoto y\) is an isomorphism in \(\cat C\).
    \item For all \(c \in \cat C\), the map
      \[
        f_*: \Hom_{\cat C}(c, x) \to \Hom_{\cat C}(c, y), \qquad
        g \xmapsto{f_*} f g
      \]
      is a bijection.
    \item For all \(c \in \cat C\), the map
      \[
         f^* : \Hom_{\cat C} (x, c) \to \Hom_{\cat C}(y, c), \qquad
         g \xmapsto{f^*} g f
      \]
      is a bijection.
  \end{enumerate}
\end{lemma}

\begin{proof}
  ((a) \(\Rightarrow\) (b)) Let \(f: x \isoto y\) be an isomorphism and \(c
  \in \cat C\) be an object and define \(\ell : y \isoto x\) to be its inverse.
  Given \(c \in \cat C\), define \(\ell_*: \Hom_{\cat C}(c, y) \to \Hom_{\cat
  C}(c, x)\). Notice that \(f_* \ell_* : \Hom_{\cat C}(c, y) \to
  \Hom_{\cat C}(c, y)\) mapping \(g \mapsto f_*(\ell_*(g)) = f \ell
  g = g\), hence \(f_* \ell_* = \Id_{\Hom_{\cat C}(c, y)}\). Moreover, we
  have \(\ell_* f_* : \Hom_{\cat C}(c, x) \to \Hom_{\cat C}(c, x)\) mapping \(h \mapsto
\ell_*(f_*(h)) = h f \ell = h\), hence
  \(\ell_* f_* = \Id_{\Hom_{\cat C}(c, x)}\). This shows that \(\ell_*\)
  is the inverse of \(f_*\) and therefore \(f_*\) is an isomorphism. ((b)
  \(\Rightarrow\) (a)) Suppose the contrary, so that \(f_*\) is an isomorphism.
  In particular, we can take \(c = y\) so that \(f_*: \Hom_{\cat C}(y, x) \to
  \Hom_{\cat C}(y, y)\). From the isomorphism property, there exists \(\ell \in
  \Hom_{\cat C}(y, x)\) such that \(f_*(\ell) = f \ell = \Id_y\). Consider
  now that \(c = x\), then \(f_*: \Hom_{\cat C}(x, x) \to \Hom_{\cat C}(x, y)\).
  Notice that \(f_*(\ell f) = f \ell f = \Id_y f = f\)
  and \(f_*(\Id_x) = f \Id_x = f\). Since \(f_*\) is supposed to be an
  isomorphism, it follows that \(\Id_x = \ell f\). With this we conclude
  that \(\ell\) is the inverse of \(f\) and hence \(f\) is an isomorphism.

  ((a) \(\Leftrightarrow\) (c)) Suppose that
  \(f^\op: y \to x \in \Mor(\cat C^\op)\), then from the last paragraph we have
  that \(f^\op\) is an isomorphism if and only if
  \((f^\op)_* : \Hom_{\cat C^\op}(c, y) \to \Hom_{\cat C^\op}(c, x)\) is an
  isomorphism. Therefore the dual of such statement is that
  \(f: x \to y \in \Mor(\cat C)\) is an isomorphism if and only if
  \((f^\op)_*^\op = f^*: \Hom_{\cat C}(x, c) \to \Hom_{\cat C}(y, c)\) is an
  isomorphism, since \(\Hom_{\cat C^\op}(*, *') = \Hom_{\cat C}(*', *)\).
\end{proof}

\begin{definition}[Monomorphism]\label{def: monomorphism}
  Let a category \(\cat{C}\). We say that \(f \in \Hom_{\cat C}(x, y)\) is a
  \emph{monomorphism} if for all \(c \in \cat{C}\), and for all \(\alpha, \beta
  \in \mathrm{Hom}(c, x)\) we have that \(f \alpha = f \beta\) implies \(\alpha
  = \beta\). Equivalently, for all \(c \in \cat C\) the map \(f_*: \Hom_{\cat
  C}(c, x) \to \Hom_{\cat C}(c, y)\) is injective.
\end{definition}

\begin{definition}[Epimorphism]\label{def: epimorphism}
  Let a category \(\cat C\). We say that morphism \(g \in \mathrm{Hom}_{\cat
  C}(x, y)\) to be an \emph{epimorphism} if for all \(c \in \cat{C}\), and for
  all \(\gamma, \delta \in \mathrm{Hom}_{\cat C}(y, c)\) we have that \(\gamma g
  = \delta g\) implies \(\gamma = \delta\). Equivalently, for all \(c \in \cat
  C\) the map \(g^*: \Hom_{\cat C}(y, c)\to \Hom_{\cat C}(x, c)\) is injective.
\end{definition}

\begin{proposition}
\label{prop:}
Let \(\cat C\) be a category. The following are properties regarding
monomorphisms and epimorphisms in \(\cat C\):
\begin{enumerate}[(a)]\setlength\itemsep{0em}
\item Every identity morphism is a monomorphism and an epimorphism
\item The composite of two monomorphisms (or epimorphisms) is a monomorphism (or
  epimorphism).
\item If the composition \(k f\) is a monomorphism, then \(f\) is a
  monomorphism. Conversely, if \(f k\) is an epimorphism, then \(f\) is an
  epimorphism.
\end{enumerate}
\end{proposition}

\begin{proof}
We prove the assertions about monomorphisms, the respective ones for
epimorphisms are dually true from the former.
\begin{enumerate}[(a)]\setlength\itemsep{0em}
\item Identities are isomorphism, so clearly they are monomorphisms and
  epimorphisms.

\item Let \(f\) and \(g\) be composable monomorphisms, then if \(g f p = g f q\)
  for two given morphisms \(p\) and \(q\), for since \(g\) is a monomorphism
  then \(f p = f q\) --- then using the fact that \(f\) is a monomorphism we
  obtain \(p = q\).

\item Suppose \(k f\) is a monomorphism and consider morphisms \(g\) and \(h\)
  such that \(f g = f h\). Composing with \(k\) one sees that \(k f g = k f h\)
  but since \(k f\) is a monomorphism, it follows that \(g = h\).
\end{enumerate}
\end{proof}

\begin{definition}
  Let \(x \xrightarrow s y \xrightarrow r x\) be morphisms such that \(r s =
  \Id_x\). We define the following terms
  \begin{enumerate}[(a)]\setlength\itemsep{0em}
    \item\label{def: split monomorphism}
      \(s\) is said to be a section of \(r\). The morphism \(s\) is always a
      monomorphism, being called a split monomorphism.
    \item\label{def: split epimorphism}
      \(r\) is said to be the retraction of \(s\). The morphism \(r\) is always
      an epimorphism, being called a split epimorphism.
    \item\label{def: retract}
      \(x\) is the retract of \(y\).
  \end{enumerate}
\end{definition}

\begin{proposition}
  A morphism \(f \in \Hom_{\cat C}(x, y)\) is a split epimorphism if and only if
  for all \(c \in \cat C\) the map \(f_*: \Hom_{\cat C}(c, x) \to \Hom_{\cat
    C}(c, y)\) is surjective. Dually, \(f\) is a split monomorphism if and only
    if for all \(c \in \cat C\) the map \(f^*: \Hom_{\cat C}(x, c) \to
    \Hom_{\cat C}(y, c)\) is surjective.
\end{proposition}

\begin{proof}
  (\(\Rightarrow\)) Suppose \(f: x \to y\) is a split epimorphism and define
  \(g: y \to x\) as a section of \(f\), that is \(f g = \Id_y\). Let \(c
  \in \cat C\), and \(\alpha \in \Hom_{\cat C}(c, y)\) be any morphism. Notice
  that \(g \alpha \in \Hom_{\cat C}(c, x)\), hence we find that \(f_*(g
  \alpha) = f g \alpha = \alpha\) and therefore \(f_*\) is
  surjective. (\(\Leftarrow\)) Suppose \(f_*\) is surjective, then in particular
  for \(c = y\) we have that \(\Id_y \in \im f_*\) and hence \(f\) is a split
  epimorphism. (Dual) Let \(f^\op: y \to x\), then from the above proposition
  \(f^\op\) is a split epimorphism if and only if \((f^\op)_*: \Hom_{\cat
  C^\op}(y, c) \to \Hom_{\cat C^\op}(x, c)\) is surjective. Dually we have that
  \(f\) is a monomorphism if and only if \((f^\op)_*^\op = f^*: \Hom_{\cat C}(c,
  x) \to \Hom_{\cat C}(c, y)\), which proves the last part.
\end{proof}

\begin{proposition}
  Let \(f \in \Mor(\cat C)\). If \(f\) is a monomorphism and also a split
  epimorphism, then \(f\) is an isomorphism. Dually, if \(f\) is an epimorphism
  and a split monomorphism, then \(f\) is an isomorphism.
\end{proposition}

\begin{proof}
  Suppose \(f: x \to y\) is a split epimorphism and \(g: y \to x\) be a section
  of \(f\), then \(f g = \Id_y\) and, moreover \(f g f = \Id_y
  f = f\) hence from if \(f\) is a monomorphism we conclude that \(g
  f = \Id_x\). Thus \(g\) is the inverse of \(f\) and hence \(f\) is an
  isomorphism. Dually, let \(f^\op: y \to x\) be a monomorphism and split
  epimorphism, then \(f^\op\) is an isomorphism, which dually means that \(f\)
  is an epimorphism and split monomorphism, then \(f\) is an isomorphism.
\end{proof}

\begin{lemma}
  Let \(f: x \to y\) and \(g: y \to z\). The following propositions hold
  \begin{enumerate}[(a).]\setlength\itemsep{0em}
    \item If \(f\) and \(g\) are monomorphisms, so is \(g f: x \mono z\).
      For the dual proposition, if \(f\) and \(g\) are epimorphisms, then \(g
      f: x \epi z\) is an epimorphism.
    \item If \(g f: x \mono z\) is a monomorphism, then \(f\) is also
      monomorphism.  Dually, if \(g f: x \epi z\) is an epimorphism, then
      \(g\) is an epimorphism.
  \end{enumerate}
\end{lemma}

\begin{proof}
  (a) Suppose that \(f, g\) are both monomorphisms. Given any \(c \in \cat C\),
  let \(\alpha, \beta \in \Hom_{\cat C}(c, x)\) be such that \(g f \alpha = g f
  \beta\). In particular, since \(g\) is monic, then \(f \alpha = f \beta\), but
  \(f\) is also monic, hence \(\alpha = \beta\). For the dual part, suppose that
  \(f^\op: y \to x\) and \(g^\op: z \to y\) be monic, then from above we have
  \(f^\op g^\op = (g f)^\op\) monic, which dually implies that \(g f\) is epic.

  (b) Let \(g f\) be monic. Then given \(c \in \cat C\) and \(\alpha, \beta \in
  \Hom_{\cat C}(c, x)\) such that \(f \alpha = f \beta\), then in particular
  \(g(f \alpha) = g(f \beta)\), then from the monic property we have \(\alpha =
  \beta\). Dually, suppose that \((g f)^\op = f^\op g^\op \in \Mor(\cat C^\op)\)
  is monic, then from above we find that \(g^\op\) is monic, which dually
  implies that \(g\) is epic.
\end{proof}

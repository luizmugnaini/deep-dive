\section{\texorpdfstring{\(k\)}{k}-Linear Categories}

\begin{definition}[\(k\)-linear category]
    \label{def:k-linear-category}
    Let \(k\) be a commutative ring. A category \(\cat C\) is said to be
    \(k\)-linear if:
    \begin{enumerate}[(a)]\setlength\itemsep{0em}
        \item For any pair of objects \(X, Y \in \cat C\) the morphism collection
              \(\Mor_{\cat C}(X, Y)\) is a \(k\)-module.

        \item Consider the following diagram in \(\cat C\):
              \[
                  \begin{tikzcd}
                      X \ar[r, "u"]
                      &A \ar[r, shift left, "f"] \ar[r, shift right, "g"']
                      &B \ar[r, "v"]
                      &Y
                  \end{tikzcd}
              \]
              Then the following relations are satisfied, which mimic the familiar
              distributive: given any two scalars \(r, t \in k\) we have equalities
              \begin{align*}
                  v (r f + t g) & = r (v f) + t (v g), \\
                  (r f + t g) u & = r (f u) + t (g u).
              \end{align*}

        \item The category \(\cat C\) admits finite products and finite coproducts.
    \end{enumerate}
    A category \(\cat D\) is said to be an \emph{additive category} if \(\cat D\) is
    \(\Z\)-linear.
\end{definition}

\begin{corollary}[Initial and final objects]
    \label{cor:k-linear-has-initial-and-final-obj}
    If \(\cat C\) is a \(k\)-linear category, then \(\cat C\) has an \emph{initial}
    and \emph{final} object.
\end{corollary}

\begin{proof}
    Since \(\cat C\) has all finite products and coproduct, the product \(F\) of the
    empty family of objects of \(\cat C\) is the final object of \(\cat C\), and the
    coproduct \(I\) of the empty family is the initial object of \(\cat C\).
\end{proof}

\begin{definition}[\(k\)-linear functor]
    \label{def:k-linear-functor}
    Given \(k\)-categories \(\cat C\) and \(\cat D\), we define a
    \emph{\(k\)-linear functor} between \(\cat C \to \cat D\) to be a functor
    \(F: \cat C \to \cat D\) such that for all parallel morphisms
    \(f, g: A \para B\) and scalars \(r, t \in k\) we have
    \[
        F(r f + t g) = r (F f) + t (F g),
    \]
    for any \(A, B \in \cat C\). In other words, the induced map
    \(\Hom_{\cat C}(A, B) \to \Hom_{\cat D}(F A, F B)\) is a morphism of abelian
    groups.
\end{definition}

\begin{corollary}
    \label{cor:opposite-category-k-linear}
    If \(\cat C\) is a \(k\)-linear category for some commutative ring \(k\) then
    \(\cat C^{\op}\) is \(k\)-linear.
\end{corollary}

\begin{definition}[Biproduct]
    \label{def:biproduct}
    Let \((X_1, \dots, X_n)\) be a finite family of objects of a \(k\)-linear
    category \(\cat C\). A pair \((X, (p_j, q_j)_{j=1}^n)\) is said to be a
    \emph{biproduct} of the family \((X_j)_j\) if:
    \begin{enumerate}[(a)]\setlength\itemsep{0em}
        \item \(X\) is an object of \(\cat C\).

        \item \(p_j: X \to X_j\) and \(q_j: X_j \to X\) are morphisms in \(\cat C\) for each
              \(1 \leq j \leq n\).

        \item The pairs \((p_j, q_j)_j\) satisfy the following rules:
              \begin{enumerate}[(1)]\setlength\itemsep{0em}
                  \item \(\sum_{j=1}^n q_j p_j = \Id_X\).
                  \item For each \(1 \leq i \neq j \leq n\) we have \(p_i q_j = 0\), and on the other
                        hand \(p_j q_j = \Id_{X_j}\).
              \end{enumerate}
    \end{enumerate}
    In other words, \(X\) is both a product and a coproduct of the family in a
    compatible way.
\end{definition}

\begin{theorem}
    \label{thm:product-iff-admits-biproduct}
    Let \((X_1, \dots, X_n)\) be a finite family of objects of a \(k\)-linear
    category \(\cat C\).
    \begin{enumerate}[(a)]\setlength\itemsep{0em}
        \item A pair \((P, (p_j: P \to X_j)_{j=1}^n)\) is a \emph{product} of
              \((X_j)_j\) if and only if there exists a family of morphisms
              \((q_j: X_j \to P)_{j=1}^n\) in \(\cat C\) such that \((P, (p_j, q_j)_j)\) is
              the \emph{biproduct} of \((X_j)_j\).

        \item A pair \((C, (q_j: X_j \to X)_{j=1}^n)\) is a \emph{coproduct} of the family
              \((X_j)_j\) if and only if there exists a collection of morphisms
              \((p_j: C \to X_j)_{j=1}^n\) for which the pair \((C, (p_j, q_j)_j)\) is the
              \emph{biproduct} of \((X_j)_j\).
    \end{enumerate}
\end{theorem}

\begin{proof}
    We shall prove only the first item since the second is merely its dual. Suppose
    that \(P\) is a product, and define morphisms \(\delta_{j i}: X_i \to X_j\) for each
    pair \(1 \leq i, j \leq n\) where
    \[
        \delta_{j i} =
        \begin{cases}
            \Id_{X_j}, & \text{if } i = j     \\
            0,         & \text{if } i \neq j. \\
        \end{cases}
    \]
    By the product universal property, for each \(1 \leq i \leq n\)
    there exists a unique morphism \(q_i: X_i \to P\) making the following diagram
    commute for all \(1 \leq j \leq n\):
    \[
        \begin{tikzcd}
            X_i
            \ar[d, dashed, "q_i"']
            \ar[rd, bend left, "\delta_{j i}"]
            & \\
            X \ar[r, "p_j"']
            &X_j
        \end{tikzcd}
    \]

    Now that we have constructed a family \((q_j: X_j \to P)_j\) notice that for each
    \(1 \leq j \leq n\) one has
    \[
        p_j \sum_{i=1}^n q_i p_i
        = \sum_{i=1}^n p_j (q_i p_i)
        = \sum_{i=1}^{n} (p_j q_i) p_i
        = \sum_{i=1}^{n} \delta_{j i} p_i
        = p_j
        = p_j \Id_P
    \]
    Notice that since both \(\Id_P\) and \(\sum_{i=1}^{n} q_i p_i\) satisfy the
    commutativity property specified above for any \(1 \leq j \leq n\), we can use the
    universal property of \(P\) to obtain that
    \[
        \sum_{i=1}^{n} q_i p_i = \Id_P.
    \]
    Therefore the pair \((P, (p_j, q_j)_j)\) satisfies all the requirements to be
    the biproduct of \((X_j)_j\).

    Suppose, on the contrary, that there exists a family \((q_j: X_j \to P)_j\) of
    morphisms such that \((P, (p_j, q_j)_j)\) is a biproduct of \((X_j)_j\). Let
    \(Y \in \cat C\) be any object, and consider a collection of morphisms
    \((f_j: Y \to X_j)_{j=1}^n\). Notice that the map \(\sum_{i=1}^{n} q_i f_i: Y \to X\)
    is such that
    \[
        p_j \sum_{i=1}^n q_i f_i
        = \sum_{i=1}^{n} p_j (q_i f_i)
        = \sum_{i=1}^{n} (p_j q_i) f_i
        = f_j
    \]
    for each \(1 \leq j \leq n\). Moreover, if \(f: Y \to X\) is another morphism such that
    \(p_j f = f_j\) for each \(j\), then
    \[
        f
        = \Id_X f
        = \Big( \sum_{i=1}^n q_i p_i \Big) f
        = \sum_{i=1}^{n} (q_i p_i) f
        = \sum_{i=1}^{n} q_i (p_i f)
        = \sum_{i=1}^{n} q_i f_i.
    \]
    This shows that \(P\) enjoys the universal property of a product.
\end{proof}

\begin{corollary}[Finite products and coproducts are isomorphic]
    \label{cor:k-linear-cat-finite-coprod-prod-are-iso}
    Let \(\cat C\) be a \(k\)-linear category. Every finite collection of objects of
    \(\cat C\) has \emph{isomorphic product and coproduct}.
\end{corollary}

\begin{corollary}[Zero object]
    \label{cor:k-linear-category-has-zero-object}
    A \(k\)-linear category has a \emph{zero object}.
\end{corollary}

\section{Properties of \texorpdfstring{\(k\)}{k}-Linear Categories}

\begin{definition}[Kernels \& cokernels]
    \label{def:k-linear-cat-kernel}
    Let \(f: X \to Y\) be a morphism in a \(k\)-linear category \(\cat C\). A
    \emph{kernel} of \(f\), if it exists, is a pair \((U, u)\) where
    \(U \in \cat C\) and \(u: U \to X\) is a morphism such that
    \begin{enumerate}[(a)]\setlength\itemsep{0em}
        \item \(f u = 0\).
        \item For any morphisms \(u': U' \to X\) in \(\cat C\) such that \(f u' = 0\),
              there exists a unique morphism \(\phi: U' \to U\) such that \(u' = u g\):
              \[
                  \begin{tikzcd}
                      U \ar[r, "u"] \ar[rr, bend left=50, "0" description]
                      &X \ar[r, "f"]
                      &Y
                      \\
                      U' \ar[u, "\phi", dashed] \ar[ru, "u'"'] \ar[rru, bend right=35, "0" description]
                      & &
                  \end{tikzcd}
              \]
    \end{enumerate}

    A \emph{cokernel} of \(f\), if it exists, is a pair \((C, c)\) where
    \(C \in \cat C\) and \(c: Y \to C\) such that
    \begin{enumerate}[(a)]\setlength\itemsep{0em}
        \item \(c f = 0\).

        \item If \(c': Y \to C'\) is a morphism such that \(c' f = 0\), then there exists
              a unique \(\psi: C \to C'\) for which the following diagram commutes:
              \[
                  \begin{tikzcd}
                      X
                      \ar[r, "f"]
                      \ar[rrd, bend right=35, "0" description]
                      \ar[rr, bend left=50, "0" description]
                      &Y
                      \ar[r, "c"]
                      \ar[dr, "c'"']
                      &C \ar[d, "\psi", dashed]
                      \\
                      &
                      &C'
                  \end{tikzcd}
              \]
    \end{enumerate}
\end{definition}

\begin{corollary}[Kernel is subobject \& cokernel is quotient]
    \label{cor:k-linear-cat-kernel-is-subobject}
    Given a morphism \(f: X \to Y\) admitting kernel \((U, u)\) and cokernel
    \((C, c)\) in a \(k\)-linear category \(\cat C\). Then the isomorphism class of
    \((U, u)\) is a subobject of \(X\), on the other hand, the isomorphism class of
    \((C, c)\) is a quotient of \(Y\).
\end{corollary}

\begin{proof}
    Indeed, given a morphism \(v: V \to U\) such that \(u v = 0\) then \(u v = 0 = u
    0\) and by the universal property of the kernel, the unicity of \(v\) and \(0\)
    yields \(v = 0\). Let \(w: C \to Z\) be a morphism such that \(w c = 0\), then \(w
    c = 0 = 0 c\) but then by the universal property of cokernels we obtain \(w = 0\).
\end{proof}

\begin{lemma}
    \label{lem:k-linear-cat-monic-epic-iff-ker-coker-zero}
    In a \(k\)-linear category \(\cat C\):
    \begin{enumerate}[(a)]\setlength\itemsep{0em}
        \item A morphism \(f\) is \emph{monic} if and only if \(\ker f = 0\).
        \item A morphism \(g\) is \emph{epic} if and only if \(\coker f = 0\).
    \end{enumerate}
\end{lemma}

\begin{definition}[(Co)image]
    \label{def:k-linear-cat-image-coimage}
    Let \(\cat C\) be a \(k\)-category and \(f: X \to Y\) a morphism of \(\cat C\). We
    define, whenever possible\footnote{Not all morphisms admit an image or coimage
        in a \(k\)-linear category since these constructions depend on the existence
        of kernels and cokernels of certain maps---which cannot be always ensured in
        this case.}, the following objects:
    \begin{enumerate}[(a)]\setlength\itemsep{0em}
        \item The \emph{image} of \(f\) is given by
              \(\im f = \ker(\coker f)\)\footnote{Here we are abusing the notation:
                  \(\coker f\) stands for the \emph{morphism} associated to the cokernel of
                  \(f\), while \(\ker(\coker f)\) is the kernel \emph{object} associated to
                  the morphism \(\coker f\). The same analogous abuse of notation is used
                  while defining the coimage of \(f\).}.

        \item The \emph{coimage} of \(f\) is given by \(\coim f = \coker(\ker f)\).
    \end{enumerate}
    In other words, we have the following commutative diagram:
    \[
        \begin{tikzcd}
            \ker f \ar[r, tail]
            &X \ar[r, "f"] \ar[d, two heads]
            &Y \ar[r, two heads]
            &\coker f
            \\
            &\coim f \ar[r, dashed, "\overline f"']
            &\im f \ar[u, tail]
            &
        \end{tikzcd}
    \]
    which describes the canonical decomposition of \(f\).
\end{definition}

\begin{definition}[Abelian category]
    \label{def:abelian-category}
    A \(k\)-linear category is said to be \emph{\(k\)-abelian} if:
    \begin{enumerate}[(a)]\setlength\itemsep{0em}
        \item Every morphism of \(\cat C\) admits both kernel and cokernel.
        \item Every morphism \(f \in \cat C\) has an associated canonical isomorphism
              \(\overline f: \coim f \isoto \im f\).
    \end{enumerate}
\end{definition}

\begin{corollary}
    \label{cor:abelian-cat-iff-op-abelian}
    A category \(\cat C\) is \(k\)-abelian if and only if the opposite category
    \(\cat C^{\op}\) is \(k\)-abelian.
\end{corollary}

\begin{lemma}[Isomorphisms in abelian categories]
    \label{lem:abelian-cat-iso-iff-monic-and-epic}
    In a \(k\)-abelian category, a morphism is an isomorphism if and only if it is
    monic and epic.
\end{lemma}

\begin{proof}
    It is trivial that an isomorphism is both monic and epic, we shall prove the
    converse Let \(f: X \to Y\) be both monic and epic in a \(k\)-abelian category
    \(\cat C\). Since \(f\) is monic then \(\ker f = 0\) and therefore
    \[
        \coim f = \coker(\ker f) = \coker(0) = (X, \Id_X).
    \]
    On the other hand, since \(f\) is epic then \(\coker f = 0\) and thus
    \[
        \im f = \ker(\coker f) = \ker(0) = (Y, \Id_Y).
    \]
    This shows that \(f = \overline f\). Now since \(\overline f\) is an isomorphism
    due to the fact that \(\cat C\) is abelian, then \(f\) is an isomorphism.
\end{proof}

\begin{theorem}
    \label{thm:k-linear-cat-all-co-kernels-then-admits-all-pullbacks-pushouts}
    Let \(\cat C\) be a \(k\)-linear category:
    \begin{enumerate}[(a)]\setlength\itemsep{0em}
        \item If \(\cat C\) is such that every morphism admits a kernel. Then every pair
              of compatible morphisms of \(\cat C\) admit a pullback.

        \item If \(\cat C\) is such that every morphism admits a cokernel. Then every
              pair of compatible morphisms of \(\cat C\) admit a pushout.
    \end{enumerate}
\end{theorem}

\begin{proof}
    \begin{enumerate}[(a)]\setlength\itemsep{0em}
        \item Let \(f: Y \to X\) and \(g: Z \to X\) be morphisms and let
              \(\pi_Y: Y \times Z \to Y\) and \(\pi_Z: Y \times Z \to Z\) be the canonical projections. Let
              \((U, u)\) be the kernel of the morphism
              \(f \pi_Y - g \pi_Z: Y \times Z \to X\). Defining \(u_Y: U \to Y\) as
              \(u_Y \coloneq \pi_Y u\) and \(u_Z: U \to Z\) as \(u_Z \coloneq \pi_Z u\), we have from definition
              that \((f \pi_Y - g \pi_Z) u = 0\) and therefore \(f u_Y = g u_Z\). Hence the
              following diagram commutes
              \[
                  \begin{tikzcd}
                      U \ar[rrd, "u_Z", bend left]
                      \ar[ddr, "u_Y"', bend right]
                      \ar[rd, "u", tail]
                      & &
                      \\
                      &Y \times Z \ar[r, "\pi_Z"] \ar[d, "\pi_Y"']
                      &Z \ar[d, "g"]
                      \\
                      &Y \ar[r, "f"']
                      &X
                  \end{tikzcd}
              \]

              To see that \((U, u_Y, u_Z)\) is the pullback of \((f, g)\), let
              \((W, w_Y: W \to Y, w_Z: W \to Z)\) be a triple in \(\cat C\) satisfying
              \(f w_Y = g w_Z\). From the product universal property we have a unique morphism
              \(w': W \to Y \times Z\) such that the following diagram commutes
              \[
                  \begin{tikzcd}
                      &W \ar[rdd, bend left, "w_Z"]
                      \ar[ldd, bend right, "w_Y"']
                      \ar[d, "w'", dashed]
                      &
                      \\
                      &Y \times Z \ar[ld, "\pi_Y"'] \ar[rd, "\pi_Z"]
                      &
                      \\
                      Y & &Z
                  \end{tikzcd}
              \]
              Therefore
              \[
                  (f \pi_Y - g \pi_Z) w' = f \pi_Y w' - g \pi_Z w' = f w_Y - g w_Z = 0.
              \]
              Hence from the universal property of the kernel of \(f \pi_Y - g \pi_Z\) there
              exists a unique morphism \(w: W \to U\) such that the following diagram commutes:
              \[
                  \begin{tikzcd}
                      &W
                      \ar[ld, "w"', dashed]
                      \ar[d, "w'"] & &
                      \\
                      U
                      \ar[r, "u"']
                      &Y \times Z
                      \ar[rr, "f \pi_Y - g \pi_Z"']
                      & &X
                  \end{tikzcd}
              \]
              This shows that the following diagram commutes:
              \[
                  \begin{tikzcd}
                      W \ar[rrd, "w_Z", bend left]
                      \ar[ddr, "w_Y"', bend right]
                      \ar[rd, "w", dashed]
                      & &
                      \\
                      &U \ar[r, "u_Z"] \ar[d, "u_Y"']
                      \ar[rd, phantom, very near start, "\lrcorner"]
                      &Z \ar[d, "g"]
                      \\
                      &Y \ar[r, "f"']
                      &X
                  \end{tikzcd}
              \]

        \item The proof of this second fact becomes dual to item (a) as soon as we use
              \cref{cor:k-linear-cat-finite-coprod-prod-are-iso}.
    \end{enumerate}
\end{proof}

\begin{corollary}
    \label{cor:abelian-cat-has-all-pullbacks}
    In a \(k\)-abelian category, every pair of compatible morphisms admit a pullback
    and pushout.
\end{corollary}

\begin{lemma}
    \label{lem:pullback-pushout-mono-epi}
    Let \(\cat C\) be a \(k\)-abelian category:
    \begin{enumerate}[(a)]\setlength\itemsep{0em}
        \item Let
              \[
                  \begin{tikzcd}
                      P \ar[r, "p_2"] \ar[d, "p_1"']
                      \ar[rd, phantom, "\lrcorner", very near start]
                      &X_2 \ar[d, "f_2"] \\
                      X_1 \ar[r, "f_1"'] & X
                  \end{tikzcd}
              \]
              be a pullback square in \(\cat C\), then:
              \begin{enumerate}[(i)]\setlength\itemsep{0em}
                  \item If \(f_1\) is a monomorphism, then so is \(p_2\).
                  \item If \(f_1\) is an epimorphism, then so is \(p_2\).
              \end{enumerate}

        \item Let
              \[
                  \begin{tikzcd}
                      Y \ar[r, "g_2"] \ar[d, "g_1"']
                      &Y_2 \ar[d, "q_2"] \\
                      Y_1 \ar[r, "q_1"']
                      & Q \ar[lu, "\ulcorner", phantom, very near start]
                  \end{tikzcd}
              \]
              be a pushout square in \(\cat C\), then:
              \begin{enumerate}[(i)]\setlength\itemsep{0em}
                  \item If \(g_2\) is a monomorphism, then so is \(q_1\).
                  \item If \(g_2\) is an epimorphism, then so is \(q_1\).
              \end{enumerate}
    \end{enumerate}
\end{lemma}

\begin{proof}
    \begin{enumerate}[(a)]\setlength\itemsep{0em}
        \item
              \begin{enumerate}[(i)]\setlength\itemsep{0em}
                  \item Suppose \(f_1\) is a monomorphism. Let \(u: U \to P\) be a morphism in
                        \(\cat C\) such that \(p_2 u = 0\), then
                        \[
                            (f_1 p_1) u = (f_2 p_2) u = f_2 (p_2 u) = f_2 0 = 0.
                        \]
                        However, since \(f_1 (p_1 u) = 0 = f_1 0\) and \(f_1\) is monic, then
                        \(p_1 u = 0\). This shows that the triple \((U, 0: U \to X_1, 0: U \to X_2)\)
                        has \(u\) as its unique corresponding morphism associated to the universal
                        property of the pullback. Since \(u\) is unique and \(0: U \to P\) also
                        satisfies the needed conditions, it follows that \(u = 0\). This shows that
                        \(p_2 u = 0\) implies \(u = 0\)---which is equivalent to \(p_2\) being monic.

                  \item Suppose \(f_2\) is an epimorphism. From definition the following
                        sequence is short exact:
                        \[
                            \begin{tikzcd}
                                0 \ar[r]
                                &P \ar[rr, "p", tail]
                                &&X_1 \times X_2 \ar[rr, "f_1 \pi_1 - f_2 \pi_2", two heads]
                                &&X \ar[r]
                                &0
                            \end{tikzcd}
                        \]
                        Let \(w: X_2 \to W\) be any morphism such that \(w p_2 = 0\), and using the
                        fact that \(p_2 = \pi_2 p\) we find that \(w \pi_2 p = 0\). This implies in the
                        existence of a morphism \(g: X \to W\) such that
                        \(g (f_1 \pi_1 - f_2 \pi_2) = w \pi_2\). On the other hand, if \(\iota_1: X_1 \emb X_1
                        \times X_2\) is the canonical inclusion, then
                        \[
                            g f_1 = g (f_1 \pi_1 - f_2 \pi_2) \iota_1 = w \pi_2 \iota_1 = w 0 = 0.
                        \]
                        Since \(f_1\) is assumed to be epic then \(g = 0\), which in turn implies in
                        \(w \pi_2 = 0\). Now using the fact that the canonical projection \(\pi_2\) is
                        also epic we conclude that \(w = 0\). This shows that for any morphism with
                        \(w p_2 = 0\) one has \(w = 0\), therefore \(p_2\) is an epimorphism.
              \end{enumerate}

        \item The proof of (b-i) is dual to the proof of (a-ii), while (b-ii) is dual to
              the proof of (a-i).
    \end{enumerate}
\end{proof}

\begin{definition}[Exact sequence]
    \label{def:abelian-cat-complex-exact-sequence}
    Let \(\cat C\) be an abelian category and consider a sequence inside \(\cat C\):
    \[
        \begin{tikzcd}
            \cdots \ar[r]
            &X_{j+1} \ar[r, "f_{j+1}"]
            &X_j \ar[r, "f_j"]
            &X_{j-1} \ar[r]
            &\cdots
        \end{tikzcd}
    \]
    Such sequence is said to be a \emph{complex} if \(f_j f_{j+1} = 0\) for every
    index \(j\). On the other hand, if \(\im f_{j+1} = \ker f_j\) then we say that
    the sequence is \emph{exact} in \(X_j\).
\end{definition}

\begin{definition}[Split short exact sequence]
    \label{def:abelian-cat-split-short-exact}
    Let \(\cat C\) be an abelian category. A short exact sequence
    \[
        \begin{tikzcd}
            0 \ar[r] &X \ar[r, "f", tail] &Y \ar[r, two heads, "g"] &Z \ar[r] &0
        \end{tikzcd}
    \]
    is \emph{split} if there exists an isomorphism \(Y \iso X \oplus Z\) such that the
    following diagram commutes:
    \[
        \begin{tikzcd}
            0 \ar[r]
            &X \ar[r, "f", tail] \ar[d, equals]
            &Y \ar[r, two heads, "g"] \ar[d, "\dis"]
            &Z \ar[r] \ar[d, equals]
            &0
            \\
            0 \ar[r]
            &X \ar[r, tail]
            &X \oplus Z \ar[r, two heads]
            &Z \ar[r]
            &0
        \end{tikzcd}
    \]
\end{definition}

\begin{theorem}
    \label{thm:abelian-cat-split-equivalent-conditions}
    Let \(\cat C\) be an abelian category and consider the following short exact
    sequence:
    \[
        \begin{tikzcd}
            0 \ar[r] &X \ar[r, "f", tail] &Y \ar[r, two heads, "g"] &Z \ar[r] &0
        \end{tikzcd}
    \]
    The following conditions are equivalent:
    \begin{enumerate}[(a)]\setlength\itemsep{0em}
        \item The sequence is split.

        \item The morphism \(f\) is a split monomorphism.

        \item The morphism \(g\) is a split epimorphism.
    \end{enumerate}
\end{theorem}

\begin{lemma}
    \label{lem:abelian-cat-middle-iso-if-lateral-are-iso}
    Let \(\cat C\) be an abelian category and consider the following commutative
    diagram with short exact rows:
    \[
        \begin{tikzcd}
            0 \ar[r]
            &A \ar[r, tail] \ar[d, "\alpha"']
            &B \ar[r, two heads] \ar[d, "\beta"]
            &C \ar[r] \ar[d, "\gamma"]
            &0
            \\
            0 \ar[r]
            &A' \ar[r, tail]
            &B' \ar[r, two heads]
            &C' \ar[r]
            &0
        \end{tikzcd}
    \]
    If \(\alpha\) and \(\gamma\) are isomorphisms, then so is \(\beta\).
\end{lemma}

\begin{theorem}
    \label{thm:abelian-cat-completing-diagram}
    Let \(\cat C\) be an abelian category.
    \begin{enumerate}[(a)]\setlength\itemsep{0em}
        \item Given a diagram in \(\cat C\) with an exact bottom row:
              \[
                  \begin{tikzcd}
                      & & &X_2 \ar[d, "f_2"] & \\
                      0 \ar[r]
                      &X_0 \ar[r, tail, "f_0"]
                      &X_1 \ar[r, two heads, "f_1"]
                      &X \ar[r]
                      &0
                  \end{tikzcd}
              \]
              Then we can complete this diagram with the pullback \((P, p_1, p_2)\) of the
              pair \((f_1, f_2)\):
              \[
                  \begin{tikzcd}
                      0 \ar[r]
                      &X_0
                      \ar[d, equals]
                      \ar[r, "p_0", tail]
                      &P
                      \ar[rd, "\lrcorner", phantom, very near start]
                      \ar[r, "p_2", two heads]
                      \ar[d, "p_1"']
                      &X_2
                      \ar[d, "f_2"]
                      \ar[r]
                      &0 \\
                      0 \ar[r]
                      &X_0 \ar[r, tail, "f_0"']
                      &X_1 \ar[r, two heads, "f_1"']
                      &X \ar[r]
                      &0
                  \end{tikzcd}
              \]

              Reciprocally, given a commutative diagram with exact rows:
              \[
                  \begin{tikzcd}
                      0 \ar[r]
                      &X_0
                      \ar[d, equals]
                      \ar[r, "p_0'", tail]
                      &P'
                      \ar[r, "p_2'", two heads]
                      \ar[d, "p_1'"']
                      &X_2
                      \ar[d, "f_2"]
                      \ar[r]
                      &0 \\
                      0 \ar[r]
                      &X_0 \ar[r, tail, "f_0"']
                      &X_1 \ar[r, two heads, "f_1"']
                      &X \ar[r]
                      &0
                  \end{tikzcd}
              \]
              Then there exists an isomorphism \(f: P' \isoto P\) such that the following
              diagram commutes:
              \[
                  \begin{tikzcd}
                      0
                      \ar[r]
                      &X_0
                      \ar[dddd, equals]
                      \ar[rr, "p_0", tail]
                      \ar[rrdd, "p_0'"', near start, tail, crossing over]
                      &
                      &P
                      \ar[rr, "p_2", two heads]
                      \ar[dddd, "p_1"', bend right=50]
                      &
                      &X_2
                      \ar[dddd, "f_2"]
                      \ar[r]
                      &0
                      \\
                      &
                      &
                      &
                      &
                      &
                      &
                      \\
                      &
                      &
                      &P'
                      \ar[uu, "f"', "\dis"]
                      \ar[uurr, "p_2'"', two heads]
                      \ar[dd, "p_1'"]
                      &
                      &
                      &
                      \\
                      &
                      &
                      &
                      &
                      \\
                      0
                      \ar[r]
                      &X_0
                      \ar[rr, tail, "f_0"']
                      &
                      &X_1
                      \ar[rr, two heads, "f_1"']
                      &
                      &X
                      \ar[r]
                      &0
                  \end{tikzcd}
              \]


        \item Given a diagram in \(\cat C\) with an exact bottom row:
              \[
                  \begin{tikzcd}
                      0 \ar[r]
                      &Y \ar[r, tail, "g_1"] \ar[d, "g_2"']
                      &Y_1 \ar[r, two heads, "g_0"]
                      &Y \ar[r]
                      &0
                      \\
                      &Y_2 & & &
                  \end{tikzcd}
              \]
              Then we can complete this diagram with the pushout \((Q, q_1, q_2)\) of the
              pair \((g_1, g_2)\):
              \[
                  \begin{tikzcd}
                      0 \ar[r]
                      &Y \ar[d, "g_2"']
                      \ar[r, "g_1", tail]
                      &Y_1
                      \ar[r, "g_0", two heads]
                      \ar[d, "q_1"]
                      &Y_0
                      \ar[d, equals]
                      \ar[r]
                      &0
                      \\
                      0 \ar[r]
                      &Y_2 \ar[r, tail, "q_2"']
                      &Q \ar[r, two heads, "q_0"']
                      \ar[ul, "\ulcorner", phantom, very near start]
                      &Y_0 \ar[r]
                      &0
                  \end{tikzcd}
              \]

              Reciprocally, given a commutative diagram with exact rows:
              \[
                  \begin{tikzcd}
                      0 \ar[r]
                      &Y \ar[d, "g_2"']
                      \ar[r, "g_1", tail]
                      &Y_1
                      \ar[r, "g_0", two heads]
                      \ar[d, "q_1'"]
                      &Y_0
                      \ar[d, equals]
                      \ar[r]
                      &0
                      \\
                      0 \ar[r]
                      &Y_2 \ar[r, tail, "q_2'"']
                      &Q' \ar[r, two heads, "q_0'"']
                      &Y_0 \ar[r]
                      &0
                  \end{tikzcd}
              \]
              Then there exists an isomorphism \(g: P' \isoto P\) such that the following
              diagram commutes:
              \[
                  \begin{tikzcd}
                      0
                      \ar[r]
                      &Y
                      \ar[rr, "g_1", tail]
                      \ar[dddd, "g_2"']
                      &
                      &Y_1
                      \ar[rr, "g_0", two heads]
                      \ar[dddd, "q_1"', bend right=50]
                      \ar[dd, "q_1'"']
                      &
                      &Y_0
                      \ar[dddd, equals]
                      \ar[r]
                      &0
                      \\
                      &
                      &
                      &
                      &
                      &
                      &
                      \\
                      &
                      &
                      &Q'
                      \ar[dd, "g", "\dis"']
                      \ar[ddrr, "q_0'", two heads]
                      &
                      &
                      &
                      \\
                      &
                      &
                      &
                      &
                      \\
                      0
                      \ar[r]
                      &Y_2
                      \ar[rr, tail, "q_2"']
                      \ar[rruu, tail, "q_2'", near start, crossing over]
                      &
                      &Q
                      \ar[rr, two heads, "q_0"']
                      &
                      &Y_0
                      \ar[r]
                      &0
                  \end{tikzcd}
              \]
    \end{enumerate}
\end{theorem}

\begin{proof}
    We shall only prove item (a) since item (b) follows from duality. Using the
    universal property for the triple
    \((X_0, f_0: X_0 \to X_1, 0: X_0 \to X_2)\)---which satisfies
    \(f_1 f_0 = 0 = f_2 0\)---we have a unique morphism \(p_0: X_0 \to P\) for which
    satisfies \(p_1 p_0 = f_0\) and \(p_2 p_0 = 0\). It suffices for us to prove
    that \(p_0 = \ker p_2\): let \(u: U \to P\) be such that \(p_2 u = 0\), then
    \[
        f_1 p_1 u = f_2 p_2 u  = f_2 0 = 0.
    \]
    Therefore there exists \(u': U \to X_0\) such that \(f_0 u' = p_1 u\), moreover,
    we have
    \[
        p_2 p_0 u' = 0 = p_2 u\, \text{ and }\, p_1 p_0 u' = f_0 u' = p_1 u.
    \]
    By the pullback universal property we can use unicity to obtain \(u = p_0
    u'\). Notice that since \(f_0 = p_1 p_0\) is a monomorphism, then \(p_0\) is
    also a monomorphism.

    For the reciprocal, notice that the triple \((P', p_1', p_2')\) induces a unique
    morphism \(f: P' \to P\) such that \(p_1 f = p_1'\) and \(p_2 f = p_2'\). On the
    other hand, if we consider the triple
    \((X_0, f_0: X_0 \to X_1, 0: X_0 \to X_2)\) one has that the map
    \(f p_0': X_0 \to P\) satisfies the commutativity conditions:
    \begin{gather*}
        p_1 (f p_0') = (p_1 f) p_0' = p_1' p_0' = f_0 = p_1 p_0, \\
        p_2 (f p_0') = (p_2 f) p_0' = p_2' p_0' = 0 = p_2 p_0,
    \end{gather*}
    as does the map \(p_0: X_0 \to P\), hence by uniqueness we find that
    \(f p_0' = p_0\). Now by \cref{lem:abelian-cat-middle-iso-if-lateral-are-iso} we
    obtain that \(f\) is an isomorphism.
\end{proof}

\section{Ideals}

\begin{definition}[Bilateral ideal]
    \label{def:k-linear-cat-bilateral-ideal}
    Let \(\cat C\) be a \(k\)-linear category. We define a \emph{bilateral ideal} of
    \(\cat C\) to be a category \(\mathcal{I}\) with
    \(\Obj \mathcal{I} \subseteq \Obj \cat C\) and where \(\mathcal{I}(X, Y) \subseteq \Mor_{\cat C}(X, Y)\) is a
    submodule for any two \(X, Y \in \cat C\) such that:
    \begin{enumerate}[(a)]\setlength\itemsep{0em}
        \item If \(f \in \mathcal{I}(X, Y)\) and \(g \in \Mor_{\cat C}(Y, Z)\) then \(g f \in \mathcal{I}(X, Z)\).

        \item If \(f \in \mathcal{I}(X, Y)\) and \(h \in \Mor_{\cat C}(W, X)\) then \(f h \in \mathcal{I}(W, Y)\).
    \end{enumerate}
\end{definition}

\begin{definition}[Quotient category]
    \label{def:k-linear-cat-quotient-category}
    Let \(\cat C\) be a \(k\)-linear category and \(\mathcal{I}\) be a bilateral ideal of
    \(\cat C\). We define the \emph{quotient category} \(\cat C/\mathcal{I}\) to be the
    \(k\)-linear category composed of the objects of \(\cat C\) and for each
    \(X, Y \in \cat C/\mathcal{I}\) we have
    \[
        \Mor_{\cat C/\mathcal{I}}(X, Y) \coloneq \Mor_{\cat C}(X, Y)/\mathcal{I}(X, Y).
    \]
\end{definition}

\begin{definition}[Functor kernel]
    \label{def:k-linear-functor-kernel}
    Let \(F: \cat C \to \cat D\) be a \(k\)-linear functor. We define the
    \emph{kernel} of \(F\) to be the bilateral ideal \(\ker F\) of \(\cat C\) given
    by the morphisms \(f \in \Mor \cat C\) such that \(F f = 0\).
\end{definition}

\begin{proposition}
    \label{prop:k-linear-functor-quotient-kernel-equivalence--cats}
    Let \(F: \cat C \to \cat D\) be a \(k\)-linear functor that is both full and
    essentially surjective. Then there exists a unique equivalence of categories
    \(\overline F: \cat C/\ker F \isoto \cat D\) such that the following diagram
    quasi-commutes:
    \[
        \begin{tikzcd}
            \cat C \ar[d, "P"'] \ar[r, "F"]               &\cat D \\
            \cat C/\ker F \ar[ru, "\overline F"', dashed] &
        \end{tikzcd}
    \]
    where \(P: \cat C \to \cat C/\ker F\) is the canonical projection functor.
\end{proposition}

\begin{proof}
    Let \(\overline F: \cat C/\ker F \to \cat D\) be the functor given by
    \(\overline F X \coloneq F X\) for any \(X \in \cat C/\ker F\) and for each morphism
    \(f: X \to Y\) in \(\cat C\) we define
    \[
        \overline F (f + (\ker F)(X, Y)) = F f.
    \]
    So that indeed \(\overline F P = F\). Moreover, it is clear that \(\overline F\)
    inherits the properties of being full and essentially surjective from \(F\). On
    the other hand, if \(\overline F (g + (\ker F)(Z, W)) = 0\) then \(F g = 0\) and
    hence \(g \in (\ker F)(Z, W)\), thus \(\overline F\) is faithful.
\end{proof}

\section{Exact Functors}

\begin{proposition}
    \label{prop:abelian-cat-left-right-exact-functors}
    Let \(F: \cat C \to \cat D\) be a \(k\)-linear functor between \(k\)-abelian
    categories:
    \begin{enumerate}[(a)]\setlength\itemsep{0em}
        \item If \(F\) is covariant, then \(F\) is \emph{left exact} if it preserves
              kernels---that is, \(F(\ker g) = \ker(F g)\) for any \(g \in \Mor \cat C\). On the
              other hand, \(F\) is said to be \emph{right exact} if it preserves
              cokernels---for any \(g \in \Mor \cat C\) we have \(F(\coker g) = \coker(F g)\).

        \item If \(F\) is contravariant, then \(F\) is \emph{left exact} if it
              transforms kernels into cokernels---that is, \(F(\ker g) = \coker(F g)\) for any
              \(g \in \Mor \cat C\). Moreover, \(F\) is \emph{right exact} if it transforms
              cokernels into kernels---that is, \(F(\coker g) = \ker(F g)\) for any
              \(g \in \Mor \cat C\).
    \end{enumerate}
\end{proposition}


%%% Local Variables:
%%% mode: latex
%%% TeX-master: "../../deep-dive"
%%% End:

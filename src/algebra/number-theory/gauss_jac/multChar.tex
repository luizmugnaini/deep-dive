\section{Multiplicative Characters}

\begin{definition}
    We define \(\chi : \F_p^{\ast} \to \mathbb{C} \setminus \{0\}\) to be a
    multiplicative character on \(\F_p\) if it satisfies for all \(a, b \in
    \F_p\) that \(\chi(ab) = \chi(a)\chi(b)\).
\end{definition}

\begin{proposition}
    Let \(\chi\) be a multiplicative character on \(\F_p\) and \(a\) be an
    element of the finite field, then
    \begin{enumerate}[i.]
        \item \(\chi(1) = 1\).
        \item  \(\chi(a)\) is a root of \(x^{p-1} - 1\) (that is, a "\(p-1\)st"
              root of unity).
        \item \(\chi(a^{-1}) = \chi(a)^{-1} = \overline{\chi(a)}\) (complex
              conjugation).
    \end{enumerate}
\end{proposition}

\begin{proof}
    We prove only the last two. Since \(a^{p-1} = 1\) then  \(\chi(a^{p-1}) =
    \chi(1) = 1\) and also  \(\chi(a^{p-1}) = \chi^{p-1}(a) = 1\) and therefore
    the claim follows.
    For the third notice that \(1 = \chi(1) = \chi(a)\chi(a^{-1}) \Rightarrow
    \chi(a^{-1}) = \chi(a)^{-1}\).
\end{proof}

\begin{definition}
    We label \(\varepsilon : \F_p^\ast \to \{1\}\) mapping \(a \mapsto 1\) to be
    a multiplicative character.
\end{definition}

In order to extend the domain of \(\chi\) for the whole finite field, we define
\(\chi(0) := 0\) if  \(\chi \neq  \varepsilon\) and \(\varepsilon(0) := 1\).

\begin{proposition}\label{prop: sum multiplicative characters}
    Let \(\chi\) denote a multiplicative character. Then
    \[
        \sum_{a \in \F_p} \chi(a) =
        \begin{cases}
            0, & \chi \neq \varepsilon \\
            p, & \chi = \varepsilon
        \end{cases}
    \]
\end{proposition}

\begin{proof}
    Let \(b \in \F_p^\ast\) such that \(\chi(a) \neq 1\), then define \(S =
    \sum_{a \in \F_p^\ast} \chi(a)\). Thus
    \[
        S\chi(b) = \sum_{a \in \F_p^\ast} \chi(a)\chi(b) = \sum_{ab \in \F_p^\ast}
        \chi(ab) = S \Rightarrow S = 0.
    \]
\end{proof}

\begin{proposition}
    The multiplicative characters form a group:
    \begin{enumerate}[i.]
        \item \(\chi\lambda : \F_p^\ast \to \mathbb{C} \setminus \{0\} \) where
              \(a \mapsto \chi(a)\lambda(a)\).
        \item  \(\chi^{-1}: \F_p^\ast \to \mathbb{C} \setminus \{0\} \) where \(a
              \mapsto \chi(a)^{-1}\)
    \end{enumerate}
\end{proposition}

\begin{proposition}
    The group of characters is a cyclic group of order \(p-1\). If \(a \in
    \F_p^\ast\) and \(a \neq 1\) then there exists a character \(\chi\) such
    that \(\chi(a) \neq 1\).
\end{proposition}

\begin{proof}
    Let \(\langle g \rangle = \F_p^\ast\) and \(\forall a \in \F_p^\ast\) let \(a
    := g^{\ell(a)}\), then if \(\chi\) is a character, we just need to know it's
    value for \(g\) since any other will be simply this value to the power
    \(\ell(a)\) where \(a\) is the element concerned. We know that  \(\chi(g)\)
    is a  \(p-1\)st root of unity and therefore we have that the group of
    characters have order less than or equal to \(p-1\).

    We now wish to construct a generator for the group of multiplicative
    characters. Notice that if we define a character  \(\lambda(g^k) :=
    e^{\frac{2\pi i k}{p-1}}\) then we have that if \(\lambda^n = \varepsilon\) for
    some \(n\) then \(\lambda^n(g) = e^{\frac{2\pi i n}{p-1}} = \varepsilon(g) = 1\)
    and therefore the least element \(n\) such that such statement is true is
    simply \(n = p-1\). Therefore, since \(\{\varepsilon,
    \lambda,\dots,\lambda^{p-2}\}\) is a set with \(p-1\) distinct elements, we
    conclude that there are exactly \(p-1\) possible constructible characters and
    thus \(\langle \lambda \rangle \) is the group of characters.

    We now focus on the second part of the statement. Let \(a = g^{\ell(a)} \in
    \F_p^\ast\) be any element such that \(p-1 \nmid \ell(a)\), then
    \(\lambda(g^{\ell(a)}) = \lambda(g)^{\ell(a)} = e^{\frac{2 \pi
                i}{p-1}\ell(a)} \neq 1\) because we imposed \(p-1 \nmid \ell(a)\).
\end{proof}

\begin{proposition}
    Let \(a \in \F_p \setminus \{1\} \), then
    \[
        \sum_{\chi \in \langle \lambda \rangle} \chi(a) = 0.
    \]
    Where \(\lambda\) is a generator for the group of multiplicative characters.
\end{proposition}

\begin{proof}
    Just use that \(\lambda(a) \neq 1\) and then apply the same technique as in
    \ref{prop: sum multiplicative characters}.
\end{proof}

\begin{proposition}[Criterion for solutions]
    Let \(a \in \F_p^\ast\), suppose \(n \mid p-1\) and \(x^n = a\) is not a
    solvable equation over \(\F_p\). Then, there exists a character \(\chi\)
    such that
    \begin{enumerate}
        \item \(\chi^n = \varepsilon\).
        \item \(\chi(a) \neq  a\).
    \end{enumerate}
\end{proposition}

\begin{proof}
    Just take \(\langle g \rangle = \F_p^\ast\) and set \(\chi :=
    \lambda^{\frac{p-1}{n}}\) where \(\lambda(g^k) := e^{\frac{2 \pi i k}{p-1}}\)
    is the generator of the characters group. From this we conclude trivially
    that both assertions are true.
\end{proof}

\begin{proposition}
    Let \(n \mid p-1\). The number of solutions to the equation \(x^n = a\)
    satisfies
    \[
        N(x^n=a) = \sum_{\chi^n = \varepsilon} \chi(a).
    \]
    Notice that the imposition of summing over \(\chi^n = \varepsilon\) simply
    states that the order of \(\chi\) is a divisor or \(n\).
\end{proposition}

\begin{proof}
    Let \(\lambda\) be a character such that its order divides \(n\), that is
    \(\lambda(g)^n = \varepsilon(g) = 1\) and therefore there are less than of
    \(n\) such \(n\)th roots of unity, showing that there are at most \(n\)
    characters with order dividing \(n\). If for instance we take \(\lambda\) to
    be the generator of the group, then we see that \(\{\varepsilon, \lambda,
    \dots, \lambda^{n-1}\} \) are all characters that are \(n\)th roots of
    unity and also distinct, thus there are exactly \(n\) characters for which
    the order divides \(n\).

    Now we inspect three cases
    \begin{enumerate}[i.]
        \item Consider \(a = 0\), then the only solution for  \(x^n = a\) is  \(x
              =0\). Since we defined \(\chi(0) := 1\) for any \(\chi \neq
              \varepsilon\) and \(\varepsilon(0) := 0\) we end up having
              \(\sum_{\chi^n = \varepsilon} \chi(0) = \varepsilon(0) + 0 + \dots + 0
              = 1\) and therefore we got the right number of solutions.
        \item Let \(a \neq  0\) and suppose \(x^n = a\) is a solvable equation.
              Then, suppose \(x_0\) is a solution, so that \(\chi(a) = \chi(x_0^n) =
              \chi(x_0)^n = \varepsilon(x_0) = 1\) and therefore \(\sum_{\chi^n =
                  \varepsilon}\chi(a) = n\) which is the number of solutions for this
              case.
        \item Suppose \(a \neq  0\) and \(x^n = a\) is unsolvable. Define \(S :=
              \sum_{\chi^n  = \varepsilon} \chi(a)\) and then let \(\lambda\) be a
              character such that  \(\lambda(a) \neq  1\) and suppose also that
              \(\lambda^n = \varepsilon\). Notice that
              \[
                  \lambda(a) S = \sum_{\chi^n = \varepsilon} \lambda(a)\chi(a) =
                  \sum_{(\lambda \chi)^n = \varepsilon} (\lambda\chi)(a) =  S
                  \Rightarrow S = 0, \text{ since } \lambda(a) \neq  1.
              \]
              which yields the correct number of solutions.
    \end{enumerate}
\end{proof}



\section{Vector Spaces and Subspaces}

\subsection{Vector Spaces}

\begin{definition}[Vector Space]\label{def: vector space}
A set \(V\) is called a \emph{vector space} over a field \(k\) (or
\(k\)-space) if it is equipped with an internal operation \(+ : V \times V
\to V\) where \((a, b) \mapsto a + b\) and the external operation \(\cdot : k
\times V \to V\) where \((r, a) \mapsto r \cdot a\). We normally call the
elements of \(V\) as vectors, and the elements of \(k\) as scalars. Also,
these operations satisfy
\begin{enumerate}[I.]
  \item \((V, +)\) is an abelian group.
  \item Multiplication of vectors by scalars is associative and distributive
    and is unitary (that is, \(1 \cdot a = a\) for every \(a \in V\)).
\end{enumerate}
\end{definition}

\subsection{Subspaces}

\begin{definition}[Subspaces]\label{def: subspaces}
Let \(V\) be a vector space. A set \(S \subseteq V\) is called a
\emph{subspace} of \(V\) if it satisfies all properties of a vector space.
Also \(S\) is called a \emph{proper subspace} if it is not equal to the
original vector space.
\end{definition}

\begin{theorem}[Cover Avoidance]\label{thm: cover avoidance in infinite field}
A non-zero vector space \(V\) over an infinite field \(k\) is not the union
of a finite number of proper subspaces.
\end{theorem}

\begin{proof}
Suppose, for the sake of contradiction, that \(V = \bigcup_{i \in  I} S_i\)
where \(I\) is a finite indexing set and \(S_i\) are all proper subspaces of
\(V\). Then, let \(S_1\) be such that it is not contained in any other
subspace \(S_i\). Define elements \(a \in S_1 \setminus \bigcup_{i \in
I \setminus \{1\} } S_i \) and \(b \in V \setminus S_1\) and construct the
set \(A \coloneq \{ra + b \colon r \in k\}\). Notice that if we have that one element
\(ra + b \in S_1\), the fact that \(a \in S_1\) makes \((ra + b) - ra = b
\in S_1\), contradicting the assumption of \(b \not\in S_1\) thus, we can't
have \(ra + b \in S_1\). Suppose now that we let \(ra + b, r'a + b \in S_i\)
different elements, for some \(i > 1\), then \((ra + b) - (r'a + b) = (r -
r')a \in S_i\) but since \(k\) is a field, then \(a \in S_i\) contradicting
again the construction and therefore we cannot have more than one element of
\(A\) contained in \(S_i\). Now, since \(A\) is an infinite set and is
contained in \(V\), then we cannot have the equality between \(V\) and the
finite union \(\bigcup_{i \in  I} S_i\).
\end{proof}

\begin{definition}[Lattice]\label{def: lattice}
A poset \(P\) is called a \emph{lattice} if for every pair of
elements of \(P\) there exists a \emph{join} (or a least upper bound) and a
\emph{meet} (greatest lower bound). The set \(P\) is called a \emph{complete
lattice} if there exists a join and a meet for every collection of sets and
also every collection contains smallest and larger elements under the partial
order.
\end{definition}

\begin{proposition}[Intersection of subspaces]\label{prop: intersection of subspaces}
The intersection of any collection of subspaces of a given vector space is a
subspace of the original vector space. This intersection contains the
greatest lower bound of subspace that is contained in every subspace of the
intersection, then, we denote it by
\[
  \bigcap_{i \in  I} S_i = \mathrm{Glb}\{S_i \colon i \in I\}.
\]
\end{proposition}

\begin{proof}
Let \(V\) be a \(k\)-space and \(S_i\) be an arbitrary subspace. Notice that
\(0 \in \bigcup_{i \in  I} S_i\). Let elements \(u, v \in \bigcap_{i \in  I}
S_i\) so that for all subspace we have that the element \(u + vt \in S_i\)
for any given \(t \in k\), thus \(u + vt \in \bigcap_{i \in  I} S_i\), which
makes the union a vector space by itself. The claim follows.
\end{proof}

\begin{definition}[Sum of subspaces]\label{def: sum of subspaces}
Let \(V\) be a vector space and \(S_i\) be subspaces of \(V\). We define the
sum of such subspaces as
\[
  \sum_{i \in I} S_i \coloneq \left\{\sum_j s_j \colon s_j \in \bigcup_{i \in  I}
  S_i\right\}.
\]
Therefore the least upper bound under set inclusion as
\[
  \mathrm{Lub}\{S_i \colon i \in I\} = \sum_{i \in I} S_i.
\]
\end{definition}

\begin{theorem}[Subspaces form a complete lattice]
\label{thm: subspaces form a lattice}
The set containing all subspaces of \(V\), denoted by \(\mathcal{S}(V)\), is
a complete lattice under set inclusion (partial order of the set), with
smallest element \(\{0\}\) (the zero subspace) and largest element \(V\). The
meet of any collection of sets \(\{S_i \colon i \in I\}\), where \(I\) is a finite
indexing set, is
\[
  \bigcap_{i \in  I} S_i = \mathrm{Glb}\{S_i \colon i \in I\}
\]
and the join is defined as
\[
  \sum_{i \in I} S_i = \mathrm{Lub}\{S_i \colon i \in I\}.
\]
\end{theorem}

\subsection{Morphisms of Vector Spaces}

\begin{definition}[Morphisms]\label{def: k-linear morphism}
Let \(V, L\) be \(k\)-vector spaces. We say that \(\varphi : V \to L\) is a
morphism of vector spaces if it satisfies
\begin{enumerate}[I.]
  \item \(\varphi(0) = 0\), that is \(V \ni 0 \mapsto 0 \in L\).
  \item For all \(u, v \in V\), \(\varphi(u + b) = \varphi(u) +
      \varphi(v)\).
  \item For all \(a \in k\) and \(u \in V\) we have \(\varphi(au) =
      a\varphi(u)\).
\end{enumerate}
\end{definition}

Notice in fact that the first property of the morphism is in fact redundant. By
means of item two we can see that \(\varphi(0) = \varphi(0) + \varphi(0)\), for
which the item one is obtained.

\begin{definition}\label{def: linear operator}
A \(k\)-linear morphism \(f : V \to V\) is called a linear operator.
\end{definition}

\begin{proposition}\label{prop: category of vector spaces}
The \(k\)-vector spaces, together with morphism between such vector spaces,
form a category, of which we'll denote by \(\Vect_k\).
\end{proposition}

\begin{proof}
Certainly, the identity map \(\Id_V : V \to V\) is a morphism of \(k\)-vector
spaces and all of the properties come directly from the fact that \(V\) is a
vector space.

Another important feature of morphism between vector spaces is that they are
closed under composition. Let \(\varphi : V \to L\) and \(\psi : L \to U\) be
morphism between \(k\)-vector spaces, then the composition \(\psi
\varphi : V \to U\) is such that, given any \(u, v \in V\) then
\[
  \psi(\varphi(u+v)) = \psi(\varphi(u) + \varphi(v)) = \psi(\varphi(u)) +
  \psi(\varphi(v))
\]
and also, being \(a \in k\), we have
\[
  \psi(\varphi(au)) = \psi(a\varphi(u)) = a \psi(\varphi(u)).
\]
Which proves the closure under composition.

We now prove that the composition of morphism of vector spaces is
associative. Let the morphisms be as before and define yet another morphism
\(\ell : W \to V\), then
\[
  (\psi  (\varphi  \ell)) (w) = \psi(\varphi(\ell(w)) = (\psi
  \varphi)(\ell(w)) = ((\psi  \varphi)  \ell) (w).
\]

Consider the morphism \(\varphi : V \to L\), then clearly \(\varphi
\Id_V = \varphi = \Id_L  \varphi\). Together with the fact that the
collection of morphisms between \(k\)-vector spaces \(V, L\) and the
collection of morphisms between \(k\)-vector spaces \(U, W\), these two
collections are clearly disjoint for \(V \neq U\) and \(L \neq W\). This
finishes the proof of the properties needed for a category.
\end{proof}

\begin{proposition}[Initial and Final object]
The \(k \)-vector space \(0\) is a initial and final object of
\(\Vect_k\).
\end{proposition}

\begin{proof}
Essentially, we need to prove that for all \(V \in \Obj(\Vect_k)\) there
exists unique morphisms \(\varphi\) and \(\psi\) where
\[
  \begin{tikzcd}
    0
    \ar[r, bend left, dashed, "\psi"]
      &V
      \ar[l, bend left, dashed, "\varphi"]
  \end{tikzcd}
\]
Notice that, since \(\varphi\) is a morphism of \(k\)-vector spaces, we'll
need to impose \(\varphi(0) = 0\) and thus this morphism is clearly unique
and satisfies the properties needed for a morphism. Moreover, the morphism
has a unique target element, thus the image of \(\psi\) is the singleton
\(\{0\}\) which is also clearly unique and satisfies the properties of
morphism.
\end{proof}

\begin{definition}[Isomorphism]
Let \(V, L \in \Obj(\Vect_k)\). We say that \(L\) and \(V\) are
isomorphic, that is \(L \iso V\), if there exists an isomorphism \(L \to V\)
in \(\Hom(L, V)\).
\end{definition}

\section{Modules Over Rings}

\subsection{Modules}

\begin{definition}[Ring action]
\label{def:ring-action}
Let \(R\) be a ring and \(A\) an abelian group. A left-\(R\)-action on \(A\) is
given a ring morphism
\[
\mu: R \to \End_{\Ab}(M).
\]
Less compactly, given \(r, s \in R\) and a group endomorphism \(\phi: A \to A\),
the ring morphism satisfies, for every \(a, b \in A\):
\begin{enumerate}[(a)]\setlength\itemsep{0em}
\item \(\mu(r)(a + b) = \mu(r)(a) + \mu(r)(b)\).
\item \(\mu(r + s)(a) = \mu(r)(a) + \mu(s)(a)\).
\item \(\mu(r s)(a) = \mu(r)(\mu(s)(a))\).
\item \(\mu(1)(a) = a\).
\end{enumerate}
\end{definition}

\begin{definition}[Module]
\label{def:module}
Given a ring \(R\), a \emph{left-\(R\)-module} is an \emph{abelian group} \(M\)
endowed with a \emph{left-\(R\)-action} \(R \times M \to M\) mapping
\((r, m) \mapsto r m\) satisfying the following properties, for every
\(r, s \in R\) and \(m, n \in M\):
\begin{enumerate}[(a)]\setlength\itemsep{0em}
\item \(r (m + n) = r m + r n\).
\item \((r + s) m = r m + s m\).
\item \((r s) m = r (s m)\).
\item \(1 m = m\).
\end{enumerate}
Right-\(R\)-modules are defined completely analogous, with a right-\(R\)-action.
\end{definition}

\begin{definition}[Opposite ring]
\label{def:opposite-ring}
Given a ring \((R, +, \cdot)\), we define its \emph{opposite ring} \(R^{\op}\)
to be ring inheriting the elements and additive structure of \(R\), while its
multiplicative structure is given by a map \(*: R \times R \to R\) defined by
\(a * b \coloneq b a \in R\).
\end{definition}

\begin{corollary}
\label{cor:id-ring-and-opposite}
The identity map \(\Id: R \to R^{\op}\) is an \emph{isomorphism} of rings if and
only if \(R\) is \emph{commutative}.
\end{corollary}

\begin{proof}
If \(R\) is commutative, clearly the identity map is an isomorphism. Conversely,
if \(\Id\) is a morphism of rings, then
\(\Id(r s) = \Id(r) * \Id(s) = \Id(s) \Id(r) = s r\) but
\(\Id(r s) = r s\) thus \(r s = s r\).
\end{proof}

\begin{example}
\label{exp:ring-of-square-matrices}
The ring of real \(n\)-square matrices, \(M_n(\R)\), is isomorphic as a ring to
its opposite ring \(M_n(\R)^{\op}\).

Indeed, if \(\phi: M_n(\R) \to M_n(R)^{\op}\) is the map sending \(A \mapsto
A^T\) then, given \(A, B \in M_n(\R)\), we have
\[
\phi(A B) = (A B)^T = B^T A^T = A^T * B^T = \phi(A) * \phi(B).
\]
That is, \(\phi\) is a morphism of rings. Injectivity and surjectivity are
clear, thus \(\phi\) is an isomorphism of rings.
\end{example}

\begin{example}
\label{exp:right-module-to-left-module}
For a \emph{commutative} ring \(R\), left-\(R\)-modules can be bijectively
assigned to a right-\(R\)-modules. To see that given an abelian group \(M\), let
\({}_RM\) be a left-\(R\)-module structure on \(M\), we proceed by constructing
a bijection \({}_RM \isoto M_R\), where \(M_R\) denotes a right-\(R\)-module on
\(M\). For every element \(r \in R\) and \(m \in M\), map \(r m \mapsto m
r\). Since \(R\) is commutative, indeed
\[
(r s) m = r (s m) \longmapsto m (r s) = m (s r) = (m s) r,
\]
translating the multiplication by \(r s\) from a left-module to a right-module
accordingly.
\end{example}

\subsection{Vector Spaces}

\begin{example}[Vector spaces]
\label{exp:vector-space-is-k-module}
A module over a field \(k\) is nothing more than a \(k\)-vector space, thus
\(\Mod{k} = \Vect_k\).
\end{example}

\begin{example}[\({k[x]}\)-module structure on \(V\)]
\label{exp:k[x]-module-structure-on-V}
Let \(k\) be a field, \(V\) be a \(k\)-vector space, and \(\phi: V \to V\) be an
endomorphism. We consider the ring action \(\mu: k \emb \End_{\Vect_k}(V)\)
given by \(a \mapsto a \Id_V\), which by
\cref{prop:universal-property-polynomial-rings} implies in the existence of a
unique morphism of rings \(\Psi: k[x] \unique \End_{\Vect_k}(V)\) such that the
following diagram commutes
\[
\begin{tikzcd}
k \ar[d, hook] \ar[r, "\mu"] & \End_{\Vect_k}{(V)} \\
k[x] \ar[ru, dashed, bend right, "\Psi"']&
\end{tikzcd}
\]
and that \(x \xmapsto{\Psi} \phi\). Since \(k\) and \(x\) generate \(k[x]\), for
any \(f(x) \coloneq \sum_{j=1}^n a_j x^j \in k[x]\), we have a mapping
\[
\begin{tikzcd}
f(x) = \sum_{j=1}^n a_j x^j \ar[r, mapsto, "\Psi"]
& a_0 \Id_V + a_1 \phi + \dots + a_{n-1} \phi^{n-1} + a_n \phi^n \coloneq f(\phi).
\end{tikzcd}
\]

The ring action \(\Psi: k[x] \to \End_{\Vect_k}(V)\) induces the structure of a
left-\(k[x]\)-module on \(V\) as
\[
p(x) \cdot v \coloneq p(\phi)(v),
\]
for all \(p(x) \in k[x]\) and \(v \in V\).
\end{example}

\begin{corollary}
\label{cor:k[x]-are-vector-spaces-with-endo}
Given a field \(k\), there exists a bijection
\[
\{k[x]\text{-modules}\} \isoto
\{(V, \phi) \colon V \in \Vect_k \text{ and } \phi \in \End_{\Vect_k}(V)\}.
\]
In other words, \(k[x]\)-modules are equivalent to \(k\)-vector spaces together
with a uniquely determined \(k\)-linear endomorphism.
\end{corollary}

\begin{proof}
As constructed in \cref{exp:k[x]-module-structure-on-V}, for every \(k\)-vector
space \(V\) and \(k\)-linear endomorphism \(\phi: V \to V\), one has a unique
left-\(k[x]\)-module structure induced on \(V\). On the other hand, let \(M\) be
a left-\(k[x]\)-module and \(\psi: M \to M\) be the group morphism given by
\(m \mapsto x m\) for all \(m \in M\). Notice that for any \(a \in k\) and
\(m \in M\) we have
\[
\psi(a m) = x (a m) = (x a) m = (a x) m = a (x m) = a \psi(a m),
\]
therefore \(\psi\) is \(k\)-linear. Thus \((M, \phi)\) is the corresponding
uniquely defined \(k\)-vector space together with a \(k\)-linear endomorphism.
\end{proof}

\subsection{Category of Modules}

\begin{definition}[Morphism of modules]
\label{def:morphisms-of-modules}
Let \(R\) be a ring, and both \(M\) and \(N\) be abelian groups. Let \(M_R\) and
\(N_R\) denote right-\(R\)-modules on \(M\) and \(N\), while \({}_RM\) and
\({}_RN\) denote left-\(R\)-modules on \(M\) and \(N\). We define the the
following:
\begin{enumerate}[(a)]\setlength\itemsep{0em}
\item A \emph{morphism} between \emph{right}-\(R\)-modules \(\phi: M_R \to N_R\)
  is a morphism of abelian groups such that, for all \(r \in R\) and \(m \in M\)
  we have
  \[
  \phi(m r) = \phi(m) r.
  \]
\item A \emph{morphism} between \emph{left}-\(R\)-modules
  \(\psi: {}_RM \to {}_RN\) is a morphism of abelian groups such that, for all
  \(r \in R\) and \(m \in M\) we have
  \[
  \psi(r m) = r \psi(m).
  \]
\end{enumerate}
Morphisms of \(R\)-modules can also be compactly named \(R\)-linear morphisms.
\end{definition}

\begin{definition}[Category of \(R\)-modules]
\label{def:category-of-R-Mod}
Given a ring \(R\), we denote by \(\rMod{R}\) the category whose objects are
\emph{right-\(R\)-modules} and morphisms between them. Analogously, we define
\(\lMod{R}\) to be the category whose objects are \emph{left-\(R\)-modules} and
morphisms between them.

If \(R\) is a \emph{commutative} ring, we simply denote the category of modules
over \(R\) and morphisms between them by \(\Mod{R}\).
\end{definition}

\begin{proposition}[\(\Z\)-modules]
\label{prop:abelian-groups-are-Z-modules}
Abelian groups are \(\Z\)-modules in exactly one way. Therefore \(\lMod{\Z}\)
and \(\Ab\) are isomorphic categories.
\end{proposition}

\begin{proof}
Let \(G\) be an abelian group. Since \(\Z\) is initial in the category of rings
and \(\End_{\Ab}(G)\) forms a ring, we find that there exists a unique morphism
of rings
\[
\begin{tikzcd}
\Z \ar[r, dashed] &\End_{\Ab}(G)
\end{tikzcd}
\]
defining an action of \(\Z\) on the group \(G\).
\end{proof}

\begin{example}[\(\Q\)-vector space]
\label{exp:Q-vector-space}
Let \(G\) be an abelian group. If there exists a \(\Q\)-vector
space structure on \(G\), this structure is \emph{unique}.

Let \(\mu, \sigma: \Q \para \End_{\Ab}(G)\) be two \(Q\)-module structures on
\(G\). Since the inclusion \(\iota: \Z \emb \Q\) is an epimorphism of
\emph{rings}, it follows that, since there exists a \emph{unique} \(\Z\)-module
structure \(\Z \to \End_{\Ab}(G)\) (see
\cref{prop:abelian-groups-are-Z-modules}), it follows that
\(\mu \iota = \sigma \iota\). On the other hand, since \(\iota\) is an
epimorphism, then \(\mu = \sigma\).
\end{example}

\begin{proposition}[Zero object]
\label{prop:zero-object-in-R-Mod}
Let \(R\) be a ring. The trivial group \(0\) has a \emph{unique} \(R\)-module
structure and defines a \emph{zero object} in the category of (left or right)
\(R\)-modules.
\end{proposition}

\begin{proof}
Notice that the only \(R\)-module structure on the trivial group \(0\) is given
by \(r \cdot 0 \coloneq 0\) for all \(r \in R\). Moreover, for any \(R\)-module
\(M\), we have unique \(R\)-linear morphisms \(0 \unique M\) mapping
\(0 \mapsto 0_M\) and \(M \unique 0\) mapping \(m \mapsto 0\) for all
\(m \in M\).
\end{proof}

\begin{proposition}[Isomorphisms]
\label{prop:R-mod-iso-iff-bij}
Let \(R\) be a ring. A morphism of (left or right) \(R\)-modules is an
\emph{isomorphism} if and only if it is a \emph{bijective} set-function.
\end{proposition}

\begin{proof}
We prove for right-\(R\)-modules, the proof for left-\(R\)-modules is completely
analogous. Let \(\phi: M \to N\) be an isomorphism of \(R\)-modules and
\(\psi: N \to M\) be its inverse. Since \(\phi \psi = \Id_N\), then
\(\im \phi = N\), that is, \(\phi\) is surjective. On the other hand, since
\(\psi \phi = \Id_M\) and \(\psi\) is a well defined set-function, it follows
that \(\phi\) is injective.

Conversely, let \(\phi: M \to N\) be an \(R\)-linear morphism and a bijective
set-function. Let \(\psi: N \to M\) be its inverse as a set-function --- we
shall prove that \(\psi\) is an \(R\)-linear morphism. Let \(n, n' \in N\) be any
elements, if \(\phi(m) = n\) and \(\phi(m') = n'\) then
\(\phi(m + m') = \phi(m) + \phi(m') = n + n'\) --- thus from construction
\(\psi(n + n') = m + m' = \psi(n) + \psi(n')\). Now, if \(r \in R\) is any ring
element, then since \(\phi(m r) = \phi(m) r = n r\), we find that
\(\psi(n r) = m r = \psi(n) r\) --- therefore \(\psi\) is indeed a morphism of
\(R\)-modules and hence an inverse morphism for \(\phi\).
\end{proof}

\begin{example}
\label{exp:int-dom-iso-principal-ideal-R-mod}
Let \(R\) be an integral domain, and \((a) \subseteq R\) be a non-zero principal
ideal of \(R\). There exists a natural isomorphism of \(R\)-modules
\(R \iso (a)\).

Consider the map \(\phi: R \to (a)\) given by \(r \mapsto r a\). Then for any
\(r, s \in R\) we have
\begin{align*}
  \phi(r + s) &= (r + s) a = r a + s a = \phi(r) + \phi(s) \\
  \phi(r s)   &= (r s) a = r (s a) = r \phi(s)
\end{align*}
thus \(\phi\) is an \(R\)-module morphism. The morphism is also clearly
surjective by the definition of an ideal. Moreover, one should note that
\(\phi(1) = a\), which is, by hypothesis, non-zero. Given any two
\(r, r' \in R\) such that \(\phi(r) = \phi(r')\) we obtain
\[
0 = \phi(r) - \phi(r') = r \phi(1) - r' \phi(1) = (r - r') \phi(1)
= (r - r') a,
\]
and since \(R\) is an integral domain, it follows that \(r = r'\) --- therefore,
\(\phi\) is injective. We conclude that \(\phi\) is a bijection between
\(R\)-modules and therefore establishes an isomorphism \(R \iso (a)\).
\end{example}

\begin{example}[Morphisms form an \(R\)-module]
\label{exp:R-mod-hom-set-is-module}
Given a ring \(R\) and left-\(R\)-modules \(M\) and \(N\), the
collection of \(R\)-linear morphisms \(\Hom_{\lMod{R}}(M, N)\) can be endowed
with the structure of an \emph{right-\(R\)-module}.

Since \(\Hom(\lMod{R}) \subseteq \Hom(\Ab)\), it follows that
\(\Hom_{\lMod{R}}(M, N)\) is an abelian group given by
\[
(f + g)(m) \coloneq f(m) + g(m)
\]
for any morphisms \(f, g: M \para N\) and element \(m \in M\). Moreover, one can
endow \(\Hom_{\lMod{R}}(M, N)\) with the \emph{right} ring action
\(\R \times \Hom_{\lMod{R}}(M, N) \to \Hom_{\lMod{R}}(M, N)\) given by
\[
(f \cdot r)(m) \coloneq f(r m)
\]
for every morphism \(f: M \to N\), and any elements \(r \in R\) and \(m \in M\).

Notice that we've emphasized that, in general, we can \emph{only} give a
\emph{right} \(R\)-module structure to \(\Hom_{\lMod{R}}(M, N)\), if on the
contrary we defined an left-\(R\)-action by \((r \cdot f)(m) \coloneq f(r m)\),
one would suffer from the following problem:
\begin{equation}\label{eq:left-R-action-problem}
((r s) f)(m) = (r(s f))(m) = (s f)(r m) = f(s (r m)) = f((s r) m) = ((s r) f)(m)
\end{equation}
where \(f: M \to N\) is a morphism, and \(r, s \in R\) and \(m \in M\) are any
elements. Mind that \cref{eq:left-R-action-problem} \emph{does not yield a
left-\(R\)-module structure unless \(R\) is commutative}.
\end{example}

\begin{proposition}
\label{prop:Mor(R,M)-iso-M-for-commutative-R}
Let \(R\) be a \emph{commutative} ring. Then there exists a canonical
isomorphism of \(R\)-modules
\[
\Hom_{\Mod{R}}(R, M) \iso M.
\]
\end{proposition}

\begin{proof}
Notice that every \(R\)-module morphism \(R \to M\) has to be of the form
\(r \mapsto r m\) for some fixed \(m \in M\) --- for convenience, name this
morphism \(f_m\). We define a map \(\phi: \Hom_{\Mod{R}}(R, M) \to M\) by
sending \(f_m \mapsto m\), which is certainly surjective. Moreover, if
\(f_m = f_n\) then in particular \(m = f_m(1) = f_n(1) = n\), therefore \(\phi\)
is injective. Notice that for any \(m, n \in M\) we have
\[
f_m(r) + f_n(r) = r m + r n = r(m + n) = f_{m + n}(r),
\]
thus \(f_m + f_n = f_{m + n}\). Also, given any \(s \in R\) we have
\[
s f_m(r) = s (r m) = (s r) m = (r s) m = r (s m) = f_{s m}(r),
\]
then \(s f_m = f_{s m}\).  The bijection \(\phi\) is also an \(R\)-module
morphism since, given any two \(f_m, f_n \in \Hom_{\Mod{R}}(R, M)\) we have
\[
\phi(f_m + f_n) = \phi(f_{m + n}) = m + n = \phi(f_m) + \phi(f_n),
\]
and given \(r \in R\),
\[
\phi(r f_m) = \phi(f_{r m}) = r m = r \phi(f_m).
\]
Since bijective morphisms are isomorphisms in \(\Mod{R}\), it follows that
\(\Hom_{\Mod{R}}(R, M) \iso M\) via \(\phi\).
\end{proof}

\begin{proposition}
\label{prop:M-iso-N-then-iso-between-End}
Let \(M\) and \(N\) be \(R\)-modules, for some ring \(R\). If \(M \iso N\), then
there exists a natural isomorphism between \emph{abelian groups}
\[
\End_{\Mod{R}}(M) \iso \End_{\Mod{R}}(N).
\]
\end{proposition}

\begin{proof}
Let \(\phi: M \isoto N\) be an isomorphism. We define a map
\(\Phi: \End_{\Mod{R}}(M) \to \End_{\Mod{R}}(N)\) given by the conjugation \(f
\mapsto \phi f \phi^{-1}\). This uniquely defines an \(R\)-module morphism for
each endomorphism \(f: M \to M\). Notice that, since \(\phi\) is a bijection,
given any endomorphism \(g: N \to N\) we may define an \(R\)-morphism \(f: M \to
M\) given by \(f \coloneq \phi^{-1} g \phi\), so that \(\Phi(f) = \phi
(\phi^{-1} g \phi) \phi^{-1} = g\). Therefore \(\Phi\) establishes an isomorphism
of abelian groups via conjugation.
\end{proof}

\subsubsection{Examples on Nilpotency}

\begin{lemma}[Nakayama's lemma, a particular case]
\label{lem:nakayama-particular-case}
Let \(R\) be a \emph{commutative} ring, and \(a \in R\) be a \emph{nilpotent}
element. Then \(M = 0\) if and only if \(a M = M\).
\end{lemma}

\begin{proof}
If \(a = 0\) then the statement is true. Suppose that \(a\) is non-zero, and let
\(n \in \Z_{>0}\) be the minimal positive integer such that \(a^n = 0\). If
\(M = 0\) then obviously \(a M = a \cdot 0 = 0 = M\). On the converse, if we
assume that \(a M = M\), let \(m \in M\) be any element. Let \(m_1 \in M\) be an
element such that \(m = a m_1\). By induction, let \(m_j \in M\) be an element
such that \(m_{j-1} = a m_j\), for \(1 < j \leq k\). If we consider the
collection \((m_j)_{j=1}^k\) we get a chain of equalities
\[
m = a m_1 = a^2 m_2 = \dots = a^k m_k = 0.
\]
Therefore \(M = 0\).
\end{proof}

\begin{proposition}
\label{prop:induced-surjective-nilpotent-ideal}
Let \(R\) be a commutative ring and \(\ideal a\) a \emph{nilpotent ideal} of
\(R\). Let \(\phi: M \to N\) be an \(R\)-module morphism. If the induced
\(R\)-module morphism
\[
\overline{\phi}: \frac{M}{\ideal a M} \longrightarrow \frac{N}{\ideal a N}
\quad\text{ mapping }\quad
m + \ideal a M \longmapsto \phi(m) + \ideal a N
\]
is a surjection, then so is \(\phi\).
\end{proposition}

\begin{proof}
Since \(\overline{\phi}\) is surjective, it follows that
\[
\overline{\phi}(M/\ideal{a} M) = \frac{\phi(M) + \ideal a N}{\ideal a N}
= \frac{N}{\ideal a N},
\]
therefore \(N = \phi(M) + \ideal a N\). Suppose that \(k \in \Z_{> 0}\) is such
that \(\ideal a^k = 0\) (which exists since \(\ideal a\) is a nilpotent
ideal). Then via induction we find
\begin{align*}
  N = \phi(M) + \ideal a N
  &= \phi(M) + \ideal a(\phi(M) + \ideal a N) \\
  &= \phi(M) + \ideal a \phi(M) + \ideal a^2 N \\
  &= \phi(M) + \ideal a^2 N \\
  &= \dots \\
  &= \phi(M) + \ideal a^k N \\
  &= \phi(M).
\end{align*}
Therefore \(\phi\) is surjective, since \(N = \phi(M)\).
\end{proof}

\subsection{Kernels \& Cokernels}

\begin{lemma}
\label{lem:ker-coker-exist-in-R-mod}
Kernels and cokernels exist in \(\Mod{R}\).
\end{lemma}

\begin{proof}
Let \(\phi: M \to N\) be any \(R\)-module and consider
\(\ker \phi \coloneq \{m \in M \colon \phi(m) = 0\}\), together with the
canonical inclusion \(\iota: \ker \phi \emb M\). Notice that
\(\phi \iota = 0 \iota = 0\). We prove that \((\ker \phi, \iota)\) is the
equalizer of \((\phi, 0)\): let \(P\) be any other \(R\)-module and
\(f: P \to M\) be any \(R\)-module morphism such that \(\phi f = 0 f = 0\). We
define a map \(\overline{f}: P \to \ker \phi\) given by \(\overline{f} = f\).
Indeed, since \(\phi f = 0\), then \(\im f \subseteq \ker \phi\), and
\(\overline{f}\) is a well defined and unique \(R\)-module morphism making the
following diagram commute
\[
\begin{tikzcd}
\ker \phi \ar[r, hook, "\iota"]
&M \ar[r, shift left, "\phi"] \ar[r, shift right, "0"']
&N \\
&P \ar[u, "f"] \ar[ul, dashed, bend left, "\overline{f}"] &
\end{tikzcd}
\]
Thus \((\ker \phi, \iota)\) is indeed the equalizer of \((f, 0)\).

Now we prove that \(\coker \phi \coloneq N/{\im \phi}\) together with the
natural projection \(\pi: N \epi \coker \phi\) is the coequalizer of
\((\phi, 0)\). Let \(C\) be an \(R\)-module and \(g: N \to C\) be an
\(R\)-module morphism such that \(g \phi = g 0 = 0\). We define a map
\(\overline{g}: \coker \phi \to C\) to be given by \(n + \im \phi \mapsto
g(n)\). This map is well defined since \(g(n) = 0\) for all \(n \in \im
\phi\). Also, since \(g\) is a morphism, \(\overline{g}\) is trivially a
morphism of \(R\)-modules too. Moreover, since \(\pi\) is an epimorphism,
\(\overline{g}\) is the unique morphism for which the following diagram commutes
\[
\begin{tikzcd}
M \ar[r, shift left, "\phi"] \ar[r, shift right, "0"']
& N \ar[r, two heads, "\pi"] \ar[d, "g"]
&\coker \phi \ar[dl, bend left, dashed, "\overline{g}"]
\\
&P &
\end{tikzcd}
\]
That is, \((\coker \phi, \pi)\) is the coequalizer of \((\phi, 0)\).
\end{proof}

\begin{proposition}[Properties of kernels \& cokernels]
\label{prop:ker-coker-in-R-mod-properties}
Kernels and cokernels exist in \(\Mod{R}\). Let \(\phi: M \to N\) be an
\(R\)-module morphism, then:
\begin{enumerate}[(a)]\setlength\itemsep{0em}
\item The following propositions are equivalent:
  \begin{itemize}\setlength\itemsep{0em}
  \item The \(R\)-module morphism \(\phi\) is a monomorphism.
  \item The kernel of \(\phi\) is trivial, that is, \(\ker \phi = 0\).
  \item The set-function induced by \(\phi\) is injective.
  \end{itemize}
\item The following propositions are equivalent:
  \begin{itemize}\setlength\itemsep{0em}
  \item The \(R\)-module morphism \(\phi\) is an epimorphism.
  \item The cokernel of \(\phi\) is trivial, that is, \(\coker \phi = 0\).
  \item The set-function induced by \(\phi\) is surjective.
  \end{itemize}
\end{enumerate}
\end{proposition}

\begin{proof}
\begin{enumerate}[(a)]\setlength\itemsep{0em}
\item
  \begin{itemize}\setlength\itemsep{0em}
\item Suppose \(\phi\) is a monomorphism, and consider
  \(\iota: \ker \phi \emb M\) and \(0\), then \(\phi \iota = \phi 0 = 0\), thus
  \(\iota = 0\) --- which implies in \(\ker \phi = 0\).

\item If we now suppose that \(\ker \phi = 0\), given any two elements
  \(m, m' \in M\) such that \(\phi(m) = \phi(m')\) we obtain
  \(0 = \phi(m) - \phi(m') = \phi(m - m')\) then \(m - m' \in \ker \phi\),
  implying in \(m = m'\).
\item Lastly, if \(\phi\) is injective, then given any two morphisms
  \(\alpha, \beta: P \para M\) such that \(\phi \alpha = \phi \beta\), we have
  for all \(p \in P\) that \(\phi(\alpha(p)) = \phi(\beta(p))\) --- which
  implies in \(\alpha(p) = \beta(p)\), thus \(\alpha = \beta\). We conclude that
  \(\phi\) is a monomorphism.
  \end{itemize}

\item
  \begin{itemize}\setlength\itemsep{0em}
\item Suppose that \(\phi\) is an epimorphism, then if we consider the canonical
  inclusiont \(\iota: \im \phi \emb N\) and the identity morphism
  \(\Id_N: N \to N\), one has that \(\iota \phi = \Id_N \phi\). Since \(\phi\)
  is an epimorphism, then \(\iota = \Id_N\) --- which is only possible if
  \(\im \phi = N\). Therefore \(\coker \phi\) is trivial.

\item If we suppose that \(\coker \phi\) is trivial, then \(\im \phi = N\) and
  \(\phi\) is therefore surjective.

\item Suppose that \(\phi\) is surjective, and let \(\alpha, \beta: N \to P\) be
  \(R\)-module morphisms such that \(\alpha \phi = \beta \phi\). Then, for every
  \(n \in N\), there exists \(m \in M\) such that \(\phi(m) = n\), thus
  \(\alpha(n) = \alpha(\phi(m)) = \beta(\phi(m)) = \beta(n)\) --- therefore
  \(\alpha = \beta\) and \(\phi\) is an epimorphism.
  \end{itemize}
\end{enumerate}
\end{proof}

\begin{example}[Right \& left inverses]
\label{exp:right-left-inverse-monic-epic}
One should not be deceived by the ideas permeating \(\Set\), just as in general
categories, \(\Mod{R}\) monomorphisms and epimorphisms need \emph{not} have
left-inverse and right-inverse, respectively. An example of a monomorphism
without a left-inverse is \(\Z \mono \Z\) given by \(a \mapsto 2 a\). On the
other hand, the projection \(\Z \epi \Z/2\Z\) is an epimorphism, although it
does not have a right-inverse.
\end{example}


\subsection{\texorpdfstring{\(R\)}{R}-Algebras}

\begin{example}[\(R\)-modules from ring morphisms]
\label{exp:R-modules-induced-by-ring-morphism}
Let \(\phi: R \to S\) be any ring morphism. We can induce on \(S\) a structure
of left-\(R\)-module by defining a map \(\rho: R \times S \to S\) to map
\((r, s) \mapsto \phi(r) s\) for every \(r \in R\) and \(s \in S\). In
particular, this shows that we can endow \(R\) itself with a left-\(R\)-module
structure. These constructions can be analogously done for right-\(R\)-modules
as well.

If \(R\) happens to be commutative and \(\im \phi \subseteq Z(S)\), then by
\cref{exp:right-module-to-left-module} we find that the left and right
\(R\)-module structures on \(S\) induced by \(\phi\) coincide. Moreover, notice
that, given any \(s, s' \in S\) and \(r, r' \in R\), we have that
\begin{align*}
\rho(r, s) \rho(r', s')
&= (\phi(r) s) (\phi(r') s')
= \phi(r) (s \phi(r')) s'
= \phi(r) (\phi(r') s) s' \\
&= (\phi(r) \phi(r')) (s s')
= \phi(r r') (s s') \\
&= \rho(r r', s s').
\end{align*}
This shows that the \(R\)-module structure induced by \(\phi\) is compatible
with the ring structure of \(S\). This kind of \(R\)-module receive the
name of \(R\)-algebra, which we now define for later reference.
\end{example}

\begin{definition}[\(R\)-algebra]
\label{def:R-algebra}
Let \(R\) be a \emph{commutative} ring. We define an \emph{\(R\)-algebra} to be
a \emph{ring morphism} \(\phi: R \to S\) such that the image of \(\phi\) is
contained in the centre of \(S\) --- inducing an \(R\)-module structure on
\(S\). If the ring \(S\) itself is commutative, we say that it has a structure
of a \emph{commutative \(R\)-algebra}.

We define a morphism \(\gamma: \alpha \to \beta\) of \(R\)-algebras
\(\alpha: R \to S\) and \(\beta: R \to Q\) to be a ring morphism
\(\gamma: S \to Q\) such that for all \(r \in R\) and \(s \in S\) we have
\(\phi(r s) = r \phi(s)\), and that the following diagram commutes in \(\Rng\)
\[
\begin{tikzcd}
&R \ar[ld, "\alpha"'] \ar[rd, "\beta"] & \\
S \ar[rr, "\gamma"']& &Q
\end{tikzcd}
\]

We denote by \(\Alg{R}\) the category consisting of \(R\)-algebras and morphisms
between them. A particularly important subcategory is that of the commutative
\(R\)-algebras, which we shall denote by \(\cAlg{R}\).
\end{definition}

\begin{corollary}
\label{cor:CAlg-subcat-CRing}
Given a commutative ring \(R\), then \(\cAlg{R}\) is a subcategory of
\(R/\cRng\) --- where \(R/\cRng\) denotes the slice category under \(R\).
\end{corollary}

\begin{example}
\label{exp:CAlg-not-full-subcat-CRing}
The category \(\cAlg{R}\), however, is \emph{not a full subcategory} of
\(\cRng\). Notice for instance that the map \(\CC \to \CC\) given by
the complex conjugation \(z \mapsto \overline{z}\) is a \emph{ring}
automorphism, nonetheless it isn't a morphism of \(\CC\)-modules.
\end{example}

\begin{proposition}[Initial object]
\label{prop:R-initial-R-alg}
Given a ring \(R\), the module structure of \(R\) over itself is the
\emph{initial} object of \(\Alg{R}\).
\end{proposition}

\begin{proof}
Given any \(R\)-algebra \(\phi: R \to A\) we have, for all \(r, r' \in R\), that
\(\phi(r r') = r \phi(r')\) --- thus \(\phi\) is an \(R\)-algebra morphism. In
particular, for \(r' = 1_R\) we obtain \(\phi(r) = r \phi(1_R) = r 1_A = r\) ---
thus \(\phi\) is the unique \(R\)-algebra morphism \(R \to A\).
\end{proof}

\begin{definition}[Field extension]
\label{def:field-extension}
Let \(k\) be a field. We define a \emph{field extension} of \(k\) to be a
\emph{commutative \(k\)-algebra} \(K\) with injective ring morphisms
\(\operatorname{inj}, \operatorname{inv}: k \para K\) such that
\(\operatorname{inj}(a) \operatorname{inv}(a) = 1\) for all \(a \in k\).
\end{definition}

\begin{example}
\label{exp:subfield-natural-field-extension}
Let \(k\) be a subfield \(k \subseteq \ell\) of a field \(\ell\). Then \(\ell\)
has a natural structure of field extension of \(k\). Indeed, we have a natural
inclusion \(k \emb \ell\) mapping \(a \mapsto a\) and an inversion map
\(k \emb \ell\) sending \(a \mapsto a^{-1}\).
\end{example}

\begin{definition}[Rees algebra]
\label{def:rees-algebra}
Let \(R\) be a \emph{commutative} ring and \(\ideal{a}\) ideal of \(R\). We
define a ring
\[
\Rees_R(\ideal{a}) \coloneq \bigoplus_{j \geq 0} \ideal{a}^j,
\]
where \(\ideal{a}^0 \coloneq R\), with a multiplication given by
\[
(a_j)_{j \geq 0} \cdot (b_j)_{j \geq 0} \coloneq
\bigg( \sum_{i + k = j} a_i b_k \bigg)_{j \geq 0}
\in \Rees_R(\ideal{a}).
\]
The ring morphism \(R \to \Rees_R(\ideal{a})\) mapping
\(r \mapsto (r, 0, \dots, 0, \dots)\) is called the \emph{Rees algebra of
  \(\ideal{a}\)}.
\end{definition}

\begin{proposition}
\label{prop:rees-algebra-iso-R[x]}
Let \(R\) be a \emph{commutative} ring and \(a \in R\) be a
non-zero-divisor. There exists a natural \emph{isomorphism of \(R\)-algebras}
\[
\Rees_R((a)) \iso R[x]
\]
\end{proposition}

\begin{proof}
Notice that \(R[x]\) is realized as an \(R\)-algebra by the inclusion
\(\iota: R \emb R[x]\) mapping to constant polynomials. Denote, for every
\(j \geq 0\), by \(e_j \in \Rees_R((a))\) the element whose \(j\)-th coordinate
is \(a^j\) and zero elsewhere. Define a map \(\phi: R[x] \to \Rees_R((a))\) by
mapping \(\phi(r) \coloneq r e_0\) and \(\phi(x) \coloneq e_1\).  Since
\(\{R, x\}\) generate \(R[x]\), this completely defines \(\phi\) as a ring
morphism, since \(\phi(r x^j) = r \phi(x)^j = r e_1^j = r e_j\). Moreover,
given any \((a_j)_{j \geq 0} \in \Rees_R((a))\), since \(R\) is commutative, we
have \(a_j = r_j a^j\) for some \(r_j \in R\). If we let \((r_j)_{j \geq 0}\) be
the collection of these associated \(R\) terms, we can build a polynomial \(p(x)
\coloneq \sum_{j \geq 0} r_j x^j\) so that
\[
\phi(p(x)) = \sum_{j \geq 0} \phi(r_j x^j) = \sum_{j \geq 0} r_j e_j
= (r_j a^j)_{j \geq 0}
= (a_j)_{j \geq 0}.
\]
Therefore \(\phi\) is surjective. Moreover, it is simple to see that if
\(p(x) \coloneq \sum_{j \geq 0} b_j x^j\) and
\(q(x) \coloneq \sum_{j \geq 0} c_j x^j\) are polynomials in \(R[x]\) such that
\(\phi(p(x)) = \phi(q(x))\), then their coefficients match --- that is,
\(b_j = c_j\) for all \(j \geq 0\), and then \(p(x) = q(x)\), making \(\phi\)
injective. Therefore \(\phi\) is an isomorphism of \(R\)-algebras.
\end{proof}

\begin{proposition}
\label{prop:rees-iso-ideal-a*ideal}
Let \(R\) be a commutative ring, \(a \in R\) be a non-zero-divisor, and
\(\ideal{b} \subseteq R\) be any ideal. Then there exists a natural isomorphism
of \(R\)-algebras
\[
\Rees_R(a \ideal{b}) \iso \Rees_R(\ideal b).
\]
\end{proposition}

\begin{proof}
Define a map \(\phi: \Rees_R(\ideal b) \to \Rees_R(a \ideal b)\) by
\((b_j)_{j \geq 0} \mapsto (a^j b_j)_{j \geq 0}\) --- since \(a\) is a
non-zero-divisor, \(a^j b_j = 0\) if and only if \(b_j = 0\), therefore \(\phi\)
is bijective. Notice that \(\phi\) satisfies \(\phi(r u) = r \phi(u)\) for any
\(r \in R\) and \(u \in \Rees_R(\ideal b)\). Moreover for any two \((u_j)_{j
  \geq 0}, (v_j)_{j \geq 0} \in \Rees_R(\ideal b)\), we have
\begin{align*}
\phi((u_j)_j \cdot (v_j)_j)
&= \phi \bigg( \sum_{i + k = j} u_i v_k \bigg)_j
= \bigg( a^j \sum_{i + k = j} u_i v_k \bigg)_j \\
&= \bigg( \sum_{i + k = j} a^{i + k} u_i v_k \bigg)_j
= (a^j u_j)_j \cdot (a^j v_j)_j \\
&= \phi((u_j)_j) \cdot \phi((v_j)_j).
\end{align*}
Thus \(\phi\) is an \(R\)-algebra isomorphism between \(\Rees_R(a \ideal{b})\)
and \(\Rees_R(\ideal b)\).
\end{proof}

\subsection{Submodules \& Quotients}

\subsubsection{Submodules}

\begin{definition}[Submodule]
\label{def:submodule}
Let \(R\) be a ring and \(M\) be an \(R\)-module. An \(R\)-module \(N\) is said
to be a \emph{submodule} of \(R\) if \(N \subseteq M\) and the inclusion map
\(N \emb M\) is a morphism of \(R\)-modules.
\end{definition}

\begin{example}[Ideals are the submodules of \(R\)]
\label{exp:R-submodules-are-ideals}
Let \(R\) be a ring endowed with the canonical left-\(R\)-module structure. The
submodules of \(R\) are correspond exactly to the left-ideals of \(R\).
\end{example}

\begin{example}[Kernel and image are submodules]
\label{exp:ker-im-are-submodules}
Given a morphism \(\phi: M \to N\) of \(R\)-modules, the kernel of \(\phi\) is a
submodule of \(M\), while the image of \(\phi\) is a submodule of \(N\).

Suppose \(M\) and \(N\) are right-\(R\)-modules, for left-\(R\)-modules the
proof is analogous. If \(m \in \ker \phi\), then for all \(r \in R\) we
have \(\phi(m r) = \phi(m) r = 0 \cdot r = 0\) --- thus \(m r \in \ker \phi\),
making \(\ker \phi \emb M\) a morphism of right-\(R\)-modules. Now, if \(n \in
\im \phi\), there must exist \(m \in M\) such that \(\phi(m) = n\), therefore
for all \(r \in R\) we have \(\phi(m r) = \phi(m) r = n r \in \im \phi\). We
conclude that, indeed, \(\ker \phi\) and \(\im \phi\) are submodules of \(M\)
and \(N\), respectively.
\end{example}

\begin{example}[Intersection \& sum of submodules]
\label{exp:intersection-sum-submodules}
Let \(M\) be an \(R\)-module and \((N_j)_{j \in J}\) be a collection of
submodules of \(M\). We have the following:
\begin{enumerate}[(a)]\setlength\itemsep{0em}
\item The intersection \(\bigcap_{j \in J} N_j\) is a submodule of \(M\).
\item The sum
  \[
  \sum_{j \in J} N_j \coloneq \bigg\{
  \sum_{j \in F} n_j \colon F \subseteq J
  \text{ is finite, and } n_j \in N_j \text{ for all } j \in F
  \bigg\}
  \]
  is a submodule of \(M\).
\end{enumerate}
\begin{proof}
For the intersection, given any two \(a, b \in \bigcap_{j \in J} N_j\) we have,
for each \(j \in J\), that \(a, b \in N_j\) and therefore \(a + b \in N_j\) ---
this implies in \(a + b \in \bigcap_{j \in J} N_j\). Moreover, given any element
\(r \in R\) it is equally clear that \(a r \in N_j\) for every \(j \in J\),
which implies in \(a r \in \bigcap_{j \in J} N_j\). Therefore
\(\bigcap_{j \in J} N_j\) is indeed a submodule of \(M\).

For the ease of notation, define \(N \coloneq \sum_{j \in J} N_j\). Notice that
since every sum in \(N\) is finite, then \(N \subseteq M\). Given any two
elements \(\sum_{j \in F} a_j, \sum_{j \in F'} b_j \in N\), we have that
\(F \cup F' \subseteq J\) is finite, hence we have an element
\(\sum_{j \in F \cup F'} m_j \in N\) given by
\[
m_j \coloneq
\begin{cases}
  a_j,  &j \in F \setminus F', \\
  b_j, &j \in F' \setminus F, \\
  a_j + b_j, &j \in F \cap F'. \\
\end{cases}
\]
It is easy to see that this constructed element satisfies
\[
\sum_{j \in F \cup F'} m_j =
\bigg( \sum_{j \in F} a_j \bigg) +
\bigg( \sum_{j \in F'} b_j \bigg),
\]
therefore \(N\) is closed under finite addition. Since multiplication is
distributive over finite sums, it follows that for any \(r \in R\) we have
\[
\bigg( \sum_{j \in F} a_j \bigg) r = \sum_{j \in F} a_j r
\]
and since \(a_j r \in N_j\) for each \(j \in F\), it follows that
\(\big( \sum_{j \in F} a_j \big) r \in N\). Thus \(N\) is a submodule of
\(M\).
\end{proof}
\end{example}

\begin{example}[Union of submodules]
\label{exp:union-submodules}
Let \(M\) be an \(R\)-module, for a ring \(R\). The following are two
propositions concerning submodules of \(M\):
\begin{enumerate}\setlength\itemsep{0em}
\item Given submodules \(S, T \subseteq M\), the union \(S \cup T\) is a
  submodule of \(M\) if and only if \(S \subseteq T\) or \(T \subseteq S\).
\item Let \((N_j)_{j \in \N}\) be an ascending chain of submodules of
  \(M\)---that is, \(N_j \subseteq N_{j+1}\) for all \(j \in \N\). Then the
  union \(\bigcup_{j \in \N} N_j\) is a submodule of \(M\).
\end{enumerate}
\begin{proof}
For item (a), if \(S \cup T\) is a submodule of \(M\),
let \(s \in S\) and \(t \in T\) be any two elements, then
\(s + t \in S \cup T\). This implies that either \(s + t \in S\) (which would
imply in \(t \in S\)) or \(s + t \in T\) (which would imply in \(s \in T\)),
therefore either \(S \subseteq T\) or \(T \subseteq S\). For the converse,
suppose, without loss of generality, that \(S \subseteq T\), then \(S \cup T =
S\), therefore \(S \cup T\) is a submodule of \(M\).

We now prove item (b). For the sake of notation, let
\(N \coloneq \bigcup_{j \in \N} N_j\). If \(n, n' \in N\) are any two elements,
there must exist indices \(j, j' \in \N\) such that \(n \in N_i\) and
\(n' \in N_{i'}\) for all \(i > j\) and for all \(i' > j'\). Define
\(k \coloneq \max(j, j')\), then \(n, n' \in N_i\) for all \(i > k\)---implying
in \(n + n' \in N_i\) and \(r n \in N_i\) for any \(r \in R\). Therefore
\(n + n' \in N\) and \(r n \in N\), hence \(N\) is a submodule of \(M\).
\end{proof}
\end{example}

\begin{example}[Totally ordered submodules]
\label{exp:totally-ordered-submodules-prime-division}
Let \(M\) be a finite \(\Z\)-module such that the collection of its submodules
is totally ordered with respect to inclusion. Then there exists a prime \(p\)
such that the number of elements of \(M\) is a power of \(p\).

\begin{proof}
Let \(|M| \coloneq d\) and suppose there exists two primes \(p\) and \(q\)
dividing \(d\). By \cref{prop:finite-abelian-p-div-then-elem-of-p-order} we know
that there must exist \(m, n \in M\) with order \(p\) and \(q\),
respectively. We now consider the submodules \(\langle m \rangle\) and
\(\langle n \rangle\) of \(M\). Since \(M\) has a totally ordered set of
submodules, we may assume without loss of generality that
\(\langle m \rangle \subseteq \langle n \rangle\)---therefore there exists \(a
\in \Z\) such that \(m = a n\). Notice that \(p m = p a n = 0\), therefore \(p
a\) must be a divisor of \(q\)---but since \(q\) is prime, \(p a\) is either
\(1\) or \(q\). Since \(p\) is also prime, it must be the case that \(a = 1\)
and \(p = q\). Thus there exists a unique prime divisor of \(d\)---hence \(d =
p^{\alpha}\) for some \(\alpha \in \Z\).
\end{proof}
\end{example}

\subsubsection{Simple \texorpdfstring{\(R\)}{R}-Modules}

\begin{definition}[Simple module]
\label{def:simple-module}
Let \(R\) be a ring. An \(R\)-module \(M\) is said to be \emph{simple} if its
only submodules are \(\{0\}\) and \(M\) itself.
\end{definition}

\begin{lemma}[Schur's]
\label{lem:schur-lemma}
Let \(M\) and \(N\) be \emph{simple} \(R\)-modules. If \(\phi: M \to N\) is an
\(R\)-module morphism, then either \(\phi = 0\) or \(\phi\) is an
\emph{isomorphism}.
\end{lemma}

\begin{proof}
Since \(\ker \phi\) is a submodule of \(M\), it can either be \(0\) or \(M\). If
\(\ker \phi = M\), then \(\phi = 0\). On the other hand, if \(\ker \phi = 0\)
then \(\phi\) is an injective morphism. Moreover, since \(\im \phi\) is a
submodule of \(N\), it's either \(0\) or \(N\) --- since \(\ker \phi\) in
trivial, then the only possibility is that \(\im \phi = N\), thus \(\phi\) is
surjective.
\end{proof}

\begin{corollary}
\label{cor:M-simple-Mor-division-ring}
Let \(M\) and \(N\) be right-\(R\)-modules and consider
\(\Hom_{\rMod{R}}(M, N)\) as a \emph{ring}. If \(M\) is \emph{simple}, then
\(\Hom_{\rMod{R}}(M, N)\) is a \emph{division ring}.
\end{corollary}

\begin{proof}
Let \(\ideal{a}\) be a left-ideal (or right-ideal) of \(\Hom_{\rMod{R}}(M,
N)\). If there exists an isomorphism \(\phi \in \ideal{a}\), then
\(\phi^{-1} \phi = \Id_M \in \ideal{a}\) and therefore, for all elements
\(\psi \in \Hom_{\rMod{R}}(M, N)\) we have \(\psi \Id_M = \psi \in \ideal{a}\)
--- thus \(\ideal{a} = \Hom_{\rMod{R}}(M, N)\) (the same equivalent arguement can
be used for right-ideals, where instead we get \(\phi \phi^{-1} = \Id_N\) in the
ideal). If there exist no isomorphism in \(\ideal{a}\), by
\cref{lem:schur-lemma} it is only composed of the zero morphism, thus
\(\ideal{a} = 0\). From \cref{prop:division-ring-ideals-are-0-or-ring} we
conclude that \(\Hom_{\rMod{R}}(M, N)\) is a division ring.
\end{proof}

\begin{example}
\label{exp:counterexp-schur-opposite}
We now show a counterexample ilustrating why the opposite of
\cref{cor:M-simple-Mor-division-ring} does not hold in general. Let \(k\) be a
field and
\[
A \coloneq
\begin{bmatrix}
  1 & 1 \\
  0 & 0
\end{bmatrix}
\qquad \qquad
B \coloneq
\begin{bmatrix}
  0 & 0 \\
  0 & 1
\end{bmatrix}
\]
be matrices. Define \(R\) to be the \(k\)-algebra generated by \(A\) and
\(B\). Consider the left-\(R\)-module \(M = k^2\), with left-multiplication by
matrices. We'll show that \(\Hom_{\lMod{R}}(M, M)\) is a division ring, while
\(M\) is not simple.

Consider the principal ideal
\(
\ideal{a} \coloneq \big(\big[
\begin{smallmatrix}
  1 \\
  0
\end{smallmatrix}
\big]\big)\) of \(M\). Notice that for any \(a, b \in k\) we have
\[
(a A + b B)
\begin{bmatrix}
  1 \\
  0
\end{bmatrix}
=
\begin{bmatrix}
  a & a \\
  0 & b
\end{bmatrix}
\begin{bmatrix}
  1 \\
  0
\end{bmatrix}
=
\begin{bmatrix}
  a \\
  0
\end{bmatrix},
\]
thus \(\ideal{a} \neq M\) and \(\ideal{a} \neq 0\), and \(M\) isn't
simple.

Let \(\ideal{h}\) be a left-ideal (or right-ideal) of the ring
\(\Hom_{\lMod{R}}(M, M)\). If \(\ideal{h} \neq 0\), consider a non-zero morphism
\(M =
\big[
\begin{smallmatrix}
  a & b \\
  c & d
\end{smallmatrix}
\big]
\in \ideal{h}\). Since \(M\) is a left-\(R\)-module morphism, it must be the
case that, for any
\(x =
\big[
\begin{smallmatrix}
  x_1 \\
  x_2
\end{smallmatrix}
\big]
\in M\), we have \(M(A x) = A M(x)\) and \(M(B x) = B M(x)\), but
%
\begin{align}
  \label{eq:M(Ax)}
  M(A x) &=
  % \begin{bmatrix}
  %   a & b \\
  %   c & d
  % \end{bmatrix}
  % \bigg(
  % \begin{bmatrix}
  %   1 & 1 \\
  %   0 & 0
  % \end{bmatrix}
  % \begin{bmatrix}
  %   x_1 \\
  %   x_2
  % \end{bmatrix}
  % \bigg)
  % =
  \begin{bmatrix}
    a (x_1 + x_2) \\
    c (x_1 + x_2)
  \end{bmatrix}
  \\ %%
  \label{eq:M(Bx)}
  M(B x) &=
  % \begin{bmatrix}
  %   a & b \\
  %   c & d
  % \end{bmatrix}
  % \bigg(
  % \begin{bmatrix}
  %   0 & 0 \\
  %   0 & 1
  % \end{bmatrix}
  % \begin{bmatrix}
  %   x_1 \\
  %   x_2
  % \end{bmatrix}
  % \bigg)
  % =
  \begin{bmatrix}
    b x_2 \\
    d x_2
  \end{bmatrix}
\end{align}
while on the other hand we have
\begin{align}
  \label{eq:A(Mx)}
  A M(x) &=
  % \begin{bmatrix}
  %   1 & 1 \\
  %   0 & 0
  % \end{bmatrix}
  % \bigg(
  % \begin{bmatrix}
  %   a & b \\
  %   c & d
  % \end{bmatrix}
  % \begin{bmatrix}
  %   x_1 \\
  %   x_2
  % \end{bmatrix}
  % \bigg)
  % =
  \begin{bmatrix}
    (a + c) x_1 + (b + d) x_2 \\
    0
  \end{bmatrix}
  \\ %%
  \label{eq:B(Mx)}
  B M(x) &=
  % \begin{bmatrix}
  %   0 & 0 \\
  %   0 & 1
  % \end{bmatrix}
  % \bigg(
  % \begin{bmatrix}
  %   a & b \\
  %   c & d
  % \end{bmatrix}
  % \begin{bmatrix}
  %   x_1 \\
  %   x_2
  % \end{bmatrix}
  % \bigg)
  % =
  \begin{bmatrix}
    0 \\
    c x_1 + d x_2
  \end{bmatrix}
\end{align}
Since \cref{eq:M(Ax)} equals \cref{eq:A(Mx)}, while \cref{eq:M(Bx)} equals
\cref{eq:B(Mx)} --- and since \(M\) is non-zero by hypothesis --- we obtain a
solution \(a = d = 1\) and \(b = c = 0\), yielding
\(M = \big[
\begin{smallmatrix}
  1 & 0 \\
  0 & 1
\end{smallmatrix}
\big]\), which implies in \(\ideal{h} = \Hom_{\lMod{R}}(M, M)\).
\end{example}

\begin{proposition}
\label{prop:simple-mod-iff-iso-R/a-with-maximal}
Let \(M\) be an \(R\)-module. Then \(M\) is simple if and only if \(M \iso
R/\ideal m\) for a maximal ideal \(\ideal m\) of \(R\).
\end{proposition}

\begin{proof}
Suppose \(\phi: M \isoto R/\ideal m\) is an isomorphism of \(R\)-modules. If
\(N \subsetneq M\) is a \emph{proper} submodule, then
\(\phi(N) \subseteq R/\ideal m\) must also be a submodule of
\(R/\ideal m\)---that is, an ideal. Since \(\ideal m\) is maximal, then the only
ideals of the field \(R/\ideal m\) are either the field itself or the zero ideal
\(\ideal m\). Since \(\phi\) is an isomorphism and \(N\) is proper, it must be
the case that \(\phi(N) = \ideal m\). But since \(\phi(0) = \ideal m\), then
\(N = \{0\}\)---thus \(M\) is simple.

For the converse, assume that \(M\) is simple, so that every non-zero element of
\(M\) generates the whole module. In particular \(1 \in M\) generates \(M\),
therefore the morphism of \(R\)-modules \(\psi: R \to M\) mapping
\(a \mapsto a \cdot 1\) is surjective, therefore
\[
M \iso R/\ker \psi.
\]
By \cref{cor:submodule-correspondence} if \(\ideal{a} \subsetneq R\) is any
proper ideal (submodule) containing \(\ker \psi\), then its corresponding
submodule is \(\ideal{a}/\ker \psi \subseteq R/\ker \psi \iso M\). Since \(M\)
is simple and \(\ideal{a}\) is proper, it follows that \(\ideal{a}/\ker \psi\)
must be the zero ideal, therefore \(\ideal{a} = \ker \psi\)---that is
\(\ker \psi\) is maximal.
\end{proof}

\subsubsection{Quotient modules}

\begin{definition}[Quotient module]
\label{def:quotient-module}
Let \(R\) be a ring, and \(M\) be a left-\(R\)-module. If \(N \subseteq M\) is a
submodule, then in particular \(N\) is a \emph{normal} subgroup of \(M\) and
therefore \(M/N\) is an \emph{abelian group}. We endow the group \(M/N\) with a
left-\(R\)-action \(R \times (M/N) \to M/N\) given by
\[
r (m + N) \coloneq r m + N,
\]
for all \(r \in R\) and \(m \in M\). This action turns \(M/N\) into a
left-\(R\)-module and the canonical projection \(\pi: M \epi M/N\) into an
\(R\)-module morphism. Therefore, the submodule \(N\) is the \emph{kernel} of
the canonical projection \(\pi\).
\end{definition}

\begin{example}
\label{exp:R/ideal-is-always-module}
Let \(R\) be a \emph{non-commutative} ring and \(\ideal{a}\) a left-submodule,
then the \emph{set} of equivalence classes \(R/\ideal{a}\) is \emph{not a ring}
--- since \(\ideal{a}\) is be required to be a two-sided-ideal for the quotient
to be a ring. Although not a ring, \(R/\ideal{a}\) is is an abelian group and
one can endow \(R/\ideal{a}\) with the canonical left-\(R\)-action given by
\(r (a + \ideal{a}) = r a + \ideal{a}\) for every \(r \in R\) and
\(a + \ideal{a} \in R/\ideal{a}\).
\end{example}

\begin{theorem}[Universal property of quotient modules]
\label{thm:univ-prop-quotient-modules}
Let \(R\) be a ring and \(N\) be a submodule of an \(R\)-module \(M\). For every
module \(Z\) together with a morphism \(\psi: M \to Z\) of \(R\)-modules such
that \(N \subseteq \ker \psi\), there exists a \emph{unique} morphism of rings
\(\phi: M/N \unique Z\) for which the following diagram commutes
\[
\begin{tikzcd}
M \ar[d, two heads] \ar[r, "\psi"] & Z \\
M/N \ar[ru, dashed, bend right, "\phi"']
\end{tikzcd}
\]
\end{theorem}

\begin{proof}
In particular, the existence and uniqueness of \(\phi\) as a \emph{group
  morphism} is shown in \cref{prop:universal-property-quotients-grp} --- we only
show that \(\phi\) is a morphism of \(R\)-modules. Let's assume that we are
working with right-\(R\)-modules, the same analogous proof would work for the
left modules. Let \(m + N \in M/N\) and \(r \in R\) be any elements, then
\(\psi(m r) = \psi(m) r\) and since \(\psi = \phi \pi\) it follows that
\[
\phi((m + N) r) = \phi(m r + N) = \psi(m r) = \psi(m) r = \phi(m + N) r,
\]
which shows that \(\phi\) is indeed a morphism of right-\(R\)-modules.
\end{proof}

\begin{theorem}[Factorization of morphisms]
\label{thm:R-module-morphism-factorization}
Every \(R\)-module morphism \(\phi: M \to N\) factors as follows
\[
\begin{tikzcd}
M \ar[r, "\phi"] \ar[d, two heads] &N \\
M/{\ker \phi} \ar[r, "\dis", "\overline{\phi}"'] & \im \phi \ar[u, hook]
\end{tikzcd}
\]
\end{theorem}

\begin{proof}
From the universal property \cref{thm:univ-prop-quotient-modules} we find that
\(\phi\) induces a unique morphism \(\overline{\phi}: M/{\ker \phi} \to N\) ---
being defined as \(\overline{\phi}(a + \ker \phi) = \phi(a)\) for any
\(a + \ker \phi \in M/{\ker \phi}\). We simply restrict the codomain of
\(\overline{\phi}\) to \(\im \phi\) so that it becomes surjective. Moreover,
given any two classes \(a + \ker \phi, b + \ker \phi \in M/{\ker \phi}\), such
that \(\overline{\phi}(a + \ker \phi) = \overline{\phi}(b + \ker \phi)\) then
\(\phi(a) = \phi(b)\) and thus \(a - b \in \ker \phi\) --- therefore \(a + \ker
\phi = b + \ker \phi\) and \(\overline{\phi}\) is injective.
\end{proof}

\begin{corollary}[First isomorphism]
\label{cor:first-iso-theorem-R-modules}
Let \(\phi: M \epi N\) be a surjective \(R\)-module morphism. There exists a
canonical isomorphism of \(R\)-modules
\[
N \iso M/{\ker \phi}
\]
\end{corollary}

\begin{proof}
Since \(\phi\) is surjective, \(\im \phi = N\) and from
\cref{thm:R-module-morphism-factorization} we conclude that the induced morphism
\(\overline{\phi}: M/{\ker \phi} \isoto N\) establishes the wanted canonical
isomorphism.
\end{proof}

\begin{proposition}[Second isomorphism]
\label{prop:second-iso-R-mod}
Let \(M\) be an \(R\)-module and \(P, N \subseteq M\) be submodules. There
exists a canonical isomorphism
\[
\frac{N + P}{N} \iso \frac{P}{N \cap P}
\]
\end{proposition}

\begin{proof}
Since \(N \emb N + P\) mapping \(n \mapsto n + 0 = n\) is clearly an
\(R\)-module morphism, \(N\) is a submodule of \(N + P\). Moreover, the
inclusion \(N \cap P \emb P\) mapping \(p \mapsto p\) is also a morphism since
\(p \in P\) for all \(p \in N \cap P\).

Consider the map \(\phi: N + P \to \frac{P}{N \cap P}\) given by
\(n + p \mapsto p + N \cap P\), which is clearly an \(R\)-module morphism. Given
any class \(p + N \cap P \in \frac{P}{N \cap P}\), one can choose a
representative \(p \in P\) so that \(\phi(p) = p + N \cap P\) --- thus \(\phi\)
is surjective. Moreover, \(\phi(n + p) = N \cap P\) if, and only if
\(p \in N \cap P\), which in this case implies in \(n + p \in N\). Therefore
\(\ker \phi = N\). By \cref{cor:first-iso-theorem-R-modules} we obtain
\(\frac{N + P}{\ker \phi} \iso \frac{P}{N \cap P}\), which is the required
isomorphism.
\end{proof}

\begin{proposition}[Third isomorphism]
\label{prop:third-iso-R-mod}
Let \(M\) be an \(R\)-module and \(N \subseteq M\) be a submodule. If
\(P \subseteq M\) is a submodule \emph{containing} \(N\) --- then \(P/N\) is a
\emph{submodule} of \(M/N\), and there exists a canonical isomorphism
\[
\frac{M/N}{P/N} \iso M/P
\]
\end{proposition}

\begin{proof}
Notice that since \(N \emb P\) is clearly an \(R\)-module morphism, thus \(N\)
is a submodule of \(P\). Moreover, if we consider the inclusion
\(i: P/N \emb M/N\) by mapping \(p + N \mapsto p + N\), since \(P \subseteq M\)
this is well defined and is a morphism of groups --- furthermore, given any
\(r \in R\), one has
\[
i((p + N) r) = i(p r + N) = p r + N = (p + N) r = i(p + N) r
\]
for any \(p + N \in P/N\), thus \(i\) is an \(R\)-module morphism. From the last
inclusion we conclude that \(P/N\) is a submodule of \(M/N\).

Let \(\phi: M/N \to M/P\) be defined by mapping \(m + N \mapsto m + P\) ---
since \(N \subseteq P\), this map is well defined and and surjective. Moreover,
for any \(p + N \in M/N\) where \(p \in P\), we have
\(\phi(p + N) = p + P = P\). Moreover, if \(m + N \in \ker \phi\) then
necessarily \(m \in P\). Therefore \(\ker \phi = P/N\) and by
\cref{cor:first-iso-theorem-R-modules} \(\phi\) induces an isomorphism
\(\frac{M/N}{\ker \phi} \iso M/P\).
\end{proof}

\begin{corollary}[Submodule correspondence]
\label{cor:submodule-correspondence}
Let \(M\) be an \(R\)-module and \(N \subseteq M\) be a submodule. There exists
a bijection
\begin{align*}
  \{\text{submodule } P \text{ of } M \text{ containing } N\}
  &\xrightarrow{\hspace*{2cm}}\{\text{submodules of } M/N
  \} \\
  P &\xmapsto{\hspace*{2cm}} P/N
\end{align*}
Moreover, given submodules \(P, P' \subseteq M\), we have that \(N \subseteq P
\subseteq P'\) if and only if \(P/N \subseteq P'/N\) in \(M/N\)---that is, the
bijection preserves inclusions.
\end{corollary}

\begin{proposition}
\label{prop:a*R/b-iso-a+b/b}
Let \(R\) be a \emph{commutative} ring, and \(\ideal{a}, \ideal{b} \subseteq R\)
be ideals. Then, there exists a natural \emph{\(R\)-module isomorphism}
\[
\ideal{a} \cdot \frac{R}{\ideal{b}}
\iso \frac{\ideal{a} + \ideal{b}}{\ideal{b}}
\]
\end{proposition}

\begin{proof}
Define a map
\(\phi: \ideal{a} + \ideal{b} \to \ideal{a} \cdot \frac{R}{\ideal{b}}\) given by
\(\phi(a + b) = a + \ideal{b}\) for every \(a \in \ideal{a}\) and
\(b \in \ideal{b}\). Clearly such a map establishes an \(R\)-module
morphism. Moreover, if
\(p \coloneq \sum_{j=1}^n a_j r_j + \ideal{b} \in \ideal{a} \cdot
\frac{R}{\ideal{b}}\) is any element, since \(R\) is commutative then
\(\ideal a\) is a two-sided ideal, hence one can choose, for every
\(1 \leq j \leq n\), an element \(a_j' \in \ideal{a}\) such that
\(a_j r_j = a_j'\). Therefore if we consider
\(q \coloneq \sum_{j=1}^n a_j' \in \ideal{a} + \ideal{b}\), one has
\(\phi(q) = p\) --- thus \(\phi\) is surjective. Furthermore, an element
\(a + b \in \ideal{a} + \ideal{b}\) belongs to \(\ker \phi\) if and only if
\(a + b \in \ideal{b}\) --- therefore \(\ker \phi = \ideal{b}\). By
\cref{cor:first-iso-theorem-R-modules} we obtain an isomorphism
\(\frac{\ideal{a} + \ideal{b}}{\ker \phi} \iso \ideal{a} \cdot
\frac{R}{\ideal{b}}\) as wanted.
\end{proof}

\subsection{Products \& Coproducts}

\begin{definition}[Direct product]
\label{def:direct-product-modules}
Let \(R\) be a ring and \((M_j)_{j \in J}\) be a collection of \(R\)-modules
(either right or left modules). We define the \emph{direct product} of
\((M_j)_{j \in J}\) as the set
\[
\prod_{j \in J} M_j \coloneq \{(m_j)_{j \in J} \colon m_j \in M_j\},
\]
together with coordinate-wise addition and multiplication by elements of
\(R\). This structure comes naturally with canonical projections
\(\pi_i: \prod_{j \in J} M_j \epi M_i\) mapping \((m_j)_{j \in J} \mapsto m_i\)
for all \(i \in J\).
\end{definition}

\begin{definition}[Direct sum]
\label{def:direct-sum-modules}
Let \(R\) be a ring and \((M_j)_{j \in J}\) be a collection of \(R\)-modules
(either right or left modules). We define the direct product of this family to
be a module
\[
\bigoplus_{j \in J} M_j \coloneq
\{(m_j) \colon m_j \in M_j \text{ and } m_j \neq 0
\text{ for finitely many } j \in J\},
\]
with a natural component-wise addition, and multiplication by elements of
\(R\). Such a structure also naturally induces canonical inclusions \(\iota_i:
M_i \emb \bigoplus_{j \in J} M_j\) given by \(m \mapsto (m_j)_{j \in J}\)
where \(m_i = m\) and \(m_j = 0\) for \(j \neq i\).
\end{definition}

\begin{theorem}
\label{thm:direct-sum-product-and-coproduct}
In the category of modules over a given ring, \emph{direct products} are
\emph{products} and \emph{direct sums} are \emph{coproducts}.
\end{theorem}

\begin{proof}
Let \((M_j)_{j \in J}\) be a collection of \(R\)-modules (either left or
right).
\begin{itemize}\setlength\itemsep{0em}
\item (Product) Define \(M \coloneq \prod_{j \in J} M_j\). Given an \(R\)-module
  \(N\) and a family of morphisms \((\phi_j: N \to M_j)_{j \in J}\), we define a
  map \(\phi: N \to M\) to be given by \(n \mapsto (\phi_j(n))_{j \in
    J}\). Notice that since each \(\phi_j\) is a morphism, then given any
  \(n, n' \in N\) we have
  \begin{align*}
    \phi(n + n')
    &= (\phi_j(n + n'))_{j \in J}
      = (\phi_j(n) + \phi_j(n'))_{j \in J}
      = (\phi_j(n))_{j \in J} + (\phi_j(n'))_{j \in J} \\
    &= \phi(n) + \phi(n'),
  \end{align*}
  moreover, if \(r \in R\) is any element then
  \[
    \phi(r n)
      = (\phi_j(r n))_{j \in J}
      = (r \phi_j(n))_{j \in J}
      = r (\phi_j(n))_{j \in J}
      = r \phi(n).
  \]
  That is, \(\phi\) is an \(R\)-module morphism and clearly \(\pi_j \phi =
  \phi_j\) for all \(j \in J\). Moreover, uniqueness comes from the fact that
  the natural projections are epimorphisms.

\item (Coproduct) Define \(M \coloneq \bigoplus_{j \in J} M_j\). Given an
  \(R\)-module \(N\) and a family of morphisms
  \((\psi_j: M_j \to N)_{j \in J}\), we define a map \(\psi: M \to N\) by
  \((m_j)_{j \in J} \mapsto \sum_{j \in J} \psi_j(m_j)\) --- which is well
  defined since \(m_j \neq 0\) only for finitely many \(j \in J\) and thus
  \(\sum_{j \in J} \psi_j(m_j)\) constitutes only of finitely many terms, since
  \(\psi_j(0) = 0\) for any \(j \in J\). This map clearly defines a morphism of
  \(R\)-modules, and since \(\iota_j\) are monomorphisms, \(\psi\) is the unique
  morphism such that \(\psi \iota_j = \psi_{j}\).
\end{itemize}
\end{proof}

\begin{proposition}
\label{prop:quotient-family-modules}
Let \((M_j)_{j \in J}\) be a collection of \(R\)-modules and \((N_j)_{j \in J}\)
be a corresponding collection of submodules \(N_j \subseteq M_j\). Then
\(\bigoplus_{j \in J} N_j\) is naturally identified as a submodule of
\(\bigoplus_{j \in J} M_j\), and there exists a natural isomorphism
\[
\frac{\bigoplus_{j \in J} M_j}{\bigoplus_{j \in J} N_j}
\iso
\bigoplus_{j \in J} M_j/N_j.
\]
\end{proposition}

\begin{proof}
For the sake of notation, define \(M \coloneq \bigoplus_{j \in J} M_j\) and
\(N \coloneq \bigoplus_{j \in J} N_j\). Since the inclusion \(N \emb M\) is an
\(R\)-module morphism, it follows that \(N\) is a submodule of \(M\). Consider
the natural projections \((\pi_j: M_j \epi M_j/N_j)_{j \in J}\), with kernels
\(\ker \pi_j = N_j\). Notice that the map
\(\pi: M \to \bigoplus_{j \in J} M_j/N_j\) given by
\(m \mapsto (\pi_j(m))_{j \in J}\) has a kernel \(\ker \pi = N\), and is
naturally surjective from its construction. Therefore by the universal property
of quotients we find \(M/\ker \pi = M/N \iso \bigoplus_{j \in J} M_j/N_j\).
\end{proof}

\begin{proposition}[Internal sum]
\label{prop:internal-sum-modules}
Given a ring \(R\), let \(M\) be an \(R\)-module (either right or left module)
and \((N_j)_{j \in J}\) be a collection of submodules of \(M\). The following
propositions are equivalent:
\begin{enumerate}[(a)]\setlength\itemsep{0em}
\item The map \(\bigoplus_{j \in J} N_j \to \sum_{j \in J} N_j\) given by
  \(x \mapsto \sum_{j \in J} \pi_j(x)\) is a \emph{unique isomorphism} of
  \(R\)-modules.

\item For every \(i \in J\), we have
  \(N_i \cap \sum_{j \in J \setminus i} N_j = 0\).

\item If \(\sum_{j \in J} x_j \in \sum_{j \in J} N_j\) is \emph{zero}, then
  \(x_j = 0\) for all \(j \in J\).

\item Every element \(x \in \sum_{j \in J} N_j\) can be \emph{uniquely}
  expressed as a sum \(x = \sum_{j \in J} x_j\) for \(x_j \in N_j\) and
  \(x_j \neq 0\) for only finitely many \(j \in J\).
\end{enumerate}
\end{proposition}

\begin{proof}
\begin{itemize}\setlength\itemsep{0em}
\item (a) \(\implies\) (b). Let
  \(n_i \in N_i \cap \sum_{j \in J \setminus i} N_j\) be any element, then one
  can express \(n_i\) as \(n_i = \sum_{j \in J \setminus i} a_j  n_j\) for,
  therefore \(n_i - \sum_{j \in J \setminus i} a_j n_j = 0\). By (a) we find
  that such an element has preimage \(0 \in \bigoplus_{j \in J} N_j\), which can
  only be the case if \(n_i = 0\).

\item (b) \(\implies\) (c). If \(\sum_{j \in J} x_j = 0\), then for every \(i
  \in J\) we have \(x_i = - \sum_{j \in J \setminus i} x_j\) --- but assuming
  (b) we conclude that \(x_i = 0\).

\item (c) \(\implies\) (d). The condition of finitely many non-zero terms is
  already satisfied from the definition of \(\sum_{j \in J} N_j\). Now, if
  \(x = \sum_{j \in J} x_j = \sum_{j \in J} y_j\) are two representations of the
  same element \(x \in \sum_{j \in J} N_j\), then
  \(\sum_{j \in J}(x_j - y_j) = 0\). By (c) this implies in \(x_j = y_j\) for
  all \(j \in J\).

\item (d) \(\implies\) (a). Unicity of expression implies that the given map is
  injective and unique. Moreover, the map is clearly surjective --- therefore it
  establishes a unique isomorphism.
\end{itemize}
\end{proof}


\begin{proposition}
\label{prop:direct-sums-isos-hom-sets}
Let \(R\) be a ring, and consider a right-\(R\)-module \(M\), and a collection
\((N_j)_{j \in J}\) of right-\(R\)-modules. The following propositions hold:
\begin{enumerate}[(a)]\setlength\itemsep{0em}
\item There exists a canonical isomorphism of \emph{abelian groups}
  \[
  \Hom_{\rMod{R}}\bigg( \bigoplus_{j \in J} N_j, M \bigg) \iso
  \prod_{j \in J} \Hom_{\rMod{R}}(N_j, M).
  \]

\item There exists a canonical isomorphism of \emph{abelian groups}
  \[
  \Hom_{\rMod{R}}\bigg( M, \prod_{j \in J} N_j \bigg) \iso
  \prod_{j \in J} \Hom_{\rMod{R}}(M, N_j).
  \]
\item If \((M_i)_{i = 1}^n\) and \((N_j)_{j=1}^m\) are \emph{finite} collections
  of right-\(R\)-modules, then there exists a canonical isomorphism of
  \emph{abelian groups}
  \[
  \Hom_{\rMod{R}} \bigg( \bigoplus_{i=1}^n M_i, \bigoplus_{j=1}^m N_j \bigg)
  \iso
  \bigoplus_{i=1}^n \bigoplus_{j=1}^m \Hom_{\rMod{R}}(M_i, N_j).
  \]
\end{enumerate}
\end{proposition}

\begin{proof}
\begin{enumerate}[(a)]\setlength\itemsep{0em}
\item Define a map
  \(\Phi: \Hom_{\rMod{R}}(\bigoplus_{j \in J} N_j, M) \to \prod_{j \in J}
  \Hom_{\rMod{R}}(N_j, M)\) by sending \(f \mapsto (f \iota_j)_{j \in J}\),
  where \(\iota_j: N_j \emb \bigoplus_{j \in J} N_j\) is the canonical
  inclusion. Notice that \(\Phi\) is indeed a group morphism. Also, morphisms
  with equal image must agree on every element \(\iota_j(n_j)\) for any
  \(j \in J\) and \(n_j \in N_j\), therefore, since these elements generate
  \(\bigoplus_{j \in J} N_j\), it follows that the morphisms need to be equal
  --- thus \(\Phi\) is injective. Moreover, if \((g_j: N_j \to M)_{j \in J}\) is
  any collection of \(R\)-linear maps, then by the coproduct universal property
  there exists a unique \(f: \bigoplus_{j \in J} N_j \to M\) such that
  \(f \iota_j = g_j\) for all \(j \in J\).

\item Define a map
  \(\Psi: \Hom_{\rMod{R}}(M, \prod_{j \in J} N_j) \to \prod_{j \in J}
  \Hom_{\rMod{R}}(M, N_j)\) by \(f \mapsto (\pi_j f)\) --- again
  this product preserves additive structure. Also, this map is injective since,
  given \(f, g: M \para \prod_{j \in J} N_j\) such that \(\Psi(f) = \Psi(g)\),
  we must have \(\pi_j f(m) = \pi_j g(m)\) for every \(j \in J\) and
  \(m \in M\), therefore \(f(m) = g(m)\) and \(f = g\). Surjectivity comes from
  the product universal property: given a collection
  \((g_j: M \to N_j)_{j \in J}\) of morphisms, there exists a unique morphism
  \(f: M \to \prod_{j \in J} N_j\) such that \(\pi_j f = g_j\) for all
  \(j \in J\).

\item For finite indexing sets direct sums and products are isomorphic, thus the
  isomorphism follows directly from the last two items.
\end{enumerate}
\end{proof}

\begin{example}
\label{exp:matrix-of-morphisms}
Let \(M_1, \dots, M_n\) be right-\(R\)-modules and define a right-\(R\)-module
\(M \coloneq \bigoplus_{j=1}^n M_j\) and a ring
\[
H \coloneq \Bigg\{
\begin{bmatrix}
  f_{1 1} &\dots  &f_{1 n} \\
  \vdots &\ddots &\vdots \\
  f_{n 1} &\dots  &f_{n n}
\end{bmatrix}
\colon f_{i j} \in \Hom_{\rMod{R}}(M_j, M_i)
\Bigg\}.
\]
Then there exists a natural isomorphism of rings
\[
H \iso \End_{\rMod{R}}(M).
\]

First we show that \(H\) is indeed a ring. Given any
\([f_{i j}], [g_{ij}] \in H\), we have entry-wise
\(f_{i j} + g_{i j} \in \Hom_{\rMod{R}}(M_j, M_i)\), thus
\([f_{i j} + g_{i j}] \in H\), and \(H\) is an abelian group via matrix
addition. Now, if we define a product on \(H\) as the matrix product
\([f_{ij}] \cdot [g_{ij}] \coloneq [\sum_{k=1}^n f_{i k} g_{k j}]\), then since
\(f_{i k} g_{k j}: M_j \to M_i\) thus the matrix product
\([f_{i j}] \cdot [g_{ij}]\) is also an element of \(H\)---which shows that
\(H\) is indeed a ring.

If \([f_{i j}] \in H\) is any matrix, define a morphism of right-\(R\)-modules
\(f: M \to M\) with projections \((f_i)_{i=1}^n\), where for every
\(1 \leq i \leq n\) the map \(f_i: M \to M_i\) is the unique morphism such that
the diagram
\[
\begin{tikzcd}
M_j \ar[r, hook] \ar[rd, bend right, "f_{ij}"'] &M \ar[d, "f_i"] \\
&M_i
\end{tikzcd}
\]
commutes for all \(1 \leq j \leq n\). We now simply define a map \(\Phi: H \to
\End_{\rMod{R}}(M)\) by \([f_{ij}] \mapsto f\).

We now show that \(\Phi\) is a ring morphism. Let \([g_{ij}] \in H\) be any
other matrix and let \(g \coloneq \Phi([g_{ij}]): M \to M\). If
\([h_{ij}] \coloneq [f_{i j}] \cdot [g_{i j}]\) then from definition we have
\(h_{ij} = \sum_{k=1}^n f_{i k} g_{k j}\). Notice that the composition \(f g: M
\to M\) with projections
\[
(f g)_i = f_i\bigg(\sum_{k=1}^n g_k\bigg)
= f_i \bigg(\sum_{k=1}^n \sum_{j=1}^n g_{k j} \bigg)
= \sum_{j=1}^n \sum_{k=1}^n f_{i k} g_{k j}
\]
for all \(1 \leq i \leq n\), therefore \((f g)_i = h_i\). This proves that
\(\phi\) is a ring morphism:
\[
\phi([f_{i j}] \cdot [g_{i j}]) = h = f g = \phi([f_{i j}]) \phi([g_{i j}]).
\]

For the injectivity of \(\phi\) it suffices to see that the image of an element
of \(H\) is uniquely defined by the universal property of the coproduct. For the
surjectivity of \(\phi\), let \(f: M \to M\) be any endomorphism, then again
from the universal property each of its projections \(f_i\) are uniquely defined
by a family of morphisms of \(R\)-modules \((f_{i j}: M_j \to M_i)_{j=1}^n\),
therefore, the matrix \([f_{i j}] \in H\) has image \(f\) under
\(\phi\). Therefore we conclude that \(H \iso \End_{\rMod{R}}(M)\).
\end{example}

\begin{example}
\label{exp:matrix-of-morphisms-over-ring}
Let \(R\) be a ring and consider \(R\) as a \emph{right}-\(R\)-module over
itself. We'll show that there exists a natural isomorphism of rings
\[
M_n(R) \iso \End_{\rMod{R}}\Big( \bigoplus_{j=1}^n R \Big).
\]

First, we shall show that \(R \iso \End_{\rMod{R}}(R)\) as rings. For that end,
define a map \(\Psi: R \to \End_{\rMod{R}}(R)\) given by \(r \mapsto {}_rm\),
where \({}_rm: R \to R\) is the \emph{left}-multiplication by \(r\), that is
\({}_rm(a) = r a\). It is easy to notice that if \(r, s \in R\), then
\({}_{r s}m = {}_rm {}_sm\), therefore \(\Psi\) is a ring morphism. For the
injectivity of \(\Psi\), if \(r, s \in R\) are elements with equal image, then
in particular \(r = {}_rm(1) = {}_sm(1) = s\). For surjectivity, if
\(f: R \to R\) is an endomorphism of right-\(R\)-modules, then given \(r \in R\)
we have \(f(r) = f(1 \cdot r) = f(1) r\), therefore \(f = {}_{f(1)}m\)---and
\(\Psi(f(1)) = f\). We thus conclude that
\[
R \iso \End_{\rMod{R}}(R).
\]

As in \cref{exp:matrix-of-morphisms}, we can define a ring \(H\) of
\(n \times n\) matrices whose entries are elements of the ring
\(\End_{\rMod{R}}(R)\). From our previous result, given a matrix
\([f_{ij}] \in H\) there exists a unique matrix \([r_{ij}] \in M_n(R)\) such
that \(\Psi(r_{ij}) = f_{ij}\) for all \(1 \leq i, j \leq n\). Therefore,
\(\Psi\) induces an isomorphism \(H \iso M_n(R)\). From
\cref{exp:matrix-of-morphisms} we know that
\(H \iso \End_{\rMod{R}}(\bigoplus_{j=1}^n R)\), therefore we conclude that
\(M_n(R) \iso \End_{\rMod{R}}(\bigoplus_{j=1}^n R)\).

As a corollary of this result, together with
\cref{cor:M-simple-Mor-division-ring}, we find that if \(S\) is a \emph{simple
  right-\(R\)-module}, then there exists a natural isomorphism of rings
\[
M_n(D) \iso \End_{\rMod{R}}\Big(\bigoplus_{j=1}^n S \Big),
\]
where \(D\) is a division ring---in particular, we have
\(D \iso \End_{\rMod{R}}(S)\).
\end{example}

\subsection{Pullbacks \& Pushouts of \texorpdfstring{\(R\)}{R}-Modules}

\begin{theorem}
\label{thm:R-mod-pullback-pushouts}
The category of \(R\)-modules has pullbacks and pushouts.
\end{theorem}

\begin{proof}
Let \(M\), \(N\), and \(Z\) be \(R\)-modules.
\begin{itemize}\setlength\itemsep{0em}
\item (Pullback) Given \(R\)-module morphisms \(\mu: M \to Z\) and \(\lambda: N
  \to Z\), we define a triple \((M \times_Z N, \pi_M, \pi_N)\) ---
  where
  \[
  M \times_Z N \coloneq \{(m, n) \in M \times N \colon \mu(m) = \lambda(n)\},
  \]
  with the natural \(R\)-module structures inherited from \(M \times N\), and
  \(\pi_M: M \times_Z N \epi M\) and \(\pi_N: M \times_Z N \epi N\) are
  canonical projections. We claim that the following commutative square is a
  pullback
  \[
  \begin{tikzcd}
  M \times_Z N \ar[d, two heads, "\pi_M"'] \ar[r, two heads, "\pi_N"]
  \ar[dr, phantom, "\lrcorner", very near start]
  &N \ar[d, "\lambda"] \\
  M \ar[r, "\mu"'] &Z
  \end{tikzcd}
  \]
  From construction we have \(\mu \pi_M = \lambda \pi_N\), thus the diagram
  commutes. Let \(W\) be any other \(R\)-module, and consider any two
  \(R\)-module morphisms \(m: W \to M\) and \(n: W \to N\) such that
  \(\mu m = \lambda n\). We define a map \(\phi: W \to M \times_Z N\) to be
  given by \(\phi(w) = (m(w), n(w))\) --- since \(\mu m(w) = \mu n(w)\) then
  indeed \(\phi(w) \in M \times_Z N\). Moreover, since \(m\) and \(n\) are
  morphisms, it follows trivially that \(\phi\) is an \(R\)-module
  morphism. Uniqueness of \(\phi\) comes from the fact that both \(\pi_M\) and
  \(\pi_N\) are epimorphisms.

\item (Pushout) Let \(\sigma: Z \to M\) and \(\varepsilon: Z \to N\) be
  \(R\)-module morphisms. Consider the submodule
  \[
  W \coloneq \{(\sigma(z), -\varepsilon(z)) \in M \oplus N \colon
  z \in Z\}.
  \]
  Define a triple \((M \oplus_Z N, \iota_M, \iota_N)\)
  to be given by
  \[
  M \oplus_Z N \coloneq \frac{M \oplus N}{W},
  \]
  inheriting the natural \(R\)-module structure from \(M \oplus N\), and
  canonical inclusions \(\iota_M: M \emb M \oplus_Z N\) and
  \(\iota_N: N \emb M \oplus_Z N\). We claim that the following commutative
  diagram is a pushout
  \[
  \begin{tikzcd}
  Z \ar[r, "\varepsilon"] \ar[d, "\sigma"']
  \ar[dr, very near end, phantom, "\ulcorner"]
  &N \ar[d, hook, "\iota_N"] \\
  M \ar[r, hook, "\iota_M"'] &M \oplus_Z N
  \end{tikzcd}
  \]
  If \(z \in Z\) is any element, then  \(\iota_M\sigma(z) = (\sigma(z), 0) + W\)
  and \(\iota_N \varepsilon(z) = (0, \varepsilon(z)) + W\) but notice that
  \[
  \iota_M \sigma(z) - \iota_N \varepsilon(z)
  = (\sigma(z), - \varepsilon(z)) + W
  = W.
  \]
  Thus \(\iota_M \sigma = \iota_N \varepsilon\). Moreover, \(P\) be any
  \(R\)-module, together with morphisms \(n: N \to P\) and \(m: M \to P\) such
  that \(m \sigma = n \varepsilon\).  We construct a map
  \(\psi: M \oplus_Z N \to P\) given by \((a, b) + W \mapsto m(a) + n(b)\),
  which clearly is such that \(\psi \iota_N = n\) and \(\psi \iota_M = m\) ---
  moreover, \(\psi\) is an \(R\)-module morphism. Its uniqueness comes from the
  fact that \(\iota_M\) and \(\iota_N\) are monomorphisms.
\end{itemize}
\end{proof}

%%% Local Variables:
%%% mode: latex
%%% TeX-master: "../../deep-dive"
%%% End:

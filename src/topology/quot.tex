\section{Quotient Space}

\subsection{Quotient Topology}

The motivation for the construction of the quotient topology is the study of
surjective set-functions \(\pi: X \epi S\) between topological spaces \(X\) and
sets \(S\), which induce an equivalence relation on the initial topological
space by means of arranging the points of \(X\) in classes where \(x \sim y\)
for \(x, y \in X\) if and only if \(\pi(x) = \pi(y)\), that is, \(x\) and \(y\)
are common points of a fiber \(\pi^{-1}(s)\) for some \(s \in S\).

The geometrically appealing version of such construction would be the idea of
gluing every element \(x \in \pi^{-1}(s)\) into a unique point --- this way we
loose some initial information about the topological space \(X\). However, we
would like to make this gluing process compatible with the theory of topology so
far constructed. To achieve that, one may make the point that we should make
\(S\) into a topological space for which its topology makes \(\pi\) a continuous
map. The first idea that may come to mind is that we can simply force the
continuity of \(\pi\) by defining a topology \(\tau\) on \(S\) so that
\(U \subseteq S\) is an open set if and only if \(\pi^{-1}(U)\) is open ---
which is exactly the definition of a continuous map. Lets give this a formal
definition.

\begin{definition}\label{def:quotient-topology}
Let \(X\) be a topological space and \(S\) be any set. Let also \(\pi: X \epi S\)
be a set-function. The quotient topology on \(S\) induced by the map \(\pi\) is
the final topology such that \(\pi\) is a continuous map.
\end{definition}

\begin{proposition}
The quotient topology is a topology.
\end{proposition}

\begin{proof}
Let \(X /{\sim}\) be a space with the topology induced by the map
\(\pi: X \epi X /{\sim}\). Consider an arbitrary collection \({\{U_{j}\}}_j\) of
open sets of \(X /{\sim}\). Notice that since
\(\pi^{-1}(\bigcup_{j} U_{j}) = \bigcup_j \pi^{-1}(U_j)\) then since
\(\pi^{-1}(U_j)\) is open for all index \(j\), we conclude that
\(\bigcup_j \pi^{-1}(U_{j}) \subseteq X\) is open, hence the set
\(\bigcup_j U_j \subseteq X /{\sim}\) is necessarily open --- since \(\pi\) is
continuous. Let now \(A, B \subseteq X /{\sim}\) be any open sets, then
\(\pi^{-1}(A \cap B) = \pi^{-1}(A) \cap \pi^{-1}(B) \subseteq X\), which is open
and therefore \(A \cap B\) is open in \(X /{\sim}\). We conclude that the
quotient topology indeed satisfies the properties of a topology.
\end{proof}

In other terms, let \(T_S\) be the collection of all topologies on \(S\) such
that \(\pi\) is a continuous map. The above definition simply says that the
quotient topology \(\tau\) on the set \(S\) is the intersection
\(\tau = \bigcap_{T \in T_S}T\). Even better than that definition is the
fact that we can determine the quotient topology by the following universal
property.

\begin{theorem}[Universal property of the quotient topology]
\label{thm:universal-property-quotient-topology}
Let \(X\) be a topological space and \(S\) be any set, together with a
surjective set-function \(\pi: X \epi S\). The quotient topology \(\tau\)
on \(S\) induced by \(\pi\) is such that, for all topological spaces \(Z\) and
every morphism \(g: X \to Z\) for which all \(x, y \in X\) such that
\(\pi(x) = \pi(y)\) then \(g(x) = g(y)\) --- i.e. \(g\) is constant on the
fibers of \(\pi\) --- there exists a unique morphism
\(f: (S, \tau) \to Z\) such that the diagram
\[
  \begin{tikzcd}
    X \ar[r, "g"] \ar[d, two heads, swap, "\pi"] &Z\\
    S \ar[ur, swap, bend right, dashed, "f"]
  \end{tikzcd}
\]
commutes in \(\Top\). Moreover, if \(\tau'\) is a topology on \(S\) such
that the diagram commutes, then necessarily \(\tau' = \tau\).
\end{theorem}

\begin{proof}
Since \(\pi\) is surjective, we can completely define a map \(f: S \to Z\)
sending \(s \mapsto g(x)\) such that \(x \in \pi^{-1}(s)\), which is well
defined because, for all \(x, y \in \pi^{-1}(s)\), we have \(g(x) = g(y)\), so
that the image of each \(s \in S\) under the map \(f\) is uniquely defined. We
now show that \(f\) is, in fact, continuous --- and hence a morphism. Let
\(U \subseteq Z\) be any open set of \(Z\), then, since \(g^{-1}(U)\) is open
and \((f \pi)^{-1}(U) = \pi^{-1} f^{-1}(U) = g^{-1}(U)\), \(f^{-1}(U)\) cannot
be a closed set --- in fact, it needs to be an open set, because
\(\pi^{-1}(f^{-1}(U)) \subseteq X\) must be open, since \(\pi\) is continuous.
It follows that \(f\) is continuous and thus the said morphism indeed exists.

For the uniqueness, let \(f\) and \(f'\) be two morphisms such that \(f\pi = g\)
and \(f'\pi = g\). Let \(s \in S\) be any point. Since \(\pi\) is surjective,
there exists \(x \in X\) for which \(x \in \pi^{-1}(s)\), therefore,
\(f \pi(x) = f(s) = f'\pi(x) = f'(s)\) for every element of their domain ---
hence \(f = f'\).

Suppose now that both \((S, \tau)\) and \((S, \tau')\) satisfy the
universal property, that is, the following diagrams commutes for unique
morphisms \(f\) and \(h\)
\[
  \begin{tikzcd}
    &Z & \\
    (S, \tau) \ar[ur, dashed, bend left, "f"]
    &X \ar[l, two heads, "\pi"] \ar[r, swap, two heads, "\pi"] \ar[u, "g"]
    &(S, \tau') \ar[ul, dashed, bend right, swap, "f'"]
  \end{tikzcd}
\]
Since the diagram commutes for all \(Z\), let \(Z = (S, \tau)\), and
consider the map \(f' = \Id': (S, \tau') \to (S, \tau)\). Then,
given any \(U \in \tau\) we find that since \(\Id'\) is continuous that
\(\Id'^{-1}(U) = U \subseteq (S, \tau')\) is open, hence
\(\tau \subseteq \tau'\). Analogously, let \(Z = (S, \tau')\)
and consider \(f = \Id: (S, \tau) \to (S, \tau')\). Let
\(U' \in \tau'\) then from the continuity of \(\Id\) we find that
\(\Id^{-1}(U') = U' \subseteq (S, \tau)\) is open, therefore
\(\tau' \subseteq \tau\). Thus indeed \(\tau = \tau'\)
as wanted.
\end{proof}

\begin{proposition}[Quotient topology as a coequalizer]
\label{prop:quotient-top-as-coequalizer}
Let \(X\) be a topological space and \(f: X \epi S\) be a surjective
map, where \(S\) is some set. Define the equivalence relation set
\[
R \coloneq \{(x, y) \in X \times X \colon f(x) = f(y)\},
\]
together with two maps \(r_1, r_2: R \rightrightarrows X\) given by the
following commutative diagram
\[
\begin{tikzcd}
R \ar[r, hook] \ar[rr, bend left=50, "r_1" description]
\ar[rr, bend right=50, "r_2" description]
&X \times X \ar[r, two heads, shift left, "\pi_1"]
\ar[r, two heads, shift right, "\pi_2"']
&X
\end{tikzcd}
\]

The \emph{quotient topology} is exactly the topology that makes \(S\) into the
\emph{coequalizer} of \(r_1\) and \(r_2\).
\end{proposition}

\begin{proof}
Let \(Y\) be any space and \(h: X \to Y\) be a continuous map such that
\(h r_1 = h r_2\). Let \(x, x' \in X\) be any two points such that
\(f(x) = f(y)\), then from construction \(x, x' \in R\). Then
\(h(x) = h r_1(x, x') = h r_2(x, x') = h(x')\). Now by means of the universal
property \cref{thm:universal-property-quotient-topology}, if \(\tau\) is the
quotient topology on \(S\) induced by \(f\), we conclude that there
exists a unique continuous map \(g: (S, \tau) \to Y\) such that the following
diagram commutes
\[
\begin{tikzcd}
Y & & \\
(S, \tau) \ar[u, dashed, "g"]
&X \ar[l, two heads, "f"]
\ar[lu, bend right, "h"']
&R \ar[l, shift right, "r_1"'] \ar[l, shift left, "r_2"]
\end{tikzcd}
\]
Therefore \((S, \tau) = \Coeq(r_1, r_2)\), with associated morphism \(f\).
\end{proof}

To ease the way in which we refer to quotients and surjective morphisms that
induce quotients between topological spaces, we define the following
terminology.

\begin{definition}[Quotient morphism]
\label{def:quotient-morphism}
A surjective morphism \(\pi: X \to Y\) of topological spaces \(X\) and \(Y\) is
said to be a \emph{quotient morphism} (or quotient map) if \(\pi\) induces the
universal property of quotients --- in other words, open sets of \(Y\) are
exactly those that have open preimage on \(\pi\), that is, \(V \subseteq Y\) is
open if and only if \(\pi^{-1}(V) \subseteq X\) is open.
\end{definition}

\begin{theorem}[Quotient descent]
\label{thm:Top-quotient-descent}
Let \(q: X \to Y\) be a quotient map between topological spaces. For any space
\(Z\) together with a morphism \(f: X \to Z\) that is constant on the fibers of
\(q\) --- that is, \(q(x_1) = q(x_2)\) implies \(f(x_1) = f(x_2)\) --- there
exists a \emph{unique} morphism \(f_{*}: Y \unique Z\) such that the following
diagram commutes
\[
\begin{tikzcd}
X \ar[d, two heads, "q", swap] \ar[dr, bend left, "f"] & \\
Y \ar[r, dashed, swap, "f_{*}"] &Z
\end{tikzcd}
\]
\end{theorem}

\begin{proof}
Since \(q\) is surjective, for every \(y \in Y\) there exists \(x \in X\) such
that \(q(x) = y\), hence we define \(f_{*}(y) \coloneq f_{*}(x)\) for every
\(x \in q^{-1}(y)\). For the uniqueness, since \(f\) is constant on the fibers
of \(q\) then \(f_{*}\) is fully defined by \(f\) --- thus unique. From the
universal property we obtain that \(f_{*}\) is continuous.
\end{proof}

\subsection{Some Examples And Applications}

Many important spaces can be obtained with the inclusion of the quotient
topology to our toolkit, I'll now briefly discuss some of those, which will most
probably come up further into this text.

\begin{example}[Projective space]\label{exp:real-projective-space}
Let \(\sim\) be the equivalence relation for which \(x \sim y\) if and only if
\(x = \gamma y\), where \(x, y \in \R^{n+1} \ \{0\}\) and \(\gamma \in \R\). We
define the \(n\)-dimensional real projective space as the quotient \((\R^{n+1}
\setminus \{0\})/{\sim}\), which is denoted by \(\R \Proj^{n}\).
\end{example}

\begin{definition}[Cone]\label{def:cone}
Let \(X\) be any topological space and \(I\) be the standard interval. The
topological space \(X \times I\) is known as the cylinder on \(X\). Via a
quotient operation, we can collapse regions of this cylinder. For instance,
we can create a cone by collapsing one of the sides of the cylinder, such as
\((X \times I) / (X \times \{1\})\). Such object is denoted \(\Cone X\), the
cone on \(X\).
\end{definition}

\begin{definition}[Suspension]
\label{def:suspension}
The \emph{suspension} of a topological space is an endofunctor \(\Susp: \Top \to
\Top\) mapping each space \(X\) to the quotient space
\[
\Susp X \coloneq (X \times I)/(X \times \{0, 1\})
\]
and for each morphism \(f: X \to Y\) we have the naturally induced morphism
\[
\Susp f: \Susp X \to \Susp Y\ \text{ mapping }\ [x, t] \mapsto [f(x), t].
\]
\end{definition}


\begin{example}
\label{exp:wedge-sum-space}
Let \(J\) be an indexing set and \(\{X_j\}_{j \in J}\) be a collection of
non-empty topological spaces. For each \(j \in J\), choose any \(p_j \in X_j\)
as a base point. We define the wedge sum of the collection \(\{X_{j}\}_{j \in
J}\) with respect to the base points \(\{p_{j}\}_{j \in J}\) as the topological
space
\[
  \bigvee_{j \in J} X_j = \bigatt_{j \in J} X_j / \{p_{j}\}_{j \in J}
\]
\end{example}

An interesting fact about wedge sums of topological spaces preserve the
Hausdorff property if every component is Hausdorff.

\begin{proposition}
\label{prop:hausdorff-wedge-sum}
Let \(\{X_{j}\}_{j \in J}\) be an indexed collection of Hausdorff topological
spaces. Then the wedge sum \(\bigvee_{j \in J} X_j\) with respect to any choice
of base points is Hausdorff.
\end{proposition}

\begin{proof}
Let \(\{p_j \in X_j\}_{j \in J}\) be any choice of base points for the given
collection. Let \(x, y \in \bigvee_{j \in J} X_j\) be any distinct points in the
wedge sum space. If \(x, y \in X_j\) for some \(j \in J\), then it is clear that
there exists non-intersecting neighbourhoods of \(x\) and \(y\) on \(X_j\) ---
of which we can take their intersection with the disjoint union \(\bigatt_{j \in
J} X_j\) and the proposition will hold for \(\bigvee_{j \in J} X_j\). On the
other hand, if \(i, j \in J\) are distinct indices and \(x \in X_i\) while \(y
\in X_j\), then there exists \(U_x \subseteq X_i\) and \(U_y \subseteq X_j\)
neighbours of \(x\) and \(y\), respectively, such that \(U_x \cap X_j =
\emptyset\) and \(U_y \cap X_i = \emptyset\), which in particular imply in \(U_x
\cap U_y= \emptyset\). For our end, we just need to consider the neighbourhoods
\(U_x' = U_x \cap \bigatt_{j \in J} X_j\) and \(U_y' = U_y \cap \bigatt_{j \in J}
X_j\) so that \(U_x' = U_y'\).
\end{proof}

\begin{proposition}
\label{prop:quot-second-count-locall-euclidean}
Let \(X\) be a second countable space and \(M = X / {\sim}\) be a quotient. If
\(M\) is locally Euclidean, then \(M\) is second countable.
\end{proposition}

\begin{proof}
Let \(\pi: X \epi M\) be the quotient map that induces the equivalence
relation \(\sim\) in \(X\). If we assume that \(M\) is locally euclidean, we can
let \(\mathcal{C}\) be a cover of \(M\) composed of coordinate balls. Since
\(\pi\) is surjective, \(\mathcal H \coloneq \{\pi^{-1}(U) \colon U \in
\mathcal{C}\}\) is a cover for the space \(X\). Moreover, since \(X\) is second
countable, any cover of \(X\) contains a countable subcover --- in particular,
let \(\mathcal U \subseteq \mathcal H\) be a countable subcover. Define the
countable set \(\mathcal{C}' = \{U \in \mathcal{C} \colon \pi^{-1}(U) \in
\mathcal{U}\}\). Since \(\mathcal{U}\) covers \(X\) and \(\pi\) is surjective,
it follows that \(\mathcal{C}' \subseteq \mathcal{C}\) is a countable subcover
of \(M\) composed of coordinate balls --- that is, \(M\) is Lindelöf. Better
than that, since coordinate balls are second countable (see
\cref{prop:coordinate-ball-second-countable}) we can apply
\cref{cor:second-countable-out-of-cover} to see that \(M\) is second
countable.
\end{proof}

\begin{corollary}[Manifold from a quotient]
\label{cor:manifold-from-quotient}
In the context of the preceding proposition, if \(M\) is both locally Euclidean
and Hausdorff, then \(M\) is a topological manifold.
\end{corollary}

\subsection{Hausdorffness \& Quotient Spaces}

\begin{remark}
\label{rem:quotient-doesnt-preserve-hausdorff}
The quotient topology \emph{doesn't preserve} Hausdorffness.

As an example, take the space \(\R \times \{-1, 1\}\) with the usual topology
and define the equivalence relation \((x, 1) \sim (x, -1)\) if and only if
\(x \neq 0\). The resulting space \((\R \times \{-1, 1\})/{\sim}\) is not
Hausdorff: consider the sequence \((1/n, 1)_{n \in \Z_{>0}}\), we have that
\(1/n \to 0\) but since \(1/n\) does never assume the value of zero, for all
\(n \in \Z_{>0}\) we have \((1/n, 1) \sim (1/n, -1)\) --- thus the has two
distinct limits \((0, 1)\) from one side and \((0, -1)\) from the other, thus
\((\R \times \{-1, 1\})/{\sim}\) isn't Hausdorff.
\end{remark}

\begin{proposition}[Hausdorff from open quotients]
\label{prop:open-quotient-hausdorff}
Let \(\pi: X \epi Y\) be a surjective morphism of topological spaces \(X\) and
\(Y\). Then \(Y\) is Hausdorff if and only if the collection of pairs of points
with common fiber,
\(C \coloneqq \{(p, q) \in X \times X \colon \pi(p) = \pi(q)\}\), is closed in
\(X \times X\).
\end{proposition}

\begin{proof}
Let \(Y\) be Hausdorff, then, given any \((p, q) \in X \setminus C\), there are
neighbourhoods \(V_p, V_q \subseteq Y\) of \(\pi(p)\) and \(\pi(q)\),
respectively, such that \(V_p \cap V_q = \emptyset\). Since these neighbourhoods
are disjoint, then in particular the collection fibers \(\pi^{-1}(V_p) \times
\pi^{-1}(V_q)\) is contained in \(X \times X \setminus C\), that is, \(X \times
X \setminus C\) is open --- hence \(C\) is closed.

Let \(C\) be closed, then given any distinct points \(a, b \in Y\), the
surjectivity of \(\pi\) implies that there exists \(p, q \in X\) such that
\(\pi(p) = a\) and \(\pi(q) = b\) --- in particular \((p, q) \in X \times X
\setminus C\) and since \(C\) is closed, there exists a neighbourhood \(U_p
\times U_q \subseteq X \times X\) of \((p, q)\) such that \(U_p \times U_q
\subseteq X \times X \setminus C\), that is, \(\pi(U_p), \pi(U_q) \subseteq Y\)
are non intersecting open sets (from the fact that \(\pi\) is open) that are
neighbourhoods of \(a\) and \(b\), respectively --- thus \(Y\) is Hausdorff.
\end{proof}

\begin{proposition}
\label{prop:top-space-is-quotient-of-hausdorff}
Every topological space is the quotient of a Hausdorff space.
\end{proposition}

\begin{proof}
Let \(X\) be any topological space. Consider the product of real lines
\(P \coloneq \prod_{x \in X} \R\) under the product topology and let \(Y\) be
the subspace of \(P\) given by all points with one, and only one, rational
coordinate --- such space \(Y\) is therefore Hausdorff. Furthermore, define a
collection \(\{Y_{x}\}_{x \in X}\) to consist of subsets \(Y_x \subseteq Y\)
given by the set of points whose rational coordinate has index \(x\) --- hence
\(Y_x\) is dense in \(Y\). It should be noted that the above construction is
not necessarily unique, so the reader may try to construct a different Hausdorff
space \(Y\) and a collection of dense sets \(\{Y_{x}\}_{x \in X}\) in \(Y\).

Define now the subspace of \(X \times Y\) given by
\(Z \coloneq \bigcup_{x \in X} x \times Y_x\) --- endowed with the subspace
topology. Let \(\sim\) be the equivalence relation on \(Z\) given by
\((x_1, y_1) \sim (x_2, y_2)\) if and only if \(x_1 = x_2 = x\) for some
\(x \in X\) and \(y_1, y_2 \in Y_x\). Define a map \(\phi: Z/{\sim} \to X\)
sending \(x \times Y_x \mapsto x\) --- which is clearly both surjective and
injective. We now show that \(\phi\) is a topological isomorphism.

Let \(V \subseteq X\) be any open set, then
\(\phi^{-1}(V) = \{x \times Y_x \colon x \in V\}\) --- which in turn is open in
\(Z/{\sim}\). On the other hand, let
\(U \coloneq \{x \times Y_{x} \colon x \in U'\}\) be any open set in
\(Z/{\sim}\) given by some indexing set of points \(U' \subseteq X\) --- our
goal will be to prove that \(U'\) is open, so that the inverse of \(\phi\) is
continuous. Consider \(x_0 \in U'\) to be any point and likewise
\((x_0, y_0) \in x_0 \times Y_{x_0}\). By the fact that
\(\bigcup_{x \in U'} x \times Y_x \subseteq Z\) is open, we are able to find
neighbourhoods \(O_X(x_0) \subseteq X\) and \(O_Y(y_0) \subseteq Y\) --- of
\(x_0\) and \(y_0\), respectively --- such that
\(N \coloneq Z \cap (O_X(x_0) \times O_Y(y_0))\) is a neighbourhood of
\((x_0, y_0)\) in \(Z/{\sim}\). If \(x' \in O_X(x_0)\) is any point, then by the
fact that \(Y_{x'}\) is dense in \(Y\), by
\cref{prop:dense-non-empty-intersects}, the sets \(x' \times Y_{x'}\) and \(N\)
have a non-empty intersection and thus \(x' \times Y_{x'}\) also intersects
\(\bigcup_{x \in U'} x \times Y_x\). Therefore \(x' \times Y_{x'} \in U\) and
hence \(x' \in U'\) --- which implies that \(U'\) is open.
\end{proof}

\subsection{Quotient Morphisms In More Depth}

So far we've been studying the construction of quotients out of surjective
set-functions, but what about being able to classifying a surjective morphisms
between given topological spaces as inducing the universal property of the
quotient space? This will be our goal with this subsection --- identifying
quotient morphisms. For that end, we shall profit from the main idea behind
quotients: fibers. For that, we define a set given by fibers of \(f\) as being
saturated.

\begin{definition}[Saturated set]
\label{def:saturated-fiber}
Let \(f: X \to Y\) be a set-function. We say that a set \(U \subseteq X\) is
saturated with respect to \(f\) if there exists \(V \subseteq Y\) such that
\(U = f^{-1}(V)\).
\end{definition}

\begin{proposition}[Equivalences for saturated sets]
\label{prop:equivalences-saturated-fiber}
Let \(f: X \to Y\) be a set-function and \(U \subseteq X\) be any subset. The
following propositions are equivalent
\begin{enumerate}[(a)]\setlength\itemsep{0em}
\item The set \(U\) is saturated with respect to \(f\).
\item \(U = f^{-1}(f(U))\).
\item Let \(p \in U\) be any point, \(U\) contains every element \(x \in X\)
  with common fiber to \(p\) --- that is, \(f(x) = f(p)\) implies \(x \in U\).
\end{enumerate}
\end{proposition}

\begin{proof}
(c) \(\Rightarrow\) (b): Suppose \(U\) satisfies proposition (c), it is clear
that \(U \subseteq f^{-1}(f(U))\), on the other hand, given \(x \in
f^{-1}(f(U))\), it follows that \(x\) has a common fiber with some point of
\(U\), which implies that \(x \in U\). (b) \(\Rightarrow\) (a): Trivial from the
definition. (a) \(\Rightarrow\) (c): Let \(V \subseteq Y\) be such that \(U =
f^{-1}(V)\), then, given any \(p \in f^{-1}(V)\), it is clear that \(f(p) \in
V\), hence every point \(x \in X\) such that \(f(x) = f(p) \in V\) then \(x \in
U\), which finishes the equivalence chain.
\end{proof}

\begin{proposition}[Classification of surjective morphisms]
\label{prop:surjective-saturated-is-quotient}
Let \(\pi: X \epi Y\) be a surjective morphism of topological spaces. The map
\(\pi\) is a quotient morphism --- that is, induces the universal property of
quotients for \(X\) and \(Y\) --- if and only if every saturated open (or
closed) set of \(X\) has an open (or closed) image in \(Y\).
\end{proposition}

\begin{proof}
Let \(\pi\) be any surjective morphism taking saturated open sets to open
images. Let \(V \subseteq Y\) be an open set. Since \(\pi\) is surjective and
continuous, then \(\pi^{-1}(V) \subseteq X\) is open. On the other hand, let \(V
\subseteq Y\) be any set of \(Y\) (not necessarily open), such that
\(\pi^{-1}(V) \coloneq U \subseteq X\) is open. This implies directly that \(U\)
is saturated with respect to \(\pi\) and from our initial hypothesis, \(\pi(U) =
V \subseteq Y\) is open. Thus \(\pi\) is a quotient morphism.

For the contrary, let \(\pi\) be a quotient morphism. Then, given any \(U
\subseteq X\) open set, saturated with respect to \(\pi\), define \(V \subseteq
Y\) such that \(U = \pi^{-1}(V)\). Since \(\pi\) is a quotient morphism, it
follows that \(V\) is necessarily open in \(Y\), thus \(\pi(U) = V\) is open.

The proof for the closed set case is completely analogous.
\end{proof}

\begin{proposition}[Properties of quotient morphisms]
\label{prop:properties-quotient-morphism}
The following properties pertain to quotient morphisms between topological
spaces.
\begin{enumerate}[(a)]\setlength\itemsep{0em}
\item The composition of quotient morphisms is a quotient morphism.
\item Injective quotient morphisms are isomorphisms.
\item Let \(\pi: X \epi Y\) be a quotient morphism. Then, \(C \subseteq Y\) is
  closed if and only if \(\pi^{-1}(C) \subseteq X\) is closed.
\item Let \(\pi: X \epi Y\) be a quotient morphism and \(U \subseteq X\) be any
  saturated set (open or closed) with respect to \(\pi\). Then, the restriction
  \(\pi|_U: U \epi \pi(U)\) is a quotient map.
\item Let \(J\) be an indexing set and \(\{\pi_j: X_j \epi Y_j\}_{j \in J}\) be
  an indexed collection of quotient morphisms. The map \(\pi: \bigatt_{j \in J}
  X_j \epi \bigatt_{j \in J} Y_j\) defined by the restrictions \(\pi(x_{j}) =
  \pi_{j}(x_j)\) for every \(x_j \in X_{j} \cap \bigatt_{j \in J} X_j\) is a
  quotient map.
\end{enumerate}
\end{proposition}

\begin{proof}
\begin{enumerate}[(a)]\setlength\itemsep{0em}
\item Let \(\pi: X \epi Y\) and \(\pi': Y \epi Z\) be quotient morphisms, and
  consider the map \(\pi' \pi: X \epi Z\), which is clearly surjective. From
  hypothesis a subset \(U \subseteq Z\) is open if and only if \(\pi'^{-1}(U)\)
  is open, moreover, \(\pi'^{-1}(U) \subseteq Y\) is open if and only if
  \(\pi^{-1}(\pi'^{-1}(U)) \subseteq X\) is open --- the proposition follows.
\item If the quotient morphism \(\pi: X \epi Y\) is injective, then \(\pi\) is a
  bijection. Let \(U \subseteq X\) be any open set, since \(\pi\) is bijective,
  there exists a set \(V \subseteq Y\) such that \(\pi^{-1}(V) = U\) ---
  moreover, such set must be open on \(Y\) from the quotient topology. This
  shows that \(\pi(U) = V\) is open and hence \(\pi\) is a topological
  isomorphism.
\item Let \(C \subseteq Y\) be any set, notice that \(Y \setminus C\) is open
  if and only if \(\pi^{-1}(Y \setminus C) = \pi^{-1}(Y) \setminus \pi^{-1}(C)
  \subseteq X\) is open --- thus the proposition follows.
\item Let \(V \subseteq \pi(U)\) be any set. Since \(\pi\) is a quotient
  morphism, \(V\) is open if and only if \(\pi^{-1}(V) \subseteq U \subseteq
  X\) --- moreover, since \(U\) is saturated, the whole set can have its subsets
  classified by \(\pi|_U\) into open or closed sets, thus \(\pi|_U\) is a
  quotient map.
\item Let \(U \subseteq \bigatt_{j \in J} Y_j\) be any set and let \(j_{0} \in
  J\) be such that \(U \subseteq Y_{j_0}\). From the mappings, \(U\) is open if
  and only if \(\pi^{-1}_{j_0}(U) \subseteq X_{j_0}\) is open --- but since
  \(\pi^{-1}(U) = \pi^{-1}_{j_0}(U)\), the proposition follows.
\end{enumerate}
\end{proof}

\begin{example}[Cones]
\label{exp:remove-section-cone-isomorphism}
An application of the last proposition takes us back to \cref{def:cone}, where
we defined the cone \(\Cone X\) of a topological space \(X\) as \((X \times I) /
(X \times \{1\})\) --- the collapse the top of the cylinder. Notice that, given
any point \((x, t) \in (X \times I) \setminus (X \times \{1\})\), there exists a
neighbourhood \(U \subseteq (X \times I) \setminus (X \times \{1\})\) of \((x,
t)\), therefore \(X \times \{1\}\) is closed in \(X \times I\). If we consider
the quotient morphism of the cone \(\pi: X \times I \epi \Cone X\), the
restriction \(\pi|_{X \times \{0\}}: X \times \{0\} \epi \pi(X \times \{0\})\)
is also a quotient map --- moreover, such quotient map is injective, thus
\(\pi|_{X \times \{0\}}\) is an isomorphism. Therefore we have a sequence of
isomorphisms
\[
  X \isoto X \times \{0\} \isoto \pi(X \times \{0\}) \subseteq \Cone X,
\]
thus we can identify \(X\) as a subspace of \(\Cone X\).
\end{example}

The following is a \emph{sufficient}, but \emph{not} necessary condition for a
quotient morphism.

\begin{proposition}
Let \(\pi: X \epi Y\) be a surjective topological morphism. If \(\pi\) is either
open or closed, then \(\pi\) is a quotient morphism.
\end{proposition}

\begin{proof}
If \(\pi\) is open (respectively, closed), in particular we have that saturated
sets open (respectively, closed) of \(X\) are mapped to open (respectively,
closed) sets of \(Y\), hence \(\pi\) is a quotient morphism.
\end{proof}

\begin{proposition}
\label{prop:map-open-or-closed-properties}
Let \(f: X \to Y\) be a topological morphism that is either open or closed. The
following are properties hold:
\begin{enumerate}[(a)]\setlength\itemsep{0em}
\item If \(f\) is injective, it is a topological \emph{embedding}.

\item If \(f\) is surjective, it is a \emph{quotient map}.

\item If \(f\) is bijective, it is an \emph{isomorphism}.
\end{enumerate}
\end{proposition}

\begin{proof}
We work out the proof for the case where \(f\) is open, the closed case is
equivalent:
\begin{enumerate}\setlength\itemsep{0em}
\item Consider the restriction of the codomain \(f': X \to f(X)\) --- which is
  certainly surjective, thus bijective. Let \(g: f(X) \to X\) denote the inverse
  of \(f\). Notice that since \(f\) is open then any \(U \subseteq X\) open
  implies in \(f(U) \subseteq Y\) also open, therefore \(g(U) = f(U)\) is open
  and hence \(g\) is continuous.

\item Let \(U \subseteq Y\) be any set. Since \(f\) is surjective, there exists
  \(V \subseteq X\) such that \(f(V) = U\). Moreover, since \(f\) is open, \(U\)
  can only be open in \(Y\) if its preimage \(f^{-1}(U) = V\) is open in \(X\)
  --- which implies that \(f\) is a quotient map.

\item If \(f\) is bijective, then by the fist item \(f\) is an embedding, that
  is, it yields an isomorphism \(X \iso f(X)\). Moreover since \(f\) is also
  surjective then \(f(X) = Y\) and thus \(f\) is an isomorphism \(X \isoto Y\).
\end{enumerate}
\end{proof}

\section{Attaching Space}

\begin{definition}[Attaching space]
\label{def:attaching-space}
Let \(X\) and \(Y\) be topological spaces, and consider a subspace
\(A \subseteq X\) together with a continuous map \(f: A \to X\). The
\emph{attaching space of \(X\) and \(Y\) along \(f\)} is defined to be the
pushout of the canonical inclusion \(\iota: A \emb X\) and \(f\), that is
\[
\begin{tikzcd}
A \ar[r, hook, "\iota"] \ar[d, "f"']
\ar[rd, phantom, very near end, "\ulcorner"]
&X \ar[d] \\
Y \ar[r] &X \cup_f Y
\end{tikzcd}
\]
\end{definition}

\begin{corollary}
\label{cor:attaching-space-as-quotient}
In the notation of \cref{def:attaching-space}, the attaching space of \(X\) and
\(Y\) along \(f\) is given by the quotient space
\[
X \cup_f Y \iso (X \disj Y)/{\sim},
\]
where \(\sim\) is the smallest equivalence relation on \(X \disj Y\) such that
\(x \sim f(x)\) for all \(x \in A\).
\end{corollary}

\begin{proof}
Let \(Z\) be any space together with two continuous maps \(p: X \to Z\) and
\(q: Y \to Z\) such that \(p \iota = q f\) --- that is, for every \(x \in A\) we
have \(p(x) = q(f(x))\). We define a map \(\phi: (X \disj Y)/{\sim} \to Z\) by
\([(x, X)] \mapsto p(x)\) and \([(y, Y)] \mapsto q(y)\). Indeed, the image of a
class point under \(\phi\) does not depend on the representative since, for any
\(x \in A\), we have \([(x, X)] \sim [(f(x), Y)]\), but \(p(x) =
q(f(x))\). Also, since \([X] \coloneq \{[(x, X)]\}_{x \in X}\) and
\([Y] \coloneq \{[(y, Y)]\}_{y \in Y}\) are open subspaces covering
\((X \disj Y)/{\sim}\), since \(\phi|_{[X]} = p\) and \(\phi|_{[Y]} = q\) are
continuous maps, by \cref{prop:continuous-from-covering-subspaces} we conclude
that \(\phi\) is continuous.

If we consider the inclusions \(\iota_X: X \emb (X \disj Y)/{\sim}\) and
\(\iota_Y: Y \emb (X \disj Y)/{\sim}\), one has that \(\phi \iota_X = p\) and
\(\phi \iota_Y = q\). On the other hand, since these inclusions are
monomorphisms in \(\Top\), we conclude that \(\phi\) is the only continuous map
making the following diagram commutative in \(\Top\)
\[
\begin{tikzcd}
A \ar[r, hook, "\iota"] \ar[d, "f"']
&X \ar[drd, bend left, "p"] \ar[d, hook, "\iota_X"]  &\\
Y \ar[r, "\iota_Y", hook] \ar[rrd, "q"', bend right]
&(X \disj Y)/{\sim} \ar[rd, "\phi"', dashed]  &\\
& &Z
\end{tikzcd}
\]
We conclude that \((X \disj Y)/{\sim}\) is the pushout of \(f\) with \(\iota\)
and since pushouts are unique up to isomorphism, then
\((X \disj Y)/{\sim} \iso X \cup_f Y\).
\end{proof}

\begin{proposition}[Attaching space properties]
\label{prop:attaching-space-properties}
Let \(A \subseteq Y\) be a \emph{closed} subspace, \(f: A \to X\) be a
continuous map, and \(\pi: X \disj Y \epi X \cup_f Y\) be the canonical
projection to the attaching space along \(f\). The following properties hold:
\begin{enumerate}[(a)]\setlength\itemsep{0em}
\item The restriction \(\pi|_X\) is a \emph{topological embedding}, and its
  image \(\pi(X)\) is a \emph{closed subspace} of \(X \cup_f Y\).

\item The restriction \(\pi|_{Y \setminus A}\) is a \emph{topological
    embedding}, and its image \(\pi(Y \setminus A)\) is an \emph{open subspace}
  of \(X \cup_f Y\).

\item The attaching space \(X \cup_f Y\) is isomorphic to the disjoint union
  \(\pi(X) \disj \pi(Y \setminus A)\).
\end{enumerate}
\end{proposition}

\begin{proof}
\begin{enumerate}[(a)]\setlength\itemsep{0em}
\item
  \todo[inline]{Continue here}
\end{enumerate}
\end{proof}

%%% Local Variables:
%%% TeX-master: "../../deep-dive"
%%% End:
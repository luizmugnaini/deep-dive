\section{Functors}

\begin{definition}[Covariant functor]\label{def: functor}
Let \(\cat C\) and \(\cat D\) be categories. A covariant functor \(F: \cat C \to
\cat D\) has the following data:
\begin{enumerate}[({DF}1)]
\item For all \(c \in \cat C\) exists a corresponding \(F c \in \cat
  D\)\footnote{When convenient, we may discard the use of parenthesis, but in
    occasions where the use of parenthesis brings more clarity to the
    situation, we shall use it.}.
\item For all \(f: c \to c' \in \Mor(\cat C)\) there exists a morphism \(F
  f: F c \to F c' \in \Mor(\cat D)\).
\end{enumerate}
Such data satisfies the two following axioms:
\begin{enumerate}[({AF}1)]
\item For all composable \(f, g \in \Mor(\cat C)\), we have \(F g \circ F f
  = F (g \circ f)\).
\item For all \(c \in \cat C\) we have \(F \Id_c = \Id_{F c} \in \Mor(\cat
  D)\).
\end{enumerate}
\end{definition}

\begin{definition}[Contravariant functor]
\label{def: contravariant functor}
A contravariant functor from categories \(\cat C\) to \(\cat D\) is a functor
\(F: \cat C^\op \to \cat D\) together with the following data:
\begin{enumerate}[({DCF}1)]
\item For all \(c \in \cat C\) exists \(F c \in \cat D\).
\item For all \(f: c \to c' \in \Mor(\cat C)\) we have \(F f: F c' \to F c
  \in \Mor(\cat D)\).
\end{enumerate}
Moreover, a contravariant functor satisfies the following axioms:
\begin{enumerate}[({ACF}1)]
\item For all composable \(f, g \in \Mor(\cat C)\) we have \(F f \circ F g =
  F (g \circ f)\).
\item For all \(c \in \cat C\) we have \(F \Id_c = \Id_{F c}\).
\end{enumerate}
This can all be comprised diagrammatically as:
\[
  \begin{tikzcd}
    \cat C^\op \ar[rr, "F"] & &\cat D
    \\
    c \ar[d, "f"] \ar[dd, bend right = 60, swap, "g f"]
    \ar[rr, mapsto]
    & &F c
    \\
    c' \ar[d, "g"] \ar[rr, mapsto]
    & &F c' \ar[u, "F f"]
    \\
    c'' \ar[rr, mapsto]
    & & F c''
    \ar[u, "F g"]
    \ar[uu, bend right = 60, swap, "F f \circ F g = F (gf)"]
  \end{tikzcd}
\]
\end{definition}

\begin{definition}[Composition of functors]
\label{def:composition-functors}
Let \(F: \cat C \to \cat D\) and \(G: \cat D \to \cat B\) be functors. We define
their composition \(G F: \cat C \to \cat B\) by \((GF)(x) = G(F(x))\), for all
\(x \in \cat C\), and \((G F)(f) = G(F(f))\) for all morphism \(f \in \Mor(\cat
C)\).
\end{definition}

\begin{example}
\label{exp:opposite-category}
An important example of contravariant functor is \(\op: \cat C \to \cat
C^{\op}\), where \(\cat C\) is some category, defined by the identity on objects
and morphisms.
\end{example}

\begin{definition}[Forgetful functor]
A functor is said to be forgetful if the functor ``forgets'' some object,
structure or property of its domain category.
\end{definition}

\begin{example}
We have some classical examples of forgetful functors, for instance, the
following are functors that forget the structure of their domain categories:
\begin{itemize}
\item The functor \(G: \Grp \to \Set\) mapping groups to its corresponding
  underlying set.
\item The functor \(T: \Top \to \Set\) maps any topological space
  to its corresponding set of points.
\item The functor \(V, E: \Graph \to \Set\) maps the vertices and
  edges of a graph to the set of such vertices an edges.
\end{itemize}
\end{example}

\begin{example}[\(\Top^\op \to \Rng\)]
Let \(C: \Top^\op \to \Rng\) be a contravariant functor such that
for all \(X \in \Top\), let \(C X\) be the ring of continuous functions
\(X \to \R\). The ring operations on \(C X\) are defined pointwise,
that is, given \(p, q: X \to \R \in C X\) we have \((p \cdot q) (x) =
p(x) \cdot q(x)\) and \((p + q)(x) = p(x) + q(x)\) for all \(x \in X\).
Moreover, given a morphism \(f: X \to Y \in \Mor(\Top)\) we define \(C f:
C Y \to C X\) as the composition \((C f)(q) = q f \in \Mor(C X)\) for all \(q
\in C Y\), that is
\[
  \begin{tikzcd}
    X \ar[r, "f"] \ar[rr, bend right = 40, swap, "C f(q) = q f"]
    & Y \ar[r, "q"] & \R
  \end{tikzcd}
\]

We now show that the axioms for the contravariant functor are satisfied by
\(C\). Let \(f: X \to Y\) and \(g: Y \to Z\), then given any \(p \in C Z\) we
have
\[
  \begin{tikzcd}
    X \ar[r, "f"]
    \ar[rrr, bend left = 60, "C f(C g (p))"]
    \ar[rrr, bend right, swap, "C (g f)(p)"]
    & Y \ar[r, "g"]
    \ar[rr, bend left = 35, "C g(p)"]
    & Z \ar[r, "p"] & \R
  \end{tikzcd}
\]
hence \(C (f) C (g) = C (g f)\). Moreover, given any \(X \in \Top^\op\) we find
that \(C \Id_X: C X \to C X\) is such that for all \(q \in C X\),
\(C \Id_X (q) = q \Id_X = q\) hence \(C \Id_X = \Id_{C X}\). This finishes the
proof that \(C: \Top^\op \to \Rng\) is a contravariant functor.
\end{example}

\begin{definition}[Presheaf]\label{def: presheaf}
Let \(\cat C\) be a \(\mathcal U\)-small category. A contravariant functor
\(\cat C^\op \to \Set\) is called a presheaf on \(\cat C\).
\end{definition}

\begin{example}[\(\mathcal O(X)^\op \to \Set\)]
Let \(X \in \Top\) we define \(\mathcal O(X)\) to be the poset category whose
objects are open sets of \(X\). That is, for sets \(U, U' \in \mathcal O(X)\), if
\(U \subseteq U'\), then there exists a morphism \(U \to U'\) in
\(\Mor(\mathcal O(X))\). A presheaf on the category \(\mathcal O(X)\) is a
functor \(F: \mathcal O(X)^\op \to \Set\) that assigns
\(F U = \{f: U \to \R \colon f \text{ continuous}\}\) for all
\(U \in \mathcal O(X)\).  Moreover, for maps \(g: U \to U'\) (that is
\(U \subseteq U'\)) we have \(F g: F U' \to F U\) such that
\(F g(f) = f|_U: U \to \R\) for all \(f: U' \to \R\) continuous. Since the
restriction of a continuous map is continuous, then \(f|_U \in F U\)
\end{example}

\begin{example}[Simplex category]
\label{exp:simplex-category}
The simplex category \(\Delta\) comprises objects that are finite non-empty
ordinals and order-preserving morphisms. Simplicial sets are defined as
presheaves \(\Delta^{\op} \to \Set\).
\end{example}

\begin{lemma}\label{lem: functor preserve iso}
Functors preserve isomorphisms. Let \(\cat C\) and \(\cat D\) be categories
and \(F: \cat C \to \cat D\) be a functor. Given an isomorphism \(f: c \isoto
c' \in \Mor(\cat C)\), we have that \(F f : F c \isoto c'\) is an isomorphism.
\end{lemma}

\begin{proof}
Denote by \(f^{-1}: c' \isoto c\) the inverse of \(f\). By the composition
axiom we have
\begin{gather*}
  F (f^{-1}) F (f) = F (f^{-1} f) = F \Id_c    = \Id_{F c}, \\
  F (f) F (f^{-1}) = F (f f^{-1}) = F \Id_{c'} = \Id_{F c'}.
\end{gather*}
This shows that \(F f^{-1}\) is the right and left inverse of \(F f\), hence
\(F f: F c \isoto F c'\) is indeed an isomorphism.
\end{proof}

\begin{example}[Group action]\label{exp:grp-action}
Let \(G\) be any group and consider the category \(\cat{B}G\) generated by \(G\)
--- that is, \(\cat BG\) consists of a unique object \(*\) and the morphisms of
the category are automorphisms \(* \to *\) given by the elements of \(G\). Given
a category \(\cat C\), a functor \(X: \cat BG \to \cat C\) --- given by mapping
\(X* \coloneq X \in \cat C\) and each object \(g \in G\) to an endomorphism
\(Xg \coloneq g_{*}: X \to X\) --- is said to define a left group action on the
object \(X \in \cat C\). Moreover, the functor \(X\) has to obey
\begin{itemize}\setlength\itemsep{0em}
\item Composition preserving: for any \(h, g \in G\), we have \((h g)_{*} =
  h_{*} g_{*}\).
\item Identity: \(e_{*} = \Id_X\).
\end{itemize}
Since functors preserve isomorphisms and every morphism of \(\cat BG\) is an
automorphism, it follows that, for all \(g \in G\), the map \(g_{*}: X \to X\)
is an automorphism --- in particular, this implies that \((g^{-1})_{*} =
g_{*}^{-1}\).

Some particular cases of interest are the following:
\begin{itemize}\setlength\itemsep{0em}
\item If \(\cat C = \Set\), then the set \(X\) together with the actions
  \(\{g_{*} \colon g \in G\}\) is called a \(G\)-set.
\item If \(\cat C = \Vect_k\), then the \(k\)-vector space \(X\) together with
  the actions generated by \(G\) is said to be a \(G\)-representation.
\item If \(\cat C = \Top\), then the topological space \(X\) endowed with the
  actions generated by \(G\) is called a \(G\)-space.
\end{itemize}

A right group action is nothing more than a contravariant functor \(X: \cat
BG^{\op} \to \cat C\) such that \(X* \coloneq X\) and \(Xg \coloneq g^{*}: X \to
X\) are endomorphisms. The rules for such a functor are the contravariant
preservation of compositions, that is, \((h g)_{*} = g_{*} h_{*}\), and that
\(e^{*} = \Id_X\) as before.
\end{example}

\begin{example}[Skeletal functor]
\label{exp:skeletal-functor}
Let \(\cat C\) be a category. The inclusion functor \(J: \Sk \cat C \to \cat C\)
equivalence of categories, indeed, one can define a quasi-inverse functor
\(F: \cat C \to \Sk \cat C\) by mapping \(a \in \cat C\) to the unique object
\(F a \in \Sk \cat C\) for which there exists an isomorphism \(a \iso F
a\). Choosing a collection \((\alpha_a: a \isoto F a)_{a \in \cat C}\) of
isomorphisms where \(\alpha_a = \Id_a\) whenever \(F a = a\), we can map each
morphism \(f: a \to b\) of \(\cat C\) to the morphism
\(F f \coloneq \alpha_b f \alpha_a^{-1}: F a \to F b\). From this construction
one gets \(F J = \Id\) and the collection \((\alpha_a)_{a \in \cat A}\) of
chosen isomorphisms define a natural isomorphism \(\alpha: \Id \isonat J F\).
\end{example}

\begin{lemma}\label{lem: func-preserve-split}
Functors preserve split monomorphisms and split epimorphisms.
\end{lemma}

\begin{proof}
Let \(\cat C\) and \(\cat D\) be categories and consider a functor \(F: \cat C
\to \cat D\). Define morphisms \(x \xrightarrow s y \xrightarrow r x\) in
\(\Mor(\cat C)\) such that \(r s = \Id_x\), that is, \(s\) is a split
monomorphism and \(r\) is a split epimorphism. Consider the morphisms \(F s: F
x \to F y\) and \(F r: F y \to F x\) in \(\Mor(\cat D)\). Notice that \(F (s)
F(r) = F(s r) = F(\Id_x) = \Id_{F x}\). Hence \(F s\) is a split monomorphism
and \(F r\) is a split epimorphism.
\end{proof}

\begin{definition}[\(\Hom\) functors]\label{def:hom-functors}
Let \(\cat C\) be a \(\mathcal U\)-category. Given any \(c \in \cat C\), there
exists a pair of covariant and contravariant functors, \(\Hom(c, -)\) and
\(\Hom(-, c)\), respectively --- represented by the object \(c\). That is:
\[
  \begin{tikzcd}
    \cat C \ar[rr, "\Hom{(c, -)}"] & & \Set
    \\
    x \ar[rr, maps to] \ar[d, swap, "f"]
    & & \Hom(c, x) \ar[d, "f_*"]
    \\
    y \ar[rr, maps to] & & \Hom(c, y)
  \end{tikzcd}
  \qquad
  \begin{tikzcd}
    \cat C^\op \ar[rr, "\Hom{(-, c)}"] & & \Set
    \\
    x \ar[rr, maps to] \ar[d, swap, "f"]
    & & \Hom(x, c)
    \\
    y \ar[rr, maps to] & & \Hom(y, c) \ar[u, swap, "f^*"]
  \end{tikzcd}
\]
\end{definition}

We now prove that such definition indeed satisfies the axioms for covariant and
contravariant functors. Given morphisms \(f: x \to y\) and \(g: y \to z\) in
\(\Mor(\cat C)\), we see that \(g_* f_* = (g f)_*\), moreover \(f^* g^* = (f
g)^*\). Let \(x \in \cat C\) be any object, then \(\Id_{x *} = \Id_{\Hom(c, x)}
= \Id_x^*\). This proves that \(\Hom(c, -)\) is covariant and \(\Hom(-, c)\) is
contravariant.

\begin{definition}[Faithful, full \& its friends]
\label{def:faithful-full-fully-faithful-essentially-surjective-conservative}
Let \(\cat C\) and \(\cat D\) be categories. A functor \(F: \cat C \to \cat D\)
is said to be
\begin{enumerate}[(a)]\setlength\itemsep{0em}
\item \emph{Faithful} if for all \(x, y \in \cat C\) the map \(\Hom_{\cat C}(x,
  y) \mono \Hom_{\cat D}(F x, F y)\) is injective.
\item \emph{Full} if for all \(x, y \in \cat C\) the map \(\Hom_{\cat C}(x, y)
  \epi \Hom_{\cat D}(F x, F y)\) is surjective.
\item \emph{Fully faithful} if for all \(x, y \in \cat C\) the map \(\Hom_{\cat
    C}(x, y) \isoto \Hom_{\cat D}(F x, F y)\) is a bijection.
\item \emph{Essentially surjective} if for each \(y \in \cat D\) there exists
  \(x \in \cat C\) and an isomorphism \(F x \isoto y\) in \(\cat D\).
\item \emph{Conservative} if, given a morphism \(f\) in \(\cat C\), if \(F f\)
  is an isomorphism in \(\cat D\), then \(f\) is an isomorphism in \(\cat C\).
\end{enumerate}
\end{definition}

\begin{proposition}[Fully faithful functors and isomorphisms]
\label{prop:fully-faithful-image-iso-then-obj-iso}
Let \(F: \cat C \to \cat D\) be a \emph{fully faithful} functor. If
\(F x \iso F y\) in \(\cat D\), for some \(x, y \in \cat C\), then \(x \iso y\)
in \(\cat C\) for a \emph{unique} isomorphism.
\end{proposition}

\begin{proof}
If \(F x \iso F y\), let \(\phi: F x \isoto F y\) be an isomorphism in
\(\cat D\). Since \(F\) is fully faithful, there exists unique morphisms
\(f: x \to y\) and \(g: y \to x\) in \(\cat C\) for which \(F f = \phi\) and
\(F g = \phi^{-1}\). In particular, one has
\[
\Id_{F x} =  \phi^{-1} \phi = F g \circ F f = F(g f).
\]
From the faithfulness of \(F\), since \(F \Id_x = \Id_{F x}\), then
\(g f = \Id_x\). On the other hand, one also has
\[
\Id_{F y} = \phi \phi^{-1} = F f \circ F g = F(f g),
\]
therefore, since \(F \Id_y = \Id_{F y}\) we obtain \(f g = \Id_y\). We can now
finally conclude that \(f\) is an isomorphism and its inverse is \(g\).
\end{proof}

\begin{definition}[Product \& disjoint union categories]
\label{def:product-disjoint-categories}
Let \(I\) be an indexing set and \(\{\cat C_{i}\}_{i \in I}\) be a collection of
categories associated with \(I\). We define the following categories:
\begin{enumerate}[(a)]\setlength\itemsep{0em}
\item The \emph{product} category \(\prod_{i \in I} \cat C_i\) consists of
  objects
  \(\Obj(\prod_{i \in I} \cat C_i) \coloneq \prod_{i \in I} \Obj(\cat C_i)\),
  and morphisms
  \(\Hom_{\prod_{i \in I} \cat C_i}((x_i)_{i \in I}, (y_i)_{i \in I}) \coloneq
  \prod_{i \in I} \Hom_{\cat C_i}(x_i, y_i)\) between any two objects
  \((x_i)_{i \in I}\) and \((y_i)_{i \in I}\) in the category. Composable
  morphisms \((f_i)_{i \in I}\) and \((g_i)_{i \in I}\) have composition defined
  component-wise --- that is,
  \((f_i)_{i \in I} (g_i)_{i \in I} \coloneq (f_i g_i)_{i \in I}\).
\item The \emph{disjoint union} category \(\bigdisj_{i \in I} \cat C_i\) consists
  of objects
  \[
    \Obj\bigg(\bigdisj_{i \in I} \cat C_i\bigg) \coloneq
    \bigdisj_{i \in I} \{(x, i) \colon i \in I \text{ and } x \in \cat C_i\},
  \]
  and morphisms
  \[
    \Hom_{\bigdisj_{i \in I} \cat C_i}((x, i), (y, j)) \coloneq
    \begin{cases}
      \Hom_{\cat C_i}(x, y), &\text{if } i = j \\
      \emptyset, &\text{otherwise}
    \end{cases}
  \]
  for any objects \((x, i)\) and \((y, j)\) in the category.
\end{enumerate}
Moreover, if \(\{\cat D_{i}\}_{i \in I}\) is another collection of categories,
and \(\{F_{i}: \cat C_i \to \cat D_i\}_{i \in I}\) is a collection of functors,
we associate to each of the above categories the functors \(\prod_{i \in I}
F_i\) and \(\bigdisj_{i \in I} F_i\).
\end{definition}

\begin{example}[Orbit category]
\label{exp:orbit-category}
Let \(G\) be a group. We define the orbit category associated to \(G\) as
\(\mathcal{O}_G\) whose objects are subgroups \(H \subseteq G\), identified by
the left \(G\)-set \(G/H\) of left cosets of \(H\). The morphisms \(\phi: G/H \to
G/Q\) are maps commuting with the left \(G\)-action, that is, \(\phi(g_{*}h) =
g_{*} \phi(h)\) --- such maps are called \(G\)-equivariant.
\end{example}

\begin{proposition}[Bifunctor]\label{def:bifunctor}
Let \(\cat A, \cat B\) and \(\cat C\) be any categories. A functor
\[
  F: \cat A \times \cat B \longrightarrow \cat C
\]
is called a \emph{bifunctor}. The functor \(F\) is defined so that, given any
objects \(x \in \cat A\) and \(y \in \cat B\),
\[
  F(x, -): \cat B \longrightarrow \cat C\ \text{ and }\
  F(-, y): \cat A \longrightarrow \cat C
\]
are both functors. Moreover, given any morphisms \(f: x \to y\) in \(\cat A\)
and \(g: x' \to y'\) in \(\cat B\), the following diagram commutes
\[
  \begin{tikzcd}
    F(x, x') \ar[rr, "F{(x, g)}"] \ar[d, swap, "F{(f, x')}"]
      & &F(x, y') \ar[d, "F{(f, y')}"] \\
    F(y, x') \ar[rr, "F{(y, g)}"]
      & &F(y, y')
  \end{tikzcd}
\]
\end{proposition}

Notice that the product of small categories can also be understood as a
bifunctor
\[
\times: \Cat \times \Cat \longrightarrow \Cat
\]
which associates to each pair of small categories \((\cat A, \cat B)\) the
product category \(\cat A \times \cat B\), and any pair of functors
\((F: \cat A \to \cat B, G: \cat C \to \cat D)\) is mapped to a bifunctor
\[
F \times G: \cat A \times \cat C \to \cat B \times \cat D
\]
such that \((F \times G)(f, g) \coloneq (F f, G g)\) and
\((F \times G)(A, C) \coloneq (F A, G C)\), for any pair of morphisms
\((f, g) \in \Hom(\cat A) \times \Hom(\cat C)\) and pair of objects
\((A, C) \in \cat A \times \cat C\).

\begin{definition}[Functor-induced categories]
\label{def:functor-induced-cats}
Let \(F: \cat C \to \cat D\) be a functor between categories \(\cat C\) and
\(\cat D\), and let \(y \in \cat D\). We define the following two categories:
\begin{enumerate}[(a)]\setlength\itemsep{0em}
\item The category \(\cat C_y\) is defined to consist of objects
  \[
    \Obj(\cat C_y) \coloneq
    \{(x, s) \colon x \in \cat C \text{ and } s: F x \to y \text{ in } \cat D\},
  \]
  and morphisms between any objects \((a, s), (b, t) \in \cat C_{y}\) are
  defined to be
  \[
    \Hom_{\cat C_y}\left( (a, s), (b, t) \right) \coloneq
    \{f \in \Hom_{\cat C}(a, b) \colon s = t F(f) \text{ in } \cat D\},
  \]
  that is, a morphism \(f: (a, s) \to (b, t)\) makes to following diagram
  commute
  \[
    \begin{tikzcd}
      F a \ar[r, "s"] \ar[dr, swap, "Ff"] &y \\
      &F b \ar[u, swap, "t"]
    \end{tikzcd}
  \]
  Together with such category, we define a faithful functor \(j_y: \cat C_y \to
  \cat C\) by \(j_y(x, s) \coloneq x\), acting as a projection.
\item The category \(\cat C^y\) is defined to consist of objects
  \[
    \Obj(\cat C^y) \coloneq
    \{(x, s) \colon x \in \cat C \text{ and } s: y \to F x \text{ in } \cat D\},
  \]
  and morphisms between any objects \((a, s), (b, t) \in \cat C^{y}\) are
  defined to be
  \[
    \Hom_{\cat C^y}\left( (a, s), (b, t) \right) \coloneq
    \{f \in \Hom_{\cat C}(a, b) \colon t = F(f) s \text{ in } \cat D\},
  \]
  that is, a morphism \(f: (a, s) \to (b, t)\) makes to following diagram
  commute
  \[
    \begin{tikzcd}
      y \ar[r, "s"] \ar[d, swap, "t"] &Fa \ar[dl, "F f"] \\
      F b &
    \end{tikzcd}
  \]
  Together with such category, we define a faithful functor \(j^y: \cat C^y \to
  \cat C\) by \(j^y(x, s) \coloneq x\), which acts as a projection.
\end{enumerate}
\end{definition}

\begin{definition}[Equivalence classes]
\label{def:equivalence-class-category}
Let \(\cat C\) be a category and \(\sim\) denote an equivalence relation on the
objects of \(\cat C\) --- where \(x \sim y\) if \(\Hom_{\cat C}(x, y) \neq
\emptyset\). We denote the collection of all equivalence classes on the objects
of \(\cat C\) by \(\pi_0(\cat C)\).
\end{definition}

\begin{corollary}
\label{cor:connected-iff-class-point}
A category \(\cat C\) is \emph{connected} if and only if \(\pi_0(\cat C)\)
consists of a single element.
\end{corollary}

\begin{proof}
If \(\pi_0(\cat C)\) is a single object, every equivalence class on \(\cat C\)
is such that every element is equivalent to each other, which implies that the
collection of morphisms \(\Hom_{\cat C}(x, y)\), between any two elements \(x, y
\in \cat C\), is non-empty --- thus clearly \(\cat C\) is connected.
\end{proof}

\begin{definition}[Isomorphisms of monomorphisms \& epimorphisms]
\label{def:iso-mono-epi}
Let \(\cat C\) be a category and \(x, y, z \in \cat C\) be any objects. We
define the following concepts:
\begin{enumerate}[(a)]\setlength\itemsep{0em}
\item Two monomorphisms \(f: x \mono z\) and \(g: y \mono z\) in \(\cat C\) are
  said to be \emph{isomorphic} in \(\cat C\) if there exists an isomorphism \(h:
  x \isoto y\) for which the following diagram commutes in
  \[
    \begin{tikzcd}
      &z & \\
      x \ar[rr, "h", "\iso"'] \ar[ur, tail, "f"] &
      &y \ar[ul, tail, swap, "g"]
    \end{tikzcd}
  \]
  This is equivalent to \(f\) and \(g\) being isomorphic in \(\cat C_z\).
\item Two epimorphisms \(f: x \epi z\) and \(g: y \epi z\) in \(\cat C\) are
  said to be \emph{isomorphic} in \(\cat C\) if there exists an isomorphism \(h:
  x \isoto y\) for which the following diagram commutes in
  \[
    \begin{tikzcd}
      x \ar[rr, "h", "\iso"'] \ar[dr, two heads, swap, "f"] &
      &y \ar[dl, two heads, "g"] \\
      &z &
    \end{tikzcd}
  \]
  This is equivalent to \(f\) and \(g\) being isomorphic in \(\cat C^z\).
\end{enumerate}
\end{definition}

\begin{definition}
\label{def:subobject-quotient}
Let \(\cat C\) be a category and \(x \in \cat C\) be any object. We define the
following:
\begin{enumerate}[(a)]\setlength\itemsep{0em}
\item An isomorphism class of a monomorphism with \emph{target} \(x\) is called a
  \emph{subobject} of \(x\).
\item An isomorphism class of an epimorphism with \emph{source} \(x\) is called a
  \emph{quotient} of \(x\).
\end{enumerate}
These isomorphism classes are given in the sense of \cref{def:iso-mono-epi}.
\end{definition}

\begin{example}[Ordering subobject]
\label{exp:order-subobject}
One can \emph{order} the collection of subobjects of a given object \(x \in \cat
C\) by defining a relation \([f: y \mono x] \leq [g: z \mono x]\) if there
exists \(h: y \to z\) such that
\[
  \begin{tikzcd}
    y \ar[r, tail, "f"] \ar[rd, dashed, swap, "h"] &x \\
    &z \ar[u, tail, swap, "g"]
  \end{tikzcd}
\]
Moreover, \emph{if} \(h\) exists, then it's unique.
\end{example}


\section{Natural Transformations}

\begin{definition}[Natural transformation]
\label{def:natural-transformation}
Let \(\cat C\) and \(\cat D\) be categories, and consider functors \(F, G:
\cat C \rightrightarrows \cat D\). A \emph{natural transformation} \(\alpha: F
\nat G\) consists of morphisms \(\alpha_x: F x \to G x\), for all \(x \in \cat
C\), such that, for any morphism \(f: x \to y\) in \(\cat C\), the following
diagram commutes
\[
  \begin{tikzcd}
    F x \ar[r, "\alpha_x"] \ar[d, swap, "F f"]
    &G x \ar[d, "G f"] \\
    F y \ar[r, "\alpha_y"] &Gy
  \end{tikzcd}
\]
Moreover, if \(L: \cat C \to \cat D\) is another functor, and \(\beta: G \nat
L\) is a natural transformation, we define the \emph{composition} of natural
transformations \(\alpha\) and \(\beta\) as the map \(\beta \alpha: F \nat L\)
such that \((\beta \alpha)_x \coloneq \beta_x \alpha_x\) for every \(x \in \cat
C\).
\end{definition}

\begin{definition}[Functor category]
\label{def:functor-category}
Let \(\cat C\) and \(\cat D\) be categories. We define a category \(\Fct(\cat C,
\cat D)\) whose objects are functors \(\cat C \to \cat D\), and morphisms are
natural transformations between functors.
\end{definition}

\begin{proposition}[Horizontal composition]
\label{prop:horizontal-composition-natural-transformation}
Let \(\cat A, \cat B\) and \(\cat C\) be categories. Consider functors \(F, F':
\cat A \rightrightarrows \cat B\), and \(G, G': \cat B \rightrightarrows \cat
C\).  Let \(\alpha: F \nat F'\) and \(\beta: G \nat G'\) be natural
transformations --- that is,
\[
  \begin{tikzcd}
    \cat A \ar[r, bend left, "F"{name=F}]
    \ar[r, bend right, "{F'}"'{name=FF}]
    &\cat B \ar[r, bend left, "G"{name=G}]
    \ar[r, bend right, "{G'}"'{name=GG}]
    & \cat C
    \ar[Rightarrow, from=F, to=FF, "\alpha"]
    \ar[Rightarrow, from=G, to=GG, "\beta"]
  \end{tikzcd}
\]
There exists an induced natural transformation
\(\beta * \alpha: G F \nat G' F'\) --- called the \emph{horizontal composition},
also called \emph{Godement product}, of \(\alpha\) and \(\beta\) --- such that,
for all \(a \in \cat A\),
\[
(\beta * \alpha)_a = \beta_{F' a} G(\alpha_a) = G'(\alpha_a) \beta_{F a}.
\]
The horizontal composition can be depicted by the following diagram
\[
  \begin{tikzcd}
    \cat A \ar[rr, bend left=40, "G F"{name=s}]
    \ar[rr, bend right=40, swap, "{G' F'}"{name=t}]
    &&\cat C
    \ar[Rightarrow, from=s, to=t, "\beta * \alpha"]
  \end{tikzcd}
\]
\end{proposition}

\begin{proof}
Notice that the naturality of \(\beta * \alpha\) solely depends on the
naturality of both \(\alpha\) and \(\beta\). Indeed, for every \(a \in \cat A\)
and morphism \(f: a \to a'\) in \(\cat A\) the following diagram
\[
\begin{tikzcd}
G F a \ar[d, "G F f"'] \ar[r, "G \alpha_a"]
\ar[rr, bend left=50, "{(\beta * \alpha)_{a}}"]
&G F' a \ar[r, "\beta_{F' a}"] \ar[d, "G F' f"]
& G' F' a \ar[d, "G' F' f"]
\\
G F a' \ar[r, "G \alpha_{a'}"']
\ar[rr, bend right=50, "{(\beta * \alpha)_{a'}}"']
&G F' a' \ar[r, "\beta_{F' a'}"']
&G' F' a'
\end{tikzcd}
\]
is commutative, which proves that \(\beta * \alpha\) is a natural
transformation.
\end{proof}

\begin{notation}
\label{not:horizontal-composition}
For the ease of notation, we denote the vertical composition \(\alpha * \Id_F\)
by \(\alpha * F\).
\end{notation}

\begin{definition}[Vertical composition]
\label{def:vertical-compostion-natural-transformation}
Let \(\cat C\) and \(\cat D\) be categories, and consider functors and natural
transformations given in the following diagram
\[
  \begin{tikzcd}
  \cat C \ar[rr, bend left=60, "F"{name=F}]
  \ar[rr, "H"{near start, description}, ""'{name=H}, ""{name=HH}]
  \ar[rr, bend right=60, swap, "G"{name=G}]
  & &\cat D
  \ar[Rightarrow, "\alpha", from=F, to=H]
  \ar[Rightarrow, "\beta", from=HH, to=G]
  \end{tikzcd}
\]
We define the \emph{vertical composition} of \(\beta\) with \(\alpha\) as the
natural transformation \(\beta \circ \alpha: F \nat G\) --- diagrammatically,
\[
  \begin{tikzcd}
  \cat C \ar[rr, bend left=50, "F"{name=F}]
  \ar[rr, bend right=50, swap, "G"{name=G}]
  &&\cat D
  \ar[Rightarrow, "\beta \circ \alpha", from=F, to=G]
  \end{tikzcd}
\]
\end{definition}

\begin{proposition}
\label{prop:mixing-vertical-and-horizontal-compositions}
Consider the following diagram, with categories \(\cat A\), \(\cat B\) and
\(\cat C\), functors \(F\), \(G\), \(H\), \(K\), \(L\), \(M\), and natural
transformations \(\alpha\), \(\beta\), \(\gamma\) and \(\delta\):
\[
\begin{tikzcd}
\cat A
\ar[rr, bend left=60, "F"{name=F}]
\ar[rr, "H"{description, near start}, ""'{name=H}, ""{name=HH}]
\ar[rr, bend right=60, "L"'{name=L}]
&& \cat B
\ar[rr, bend left=60, "G"{name=G}]
\ar[rr, "K"{description, near start}, ""'{name=K}, ""{name=KK}]
\ar[rr, bend right=60, "M"'{name=M}]
&&\cat C
\ar[Rightarrow, "\alpha", from=F, to=H]
\ar[Rightarrow, from=HH, to=L, "\gamma"]
\ar[Rightarrow, "\beta", from=G, to=K]
\ar[Rightarrow, from=KK, to=M, "\delta"]
\end{tikzcd}
\]
This diagram is such that the following equality holds:
\[
(\delta * \gamma) \circ (\beta * \alpha)
= (\delta \circ \beta) * (\gamma \circ \alpha)
\]
Where by \(\circ\) we denote the \emph{vertical composition} and by \(*\) we
denote the \emph{horizontal composition}.
\end{proposition}

\begin{proof}
The composition \((\delta * \gamma) \circ (\beta * \alpha): GF \nat ML\) is
given by the following diagrams
\[
\begin{tikzcd}
\cat A \ar[rr, bend left=60, "GF"{name=GF}]
\ar[rr, "HK"{description, near start}, ""'{name=HK}]
\ar[rr, "ML"'{name=ML}, bend right=60]
& &\cat C
\ar[Rightarrow, "\beta * \alpha", from=GF, to=HK]
\ar[Rightarrow, "\delta* \gamma", from=HK, to=ML]
\end{tikzcd}
\longmapsto
\begin{tikzcd}
\cat A \ar[rr, bend left=60, "GF"{name=GF}]
\ar[rr, "ML"'{name=ML}, bend right=60]
& &\cat C
\ar[Rightarrow, "(\delta * \gamma) \circ (\beta * \alpha)" description, from=GF,
to=ML]
\end{tikzcd}
\]
On the other hand, the composition \((\delta \circ \beta) * (\gamma \circ
\alpha): GF \nat ML\) is given by the following diagrams
\[
\begin{tikzcd}
\cat A
\ar[rr, "F"{name=F}, bend left=60]
\ar[rr, "L"'{name=L}, bend right=60]
&& \cat B
\ar[rr, "G"{name=G}, bend left=60]
\ar[rr, "M"'{name=M}, bend right=60]
&& \cat C
\ar[Rightarrow, "\gamma \circ \alpha", from=F, to=L]
\ar[Rightarrow, "\delta \circ \beta", from=G, to=M]
\end{tikzcd}
\longmapsto
\begin{tikzcd}
\cat A \ar[rr, bend left=60, "GF"{name=GF}]
\ar[rr, "ML"'{name=ML}, bend right=60]
& &\cat C
\ar[Rightarrow, "(\delta \circ \beta) * (\gamma \circ \alpha)" description,
from=GF, to=ML]
\end{tikzcd}
\]
This is sufficient to prove the assertion.
\end{proof}

Notice that, given a functor \(\phi: \cat C \to \cat D\) between categories
\(\cat C\) and \(\cat D\), for every category \(\cat A\), there arises a natural
functor
\[
  \phi^{*}: \Fct(\cat A, \cat C) \longrightarrow \Fct(\cat I, \cat D)\text{,}
  \ \text{ mapping }\
  F \longmapsto \phi F.
\]

\begin{lemma}
\label{lem:faithful-pushforward-functor}
If \(\phi\) is a faithful functor (respectively, fully faithful), then so is the
functor \(\phi^{*}\) for any category \(\cat A\).
\end{lemma}

\begin{proof}
Given any two functors \(F, G: \cat A \rightrightarrows \cat C\), let \(\eta: F
\nat G\) be any natural transformation. If we apply \(\phi^{*}\), we get the
following commutative diagram --- for every pair \(x, y \in \cat A\) and every
morphism \(f: x \to y\) in \(\cat A\),
\[
  \begin{tikzcd}
    \phi F x \ar[r, "\phi \eta_x"] \ar[d, swap, "\phi F(f)"]
    &\phi G x \ar[d, "\phi G(f)"] \\
    \phi F y \ar[r, "\phi \eta_y"]
    &\phi G y
  \end{tikzcd}
\]
Since \(\phi\) is faithful (or fully faithful), the mappings \(\eta_x \mapsto
\phi \eta_x\) and \(\eta_y \mapsto \phi \eta_y\) are both injective (or
bijective), thus the natural map
\[
  \Hom_{\Fct(\cat A, \cat C)}(F, G) \longrightarrow
  \Hom_{\Fct(\cat A, \cat D)}(\phi F, \phi G)\text{,}
  \ \text{ mapping }\ \eta \mapsto \phi^{*} \eta,
\]
is ensured to be injective (or bijective).
\end{proof}

We consider now the category consisting of \(\mathcal U\)-small categories and
the morphisms are functors between them, we denote this category by
\(\UCat\). Notice that, given any two \(\mathcal U\)-small categories \(\cat C\)
and \(\cat D\), the collection of functors between them also forms a category
\(\Hom_{\UCat}(\cat C, \cat D) = \Fct(\cat C, \cat D)\) --- this emergent
structure gives birth to the concept of a \(2\)-category.

\begin{definition}[Isomorphism of categories]
\label{def:isomorphism-categories}
Let \(\cat C\) and \(\cat D\) be categories. We say that \(\cat C\) is
isomorphic to \(\cat D\) if there are morphisms \(F: \cat C \to \cat D\) and
\(G: \cat D \to \cat C\) such that \(GF = \Id_{\cat C}\), and \(FG = \Id_{\cat
D}\).
\end{definition}

A weaker and even more important concept it that of an equivalence between
categories.

\begin{definition}[Equivalence of categories]
\label{def:equivalence-categories}
Let \(\cat C\) and \(\cat D\) be categories. We say that a functor
\(F: \cat C \isoto \cat D\) is an \emph{equivalence} of the categories
\(\cat C\) and \(\cat D\) if there exists a functor \(G: \cat D \to \cat C\), and
two natural isomorphisms \(\alpha: G F \isonat \Id_{\cat C}\) and
\(\beta: F G \isonat \Id_{\cat D}\). If this is the case, we say that \(F\) and
\(G\) are \emph{quasi-inverses} of each other.
\end{definition}

\begin{lemma}
\label{lem:quasi-inverse}
Let \(F: \cat C \isoto \cat D\) and \(G: \cat D \isoto \cat C\) be equivalences
of given categories \(\cat C\) and \(\cat D\), and suppose that \(F\) and \(G\)
are quasi-inverses. Then there are natural isomorphisms \(\alpha: G F \isonat
\Id_{\cat C}\) and \(\beta: F G \isonat \Id_{\cat D}\) for which
\[
  F \alpha = \beta F\quad \text{ and } \quad \alpha G = G \beta.
\]
\end{lemma}

\begin{proof}
Let \(x \in \cat C\) be any object. Notice that, since \(\Id_{\cat C} x = x\)
and \(\Id_{\cat D} F x = F x\), it follows that
\[
  \begin{tikzcd}
    G F x \ar[r, "\alpha_x", "\dis"'] \ar[d, swap, "F"] &x \ar[d, "F"] \\
    FG(Fx) \ar[r, "\beta_{Fx}"', "\dis"] &Fx
  \end{tikzcd}
\]
is commutative --- thus indeed \(F \alpha = \beta F\). Now if we let \(y \in
\cat D\) be any other object, since \(\Id_{\cat D} y = y\) and \(\Id_{\cat C} G
y = G y\), we get the following commutative diagram
\[
  \begin{tikzcd}
    G F (G y) \ar[r, "\alpha_{G_y}", "\dis"'] &G y \\
    F G y \ar[u, "G"] \ar[r, "\beta_y"', "\dis"] &y \ar[u, swap, "G"]
  \end{tikzcd}
\]
therefore \(\alpha G = G \beta\) as wanted.
\end{proof}

\begin{lemma}
\label{lem:full-subcategory-functor-correspondence}
Let \(F: \cat C \to \cat D\) be a functor, and \(\cat D_0\) be a full
subcategory of \(\cat D\) such that, for all \(x \in \cat C\), there exists
\(y \in \cat D_0\) and an isomorphism \(Fx \iso y\).

Denote by \(\iota: \cat D_0 \emb \cat D\) the canonical embedding functor. Then
there exists a functor \(F_0: \cat C \to \cat D_0\) and a natural isomorphism
\(\alpha: F \isonat \iota F_0\).  Moreover, \(F_0\) is unique up to
unique isomorphism\footnote{We say that \(F_0\) is \emph{unique up to unique
isomorphism} when, given another functor \(G: \cat C \to \cat D_0\) and natural
isomorphism \(\beta: F \isonat \iota G\), there exists a \emph{unique} natural
isomorphism \(\eta: G \isonat F_0\) for which \(\alpha = \iota \eta
\beta\).}. This can be diagrammatically expressed by the
following quasi-commutative diagram\footnote{A diagram whose nodes are
categories an arrows are morphisms is said to be \emph{quasi-commutative} if it
commutes up to natural isomorphism of functors.}
\[
  \begin{tikzcd}
    \cat C \ar[r, "F"] \ar[dr, swap, dashed, bend right, "F_0"] &\cat D \\
    &\cat D_0 \ar[u, hook, swap, "\iota"]
  \end{tikzcd}
\]
\end{lemma}

\begin{proof}
We first build the functor \(F_0: \cat C \to \cat D_0\):
\begin{itemize}\setlength\itemsep{0em}
\item Let \(x \in \cat C\) be any object. By means of Zorn's Lemma (see
  \cref{lem:zorn}), we choose \(a \in \cat D_0\) for which exists an
  isomorphism \(\phi_x: a \isoto F(x)\) in \(\cat D\) --- and consequently we let
  \(F_0 x \coloneq a\).
\item Given any morphism \(f: x \to y\) in \(\cat C\), we know from
  construction that the objects \(F_0 x \coloneq a\) and \(F_0 y \coloneq b\)
  are defined so that there exists isomorphisms \(\phi_x: a \isoto F x\) and
  \(\phi_y: b \isoto F y\) in the category \(\cat D\). This allows us to define
  the morphism \(F_0 f: F_0 x \to F_0 y\) in \(\cat D\) as the mapping
  \(F f \coloneq \phi_y^{-1} (F f) \phi_x\) --- that is, so that the following
  diagram commutes
  \[
    \begin{tikzcd}
      F_0 x \ar[r, "F_0 f"] \ar[d, "\dis", "\phi_x"']
      &F_0 y \ar[d, "\dis"', "\phi_y"] \\
      F x \ar[r, swap, "F f"] &F y
    \end{tikzcd}
  \]
  Notice that from this definition we find naturally that the composition
  condition is met --- given any other morphism \(g: y \to z\) in \(\cat C\)
  have that \(F_0(g f) = (F_0 g)(F_0 f)\).
\end{itemize}
This proves the existence of \(F_0\) as a functor. For the isomorphism
\(\alpha\), we can define for each pair \(x, y \in \cat C\) the morphisms
\(\alpha_x \coloneq \phi_x^{-1}\) and \(\alpha_y \coloneq \phi_y^{-1}\), so that
the following diagram commutes
\[
  \begin{tikzcd}
    F x \ar[rr, "\alpha_x", "\dis"'] \ar[d, swap, "F f"]
    & &\iota F_0(x) = a \ar[d, "\iota F_0 (f)"] \\
    F y \ar[rr, "\alpha_y"', "\dis"]
    & &\iota F_0(y) = b
  \end{tikzcd}
\]
For the uniqueness of \(F_0\) up to unique isomorphism, let
\(G: \cat C \to \cat D_0\) be another functor, together with a natural
isomorphism \(\beta: F \isonat \iota G\). Define \(\eta: G \isonat F_0\) so
that, for each \(x \in \cat C\), we have
\(\eta_x \coloneq \alpha_x \beta_{x}^{-1}\) --- then, \(\eta\) is clearly an
isomorphism and also uniquely defined, thus the proposition follows.
\end{proof}

\begin{lemma}
\label{lem:full-subcategory-embedding-equivalence}
Let \(\cat C\) be any category. There exists a \emph{full subcategory}
\(\cat C_0\) of \(\cat C\) such that the embedding functor
\(\iota: \cat C_0 \emb \cat C\) is a \emph{category equivalence} and any two
isomorphic elements in \(\cat C_0\) are equal to each other.
\end{lemma}

\begin{proof}
Let \(\sim\) be the equivalence relation on the set \(\Obj(\cat C)\) where
\(x \sim y\) if and only if we have \(x \iso y\) in \(\cat C\). By means of
Zorn's lemma, choose for each equivalence class a representative --- and
define \(\cat C_0\) as the full subcategory of \(\cat C\) consisting of such
representatives. In this way, if \(x, y \in \Obj(\cat C_0)\) satisfy
\(x \iso y\) in \(\cat C\), then necessarily \(x = y\).

If we apply \cref{lem:full-subcategory-functor-correspondence} to the identity
functor \(\Id_{\cat C}\), we find the existence of a unique functor
\(F_0: \cat C \to \cat C_0\) and a natural isomorphism \(\alpha: \Id_{\cat C}
\isonat \iota F_0\) for which \(\iota F_0 \iso \Id_{\cat C}\). Moreover,
we have the chain of isomorphisms:
\[
\iota (F_0 \iota) = (\iota F_0) \iota \iso \Id_{\cat{C}} \iota \iso \iota \iso
\iota \Id_{\cat C_0}.
\]
From the fact that \(\iota\) is fully faithful, we obtain \(F_0 \iota \iso
\Id_{\cat C_0}\).
\end{proof}

\begin{proposition}
\label{prop:equiv-cats-iff-fully-faith-and-essen-surj}
A functor \(F: \cat C \to \cat D\) is an \emph{equivalence} of categories if and
only if \(F\) is \emph{fully faithful} and \emph{essentially surjective}.
\end{proposition}

\begin{proof}
Suppose \(F\) is an equivalence of categories, then exists a quasi inverse
\(G: \cat D \to \cat C\) and natural isomorphisms
\(\alpha: G F \isonat \Id_{\cat C}\) and \(\beta: F G \isonat \Id_{\cat D}\). We
now prove that \(F\) is both fully faithful and essentially surjective.

\begin{itemize}\setlength\itemsep{0em}
\item (Fully Faithful) Let \(g: Fx \to Fy\) be any morphism in \(\cat D\), thus
  if there exists an \(f: x \to y\) in \(\cat C\) for which \(F f = g\), it must
  be the case that
  \(f = \alpha_y (G F f) \alpha_x^{-1} = \alpha_y (G g) \alpha_x^{-1}\). Indeed,
  if \(f = \alpha_y (G g) \alpha_x^{-1}: x \to y\) is a morphism in \(\cat C\),
  from the distributivity over composition, we have
  \[
  G F f = G F \alpha_y \circ G F (G g) \circ G F \alpha_x^{-1}
  \]
  then by the naturality of \(\alpha\) on the map \(\alpha_x^{-1}\) implies
  \[
  \begin{tikzcd}
  G F x \ar[d, "G F \alpha_x^{-1}"] \ar[r, "\alpha_x"']
  & x \ar[d, "\alpha_x^{-1}"] \\
  G F (G F x) \ar[r, "\alpha_{G F x}"'] & G F x
  \end{tikzcd}
  \]
  but since \(\alpha_x^{-1}\) is an isomorphism, then
  \(G F \alpha_x^{-1} \alpha_x^{-1} = \alpha_{G F x}^{-1} \alpha_x^{-1}\)
  implies in
  \[
  G F \alpha_x^{-1} = \alpha_{G F x}^{-1}.
  \]
  On the other hand, the same can be done for \(G F \alpha_y\), yielding
  \(G F \alpha_y = \alpha_{G F y}\), therefore
  \begin{equation}\label{eq:GFf=alphas}
  G F f = \alpha_{G F y} (G F G g) \alpha_{G F x}^{-1}.
  \end{equation}
  The naturality of \(\alpha\) on the map \(G g\) gives
  \[
  \begin{tikzcd}
  G F (G F x) \ar[r, "\alpha_{G F x}"] \ar[d, "G F (G g)"']
  & G F x \ar[d, "G g"] \\
  G F (G F y) \ar[r, "\alpha_{G F y}"'] & G F y
  \end{tikzcd}
  \]
  that is, \(G F (G g) \alpha_{G F x}^{-1} = \alpha_{G F y}^{-1} G g\), thus
  substituting in \cref{eq:GFf=alphas} we obtain finally that
  \[
  G F f = G g.
  \]
  From this we conclude that \(F\) is fully faithful.

\item (Essentially surjective) Let \(a \in \cat D\) be any element and simply
  consider \(x = G a \in \cat D\) then from \(\beta\) we obtain that
  \(Fx = F G a \iso a\).
\end{itemize}

For the converse, suppose \(F\) is both fully faithful and essentially
surjective. Using \cref{lem:full-subcategory-embedding-equivalence}, let
\(\cat C_0\) be the full subcategory of \(\cat C\) such that
\(\iota_{\cat C}: \cat C_0 \emb \cat C\) is an equivalence and if \(x \iso y\)
in \(\cat C\) then \(x = y\) in \(\cat C_0\). Let \(\kappa_{\cat C}\) be a
quasi-inverse of \(\iota_{\cat C}\). Apply
\cref{lem:full-subcategory-embedding-equivalence} again, but now to the category
\(\cat D\), yielding a category \(\cat D_0\), an embedding \(\iota_{\cat D}\)
and a quasi-inverse \(\kappa_{\cat D}\). Notice that the composition of functors
\(\kappa_{\cat D} F \iota_{\cat C}: \cat C_0 \isoto \cat D_0\) is an
isomorphism. Let \(H\) be the inverse of \(\kappa_{\cat D} F \iota_{\cat C}\),
and define \(G \coloneq \iota_{\cat C} H \kappa_{\cat D}\), then \(G\) is a
quasi-inverse of \(F\).
\end{proof}

\begin{corollary}
\label{cor:fully-faithful-induces-equivalence-to-full-subcategory}
Let \(F: \cat C \to \cat D\) be a \emph{fully faithful} functor. Then there
exists a \emph{full subcategory} \(\cat D_0\) of \(\cat D\) and an
\emph{equivalence} of categories \(G: \cat C \isoto \cat D_0\) for which \(F\)
is \emph{isomorphic} to \(\iota_{\cat D} G\) --- where
\(\iota_{\cat D}: \cat D_0 \emb \cat D\) is the embedding functor.
\end{corollary}

\begin{proof}
Define \(\cat D_0\) to be the full category with
\(\Obj(\cat D_0) \coloneq \{F x \colon x \in \cat C\}\). Now define
\(G: \cat D \to \cat D_0\) to be the functor \(G = F\), we find that clearly
\(F \iso \iota_{\cat D} G\).
\end{proof}

\begin{example}
\label{exp:equivalence-category-functor-opposite}
For any two given categories \(\cat C\) and \(\cat D\), there is an isomorphism
of categories \(\Fct(\cat C, \cat D)^{\op} \iso \Fct(\cat C^{\op}, \cat
D^{\op})\). Explicitly, such isomorphism maps \(F \mapsto \op F \op\).
\end{example}

\begin{definition}[Essentially \(\mathcal{U}\)-small]
\label{def:essentially-U-small}
A category \(\cat C\) is said to be \emph{essentially \(\mathcal{U}\)-small} if
it is equivalent to a \(\mathcal{U}\)-small category. Equivalently, \(\cat C\)
is essentially \(\mathcal{U}\)-small if and only if \(\cat C\) is a
\(\mathcal{U}\)-category and there exists a subset \(S \subseteq \Obj(\cat C)\)
that is \(\mathcal{U}\)-small for which, given any \(x \in \cat C\), there
exists \(y \in S\) such that \(x \iso y\).
\end{definition}

\begin{definition}[Half-full]
\label{def:half-full}
We define the concepts of \emph{half-fullness} for functors and subcategories:
\begin{enumerate}[(a)]\setlength\itemsep{0em}
\item A functor \(F: \cat C \to \cat D\) is said to be \emph{half-full} if for
  any two \(x, y \in \cat C\) for which \(F x \iso F y\) in \(\cat D\), there
  exists \emph{an} isomorphism \(x \iso y\) in \(\cat C\).

\item A subcategory \(\cat C_0\) of \(\cat C\) is said to be \emph{half-full} if
  the embedding functor \(\iota: \cat C_0 \emb \cat C\) is half-full.
\end{enumerate}
\end{definition}

An analogous but \emph{less strict} proposition when compared to
\cref{cor:fully-faithful-induces-equivalence-to-full-subcategory} is done by
substituting the condition of fully faithfulness to only that of faithful and
half-full. It goes as follows.

\begin{proposition}
\label{prop:faithful-half-full-induces-equivalence-to-subcategory}
Let \(F: \cat C \to \cat D\) be a \emph{faithful} and \emph{half-full}
functor. Then there exists a subcategory \(\cat D_0\) of \(\cat D\) for which
\[
F(\Obj(\cat C)) \subseteq \Obj(\cat D_0),
\text{ and }
F(\Hom(\cat C)) \subseteq \Hom(\cat D_0).
\]
Moreover, \(F\) induces an \emph{equivalence} of categories
\(\cat C \iso \cat D_0\) and the embedding
\(\cat D_0 \emb \cat D\) is \emph{faithful} and
\emph{half-full}.
\end{proposition}

\begin{proof}
Let \(\cat D_0\) be the category with
\(\Obj(\cat D_0) \coloneq \{F(x) \colon x \in \cat C\}\), and for each two
\(x, y \in \cat C\) define
\(\Hom_{\cat D_0}(F x, F y) \coloneq F(\Hom_{\cat C}(x, y))\) --- so that
\(\Hom_{\cat D_0}(F x, F y) \subseteq \Hom_{\cat D}(F x, F y)\). Since \(F\) is
half-full, the definition of \(\Hom_{\cat D_0}(F x, F y)\) is independent of the
initial choice of \(x, y \in \cat C\) --- indeed, if \(x', y' \in \cat C\) are
such that \(F x' \iso F x\) and \(F y' \iso F y\) in \(\cat D_0\), then \(x'
\iso x\) and \(y' \iso y\) in \(\cat C\), thus \(\Hom_{\cat C_0}(x, y) \iso
\Hom_{\cat C_0}(x', y')\).

Restricting the codomain of \(F\) to \(\cat D_0\), we find that the induced
functor \(F: \cat C \to \cat D_0\) is fully faithful and essentially surjective
--- thus by \cref{prop:equiv-cats-iff-fully-faith-and-essen-surj} \(F\) is an
equivalence of categories.
\end{proof}

\section{Comma Categories}

\begin{definition}[Comma category]
\label{def:comma-category}
Let \(F: \cat A \to \cat C\) and \(G: \cat B \to \cat C\) be two functors. The
comma category \(F \comma G\) induced by the functors \(F\) and \(G\) is defined
as follows:
\begin{itemize}\setlength\itemsep{0em}
\item The objects of \(F \comma G\) are triples \((a, f, b)\) --- where
  \(a \in \cat A\), \(b \in \cat B\), and \(f: F a \to G b\) is a morphism of
  \(\cat C\).

\item A morphism \(\phi: (a, f, b) \to (a', f', b')\) in \(F \comma G\) is a
  pair of \((\alpha, \beta)\) where \(\alpha: a \to a'\) is a morphism of
  \(\cat A\), while \(\beta: b \to b'\) is a morphism of \(\cat B\) ---
  moreover, such morphisms are such that the following diagram commutes
  \[
  \begin{tikzcd}
  F a \ar[r, "f"] \ar[d, "F \alpha"'] & G b \ar[d, "G \beta"] \\
  F a' \ar[r, "f'"'] & G b'
  \end{tikzcd}
  \]

\item The composition of \emph{compatible} morphisms \((\alpha, \beta)\) and
  \((\alpha', \beta')\) in \(F \comma G\) is induced by the composition law of
  \(\cat A\) and \(\cat B\) as follows:
  \[
  (\alpha', \beta') \circ (\alpha, \beta)
  \coloneq (\alpha' \alpha, \beta' \beta).
  \]
\end{itemize}
\end{definition}

\begin{proposition}[Projection functors in \(F \comma G\)]
\label{prop:comma-cat-proj-functors}
Let \(F: \cat A \to \cat C\) and \(G: \cat B \to \cat C\) be functors. There are
two functors \(A: F \comma G \to \cat A\) and \(B: F \comma G \to \cat B\), and
a canonical natural transformation \(\eta: F A \nat G B\). The scenario can be
depicted in the following diagram
\[
\begin{tikzcd}
F \comma G
\ar[rrr, "B"]
\ar[rrrdddd, bend left=18, "G B", ""'{name=GB}]
\ar[rrrdddd, bend right=18, "F A"', ""{name=FA}]
\ar[dddd, "A"]
& & & \cat B \ar[dddd, "G"] \\ \\ \\ \\
\cat A \ar[rrr, "F"] & & &\cat C
\ar[Rightarrow, "\eta", from=FA, to=GB]
\end{tikzcd}
\]
It is to be noted that such diagram \emph{does not commute in general} --- that
is, we may have comma categories where \(FA \neq G B\).
\end{proposition}

\begin{proof}
Define the functors \(A\) and \(B\) as follows --- for every object
\((a, f, b) \in F \comma G\) and morphism
\((\alpha, \beta) \in \Hom(F \comma G)\):
\begin{itemize}\setlength\itemsep{0em}
\item Map \(A(a, f, b) \coloneq a\), and \(A(\alpha, \beta) \coloneq \alpha\).
\item Map \(B(a, f, b) \coloneq b\), and \(A(\alpha, \beta) \coloneq \beta\).
\end{itemize}
For the natural transformation, simply define
\(\eta_{(a, f, b)} \coloneq f: F a \to F b\) which must be natural from the
construction of comma categories.
\end{proof}

\begin{proposition}[Comma category property]
\label{prop:comma-cat-univ-property}
Let \(F: \cat A \to \cat C\) and \(G: \cat B \to \cat C\) be two functors. If
there exists a category \(\cat D\) with two functors \(A': \cat D \to \cat A\)
and \(B': \cat D \to \cat B\), and a natural transformation
\(\eta': F A' \nat G B'\) --- then there \emph{exists a unique} functor
\(W: \cat D \to F \comma G\) such that
\[
\alpha * W = \alpha',
\]
and that the following diagram commutes
\[
\begin{tikzcd}
&\cat D \ar[drd, bend left, "B'"] \ar[dld, bend right, "A'"']
\ar[d, dashed, "W"]
& \\
&F \comma G \ar[dr, "B"'] \ar[dl, "A"]& \\
\cat A & &\cat B
\end{tikzcd}
\]
\end{proposition}

\begin{proof}
For every \(d \in D\), define \(W d \coloneq (A' d, \eta'_d, B' d)\), and for
each morphism \(f \in \Hom(\cat D)\), we define \(W f \coloneq (A' f, B'
f)\). This completely determines \(W\) and thus shows that, if it exists, it is
unique. To show that \(W: \cat D \to F \comma G\) is indeed a functor, we note
that:
\begin{itemize}\setlength\itemsep{0em}
\item Given morphisms \(f: x \to y\) and \(g: x \to z\) of \(\cat D\), we have
  \begin{align*}
  W g \circ W f
  &= (A' g, B' g) \circ (A' f, B' f) \\
  &= (A' g \circ A' f, B' g \circ B' f) \\
  &= (A' (g f), B' (g f)) \\
  &= W (g f).
  \end{align*}
  Note that although we used the same symbol \(\circ\) for composition, one
  should note that they have different laws.
\item Moreover, for any \(d \in D\) we have
  \[
  W \Id_d
  = (A' \Id_d, B' \Id_d)
  = (\Id_{A' d}, \Id_{B' d})
  = \Id_{W d}.
  \]
\end{itemize}
\end{proof}

\begin{definition}[Category of elements]
\label{def:category-of-elements}
Let \(F: \cat C \to \Set\) be a covariant functor. The \emph{category of
  elements} of \(\cat C\) associated to \(F\) is denoted by \(\El_F(\cat C)\)
whose objects are pairs \((c, s)\) for \(c \in \cat C\) and \(s \in F c\), and
morphisms between any two objects \((c, s), (c', s') \in \El_F(\cat C)\) is
defined as
\[
\Hom_{\El_F(\cat C)}((c, s), (c', s')) \coloneq
\{u \in \Hom_{\cat C}(c, c') \colon F(u)(s) = s'\}.
\]

For a contravariant \(G: \cat C^{\op} \to \Set\), the category of elements of
\(\cat C\) associated to \(G\) is composed of objects \((c, s)\) for
\(c \in \cat C\) and \(s \in G c\), and the collection of morphisms between
objects \((c, s), (c', s') \in \El_G(\cat C)\) is given by
\[
\Hom_{\El_G(\cat C)}((c, s), (c', s')) \coloneq
\{u \in \Hom_{\cat C}(c', c) \colon F(u)(s') = s\}.
\]
\end{definition}

\begin{corollary}
\label{cor:cat-elements-is-comma-cat}
Let \(F: \cat C \to \cat D\) be a functor. The category of elements \(\El_F(\cat
C)\) is exactly the comma category \(1 \comma F\) --- where \(1: \mathbf{1} \to
\Set\) is the functor from the discrete category \(\mathbf{1}\) with a single
object \(\star\) to the category of sets, mapping \(\star \mapsto \{*\}\).
\end{corollary}

\section{Yoneda Lemma}

\begin{remark}
\label{rem:Set-is-U-Set}
I again stress that \(\Set\), for us, is defined to be the category whose
objects are \(\mathcal{U}\)-sets for a given universe \(\mathcal{U}\) and
set-functions between these sets --- this is a relevant remark, since confusions
with that would lead one to undesirable size issues.
\end{remark}

\begin{definition}[Category of presheaves \& Yoneda functors]
\label{def:category-of-presheaves-and-yoneda-functors}
Given a \(\mathcal{U}\)-category \(\cat C\), we define the \emph{big} category
of \emph{presheaves} \(\Psh{\cat C}\) and a \emph{big} category of functors
\(\Psh{\cat C^{\op}} = [C, \Set]\). Together with such categories we define
functors
\begin{align*}
  \yo_{\cat C}:&\, \cat C \longrightarrow \Psh{\cat C},
                 \text{ mapping } x \mapsto \Hom_{\cat C}(-, x)
                 \text{ and } f \mapsto f_{*}, \\
  \yo'_{\cat C}:&\, \cat C \longrightarrow \Psh{\cat C^{\op}},
                  \text{ mapping } x \mapsto \Hom_{\cat C}(x, -)
                  \text{ and } f \mapsto f^{*}.
\end{align*}
The functors \(\yo_{\cat C}\) and \(\yo'_{\cat C}\) are called \emph{Yoneda
  functors}.
\end{definition}

\begin{remark}[\(f_{*}\) and \(f^{*}\) as natural transformations]
\label{rem:yoneda-f*-notation}
The attentive reader may note that setting \(\yo_{\cat C} f = f_{*}\) is not
quite right since \(f_{*}\) is not a morphism in the category of presheaves
\(\Psh{\cat C}\) according to our definition that roots from
\cref{lem:f-iso-iff-f*-iso} --- you are right, but we are being sloppy here just
to simplify how we treat our objects and notations. The arrow \(f_{*}\) (and
\(f^{*}\)) is to be interpreted as the \emph{natural transformation}
\(f_{*}: \yo_{\cat C} x \nat \yo_{\cat C} y\), where \(f: x \to y\) --- for which
we define, for every object \(z \in \cat C\), the morphism
\(f_{*}: \Hom_{\cat C}(z, x) \to \Hom_{\cat C}(z, y)\) mapping
\(g \mapsto f g\).
\end{remark}

\begin{remark}[Size issues]
\label{rem:presheaf-category-size-issues}
Although \(\cat C\) is a \(\mathcal{U}\)-category, its categories of presheaves
described above \emph{need not} be a \(\mathcal{U}\)-category --- however, if
\(\cat C\) happens to be \emph{\(\mathcal{U}\)-small}, then
\(\Psh{\cat C}\) and \(\Psh{\cat C^{\op}}\) are both
\emph{\(\mathcal{U}\)-categories}.
\end{remark}

%%%% This is a really confusing passage on Categories and Sheaves, I'll prefer
%%%% to omit it
% Notice that by \cref{exp:equivalence-category-functor-opposite} we have
% natural isomorphisms
% \begin{equation}\label{eq:nat-iso-presheaf-categories}
% [\cat C^{\op}, \Set^{\op}] \iso [\cat C, \Set]^{\op},
% \end{equation}
% that is, \([\cat C^{\op}, \Set^{\op}]\) is isomorphic to the category of
% presheaves of the opposite category \(\cat C^{\op}\), namely,
% \([\cat C, \Set]\).  Therefore, for every \(x \in \cat C\) we have
% \[
% \yo'_{\cat C}x = (\yo_{\cat C^{\op}}x^{\op})^{\op}.
% \]

\begin{lemma}[Yoneda]
\label{lem:yoneda}
The Yoneda lemma states that:
\begin{enumerate}[(a)]\setlength\itemsep{0em}
\item For all functors \(F: \cat C^{\op} \to \Set\) and \(x \in \cat C\), there
  is a natural isomorphism:
  \[
  \Hom_{\Psh{\cat C}}(\yo_{\cat C}x, F) \iso F x.
  \]

\item For all functors \(G: \cat C \to \Set\) and
  \(x \in \cat C\), there is a natural isomorphism:
  \[
  \Hom_{\Psh{\cat C^{\op}}}(G, \yo'_{\cat C}x) \iso G x.
  \]
\end{enumerate}
These natural isomorphisms have a \emph{functorial} nature --- they define,
respectively, functors \(\cat C^{\op} \times \Psh{\cat C} \to \Set\) and
\(\Psh{\cat C^{\op}}^{\op} \times \cat C \to \Set\)
\end{lemma}

\begin{proof}
By means of \cref{eq:nat-iso-presheaf-categories} one only needs to prove one of
the two statements, since the other follows from duality. We prove the item (a).

Let \(\phi\) be the morphism making
\[
\begin{tikzcd}
\Hom_{\Psh{\cat C}}(\yo_{\cat C}x, F) \ar[r]
\ar[rr, bend left, "\phi" description]
&\Hom_{\Set}(\Hom_{\cat C}(x, x), F x) \ar[r]
& F x \\
\eta \ar[r, mapsto] &\eta_x \ar[r, mapsto] & \eta_x(\Id_x)
\end{tikzcd}
\]
commute. Notice that the choice of
\(\Id_x \in \Hom_{\cat C}(x, x) = \yo_{\cat C}x (x)\) is done so that we can
define a distinguished point \(\eta_x(\Id_x)\) from the set \(F_x\).

Define now \(\psi: F x \to \Hom_{\Psh{\cat C}}(\yo_{\cat C}(x), F)\) to
be the map \(F x \ni a \mapsto \eta^a: \yo_{\cat C} x \nat F\), where the
natural transformation \(\eta^a\) associated to \(a\) is defined, for every
object \(y \in \cat C\), by the map \(\eta_y^a: \Hom_{\cat C}(y, x) \to F x\)
sending \(f \mapsto F (f) (a)\).

Notice that any natural transformation
\(\eta \in \Hom_{\Psh{\cat C}}(\yo_{\cat C} x, F)\) is completely
determined by \(\eta_x(\Id_x)\). Indeed, given \(f \in \Hom_{\cat C}(y, x)\) one
has the following diagram, which comes from the naturality of \(\eta\):
\[
\begin{tikzcd}
\Hom_{\cat C}(x, x) \ar[r, "\eta_x"] \ar[d, "f^{*}"']
&F x \ar[d, "F f"] \\
\Hom_{\cat C}(y, x) \ar[r, "\eta_y"] &F y
\end{tikzcd}
\]
Thus we have \(\eta_y(f) = (F f)(\eta_x(\Id_x))\) since
\(f^{*}(\Id_{x}) = \Id_x f = f\) --- which shows that \(\eta\) is determined by
the distinguished map \(\Id_x\). That is, \(\psi\) is determined, for each
\(a \in F x\) by \(\eta_x^a(\Id_x)\). This is thus sufficient to prove that
\(\phi\) and \(\psi\) are mutual inverses and therefore stablish the wanted
isomorphism.
\end{proof}

\begin{corollary}
\label{cor:yoneda-fct-fully-faithful}
The Yoneda functors \(\yo\) and \(\yo'\) are \emph{fully faithful}.
\end{corollary}

\begin{proof}
From duality, we only check that the map
\(\Hom_{\cat C}(x, y) \to \Mor_{\Psh{\cat C}}(\yo_{\cat C}x, \yo_{\cat
  C} y)\) sending \(f \mapsto \eta^f\), where
\(\eta_z^f(g) = f_{*}(g) = f g\) for every \(z \in \cat C\) and map
\(g \in \Hom_{\cat C}(z, x)\) --- is a bijection for all \(x, y \in \cat
C\). Such map is certainly injective, since given two distinct maps
\(f, h \in \Hom_{\cat C}(x, y)\) we have \(f_{*} \neq h_{*}\).  Moreover, we
have the isomorphism
\[
\Hom_{\Psh{\cat C}}(\yo_{\cat C} x, \yo_{\cat C} y)
\iso \yo_{\cat C} y (x)
= \Hom_{\cat C} (x, y)
\]
from \cref{lem:yoneda}. Let \(\eta: \yo_{\cat C} x \nat \yo_{\cat C} y\) be any
natural transformation --- we wish to find \(f \in \Hom_{\cat C}(x, y)\) for
which \(\eta_z(g) = f g\) for every \(z \in \cat C\) and
\(g \in \Hom_{\cat C}(z, x)\). The natural choice is given by
\(f \coloneq \eta_x(\Id_x)\). By the naturality of \(\eta\) we have, for any
\(w \in \cat C\)
\[
\begin{tikzcd}
\Hom_{\cat C}(w, x) \ar[r, "\eta_w"] \ar[d, "g^{*}"']
&\Hom_{\cat C}(w, y) \ar[d, "g^{*}"] \\
\Hom_{\cat C}(z, x) \ar[r, "\eta_z"] &\Hom_{\cat C}(z, y)
\end{tikzcd}
\]
Thus given \(h: w \to x\) in \(\cat C\) we have the equality
\[
\eta_w(h) g = \eta_z(h g).
\]
Therefore, if we restrict \(w = x\) and \(h = \Id_x\) we get
\(\eta_x(\Id_x)g = f g = \eta_z(g)\) --- which is what we wanted, because this
means that \(f\) will have image \(\eta\) under our mapping.
\end{proof}

\begin{remark}[Full subcategory of the presheaf category]
\label{rem:C-is-full-subcategory-of-presheaf-cat}
From \cref{cor:yoneda-fct-fully-faithful} one can conclude that \(\cat C\) can
be viewed as a \emph{full subcategory} of the presheaf category
\(\Psh{\cat C}\) (or of the category \(\Psh{\cat C^{\op}}\)).
\end{remark}

The following corollary establishes the main idea of the Yoneda lemma --- one
knows about an object by knowing how it interacts with other objects.

\begin{corollary}
\label{cor:yoneda-obj-iso-iff-fct-iso}
Given a category \(\cat C\) and two objects \(x, y \in \cat C\). Then there
exists an isomorphism \(x \iso y\) if and only if there exists an isomorphism
\(\Hom_{\cat C}(-, x) \iso \Hom_{\cat C}(-, y)\) --- or, for the covariant
version, \(\Hom_{\cat C}(x, -) \iso \Hom_{\cat C}(y, -)\).
\end{corollary}

\begin{proof}
Since functors preserve isomorphisms, if \(x \iso y\) for any two objects
\(x, y \in \cat C\), we obtain that \(\yo_{\cat C} x \iso \yo_{\cat C} y\). From
\cref{prop:fully-faithful-image-iso-then-obj-iso} and since \(\yo_{\cat C}\) is
fully faithful we find that \(\yo_{\cat C} x \iso \yo_{\cat C} y\) implies
\(x \iso y\).
\end{proof}

\begin{corollary}
\label{cor:yoneda-corollary-something}
Let \(F: \cat C \to \cat D\) be a functor of \(\mathcal{U}\)-categories and
assume that \(\cat C\) is \(\mathcal{U}\)-small. For every \emph{functor}
\(A \in \Psh{\cat C}\), the category \(\cat C_A\)
\emph{associated}\footnote{Recall \cref{def:functor-induced-cats}.} with the
functor
\[
\begin{tikzcd}
\cat C \ar[r, "F"] &\cat D \ar[r, "\yo_{\cat D}"] &{\Psh{\cat C}}
\end{tikzcd}
\]
is \emph{\(\mathcal{U}\)-small}. Analogously, given a \emph{functor} \(B \in
\Psh{\cat C^{\op}}\), the category \(\cat C^B\) \emph{associated} with
the functor
\[
\begin{tikzcd}
\cat C \ar[r, "F"] &\cat D \ar[r, "\yo_{\cat D}'"] &{\Psh{\cat C^{\op}}}
\end{tikzcd}
\]
is \emph{\(\mathcal{U}\)-small}.
\end{corollary}

\begin{proof}
Recalling the definition, the category \(\cat C_A\) associated with \(\yo_{\cat
  C} F\) is given by the following collections:
\begin{gather*}
\Obj(\cat C_A) = \{
(x, \phi) \colon x \in \cat C \text{ and }
\phi: \yo_{\cat C} F x \to A \text{ in } \cat D
\}, \\
\Hom_{\cat C_A}((x, \phi), (y, \psi)) = \{
f \in \Hom_{\cat C}(x, y) \colon
\phi = \psi \yo_{\cat C} F(f)
\},
\end{gather*}
where \((x, \phi), (y, \psi) \in \cat C_A\) are any two objects, in other words,
the collection of morphisms so that
\[
\begin{tikzcd}
\yo_{\cat C} F x \ar[r, "\phi"] \ar[rd, bend right, "\yo_{\cat C} F(f)"'] & A \\
&\yo_{\cat C} F y \ar[u, "\psi"']
\end{tikzcd}
\]
commutes in \(\cat D\). Since \(\cat C\) is assumed to be \(\mathcal{U}\)-small
by hypothesis, it follows that the collection of arrows
\[
\bigdisj_{x \in \cat C} \Hom_{\cat C}(F x, A)
\]
is a disjoint union of \(\mathcal{U}\)-small sets, thus is itself
\(\mathcal{U}\)-small. The same proof can be analogously constructed for the
covariant case \(\cat C^B\).
\end{proof}

\begin{corollary}
\label{cor:f-iso-if-f*-iso}
Let \(\cat C\) be a category and \(f: x \to y\) a morphism in \(\cat C\). If for
every \(z \in \cat C\) the morphism
\begin{align*}
  f_{*}:& \Hom_{\cat C}(z, x) \to \Hom_{\cat C}(z, y) \\
\end{align*}
is an \emph{isomorphism} (or \(f^{*}: \Hom_{\cat C}(x, z) \to \Hom_{\cat C}(y,
z)\) for the covariant case), then \(f\) is an \emph{isomorphism}.
\end{corollary}

\begin{proof}
If the condition is met, then \(\yo_{\cat C} f: \yo_{\cat C} x \to \yo_{\cat C}
y\) is an isomorphism. Since \(\yo_{\cat C}\) is fully faithful, then \(f\) is
an isomorphism. Notice that we already stated this proposition back when we
were studying dual categories, with the Yoneda lemma we were able to prove it
functorially --- see \cref{lem:f-iso-iff-f*-iso}.
\end{proof}

\subsection{Functor Representation}

\begin{definition}[Representable functor]
\label{def:representable-functor}
A presheaf \(F: \cat C^{\op} \to \Set\) (or a functor \(\cat C \to \Set\)) is
said to be \emph{representable} if there exists a natural isomorphism
\(\yo_{\cat C} x \isonat F\) (or \(F \isonat \yo_{\cat C}' x\)) for some
\(x \in \cat C\) --- the object \(x\) is called a \emph{representative} of
\(F\).
\end{definition}

\begin{corollary}[Representative uniqueness]
\label{cor:representative-is-unique-up-to-unique-iso}
The representative of a representable functor is \emph{unique up to
unique isomorphism}.
\end{corollary}

\begin{proof}
Since \(\yo_{\cat C}\) (and \(\yo_{\cat C}'\)) is fully faithful, it follows
that, if \(x, y \in \cat C\) are representatives of \(F\), then there exists a
natural isomorphism \(\yo_{\cat C} x \iso \yo_{\cat C} y\) hence, evoking
\cref{prop:fully-faithful-image-iso-then-obj-iso}, we find a unique
isomorphism \(x \iso y\).
\end{proof}

\begin{corollary}[Universal element]
\label{cor:functor-universal-element}
Let \(F: \cat C^{\op} \to \Set\) (or \(\cat C \to \Set\)) be any representable
presheaf, with representative object \(x_0 \in \cat C\). Then the natural
isomorphism
\[
\eta: \yo_{\cat C} x_0 \isonat F
\]
is uniquely determined by an element \(s_0 \in F x_0\) --- such an element is
called a \emph{universal element} of \(F\).
\end{corollary}

\begin{proof}
Indeed, if \(F\) is represented by \(x_0\), then by the Yoneda lemma we obtain
an isomorphism
\[
\Hom_{\Psh{\cat C}}(\yo_{\cat C} x_0, F) \iso F x_0,
\]
therefore for each \(s_0 \in F x_0\) there exists a unique corresponding natural
transformation \(\eta: \yo_{\cat C} x_0 \nat F\) (notice we didn't require it to
be an isomorphism) and \(\eta\) is completely determined by \(s_0\) in the sense
that, for every \(y \in \cat C\) and element \(t \in F y\), there exists a
unique morphism \(f: x_0 \to y\) such that \(F(f)(s_0) = t\).
\end{proof}

\begin{corollary}
\label{cor:fct-to-presheaf-recover-fct-to-cat}
Let \(F: \cat C \to \Psh{\cat D}\) be a functor. If, for every
\(c \in \cat C\), there exists an object \(d \in \cat D\) such that
\(F c \iso d\), then there exists a \emph{unique} --- up to unique isomorphism
--- functor \(F_0: \cat C \to \cat D\) for which
\[
F \iso \yo_{\cat D} F_0.
\]
\end{corollary}

\begin{proof}
From \cref{rem:C-is-full-subcategory-of-presheaf-cat} we know that \(\cat D\) is
a full subcategory of the presheaf category \(\Psh{\cat D}\). Since
\(\yo_{\cat D}: D \to \Psh{\cat D}\) is fully faithful, by means of
\cref{lem:full-subcategory-functor-correspondence} one finds that there exists a
unique functor \(F_0: \cat C \to \cat D\) (up to unique isomorphism) and a
natural isomorphism \(F \iso \yo_{\cat D} F_0\). In other words, the following
diagram is quasi-commutative
\[
\begin{tikzcd}
\cat C \ar[r, "F"] \ar[rd, bend right, "F_0"', dashed]
&{\Psh{\cat D}} \\
&\cat D \ar[u, "\yo_{\cat D}"']
\end{tikzcd}
\]
\end{proof}

\subsection{Properties of the Category of Elements}

\begin{proposition}
\label{prop:representable-iff-El-has-final-object}
Let \(F: \cat C^{\op} \to \Set\) be a functor (or \(F: \cat C \to \Set\) for the
covariant case). Then \(F\) is \emph{representable} if and only if the category
of elements \(\El_F(\cat C)\) has a \emph{final} object (or \emph{initial}
object for the covariant case).
\end{proposition}

\begin{proof}
We prove the proposition for the contravariant case, the covariant case has a
completely analogous construction and can be obtained by duality. Let \(F\) be
representable, with representative \(x_0 \in \cat C\), and universal element
\(s_0\) --- that is, defining a natural isomorphism \(\yo_{\cat C} x_0 \iso
F\). Consider any element \((x, s) \in \El_F(\cat C)\) --- the morphisms
\((x, s) \to (x_0, s_0)\) in \(\El_F(\cat C)\) are arrows \(f: x \to x_0\) in
\(\cat C\) for which \(F(f)(s_0) = s\). Nonetheless, since \(\yo_{\cat C}\) is
fully faithful the morphism \(f\) satisfying such condition is unique ---
therefore \((x_0, s_0)\) is final in the category of elements \(\El_F(\cat C)\).

On the other hand, assume that \(\El_F(\cat C)\) has a final object
\((x_0, s_0)\). We construct a natural transformation
\(\theta: F \nat \yo_{\cat C} x_0\) by the maps
\(\theta_x: F x \to \Hom_{\cat C}(x, x_0)\), for \(x \in \cat C\), sending
\(s \mapsto f\) where \(F(f)(s_0) = s\) --- this is well defined since
\((x_0, s_0)\) is final and thus \(f: x \to x_0\) is the unique morphism in
\(\cat C\) with such property. Moreover, \(\theta_x\) is clearly injective: if
\(s, s' \in F x\) are any two elements such that
\(\theta_x(s) = \theta_x(s') = f\), then \(F(f)(s_0) = s\) and also
\(F(f)(s_0) = s'\) --- which can only be the case for \(s = s'\). The
surjectivity of \(\theta_x\) is ensured by the fact that, given a morphism
\(f: x \to x_0\) in \(\cat C\), one can choose the element \(F(f)(s_0) \in F x\)
so that \(\theta_x(F(f)(s_0)) = f\). We conclude that \(\theta_x\) is a
bijection and hence \(\theta\) is a natural isomorphism
\(F \iso \yo_{\cat C} x_0\).
\end{proof}

\begin{definition}[Two sided represented functor]
\label{def:two-sided-represented-functor}
Let \(\cat C\) be a \(\mathcal U\)-category, then there is a functor
\(\Hom(-, -): \cat C^{\op} \times \cat C \to \Set\) defined by mapping objects
\((x, y) \mapsto \Hom(x, y)\) and morphisms \((f, g): (a, y) \to (x, b)\) to a
set-function \((f^{*}, g_{*}): \Hom(x, y) \to \Hom(a, b)\) given by the mapping
\(g \mapsto h g f\).
\end{definition}

\section{Metric Spaces}

\todo[inline]{Write on metric spaces}

\begin{definition}
    \label{def:premetric-and-metric-space}
    Let \(M\) be a set. We say that a map \(d: M \times M \to \R_{\geq 0}\) is a
    \emph{pre-metric} if for all points \(x, y, z \in M\) it satisfies
    \begin{enumerate}[(a)]\setlength\itemsep{0em}
        \item Symmetry: \(d(x, y) = d(y, x)\).
        \item Triangle inequality: \(d(x, z) \leq d(x, y) + d(y, z)\).
    \end{enumerate}
    Moreover, if \(d\) happens to satisfy the condition that
    \begin{enumerate}[(a)]\setlength\itemsep{0em}\setcounter{enumi}{2}
        \item \(d(x, y) = 0\) implies in \(x = y\).
    \end{enumerate}
    then we say that \(d\) establishes a \emph{metric} in \(M\).

    The set \(M\) together with the (pre)metric \(d\) is called a (pre)metric space.
\end{definition}

%%% Local Variables:
%%% TeX-master: "../../deep-dive"
%%% End:
\documentclass[11pt, reqno]{amsart}
\usepackage[english]{babel}

\usepackage{layout}
\usepackage{afterpage}
\usepackage[
  asymmetric,
  textheight     = 673pt,
  marginparsep   = 7pt,
  footskip       = 27pt,
  hoffset        = 0pt,
  paperwidth     = 597pt,
  textwidth      = 452pt,
  marginparwidth = 101pt,
  voffset        = 0pt,
  paperheight    = 845pt,
]{geometry}

\newcommand{\changegeometry}{\newgeometry{includehead,headheight=89pt}%
  \afterpage{\aftergroup\restoregeometry}%
}

% Math stuff: do not mess with the ordering!
\usepackage{mathtools}
\usepackage{amsthm}
\usepackage{amssymb}
\usepackage{stmaryrd}
\usepackage{tikz-cd}
\usepackage{tqft}

% Font
\usepackage[no-math]{newpxtext}
\usepackage{newpxmath}
% Set arrow tip to that of newpxmath
\tikzset{>=Straight Barb, commutative diagrams/arrow style=tikz}

% Utilities
\usepackage{enumerate}
\usepackage{todonotes}
\usepackage{graphicx}

\newcommand{\HRule}{\rule{\linewidth}{0.5mm}} % Horizontal rule
\usepackage{multirow}

% Color
\usepackage{xcolor}
\definecolor{brightmaroon}{rgb}{0.76, 0.13, 0.28}

% References
\usepackage{hyperref}
\hypersetup{
  colorlinks = true,
  allcolors  = brightmaroon,
}
\usepackage[capitalize,nameinlink]{cleveref}
\usepackage[
  backend = biber,
  style   = alphabetic,
]{biblatex}
\addbibresource{../../src/bibliography.bib}

% Table of contents: show subsections
\setcounter{tocdepth}{2}

\linespread{1.05}
\vfuzz=14pt % No more vbox errors all over the place

%%%%%%%%%%%%%%%%%%%%%%%%%%%%%%%%%%%%%%%%%%%%%%%%%%%%%%%%%%%%%%%%%%%%%%%%%%%%%%%
% ** Environments **

\theoremstyle{definition}
\newtheorem{theorem}{Theorem}[section]
\newtheorem{proposition}[theorem]{Proposition}
\newtheorem{lemma}[theorem]{Lemma}
\newtheorem{corollary}[theorem]{Corollary}
\newtheorem{axiom}[theorem]{Axiom}
\newtheorem{definition}[theorem]{Definition}
\newtheorem{remark}[theorem]{Remark}
\newtheorem{example}[theorem]{Example}
\newtheorem{notation}[theorem]{Notation}

%%%%%%%%%%%%%%%%%%%%%%%%%%%%%%%%%%%%%%%%%%%%%%%%%%%%%%%%%%%%%%%%%%%%%%%%%%%%%%%
% ** symbols **

\renewcommand{\qedsymbol}{\(\natural\)}
\renewcommand{\leq}{\leqslant}
\renewcommand{\geq}{\geqslant}
\renewcommand{\setminus}{\smallsetminus}
\renewcommand{\preceq}{\preccurlyeq}

% ':' for maps and '\colon' for relations on collections
\DeclareMathSymbol{:}{\mathpunct}{operators}{"3A}
\let\colon\relax
\DeclareMathSymbol{\colon}{\mathrel}{operators}{"3A}

% Disjoint unions over sets
\newcommand{\disj}{\amalg}     % Disjoint union
\newcommand{\bigdisj}{\coprod} % Indexed disjoint union

\DeclareMathOperator{\Log}{Log}
\newcommand{\img}{\text{i}}

% Constant map
\DeclareMathOperator{\const}{cons}

% Multiplication map
\DeclareMathOperator{\mul}{mul}

%%%%%%%%%%%%%%%%%%%%%%%%%%%%%%%%%%%%%%%%%%%%%%%%%%%%%%%%%%%%%%%%%%%%%%%%%%%%%%%
% ** arrows **

% Alias for Rightarrow
\newcommand{\To}{\Rightarrow}

% Monomorphism arrow
\newcommand{\mono}{\rightarrowtail}

% Epimorphism arrow
\newcommand{\epi}{\twoheadrightarrow}

% Unique morphism
\newcommand{\unique}{\to}%{\dashrightarrow}
\newcommand{\xdashrightarrow}[2][]{\ext@arrow 0359\rightarrowfill@@{#1}{#2}}

% Isomorphism symbol
\newcommand{\iso}{\simeq}

\newcommand{\arrowiso}{\iso}
% Isomorphism arrow
\newcommand{\isoto}{\xrightarrow{\raisebox{-.6ex}[0ex][0ex]{\(\arrowiso\)}}}

% How isomorphisms are depicted in diagrams: either \sim or \simeq
\newcommand{\dis}{\iso}

\newcommand{\isounique}{%
  \xdashrightarrow{\raisebox{-.6ex}[0ex][0ex]{\(\arrowiso\)}}
}%

% Natural transformation arrow
\newcommand{\nat}{\Rightarrow}

% Natural isomorphism
\newcommand{\isonat}{\xRightarrow{\raisebox{-.6ex}[0ex][0ex]{\(\arrowiso\)}}}

% Embedding arrow
\newcommand{\emb}{\hookrightarrow}

% Parallel arrows
\newcommand{\para}{\rightrightarrows}

% Adjoint arrows
\newcommand{\adj}{\rightleftarrows}

% Implication
\renewcommand{\implies}{\Rightarrow}
\renewcommand{\impliedby}{\Leftarrow}

% Limits
\DeclareMathOperator{\Lim}{lim}     % Limit
\DeclareMathOperator{\Colim}{colim} % Colimit
\DeclareMathOperator{\Eq}{eq}       % Equalizer
\DeclareMathOperator{\Coeq}{coeq}   % Coequalizer

%%%%%%%%%%%%%%%%%%%%%%%%%%%%%%%%%%%%%%%%%%%%%%%%%%%%%%%%%%%%%%%%%%%%%%%%%%%%%%%
% ** Common collections **

\newcommand{\Z}{\mathbf{Z}}
\newcommand{\N}{\mathbf{N}}
\newcommand{\Q}{\mathbf{Q}}
\newcommand{\CC}{\mathbf{C}}
\newcommand{\R}{\mathbf{R}}
\newcommand{\FF}{\mathbf{F}}

\renewcommand{\emptyset}{\varnothing}

\newcommand{\Uhs}{\mathbf{H}}  % Upper half space
\newcommand{\Proj}{\mathbf{P}} % Projective space

%%%%%%%%%%%%%%%%%%%%%%%%%%%%%%%%%%%%%%%%%%%%%%%%%%%%%%%%%%%%%%%%%%%%%%%%%%%%%%%
% ** Categories **

% Font for categories
\newcommand{\cat}{\texttt}
\newcommand{\catfont}{\texttt}

% Opposite category
\newcommand{\op}{\mathrm{op}}

% Common categories
\newcommand{\Set}{{\catfont{Set}}}          % Sets
\newcommand{\FinSet}{{\catfont{FinSet}}}    % Finite sets
\newcommand{\pSet}{{\catfont{pSet}}}        % Pointed sets
\newcommand{\tOrd}{{\catfont{tOrd}}}        % Totally ordered sets

\newcommand{\Vect}{{\catfont{Vect}}}        % Vector spaces
\newcommand{\FinVect}{{\catfont{FinVect}}}  % Finite vector spaces

\newcommand{\TVect}{{\catfont{TVect}}}      % Topological vector spaces
\newcommand{\Ban}{{\catfont{Ban}}}          % Banach spaces

\newcommand{\Grp}{{\catfont{Grp}}}          % Groups
\newcommand{\Ab}{{\catfont{Ab}}}            % Abelian groups
\newcommand{\GSet}[1]{{{#1}\text{-}\Set}}   % G-sets
\newcommand{\Grpd}{{\catfont{Grpd}}}        % Groupoids

\newcommand{\Mon}{{\catfont{Mon}}}          % Monoidal category
\newcommand{\coMon}{{\catfont{coMon}}}      % coMonoidal category
\newcommand{\rActMon}{{\catfont{rActMon}}}  % right action category
\newcommand{\lActMon}{{\catfont{lActMon}}}  % left action category
\newcommand{\BrMonCat}{{\catfont{BrMonCat}}} % Cat of braided monoidal cats

\newcommand{\Graph}{{\catfont{Graph}}}      % Graphs
\newcommand{\SimpGraph}{{\catfont{sGraph}}} % Simple graphs
\newcommand{\ProfCol}{{\catfont{Prof}(\Col)}}   % C-profile category


\newcommand{\Rng}{{\catfont{Ring}}}             % Rings
\newcommand{\cRng}{{\catfont{CRing}}}           % Commutative rings
\newcommand{\rMod}[1]{{\texttt{Mod}_{#1}}}      % Right modules
\newcommand{\lMod}[1]{{{}_{#1}\catfont{Mod}}}   % Left modules
\newcommand{\Mod}[1]{{#1\text{-}\catfont{Mod}}} % Modules over comm. ring
\newcommand{\Alg}[1]{{#1\text{-}\catfont{Alg}}} % Algebras
\newcommand{\cAlg}[1]{{#1\text{-}\catfont{CAlg}}} % Commutative algebras

\newcommand{\Cat}{{\catfont{Cat}}}          % Small categories
\newcommand{\CAT}{{\catfont{CAT}}}          % Big categories
\newcommand{\UCat}{{\mathcal{U}\text{-}\catfont{Cat}}} % U-Categories

\newcommand{\Psh}[1]{{\catfont{Psh}({#1})}}   % Category of presheaves
\newcommand{\comma}{\downarrow} % Comma category separator
\DeclareMathOperator{\El}{El}              % Category of elements

% Operators
\DeclareMathOperator{\Hom}{Mor}   % Morphisms
\DeclareMathOperator{\Mono}{Mono} % Monomorphisms
\DeclareMathOperator{\Epi}{Epi}   % Epimorphisms
\DeclareMathOperator{\Fct}{Fct}   % Functors
\DeclareMathOperator{\Obj}{Obj}   % Objects
\DeclareMathOperator{\Mor}{Mor}   % Morphisms, again
\DeclareMathOperator{\End}{End}   % Endomorphisms
\DeclareMathOperator{\Aut}{Aut}   % Automorphisms
\DeclareMathOperator{\Id}{id}     % Identity
\DeclareMathOperator{\im}{im}     % Image
\DeclareMathOperator{\dom}{dom}   % Domain
\DeclareMathOperator{\codom}{cod} % Codomain
\DeclareMathOperator{\supp}{supp} % Support

% Yoneda embedding
\newcommand{\yo}{\text{\usefont{U}{min}{m}{n}\symbol{'210}}}
\DeclareFontFamily{U}{min}{}
\DeclareFontShape{U}{min}{m}{n}{<-> udmj30}{}

%%%%%%%%%%%%%%%%%%%%%%%%%%%%%%%%%%%%%%%%%%%%%%%%%%%%%%%%%%%%%%%%%%%%%%%%%%%%%%%
% ** algebra **
\DeclareMathOperator{\rank}{rank}
\DeclareMathOperator{\coker}{coker}
\DeclareMathOperator{\codim}{codim}
\DeclareMathOperator{\Tr}{tr}   % Trace
\DeclareMathOperator{\Sym}{Sym} % Symmetric space
\DeclareMathOperator{\Alt}{Alt} % Alternating space
\DeclareMathOperator{\Ann}{Ann} % Annihilator
\DeclareMathOperator{\Char}{char}
\DeclareMathOperator{\Span}{span}
\DeclareMathOperator{\Inn}{Inn}     % Inner automorphisms
\DeclareMathOperator{\Spec}{Spec}   % Prime spectrum
\DeclareMathOperator{\Specm}{Spec_m} % Maximal spectrum
\newcommand{\lie}[1]{\mathfrak{#1}} % Font for Lie structures
\DeclareMathOperator{\Rees}{Rees}   % Rees algebra
\DeclareMathOperator{\Frac}{Frac}   % Field of fractions

\DeclareMathOperator{\torsion}{\texttt{tor}}   % Torsion

\DeclareMathOperator{\eval}{eval}
\DeclareMathOperator{\sign}{sign}

% Matrices
\DeclareMathOperator{\Mat}{Mat}
\DeclareMathOperator{\GL}{GL}
\DeclareMathOperator{\SL}{SL}
\DeclareMathOperator{\PSL}{PSL}
\DeclareMathOperator{\SO}{SO}
\DeclareMathOperator{\SU}{SU}
\DeclareMathOperator{\Unit}{U}
\DeclareMathOperator{\Orth}{O}

% Symbol for the group of units --- for instance, the group of units of a ring
% \(R\) will be denoted by \(R^{\unit}\).
\newcommand{\unit}{\times}

% Orbit and stabilizer
\DeclareMathOperator{\Orb}{Orb}
\DeclareMathOperator{\Stab}{Stab}

% Left and right group actions
\newcommand{\laction}{\circlearrowright}
\newcommand{\raction}{\circlearrowleft}

% Ring ideals font
\newcommand{\ideal}[1]{\mathfrak{#1}}

%%%%%%%%%%%%%%%%%%%%%%%%%%%%%%%%%%%%%%%%%%%%%%%%%%%%%%%%%%%%%%%%%%%%%%%%%%%%%%%
% ** Topology **

\let\Top\relax
\newcommand{\Top}{{\catfont{Top}}}                       % Topological spaces

\newcommand{\wHTop}{{\catfont{wH}\text{-}\catfont{Top}}} % Weak Hausdorff
\newcommand{\kTop}{{k\text{-}\catfont{Top}}}             % k-spaces
\newcommand{\cgTop}{{\catfont{cgTop}}} % Compactly generated

\newcommand{\pTop}{{\catfont{Top}^{*}}}   % Pointed top spaces
\newcommand{\bpTop}{{\catfont{Top}^{*/}}} % Base point top spaces

\newcommand{\Ho}[1]{{\catfont{Ho}(#1)}}            % Homotopy category
\newcommand{\HoTop}{{\catfont{Ho}(\catfont{Top})}} % Classical Homotopy cat

\newcommand{\Splx}{{\mathbf{\Delta}}}           % Simplex category
\newcommand{\sSet}{{\catfont{sSet}}}            % Simplicial sets
\newcommand{\Simp}[1]{{\catfont{Simp}(#1)}}     % Simplicial category
\newcommand{\CoSimp}[1]{{\catfont{CoSimp}(#1)}} % Cosimplicial category

\newcommand{\splx}{\Delta}
\newcommand{\splxtop}{\Delta_{\text{top}}}

\DeclareMathOperator{\Sing}{Sing} % Singular complex functor
\DeclareMathOperator{\Nondeg}{nd} % Non-degenerate simplices
\newcommand{\disc}{\text{disc}}   % discrete

\DeclareMathOperator{\Sk}{sk} % Skeleton

% attaching spaces
\newcommand{\att}{\amalg}     % Disjoint union
\newcommand{\bigatt}{\coprod} % Indexed disjoint union

% Homotopy
\newcommand{\simht}{\sim_{\text{h}}}              % Homotopy between maps
\newcommand{\simhtrel}[1]{\sim_{\text{rel }{#1}}} % Relative htpy
\newcommand{\isoht}{\iso_{\text{h}}}              % Homotopy equivalence
\newcommand{\htpy}{\Rightarrow}                   % Htpy arrow
\newcommand{\htpyrel}[1]{\Rightarrow_{\text{rel }{#1}}} % Relative htpy arrow


% Functors on topological spaces
\DeclareMathOperator{\Disc}{Disc}       % Discrete topology: Set -> Top
\DeclareMathOperator{\Cone}{Cone}       % Cone
\DeclareMathOperator{\Cyl}{Cyl}         % Cylinder
\DeclareMathOperator{\Susp}{S}          % Suspension
\DeclareMathOperator{\rSusp}{\Sigma}    % Reduced suspension
\DeclareMathOperator{\Path}{Path}       % Path object
\DeclareMathOperator{\Loop}{\Omega}       % Loop space
\DeclareMathOperator{\Eval}{eval}       % Evaluation map
\DeclareMathOperator{\curry}{curry}     % Currying a map
\DeclareMathOperator{\uncurry}{uncurry} % Currying a map
\DeclareMathOperator{\co}{co}           % Compact open co(K, U)

\newcommand{\lift}{\widehat} % Lifting of path and stuff

% Set operators
\DeclareMathOperator{\Cl}{Cl}       % Closure
\DeclareMathOperator{\Bd}{\partial} % Boundary
\DeclareMathOperator{\Int}{Int}     % Interior
\DeclareMathOperator{\Ext}{Ext}     % Exterior

%%%%%%%%%%%%%%%%%%%%%%%%%%%%%%%%%%%%%%%%%%%%%%%%%%%%%%%%%%%%%%%%%%%%%%%%%%%%%%%
% ** Differentiable structures **

\newcommand{\Bun}{{\catfont{Bun}}} % Bundle category
\newcommand{\VecBun}{{\catfont{VecBun}}} % Vector Bundle category
\newcommand{\Man}{{\catfont{Man}}} % Manifolds

% Norm
\DeclarePairedDelimiter{\norm}{\lVert}{\rVert}

% Differential operators
\newcommand{\diff}{\mathrm{d}}
\newcommand{\Diff}{\mathrm{D}}
\DeclareMathOperator{\grad}{grad} % Gradient
\DeclareMathOperator{\Hess}{Hess} % Hessian
\DeclareMathOperator{\Jac}{Jac}   % Jacobian
\DeclareMathOperator{\Curl}{Curl} % Curl
\DeclareMathOperator{\VecField}{\mathfrak{X}}

\DeclareMathOperator{\Grass}{Grass} % Grassmann variety
\DeclareMathOperator{\Stie}{Stie}      % Stiefel variety

% Vector bundle
\DeclareMathOperator{\zerosec}{Zero} % Zero section

\newcommand{\trans}{\pitchfork} % Transversality

% Set operators
\DeclareMathOperator{\Vol}{vol}   % Volume
\DeclareMathOperator{\Mesh}{mesh} % Mesh

%%%%%%%%%%%%%%%%%%%%%%%%%%%%%%%%%%%%%%%%%%%%%%%%%%%%%%%%%%%%%%%%%%%%%%%%%%%%%%%
% ** Graphs **

% Colouring
\newcommand{\Col}{\mathfrak{C}}
\newcommand{\prof}[1]{\underline{#1}}

\DeclareMathOperator{\Edge}{Edge}
\DeclareMathOperator{\Vertex}{Vert}
\DeclareMathOperator{\Circ}{circ}
\DeclareMathOperator{\diam}{diam}
\newcommand{\emptygraph}{\varnothing}
\DeclarePairedDelimiterX{\size}[1]{\lVert}{\rVert}{#1}

%%%%%%%%%%%%%%%%%%%%%%%%%%%%%%%%%%%%%%%%%%%%%%%%%%%%%%%%%%%%%%%%%%%%%%%%%%%%%%%
% ** MACROS END HERE **
%%%%%%%%%%%%%%%%%%%%%%%%%%%%%%%%%%%%%%%%%%%%%%%%%%%%%%%%%%%%%%%%%%%%%%%%%%%%%%%

\begin{document}
\begin{titlepage}
 \vfill
  \begin{center}
       \textsc{\LARGE \textbf{University of São Paulo}} \\[2.0cm]

       \vskip 0.5cm
       \textsc{\large Luiz Gustavo Mugnaini Anselmo}

       {\normalsize Molecular Sciences Course \\
         Class 30, n\(^{\text{o}}\)
         USP 11809746

       E-mail: \texttt{luizmugnaini@usp.br}}\\[2.0cm]

       \HRule\\
       \vskip 0.5cm
       {\LARGE \textbf{Dendroidal Homotopy Theory \& Operads}}
       \HRule\\[1.5cm]

       \hspace{.45\textwidth}
       \begin{minipage}{.5\textwidth}
       \normalsize \textbf{Advanced Studies Report I}\\[0.5cm]

       \textsc{\large Prof.~Ivan Struchiner}\\
       University of São Paulo \\
       Institute of Mathematics and Statistics \\
       E-mail: \texttt{ivanstru@ime.usp.br}\\[1cm]

       \normalsize São Paulo, December of 2022
       \end{minipage}
  \end{center}
\end{titlepage}

% Title stuff
\title[Dendroidal Homotopy Theory \& Operads]{%
{\footnotesize\sl Advanced Studies Report I} \\ \smallskip
  Dendroidal Homotopy Theory \& Operads
}%

\author{%
  Luiz Gustavo Mugnaini Anselmo and Ivan Struchiner
}%

\address{%
  Institute of Mathematics and Statistics, University of São
  Paulo, Rua do Matão 1010, 05508--090~São Paulo, SP
}%

\email{luizmugnaini@usp.br, ivanstru@ime.usp.br}

\begin{abstract}
This report documents the first research advances of the student Luiz
G. Mugnaini A. during his first semester of the advanced cycle of the Molecular
Sciences course. The main topic of research explored this semester was the
study of simplicial methods in homotopy theory.
\end{abstract}
\maketitle

\section{Research}

This semester our study focused in the theory of simplicial homotopy theory,
which will play a key role in our next advances through the field of homotopy
theory. In what follows, we lay a brief summary of what was discussed in our
reading group and seminars---which where held weekly.

\subsection{Simplicial Sets}

\subsubsection{Construction}

\begin{definition}[Simplicial object]
\label{def:simplicial-object}
For any category \(\cat C\), a \emph{simplicial object} in \(\cat C\) is a
contravariant functor \(\Splx^{\op} \to \cat C\). We define \(\Simp{\cat C}\) to
be the category whose objects are simplicial objects and natural transformations
between them---such category is commonly referred to as the \emph{simplicial
  category of \(\cat C\)}.
\end{definition}

\begin{notation}[Simplicial sets, notations and nomenclatures]
\label{not:simplicial-sets-notation-and-nomenclature}
In particular, the most important case we'll study for the time being is the
\emph{simplicial category of sets}, which we denote by \(\sSet\). The objects of
\(\sSet\) shall be called \emph{simplicial sets}.\

Given a simplicial set \(X \in \sSet\), the points of \(X_n \in \Set\) will be
referred to as the \emph{\(n\)-cells} (or \emph{\(n\)-simplices}) of
\(X \in \sSet\).
\end{notation}

Since every map \(\alpha\) in \(\Splx\) may be decomposed into elementary face
and degeneracies, the the collection \(\delta_i^{*}\) and \(\sigma_j^{*}\) also
generate the morphisms \(\alpha^{*}\) between the objects of
\((X_n)_{n \in \N}\) of \(\cat C\). In the context of the simplicial category,
we denote them by
\begin{align*}
  d_i^n &\coloneq (\delta_i^n)^{*}: X_n \longrightarrow X_{n-1}, \\
  s_j^n &\coloneq (\sigma_j^n)^{*}: X_n \longrightarrow X_{n+1},
\end{align*}
for all \(0 \leq i, j \leq n\). The arrows \(d_i\) are called \emph{face maps}
(or \emph{cofaces}), while the arrows \(s_j\) are called \emph{degeneracy maps}
(or \emph{codegeneracies}) of the simplicial object \(X\).

A map \(\eta: X \nat Y\) is a natural transformation between simplicial objects
if and only if it is compatible with the face and degeneracy maps, that is, the
following two diagrams commute in \(\cat C\) for all \(n \in \N\),
\(0 \leq i, j\leq n\):
\[
\begin{tikzcd}
X_n \ar[r, "\eta_n"] \ar[d, "d_i^n"'] &Y_n \ar[d, "d_i^n"] \\
X_{n-1} \ar[r, "\eta_{n-1}"'] &Y_{n-1}
\end{tikzcd}
\qquad
\qquad
\begin{tikzcd}
X_n \ar[r, "\eta_n"] \ar[d, "s_j^n"'] &Y_n \ar[d, "s_j^n"] \\
X_{n+1} \ar[r, "\eta_{n+1}"'] &Y_{n+1}
\end{tikzcd}
\]

\begin{corollary}[Simplicial identities]
\label{cor:simplicial-identities}
Since the face and degeneracy maps \(d_i\) and \(s_j\) are \emph{dual} to the
elementary face and degeneracy maps \(\delta_i\) and \(\sigma_j\), they satisfy
the following identities:
\begin{enumerate}[(1)]\setlength\itemsep{0em}
\item \(d_i d_j = d_{j-1} d_i\), for \(i < j\).
\item \(s_j s_i = s_i s_{j-1}\), for \(i < j\).
\item \(d_j s_i = s_i d_{j-1}\), for \(i < j-1\).
\item \(d_j s_i = \Id\), if \(i = j\) or \(i = j-1\).
\item \(d_j s_i = s_{i-1} d_j\), for \(i > j\).
\end{enumerate}
\end{corollary}

\begin{definition}[Subcomplex]
\label{def:subcomplex}
By a \emph{subcomplex} of a simplicial set we mean a subobject.
\end{definition}

\subsubsection{Discrete Simplicial Sets}

\begin{definition}[Discrete simplicial set]
\label{def:discrete-simplicial-set}
We say that a simplicial set \(X\) is \emph{discrete} if every simplicial
operator \(f: [m] \to [n]\) induces a \emph{bijective set-function}
\(f^{*}: X_n \to X_m\). We denote by \(\sSet^{\disc}\) the full subcategory of
\(\sSet\) consisting of discrete simplicial sets.

For every set \(S\), there exists a discrete simplicial set \(S^{\disc}\) where
\(S_n^{\disc} \coloneq S\) for each \(n \in \N\), and for any \(f: [n] \to [m]\)
we have \(S^{\disc} f \coloneq \Id_S\).
\end{definition}

\begin{corollary}
\label{cor:bijection-discrete-simplicial-set-and-set-functions}
For any set \(S\) and simplicial set \(X\), there exists a bijection
\[
\Hom_{\sSet}(S^{\disc}, X) \iso \Hom_{\Set}(S, X_0).
\]
\end{corollary}

\begin{proof}
Define a set-function
\(\Phi: \Hom_{\sSet}(S^{\disc}, X) \to \Hom_{\Set}(S, X_0)\) by mapping each
simplicial set morphism \(f: S^{\disc} \to X\) to the corresponding set function
\(f_0: S \to X_0\). To prove that \(\Phi\) is injective, consider simplicial
morphisms \(f, g: S^{\disc} \para X\) such that \(f_0 = g_0\). Applying the
naturallity of \(f\) and \(g\) to the simplicial operator \(j: [n] \to [0]\) we
obtain that \(j^{*} f_0 = f_n\) and \(j^{*} g_0 = g_n\), but since
\(f_0 = f_0\), then \(f_n = g_n\)---proving that \(f = g\). Surjectivity is
immediate, therefore \(\Phi\) is a bijection.
\end{proof}

\begin{proposition}[\(\sSet^{\disc}\) \& \(\Set\)]
\label{prop:sSet-disc-equivalent-to-Set}
The full subcategory of discrete simplicial sets is \emph{equivalent} to the
category of sets.
\end{proposition}

% \begin{proof}
% We start by constructing a functor \(\disc: \Set \to \sSet^{\disc}\) where we
% define the simplicial set \(S^{\disc}\) associated with a given set \(S\) to be
% the simplicial set with \(S_n^{\disc} = S\) for every \(S\). Each set-function
% \(f: A \to B\) is mapped to their corresponding simplicial set morphism
% \(f: A^{\disc} \to B^{\disc}\) given by \(f_n \coloneq f: A \to B\) for each
% \(n \in \N\). We now show that \(\disc\) is an equivalence of categories:

% \begin{itemize}\setlength\itemsep{0em}
% \item (Fully faithful) Given any two sets, \(S\) and \(T\), define a
%   set-function
%   \(\Phi: \Mor_{\Set}(S, T) \to \Mor_{\sSet^{\disc}}(S^{\disc}, T^{\disc})\) by
%   mapping each set function \(f: S \to T\) to \(f: S^{\disc} \to T^{\disc}\) as
%   above. Then it's clear that \(\Phi\) is a \emph{bijection}.

% \item (Essentially surjective) Given any \(X \in \sSet^{\disc}\), consider the
%   set \(S \coloneq X_0\)---we'll show that \(X \iso S^{\disc}\). Define a
%   morphism of simplicial sets \(f: S^{\disc} \to X\) where \(f_0 \coloneq
%   \Id_{X_0}\). Let \(\alpha: [n] \to [0]\) be any simplicial operator. Since
%   \(\alpha^{*}: X_0 \isoto X_n\) is a bijection, then by naturality of \(f\) we
%   find that
%   \[
%   \begin{tikzcd}
%   &X_0 \ar[dd, "\dis"', "\alpha^{*}"]
%   \\
%   S \ar[ru, "\Id_{X_0}"] \ar[rd, "f_n"'] &
%   \\
%   &X_n
%   \end{tikzcd}
%   \]
%   commutes in \(\Set\), making \(f_n\) a bijection, which proves that \(f\) is a
%   natural isomorphism of simplicial sets.
% \end{itemize}
% \end{proof}

\subsubsection{Standard Simplices}

\begin{definition}[Standard simplex]
\label{def:standard-n-simplex}
We define the \emph{standard \(n\)-simplex} \(\splx^n\) to be the contravariant
functor \(\Splx^{\op} \to \Set\) represented by \([n]\), that is, the
\emph{simplicial set}
\[
\splx^n \coloneq \Hom_{\Splx}(-, [n]).
\]
The \emph{generator} of the standard \(n\)-simplex corresponds to the \(n\)-cell
\[
\iota_n \coloneq \Id_{[n]} \in \splx^n [n].
\]
\end{definition}

Let \(\phi: [\ell] \to [k]\) be any simplicial operator, then the action of
\(\phi\) on the cells of \(\splx^n\) is given by the map
\(\splx^n \phi = \phi^{*}: \splx^k \to \splx^{\ell}\). Explicitly, given a
\(k\)-cell \(\alpha: [k] \to [n]\) of \(\splx^n\), the action of the simplicial
operator \(\phi\) on \(\alpha\) yields an \(\ell\)-cell
\[
\big(\phi^{*} \alpha = \alpha \phi: [\ell] \longrightarrow [n]\big)
\in \splx^n[\ell].
\]

\begin{lemma}[Yoneda consequences]
\label{lem:yoneda-lemma-on-Splx-category}
Given a simplicial set \(X\), there exists a bijection
\[
\Hom_{\sSet}(\splx^n, X) \iso X_n,
\]
which is explicitly given by the map \(\eta \mapsto \eta_n
\iota_n\). Equivalently, for each \(n\)-cell \(x \in X_n\), there \emph{exists}
a \emph{unique} morphism of simplicial sets \(\eta^x: \splx^n \to X\) such that
\(\eta_n^x \iota_n = x\).
\end{lemma}

\begin{proof}
Consider the Yoneda functor \(\yo_{\Splx}: \Splx \to \sSet\), mapping each
object \([n] \mapsto \splx^n\) and each simplicial operator \(f: [n] \to [m]\)
to the morphism of simplicial sets \(f_{*}: \splx^n \to \splx^m\). Explicitly,
for each \(k \in \N\) we have \((f_{*})_k: \splx^n[k] \to \splx^m[k]\) mapping
\(\alpha \mapsto f \alpha\). By the Yoneda lemma we know that there exists, for
each \(n \in \N\), a bijection
\[
\Hom_{\sSet}(\splx^n, X) = \Hom_{\sSet}(\yo_{\Splx}[n], X) \iso X_n.
\]
\end{proof}

\begin{definition}[Representing map of a cell]
\label{def:representing-map-of-a-cell}
Regarding \cref{lem:yoneda-lemma-on-Splx-category}, we shall refer to \(\eta^x\)
as the \emph{representing map} of the \(n\)-cell \(x\). When convenient,
we'll simply denote \(\eta^x\) by \(x: \splx^n \to X\).
\end{definition}

\begin{corollary}
\label{cor:bijection-mor-set-standard-simplices}
There exists a bijection
\[
\Hom_{\sSet}(\splx^n, \splx^m) \iso \splx^m[n].
\]
The \emph{inverse} of this set-function is explicitly given by mapping each
simplicial operator \(\alpha: [n] \to [m]\) to the corresponding morphism of
simplicial sets \(\alpha_{*}: \splx^n \to \splx^m\)---where \(\alpha_{*}\) acts
on any \(k\)-cell \(x \in \splx^n[k]\) as
\(\alpha_{*} x = \alpha x \in \splx^m[k]\).
\end{corollary}

\begin{proof}
In the proof of \cref{lem:yoneda-lemma-on-Splx-category}, merely consider the
case where \(X \coloneq \splx^m\).
\end{proof}

\begin{corollary}[Standard simplices \& \(\Splx\)]
\label{cor:standard-simplices-equivalent-Splx}
The full subcategory \(\sSet_{\splx}\) of \(\sSet\) consisting of standard
simplices is \emph{equivalent} to the simplex category \(\Splx\).
\end{corollary}

\begin{proof}
Consider the functor \(F: \Splx \to \sSet_{\splx}\) given by
\([n] \mapsto \splx^n\) and mapping each simplicial operator \(f: [n] \to [m]\)
to its corresponding morphism of simplicial sets \(f_{*}: \splx^n \to
\splx^m\). By \cref{lem:yoneda-lemma-on-Splx-category} we see that \(F\) is
fully faithful and essentially surjective---thus an equivalence of categories.
\end{proof}

\begin{definition}[Standard simplices on totally ordered sets]
\label{def:standard-simplex-on-totally-ordered-set}
Given a totally ordered set \(S\), we define the \emph{standard \(S\)-simplicial
set} \(\splx^S\) to be the functor\footnote{We denote by \(\tOrd\) the category
of totally ordered sets.}
\[
\splx^{S} \coloneq \Hom_{\tOrd}(-, S)|_{\Splx},
\]
that is, for each \(n \in \N\) we define \(\splx^S_n\) to be the collection of
order preserving maps \([n] \to S\).
\end{definition}

\begin{definition}[Boundary of a standard simplex]
\label{def:boundary-of-standard-simplex}
For every \(n \in \N\), we define the \emph{boundary} of the standard
\(n\)-simplex \(\splx^n\) to be the subobject
\[
\Bd \splx^n \coloneq \Colim_{k \in [n]} \splx^{[n] \setminus k}
= \bigcup_{k \in [n]} \splx^{[n] \setminus k}
\]
composed of all \emph{codimension-one faces} of \(\splx^n\). Explicitly, for
each \(k \in \N\) we have
\[
\Bd \splx^n [k] = \{ f \in \Hom_{\Splx}([k], [n]) \colon f[k] \neq [n] \}
\subseteq \splx^n[k],
\]
composed of all \emph{non-surjective \(k\)-cells} of \(\splx^n\).
\end{definition}

\begin{proposition}
\label{prop:boundary-maximal-proper-subcomplex}
The boundary \(\Bd \splx^n\) is the \emph{maximal proper subcomplex of
  \(\splx^n\)}.
\end{proposition}

\subsubsection{Geometric Realization of the Simplex Category}

We define, for each \(n \in \N\), a corresponding \emph{standard topological
  \(n\)-simplex} \(\splxtop^n\) given by
\[
\splxtop^n \coloneq \bigg\{(t_0, \dots, t_n) \in \R^{n+1} \colon \sum_{j=0}^n
t_j = 1 \text{ and } t_j \geq 0 \text{ for all } j \bigg\}.
\]
Each \(\splxtop^n\) is composed of \(n+1\) vertices
\(v_j \coloneq (\delta_{ij})_{i=0}^n\).

From a categorical point of view, standard topological simplices are nothing
more than a functor
\[
\splxtop^{\bullet}: \Splx \longrightarrow \Top,
\]
mapping objects \(\splxtop^{\bullet}[n] \coloneq \splxtop^n\)---where
\(\splxtop^n\) is endowed with the standard euclidean topology---and for each
morphism \(f: [n] \to [m]\) in \(\Splx\), we map
\(\splxtop^{\bullet} f \coloneq f_{*}\), where
\(f_{*}: \splxtop^n \to \splxtop^m\) is a uniquely determined continuous map
such that \(f_{*}(v_j) \coloneq v_{f(j)}\). From this definition we obtain
\[
f_{*}(t_0, \dots, t_n) = (s_0, \dots, s_m) \text{,\ \ where \ }
s_j = \sum_{f(i) = j} t_i,
\]
that is, \(s_j\) is the sum of the points that are collapsed to the \(j\)-th
coordinate.

This functor gives us a geometric visualization of the action of the elementary
face and degeneracies:
\begin{itemize}\setlength\itemsep{0em}
\item Given an elementary \emph{face} map \(\delta_j: [n-1] \mono [n]\), for any
  \(0 \leq j \leq n\), the corresponding map
  \((\delta_j)_{*}: \splxtop^{n-1} \emb \splxtop^n\) is given by
  \[
  (\delta_j)_{*} v_i =
  \begin{cases}
    v_i, &\text{if } i < j, \\
    v_{i+1}, &\text{if } i \geq j.
  \end{cases}
  \]
  That is, \((\delta_j)_{*}\) embeds \(\splxtop^{n-1}\) as a face of
  \(\splxtop^n\) opposite to the \(j\)-th vertex.

\item An elementary \emph{degeneracy} map \(\sigma_j: [n+1] \epi [n]\), for any
  \(0 \leq j \leq n\), has a map
  \((\sigma_j)_{*}: \splxtop^{n+1} \epi \splxtop^n\) mapping the vertices as
  follows
  \[
  (\sigma_j)_{*} v_i =
  \begin{cases}
    v_i, &\text{if } i \leq j, \\
    v_{i-1}, &\text{if } i > j.
  \end{cases}
  \]
  Geometrically, the degeneracy map makes \(\splxtop^{n+1}\) into \(\splxtop^n\)
  by removing a face of dimension \(1\)---through the projection parallel to the
  line connecting \(v_j\) and \(v_{j+1}\).
\end{itemize}

\subsubsection{Geometric Realization of a Simplicial Set}

Given a simplicial set \(X: \Splx^{\op} \to \Set\), we consider the topological
space
\[
\coprod_{n \in \N} X_n \times \splxtop^n
\]
and construct in this space a minimal equivalence relation \(\sim_{\text{gr}}\)
for which points \((x, t) \in X_n \times \splxtop^n\) and
\((x', t') \in X_m \times \splxtop^m\) are
\emph{equivalent}---\((x, t) \sim_{\text{gr}} (x', t')\)---if and only if there
exists a morphism \(\alpha: [m] \to [n]\) in \(\Splx\) such that
\(x' = \alpha^{*} x\) and \(t = \alpha_{*} t'\). In an equivalent manner, we may
summarize this equivalence relation as gluing points of the form
\[
(x, \alpha_{*} t) \sim_{\text{gr}} (\alpha^{*} x, t).
\]
The points of the resulting quotient space
\((\coprod_{n \in \N} X_n \times \splxtop^n)/{\sim_{\text{gr}}}\) are denoted by
\(x \otimes t\)---corresponding to the class of a pair \((x, t)\).

\begin{definition}[Geometric realization functor]
\label{def:geometric-realization-functor}
We define the \emph{geometric realization} of the category of simplicial sets to
be a functor
\[
|-|: \sSet \longrightarrow \Top,
\]
mapping
\(|X| \coloneq (\coprod_{n \in \N} X_n \times \splxtop^n)/{\sim_{\text{gr}}}\) and
for each natural transformation \(\eta: X \to Y\) we have a topological morphism
\(|\eta|: |X| \to |Y|\) given by \(x \otimes t \mapsto \eta_n x \otimes t\),
for \((x, t) \in X_n \times \splxtop^n\).

For any simplicial set \(X\), each \(n\)-cell \(x \in X_n\) induces
\emph{topological} morphism
\[
\widehat{x}: \splxtop^n \longrightarrow |X|\text{, \ mapping }\
t \longmapsto x \otimes t.
\]
From construction, given a morphism \(\alpha: [n] \to [m]\) in \(\Splx\) and a
point \(y \in X_m\) such that \(y = \alpha^{*} x\), the diagram
\begin{equation}\label{eq:commutativity-n-simplexes}
\begin{tikzcd}
\splxtop^m \ar[rr, "\alpha_{*}"] \ar[rd, "\widehat y"']
& &\splxtop^n \ar[ld, "\widehat x"] \\
&{|X|} &
\end{tikzcd}
\end{equation}
commutes in \(\Top\).
\end{definition}

Given any category \(\cat C\) and a functor \(F: \Splx \to \cat C\), we can
define a \emph{simplicial set} induced by any object \(C \in \cat C\) given by
\[
\Hom_{\cat C}(F(-), C): \Splx^{\op} \longrightarrow \Set.
\]
This simplicial set maps each \([n] \in \Splx\) to the set of morphisms
\(\Hom_{\cat C}(F n, C)\), and each morphism \(\alpha: [m] \to [n]\) of
\(\Splx\) to the set-function
\(\alpha^{*}: \Hom_{\cat C}(F n, C) \to \Hom_{\cat C}(F m, C)\).


% \begin{definition}[Singular complex]
% \label{def:singular-complex-functor}
% Let \(\cat C\) be a category and \(F: \Splx \to \cat C\) be a functor. We define
% the \emph{singular complex} of \(\cat C\) to be the functor
% \[
% \Sing_F: \cat C \longrightarrow \sSet
% \]
% mapping each object \(C \in \cat C\) to the simplicial set
% \[
% \Sing_F(C) \coloneq \Hom_{\cat C}(F(-), C): \Splx^{\op} \longrightarrow \Set,
% \]
% and each map \(\alpha: [m] \to [n]\) to the set-function
% \[
% \Sing_F(\alpha) \coloneq \alpha^{*}:
% \Hom_{\cat C}(F n, C) \longrightarrow \Hom_{\cat C}(F m, C).
% \]

% In particular, the standard topological simplices functor \(\splxtop^{\bullet}: \Splx \to
% \Top\) induces a singular complex on each topological space. Since we shall be
% mostly interested in this particular case for the time being, we shall reserve
% the notation
% \[
% \Sing \coloneq \Sing_{\splxtop^{\bullet}}: \Top \longrightarrow \sSet,
% \]
% with no subscripts, for the standard topological simplices functor. In this
% case, given a topological space \(T\), we shall denote by \(\Sing(T)_n\) the
% image of \([n] \in \Splx\) under the simplicial set \(\Sing(T)\).
% \end{definition}

% Given a simplicial set \(X\), the collection of simplexes
% \((\widehat{x}_n)_{n \in \N}\), where \(x_n \in X_n\), covers the whole
% topological space \(|X|\)---in the sense that collection of images forms a cover
% of \(|X|\). Therefore, given any topological morphism \(\phi: |X| \to T\), this
% map is completely defined by the family of compositions
% \((\phi \widehat{x}_n: \splxtop^n \to T)_{n \in \N}\). Given a morphism
% \(\alpha: [n] \to [n]\) in \(\Splx\), by \cref{eq:commutativity-n-simplexes},
% the diagram
% \[
% \begin{tikzcd}
% \splxtop^m \ar[rr, "\alpha_{*}"]
% \ar[rd, "\widehat y_m"']
% \ar[drd, bend right, "\phi \widehat y_m"']
% & &\splxtop^n \ar[ld, "\widehat x_n"]
% \ar[dld, bend left, "\phi \widehat x_n"] \\
% &{|X|} \ar[d, "\phi"] & \\
% &T &
% \end{tikzcd}
% \]
% commutes in \(\Top\). This construction induces unique a collection of maps
% \[
% (\phi_n: X_n \longrightarrow \Sing(T)_n)_{n \in \N},
% \]
% where \(\phi_n(x) \mapsto \phi \widehat x\). Notice that this family of arrows
% is nothing more than a natural transformation \(\phi: X \nat \Sing(T)\) between
% simplicial sets. From this we conclude that there exists a natural bijection
% \[
% \Hom_{\Top}(|X|, T) \iso \Hom_{\sSet}(X, \Sing(T)),
% \]
% thus the singular complex functor is \emph{right adjoint} to the geometric
% realization,
% \[
% \begin{tikzcd}
% \sSet \ar[rr, shift left, "{|-|}"] &&\Top \ar[ll, shift left, "\Sing"]
% \end{tikzcd}
% \]

\subsubsection{Geometric Realization as a CW-complex}

\begin{definition}[Degenerate \(n\)-cell]
\label{def:degenerate-n-cell}
Given a simplicial set \(X\), we say that an \(n\)-cell \(x \in X_n\) is
\emph{degenerate} if \(x \in s_j X_{n-1}\) for some \emph{codegeneracy} map
\(s_j: X_{n-1} \to X_n\), where \(0 \leq j \leq n-1\).

Equivalently, \(x\) is denenerate if there exists a \emph{surjective} map
\(\alpha: [n] \epi [m]\) and \(m\)-cell \(y \in X_m\) for which
\(x = \alpha^{*} y\).
\end{definition}

\begin{lemma}[Eilenberg-Zilber]
\label{lem:Eilenberg-Zilber}
Let \(x\) be a \(n\)-cell of a given simplicial set \(X\). There \emph{exists
  a unique} pair \((\alpha, y)\) such that \(\alpha: [n] \epi [k]\) is a
\emph{surjective} map and \(y\) is a \emph{non-degenerate} \(k\)-cell of
\(X\) satisfying \(\alpha^{*} y = x\).
\end{lemma}

\begin{proof}
The existence of the pair \((\alpha, y)\) comes straight from definition. Now
suppose \((\beta, z)\) is another pair satisfying the said property, where \(z\)
is a non-degenerate \(\ell\)-cell of \(X\). Since pushouts of a pair of
surjections in \(\Splx\) are absolute, the pushout of the pair
\((\alpha, \beta)\):
\[
\begin{tikzcd}
{[n]} \ar[r, two heads, "\alpha"]
\ar[d, two heads, "\beta"']
\ar[dr, phantom, "\ulcorner", very near end]
&{[k]} \ar[d, "\gamma"] \\
{[\ell]} \ar[r, "\omega"'] &{[s]}
\end{tikzcd}
\]
is turned into a pullback by the simplicial set \(X: \Splx^{\op} \to \Set\),
that is:
\[
\begin{tikzcd}
X_n \ar[rd, phantom, very near start, "\lrcorner"]
&X_k \ar[l, "\alpha^{*}"']
\\
X_{\ell} \ar[u, "\beta^{*}"]
&X_s \ar[u, "\gamma^{*}"'] \ar[l, "\omega^{*}"]
\end{tikzcd}
\]
From the pushout property, \(\gamma\) and \(\omega\) are epimorphisms, thus
split, \(X\) ensures that \(\gamma^{*}\) and \(\omega^{*}\) are
split-epimorphisms in \(\Set\). Therefore there exists \(s\)-cells
\(a, b \in X_s\) such that \(\gamma^{*} a = y\) and \(\omega^{*} b = z\). Notice
however that we assumed \(y\) and \(z\) to be both non-degenerate, hence it must
be the case that \(\gamma\) and \(\omega\) are \emph{identities}. This implies
in \(\beta = \alpha\) and \(y = z\).
\end{proof}

Notice that, given a simplicial set \(X\), we can naturally construct a
\emph{filtration} for \(X\) by defining for each \(n \in \N\) the subspace
\(|X|_n\) of \(|X|\) given by
\[
|X|_n \coloneq \{
x \otimes t \in |X| \colon
(x, t) \in X_k \times \splxtop^k
\text{ for some } k \leq n
\},
\]
so that the collection \((|X|_n)_{n \in \N}\) defines a filtration---that is,
\(|X|_n \leq |X|_{n+1}\) and \(|X| = \bigcup_{n \in \N} |X|_n\) is endowed with
the \emph{weak topology}.

\begin{lemma}
\label{lem:X0-simplicial-set-is-discrete}
Given a simplicial set \(X\), the subspace \(|X|_0 \subseteq |X|\) is
\emph{discrete} and given by
\[
|X|_0 = X_0 \times \splxtop^0.
\]
\end{lemma}

\begin{proof}
We shall create an isomorphism between both spaces. Define a surjective map
\(X_0 \times \splxtop^0 \epi |X|_0\) by mapping \((x, 1) \mapsto x \otimes
1\). Now, suppose that for some \(x, y \in X_0\) one has
\(x \otimes 1 = y \otimes 1\), then there must exist a pair
\((z, t) \in X_n \times \splxtop^n\), for some \(n \in \N\), and parallel
morphisms \(\alpha, \beta: [0] \para [n]\) in \(\Splx\) for which
\[
(x, \alpha_{*} 1) = (\alpha^{*} z, t)
\quad \text{ and } \quad
(y, \beta_{*} 1) = (\beta^{*} z, t).
\]
Since \(\splxtop^0 = 1\) is a single point, then \(\alpha_{*} = \beta_{*}\) and
hence \(\alpha = \beta\), which shows that \(x = y\).
\end{proof}

\begin{notation}
\label{not:non-degenerate-cells}
Denote by \(X_n^{\Nondeg}\) the collection of all \emph{non-degenerate
  \(n\)-cells} of a given simplicial set \(X\).
\end{notation}

\begin{lemma}
\label{lem:unique-non-degenerate-point-geometric-realization}
Let \(\xi \in |X|\) be any point, and let \(n \in \N\) be the \emph{minimal}
index such that there exists \((x, t) \in X_n \times \splxtop^n\) for which
\(\xi = x \otimes t\). Then \(x\) is a non-degenerate simplex, and if
\(n \geq 1\) then \(t \in \Int \splxtop^n\)---also, if that is the case, then the
pair \((x, t)\) is unique with such property.
\end{lemma}

% \begin{proof}
% We shall prove this lemma in three steps:
% \begin{itemize}\setlength\itemsep{0em}
% \item (\(x\) is non-degenerate) Suppose for the sake of contradiction that \(x\)
%   is degenerate, so that there exists a pair \((\alpha: [n] \epi [m], y)\) with
%   \(y \in X_m\) and \(m < n\) such that \(\alpha^{*} y = x\). From this one
%   obtains
%   \[
%   x \otimes t = \alpha^{*} y \otimes t = y \otimes \alpha_{*} t
%   \]
%   then \((y, \alpha_{*} t) \in X_m \times \splxtop^m\) satisfies
%   \(\xi = y \otimes \alpha^{*} t\), which contradicts the assumption that \(n\)
%   was the minimal element of \(\N\) with this property.

% \item (\(t \in \Int \splxtop^n\)) Assume that \(n \geq 1\) and, for the sake of
%   contradiction, suppose that \(t \in \Bd \splxtop^n\)---so that there exists
%   at least one coordinate of \(t\) that is zero. From this last comment, it
%   follows that there exists an injective map \(\beta: [k] \mono [n]\), for some
%   \(k < n\), such that \(t \in \beta_{*} \splxtop^k\). If \(s \in \splxtop^k\) is
%   the point such that \(\beta_{*} s = t\), then
%   \[
%   \xi = x \otimes t = x \otimes \beta_{*} s = \beta^{*} x \otimes s,
%   \]
%   but \(\beta^{*} x \in \splxtop^k\), which contradicts again the minimality of
%   \(n\).

% \item (Uniqueness of the pair \((x, t)\)) Assume again that \(n \geq 1\), and
%   suppose that \(\xi = x \otimes t = y \otimes s\) for some
%   \(x, y \in X_n^{\Nondeg}\) and \(t, s \in \Int \splxtop^n\)---which can be
%   assumed using the result of the last item. From the definition of \(\otimes\)
%   we may consider a finite zig-zag \((\alpha_j, \beta_j)_{j=1}^N\) in \(\Splx\):
%   \[
%   \begin{tikzcd}
%   {[n_{j-1}]}
%   &&{[m_j]}
%   \\
%   &{[k_j] \ar[lu, "\alpha_j"]} \ar[ru, "\beta_j"']
%   &
%   \end{tikzcd}
%   \]
%   where \(n_0\) and \(m_N\) are both defined to be \(n\) and the intersection
%   \(\im \beta_j \cap \im \alpha_{j+1}\) is \emph{non-empty} for every
%   \(1 \leq j < N\)---together with a collection of pairs
%   \[
%   \big((x_j, t_j), (y_j, s_j)\big)_{j=0}^N
%   \in \prod_{j=0}^N (X_{k_j} \times \splxtop^{k_j})
%   \times (X_{m_j} \times \splxtop^{m_j})
%   \]
%   for which one has the following relations:
%   \[
%   (x_j, (\alpha_j)_{*} t_j) = (\alpha_j^{*} y_{j-1}, s_{j-1})
%   \quad \text{ and } \quad
%   (x_j, (\beta_j)_{*} t_j) = (\beta_j^{*} y_j, s_j),
%   \]
%   and \((x_0, t_0) \coloneq (x, t)\), while \((y_N, s_N) \coloneq (y, s)\).

%   By hypothesis, since \(t\) and \(s\) are interior points of \(\splxtop^n\), then
%   it must be the case that both \(\alpha_1\) and \(\beta_N\) are
%   \emph{surjective}, so that \(k_1, k_N \geq n\). We proceed by induction on the
%   zig-zag length \(N\). If \(N = 1\), then the zig-zag is composed by two
%   \emph{surjective} maps \(\alpha, \beta: [k] \para [n]\). Since pushouts of
%   surjective maps exist in \(\Splx\) and are \emph{absolute}, if we consider the
%   pushout of the pair \((\alpha, \beta)\):
%   \[
%   \begin{tikzcd}
%   {[k]} \ar[r, "\alpha"]
%   \ar[d, "\beta"']
%   \ar[rd, phantom, "\ulcorner", very near end]
%   &{[n]} \ar[d, "\gamma"] \\
%   {[n]} \ar[r] &{[\ell]}
%   \end{tikzcd}
%   \]
%   the contravariant functor \(X\) maps this universal square to a
%   \emph{pullback} in \(\Set\):
%   \[
%   \begin{tikzcd}
%   X_k \ar[rd, "\lrcorner", very near start, phantom]
%   &X_n \ar[l, "\alpha^{*}"'] \\
%   X_n \ar[u, "\beta^{*}"]
%   &X_{\ell} \ar[u, "\gamma^{*}"'] \ar[l]
%   \end{tikzcd}
%   \]
%   Then since \(\gamma\) is a split epimorphism, so is \(\gamma^{*}\), which
%   proves the existence of an \(\ell\)-cell \(z \in X_{\ell}\) such that
%   \(\gamma^{*} z = x\). However, since \(x\) is non-degenerate and \(n\) is
%   minimal, this can only happen if \(\ell = n\) so that \(\alpha, \beta =
%   \Id_{[n]}\)---implying on \((x, t) = (y, s)\).

%   Assume, for the hypothesis of induction, that equality of the points is
%   stablished for some \(N-1 > 1\). We now work on the case \(N > 1\): our goal
%   will be to shorten the zig-zag sequence by one. Start by factoring the
%   morphism \(\beta_1\) as follows
%   \[
%   \begin{tikzcd}
%   {[k_1]} \ar[rr, "\beta_1"] \ar[rd, two heads, "\varepsilon"']
%   & &{[m_1]} \\
%   &{[m_1']} \ar[ur, tail, "\delta"'] &
%   \end{tikzcd}
%   \]
%   If we now consider the pushout of the pair of \emph{surjective} maps
%   \((\alpha_1, \varepsilon)\), the simplicial set takes this pushout to a
%   \emph{pullback} in \(\Set\) and from an analogous argument as done above, we
%   conclude that we must have \(\varepsilon = \Id_{[k_1]}\). Therefore the
%   factorization of \(\beta_1\) reduces solely to \(\beta_1 = \delta\), which is
%   a \emph{monomorphism}. Since \(\im \alpha_2 \cap \im \beta_1 \neq \emptyset\),
%   then by \cref{lem:splx-cat-pullback-mono-pushout-epi} we know that the
%   pullback of the pair \((\beta_1, \alpha_2)\) exists in \(\Splx\):
%   \begin{equation}\label{eq:pullback-unique-nd-pt-xi}
%   \begin{tikzcd}
%   {[k_1']} \ar[r, tail, "\theta"]
%   \ar[rd, phantom, very near start, "\lrcorner"]
%   \ar[d, "\eta"']
%   &{[k_2]} \ar[d, "\alpha_2"] \\
%   {[k_1]} \ar[r, tail, "\beta_1"'] &{[m_1]}
%   \end{tikzcd}
%   \end{equation}
%   Considering the simplex covariant functor \(\splxtop^{\bullet}\), we have a
%   corresponding square
%   \[
%   \begin{tikzcd}
%   \splxtop^{k_1'} \ar[r, tail, "\theta_{*}"]
%   \ar[d, "\eta_{*}"']
%   \ar[rd, phantom, very near start, "\lrcorner"]
%   &\splxtop^{k_2} \ar[d, "(\alpha_2)_{*}"] \\
%   \splxtop^{k_1} \ar[r, tail, "(\beta_1)_{*}"'] &\splxtop^{m_1}
%   \end{tikzcd}
%   \]
%   in \(\Top\), which is again a pullback. Therefore can find \(c \in X_{k_1'}\)
%   such that \(c = \eta^{*} x_1 = \theta^{*} y_1\), and take the \emph{unique}
%   \(w \in \splxtop^{k_1'}\) such that \(\eta_{*} w = t_1\) and
%   \(\theta_{*} w = s_1\).

%   Notice that by the pullback \cref{eq:pullback-unique-nd-pt-xi} one has
%   \[
%   \begin{tikzcd}
%   {[n]} & &{[m_1]} & &{[m_2]}
%   \\
%   &{[k_1]} \ar[ul, "\alpha_1"]
%   \ar[ru, tail, "\beta_1"]
%   &
%   &{[k_2]} \ar[ru, "\beta_2"']
%   \ar[lu, "\alpha_2"']
%   &
%   \\
%   & &{[k_1']} \ar[ul, "\eta"] \ar[ru, tail, "\theta"'] & &
%   \end{tikzcd}
%   \]
%   so together with the pair \((c, w) \in X_{k_1'} \times \splxtop^{k_1'}\) and
%   \([n] \overset{\alpha_1 \eta} \longleftarrow [k_1'] \overset{\beta_2
%     \theta}\longrightarrow [m_2]\) we can shorten the zig-zag length by one,
%   yielding the case \(N - 1\), which is true by hypothesis. Therefore
%   \((x, t) = (y, s)\) and the lemma follows.
% \end{itemize}
% \end{proof}

\begin{theorem}
\label{thm:geometric-realization-is-CW-complex}
Let \(X\) be a simplicial set. The geometric realization of \(X\) has a natural
structure of a CW-complex with exactly one closed \(n\)-cell
\(\widehat x: \splxtop^n \to |X|\) for each non-degenerate \(n\)-cell
\(x \in X_n\).
\end{theorem}

\begin{proof}
Let \((x, t), (y, s) \in \coprod_{x \in X_n^{\Nondeg}} \splxtop^n\) be any two
points.  We analyse the case where \(x \otimes t = y \otimes s\) are equivalent
points of \(|X|_n\) by means of
\cref{lem:unique-non-degenerate-point-geometric-realization}:
\begin{itemize}\setlength\itemsep{0em}
\item If both \(t, s \in \Int \splxtop^n\) then by the uniqueness of
  representatives we find that
  \[
  (x, t) = (y, s).
  \]

\item Now if for instance \(t \in \Bd \splxtop^n\), then there exits a
  unique \((z, r) \in X_k \times \Int \splxtop^k\) for some minimal \(k < n\)
  such that \(x \otimes t = z \otimes t\). Notice however that since \(k\) is
  minimal and unique with such property, then it must be the case that
  \(s \in \Bd \splxtop^n\)---therefore in this case one has both points
  \[
  (x, t), (y, s) \in \coprod_{x \in X_n^{\Nondeg}} \Bd \splxtop^n.
  \]
\end{itemize}
From this we can conclude that the following square is a pushout in \(\Top\):
\[
\begin{tikzcd}
\coprod_{x \in X_n^{\Nondeg}} \Bd \splxtop^n
\ar[dd, hook]
\ar[rr]
\ar[rdrd, "\ulcorner", phantom, very near end]
&&{|X|_{n-1}} \ar[dd, hook]
\\
&&
\\
\coprod_{x \in X_n^{\Nondeg}} \splxtop^n
\ar[rr, "(\widehat x)_{x \in X_n^{\Nondeg}}"']
&&{|X|_n}
\end{tikzcd}
\]
which proves that \(|X|\) is a \(CW\)-complex.
\end{proof}

\begin{definition}[Skeletal filtrations in \(\sSet\)]
\label{def:skeletal-filtration-sset}
Let \(X\) be a simplicial set. We define the \emph{skeletal filtration} of \(X\)
to be the collection \((\Sk_n X)_{n \in \N}\) of simplicial sets, where
\(\Sk_n X\) is the \emph{smallest subcomplex} of \(X\) containing every
\(k\)-cell \(x \in X_k\) for each \(0 \leq k \leq n\). Explicitly, the
collection of \(k\)-cells of \(\Sk_n X\) is
\[
(\Sk_n X)_k = \bigcup_{0 \leq j \leq k}
\big\{ x f \in X_k
\colon x \in X_j
\text{ and }
f \in \Hom_{\Splx}([k], [j])\big\}.
\]
Therefore for each \(n \in \N\) the simplicial set \(\Sk_n X\) is a
\emph{subcomplex} of \(\Sk_{n+1} X\).

From construction we see that
\[
X = \Colim_{n \in \N} \Sk_n X,
\]
which is, loosely, the \emph{union} of all \(\Sk_n X\).
\end{definition}

% \begin{remark}[On \cref{def:skeletal-filtration-sset}]
% \label{rem:on-the-skeletal-filtration-of-a-sset}
% Note by a subobject \(S \in \sSet\) of the simplicial set \(X\) we actually
% mean an isomorphism class of \emph{natural monomorphisms}---meaning, a class of
% natural transformations \(\eta: S \nat X\) for which each associated morphism
% \(\eta_{[n]}: S_n \mono X_n\) is a monomorphism in \(\Set\). We say that
% \((S, \eta)\) is \emph{equivalent} to another subobject \((B, \xi: B \nat X)\)
% of \(X\) if there exists a \emph{natural isomorphism} \(\sigma: S \isonat B\)
% such that, for any \([n] \in \Splx\), the diagram
% \[
% \begin{tikzcd}
% &X_k &
% \\
% S_k \ar[ru, tail, "\eta_{[k]}"]
% \ar[rr, "\dis", "\sigma_{[k]}"']
% &
% &B_k \ar[lu, tail, "\xi_{[k]}"']
% \end{tikzcd}
% \]
% commutes in \(\Set\).
% \end{remark}

% \begin{corollary}[Pushouts \& pullbacks in \(\sSet\)]
% \label{cor:pushout-and-pullback-of-simplicial-sets}
% Let \(X\), \(Y\) and \(Z\) be simplicial sets. We define the following:
% \begin{enumerate}[(a)]\setlength\itemsep{0em}
% \item Given morphisms of simplicial sets \(f: X \to Z\) and \(g: Y \to Z\), the
%   pullback of the pair \((f, g)\) to be the simplicial set \(P\) defined as
%   follows: for each \(n \in \N\), the set \(P_n\) is the pullback
%   \[
%   \begin{tikzcd}
%   P_n \ar[rd, phantom, "\lrcorner", very near start]
%   \ar[r] \ar[d]
%   &Y_n \ar[d, "g_n"] \\
%   X_n \ar[r, "f_n"']
%   &Z_n
%   \end{tikzcd}
%   \]
%   in the category of sets.
% \item Given morphisms of simplicial sets \(u: X \to Z\) and \(v: Y \to Z\), the
%   pushout of the pair \((u, v)\) to be the simplicial set \(Q\) defined as
%   follows: for each \(n \in \N\), the set \(Q_n\) is the pushout
%   \[
%   \begin{tikzcd}
%   Z_n \ar[rd, phantom, "\ulcorner", very near end]
%   \ar[r, "v_n"] \ar[d, "u_n"']
%   &Y_n \ar[d] \\
%   X_n \ar[r]
%   &Q_n
%   \end{tikzcd}
%   \]
%   in the category of sets.
% \end{enumerate}
% \end{corollary}

% \begin{proof}
% These limits all exist in \(\Set\), therefore each limit is well defined in
% \(\sSet\).
% \end{proof}

\begin{proposition}
\label{prop:filtration-attaching-cells}
Given a simplicial set \(X\), the square
\[
\begin{tikzcd}
\coprod_{x \in X_n^{\Nondeg}} \Bd \Delta^n
\ar[rd, phantom, very near end, "\ulcorner"]
\ar[r] \ar[d, tail]
&\Sk_{n-1} X \ar[d, tail]
\\
\coprod_{x \in X_n^{\Nondeg}} \Delta^n
\ar[r]
&\Sk_n X
\end{tikzcd}
\]
is a pushout in \(\sSet\).
\end{proposition}

% \begin{proof}
% Let \(P \in \sSet\) be the pushout of the given square, and define
% \(p: P \to \Sk_n X\) to be unique morphism of simplicial sets making the diagram
% \[
% \begin{tikzcd}
% \coprod_{x \in X_n^{\Nondeg}} \Bd \Delta^n
% \ar[rd, phantom, very near end, "\ulcorner"]
% \ar[r] \ar[d]
% &\Sk_{n-1} X
% \ar[d, tail]
% \ar[rdd, tail, bend left]
% &
% \\
% \coprod_{x \in X_n^{\Nondeg}} \Delta^n
% \ar[rrd, bend right]
% \ar[r]
% &P \ar[rd, dashed, "p"]
% &
% \\
% &&\Sk_n X
% \end{tikzcd}
% \]
% commute in \(\sSet\). We'll show that \(p\) is an isomorphism of simplicial
% sets:
% \begin{itemize}\setlength\itemsep{0em}
% \item (Epimorphism) Let \(x \in (\Sk X_n)_k\) be any \(k\)-cell, and let
%   \((y, \alpha: [k] \epi [m])\) be the unique representative pair of \(x\) with
%   \(y \in X_m\) being a non-degenerate \(m\)-cell such that
%   \(x = y \alpha\) and \(m \leq k\). If it is the case that \(m < n\), then
%   \(x\) is a degenerate \(k\)-cell and also \(y \in \Sk_{n-1} X\)---but then
%   \(x \in \Sk_{n-1} X\) via \(\alpha^{*}\), showing that \(x \in \im p_k\). For
%   the case where \(m = n\) we see that \(x\) is non-degenerate and therefore
%   \((x, \splx^n) \in \coprod_{x \in X_n^{\Nondeg}} \splx^n\), thus
%   \(x \in \im p_k\) by the commutativity of
%   \[
%   \begin{tikzcd}
%   {(x, \splx^n)}
%   \ar[r]
%   \ar[rd, "x"']
%   &P_k \ar[d, "p_k"] \\
%   &{(\Sk_n X)_{k}}
%   \end{tikzcd}
%   \]
%   where \(x: \splx^n \to (\Sk_n X)_k\) is the unique representative morphism of
%   \(x\).

% \item (Monomorphism) To prove that \(p\) is a monomorphism it suffices to show
%   the following two properties:
%   \begin{enumerate}[(i)]\setlength\itemsep{0em}
% \item Let \(x \in X_n^{\Nondeg}\) be any non-degenerate \(n\)-cell and consider
%   the canonical morphisms \(x: \splx^n \to \Sk_n X\) and
%   \(\Sk_{n-1} X \mono \Sk_n X\)---we'll show that the \emph{pullback} of this
%   pair of morphisms is \(\Bd \splx^n\).

%   Let \(\alpha \in \splx^n[k]\) be a \(k\)-cell such that the \(k\)-cell
%   \(x \alpha \in X_k\) is an element of \((\Sk_{n-1} X)_k\). Suppose, for the
%   sake of contradiction, that \(\alpha\) is an epimorphism, so that
%   \(n \leq k\). Since \(x \alpha\) a \(k\)-cell of the \((n-1)\)-skeleton, the
%   unique pair \((y, \beta: [k] \epi [m])\)---where \(y \in X_m\) is
%   non-degenerate with \(m < n\) and \(y \beta = x \alpha\)---is such that
%   \(y \in (\Sk_n X)_m\). Since \(\alpha\) and \(\beta\) are epimorphisms in
%   \(\Splx\), their pushout exists and is absolute:
%   \[
%   \begin{tikzcd}
%   {[k]} \ar[r, "\alpha", two heads]
%   \ar[d, two heads, "\beta"']
%   \ar[dr, phantom, very near end, "\ulcorner"]
%   &{[n]} \ar[d, "\delta"] \\
%   {[m]} \ar[r, "\gamma"']
%   &{[\ell]}
%   \end{tikzcd}
%   \]
%   Since \(X\) is covariant, the corresponding square is a pullback in \(\sSet\):
%   \[
%   \begin{tikzcd}
%   X_k
%   \ar[dr, phantom, very near start, "\lrcorner"]
%   &X_n \ar[l, "\alpha^{*}"', two heads] \\
%   X_m
%   \ar[u, two heads, "\beta^{*}"]
%   &X_{\ell} \ar[l, "\gamma^{*}"]
%   \ar[u, "\delta^{*}"']
%   \end{tikzcd}
%   \]
%   Using the fact that \(y \beta = x \alpha\), there exists \(z \in X_{\ell}\)
%   such that \(z \delta = x\) and \(z \gamma = y\). Since \(x\) is assumed to be
%   non-degenerate, and \(\delta\) is an epimorphism with \(\ell \leq n\) and
%   \(z \delta = x\), then it must be the case that \(\delta = \Id_{[n]}\) and
%   hence \(\ell = n\). Note however that \(m < n\) so that
%   \(\gamma: [m] \to [n]\) should not be able to be an epimorphism, which is a
%   contradiction.  From this it follows that \(\alpha\) isn't an epimorphism and
%   hence \(\alpha \in \Bd \splx^n[k]\) is a \(k\)-cell of the boundary. We thus
%   conclude that
%   \[
%   \begin{tikzcd}
%   \Bd \splx^n
%   \ar[rd, phantom, "\lrcorner", very near start]
%   \ar[r] \ar[d, tail]
%   &\Sk_{n-1} X \ar[d, tail]
%   \\
%   \splx^n \ar[r, "x"']
%   &\Sk_n X
%   \end{tikzcd}
%   \]

% %%%%%%%%%%%%%%%%%%

% \item Given any two \emph{distinct} non-degenerate \(n\)-cells \(x, y \in
%   X_n^{\Nondeg}\), let the square
%   \[
%   \begin{tikzcd}
%   Q \ar[r, "v"]
%   \ar[d, "w"']
%   &\splx^n \ar[d, "x"] \\
%   \splx^n \ar[r, "y"']
%   &\Sk_n X
%   \end{tikzcd}
%   \]
%   be a pullback in \(\sSet\). Our goal will be to show that \(v\) and \(w\) can
%   be factored through the monomorphism \(\Bd \splx^n \mono \splx^n\).

%   Consider a pair of \(k\)-cells \(\alpha, \beta \in \splx^n[k]\) such that
%   \(x \alpha = y \beta\). First, notice that if one of them was an identity map,
%   then \(k\) would coincide with \(n\) and it would follow that both \(\alpha\)
%   and \(\beta\) are identities, yielding \(x = y\), which contradicts the
%   assumption of \(x\) and \(y\) being distinct cells. We can therefore assume
%   that neither \(\alpha\) nor \(\beta\) are identities. Consider the
%   factorization
%   \[
%   \begin{tikzcd}
%   {[k]} \ar[rr, "\alpha"]
%   \ar[rd, two heads, "\alpha_{-}"']
%   &&{[n]} \\
%   &{[m_1]} \ar[ru, tail, "\alpha_{+}"'] &
%   \end{tikzcd}
%   \qquad
%   \qquad
%   \begin{tikzcd}
%   {[k]} \ar[rr, "\beta"]
%   \ar[rd, two heads, "\beta_{-}"']
%   & &{[n]}
%   \\
%   &{[m_2]} \ar[ru, tail, "\beta_{+}"'] &
%   \end{tikzcd}
%   \]
%   Since \(\alpha_{-}\) and \(\beta_{-}\) are epimorphisms, their pushout exists
%   and is absolute:
%   \begin{equation}\label{eq:pushout-proving-p-is-monic}
%   \begin{tikzcd}
%   {[k]} \ar[r, "\alpha_{-}", two heads]
%   \ar[d, two heads, "\beta_{-}"']
%   \ar[dr, phantom, very near end, "\ulcorner"]
%   &{[m_1]} \ar[d, "\delta"] \\
%   {[m_2]} \ar[r, "\gamma"']
%   &{[\ell]}
%   \end{tikzcd}
%   \end{equation}
%   Since \(X\) is covariant, the corresponding square is a pullback in \(\sSet\):
%   \[
%   \begin{tikzcd}
%   X_k
%   \ar[dr, phantom, very near start, "\lrcorner"]
%   &X_{m_1} \ar[l, "\alpha_{-}^{*}"', two heads] \\
%   X_{m_2}
%   \ar[u, two heads, "\beta_{-}^{*}"]
%   &X_{\ell} \ar[l, "\gamma^{*}"]
%   \ar[u, "\delta^{*}"']
%   \end{tikzcd}
%   \]
%   Therefore there exists \(z \in X_{\ell}\) for which
%   \(z \alpha_{-} \gamma = x\) while \(z \delta \beta_{-} = y\). Now, since both
%   \(x\) and \(y\) are non-degenerate cells, it follows that all maps in
%   \cref{eq:pushout-proving-p-is-monic} are identity morphisms. It then follows
%   that \(\alpha = \alpha_{+}\) and \(\beta = \beta_{+}\) are
%   monomorphisms. Since neither of them is an identity, it follows that
%   \(\alpha, \beta \in \Bd \splx^n\) are \(k\)-cells of the boundary. This shows
%   that \(v\) and \(w\) can be both factorized through
%   \(\Bd \splx^n \mono \splx^n\).
%   \end{enumerate}
% \end{itemize}
% \end{proof}

\section{Next Semester}

The following table displays the planned courses that the student will
participate in the following semester.

\begin{table}[h!]\label{tab:disc}
  \centering
  \caption{Types are divided in undergraduate course (G) and graduate (PG).}
  \begin{tabular}{ |c|c|c|c|c| }
    \hline
    Year and Semester & Code & Discipline & Type & Credits \\
    \hline
    \multirow{3}{*}{2023 / 1}
                   & CCM0328 & Iniciação à Pesquisa II            & G  & 12 \\
                   & MAT5761 & Introdução à Geometria Algébrica   & PG & 8 \\
                   & MAT6684 & Topologia Algébrica                & PG & 8 \\
    \hline
\end{tabular}
\end{table}

\end{document}


%%% Local Variables:
%%% mode: latex
%%% TeX-master: t
%%% End:

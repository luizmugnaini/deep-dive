\section{Projective Modules}

\subsection{Lifting Property For Projective Modules}

\begin{theorem}[Free modules have the lifting property]
\label{thm:free-modules-are-projective}
Let \(R\) be a ring and \(F\) be a free \(R\)-module. For every surjective
morphism of \(R\)-modules \(p: M \to N\) and morphism \(h: F \to N\), there
\emph{exists a unique morphism} \(\ell: F \to M\), called \emph{lifting of
  \(h\)}, such that the diagram
\[
\begin{tikzcd}
&F \ar[d, "h"] \ar[dl, bend right, dashed, "\ell"'] & & \\
M \ar[r, two heads, "p"'] &N \ar[r] &0
\end{tikzcd}
\]
commutes in \(\Mod{R}\).
\end{theorem}

\begin{proof}
Since \(F\) is free, let \(B \coloneq (b_j)_{j \in J}\) be a basis for
\(F\). From the surjectivity of \(p\), for every \(j \in J\) there exists
\(m_j \in M\) such that \(p(m_j) = h(b_j)\). From the free module universal
property there exists a unique \(\ell: F \to M\) such that \(\ell(b_j) = m_j\)
for each \(j \in J\). From construction we have
\(p \ell(b_j) = p(m_j) = h(b_j)\), therefore \(p \ell = h\), since \(B\)
generates \(F\), and the diagram commutes.
\end{proof}

\begin{remark}[Uniqueness of the lift]
\label{rem:uniqueness-of-lifting}
The lift \(\ell\) of \(h\) \emph{need not be unique} in case \(F\) isn't free!
\end{remark}

\begin{definition}[Projective module]
\label{def:projective-module}
Let \(R\) be a ring and \(P\) be an \(R\)-module. We say that \(P\) is a
\emph{projective \(R\)-module} if
\[
\Hom_{\Mod{R}}(P, -): \Mod{R} \longrightarrow \Ab
\]
is an \emph{exact covariant functor}.
\end{definition}


\begin{proposition}[Equivalences for projective modules]
\label{prop:equivalences-projective-module}
Let \(R\) be a ring and \(P\) be an \(R\)-module. The following properties are equivalent:
\begin{enumerate}[(a)]\setlength\itemsep{0em}
\item The module \(P\) is \emph{projective}.

\item For every exact sequence of \(R\)-modules \(M \overset{g}\epi N \to 0\)
  the sequence of abelian groups
  \[
  \begin{tikzcd}
  \Hom_{\Mod{R}}(P, M) \ar[r, "g_{*}", two heads]
  &\Hom_{\Mod{R}}(P, N) \ar[r] &0
  \end{tikzcd}
  \]
  is \emph{exact}. Equivalently, for every \(h: P \to N\), there \emph{exists a
    lifting} \(\ell: P \to N\) of \(h\)---\emph{not necessarily unique}---such
  that the diagram
  \[
  \begin{tikzcd}
  &P \ar[d, "h"] \ar[dl, bend right, "\ell"'] & & \\
  M \ar[r, two heads, "p"'] &N \ar[r] &0
  \end{tikzcd}
  \]
  is commutative in \(\Mod{R}\).

\item Every short exact sequence of \(R\)-modules of the form
  \[
  \begin{tikzcd}
  0 \ar[r] &L \ar[r, tail] &M \ar[r, two heads] &P \ar[r] &0
  \end{tikzcd}
  \]
  is a \emph{split} sequence.

\item The module \(P\) is a \emph{direct summand} of a \emph{free} \(R\)-module.

\item There exists a collection \((x_j)_{j \in J}\) of elements \(x_j \in P\),
  and a collection of associated morphisms of \(R\)-modules
  \((\phi_j: P \to R)_{j \in J}\)\footnote{The collection of pairs
    \((x_j, \phi_j: P \to R)_{j \in J}\) is sometimes referred to as the
    \emph{dual ``basis''}, but it should be noted right away that such family
    \emph{may not form a basis} for the module \(P\)---\emph{not every
      projective module is free}!} such that, for all \(x \in P\) we have:
\begin{itemize}\setlength\itemsep{0em}
\item The elements \(\phi_j(x) \in R\) are \emph{non-zero for only finitely many
  \(j \in J\)}.
\item The element \(x\) can be written as \(x = \sum_{j \in J} x_j \phi_j(x)\).
\end{itemize}
\end{enumerate}
\end{proposition}

\begin{proof}
From the definition, the equivalence of (a) and (b) is immediate. We prove the
following:
\begin{itemize}\setlength\itemsep{0em}
\item (b) \(\implies\) (c). Let \(g: M \epi P\) be the epimorphism depicted in
  the sequence. Consider the identity morphism \(\Id_P: P \to P\) and apply (b)
  to obtain \(\rho: P \to M\) such that
  \[
  \begin{tikzcd}
  &P \ar[d, "\Id_P"] \ar[ld, "\rho"', bend right] & \\
  M \ar[r, "g"', two heads] &P \ar[r] &0
  \end{tikzcd}
  \]
  is a commutative diagram of \(R\)-modules. Notice that \(g \rho = \Id_P\),
  therefore \(\rho\) is a section of \(g\)---thus the sequence splits.

\item (c) \(\implies\) (d). Via
  \cref{thm:any-module-is-quotient-of-free-module} let \(p: F \epi M\) be a
  surjective morphism of \(R\)-modules, where \(F\) is free. Thus we have a
  short exact sequence
  \[
  \begin{tikzcd}
  0 \ar[r] &\ker p \ar[r, hook] &F \ar[r, "p", two heads] &P \ar[r] &0.
  \end{tikzcd}
  \]
  By item (c) we find that the above sequence is split, therefore if \(\iota: P
  \mono F\) is a section of \(p\), then
  \[
  F = \ker p \oplus \im \iota \iso \ker p \oplus P.
  \]

\item (d) \(\implies\) (b). Let \(f: M \epi N\) be a surjective morphism of
  \(R\)-modules, and \(\psi: P \to N\) be any morphism. By item (d), let \(F\)
  be a free module with \(F \iso P \oplus P'\), where \(P'\) is the complement
  \(R\)-module of \(P\) with respect to \(F\)---also, let \(B\) be a basis of
  \(F\). Considering the natural projection \(\pi_P: F \epi P\), define a map
  \(\phi': F \to M\) as follows: given \(b \in B\), by the surjectivity of
  \(f\), there exists \(m \in M\) such that \(f(m) = \psi \pi_P(b)\)---we shall
  define \(\phi'(b) \coloneq m\). It is easily seen that \(\phi'\) is
  \(R\)-linear, and that \(f \phi' = \psi \pi_P\). Moreover, the surjectivity of
  \(f\) implies that \(\phi'\) is the unique morphism of modules with such
  property. Considering the natural inclusion \(\iota_P: P \emb F\)---which is a
  section of \(\pi_P\)---define \(\phi \coloneq \phi' \iota_P: P \to M\) and
  notice that
  \[
  f \phi
  = f (\phi' \iota_P)
  = (f \phi') \iota_P
  = (\psi \pi_P) \iota_P
  = \psi (\pi_P \iota_P)
  = \psi.
  \]
\end{itemize}
This finishes the proof of the equivalence of the items (a), (b), (c), and
(d). For item (e), we shall prove its equivalence with (d).
\begin{itemize}\setlength\itemsep{0em}
\item (d) \(\implies\) (e). Via item (d), there exists an indexing set \(J\) and
  an isomorphism \(\psi: \bigoplus_{j \in J} R \isoto P \oplus P'\), where
  \(P'\) is the complement of \(P\). If \(\pi_P: P \oplus P' \epi P\) denotes
  the canonical projection, define a collection \((x_j)_{j \in J}\) by
  \(x_j \coloneq \pi_P \psi(e_j)\)---where
  \(e_j \coloneq (\delta_{ij})_{i \in J}\). Let \(\iota_P: P \emb P \oplus P'\)
  be the canonical inclusion of \(P\), and
  \(\pi_j: \bigoplus_{j \in J} R \epi R\) be the canonical projection of the
  \(j\)-th coordinate. Define a collection of morphisms \((\phi_j)_{j \in J}\)
  by \(\phi_j \coloneq \pi_j \psi^{-1} \iota_P: P \to R\), so that---since
  \(\psi^{-1} \iota_P(x) \in \bigoplus_{j \in J} R\) has finitely many non-zero
  components---there are finitely many \(j \in J\) such that
  \(\pi_j \psi^{-1} \iota_P(x) \in R\) is non-zero. For the last condition of
  item (e), if \(x \in P\) is any element, we have
  \begin{align*}
  x &= \pi_P \iota_P(x)
  = \pi_P(\psi \psi^{-1}) \iota_P(x)
  = \pi_P \psi (\psi^{-1} \iota_P)(x)
  = \pi_P \psi (\phi_j(x))_{j \in J} \\
  &= \pi_P \psi \bigg( \sum_{j \in J} e_j \phi_j(x) \bigg)
  = \sum_{j \in J} \pi_P \psi(e_j \phi_j(x))
  = \sum_{j \in J} \pi_P \psi(e_j) \phi_j(x) \\
  &= \sum_{j \in J} x_j \phi_j(x).
  \end{align*}

\item (e) \(\implies\) (d). The collection \((\phi_j)_{j \in J}\) induces, by
  the universal property of the product, a unique morphism of \(R\)-modules
  \(\phi: P \to \prod_{j \in J} R\) mapping \(x \mapsto (\phi_j(x))_{j \in
    J}\). For any \(x \in P\) we know from hypothesis that the collection
  \((\phi_j(x))_{j \in J}\) has finitely many non-zero elements, therefore
  \(\im \phi \subseteq \bigoplus_{j \in J} R\). Then we may naturally restrict
  the codomain of \(\phi\), obtaining a morphism
  \(\phi: P \to \bigoplus_{j \in J} R\). Define a collection \((e_j)_{j \in J}\)
  where \(e_j \coloneq (\delta_{ij})_{i \in J}\), and consider an \(R\)-linear
  map \(\lambda: \bigoplus_{j \in J} R \to P\) defined by sending
  \(e_j \mapsto x_j\). By hypothesis, one has
  \[
  \lambda \phi(x)
  = \lambda(\phi_j(x))_{j \in J}
  = \sum_{j \in J} x_j \phi_j(x)
  = x,
  \]
  therefore \(\phi\) is a section of \(\lambda\), showing that \(\lambda\) is a
  split epimorphism. Thus \(P\) is a direct summand of the free module
  \(\bigoplus_{j \in J} R\).
\end{itemize}
\end{proof}

\begin{example}
\label{exp:free-mod-is-projective}
From \cref{thm:free-modules-are-projective} we find that every free module is a
projective module.
\end{example}

\begin{example}
\label{exp:idempotent-ring-projective}
Let \(R\) be a ring. If there exists an idempotent element \(e \in R\) (that is,
\(e^2 = e\)), then we have a decomposition \(R = e R \oplus (1 - e)
R\). Therefore \(e R\) is projective.

Indeed, given \(r \in R\) we can write it as \(r = e r + (1 - e) r\)---so that
\(R = eR + (1 - e)R\). Moreover, if \(a \in eR \cap (1 - e)R\), let \(r, r' \in
R\) be such that \(a = e r = (1 - e) r'\), however, notice that
\[
a = e r = e (e r) = e ((1 - e) r') = (e - e^2) r' = (e - e) r' = 0.
\]
Therefore \(e R \cap (1 - e) R = 0\), thus \(R = eR \oplus (1 - e)R\).

As an example of a \emph{projective module that isn't free}: if \(e\) is a
central in \(R\), then given any \(e r \in e R\) we have
\[
(e r) (1 - e) = e r - (e r) e = e r - e (r e) = e r - e (e r)
= er - er = 0.
\]
That is, the singleton \(\{e r\}\) is \emph{\(R\)-linearly
  dependent}---therefore \(eR\) does \emph{not} admit a basis.
\end{example}

\begin{proposition}
\label{prop:direct-sum-projective}
Let \((P_j)_{j \in J}\) be a family of \(R\)-modules. Then the module
\(P \coloneq \bigoplus_{j \in J} P_j\) is projective if and only if \(P_j\) is
projective for each \(j \in J\).
\end{proposition}

\begin{proof}
Suppose that \(P\) is projective, then we let \(F\) be a free module of which
\(P\) is a direct summand, say \(F = P \oplus P'\). If \(\sigma: J \to J\) is
any permutation, we know that \(\bigoplus_{j \in J} P_j \iso
\bigoplus_{j \in J} P_{\sigma(j)}\), therefore for all \(i \in J\) we have
\[
F = P \oplus P' = \Big( \bigoplus_{j \in J} P_j \Big) \oplus P'
\iso \Big( P_i \oplus \bigoplus_{j \in J \setminus i} P_j \Big) \oplus P'
= P_i \oplus \bigg(\Big(\bigoplus_{j \in J \setminus i} P_j \Big) \oplus P'\bigg)
\]
Since \(P_i\) is a direct summand of a free module, it is a projective module.

For the converse, suppose that \(P_j\) is projective for all \(j \in J\). Let
\(g: M \epi N\) be a surjective morphism of \(R\)-modules, and \(\phi: P \to N\)
be any morphism. Since \(P_j\) is projective, if \(\iota_j: P_j \emb P\) is the
canonical inclusion, there exists a morphism \(\psi_j: P_j \to M\) making the
diagram
\[
\begin{tikzcd}
&P_j \ar[d, "\phi \iota_j"] \ar[dl, bend right, "\psi_j"'] &\\
M \ar[r, two heads, "g"'] &N \ar[r] &0
\end{tikzcd}
\]
commute in \(\Mod{R}\). By the universal property of the coproduct \(P\), the
collection of morphisms \((\psi_j)_{j \in J}\) induce a unique morphism \(\psi:
P \to M\) such that \(\psi \iota_j = \psi_j\) for all \(j \in J\). To show that
\(g \psi = \phi\), notice that, given any \((x_j)_{j \in J} \in P\), one has
\[
g \psi(x_j)_{j \in J} = g \Big(\sum_{j \in J} \psi_j(x_j)\Big)
= \sum_{j \in J} g \psi_j(x_j)
= \sum_{j \in J} \phi \iota_j(x_j)
= \phi(x_j)_{j \in J}.
\]
Therefore, the following diagram commutes in \(\Mod{R}\):
\[
\begin{tikzcd}
&P \ar[d, "\phi"] \ar[dl, bend right, "\psi"'] &\\
M \ar[r, two heads, "g"'] &N \ar[r] &0
\end{tikzcd}
\]
which shows that \(P\) is a projective module.
\end{proof}

\begin{proposition}[Eilenberg's trick]
\label{prop:eilenberg-trick}
Let \(P\) be a \emph{projective} \(R\)-module. Then there exists a \emph{free
  \(R\)-module} \(F\) such that the direct sum \(P \oplus F\) is a \emph{free
  mocule}.
\end{proposition}

\begin{proof}
Since \(P\) is projective, there exists a free \(R\)-module \(F'\) and an
\(R\)-module \(Q\) such that \(F' = P \oplus Q\). Notice that
\[
(P \oplus Q) \oplus F'
= P \oplus (Q \oplus (P \oplus Q))
= P \oplus ((Q \oplus P) \oplus Q)
\]
is a direct sum of free modules, thus also free---however, \((Q \oplus P) \oplus
Q\) may not be free, since \(Q\) is only ensured to be projective. However, if
we define \(F \coloneq \bigoplus_{j \in \N} (Q \oplus P)\), then \(F\) is a
direct sum of projective modules and hence projective itself. Using the
fact that \(P \oplus (Q \oplus P) = (P \oplus Q) \oplus P\), we obtain
\[
P \oplus F
= P \oplus \bigoplus_{j \in \N} (Q \oplus P)
= \bigoplus_{j \in \N} (P \oplus Q),
\]
which is a direct sum of free modules---hence a projective module. Therefore
\(F\) satisfies the requirement of the statement.
\end{proof}

\begin{proposition}[Schanuel]
\label{prop:schanuel-lemma}
Let \(P\) and \(P'\) be projective \(R\)-modules. If there exists short exact
sequences
\[
\begin{tikzcd}
0 \ar[r] &K \ar[r, tail, "f"] &P \ar[r, two heads, "g"] &M \ar[r] &0
\\
0 \ar[r] &K' \ar[r, tail, "\phi"] &P' \ar[r, two heads, "\psi"] &M \ar[r] &0
\end{tikzcd}
\]
of \(R\)-modules, then there exists an isomorphism of \(R\)-modules
\[
K \oplus P' \iso K' \oplus P.
\]
\end{proposition}

\begin{proof}
Since \(P\) is projective, there exists a morphism of \(R\)-modules
\(\varepsilon: P \to P'\) such that the diagram
\[
\begin{tikzcd}
&P \ar[d, two heads, "g"] \ar[ld, bend right, "\varepsilon"'] &
\\
P' \ar[r, two heads, "\psi"'] &M \ar[r] &0
\end{tikzcd}
\]
commutes in \(\Mod{R}\). Notice that for any \(k \in K\) one has
\[
\psi \varepsilon f(k) = g f(k) = 0,
\]
therefore \(\varepsilon f(k) \in \ker \psi\). Since
\(\ker \psi \subseteq \im \phi\), there must exist \(k' \in K'\) such that
\(\phi(k') = \varepsilon f(k)\)---which needs to be unique with such image since
\(\phi\) is injective. Define \(\delta: K \to K'\) to be the map
\(k \mapsto k'\), where \(k' \in K'\) is as described above. So far, we have the
following commutative diagram in \(\Mod{R}\)
\[
\begin{tikzcd}
0 \ar[r]
&K \ar[r, tail, "f"] \ar[d, "\delta"']
&P \ar[r, two heads, "g"] \ar[d, "\varepsilon"]
&M \ar[r] \ar[d, equals]
&0
\\
0 \ar[r] &K' \ar[r, tail, "\phi"'] &P' \ar[r, two heads, "\psi"'] &M \ar[r] &0
\end{tikzcd}
\]

We now define a map \(\omega: K \to K' \oplus P\) given by
\(k \mapsto (\delta(k), f(k))\), and \(\gamma: K' \oplus P \to P'\) by
\((k', p) \mapsto \varepsilon(p) - \phi(k')\). Notice that both maps are clearly
\(R\)-linear and from definition \((k', p) \in \ker \gamma\), that is,
\(\varepsilon(p) = \phi(k')\) if and only if there exists a common \(k \in K\)
for which \(f(k) = p\) and \(\delta(k) = k'\)---therefore
\(\ker \gamma = \im \omega\). To show that \(\omega\) is injective, let
\(k \in \ker \omega\) be any element, then from definition
\((\delta(k), f(k)) = (0, 0)\), but since \(f\) is injective, then
\(k = 0\)---thus \(\ker \omega = 0\) and \(\omega\) is injective. For the
surjectivity of \(\gamma\), if \(p' \in P'\), let \(p \in P\) be such that
\(g(p) = \psi(p')\)---which exists because \(g\) is surjective. Then by the
commutativity of the right square one has
\[
\psi \varepsilon(p) = g(p) = \psi(p'),
\]
that is, \(\varepsilon(p) - p' \in \ker \psi\) and thus exists \(k' \in K'\) for
which \(\phi(k') = \varepsilon(p) - p'\). This shows that
\[
\gamma(k', p') = \varepsilon(p) - \phi(k') = p',
\]
so that \(\gamma\) is surjective.

We've just shown that the sequence
\[
\begin{tikzcd}
0 \ar[r]
&K \ar[r, tail, "\omega"]
&K' \oplus P \ar[r, two heads, "\gamma"]
&P' \ar[r]
&0
\end{tikzcd}
\]
is short exact. Since \(P'\) is projective, the sequence is also split and
therefore there exists an isomorphism of \(R\)-modules
\(K' \oplus P \iso K \oplus P'\).
\end{proof}

\subsection{Finitely Generated Projective Modules}

\begin{proposition}
\label{prop:fg-projective-then-dual-module-is-projective}
Let \(P\) be a finitely generated projective right-\(R\)-module (or left). Then
the dual module \(P^{*}\) is a projective left-\(R\)-module (or right).
\end{proposition}

\begin{proof}
If \(P\) is finitely generated and projective we can choose a finitely generated
free right-\(R\)-module \(F\) of which \(P\) is a direct summand---say, \(F = P
\oplus P'\). Since \(F\) is finitely generated, there exists \(n \in \Z_{> 0}\)
such that \(F \iso \bigoplus_{j=1}^n R\). We know that there exists a natural
isomorphism of \emph{left}-\(R\)-modules
\[
F^{*} = \Hom_{\rMod{R}}(P \oplus P', R)
\iso \Hom_{\rMod{R}}(P, R) \oplus \Hom_{\rMod{R}}(P', R)
= P^{*} \oplus P'^{*}.
\]
Moreover, we can rewrite the morphism set using the fact that \(F\) is finitely
generated:
\[
F^{*} \iso \Hom_{\rMod{R}}\Big(\bigoplus_{j=1}^n R, R\Big)
\iso \bigoplus_{j=1}^n \Hom_{\rMod{R}}(R, R)
\iso \bigoplus_{j=1}^n R,
\]
therefore \(F^{*}\) is a finitely generated free left-\(R\)-module. Since
\(P^{*}\) is a direct summand of \(F^{*}\), then \(P^{*}\) is a finitely
generated \emph{projective} left-\(R\)-module.
\end{proof}

\begin{proposition}
\label{prop:projective-module-evaluation-morphism-is-injective}
Let \(P\) be a right-\(R\)-module (or left). The \emph{double dual} \(P^{* *}\)
is a \emph{right}-\(R\)-module (or left), and the natural evaluation map
\[
\eval: P \longrightarrow P^{* *}
\]
sending \(x \mapsto \eval_x: P^{*} \to R\)---where \(\eval_x(f) = f(x)\)---is an
\emph{injective morphism of right-\(R\)-modules}.
\end{proposition}

\begin{proof}
Let \((x_j, \phi_j)_{j \in J}\) be the dual ``basis'' of \(P\), where
\(x_j \in P\) and \(\phi_j \in P^{*}\). If \(x \in \ker \eval\), then
\(f(x) = 0\) for all \(f \in P^{*}\)---thus in particular \(x = \sum_{j \in J}
x_j \phi_j(x) = 0\). This shows that \(\ker \eval = 0\), thus the evaluation
morphism is injective.
\end{proof}

\begin{proposition}
\label{prop:properties-of-fg-proj-modules-and-duals}
Let \(P\) be a finitely generated projective right-\(R\)-module (or left), with
a dual ``basis'' \((x_j, \phi_j)_{j=1}^n\). The following properties hold:
\begin{enumerate}[(a)]\setlength\itemsep{0em}
\item The finite collection \((\phi_j, \eval_{x_j})_{j=1}^n\) forms a \emph{dual
  ``basis''} for the dual module \(P^{*}\).
\item The dual \(P^{*}\) is a \emph{finitely generated projective
    left-\(R\)-module} (or right), with a generating set \((\phi_j)_{j=1}^n\).
\item The double dual \(P^{* *}\) is a \emph{finitely generated projective
    right-\(R\)-module} (or left), with a generating set
  \((\eval_{x_j})_{j=1}^n\).
\item The natural evaluation morphism \(\eval: P \mono P^{* *}\) is an
  \emph{isomorphism} of right-\(R\)-modules (or left).
\end{enumerate}
\end{proposition}

\begin{proof}
\begin{enumerate}[(a)]\setlength\itemsep{0em}
\item Let \(f \in P^{*}\) be any functional, then given any \(x \in P\) one has
  \[
  \sum_{j=1}^n \eval_{x_j}(f) \phi_j(x)
  = \sum_{j=1}^{n} f(x_j \phi_j(x))
  = f\Big( \sum_{j=1}^{n} x_j \phi_j(x) \Big)
  = f(x),
  \]
  therefore \(f = \sum_{j=1}^n \eval_{x_j}(f) \phi_j\)---which proves that
  \((\phi_j, \eval_{x_j})_j\) is a dual basis for \(P^{*}\).

\item It is immediate from the last item's proof that \((\phi_j)_{j=1}^n\) is a
  generating set for \(P^{*}\). Moreover, since \(P^{*}\) admits a dual basis it
  follows that \(P^{*}\) is projective.

\item Define a map \(\eval^{*}: P^{*} \to (P^{* *})^{*}\) given by \(f \mapsto
  \eval_f^{*} \in \Hom_{\rMod{R}}(P^{* *}, R)\), where \(\eval_f^{*}(\Phi)
  \coloneq \Phi(f)\) for any \(\Phi \in \Hom_{\lMod{R}}(P^{*}, R)\). Notice
  that, for any \(f \in P^{*}\) one has
  \[
  \sum_{j=1}^n \eval_{x_j}(f) \eval_{\phi_j}^{*}(\Phi)
  = \sum_{j=1}^n f(x_j) \Phi(\phi_j)
  = \sum_{j=1}^n \Phi(f(x_j) \phi_j)
  = \Phi\Big( \sum_{j=1}^n f(x_j) \phi_j \Big)
  = \Phi(f),
  \]
  that is, \(\Phi = \sum_{j=1}^n \eval_{x_j} \cdot \eval_{\phi_j}^{*}
  (\Phi)\). Therefore \((\eval_{x_j})_{j=1}^n\) is a generating set for
  \(P^{* *}\), and \((\eval_{\phi_j}^{*}, \eval_{x_j})_{j=1}^n\) is a dual
  ``basis'' for \(P^{* *}\). Therefore \(P^{* *}\) is a finitely generated
  projective right-\(R\)-module.

\item Given an element \(\Phi \in P^{* *}\), we can rewrite it as
  \(\Phi = \sum_{j=1}^n \eval_{x_j} \cdot \eval_{\phi_j}^{*}(\Phi)\), therefore,
  if we take \(\sum_{j=1}^n x_j \Phi(\phi_j) \in P\), one obtains
  \begin{align*}
  \eval\Big(\sum_{j=1}^n x_j \Phi(\phi_j)\Big)
  &= \sum_{j=1}^n \eval(x_j \Phi(\phi_j))
  = \sum_{j=1}^n \eval(x_j) \Phi(\phi_j) \\
  &= \sum_{j=1}^n \eval_{x_j} \cdot \Phi(\phi_j)
  = \sum_{j=1}^n \eval_{x_j} \cdot \eval_{\phi_j}^{*}(\Phi) \\
  &= \Phi.
  \end{align*}
  This shows that \(\eval\) is also a surjective morphism. Therefore \(\eval\)
  is an isomorphism of right-\(R\)-modules
  \[
  P \iso P^{* *}.
  \]
\end{enumerate}
\end{proof}

\section{Injective Modules}

\subsection{Lifting Property for Injective Modules}

\begin{definition}[Injective module]
\label{def:injective-modules}
Let \(R\) be a ring, and \(E\) be an \(R\)-module. We say that \(E\) is an
\emph{injective \(R\)-module} if
\[
\Hom_{\Mod{R}}(-, E): \Mod{R}^{\op} \longrightarrow \Ab
\]
is an \emph{exact contravariant functor}.
\end{definition}

\begin{proposition}
\label{prop:injective-modules-equivalences}
Let \(E\) be an \(R\)-module. The following are equivalent properties:
\begin{enumerate}[(a)]\setlength\itemsep{0em}
\item The module \(E\) is \emph{injective}.

\item Given any exact sequence of \(R\)-modules \(0 \to M \overset{f}\mono N\),
  the sequence
  \[
  \begin{tikzcd}
  \Hom_{\Mod{R}}(N, E) \ar[r, two heads, "f^{*}"] &\Hom_{\Mod{R}}(M, E) \ar[r] &0
  \end{tikzcd}
  \]
  is an \emph{exact sequence of abelian groups}

\item For any injective morphism of \(R\)-modules \(f: M \mono N\), and morphism
  \(\phi: M \to E\), there \emph{exists} a morphism \(\psi: N \to E\)---we do
  not require uniqueness---for which the diagram
  \[
  \begin{tikzcd}
  &E & \\
  0 \ar[r] &M \ar[r, tail, "f"'] \ar[u, "\phi"] &N \ar[lu, bend right, "\psi"']
  \end{tikzcd}
  \]
  is \emph{commutative} in \(\Mod{R}\).

\item Any
  short exact sequence of \(R\)-modules of the form
  \[
  \begin{tikzcd}
  0 \ar[r] &E \ar[r, tail] &A \ar[r, two heads] &B \ar[r] &0
  \end{tikzcd}
  \]
  is \emph{split}.
\end{enumerate}
\end{proposition}

\begin{proof}
The equivalence between (a) and (b) comes exactly from the definition of
what an injective module is. We first prove the equivalence between (b) and (c),
then the equivalence between (c) and (d):
\begin{itemize}\setlength\itemsep{0em}
\item (b) \(\implies\) (c). If we let \(0 \to M \overset{f}\mono N\) be any
  exact sequence of \(R\)-modules, then by (b) we know that
  \(f^{*}: \Hom_{\Mod{R}}(N, E) \epi \Hom_{\Mod{R}}(M, E)\) is a surjective
  morphism. Therefore, if we are given any \(\phi \in \Hom_{\Mod{R}}(M, E)\),
  one can choose a \(\psi \in \Hom_{\Mod{R}}(N, E)\) such that
  \(f^{*}(\psi) = \psi f = \phi\).

\item (c) \(\implies\) (b). Let \(0 \to M \overset{f}\mono N\) be any exact
  sequence of \(R\)-modules, then if we are given any morphism
  \(\phi: M \to E\), we can find \(\psi: N \to E\) such that \(\phi = f \psi\),
  but \(f \psi = f^{*}(\psi)\), therefore we've just shown that the map
  \(f^{*}\) is surjective.

\item (c) \(\implies\) (d). Let \(0 \to E \overset f \mono A \epi B \to 0\) be
  a short exact sequence of \(R\)-modules. If we consider the identity morphism
  \(\Id_E\), by item (c) we are able to find a morphism \(\psi: A \to E\) such
  that \(\psi f = \Id_E\)---therefore \(f\) is a split monomorphism and the
  sequence splits.

\item (d) \(\implies\) (c). Let \(0 \to L \overset f \mono M\) be an exact
  sequence of \(R\)-modules and let \(\phi: L \to E\) be a morphism. We may
  define an \(R\)-module
  \[
  X \coloneq \frac{E \oplus M}{\{(\phi(x), -f(x)) \colon x \in L\}},
  \]
  and---given the natural projection \(\pi: E \oplus M \epi X\) and the
  canonical inclusion morphisms \(\iota_E: E \emb E \oplus M\) and
  \(\iota_M: M \emb E \oplus M\)---we define \(\ell_E: E \to X\) and
  \(\ell_M: M \to X\) by \(\ell_E \coloneq \pi \iota_E\) and
  \(\ell_M \coloneq \pi \iota_M\). By \cref{prop:pushout-modules}, we know that
  \((X, \ell_E, \ell_M)\) is the pushout of the pair \((f, \phi)\):
  \[
  \begin{tikzcd}
  L \ar[r, tail, "f"] \ar[d, swap, "\phi"]
  \ar[rd, phantom, very near end, "\ulcorner"]
  &M \ar[d, "\ell_M"] \\
  E \ar[r, "\ell_{E}"'] &X
  \end{tikzcd}
  \]
  Let \(e \in \ker \ell_{E}\) be any element, then
  \[
  \ell_E(e) = [e, 0] \in \{[\phi(x), -f(x)] \colon x \in L\}.
  \]
  Let \(x_e \in L\) be such that \((e, 0) = (\phi(x_e), -f(x_e))\), then
  \(x_e \in \ker f\)---but since \(f\) is injective, it follows that
  \(x_e = 0\). Therefore \(e = \phi(0) = 0\).

  From this we can consider the following short exact sequence
  \[
  \begin{tikzcd}
  0 \ar[r] &E \ar[r, tail, "\ell_E"]
  &X \ar[r, two heads]
  &\coker \ell_E \ar[r] &0
  \end{tikzcd}
  \]
  which by item (d) is split---therefore there exists a retract \(r: X \to E\)
  of \(\ell_E\), that is, \(r \ell_E = \Id_E\). From this we can create a
  morphism \(\psi: M \to E\) given by \(\psi \coloneq r \ell_M\)---we'll show
  that this map satisfies the condition for item (c). Let \(x \in L\) be any
  element and notice that
  \[
  \psi f(x)
  = (r \ell_M) f(x)
  = r[0, f(x)]
  = r[\phi(x), 0]
  = r(\ell_E(\phi(x)))
  = (r \ell_E)\phi(x)
  = \phi(x),
  \]
  therefore \(\psi f = \phi\) as needed, which proves that \(E\) is injective.
\end{itemize}
\end{proof}

\begin{corollary}
\label{cor:injective-submodule-is-direct-summand}
Let \(M\) be an \(R\)-module, and \(E\) be a submodule of \(M\). If \(E\) is
\emph{injective}, then it is a \emph{direct summand} of \(M\).
\end{corollary}

\begin{proof}
Indeed, since the short exact sequence
\[
\begin{tikzcd}
0 \ar[r] &E \ar[r, hook, "\iota"] &M \ar[r, two heads, "\pi"]
&M/E \ar[l, bend left=50, "\rho" description] \ar[r] &0
\end{tikzcd}
\]
is split, where \(\rho\) is a section of \(\pi\). Therefore
\[
M = \im \iota \oplus \im \rho = E \oplus \im \rho.
\]
\end{proof}

\begin{proposition}
\label{prop:direct-summand-of-injective-is-injective}
Any \emph{direct summand of an injective} \(R\)-module is an \emph{injective}
\(R\)-module
\end{proposition}

\begin{proof}
Let \(E = X \oplus Y\), and consider a morphism \(\phi: L \to X\), and an
injective morphism \(f: L \mono M\). If \(\iota_X: X \emb E\), we can define a
morphism \(\phi': L \to E\) given by \(\phi' \coloneq \iota_X \phi\). Since
\(E\) is injective, there exists \(\psi': M \to E\) such that
\(\psi' f = \phi'\). However, taking the canonical projection
\(\pi_X: E \epi X\) we can define \(\psi: M \to X\) to be given by
\(\psi \coloneq \pi_X \psi'\)---therefore the diagram
\[
\begin{tikzcd}
&X &
\\
0 \ar[r] &L \ar[u, "\phi"] \ar[r, tail, "f"'] &M \ar[lu, bend right, "\psi"']
\end{tikzcd}
\]
is commutative in \(\Mod{R}\), hence \(X\) is injective.
\end{proof}

\begin{proposition}
\label{prop:module-product-injective-iff-collection-injective}
Let \((E_j)_{j \in J}\) be a collection of \(R\)-modules. Then the product
\(\prod_{j \in J} E_j\) is an injective \(R\)-module if and only if \(E_j\) is
injective for each \(j \in J\).
\end{proposition}

\begin{proof}
(\(\implies\)) We know that any permutation \(\sigma: J \isoto J\) is such that
\(\prod_{j \in J} E_j \iso \prod_{j \in J} E_{\sigma(j)}\) in \(\Mod{R}\).
Therefore, for any \(i \in J\) we have a natural isomorphism
\[
\prod_{j \in J} E_j \iso E_i \times \Big( \prod_{j \in J \setminus i} E_j \Big),
\]
which by \cref{prop:direct-summand-of-injective-is-injective} implies in the
injectivity of \(E_i\).

(\(\impliedby\)) Let \(E_j\) be injective for each \(j \in J\). Let
\(f: L \mono M\) be any injective morphism of \(R\)-modules and consider a
morphism \(\phi: L \to E\). Since each \(E_j\) is injective, define
\((\psi_j: M \to E_j)_{j \in J}\) to be the collection of morphisms such that
the diagram
\[
\begin{tikzcd}
&\prod_{j \in J} E_j \ar[r, two heads, "\pi_j"] & E_j
\\
0 \ar[r] &L \ar[u, "\phi"] \ar[r, tail, "f"] &M \ar[u, "\psi_j"']
\end{tikzcd}
\]
commutes for all \(j \in J\)---where \(\pi_j\) is the natural \(j\)-th
projection. From the universal property of products, the collection
\((\psi_j)_{j \in J}\) defines a unique morphism \(\psi: M \to E\) such that
\(\pi_j \psi = \psi_j\) for each \(j \in J\). Note that \(\psi f\) and \(\phi\)
are equal if and only if each of their projections match, but since
\[
\pi_j(\psi f) = (\pi_j \psi) f = \psi_j f = \pi_j \phi,
\]
then \(\psi f = \phi\) and the diagram
\[
\begin{tikzcd}
&\prod_{j \in J} E_j &
\\
0 \ar[r] &L \ar[u, "\phi"] \ar[r, tail, "f"'] &M \ar[lu, bend right, "\psi"']
\end{tikzcd}
\]
commutes in \(\Mod{R}\)---therefore \(\prod_{j \in J} E_j\) is an injective
module.
\end{proof}

\begin{remark}
\label{rem:direct-sum-injective-modules-not-always-injective}
Contrary to the behaviour of projective modules, \emph{the direct sum of a
  collection of injective modules need not be injective}.
\end{remark}

\begin{lemma}
\label{lemma:field-of-fractions-lifting-property}
Let \(R\) be an integral domain and consider its field of fractions
\(\Frac(R)\) as an \emph{\(R\)-module}. The following holds:
\begin{enumerate}[(a)]\setlength\itemsep{0em}
\item Let \(I \subseteq \Frac(R)\) be an \(R\)-submodule of the \(R\)-module
  \(\Frac(R)\), and let \(\phi: I \to \Frac(R)\) be a \emph{morphism of
    \(R\)-modules}. Then there exists \(q \in \Frac(R)\) such that
  \(\phi(y) = q y\) for all \(y \in I\).

\item Let \(\ideal a\) be a submodule (ideal) of \(R\). Then any morphism of
  \(R\)-modules \(\phi: \ideal a \to \Frac(R)\) can be extended to a morphism
  \(\overline{\phi}: R \to \Frac(R)\).
\end{enumerate}
\end{lemma}

\begin{proof}
\begin{enumerate}[(a)]\setlength\itemsep{0em}
\item If \(I = 0\), then \(\phi = 0\) and any \(q \in \Frac(R)\) satisfies the
  requirement. On the contrary, suppose \(I\) is non-zero, and fix any non-zero
  \(b \coloneq w/z \in I\). If \(a \coloneq u/v \in \Frac(R)\) is any other
  element, let \(r \coloneq z v\) so that
  \[
  b r = b (z v) = (b z) v = w v \in R
  \quad\text{ and }\quad
  a r = a(z v) = a(v z) = (a v) z = u z \in R,
  \]
  where we used the commutativity of \(R\) in order to have \(a r \in
  R\). Therefore one has
  \begin{gather*}
  \phi((b a) r) = \phi(b (a r)) = \phi(b) \cdot (a r), \\
  \phi((b a) r) = \phi((a b) r) = \phi(a (b r)) = \phi(a) \cdot (b r).
  \end{gather*}
  Thus \(\phi(a) b r = \phi(b) a r\)---multiplying this equality by
  \((b r)^{-1} \in \Frac(R)\) we obtain
  \[
  \phi(a)
  = \phi(b) \cdot (a r) (b r)^{-1}
  = \phi(b) \cdot (a b^{-1})
  = \phi(b) \cdot (b^{-1} a)
  = (\phi(b) \cdot b^{-1}) a.
  \]
  Therefore we may define \(q \coloneq \frac{\phi(b)}{b} \in \Frac(R)\), so that
  \(\phi(y) = q y\).

\item Given any morphism of \(R\)-modules \(\phi: \ideal a \to \Frac(R)\), let
  \(q \in \Frac(R)\) be such that \(\phi(y) = q y\)---which exists because of
  the last items's result. Define \(\overline{\phi}: R \to \Frac(R)\) to be an
  \(R\)-module morphism given by \(\overline{\phi}(1) \coloneq q\)---this
  completely defines \(\psi\) since
  \(\overline{\phi}(r) = \overline{\phi}(1 \cdot r) = \overline{\phi}(1) r = q
  r\). Therefore the diagram
  \[
  \begin{tikzcd}
  &\Frac{(R)} &
  \\
  0 \ar[r] &\ideal a \ar[r, hook] \ar[u, "\phi"]
  &R \ar[lu, bend right, "\overline{\phi}"']
  \end{tikzcd}
  \]
\end{enumerate}
\end{proof}

\begin{theorem}[Baer's criterion]
\label{thm:baer-criterion}
Let \(E\) be a right-\(R\)-module (or left). Then \(E\) is \emph{injective} if
and only if for every \emph{right ideal} (or left) \(\ideal a\) of \(R\), and
morphism of right-\(R\)-modules (or left) \(\phi: \ideal a \to E\), there exists
an extension \(\overline{\phi}: R \to E\) of \(\phi\)---that is, the diagram
\[
\begin{tikzcd}
&E & \\
0 \ar[r] &\ideal a \ar[r, hook] \ar[u, "\phi"]
&R \ar[lu, bend right, "\overline{\phi}"']
\end{tikzcd}
\]
is commutative in \(\rMod{R}\) (or \(\lMod{R}\)).
\end{theorem}

\begin{proof}
(\(\implies\)) If \(E\) is injective, then the existence of the extension
follows immediatly.

(\(\impliedby\)) Suppose the latter condition is satisfied. Consider morphisms
\(\phi: L \to E\), and \(f: L \mono M\) injective, in \(\rMod{R}\). Define
\(\mathcal{L}\) to be the collection of all pairs \((L', \psi': L' \to E)\) such
that \(f(L) \subseteq L' \subseteq M\), where \(L'\) is a submodule of \(M\),
and \(\psi' f = \phi\). Notice that since \(f\) is injective we may define
\(\psi' \coloneq \phi \circ (f|_{f(L)})^{-1}\) and take \(L' \coloneq f(L)\),
proving that \(\mathcal{L}\) is a non-empty set.

We define a partial order \(\preceq\) on \(\mathcal{L}\) as follows:
\((L', \psi') \preceq (L'', \psi'')\) if and only if \(L' \subseteq L''\) and
\(\psi''|_{L'} = \psi'\). Define an ascending chain (with respect to
\(\preceq\)) of pairs \((L_j', \psi_j')_{j \in J}\) of elements of
\(\mathcal{L}\), and let \(L'_{\text{M}} \coloneq \bigcup_{j \in J} L'_j\). From
construction, one has that \(f(L) \subseteq L'_{\text{M}} \subseteq M\), where
\(L'_{\text{M}}\) is a submodule of \(M\). Define a morphism
\(\psi'_{\text{M}}: L'_{\text{M}} \to E\) as follows: if \(x \in L_j'\), define
\(\psi'_{\text{M}}(x) \coloneq \psi_j'(x)\)---which is well defined because
given \((L_j', \psi_j') \preceq (L_i', \psi_i')\), one has
\(\psi_i|_{L_j} = \psi_j\). From its construction the pair
\((L'_{\text{M}}, \psi'_{\text{M}})\) is a maximal element of the chain
\((L_j', \psi_j')\).

Since \(\mathcal{L}\) is non-empty and every chain of elements has a maximal
element, we can use Zorn's lemma to conclude that \(\mathcal{L}\) admits a
maximal element \((L_0, \psi_0)\). If, for the sake of contradiction, there
exists an element \(x \in M \setminus L_0\), define a right-ideal
\[
\ideal a \coloneq \{r \in R \colon x r \in L_0\} \subseteq R,
\]
and a morphism \(\lambda: \ideal a \to E\) by
\(r \mapsto \psi_0(x r) = \psi_0(x) r\). From hypothesis, \(\lambda\) admits an
extension \(\overline{\lambda}: R \to E\) making the diagram
\[
\begin{tikzcd}
&E &
\\
0 \ar[r] &\ideal a \ar[u, "\lambda"] \ar[r, hook]
&R \ar[lu, "\overline{\lambda}"', bend right]
\end{tikzcd}
\]
commute in \(\rMod{R}\). Define a map \(\psi_1: L_0 + x R \to E\) by
\(\psi_1(y + x r) \coloneq \psi_0(y) + \overline{\lambda}(r)\), which is
certainly a morphism of right-\(R\)-modules. To prove that \(\psi_1\) is well
defined, if \(y + x r = y' + x r' \in L_0 + x R\) then
\(y - y' = x(r' - r) \in x R\), therefore \(y - y' \in L_0\), while
\(x (r' - r) \in x R\). This implies in \(r' - r \in \ideal a\), therefore both
\(\psi_0(y - y')\) and \(\lambda(r' - r)\). Finally, we see that
\[
\psi_0(y - y')
= \psi_0(x (r' - r))
= \lambda(r' - r)
= \overline{\lambda}(r' - r)
= \overline{\lambda}(r') - \overline{\lambda}(r),
\]
therefore \(\psi_1(y + x r) = \psi_1(y' + x r')\), thus \(\psi_1\) is indeed
well defined morphism. From construction, we have \(L_0 \subseteq L_0 + x R\)
and \(\psi_1|_{L_0} = \psi_0\), which implies in
\[
(L_0, \psi_1) \preceq (L_0 + x R, \psi_1),
\]
contradicting the maximality of \((L_0, \psi_1)\). Therefore it must be the case
that \(L_0 = M\), which implies in \(\psi_0: M \to E\) being such that
\(\psi_0 f = \phi\). We conclude that the morphism of right-\(R\)-modules
\(\psi_0\) makes the diagram
\[
\begin{tikzcd}
&E &
\\
0 \ar[r] &L \ar[u, "\phi"] \ar[r, tail, "f"']
&M \ar[lu, "\psi_0"', bend right]
\end{tikzcd}
\]
commutative in \(\rMod{R}\), showing that \(E\) is an injective module.
\end{proof}

\begin{corollary}
\label{cor:field-of-fractions-injective}
Let \(R\) be an integral domain. Then the following holds:
\begin{enumerate}[(a)]\setlength\itemsep{0em}
\item The field of fractions \(\Frac(R)\) is an \emph{injective \(R\)-module}

\item Every \emph{\(\Frac(R)\)-module} is an \emph{injective \(R\)-module}.

\item Every \emph{vector space} is \emph{injective}.
\end{enumerate}
\end{corollary}

\begin{proof}
\begin{enumerate}[(a)]\setlength\itemsep{0em}
\item Notice that by \cref{lemma:field-of-fractions-lifting-property} item (b)
  we find that \(\Frac(R)\) satisfies the lifting property of injective modules
  described by Baer's criterion, therefore \(\Frac(R)\) is an injective
  \(R\)-module.

\item Let \(M\) be a \(\Frac(R)\)-module, and \(\ideal a \subseteq R\) be a
  non-zero ideal of \(R\)---if \(\ideal a\) where zero, then the extension
  property would be satisfies right away. Since \(M\) is a module over a field,
  it's a free module and we can assume that
  \(M \iso \bigoplus_{j \in J} \Frac(R)\) for some set \(J\).

  Let \(\phi: \ideal a \to M\) be any morphism of \(R\)-modules. If
  \(\pi_j \in \Hom_{\Mod{R}}(M, \Frac(R))\) denotes the canonical projection of
  the \(j\)-th coordinate, we define a collection
  \((f_j: \ideal a \to \Frac(R))_{j \in J}\) of \(R\)-module morphisms by
  \(f_j \coloneq \pi_j \phi\) for each \(j \in J\). Notice that since
  \[
  \phi(a) = (f_j(a))_{j \in J} = (f_j(1) a)_{j \in J}
  \in \bigoplus_{j \in J} \Frac(R) \iso M,
  \]
  it must be the case that \(f_j(1)\) is non-zero only for finitely many
  \(j \in J\). Therefore \(u \coloneq (f_j(1))_{j \in J}\) is an element of
  \(M\) and we can define completely define a morphism of \(R\)-modules
  \(\overline{\phi}: R \to M\) by mapping \(\overline{\phi}(1) \coloneq u\). We
  conclude that such map extends \(\phi\), that is
  \[
  \begin{tikzcd}
  &M & \\
  0 \ar[r] &\ideal a \ar[r, hook] \ar[u, "\phi"]
  &R \ar[lu, bend right, "\overline{\phi}"']
  \end{tikzcd}
  \]
  is a commutative diagram in \(\Mod{R}\)---which, by Baer's criterion, proves
  that \(M\) is injective.

\item Given a \(k\)-vector space \(V\), where \(k\) is any field, we know that
  \(k = \Frac(k)\). Therefore \(V\) is an injective \(k\)-module by item (b).
\end{enumerate}
\end{proof}

\begin{proposition}[Schanuel's dual (see \cref{prop:schanuel-lemma})]
\label{prop:schanuel-lemma-dual}
Let \(E\) and \(E'\) be injective \(R\)-modules. If there exists short exact
sequences
\[
\begin{tikzcd}
0 \ar[r] &M \ar[r, tail, "f"] &E \ar[r, two heads, "g"] &L \ar[r] &0
\\
0 \ar[r] &M \ar[r, tail, "\phi"] &E' \ar[r, two heads, "\psi"] &L' \ar[r] &0
\end{tikzcd}
\]
of \(R\)-modules, then there exists an isomorphism of \(R\)-modules
\[
E \oplus L' \iso E' \oplus L.
\]
\end{proposition}

\begin{proof}
From the injectivity of \(E'\), there exists a morphism of \(R\)-modules
\(\varepsilon: E \to E'\) such that the diagram
\[
\begin{tikzcd}
&E' & \\
0 \ar[r]
&M \ar[u, tail, "\phi"] \ar[r, tail, "f"']
&E \ar[ul, bend right, "\varepsilon"']
\end{tikzcd}
\]
commutes in \(\Mod{R}\). Define a collection \((e_{\ell})_{\ell \in L}\) where
\(g(e_{\ell}) = \ell\)---which is possible because \(g\) is surjective---and
define the map \(\lambda: L \to L'\) to be given by
\(\ell \mapsto \psi \varepsilon(\ell)\), then \(\lambda\) is certainly a
morphism of \(R\)-modules and \(\lambda g = \psi \varepsilon\). We've
constructed the following commutative diagram
\[
\begin{tikzcd}
0 \ar[r]
&M \ar[d, equals] \ar[r, tail, "f"]
&E \ar[r, two heads, "g"] \ar[d, "\varepsilon"']
&L \ar[r] \ar[d, "\lambda"] &0
\\
0 \ar[r] &M \ar[r, tail, "\phi"'] &E' \ar[r, two heads, "\psi"'] &L' \ar[r] &0
\end{tikzcd}
\]
in \(\Mod{R}\). Define mappings \(\alpha: E \to E' \oplus L\) given by
\(e \mapsto (\varepsilon(e), g(e))\), and \(\beta: E' \oplus L \to L'\) by
\((e', \ell) \mapsto \lambda(\ell) - \psi(e')\). From construction, both are
\(R\)-module morphisms. Moreover, notice that \((e', \ell) \in \ker \beta\) if
and only if \(\lambda(\ell) = \psi(e')\), hence if we take \(e \in E\) such that
\(g(e) = \ell\), we find by the commutativity of the right square that
\[
\psi \varepsilon(e) = \lambda g(e) = \lambda(\ell) = \psi(e').
\]
Therefore \(\varepsilon(e) - e' \in \ker \psi\) and by exactness there exists
\(m \in M\) for which \(\phi(m) = \varepsilon(e) - e'\). From the commutativity
of the left square one has
\[
\varepsilon f(m) = \varepsilon(e) - e',
\]
thus \(\varepsilon(e - f(m)) = e'\). Notice that
\[
\alpha(e - f(m)) = (\varepsilon(e - f(m)), g(e - f(m)))
= (e', g(e) - g f(m))
= (e', g(e))
= (e', \ell),
\]
which proves that \((e', \ell) \in \im \alpha\). Therefore
\(\ker \beta \subseteq \im \alpha\). For the converse, given any \(e \in E\) we have
\[
\beta \alpha(e) = \beta(\varepsilon(e), g(e))
= \lambda g(e) - \psi \varepsilon(e)
= 0
\]
since \(\lambda g = \psi \varepsilon\)---thus
\(\im \alpha \subseteq \ker \beta\).

We now show that \(\alpha\) is injective, while
\(\beta\) is surjective. If \(e \in \ker \alpha\) then by definition
\((\varepsilon(e), g(e)) = 0\). Since \(\im f = \ker g\), let \(m \in M\) be
such that \(f(m) = e\). By the commutativity of the left square we know that
\[
\phi(m) = \varepsilon f(m) = \varepsilon(e) = 0,
\]
but \(\phi\) is injective, thus \(m = 0\). This proves that \(e = f(m) = 0\) and
therefore \(\ker \alpha = 0\). For surjectivity, let \(\ell' \in L'\) be any
element. Since \(\psi\) is surjective, choose \(e' \in E'\) with image
\(\psi(e') = \ell'\). Taking the pair \((-e', 0) \in E' \oplus L\) we get
\[
\beta(-e', 0) = \lambda(0) - \psi(-e') = \psi(e') = \ell,
\]
therefore \(\beta\) is surjective.

In the last two paragraphs we've shown that the sequence of \(R\)-modules
\[
\begin{tikzcd}
0 \ar[r]
&E \ar[r, tail, "\alpha"]
&E' \oplus L \ar[r, two heads, "\beta"]
&L' \ar[r]
&0
\end{tikzcd}
\]
is short exact. Since \(E\) is injective, then the sequence splits, proving the
existence of an isomorphism \(E' \oplus L \iso E \oplus L'\).
\end{proof}

\subsection{Divisible Modules}

\begin{definition}[Divisible module]
\label{def:divisible-module}
Let \(R\) be a \emph{domain}, and \(D\) be an \(R\)-module. We say that \(D\) is
a \emph{divisible \(R\)-module} if for all \(x \in D\) and
\(r \in R \setminus \{0\}\), there exists \(y \in D\) for which \(y r = x\).
\end{definition}

\begin{proposition}
\label{prop:properties-divisible-modules}
Let \(R\) be an integral domain. The following holds:
\begin{enumerate}[(a)]\setlength\itemsep{0em}
\item The field of fractions \(\Frac(R)\) is a divisible \(R\)-module.

\item The sum (either direct or not) of divisible \(R\)-modules is a divisible
  \(R\)-module.

\item The direct product of divisible \(R\)-modules is a divisible \(R\)-module.

\item If \(D\) is a divisible \(R\)-module and \(\phi: D \to M\) is a morphism
  of \(R\)-modules, then the image \(\phi(D) \subseteq M\) is a divisible
  \(R\)-module.

\item The quotient of a divisible \(R\)-module is a divisible \(R\)-module.

\item Every direct summand of a divisible \(R\)-module is itself a divisible
  \(R\)-module.
\end{enumerate}
\end{proposition}

\begin{proof}
\begin{enumerate}[(a)]\setlength\itemsep{0em}
\item Given any element \(x \coloneq u/v \in \Frac(R)\) and a non-zero
  \(r \in R\), one has \(y \coloneq u/(v r) \in \Frac(R)\) such that
  \[
  y r = \frac{u}{v r} \cdot r = \frac{u}{v} = x.
  \]

\item Let \((D_j)_{j \in J}\) be a family of divisible \(R\)-modules. First we
  prove that the sum of the family is divisible. Let
  \(\sum_{j \in F} x_j \in \sum_{j \in J} D_j\) be any element, where
  \(F \subseteq J\) is a finite subset. If \(r \in R\) is any non-zero element,
  since \(D_j\) is divisible, let \(y_j \in D_j\) be such that \(y_j r =
  x_j\). Since \(F\) is finite, we have \(\sum_{j \in F} y_j \in \sum_{j \in J}
  D_j\), therefore
  \[
  \Big( \sum_{j \in F} y_j \Big) r = \sum_{j \in F} y_j r = \sum_{j \in F} x_j.
  \]
  This proves that \(\sum_{j \in J} D_j\) is a divisible \(R\)-module.

  We now consider the direct sum \(\bigoplus_{j \in J} D_j\) and any element
  \((x_j)_{j \in J} \in \bigoplus_{j \in J} D_j\). If \(r \in R\) is non-zero,
  define a collection \((y_j)_{j \in J}\) as follows: if \(x_j = 0\), let
  \(y_j = 0 \in D_j\), otherwise we use the divisibility of \(D_j\) and let
  \(y_j \in D_j\) be an element such that \(y_j r = x_j\). Since finitely many
  \(j \in J\) have a non-zero \(x_j\), it follows that \((y_j)_{j \in J} \in
  \bigoplus_{j \in J} D_j\). Therefore we have
  \[
  (y_j)_{j \in J} r = (y_j r)_{j \in J} = (x_j)_{j \in J},
  \]
  proving that \(\bigoplus_{j \in J} D_j\) is a divisible \(R\)-module.

\item Let \((D_j)_{j \in J}\) be a collection of divisible \(R\)-modules, and
  let \((x_j)_{j \in J} \in \prod_{j \in J} D_j\) be any element. If \(r \in R\)
  is a non-zero element, we define a collection
  \((y_j)_{j \in J} \in \prod_{j \in J} D_j\) such that \(y_j r = x_j\)---which
  is possible since each \(D_j\) is divisible. Therefore
  \((y_j)_{j \in J} r = (y_j r)_{j \in J} = (x_j)_{j \in J}\), proving that the
  direct product \(\prod_{j \in J} D_j\) is a divisible \(R\)-module.

\item Let \(m \in \phi(D)\) be any element and take \(d \in
  \phi^{-1}(m)\). Given any non-zero \(r \in R\), since \(D\) is divisible, let
  \(y \in D\) be such that \(y r = d\). Applying \(\phi\) to such element,
  \[
  m = \phi(d) = \phi(y r) = \phi(y) r.
  \]
  Since \(\phi(y) \in \phi(D)\), this shows that \(m\) is divisible by
  \(r\)---hence \(\phi(D)\) is a divisible \(R\)-module.

\item Given a divisible \(R\)-module \(D\) and a submodule \(Q \subseteq D\),
  the natural projection \(\pi: D \epi D/Q\) shows that \(\pi(D) = D/Q\) is a
  divisible \(R\)-module via last item's result.

\item Let \(D\) be a divisible \(R\)-module and suppose that \(D = X \oplus
  Y\). Then the natural projection \(\pi_X: D \epi X\) has an image \(\pi_X(D) =
  X\)---therefore \(X\) is divisible by the result of item (d).
\end{enumerate}
\end{proof}

\begin{proposition}[Injective module is divisible]
\label{prop:int-domain-injective-mod-is-divisible}
Let \(R\) be an \emph{integral domain}. Then every \emph{injective} \(R\)-module
\emph{is} a \emph{divisible} \(R\)-module.
\end{proposition}

\begin{proof}
Let \(E\) be an injective \(R\)-module, and \(x \in E\) be any element. If
\(r \in R\) is any non-zero element, consider the submodule (ideal)
\(r R \subseteq R\)---which is free with a basis \(\{r\}\). By the injectivity
of \(E\), if \(\phi: r R \to E\) is defined by \(\phi(r) \coloneq x\), there
exists an extension \(\psi: R \to E\) such that \(\psi|_{r R} =
\phi\). Therefore one has
\[
x = \psi(r) = \psi(1 \cdot r) = \psi(1) r,
\]
which proves that \(x\) is divisible by \(r\), and hence \(E\) itself is a
divisible \(R\)-module.
\end{proof}

\begin{proposition}[Divisible modules in PIDs]
\label{prop:int-dom-PID-divisible-module-iff-injective-and-quotient}
Let \(R\) be a \emph{principal ideal domain}. The following holds:
\begin{enumerate}[(a)]\setlength\itemsep{0em}
\item An \(R\)-module is \emph{injective} if and only if it is
  \emph{divisible}.

\item Every \emph{quotient} of an \emph{injective} \(R\)-module is itself
  \emph{injective}.
\end{enumerate}
\end{proposition}

\begin{proof}
\begin{enumerate}[(a)]\setlength\itemsep{0em}
\item If \(R\) is a principal ideal domain, then in particular it's an integral
  domain, therefore any injective \(R\)-module is divisible. We prove the
  converse. Let \(D\) be a divisible \(R\)-module and take any non-zero
  submodule (ideal) \(\ideal a\) of \(R\), together with a morphism of
  \(R\)-modules \(\phi: \ideal a \to E\). Since \(R\) is a PID, assume
  \(\ideal a = a R\) for some non-zero \(a \in R\). By the divisibility of
  \(D\), we know that there exists \(d \in D\) such that
  \(\phi(a) = d a\)---that is, \(\phi(a) \in D\) is divisible by \(d\). We
  define a map \(\psi: R \to D\) by mapping \(r \mapsto d r\), which is clearly
  a morphism of \(R\)-modules. Notice that if \(x \coloneq a r \in \ideal a\) is
  any element, then
  \[
  \psi(x) = \psi(a r) = \psi(a) r = (d a) r = \phi(a) r = \phi(a r) = \phi(x),
  \]
  that is, \(\psi\) extends \(\phi\), making the diagram
  \[
  \begin{tikzcd}
  &D &
  \\
  0 \ar[r] &\ideal a \ar[u, "\phi"] \ar[r, hook]
  &R \ar[lu, "\psi"', bend right]
  \end{tikzcd}
  \]
  commutative in \(\Mod{R}\) and showing that \(D\) is injective.

\item Let \(E\) be an injective module, hence divisible since \(R\) is in
  particular an integral domain. Therefore, if \(M \subseteq E\) is any
  submodule, we know that \(E/M\) is a divisible \(R\)-module by
  \cref{prop:properties-divisible-modules} item (e). Therefore, by the fact that
  \(R\) is a PID, we use last item's result to obtain that \(E/M\) is
  injective.
\end{enumerate}
\end{proof}

\begin{corollary}[Embedding abelian groups into injective ones]
\label{cor:abelian-group-embedded-injective-abelian-group}
Every \emph{abelian group} can be \emph{embedded} as a subgroup of an
\emph{injective abelian group}.
\end{corollary}

\begin{proof}
Let \(M\) be an abelian group (\(\Z\)-module) and by
\cref{thm:any-module-is-quotient-of-free-module} we know that there exists a
free abelian group \(F \coloneq \bigoplus_{j \in J} \Z\) such that \(M = F/K\)
for some subgroup \(K \subseteq F\). Considering the natural inclusion morphism
\(\iota: \Z \emb \Q\) of \(\Z\)-modules, we find that \(F\) is a submodule of
the \(\Z\)-module \(\bigoplus_{j \in J} \Q\). Since \(\Q\) is a divisible
\(\Z\)-module, by \cref{prop:properties-divisible-modules} we find that
\(E \coloneq (\bigoplus_{j \in J} \Q)/K\) is again a divisible
\(\Z\)-module. Since \(\Z\) is a PID, by
\cref{prop:int-dom-PID-divisible-module-iff-injective-and-quotient} we find that
\(E\) is injective. Therefore the proposition follows by the natural inclusion
\(M = F/K \emb E\).
\end{proof}

\begin{theorem}[Embedding modules into injective ones]
\label{thm:module-embedded-injective-module}
Every right-\(R\)-module (or left) can be \emph{embedded as a submodule} of some
\emph{injective right-\(R\)-module} (or left).
\end{theorem}

\begin{proof}
Let \(M\) be any right-\(R\)-module. By
\cref{cor:abelian-group-embedded-injective-abelian-group}, let \(D\) be an
injective abelian group containing the abelian group \(M\). Considering the
isomorphism \(M \iso \Hom_{\rMod{R}}(R, M)\) given by \(m \mapsto f_m\)---where
\(f_m(r) \coloneq m r\)---we have an injective right-\(R\)-module morphism
\(M \mono \Hom_{\rMod{\Z}}(R, D)\) given by the composition:
\[
\begin{tikzcd}
M \ar[r, "\dis"] &\Hom_{\rMod{R}}(R, M) \ar[r, hook]
&\Hom_{\rMod{\Z}}(R, M) \ar[r, hook]
&\Hom_{\rMod{\Z}}(R, D)
\end{tikzcd}
\]
This proves that \(M\) can be embedded as a subgroup of the right-\(R\)-module
\(\Hom_{\rMod{\Z}}(R, D)\).
\end{proof}

\begin{proposition}
\label{prop:abelian-grp-divisible-then-MorR-D-injective}
Let \(D\) be a \emph{divisible abelian group} and \(R\) be any ring. Then the
abelian group \(\Hom_{\rMod{\Z}}(R, D)\) is an \emph{injective
  right-\(R\)-module} (also, \(\Hom_{\lMod{\Z}}(R, D)\) is an injective
left-\(R\)-module), with a product
\[
(f \cdot r)(x) \coloneq f(r x)
\]
for every \(f \in \Hom_{\rMod{\Z}}(R, D)\), and \(r, x \in R\).
\end{proposition}

\begin{proof}
We take \(D\) to be a right-\(\Z\)-module---which is isomorphic to the left
module since \(\Z\) is commutative. Any ring \(R\) has the structure of
\((R, \Z)\)-bimodule, thus the product is well defined and produces a structure
of right-\(R\)-module in the abelian group \(\Hom_{\rMod{\Z}}(R, D)\).

We now prove that \(\Hom_{\rMod{\Z}}(R, D)\) is injective using Baer's
criterion. Since \(\Z\) is a PID, the divisibility of \(D\) implies that it is
an injective \(\Z\)-module. Let \(\ideal a \subseteq R\) be any right-ideal of
\(R\) and \(\phi: \ideal a \to \Hom_{\rMod{\Z}}(R, D)\) be a morphism of
\(R\)-modules. Define a map \(\phi': \ideal a \to D\) by
\(a \mapsto \phi(a)(1)\), which is a morphism of right-\(\Z\)-modules. By the
injectivity of \(D\), there exists a morphism of right-\(\Z\)-modules
\(\psi': R \to D\) such that the diagram
\[
\begin{tikzcd}
&D &
\\
0 \ar[r] &\ideal a \ar[u, "\phi'"] \ar[r, hook]
&R \ar[lu, "\psi'"', bend right]
\end{tikzcd}
\]
commutes in \(\rMod{\Z}\). Define a morphism of right-\(R\)-modules
\(\psi: R \to \Hom_{\rMod{\Z}}(R, D)\) by assigning \(\psi(1) \coloneq \psi'\),
which completely determines \(\psi\) since
\[
\psi(s)(r) = (\psi(1) s) (r) = (\psi' s)(r) = \psi'(s r)
\]
for any \(s, r \in R\). Moreover, if \(a \in \ideal a\) is any element, then for
any \(r \in R\) one has
\[
\psi(a)(r)
= \psi'(a r)
= \phi'(a r)
= \phi(a r)(1)
= (\phi(a) r)(1)
= \phi(a)(r),
\]
since \(a r \in \ideal a\). Therefore \(\psi|_{\ideal a} = \phi\), making the
diagram
\[
\begin{tikzcd}
&\Hom_{\rMod{\Z}}(R, D) &
\\
0 \ar[r] &\ideal a \ar[u, "\phi"] \ar[r, hook]
&R \ar[lu, "\psi"', bend right]
\end{tikzcd}
\]
commute in \(\rMod{R}\)---which shows that \(\Hom_{\rMod{\Z}}(R, D)\) is an
injective right-\(R\)-module.
\end{proof}

\begin{lemma}
\label{lem:injective-module-completing-sequences}
Let \(E\) be an injective \(R\)-module, and
\(0 \to A \overset \alpha \mono B \overset \beta \epi C \to 0\) be a short exact
sequence of \(R\)-modules. If there exists a morphism \(\gamma: A \to E\), then
one can complete the given exact sequence to a commutative diagram in
\(\Mod{R}\):
\[
\begin{tikzcd}
0 \ar[r]
&A \ar[r, tail, "\alpha"] \ar[d, "\gamma"']
&B \ar[r, two heads, "\beta"] \ar[d, "\gamma'"]
&C \ar[d, equals] \ar[r]
&0
\\
0 \ar[r]
&E \ar[r, tail, "\alpha'"]
&P \ar[r, two heads, "\beta'"]
&C \ar[r]
&0
\end{tikzcd}
\]
whose rows are short exact sequences.
\end{lemma}

\begin{proof}
Let \(P\) be the pushout of the pair \((\gamma, \alpha)\), that is:
\[
P \coloneq
\frac{E \oplus B}{\{(\gamma(a), - \alpha(a)) \in E \oplus B \colon a \in A\}}
\]
together with natural inclusion maps \(\alpha': E \to P\) given by
\(e \mapsto [e, 0]\) and \(\gamma': B \to P\) sending \(b \mapsto [0,
b]\). Define \(\beta': P \to C\) to be the mapping \([e, b] \mapsto
\beta(b)\). To see that \(\beta'\) is well defined, let \([e, b] = [e', b']\)
and notice this means that there exists a common \(a \in A\) such that
\(\alpha(a) = b - b'\) and \(\gamma(a) = e - e'\). Since the top row is exact,
then
\[
0 = \beta\alpha(a) = \beta(b - b') = \beta(b) - \beta(b'),
\]
that is, \(\beta'[e, b] = \beta'[e', b']\). Also it is clear that \(\beta'\) is
a morphism of \(R\)-modules and that \(\beta' \gamma' = \beta\).

Given any \(e \in \ker \alpha'\) one has \(\alpha'(e) = [e, 0] = [0, 0]\), that
is, there exists \(a \in A\) for which \((\gamma(a), -\alpha(a)) = (e, 0)\) but
since \(\alpha\) is injective, then \(a = 0\)---proving that
\(\ker \alpha' = 0\) and that \(\alpha'\) is injective. For the surjectivity of
\(\beta'\), we use the fact that \(\beta\) is surjective: given any \(c \in C\),
there exists \(b \in B\) for which \(\beta(b) = c\)---therefore
\(\beta' \gamma'(b) = \beta'[0, b] = \beta(b) = c\), showing that \(\beta'\) is
surjective.

We now show the exactness of the bottom row. Since
\(\beta'\alpha'(e) = \beta'[e, 0] = \beta(0) = 0\), then
\(\im \alpha' \subseteq \ker \beta'\). Given any \([e, b] \in \ker \beta'\) by
definition we have \(\beta(b) = 0\), therefore by exactness of the top row there
must exist \(a \in A\) such that \(\alpha(a) = b\). Therefore
\[
[e, b] = [e, \alpha(a)]
= [e, \alpha(a)] + [\gamma(a), -\alpha(a)]
= [e + \gamma(a), 0]
\]
since \([\gamma(a), -\alpha(a)] = [0, 0]\). From this we obtain that
\(\alpha'(e + \gamma(a)) = [e, b]\) and hence
\(\ker \beta' \subseteq \im \alpha'\).
\end{proof}

\begin{proposition}
\label{prop:injective-iff-split-cyclic}
An \(R\)-module \(E\) is \emph{injective} if and only if \emph{every} short
exact sequence of \(R\)-modules
\[
\begin{tikzcd}
0 \ar[r] &E \ar[r, tail, "f"] &B \ar[r, two heads, "g"] &C \ar[r] &0
\end{tikzcd}
\]
\emph{ending} with a \emph{cyclic} module \(C\) is \emph{split}.
\end{proposition}

\begin{proof}
The forward implication is immediate since any short exact sequence of
right-\(R\)-modules starting with an injective module is split. For the
converse, assume that \(E\) has the described property, and let
\(\ideal b \subseteq R\) be any right-ideal together with a morphism of
right-\(R\)-modules \(\phi: \ideal b \to E\). Considering the short exact
sequence \(0 \to \ideal b \emb R \epi R/{\ideal b}\), using
\cref{lem:injective-module-completing-sequences} there exists a commutative
diagram
\[
\begin{tikzcd}
0 \ar[r]
&\ideal b \ar[r, hook, "\iota"] \ar[d, "\phi"']
&R \ar[r, two heads] \ar[d, "\phi'"]
&R/{\ideal b} \ar[d, equals] \ar[r]
&0
\\
0 \ar[r]
&E \ar[r, tail, "\alpha"]
&P \ar[r, two heads]
&R/{\ideal b} \ar[r]
&0
\end{tikzcd}
\]
Since \(R/{\ideal b}\) is a cyclic right-\(R\)-module, by hypothesis the bottom
sequence splits, proving the existence of a retract \(q: P \to E\) such that
\(q \alpha = \Id_E\). Therefore the composition \(q \phi': R \to E\) is a
morphism of right-\(R\)-modules such that \((q \phi) \iota = \phi\), making the
diagram
\[
\begin{tikzcd}
&E &
\\
0 \ar[r]
&\ideal b \ar[r, hook, "\iota"'] \ar[u, "\phi"]
&R \ar[lu, bend right, "q \phi"']
\end{tikzcd}
\]
commute in \(\rMod{R}\). Thus by Baer's criterion \(E\) is an injective module.
\end{proof}

%%% Local Variables:
%%% mode: latex
%%% TeX-master: "../../deep-dive"
%%% End:

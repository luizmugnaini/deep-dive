\section{Morphisms of Topological Spaces}

\begin{definition}[Continuous map]\label{def: continuous map}
  Let \(X\) and \(Y\) be topological spaces and consider the map \(f : X \to
  Y\). We say that \(f\) is continuous if for all \(U \subseteq Y\) open, the
  preimage \(f^{-1}(U)\) is open in \(X\).
\end{definition}

\begin{proposition}
  A map \(f : X \to Y\) is continuous if and only if the preimage of every
  closed subset is closed.
\end{proposition}

\begin{proof}
  (\(\Leftarrow\)) Suppose that \(f\) satisfies the latter, the given any \(U
  \subseteq Y\) there exists a closed set \(C \subseteq Y\) such that \(U = Y
  \setminus C\), thus \(f^{-1}(C)\) is closed, then \(f^{-1}(X \setminus C) =
  f^{-1}(U)\) is open. \(\Rightarrow\) The analogous argument is used.
\end{proof}

\begin{proposition}\label{prop: continuous maps properties}
  Let \(X, Y, Z\) be topological spaces. The following properties of continuous
  maps between topological spaces hold
  \begin{enumerate}[(CM1)]
    \item Every constant map \(f: X \to Y\) is continuous.
    \item The identity map is continuous.
    \item If \(f: X \to Y\) is continuous, then for all open set \(U \subseteq
      X\) we have \(f|_U\) continuous.
    \item If \(f: X \to Y\) and \(g: Y \to Z\) are continuous maps, then \(g
      \circ f : X \to Z\).
  \end{enumerate}
\end{proposition}

\begin{proof}
  (a) Let any point \(a \in Y\) and consider the constant map \(x \xmapsto f a\)
  for all \(x \in X\). Then, for all \(U \subseteq Y \setminus \{a\}\) open, we
  have that \(f^{-1}(U) = \emptyset\), thus have open preimage. On the other
  hand, the fiber \(f^{-1}(a) = X\), hence also open, thus \(f\) is continuous.
  (b) Notice that if \(U \subseteq Y\) is any open set, then \(\Id_X^{-1}(U) =
  U\) and thus open.
  (c) Let \(g : X \to Y\) be a continuous map, and \(U \subseteq X\) be any open
  set, then we can take any open \(V \subseteq g(U)\) and conclude that
  \(g^{-1}(V)\) is open (from the hypothesis that \(g\) is continuous); let now
  an open \(H \subseteq Y \setminus g(U)\), then certainly \(g|_U^{-1}(H) =
  \emptyset\), thus open. Hence \(g|_U : U \to Y\) is indeed continuous.
  (d) Let \(U \subseteq Z\) be open, then \((g \circ f)^{-1}(U) = f^{-1}
  (g^{-1}(U))\). Moreover, since from hypothesis \(g\) is continuous, then
  \(g^{-1}(U)\) is open, now, from the continuity of \(f\) we conclude that
  \(f^{-1}(g^{-1}(U))\) is open, hence \(g \circ f\) is continuous.
\end{proof}

\begin{definition}
  We define the category \(\cat{Top}\) as a collection of topological spaces as
  objects and continuous maps as morphisms between the topological spaces.
\end{definition}

\begin{proposition}[Local criterion for continuity]
  \label{prop: local criterion for continuity}
  Let \(f: X \to Y\) be a map between topological spaces. Then \(f\) is
  continuous if and only if for all points \(x \in X\) there exists a
  neighbourhood \(U_x \subseteq X\) of \(x\) such that \(f|_{U_x}\) is
  continuous.
\end{proposition}

\todo[inline]{add proof}

\begin{definition}[Homeomorphism]\label{def: homeomorphism}
  Let \(f \in \Mor(\cat{Top})\). If \(f\) is bijective and has a continuous
  inverse, that is, \(f^{-1} \in \Mor(\cat{Top})\), then we say that \(f\) is a
  homeomorphism (isomorphism of topological spaces). We say that \(X\) and
  \(Y\) are homeomorphic topological spaces, and write \(X \iso Y\), if there
  exists a homeomorphism between them.
\end{definition}

\begin{definition}[Open/Closed maps]
  \label{def: open/closed maps}
  Let \(f : X \to Y\) be any map between topological spaces. We have the
  following definitions:
  \begin{enumerate}[(a)]
    \item We say that \(f\) is an open map if for all \(U \subseteq X\) open the
      image \(f(U) \subseteq Y\) is open.
    \item We say that \(f\) is a closed map if for all \(C \subseteq X\) closed,
      the image \(f(C) \subseteq Y\) is closed.
  \end{enumerate}
\end{definition}

\begin{proposition}
  Let \(f: X \to Y\) be a map of topological spaces and consider that \(f\) is a
  homeomorphism, then \(f\) is an open and a closed map.
\end{proposition}

\begin{proof}
  Let \(U \subseteq X\) be an open (resp. closed) set and consider \(V := f(U)
  \subseteq Y\).  Since \(f\) is a bijection, we find that \(f(U) = V\) is open
  (resp. closed) by the continuity of \(f^{-1}\). Hence \(f\) is an open (resp.
  closed) map.
\end{proof}

\subsection{Topology Generated by Mappings}

\begin{proposition}[Topology generated by a collection of mappings]
  \label{prop: top generated by collection of maps}
  Let \(X\) be a topological space and \(\{Y_i\}_{i \in I}\) be a collection of
  topological spaces. Let \(\{f_i : X \to Y_i\}_{i \in I}\) be the collection of
  mappings between such topological spaces. Then there exists a coarsest
  topology on \(X\) such that \(f_i\) is continuous, for all \(i \in I\). Such
  topology is generated by the base
  \[
    \mathcal B = \left\{ \bigcap_{j \in J} f_j^{-1}(U_j) : U_i \subseteq Y_i
    \text{ is open, } J \subseteq I \text{ is finite}\right\}.
  \] 
  We call such topology as the topology generated by the collection of mappings
  \(\{f_i\}_{i \in I}\).
\end{proposition}

\begin{proof}
  First we show that \(\mathcal B\) is indeed a basis for \(X\). Let \(x \in X\) 
  be any point, then clearly \(x \in f_i^{-1}(Y_i)\) for all \(i \in I\), hence
  \(x \in \bigcap_{j \in  J} f_j^{-1}(Y_j) \in \mathcal B\) for a finite
  indexing set \(J \subseteq I\).
  Let \(J, S \subseteq I\) be finite indexing sets, then consider the sets
  \(A := \bigcap_{j \in J} f_j^{-1}(U_j), B := \bigcap_{s \in S} f_s^{-1}(V_s) \in
  \mathcal B\) and let any point \(x \in A \cap B\). Then in particular we can
  let an nonempty indexing set \(T = J \cap S\) so that \(x \in f_t^{-1}(U_t)\)
  for all \(t \in T\) and therefore \(C := \bigcap_{t \in  T} f_t^{-1}(U_t)\) is
  such that \(x \in C \subseteq A \cap B\). This shows that \(\mathcal B\) 
  indeed satisfies the basis properties.

  Now we show the coarsest topology property. Let \(\mathcal T\) be the topology
  generated by \(\mathcal B\) and consider \(\mathcal T'\) to be any other
  topology on \(X\) for which the functions \(f_i\) are continuous for all \(i
  \in I\). Trivially we must have \(\mathcal B \subseteq \mathcal T'\) and
  therefore \(\mathcal T \subseteq \mathcal T'\). This says that \(\mathcal T\) 
  is coarser than \(\mathcal T'\), which proves the proposition.
\end{proof}

\begin{proposition}
  Let \(X\) and \(Y\) be topological spaces, and the topology of \(Y\) be
  generated by the collection of maps \(\{f_i : Y \to Y_i\}_{i \in I}\), where
  \(\{Y_i\}_{i \in I}\) is a collection of topological spaces. Then a map \(f :
  X \to Y\) is continuous if and only if the composition \(f_i \circ f : X \to
  Y_i\) is continuous for every \(i \in I\).
\end{proposition}

\begin{proof}
  (\(\Leftarrow\)) Suppose \(f_i \circ f\) is continuous for all \(i \in I\),
  then since \(f_i\) is continuous on the topology of \(Y\) it follows that for
  any given open set \(U \subseteq Y_i\) we have \(f_i^{-1}(U) = V \subseteq Y\)
  open. In particular, notice that \((f_i \circ f)^{-1}(U) = f^{-1}(f_i^{-1}(U))
  = f^{-1}(V) \subseteq X\) which must be open from the hypothesis of the
  continuity of \(f_i \circ f\). (\(\Rightarrow\)) Moreover, if \(f\) is
  continuous, then clearly the composition of continuous functions is
  continuous.
\end{proof}

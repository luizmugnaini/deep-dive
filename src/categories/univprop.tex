\section{Products}

\begin{definition}[Product]
\label{def:product}
Let \(J\) be a set, and \((C_j)_{j \in I}\) be a
collection of objects in a category \(\cat C\).

A \emph{product} of such collection, \emph{if it exists}, is a pair
\((P, (\pi_j)_{j \in J})\) --- where \(P\) is an object of \(\cat C\), and for
every \(j \in J\), \(\pi_j: P \to C_j\) is a morphism of \(\cat C\).

Furthermore, this pair has to satisfy the following \emph{universal property}:
for every pair \((Q, (q_j)_{j \in J})\) --- where \(Q \in \cat C\) and for every
\(j \in J\), \(q_j: Q \to C_j\) is a morphism in \(\cat C\) --- there exists a
\emph{unique morphism} \(f: Q \unique P\) of \(\cat C\) such that, for every
\(j \in J\) the following diagram commutes
\[
\begin{tikzcd}
Q \ar[rd, "q_j", bend left] \ar[d, dashed, "f"'] & \\
P \ar[r, "\pi_j"'] & C_j
\end{tikzcd}
\]
\end{definition}

\begin{proposition}[Uniqueness]
\label{prop:product-unique-up-to-iso}
The product of a collection of objects, if existent, is \emph{unique up to
  isomorphism}.
\end{proposition}

\begin{proof}
Let \(\cat C\) be a category admitting the product of a collection
\((C_j)_{j \in J}\) of objects of \(\cat C\), and \(J\) is a set. Let
\((P, (p_j)_{j \in J})\) and \((Q, (q_j)_{j \in J})\) be products of
\((C_j)_{j \in J}\) in the category \(\cat C\). Since \(P\) and \(Q\) are
products, there exists unique morphisms \(f: Q \to P\) and \(g: P \to Q\) such
that, for all \(j \in J\), we have \(q_j = p_j f\) and \(p_j = q_j g\).

Moreover, one can apply the product property of \(P\) to \(P\) itself: there
exists a unique morphism \(h: P \to P\) such that, for all \(j \in J\), we have
\(p_j = p_j h\). For that to be true, it is clear that we must have
\(h = \Id_{P}\). Note, however, that for each \(j \in J\),
\[
p_j = q_j g = p_j f g.
\]
Since \(h\) is unique, we obtain \(f g = h = \Id_P\).  On the other hand,
applying the product property of \(Q\) in \(Q\) will yield \(g f = \Id_Q\). This
shows that \(Q \iso P\) in \(\cat C\), via \(f\) and \(g\).
\end{proof}

\begin{proposition}[Products are independent of ordering]
\label{prop:product-ordering-independent}
Let \(I\) be a set and \((J_k)_{k \in K}\) be a partition of \(I\) by disjoint
subsets \(J_k \subseteq I\). Let \((C_i)_{i \in I}\) be a collection of objects
in a category \(\cat C\). Then, if all the products presented below exist in
\(\cat C\), they are isomorphic:
\[
\prod_{i \in I} C_i \iso \prod_{k \in K} \bigg(\prod_{j \in J_k} C_j \bigg).
\]
\end{proposition}

\begin{proof}
Define the following collections:
\begin{itemize}\setlength\itemsep{0em}
\item Let \((q_k)_{k \in K}\) be the collection of morphisms
  \(q_{k_0}: \prod_{k \in K} \bigg( \prod_{j \in J_k} C_j \bigg) \to C_{k_0}\),
  for each \(k_0 \in K\), associated with the product
  \(\prod_{k \in K} \bigg( \prod_{j \in J_k} C_j \bigg)\) in \(\cat C\).

\item For all \(k \in K\), let \((p_j)_{j \in J_k}\) be the collection of
  morphisms \(p_{j_0}: \prod_{j \in J_k} C_j \to C_{j_0}\), for each
  \(j_0 \in J_k\) associated with the product \(\prod_{j \in J_k} C_j\) in
  \(\cat C\).
\end{itemize}
Let \((L, (\ell)_{i \in I})\) be a pair where \(L \in \cat C\) and
\(\ell_i: L \to C_i\) is a morphism in \(\cat C\) for each \(i \in I\).

Define a collection \((g_k)_{k \in K}\) where, for each \(k_0 \in K\), we let
\(g_{k_0}: L \to \prod_{j \in J_{k_0}} C_j\) be the unique morphism of
\(\cat C\) such that \(\ell_i = g_{k_0} p_i\) for every \(i \in J_k\). Let
\(f: \prod_{k \in K} \bigg( \prod_{j \in J_k} C_j \bigg)\) be the unique
morphism of \(\cat C\) such that \(g_k = q_k f\).

We see that the following diagram commutes for every \(k_0 \in K\) and
\(i \in J_k\):
\[
\begin{tikzcd}
L \ar[rrrrdd, bend left, "\ell_i"]
\ar[dd, dashed, "f"']
\ar[rrdd, dashed, "g_{k_0}" description]
& & & &\\ & & & & \\
\prod_{k \in K} \bigg( \prod_{j \in J_k} C_j \bigg)
\ar[rr, "q_{k_0}"']
&
&\prod_{j \in J_{k_0}} C_j \ar[rr, "p_i"']
&
&C_i
\end{tikzcd}
\]
Since \(I = \bigcup_{k \in K} J_k\), the diagram commutes for any \(i \in I\)
--- thus \(\prod_{k \in K} \bigg( \prod_{j \in J_k} C_j \bigg)\) is a product of
the collection \((C_i)_{i \in I}\). Now, \cref{prop:product-unique-up-to-iso}
finishes the proof.
\end{proof}

\section{Coproducts}

\begin{definition}[Coproduct]
\label{def:coproduct}
Let \(J\) be a set, and \((C_j)_{j \in I}\) be a
collection of objects in a category \(\cat C\).

A \emph{coproduct} of such collection, \emph{if it exists}, is a pair
\((P, (\iota_j)_{j \in J})\) --- where \(P\) is an object of \(\cat C\), and for
every \(j \in J\), \(\iota_j: P \to C_j\) is a morphism of \(\cat C\).

Furthermore, this pair has to satisfy the following \emph{universal property}:
for every pair \((Q, (q_j)_{j \in J})\) --- where \(Q \in \cat C\) and for every
\(j \in J\), \(q_j: C_j \to Q\) is a morphism in \(\cat C\) --- there exists a
\emph{unique morphism} \(f: P \unique Q\) of \(\cat C\) such that, for every
\(j \in J\) the following diagram commutes
\[
\begin{tikzcd}
P \ar[d, dashed, "f"']
&C_j \ar[l, "\iota_j"'] \ar[ld, bend left, "q_j"] \\
Q &
\end{tikzcd}
\]
\end{definition}

We can see right away that coproducts are dual to products, yielding the
following two dual properties.

\begin{proposition}[Uniqueness]
\label{prop:coproduct-unique-up-to-iso}
If a coproduct of a collection of objects of a category exists, then this
coproduct is \emph{unique up to isomorphism}.
\end{proposition}

\begin{proposition}[Coproducts are independent of ordering]
\label{prop:coproduct-ordering-independent}
Let \(I\) be a set and \((J_k)_{k \in K}\) be a partition of \(I\) by disjoint
subsets \(J_k \subseteq I\). Let \((C_i)_{i \in I}\) be a collection of objects
in a category \(\cat C\). Then, if all the coproducts presented below exist in
\(\cat C\), they are isomorphic:
\[
\coprod_{i \in I} C_i \iso \coprod_{k \in K} \bigg(\coprod_{j \in J_k} C_j \bigg).
\]
\end{proposition}

\begin{remark}[Duality]
\label{rem:coproducts-dual-product}
Coproducts are \emph{dual} to products in the following sense: if \(P\) is a
product in a category \(\cat C\), then \(P^{\op}\) is a coproduct in the
opposite category \(\cat C^{\op}\). Notice however that one \emph{cannot} simply
reverses the arrows of a product in \(\cat C\) and end up with a coproduct in
\(\cat C\) --- in fact, there are categories where product do exist, while
coproducts do not.
\end{remark}

\section{Equalizers and Coequalizers}

\begin{definition}[Equalizer]
\label{def:equalizer}
Let \(f, g: A \para B\) be parallel morphisms in a category \(\cat C\). An
\emph{equalizer of \(f\) and \(g\)} is a pair \((K, k)\) --- where
\(K \in \cat C\) is an object and \(k: K \to A\) is a morphism of \(\cat C\) for
which \(f k = g k\) --- such that, for any object \(M \in \cat C\) and morphism
\(m: M \to A\) of \(\cat C\) satisfying \(f m = g m\), there exists a
\emph{unique} morphism \(n: M \to K\) such that the following diagram commutes
\[
\begin{tikzcd}
M \ar[d, dashed, "n"'] \ar[rd, bend left, "m"] & &
\\
K \ar[r, "k"'] &A \ar[r, shift left, "f"] \ar[r, shift right, "g"'] &B
\end{tikzcd}
\]
\end{definition}

Just as with products and coproducts, one has a dual notion of an equalizer, we
call it a \emph{coequalizer}. For the sake of later reference, we write down its
definition.

\begin{definition}[Coequalizer]
\label{def:coequalizer}
Let \(f, g: A \para B\) be parallel morphisms in a category \(\cat C\). An
\emph{equalizer of \(f\) and \(g\)} is a pair \((C, c)\) --- where
\(C \in \cat C\) is an object and \(c: B \to C\) is a morphism of \(\cat C\) for
which \(c f = c g\) --- such that, for any object \(M \in \cat C\) and morphism
\(m: B \to M\) of \(\cat C\) satisfying \(m f = m g\), there exists a
\emph{unique} morphism \(n: C \to M\) such that the following diagram commutes
\[
\begin{tikzcd}
C \ar[d, dashed, "n"]
& B \ar[l, "c"'] \ar[ld, "m", bend left]
& A \ar[l, shift left, "g"] \ar[l, shift right, "f"']
\\
M& &
\end{tikzcd}
\]

\end{definition}

\begin{proposition}[Uniqueness]
\label{prop:(co)equalizer-unique-up-to-iso}
Given two parallel morphisms \(f, g: A \para B\) in a category \(\cat C\), if
the (co)equalizer of them exists, then it is \emph{unique up to isomorphism}. We
denote \emph{the} equalizer of \(f\) and \(g\) by \(\Eq(f, g)\) and the
coequalizer of \(f\) and \(g\) by \(\Coeq(f, g)\).
\end{proposition}

\begin{proof}
We prove for equalizers. Let \((K, k)\) and \((K', k')\) be equalizers of \(f\)
and \(g\). If we apply the equalizer property of \((K', k')\) in \((K, k)\) and
vice versa, we obtain unique morphisms \(n': K \to K'\) and \(n: K' \to K\) such
that \(k = k' n'\) and \(k' = k n\).

Moreover, applying the equalizer property of \((K, k)\) in itself we obtain a
unique \(t: K \to K\) such that \(k = k t\) --- therefore \(t = \Id_K\). Since
\(k = k' n = k \Id_K\) and \(\Id_K\) is unique with such property, it follows
that, since \(k = k' n' = (k n) n' = k (n n')\) we find that \(n n' =
\Id_K\). On the other hand, doing the same for \((K', k')\) implies in \(n' n =
\Id_{K'}\). Therefore \(K \iso K'\) in \(\cat C\), via \(n\) and \(n'\).
\end{proof}

\begin{corollary}
\label{cor:self-(co)equalizer}
Given a morphism \(f: A \to B\) in a category \(\cat C\), the (co)equalizer of
\(f\) with itself always exists --- in fact \(\Eq(f, f) = (A, \Id_A)\) and
\(\Coeq(f, f) = (B, \Id_{B})\).
\end{corollary}

\begin{proposition}
\label{prop:eq-monic-coeq-epic}
Let \(\cat C\) be a category and \(f, g: A \para B\) be parallel morphisms in
\(\cat C\). The following are properties concerning equalizers and coequalizers:
\begin{enumerate}\setlength\itemsep{0em}
\item If \(f\) and \(g\) have an equalizer \(\Eq(f, g)\) in \(\cat C\), the
  morphism \(\Eq(f, g) \mono A\) associated to the equalizer is a
  \emph{monomorphism}.

\item If \(f\) and \(g\) have an coequalizer \(\Coeq(f, g)\) in \(\cat C\), the
  morphism \(B \epi \Coeq(f, g)\) associated to the coequalizer is an
  \emph{epimorphism}.
\end{enumerate}
\end{proposition}

\begin{proof}
The properties are dual to each other, thus we may simply prove the one
concerning equalizers. Let \(k: \Eq(f, g) \to A\) be the morphism in \(\cat C\)
associated to the equalizer. Consider any two parallel morphisms \(x, y: C \para
\Eq(f, g)\) such that \(k x = k y\) --- therefore, the following diagram
commutes
\[
\begin{tikzcd}
C \ar[r, shift left, "x"] \ar[r, shift right, "y"']
& \Eq(f, g) \ar[r, "k"]
& A \ar[r, shift left, "f"] \ar[r, shift right, "g"']
& B
\end{tikzcd}
\]
From the diagram we obtain the relation \(f (k x) = g (k x)\). Since \(k x: C
\to A\), the universal property implies that \(x\) must be the unique --- hence
\(y = x\).
\end{proof}

\begin{proposition}
\label{prop:epi-and-eq-is-iso}
Let \(f: A \to B\) be a morphism of a category \(\cat C\). If \(f\) is both an
\emph{epimorphism and equalizer}\footnote{This is clearly an abuse of language,
  the equalizer is in fact \(A\) and \(f\) is the morphism associated with
  \(A\).}, then \(f\) is an \emph{isomorphism}.
\end{proposition}

\begin{proof}
Let \(x, y: B \para C\) be parallel morphisms in \(\cat C\) with \(\Eq(x, y) =
f\), then
\[
\begin{tikzcd}
A \ar[r, "f"] & B \ar[r, shift right, "g"'] \ar[r, shift left, "f"] &C
\end{tikzcd}
\]
which implies in \(x f = y f\). By hypothesis \(f\) is an epimorphism, thus
\(x = y\). Notice however that the equalizer of a morphism \(B \to C\) with
itself is the identity on \(B\). Since equalizers are unique up to isomorphism,
it follows that \(f \iso \Id_B\) --- thus \(f\) itself is an isomorphism (and
\(A \iso B\)).
\end{proof}

\section{Pullbacks \& Pushouts}

\begin{definition}[Pullback]
\label{def:pullback}
Let \(f: A \to C\) and \(g: B \to C\) be any two morphisms in a category
\(\cat C\). A \emph{pullback of \((f, g)\)} is a triple \((P, f', g')\)
where the diagram
\[
\begin{tikzcd}
P \ar[r, "f'"] \ar[d, "g'"'] &B \ar[d, "g"]
\\
A \ar[r, "f"'] &C
\end{tikzcd}
\]
commutes in \(\cat C\), and for every other triple \((Q, f'', g'')\) making the
diagram
\[
\begin{tikzcd}
Q \ar[r, "f''"] \ar[d, "g''"'] &B \ar[d, "g"]
\\
A \ar[r, "f"'] &C
\end{tikzcd}
\]
commute in \(\cat C\), there exists a unique morphism \(\ell: Q \to P\) in
\(\cat C\) such that the following diagram commutes
\[
\begin{tikzcd}
Q \ar[dr, dashed, "\ell"]
\ar[rrd, bend left, "f''"]
\ar[drd, bend right, "g''"']
& &
\\
&P \ar[r, "f'"] \ar[d, "g'"'] &B \ar[d, "g"]
\\
&A \ar[r, "f"'] &C
\end{tikzcd}
\]
\end{definition}

We usually denote that a square is a pullback in \(\cat C\) by marking it as
follows
\[
\begin{tikzcd}
P \ar[r, "f'"] \ar[d, "g'"'] \ar[dr, phantom, "\lrcorner", very near start]
&B \ar[d, "g"]
\\
A \ar[r, "f"'] &C
\end{tikzcd}
\]

The \emph{dual} notion of a pullback is that of a pushout, just as before,
we'll write it down just for the sake of later reference.

\begin{definition}[Pushout]
\label{def:pushout}
Let \(f: C \to A\) and \(g: C \to B\) be any two morphisms in a category
\(\cat C\). A \emph{pushout of \((f, g)\)} is a triple \((P, f', g')\)
where the diagram
\[
\begin{tikzcd}
C \ar[r, "f"] \ar[d, "g"'] &A \ar[d, "g'"] \\
B \ar[r, "f'"'] &P
\end{tikzcd}
\]
commutes in \(\cat C\), and for every other triple \((Q, f'', g'')\) making the
diagram
\[
\begin{tikzcd}
C \ar[r, "f"] \ar[d, "g"'] &A \ar[d, "g''"] \\
B \ar[r, "f''"'] &Q
\end{tikzcd}
\]
commute in \(\cat C\), there exists a unique morphism \(\ell: P \to Q\) in
\(\cat C\) such that the following diagram commutes
\[
\begin{tikzcd}
C \ar[r, "f"] \ar[d, "g"']
&A \ar[d, "g'"] \ar[drd, bend left, "g''"]
&
\\
B \ar[r, "f'"'] \ar[rrd, bend right, "f''"']
&P \ar[dr, dashed, "\ell"]
&
\\
& &Q
\end{tikzcd}
\]
\end{definition}

Analogously, if we want to visually say that a square is a pushout in \(\cat
C\), we mark it as follows
\[
\begin{tikzcd}
C \ar[r, "f"] \ar[d, "g"'] &A \ar[d, "g'"] \\
B \ar[r, "f'"'] &P \ar[lu, phantom, "\ulcorner", very near start]
\end{tikzcd}
\]

\begin{proposition}[Uniqueness]
\label{prop:pullback-pushout-uniqueness}
The pullback (or pushout) of two morphisms, if existent, is \emph{unique up to
  isomorphism}.
\end{proposition}

\begin{proof}
Let \(f: A \to C\) and \(g: B \to C\) be two morphisms in a category \(\cat
C\). Suppose there exists two pullbacks of \((f, g)\), namely \((P, f', g'')\)
and \((Q, f'', g'')\). Let \(\phi: P \to Q\) and \(\psi: Q \to P\) be the
uniquely defined morphisms given by the universal property of the
pullback. Consider the following two commutative diagrams
\[
\begin{tikzcd}
P \ar[rd, dashed, "\phi"]
\ar[rrrdd, bend left=50, "f'"]
\ar[ddrrd, bend right=50, "g'"']
& & &
\\
&Q \ar[dr, dashed, "\psi"]
\ar[rrd, bend left, "f''"]
\ar[drd, bend right, "g''"']
& &
\\
& &P \ar[r, "f'"] \ar[d, "g'"']
&B \ar[d, "g"]
\\
& &A \ar[r, "f"'] &C
\end{tikzcd}
\qquad \qquad
\begin{tikzcd}
Q \ar[rd, dashed, "\psi"]
\ar[rrrdd, bend left=50, "f''"]
\ar[ddrrd, bend right=50, "g''"']
& & &
\\
&P \ar[dr, dashed, "\phi"]
\ar[rrd, bend left, "f'"]
\ar[drd, bend right, "g'"']
& &
\\
& &Q \ar[r, "f''"] \ar[d, "g''"']
&B \ar[d, "g"]
\\
& &A \ar[r, "f"'] &C
\end{tikzcd}
\]
Notice, however, that since \(\Id_P: P \to P\) and \(\Id_Q: Q \to Q\) also make
them commute --- respectively, the left and right diagrams. From uniqueness we
obtain \(\phi \psi = \Id_P\) and \(\psi \phi = \Id_Q\) --- therefore \(P \iso
Q\) in \(\cat C\), via \(\phi\) and \(\psi\).
\end{proof}

\begin{proposition}
\label{prop:preservation-monic-epic-iso-by-pull-push}
Let \((P, f', g')\) be the pullback of a pair of morphisms \((f, g)\) in a
category \(\cat C\), then:
\begin{enumerate}[(a)]\setlength\itemsep{0em}
\item If \(g\) is a monomorphism, then \(g'\) is also a monomorphism.
\item If \(g\) is an isomorphism, then \(g'\) is also an isomorphism.
\end{enumerate}
Dually, if \((B, f'', g'')\) is the pushout of the pair \((f, g)\), then:
\begin{enumerate}[(a)]\setcounter{enumi}{2}\setlength\itemsep{0em}
\item If \(g\) is an epimorphism, then \(g''\) is also an epimorphism.
\item If \(g\) is an isomorphism, then \(g''\) is also an isomorphism.
\end{enumerate}
\end{proposition}

\begin{proof}
We only prove items (a) and (b), since (c) and (d) are merely dual consequences
of the former items.
\begin{enumerate}[(a)]\setlength\itemsep{0em}
\item Let \(g\) be a monomorphism and consider parallel morphisms \(u, v: Q
  \para P\) such that \(g' u = g' v\) --- we want to prove that \(u = v\). Take
  into account the following commutative diagram
  \[
  \begin{tikzcd}
  Q \ar[rd, shift left, "u"] \ar[rd, shift right, "v"'] & & \\
  &P \ar[r, "f'"] \ar[d, "g'"']
  \ar[rd, phantom, very near start, "\lrcorner"]
  &B \ar[d, "g"] \\
  &A \ar[r, "f"'] &C
  \end{tikzcd}
  \]

  Lets first consider the morphism \(u\). Notice that \(p \coloneq g' u\) and
  \(q \coloneq f' u\) are such that \(f p = g q\). From the universal property
  of the pullback we have that \(u\) is the unique morphism factorizing
  \((p, q)\) through \((g, f)\).

  On the other hand, if we consider the arrow \(v\), one can define
  \(p' \coloneq g' v\) and \(q' \coloneq f' v\) so that \(f p' = g q'\) --- thus
  \(v\) is the unique factorization of \((p', q')\) through \((g, f)\). Notice,
  however, that from construction \(p' = g' v = g' u = p\). Moreover,
  \[
  g q' = f p' = f (g' v) = f (g' u) = f p = g q,
  \]
  since \(g\) is a monomorphism, it follows that \(g q' = g q\) implies
  \(q' = q\). Therefore both \(u\) and \(v\) are factorizations of the same pair
  of morphisms --- and from uniqueness, it can only be the case that \(u = v\).

\item Suppose \(g\) is an isomorphism and consider the following commutative
  diagram
  \[
  \begin{tikzcd}
  A \ar[rrd, "g^{-1} f", bend left]
  \ar[drd, "\Id_{A}"', bend right]
  \ar[rd, "\ell", dashed]
  & &
  \\
  &P \ar[r, "f'"] \ar[d, "g'"']
  &B \ar[d, "g"]
  \\
  &A \ar[r, "f"'] &C
  \end{tikzcd}
  \]
  Thus \(g' \ell = \Id_A\) is already given. We can now consider the diagram
  \begin{equation}\label{eq:pullback-pres-iso}
  \begin{tikzcd}
  P \ar[rrd, "f' (\ell g')", bend left]
  \ar[drd, "g' (\ell g')"', bend right]
  \ar[rd, "\ell g'", dashed]
  & &
  \\
  &P \ar[r, "f'"] \ar[d, "g'"']
  &B \ar[d, "g"]
  \\
  &A \ar[r, "f"'] &C
  \end{tikzcd}
  \end{equation}
  Moreover, we consider the following composition
  \[
  g' (\ell g') = (g' \ell) g' = \Id_A g' = g' = g' \Id_P,
  \]
  on the other hand we have
  \[
  f' (\ell g') = (f' \ell) g'
  = (g^{-1} f) g' = g^{-1} (f g')
  = g^{-1} (g f') = f' = f' \Id_P.
  \]
  Therefore \(g' (\ell g') = g' \Id_P\) and \(f' (\ell g') = f' \Id_P\) but by
  uniqueness, since \(\Id_P\) also makes \cref{eq:pullback-pres-iso} commute, it
  follows that \(\ell g' = \Id_P\). Thus \(g'\) is an isomorphism with inverse
  \(\ell\).
\end{enumerate}
\end{proof}

\begin{definition}[Kernel \& cokernel]
\label{def:kernel-cokernel}
Let \(f: A \to B\) be a morphism in a category \(\cat C\). The \emph{(co)kernel}
of \(f\), if existent, is defined to be the pullback of \(f\) with itself --- or
pushout in the case of cokernels.
\end{definition}

\begin{proposition}
\label{prop:morphisms-kernel-cokernel}
Let \(\cat C\) be a category and \(f: A \to B\) be a morphism of \(\cat C\). If
the (co)kernel of \(f\) exists, its associated morphisms are both epimorphisms
(monomorphisms for the case of cokernels).
\end{proposition}

\begin{proof}
Let \(\ker f \coloneq (K, \alpha, \beta)\) and consider \(A\) itself, together
with the identity morphisms. Since \(K\) is a pullback of \(f\) with itself,
there exists a unique morphism \(\gamma: A \to K\) such that \(\beta \gamma =
\Id_A = \alpha \gamma\) --- therefore \(\gamma\) is a split monomorphism and
both \(\alpha\) and \(\beta\) are split epimorphisms.
\end{proof}

\begin{proposition}
\label{prop:(co)kernel-properties}
Let \(f: A \to B\) be a morphism in a category \(\cat C\). Then the following
properties are equivalent:
\begin{enumerate}[(a)]\setlength\itemsep{0em}
\item The morphism \(f\) is monic (conversely, epic).
\item The kernel of \(f\) exists, furthermore \(\ker f = (A, \Id_A, \Id_A)\)
  (conversely we have \(\coker f = (B, \Id_B, \Id_B)\)).
\item The (co)kernel \((K, \alpha, \beta)\) of \(f\) exists, and
  \(\alpha = \beta\).
\end{enumerate}
\end{proposition}

\begin{proof}
\begin{itemize}\setlength\itemsep{0em}
\item (a) \(\implies\) (b): Since \(f\) is monic, if \(P\) is any object
  together with morphisms \(\phi, \psi: P \para A\) such that
  \(f \phi = f \psi\) then \(\phi = \psi\) and we may simply take \(A\) to be
  the kernel of \(f\) together with the identity morphisms --- while the unique
  morphism \(P \to A\) is given by \(\phi = \psi\).

\item (b) \(\implies\) (c): If \((K, \alpha, \beta)\) is a kernel for \(f\) then
  in particular there exists an isomorphism \(\phi: A \to K\) and therefore
  \(\alpha \phi = \Id_A\) and \(\beta \phi = \Id_A\). Since \(\phi\) is
  epic, it follows that \(\alpha = \beta\).

\item (c) \(\implies\) (a): Suppose \((K, \alpha, \alpha)\) is kernel for \(f\),
  then for all objects \(P\) and morphisms \(g, h: P \para A\) such that\(f g =
  f h\), we have a unique \(\phi: P \to K\) such that \(\alpha \phi = g\) and
  \(\alpha \phi = h\) --- thus \(g = h\).
\end{itemize}
\end{proof}

\begin{proposition}[Coequalizers \& kernels]
\label{prop:coeq-and-ker}
Let \(\cat C\) be a category. The
following properties relate coequalizers and kernels:
\begin{enumerate}[(a)]\setlength\itemsep{0em}
\item Consider parallel morphisms \(x, y: X \para A\). If the coequalizer
  \(f = \Coeq(x, y)\) exists and has a kernel pair \(\ker f = (\alpha, \beta)\),
  then \(\Coeq(x, y)\) is the coequalizer of \(\ker f\).

\item Consider a morphism \(h: A \to B\). If the kernel
  \(\ker h = (\varepsilon, \delta)\) exists and has a coequalizer
  \(w = \Coeq(\varepsilon, \delta)\), then \(\ker h\) is the kernel of
  \(\Coeq(\varepsilon, \delta)\).
\end{enumerate}
\end{proposition}

\begin{proof}
\begin{enumerate}[(a)]\setlength\itemsep{0em}
\item Since \(f\) is the coequalizer of \(x\) and \(y\), it cleartly satisfies
  \(f x = f y\). Therefore, the triple \((X, x, y)\) can be used to apply the
  universal property of the pullback of \(\ker f\) to obtain a unique
  factorization \(\phi: X \to \ker f\) such that \(x = \alpha \phi\) and
  \(y = \beta \phi\).

  Notice that \(\alpha, \beta: \ker f \para A\) are parallel morphisms, so one
  can be tempted to find its coequalizer --- we shall prove that
  \(\Coeq(\alpha, \beta) = \Coeq(x, y)\). Let \(C\) be an object together with a
  morphism \(g: A \to C\) such that \(g \alpha = g \beta\). We may precompose
  this morphism with \(\phi\), obtaining \(g \alpha \phi = g \beta \phi\), but
  using the result fro the last paragraph we conclude that \(g x = g y\). Using
  the universal property for the coequalizer of \(x\) and \(y\), we find a
  unique factorization morphism \(\psi: \Coeq(x, y) \to C\) such that
  \(g = \psi f\). Therefore, \(\Coeq(x, y)\) satisfies the universal property
  for the coequalizer of the kernel pair \((\alpha, \beta)\) of \(f\). The whole
  construction of this item's proof can be seen in the following commutative
  diagram:
  \[
  \begin{tikzcd}
  X \ar[rd, dashed, "\phi"]
  \ar[rrd, bend left, "x"]
  \ar[drd, bend right, "y"'] & & &
  \\
  &\ker f \ar[r, "\alpha"]
  \ar[rd, phantom, "\lrcorner", very near start]
  \ar[d, "\beta"']
  &A \ar[d, "f"] \ar[drd, bend left, "g"]
  &
  \\
  &A \ar[r, "f"'] \ar[rrd, "g"', bend right]
  &\Coeq(x, y) \ar[rd, dashed, "\psi"]
  &
  \\
  & & &C
  \end{tikzcd}
  \]

\item From the pullback definition, we know that \(h \varepsilon = h \delta\)
  --- since \(w\) is the coequalizer of \((\varepsilon, \delta)\), it follows
  that there exists a unique morphism
  \(\gamma: B \to \Coeq(\varepsilon, \delta)\) such that \(\gamma w = h\).

  Consider now two parallel morphisms \(a, b: Y \para A\) such that
  \(w a = w b\). Then we have that \(h a = (\gamma w) a\), and
  \(h b = (\gamma w) b\) since \(w b = w a\) by assumption, it follows that
  \(h a = h b\). From the kernel property, there must exist a unique morphism
  \(\eta: Y \to \ker h\) such that \(\varepsilon \eta = a\) and
  \(\delta \eta = b\). Therefore \(\ker w = \ker h\). All constructions can be
  visualized in the following commutative diagram:
  \[
  \begin{tikzcd}
  Y \ar[rd, dashed, "\eta"]
  \ar[rrd, bend left, "a"]
  \ar[drd, bend right, "b"'] & & &
  \\
  &\ker h \ar[r, "\varepsilon"]
  \ar[rd, phantom, "\lrcorner", very near start]
  \ar[d, "\delta"']
  &A \ar[d, "h"] \ar[drd, bend left, "w"]
  &
  \\
  &A \ar[r, "h"'] \ar[rrd, "w"', bend right]
  &B
  &
  \\
  & & &\Coeq(\varepsilon, \delta) \ar[lu, dashed, "\gamma"']
  \end{tikzcd}
  \]
\end{enumerate}
\end{proof}

\begin{proposition}[Associativity property]
\label{prop:associativity-pullbacks}
Let \(\cat C\) be a category and consider the following commutative diagram in
\(\cat C\):
\[
\begin{tikzcd}
A \ar[r, "a"] \ar[d, "c"']
&B \ar[r, "b"] \ar[d, "d"']
&C \ar[d, "e"] \\
D \ar[r, "f"']
&E \ar[r, "g"']
&F
\end{tikzcd}
\]
The following are properties which regard pullbacks of such commutative squares:
\begin{enumerate}[(a)]\setlength\itemsep{0em}
\item If both squares are pullbacks, then the outer-square
  \begin{equation}\label{eq:assoc-outer-pullback}
  \begin{tikzcd}
  A \ar[r, "b a"] \ar[d, "c"']
  \ar[rd, phantom, "\lrcorner", very near start]
  &C \ar[d, "e"] \\
  D \ar[r, "g f"']
  &F
  \end{tikzcd}
  \end{equation}
  is a pullback.

\item If the second square is a pullback and the outer diagram
  \cref{eq:assoc-outer-pullback} is a pullback, then the first square is a
  pullback.
\end{enumerate}
\end{proposition}

\begin{proof}
\begin{enumerate}[(a)]\setlength\itemsep{0em}
\item Let \(x: Z \to D\) and \(y: Z \to C\) be two morphisms of \(\cat C\) such
  that \(g f x = e y\). From the pullback property of the second square, there
  exists a unique morphism \(z: Z \to B\) such that \(b z = y\) and
  \(d z = f x\). Using the pullback property of the first square on the
  morphisms \(z\) and \(x\) we find a unique \(w: Z \to A\) such that
  \(a w = z\) and \(c w = x\).

  Notice that since \(a w = z\) then \(b (a w) = b z = y\) --- we'll show that
  \(w\) is the unique morphism of \(\cat C\) such that \((b a) w = y\) and
  \(c w = x\). Suppose the existence of another morphism \(w': Z \to A\) such
  that \((b a) w' = y\) and \(c w' = x\). In particular, it follows that \(b (a
  w') = b (a w)\), and
  \[
  d (a w') = f (c w') = f x = f (c w) = d (a w).
  \]
  By the uniqueness of the morphism \(Z \to B\) from the second pullback square
  diagram, \(a w' = z = a w\). Using this last equality and the fact that
  \(c w' = x = c w\), by the pullback property of the first square we obtain
  \(w' = w\) --- which finally settles that \(w\) is unique and the outer-square
  is indeed a pullback.

\item We assume the existence of a pullback \((A', c', a')\) of \((f, d)\) and
  show that \(A'\) must be isomorphic to \(A\). From the commutativity of the
  diagram, since \(d a = c f\), we use the pullback property of \((A', c', a')\)
  to get a unique morphism \(h: A \to A'\) --- we'll show that \(h\) is the
  required isomorphism. Via the result of item (a), we know that
  \((A', c, b a')\) is a pullback for the outer-square, therefore it follows
  that \(c' h = c\) and \(b a' h = b a\). Hence from the last two equalities one
  concludes that \(h\) is a factorization between two pullbacks of the
  outer-square. Since \(h\) is unique and pullbacks are unique up to
  isomorphism, it must be the case that \(h\) is an isomorphism --- thus
  \(A \iso A'\), then \((A, c, a)\) is a pullback for the first square.
\end{enumerate}
\end{proof}

\section{Limits \& Colimits}

\begin{definition}[Cone]
\label{def:cone-on-functor}
Let \(F: \cat D \to \cat C\) be a functor. We define a \emph{cone} on \(F\) to
consist of the following data:
\begin{itemize}\setlength\itemsep{0em}
\item An object \(C \in \cat C\).
\item For each object \(D \in \cat D\), a corresponding morphism \(p_D: C \to F
  D\) in \(\cat C\). Moreover, for every morphism \(d: D \to D'\) in \(\cat D\),
  one has that \(p_{D'} = F d \circ p_D\).
\end{itemize}
\end{definition}

\begin{definition}[Limit of a functor]
\label{def:limit-of-functor}
Let \(F: \cat D \to \cat C\) be a functor. We define a \emph{limit} of \(F\) to
be a \emph{cone} \(\Lim F = (L, (p_{D})_{D \in \cat D})\) with the property
that, for any cone \((M, (q_D)_{D \in \cat D})\) on \(F\), there exists a
\emph{unique} morphism \(m: M \to L\) of \(\cat C\) such that, for every
\(D \in \cat D\), the following diagram commutes
\[
\begin{tikzcd}
M \ar[d, dashed, "m"'] \ar[rd, bend left, "q_D"] & \\
\Lim F \ar[r, "p_D"'] &F D
\end{tikzcd}
\]
\end{definition}

\begin{proposition}[Uniqueness]
\label{prop:limit-functor-uniqueness}
The limit of a functor, when existent, is unique up to isomorphism.
\end{proposition}

\begin{proof}
Let \(F: \cat D \to \cat C\) be a functor admitting limit cones
\((L, (p_D)_{D \in \cat D})\) and \((L', (p_D')_{D \in \cat D})\). Then there
exists unique morphisms \(m: L' \to L\) and \(m: L \to L'\) such that
\(p_D m = p_D'\) and \(m' p_D' = p_D\). Stacking these two we find that
\(m m': L \to L\) is such that \(p_D (m m') = p_D\), but since \(\Id_L\) has the
same property, by the uniqueness of the factoring morphism, it follows that
\(m m' = \Id_L\). The same analogous argument goes for \(m' m = \Id_{L'}\). We
conclude that \(L \iso L'\) in \(\cat C\) via \(m\) and \(m'\).
\end{proof}

\begin{proposition}[Parallel factorizations]
\label{prop:equal-parallel-factorizations}
Let \(F: \cat D \to \cat C\) be a functor admiting a limit, and let \(M\) be any
object in \(\cat C\). Two parallel morphisms \(f, g: M \para \Lim F\) in
\(\cat C\) are \emph{equal} if, for every \(D \in \cat D\), one has
\(p_D f = p_D g\).
\end{proposition}

\begin{proof}
Notice that \((M, (p_D f)_{D \in \cat D})\) forms a cone on \(F\), therefore,
from the universal property of the limit of \(F\), we obtain \(f = g\).
\end{proof}

\begin{definition}[Cocone]
\label{def:cocone}
Let \(F: \cat D \to \cat C\) be a functor. We define a \emph{cocone} on \(F\) to
consist of the following data:
\begin{itemize}\setlength\itemsep{0em}
\item An object \(C \in \cat C\).

\item For each object \(D \in \cat D\), a corresponding morphism \(s_C: F D \to
  C\) of \(\cat C\) such that, for every morphism \(d: D' \to D\) in \(\cat D\),
  we have that \(s_{D'} = s_D \circ F d\).
\end{itemize}
\end{definition}

\begin{definition}[Colimit]
\label{def:colimit}
Let \(F: \cat D \to \cat C\) be a functor. We define a \emph{colimit} on \(F\)
to be a \emph{cocone} \(\Colim F = (L, (s_D)_{D \in \cat D})\) such that, for
every cocone \((M, (t_D)_{D \in \cat D})\) on \(F\), there exists a
\emph{unique} morphism \(m: L \to M\) such that, for every \(D \in \cat D\), the
following diagram commutes
\[
\begin{tikzcd}
M \ar[r, "t_D"] &D \\
L \ar[u, dashed, "m"] \ar[ru, bend right, "p_D"'] &
\end{tikzcd}
\]
\end{definition}

\begin{example}[Products]
\label{exp:product-is-limit-over-indexing-set-diagram}
Consider a set \(I\) as a discrete category (as in \cref{exp:index-category}),
and any category \(\cat C\). The limit of a functor \(F: I \to \cat C\), if
existent, is simply a \emph{product} in \(\cat C\), that is
\[
\Lim F \iso \prod_{i \in I} F i.
\]
\end{example}

\subsection{Complete Categories}

\begin{proposition}[All limits]
\label{prop:all-limits-is-preorder}
Let \(\cat C\) be a \(\mathcal{U}\)-category. If for all
\(\mathcal{U}\)-categories \(\cat D\) and functors \(F: \cat D \to \cat C\), the
limit of \(F\) exists in \(\cat C\) --- that is, the category \emph{admits all
  limits} --- then \(\cat C\) is a \emph{preorder class}.
\end{proposition}

\begin{proof}
For \(\cat C\) to be a preorder, there must exist at most one morphism between
every pair of objects of \(\cat C\), so this is what we settle to do. Let
\(A, B \in \cat C\) be any two objects and suppose there exists a pair of
\emph{distinct} parallel morphisms \(f, g: A \para B\). Since every limit
exists, then in particular the product \(B^{|\Hom(\cat C)|}\) is a well defined
object of \(\cat C\). From \(f\) and \(g\), one can create
\(2^{|\Hom(\cat C)|}\) distinct collections of morphisms
\((h: A \to B)_{|\Hom(\cat C)|}\) --- where each \(h\) is either \(f\) or \(g\)
--- which are cones over an ``inclusion'' functor \(F: \cat A \to \cat C\),
where \(\cat A\) is composed of objects \(A\) and \(B\), and morphisms are
\(f, g: A \para B\). The admittance of a limit over this functor is to state the
existence of \(2^{|\Hom(\cat C)|}\) distinct factorizations of
\(A \to B^{|\Hom(\cat C)|}\) in \(\cat C\). Since all of these factorizations
are morphisms of \(\cat C\), it should be the case that
\(2^{|\Hom(\cat C)|} < |\Hom(\cat C)|\), which contradicts Cantor's theorem (see
\cref{thm:cantor-power-set-theorem}) since \(\cat C\) is a
\(\mathcal{U}\)-category. If follows that there cannot exist more than one
morphism between the objects of \(\cat C\), proving that it is a preorder.
\end{proof}

\begin{definition}[Completeness]
\label{def:completeness-categories}
We define the following notions concerning categories and the existence of
limits:
\begin{enumerate}[(a)]\setlength\itemsep{0em}
\item A category \(\cat C\) is said to be \emph{(co)complete} if, for every
  \emph{small category} \(\cat D\), any functor \(F: \cat D \to \cat C\)
  \emph{has a (co)limit} in \(\cat C\).

\item A category \(\cat C\) is said to be \emph{finitely (co)complete} if, for
  every \emph{finite category} \(\cat D\), any functor \(F: \cat D \to \cat C\)
  \emph{has a (co)limit} in \(\cat C\).
\end{enumerate}
\end{definition}

\subsection{Existence Theorem for Limits}

\begin{theorem}
\label{thm:complete-category-prod-eq}
A category \(\cat C\) is \emph{complete} if and only if each collection of
objects, indexed by a set, has a \emph{product} and each pair of parallel
morphisms has an \emph{equalizer}.
\end{theorem}

\begin{proof}
Let \(\cat C\) be complete and \(I\) be any set, which defines a small
category. If \((C_i)_{i \in I}\) is any collection of objects in \(\cat C\), one
can define a functor \(F: I \to \cat C\) by \(F i \coloneq C_i\) and, since
\(F\) has a limit by hypothesis, it follows that
\(\prod_{i \in I} F i = \prod_{i \in I} C_i\) is a product in \(\cat C\). For
the equalizer, let \(f, g: A \para B\) be two parallel morphisms in \(\cat
C\). Define a category \(\cat D\) whose objects are \(A\) and \(B\), and
morphisms are the identities together with both \(f\) and \(g\). Defining \(F:
\cat D \to \cat C\) to be simply an inclusion, since \(F\) has a limit, then
there exists \(\Lim F\) such that
\[
\begin{tikzcd}
\Lim F \ar[r] &A \ar[r, shift left, "f"] \ar[r, shift right, "g"']
&B
\end{tikzcd}
\]
is an equalizer in \(\cat C\), which proves the first proposition.

For the second proposition, suppose that \(\cat C\) admits products and
equalizers. Let \(\cat D\) be a small category and \(F: \cat D \to \cat C\) be
any functor. Define the following:

\begin{itemize}\setlength\itemsep{0em}
\item Consider two pairs of products
  \[
  \bigg(\prod_{D \in \cat D} F D, (r_D)_{D \in \cat D}\bigg)
  \quad\text{ and }\quad
  \bigg(\prod_{f \in \Hom(\cat D)} F(\codom f),
    (q_f)_{f \in \Hom(\cat D)}\bigg),
  \]
  where \(r_{D'}: \prod_{D \in \cat D} F D \to F D'\) and
  \(q_{f'}: \prod_{f \in \Hom(\cat D)} F (\codom f) \to F (\codom f')\) are the
  projections associated with the products.

\item Let
  \(\alpha, \beta: \prod_{D \in \cat D} F D \para \prod_{f \in \Hom(\cat D)} F
  (\codom f)\) be the \emph{unique} factorizations such that
  \[
  q_f \alpha = r_{\codom f}
  \quad\text{ and }\quad
  q_f \beta = F f \circ r_{\dom f}
  \]
  for all \(f \in \Hom(\cat D)\).
\item Since equalizers always exist on \(\cat C\), let
  \((L, \ell) \coloneq \Eq(\alpha, \beta)\). Define a collection
  \((p_D)_{D \in \cat D}\) for which \(p_D \coloneq r_D \ell\). We shall prove
  that \(\Lim F = (L, (p_D)_{D \in \cat D})\).
\end{itemize}

We first prove that \((L, (p_D)_{D \in \cat D})\) is a cone on \(F\). To that
end, consider any morphism \(f: D \to D'\) in \(\cat D\). From the construction
of the collection of morphisms, we have that
\begin{align*}
F f \circ p_D
&= F f \circ (r_D \ell)
= (F f \circ r_D) \ell \\
&= (q_f \beta) \ell
= q_f (\beta \ell) \\
&= q_f (\alpha \ell)
= (q_f \alpha) \ell \\
&= r_{D'} \ell \\
&= p_{D'}
\end{align*}
therefore, \((p_D)_{D \in \cat D}\) satisfies the conditions of a cone on \(F\).

Let \((M, (h_D)_{D \in \cat D})\) be a cone on \(F\). From the product universal
property, there exists a \emph{unique} factorization
\(\gamma: M \to \prod_{D \in \cat D} F D\) such that \(r_D \gamma = h_D\). On
the other hand, for any morphism \(f: D \to D'\) of \(\cat D\) we have
\begin{align*}
  (q_f \alpha) \gamma
  = r_{D'} \gamma
  = h_{D'}
  = F f \circ h_D
  = F f \circ (r_D \gamma)
  = (F f \circ r_D) \gamma
  = (q_f \beta) \gamma
\end{align*}
Since the factorization on the product \(\prod_{f \in \Hom(\cat C)} F(\codom
f)\) is unique, it follows that \(\alpha \gamma = \beta \gamma\). Since \(L\) is
the object of the equalizer of \(\Eq(\gamma, \beta)\), there must exist a unique
factorization \(u: M \to L\) such that \(\ell u = \gamma\). Therefore, one has
\[
p_D u  = (r_D \ell) u = r_D (\ell u) = r_D \gamma = h_D,
\]
showing that \(u\) is a factorization of of the cone \(M\) via the cone \(L\) on
\(F\).

We must show that the factorization \(u\) is unique, so that \(L\) is the limit
of \(F\). Suppose \(v: M \to L\) is another factorization, that is, \(p_D v =
h_D\) for all \(D \in \cat D\). Notice that
\[
r_D (\ell u)
= (r_D \ell) u
= p_D u
= h_D
= p_D v
= (r_D \ell) v
= r_D (\ell v)
\]
holds for all \(D \in \cat D\), therefore both \(\ell u\) and \(\ell v\) are
parallel factorizations factorizations --- which by
\cref{prop:equal-parallel-factorizations} implies \(\ell u = \ell v\).
From \cref{prop:eq-monic-coeq-epic} we know that \(\ell\) is monic, therefore
\(u = v\). This shows the uniqueness of \(u\), thus we may conclude that
\[
\Lim F = (L, (p_D)_{D \in \cat D}).
\]
\end{proof}

\begin{proposition}[Finite completeness]
\label{prop:equiv-cat-finitely-complete}
Let \(\cat C\) be a category. The following properties are equivalent:
\begin{enumerate}[(a)]\setlength\itemsep{0em}
\item The category \(\cat C\) is \emph{finitely complete}.
\item The category \(\cat C\) has a \emph{terminal object, binary
    products and equalizers}.
\item The category \(\cat C\) has a \emph{terminal object, and pullbacks}.
\end{enumerate}
\end{proposition}

\begin{proof}
\begin{itemize}\setlength\itemsep{0em}
\item If \(\cat C\) is finitely complete, then from definition items (b) and (c)
  hold.
\item If (b) holds, then by \cref{prop:product-ordering-independent} we find
  that any \emph{finite} collection has a product. Since \(\cat C\) has
  equalizers, by \cref{thm:complete-category-prod-eq} and taking the particular
  case where \(\cat D\) is a finite category, we conclude that \(\cat C\) is
  finitely complete, therefore (b) implies (a).
\item If (c) holds, let \(1 \in \cat C\) be the terminal object. For any two
  objects \(A, B \in \cat C\), the pullback
  \[
  \begin{tikzcd}
  P \ar[r] \ar[d] \ar[rd, phantom, very near start, "\lrcorner"]
  &A \ar[d, dashed] \\
  B \ar[r, dashed] &1
  \end{tikzcd}
  \]
  is the product of \(A\) with \(B\) in \(\cat C\) --- thus binary products
  exist. Now consider two parallel morphisms \(f, g: A \para B\) in \(\cat
  C\). In this case, one can consider the pullback
  \[
  \begin{tikzcd}
  E \ar[rr, "k"]  \ar[d, "\ell"']
  \ar[rrd, phantom, very near start, "\lrcorner"]
  &&A \ar[d, "\Id_A \times f"] \\
  A \ar[rr, "\Id_A \times f"']
  &&A \times B
  \end{tikzcd}
  \]
  Then for \(E\) to be the equalizer of \((f,g)\) it suffices to show that
  \(k = \ell\) and \(f k = g k\). For the former, notice that
  \[
  k = \pi_A \circ (\Id_A \times f) \circ k
  = \pi_A \circ (\Id_A \times g) \circ \ell
  = \ell.
  \]
  For the last equality, we proceed with a similar argument
  \[
  f k = \pi_B \circ (\Id_A \times f) \circ k
  = \pi_B \circ (\Id_A \times g) \circ \ell
  g \ell
  = g k.
  \]
  Therefore \(\Eq(f, g) = (E, k)\), and \(\cat C\) has equalizers. This proves
  that (c) implies (b).
\end{itemize}
\end{proof}

\begin{proposition}
\label{prop:equiv-def-cone-on-func-from-graph}
Let \(F: \cat D \to \cat C\) be a functor, and suppose there exists a collection
\((f_j)_{j \in J}\) of morphisms of \(\cat D\) \emph{generating} every other
morphism of \(\cat D\) --- that is, if \(g\) is a morphism in \(\cat D\), then
\(g\) can be written as the composition of \emph{finitely} many arrows of
\((f_j)_{j \in J}\). A \emph{cone} on the functor \(F\) is a pair \((L, (p_D: L
\to F D)_{D \in \cat D})\) such that, for any morphism \(f_j: D \to D'\), we
have \(p_{D'} = F f_j \circ p_D\).
\end{proposition}

\begin{proof}
To see that this is equivalent to the definition of a cone is simple since,
given any \(g: D \to \cat D'\) in \(\cat D\), one has
\(g = f_{j_n} \dots f_{j_1}\) for a finite collection of maps
\(f_{j_i} \in (f_j)_{j \in J}\) where \(f_{j_i}: D_{i-1} \to D_i\) with
\(D_0 = D\) and \(D_n = D'\). Then
\begin{align*}
F g \circ p_D
&= F (f_{j_n} \dots f_{j_1}) \circ p_D \\
&= (F f_{j_n} \circ \dots \circ F f_{j_1}) \circ p_D \\
&= (F f_{j_n} \circ \dots \circ F f_{j_2}) \circ (F f_{j_1} \circ p_D) \\
&= (F f_{j_n} \circ \dots \circ F f_{j_2}) \circ p_{D_1} \\
&= \dots \\
&= F f_{j_n} \circ p_{D_{n-1}}\\
&= p_{D'},
\end{align*}
which proves the equivalence.
\end{proof}

\begin{definition}[Finitely generated categories]
\label{def:finitely-generated-category}
A category \(\cat C\) is said to be \emph{finitely generated} if \(\cat C\) is
comprised of finitely many objects, and there exists a finite set of
morphisms \(\{f_1, \dots, f_n\}\) of \(\cat C\) such that every morphism of
\(\cat C\) can be written as a composition of finitely many arrows \(f_j\).
\end{definition}

\begin{proposition}
\label{prop:fin-gen-to-fin-complete-has-limits}
Let \(\cat D\) be \emph{finitely generated} and \(\cat C\) be \emph{finitely
  complete}, then \emph{any} functor \(F: \cat D \to \cat C\) \emph{has a
  limit}.
\end{proposition}

\begin{proof}
Since \(\cat C\) is finitely complete, we can recycle the proof of
\cref{thm:complete-category-prod-eq} replacing the pair of products with the
\emph{finite} products \(\prod_{D \in \cat D} F D\) and
\(\prod_{j=1}^n F(\codom f_j)\).
\end{proof}

\subsection{Limit Preserving Functors}

\begin{definition}[Limit preserving functor]
\label{def:limit-preserving-functor}
A functor \(F: \cat B \to \cat C\) is said to \emph{preserve limits} if --- for
every small category \(\cat A\) and functor \(G: \cat A \to \cat B\) --- the
limit of \(G\) \emph{exists} and the limit of \(F G: \cat A \to \cat C\) exists
and is given by
\[
\Lim(F G) = F(\Lim G).
\]
To put more concretely, if \((L, (p_A)_{A \in \cat A})\) is the limit of \(G\),
then the limit of \(F G\) is given by \((F L, (F p_A)_{A \in \cat A})\).
\end{definition}

\subsection{Absolute limits}

\begin{proposition}[Absolute pushout]
\label{prop:absolute-pushout}
Let \(\cat C\) be a category, and morphisms \(p: A \epi C\) and \(q: A
\epi B\) in \(\cat C\) such that the following square commutes in \(\cat C\)
\[
\begin{tikzcd}
A \ar[r, "q"] \ar[d, "p"'] &B \ar[d] \\
C \ar[r] &D
\end{tikzcd}
\]
If \(p\) is a \emph{split epimorphism} and \(q\) is an \emph{epimorphism},
then the square is an \emph{absolute pushout}.
\end{proposition}

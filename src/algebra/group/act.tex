\section{Group Actions on Sets}

\begin{definition}[Group action]
\label{def:group-action}
Let \(G\) be a group and \(A \in \cat C\) be an object in some category
\(\cat C\). An \emph{action} of \(G\) on the object \(A\) is a \emph{group
  morphism}
\[
G \longrightarrow \Aut_{\cat C}(A),
\]
that is, the elements of \(G\), which are automorphisms in the category
\(\cat{B}G\), define automorphisms in \(A\)---this shows that group actions
are functors \(\cat{B}G \to \cat C\) (see \cref{exp:grp-action} for more). Left
actions are covariant functors, while right actions are contravariant.
Naturally the action is said to be \emph{faithful} if the functor is faithful,
that is, if \(G \mono \Aut_{\cat C}(A)\) is injective.
\end{definition}

\begin{definition}
\label{def:free-effective-action}
Let \(G\) be a group acting on a \emph{set} \(X\). We define the following
concepts concerning such action:
\begin{enumerate}[(a)]\setlength\itemsep{0em}
\item The action is said to be \emph{free} if the identity \(e_G\) is the only
  element fixing \emph{any} of the elements of \(X\). In other words, \(G\) acts
  freely on \(X\) if given any \(g \in G\) for which there exists \(x \in X\)
  with \(g \cdot x = x\), then \(g = e\) is the identity element.

\item An action is said to be \emph{effective} if the \emph{only} member of
  \(G\) acting trivially is the identity \(e_G\).
\end{enumerate}
\end{definition}

\begin{remark}
\label{rem:free-versus-effective-actions}
Mind you, free actions are effective, but not all effective actions are free.
\end{remark}

\begin{definition}[Conjugation]
\label{def:group-conjugation-action}
Let \(G\) be a group. We define a \emph{conjugation} to be a group action of
\(G\) on itself, \(G \times G \to G\), mapping
\[
(g, h) \longmapsto g h g^{-1}.
\]
\end{definition}

\begin{theorem}[Cayley]
\label{thm:cayley-faithful}
Every group acts faithfully on \emph{some} set. That is, there exists a set
\(A\) and an injective action \(G \mono \Aut_{\Set}(A) = S_A\)---where \(S_A\)
is the symmetry group of \(A\)---making \(G\) a subgroup of \(S_A\).
\end{theorem}

\begin{proof}
Simply consider the action \(G \to \Aut_{\Grp}(G)\) mapping \(g \mapsto f_g\),
where \(f_g(x) = g x\) is the left-multiplication by \(g\). This action is
certainly faithful, making \(G\) isomorphic to a subgroup of \(\Aut_{\Grp}(G)\).
\end{proof}

\begin{definition}[Opposite group]
\label{def:opposite-group}
Given a group \(G\), we define its \emph{opposite group} \(G^{\op}\) to be
composed of the elements of \(G\), and endowed with a contravariant action \(G
\times G \to G\) mapping \((g, h) \mapsto g \cdot h \coloneq h g\).
\end{definition}

\begin{corollary}
\label{cor:Gop-iso-G-id-iff-commutative}
The following are properties relating a group \(G\) with its opposite group:
\begin{enumerate}[(a)]\setlength\itemsep{0em}
\item The set-function \(\phi: G^{\op} \to G\) mapping \(g \mapsto g\) is an
  isomorphism of groups \(G^{\op} \iso G\) if and only if \(G\) is commmutative.

\item There exists a natural isomorphism of groups \(G \iso G^{\op}\) even when
  \(G\) is non-commutative.
\end{enumerate}
\end{corollary}

\begin{proof}
We first prove item (a). If the said map is an isomorphism, then for all
\(g, h \in G\) we have
\[
g h = \phi(g h) = \phi(h \cdot g) = \phi(h) \phi(g) = h g,
\]
thus \(G\) is commutative. Conversely, if we assume that \(G\) is commutative
then the map, besides being bijective, is also a group morphism, since
\[
\phi(g \cdot h) = \phi(h g) = h g = g h = \phi(g) \phi(h).
\]

For item (b), define a map \(\psi: G \to G^{\op}\) to be given by
\(g \mapsto g^{-1}\), which is bijective since inverses are unique. The
set-function \(\psi\) is also a group morphism since
\[
\psi(g h) = (g h)^{-1} = h^{-1} g^{-1}
= g^{-1} \cdot h^{-1} = \psi(g) \cdot \psi(h).
\]
Therefore \(G \iso G^{\op}\) via \(\psi\).
\end{proof}

\begin{proposition}[Left to right \& back]
\label{prop:left-and-right-actions}
Let \(G\) be a group and \(A\) be a set. Given any \emph{left-action}
\(\sigma: G \to \Aut_{\cat C}(A)\), we can turn \(\sigma\) into a unique
corresponding \emph{right-action} \(\sigma^{\op}: G \to \Aut_{\cat C}(A)\) given
by \(\sigma^{\op}(g)(a) = \sigma(g^{-1})(a)\) for any \(g \in G\) and
\(a \in A\). The conversion of a right-action into a left-action is also unique.
\end{proposition}

\begin{proof}
Let \(g, h \in G\) and \(a \in A\) be any elements. The map \(\sigma^{\op}\) is
indeed a right action:
\[
\sigma^{\op}(g h)(a)
= \sigma((gh)^{-1})(a)
= \sigma(h^{-1} g^{-1})(a)
= \sigma(h^{-1})(\sigma(g^{-1})(a))
= \sigma^{\op}(h)(\sigma^{\op}(g)(a)).
\]
The uniqueness comes from the fact that inverses are unique.
\end{proof}

\begin{definition}[Transitive action]
\label{def:transitive-action}
A group \(G\) is said to act \emph{transitively} on a set \(A\) if for all pairs
of elements \(a, b \in A\), there exists a group element \(g\) such that
\(g a = b\)
\end{definition}

\begin{definition}[Orbit \& Stabilizer]
\label{def:orbit-and-stabilizer}
Given a group action \(G \to \Aut_{\Set}(A)\) on a set \(A\), the \emph{orbit}
of an element \(a \in A\) under the action of \(G\) is defined to be the
set
\[
\Orb_G(a) \coloneq \{g a \colon g \in G\} \subseteq A.
\]
The \emph{stabilizer} of \(a\) under the action of \(G\) is the subgroup
\[
\Stab_G(a) \coloneq \{g \in G \colon g a = a\} \subseteq G.
\]
\end{definition}

Let \(\sigma\) be any action of \(G\) on \(A\). Stabilizers indeed define a
subgroup of \(G\), notice that if \(g, h \in \Stab_G(a)\), then
\(\sigma(g h)(a) = \sigma(g)(\sigma(h)(a)) = \sigma(g)(a) = a\) thus
\(g h \in \Stab_G(a)\). Moreover, since \(\sigma\) is a group morphism, we have
\(\sigma(g^{-1})(a) = \sigma(g)^{-1}(a) = a\) thus \(g^{-1} \in
\Stab_G(a)\).

\begin{definition}[\(\GSet{G}\) category]
\label{def:G-set-category}
Given a group \(G\), we define a category \(\GSet{G}\) whose objects are pairs
\((\sigma, A)\)---where \(A\) is a set, and \(\sigma: G \to \Aut_{\Set}(A)\)
is a group action on \(A\)---and morphisms \(\phi: (\sigma, A) \to (\rho, B)\)
are set-functions \(\phi: A \to B\) such that the following diagram commutes
\[
\begin{tikzcd}
G \times A \ar[d, "\sigma"'] \ar[rr, "\Id_G \times \phi"]
&&G \times B \ar[d, "\rho"] \\
A \ar[rr, "\phi"'] &&B
\end{tikzcd}
\]
That is, \(\phi(\sigma(g)(a)) = \rho(g)(\phi(a))\)---or put even more simply
as \(\phi(g a) = g \phi(a)\) when there is no chance of confusion. These
functions are called \emph{\(G\)-equivariant}.
\end{definition}

\begin{proposition}
\label{prop:transitive-G-set-iso-left-mul-G/Stab}
Let \((\sigma, A) \in \GSet{G}\), where \(A\) is non-empty and \(\sigma\)
is a \emph{transitive} left-action on \(A\). Then there exists an
\emph{isomorphism}
\[
(\sigma, A) \iso (\ell, G/\Stab_{\ell}(a)),
\]
in \(\GSet{G}\), where \(\ell\) is the left-multiplication of \(G\)
on \(G/\Stab_{\ell}(a)\), and \(a \in A\) is \emph{any} element.
\end{proposition}

\begin{proof}
For the sake of brevity, define \(S_a \coloneq \Stab_{\ell}(a)\). Let
\(a \in A\) be any element, and define a \emph{set}-function
\(\phi: G/S_a \to A\) given by \(\phi(g S_a) \coloneq g a\). This function is
well defined since, for any \(g S_a = g' S_a\) we have \(g'^{-1} g \in S_a\)
thus \((g'^{-1} g) a = a\), which implies in \(g a = g' a\). To show that
\(\phi\) is bijective, we construct its inverse: define \(\psi: A \to G/S_a\) by
mapping \(g a \mapsto g S_a\), which is well defined because if \(g a = g' a\)
then \(g'^{-1} g a = a\) thus \(g'^{-1} g \in S_a\) so that \(g S_a = g' S_a\)
by definition. One sees easely that \(\phi\) and \(\psi\) are inverses of each
other. Finally, the map \(\phi\) is also equivariant since
\[
\phi(g(g' S_a)) = \phi((g g') S_a) = (g g') a = g(g'a) = g \phi(g' S_a).
\]
\end{proof}

\begin{corollary}
\label{cor:orbit-divides-order-of-group}
Let \(G\) be a \emph{finite group} and \(A\) be a set. For \emph{any} action of
\(G\) on a group \(A\), the orbit \(\Orb_G(a)\) of \emph{any} element
\(a \in A\) is a \emph{finite subset of \(A\)}. Moreover the we have that
\(|\Orb_G(a)|\) \emph{divides} \(|G|\).
\end{corollary}

\begin{proof}
Given a \(G\)-set \((\sigma, A)\) for any action \(\sigma\), if \(a \in A\) is
any element, then the restriction
\[
\overline{\sigma}: G \longrightarrow \Aut_{\Set}(\Orb_G(a)),
\]
is a transitive action by the definition of the orbit---where
\(\overline{\sigma}(g)(g' a) \coloneq \sigma(g)(g a)\) for all \(g \in G\) and
\(g' a \in \Orb_G(a)\).

Therefore by \cref{prop:transitive-G-set-iso-left-mul-G/Stab} we have
\((\overline{\sigma}, \Orb_G(a)) \iso (\ell, G/\Stab_G(g a))\) in
\(\GSet{G}\), for any \(g a \in \Orb_G(a)\)---which means the existence of a
\(G\)-equivariant bijection between the sets \(\Orb_G(a)\) and
\(G/\Stab_G(g a)\). From \cref{prop:grp-index-subgroup} we have
\(|G/\Stab_G(g a)| = |G|/|\Stab_G(g a)|\) and therefore
\[
|\Orb_G(a)| \cdot |\Stab_G(g a)| = |G|.
\]
\end{proof}

\begin{theorem}
\label{thm:stabilizers-conjugation}
Let \(G\) be a group acting on a set \(A\). Consider elements \(a \in A\),
\(g \in G\), and define \(b \coloneq g a\). It follows that
\[
\Stab_G(b) = g \Stab_G(a) g^{-1}.
\]
\end{theorem}

\begin{proof}
Let \(h \in \Stab_G(a)\) be any element, then
\[
(g h g^{-1}) b = (g h) (g^{-1} b) = (g h) a = g (h a) = g a = b,
\]
therefore \(g h g^{-1} \in \Stab_G(a)\). Now if \(\ell \in \Stab_G(b)\), we have
\(\ell b = b\) but since \(b = g a\) then \(\ell (g a) = g a\) thus multiplying
by \(g^{-1}\) in both sides we obtain \((g^{-1} \ell g) a = a\). Therefore,
\(\Stab_G(b) = g \Stab_G(a) g^{-1}\).
\end{proof}

\section{Topological Groups}

\subsection{Construction and Properties}

\todo[inline]{Study topological groups on Dieck}

\subsection{Actions on Topological Spaces}

\begin{definition}[Orbit space]
\label{def:orbit-space}
Let \(X\) be a topological space, and a topological group \(G\) together with a
left action on \(X\). We define the \emph{orbit space} of the \(G\)-space \(X\)
to be the set
\[
X/G \coloneq X/{x \sim \Orb_G(x)}
\]
together with the quotient topology given by the canonical projection
\(X \epi X/G\).
\end{definition}

\begin{definition}[\(G\)-stable]
\label{def:G-stable}
Given a \(G\)-space \(X\), a subset \(A \subseteq X\) is said to be \(G\)-stable
if for every pair \((g, a) \in G \times A\) we have \(g a \in A\).
\end{definition}

\section{Group Objects}

\begin{definition}[Group object]
\label{def:group-object}
Let \(\cat C\) be a category with (finite) products and with a terminal object
\(1\). A triple \((G, m, e, i)\) is said to be a \emph{group object} of
\(\cat C\) if:
\begin{itemize}\setlength\itemsep{0em}
\item \(G\) is an object of \(\cat C\).

\item \(m: G \times G \to G\) is a morphism of \(\cat C\) satisfying the
  commutativity of
  \[
  \begin{tikzcd}
  (G \times G) \times G \ar[d, "\dis"'] \ar[rr, "m \times \Id_G"]
  &
  &G \times G \ar[rr, "m"]
  &
  &G \ar[d, equals]
  \\
  G \times (G \times G) \ar[rr, "\Id_G \times m"]
  &
  &G \times G \ar[rr, "m"]
  &
  &G
  \end{tikzcd}
  \]
  in \(\cat C\). This arrow defines the notion of multiplication of group
  members.

\item \(e: 1 \to G\) is a morphism of \(\cat C\) making the diagram
  \[
  \begin{tikzcd}
  1 \times G \ar[rr, "e \times \Id_{G}"]
  \ar[rrd, "\dis"']
  &
  &G \times G \ar[d, "m"]
  &
  &G \times 1 \ar[ll, "\Id_G \times e"']
  \ar[lld, "\dis"]
  \\
  &
  &G
  &
  \end{tikzcd}
  \]
  commute in \(\cat C\). The map \(e\) defines the notion of a neutral member
  of the group object.

\item \(i: G \to G\) is a morphism of \(\cat C\) such that---if \(\Delta
  \coloneq \Id_G \times \Id_G\) is the diagonal morphism of \(G\)---the diagram
  \[
  \begin{tikzcd}
  G \ar[rr, "\Delta"] \ar[d, dashed]
  &
  &G \times G \ar[rr, "\Id_G \times i"]
  &
  &G \times G \ar[d, "m"]
  %
  &
  &G \times G \ar[ll, "i \times \Id_G"']
  &
  &G \ar[d, dashed] \ar[ll, "\Delta"']
  \\
  1 \ar[rrrr, "e"']
  &
  &
  &
  &G
  &
  &
  &
  &1 \ar[llll, "e"]
  \end{tikzcd}
  \]
  is commutative in \(\cat C\). The map \(i\) defines the notion of inversion of
  members of the group.
\end{itemize}
\end{definition}

%%% Local Variables:
%%% mode: latex
%%% TeX-master: "../../deep-dive"
%%% End:

\section{Topology}

\begin{definition}[Topology]\label{def:topology}
    Let \(X\) be a set and \(\tau \subseteq 2^X\). We say that \(\tau\)
    is a topology for \(X\) if the following properties are satisfied
    \begin{enumerate}[(T1)]\setlength\itemsep{0em}
        \item\label{item:topology-T1}
        \(X, \emptyset \in \tau\).

        \item\label{item:topology-T2}
        The arbitrary union of elements of \(\tau\) is an element of
        \(\tau\).

        \item\label{item:topology-T3}
        The finite intersection of elements of \(\tau\) is an element of
        \(\tau\).
    \end{enumerate}
    The elements of the topology \(\tau\) are called open sets of \(X\).
\end{definition}

\begin{example}\label{exp:some-topologies}
    We proceed by listing some examples of topologies that are somewhat interesting,
    they are included here in order to familiarize the reader with the possible
    constructions for the topology on a given set \(X\):
    \begin{itemize}\setlength\itemsep{0em}
        \item \(\tau_1 = \{U \subseteq X \colon U = \emptyset \text{ or
              } X \setminus U \text{ is finite}\}\) is the \emph{cofinite topology} on \(X\).

        \item \(\tau_2 = \{U \subseteq X \colon U = \emptyset \text{ or } X \setminus U
              \text{ is countable}\}\) is the \emph{cocountable topology} on \(X\).

        \item Let \(p \in X\), then \(\tau_3 = \{U \subseteq X \colon U = \emptyset
              \text{ or } p \in U\}\) is the \emph{particular point topology} on \(X\).

        \item Let \(p \in X\), then \(\tau_4 = \{U \subseteq X \colon U = X \text{ or }
              p \not\in U\}\) is the \emph{excluded point topology} on \(X\).
        \item The collection \(2^X\) forms what is called the \emph{discrete topology}
              on \(X\).
    \end{itemize}
\end{example}

\begin{definition}[Comparing topologies]
    \label{def:coarser-finer-topology}
    Let \(X\) be a set, and \(\tau\) and \(\tau'\) be topologies on
    \(X\). If \(\tau' \supseteq \tau\), then we say that
    \begin{itemize}\setlength\itemsep{0em}
        \item \(\tau'\) is finer than \(\tau\). Moreover, if \(\tau\)
              is strictly contained in \(\tau'\), we say that \(\tau'\) is
              strictly finer than \(\tau\)

        \item \(\tau\) is coarser than \(\tau'\). Moreover, if \(\tau\)
              is strictly contained in \(\tau'\), we say that \(\tau'\) is
              strictly coarser than \(\tau\)
    \end{itemize}
    In general, we say that two topologies are comparable if either of them contains
    the other.
\end{definition}

\subsection{Topological Basis}

\begin{definition}[Basis]\label{def: base}
    Let \(X\) be a topological space. A collection \(\mathcal B \subseteq 2^X\) is
    said to be a basis for the topology of \(X\) if it satisfies the following
    \begin{enumerate}[(B1)]\setlength\itemsep{0em}
        \item Every element of \(\mathcal B\) is an open set of \(X\).
        \item Every open subset \(U \subseteq X\) can be written as a union of
              elements of \(B\), that is, exists \(\{B_i\}_{i \in I} \subseteq \mathcal
              B\) for which \(U = \bigcup_{i \in  I} B_i\).
    \end{enumerate}
\end{definition}

\begin{proposition}[Necessary and sufficient condition for a basis]
    \label{prop:equivalent-basis}
    Let \(X\) be a set and \(\mathcal B \subseteq 2^X\). Then \(\mathcal B\) is a
    basis for some topology of \(X\) if and only if it satisfies
    \begin{enumerate}[B1]\setlength\itemsep{0em}
        \item\label{item:basis-union-property}
        \(X = \bigcup_{B \in \mathcal B} B\).

        \item\label{item:basis-intersection-property}
        If \(x \in A \cap B\), where \(A, B \in \mathcal B\), then there exists
        \(C \in \mathcal B\) such that \(x \in C \subseteq A \cap B\).
    \end{enumerate}
\end{proposition}

\begin{proof}
    Let \(\mathcal{B}\) be a basis for the space \(X\). Let \(p \in X\) be any point
    and let \(U \subseteq X\) be a neighbourhood of \(x\). From the definition of a
    basis, there exists a subcollection of sets such that their union equals
    \(U\), which implies in the existence of \(B \in \mathcal B\) such that \(x \in
    B\). Moreover, since \(B \subseteq X\) for all \(B \in \mathcal B\), it follows
    that \(X = \bigcup_{B \in \mathcal B} B\). Let \(A, B \subseteq \mathcal B\) be
    any intersecting sets of \(\mathcal B\) and take any point \(p \in A \cap
    B\). Notice that \(A \subseteq B \subseteq X\) is open, hence there exists a
    subcollection of open sets of the basis \(\mathcal B\) whose union equals \(A
    \cap B\). It is immediate that there exist \(C \in \mathcal B\) such that \(p
    \in C\) and necessarily \(C \subseteq A \cap B\).

    Let \(\mathcal B \subseteq 2^X\) satisfying both conditions specified above. We
    first show that \(\mathcal B\) is a collection of open sets. Let \(\tau\)
    be the collection of all possible unions of sets of \(\mathcal B\). From the
    first property, \(X \in \tau\) and clearly \(\emptyset \in \tau\)
    --- satisfying \cref{item:topology-T1}. From construction, unions of sets in
    \(\tau\) are unions of unions of sets of \(\mathcal B\), which is
    certainly contained in \(\tau\) --- hence the collection satisfies
    \cref{item:topology-T2}. Let \(T, T' \in \tau\) be any intersecting sets
    and, for every \(p \in T \cap T'\), choose any \(B, B' \subseteq \mathcal B\)
    such that \(p \in B \subseteq T\) and \(p \in B' \subseteq T'\) --- which are
    ensured to exist. From the second property of \(\mathcal B\), there exists \(C
    \in \mathcal B\) such that \(p \in C \subseteq B \cap B'\), thus \(C \subseteq B
    \cap B'\) and, in particular \(C \subseteq T \cap T'\). We can see that \(T \cap
    T'\) is again the union of a collection of elements of \(\mathcal B\), hence
    \(\tau\) is closed under finite intersections, satisfying
    \cref{item:topology-T1}. We can now finally conclude that \(\tau\) is a
    topology on \(X\) and hence \(\mathcal B\) is composed of open sets of \(X\) ---
    and, even better than that, \(\tau\) is the unique topology generated by
    \(\mathcal B\).
\end{proof}

\begin{definition}[Subbase]\label{def: subbase}
    Let \((X, \tau)\) be a topological space. A collection \(\mathcal S
    \subseteq \tau\) is called a subbase for \((X, \tau)\) if the
    collection of all finite intersections \(U_1 \cap \dots \cap U_n\), where
    \(U_i \in \mathcal S\), is a base for \((X, \tau)\).
\end{definition}

\begin{definition}[Weight]\label{def: weight}
    Let \(X\) be a topological space and \(\mathfrak B\) be the collection of
    all bases for the topology of \(X\). We define the weight of \(X\) as
    \[
        w(X) = \min_{\mathcal B \in \mathfrak B} |\mathcal B|.
    \]
\end{definition}

\begin{definition}[Basis at a point]\label{def: basis at a point}
    Let \(X\) be a topological space and \(p \in X\) be any fixed point. We define
    the collection \(\mathcal B_p \subseteq 2^X\) of neighbourhoods of \(p\) to be
    the \emph{neighbourhood basis for the topology of \(X\) at \(p\)} if for any
    neighbourhood \(U_p \subseteq X\), there exists \(B \in \mathcal B_p\) such
    that \(B \subseteq U_p\).
\end{definition}

\begin{definition}\label{def: character}
    Let \((X, \tau)\) be a topological space. Let \(x \in X\) be any point
    and consider \(\mathfrak B_x\) the collection of all bases at \(x\). Then we
    define the character of \(X\) at the point \(x\) as
    \[
        \chi(x, (X, \tau))
        = \min_{\mathcal B_x \in \mathfrak B_x} |\mathcal B_x|
    \]
\end{definition}


\subsection{Closed and Open Sets}

The notion of a closed set and an open set are closely related --- pardon for
the pun. They have a dual relationship, allowing for us to define topologies via
either of them. Lets first define what we mean by a closed set.

\begin{definition}[Closed set]\label{def:closed-set}
    Let \(X\) be a topological space. We define a set \(A \subseteq X\) to be
    closed if \(X \setminus A\) is open.
\end{definition}

The following proposition realizes the idea that the duality of open and closed
sets allow us to work with topological spaces by analysing both open and closed
elements of the space of interest.

\begin{proposition}\label{prop:equiv-closed-topology}
    If \(X\) is a topological space, then
    \begin{enumerate}\setlength\itemsep{0em}
        \item The sets \(X\) and \(\emptyset\) are closed.

        \item The finite union of closed sets is closed.

        \item The arbitrary intersection of closed sets is closed.
    \end{enumerate}
\end{proposition}

\begin{proof}
    Notice that \(X \setminus X = \emptyset\) and \(X \setminus \emptyset = X\) are
    both open sets from \cref{def:topology}, hence \(X\) and \(\emptyset\) are
    closed. Let \(\{C_{j}\}_{j=1}^n\) be a finite collection of closed sets, then
    \(X \setminus \bigcup_{j=1}^n C_j = \bigcap_{j=1}^n X \setminus C_j\) but since
    \(X \setminus C_j\) is open for all \(j\), then their finite intersection is
    open and hence \(X \setminus \bigcup_{j=1}^n C_j\) is also open, which implies
    by definition that \(\bigcup_{j=1}^n C_j\) is closed. Consider now any
    collection of closed sets \(\{C_{j}\}_{j \in J}\) --- where \(J\) is possibly
    infinite. Then \(X \setminus \bigcap_{j \in J} C_j = \bigcup_{j \in J} X
    \setminus C_j\) is open by the arbitrary union of open sets being open, hence
    \(\bigcap_{j \in J} C_j\) is closed.
\end{proof}

We now define four important operations on sets of a topological space, which
will accompany us for the rest of these notes on general point-set topology.

\begin{definition}
    Let \(X\) be a topological space and \(A \subseteq X\) be a set. We define
    \begin{enumerate}[(a)]\setlength\itemsep{0em}
        \item\label{def: closure}
        The closure of \(A\) is the least closed set, \(\Cl A\), that
        contains \(A\). This can be equivalently described as
        \[
            \Cl A = \bigcap \{F \subseteq X \colon A \subseteq F \text{ and } F
            \text{ is closed}\}.
        \]

        \item\label{def: interior}
        The interior of \(A\) is the biggest open set \(\Int A\) contained in
        \(A\). That is
        \[
            \Int A = \bigcup \{U \subseteq X \colon U \subseteq A, U \text{ is open}\}.
        \]

        \item\label{def: exterior}
        The exterior of \(A\) in \(X\) is defined as
        \[
            \Ext A = X \setminus \Cl A.
        \]

        \item\label{def: boundary}
        The boundary of \(A\) in \(X\) is defined as
        \[
            \Bd A = X \setminus (\Int A \cup \Ext A).
        \]
    \end{enumerate}
\end{definition}

\begin{remark}
    For the remainder of this section, unless specified on the contrary, we let
    \(X\) be any topological space and \(A \subseteq X\) be any subset of \(X\).
\end{remark}

\begin{proposition}\label{prop:open-close-int-ext-closure-boundary}
    \(\Int A\) and \(\Ext A\) are open sets, while on the other hand \(\Cl A\)
    and \(\Bd A\) are closed sets.
\end{proposition}

\begin{proof}
    Since \(\Int A\) is the union of a collection of open sets, \(\Int A\) is
    open. \(\Cl A\) is the intersection of a collection of closed sets, hence
    \(\Cl A\) is closed. The exterior set \(\Ext A\) is simply the complement
    of a closed set, hence it's open. The boundary of \(A\) is the complement of the
    union of open sets (which is open), hence \(\Bd A\) is closed.
\end{proof}

\begin{proposition}
    \label{prop:equiv-open-and-closed-set}
    The following are equivalences on the definitions of open and closed sets.
    \begin{enumerate}[(a)]\setlength\itemsep{0em}
        \item Open set equivalences:
              \begin{itemize}\setlength\itemsep{0em}
                  \item \(A\) is open.

                  \item \(A = \Int A\).

                  \item \(A\) contains none of its boudary points, i.e. \(A \cap \Bd A =
                        \emptyset\).

                  \item For all \(x \in A\) there exists a neighbourhood \(U \subseteq A\)
                        containing \(x\).
              \end{itemize}

        \item Closed set equivalences:
              \begin{itemize}\setlength\itemsep{0em}
                  \item \(A\) is closed.

                  \item \(A = \Cl A\).

                  \item For all \(x \in X \setminus A\), there exists a neighbourhood \(U
                        \subseteq X \setminus A\) of \(x\).
              \end{itemize}
    \end{enumerate}
\end{proposition}

\begin{proof}
    \begin{enumerate}[(a)]\setlength\itemsep{0em}
        \item Let \(A\) be open, then, in particular,
              \(A \in \{U \subseteq X \colon U \subseteq A, U \text{ is open}\}\), thus
              \(A = \Int A\). Moreover, from the definition of \(\Bd A\) it follows that
              \(A \cap \Bd A = \Int A \cap A \setminus (\Int A \cup \Ext A) =
              \emptyset\). On the other hand, if \(p \in A\) is any point, then \(A\) itself
              is a neighbourhood of \(p\).

              In order to finish the equivalence chain, let \(A\) be such that all of its
              points have a neighbourhood contained in \(A\). We can then define the
              collection of neighbourhoods
              \(\mathcal U = \{U_p \subseteq A \colon p \in A, p \in U\}\). Notice that
              \(A \subseteq \bigcup_{U \in \mathcal U} U\) and the opposite inclusion is
              clearly true --- thus \(A\) is the union of a collection of open sets of
              \(X\), hence \(A\) is open.

        \item Let \(A\) be closed, then \(A\) is clearly the least closed set containing
              itself, implying in \(A = \Cl A\). Let \(x \in X \setminus A\) be any
              point. Since \(X \setminus A\) is open, we use the previous equivalence for
              open sets to conclude that there exists \(U \subseteq X \setminus A\)
              neighbourhood of \(x\).  To conclude the equivalence chain, suppose the last
              property is true for \(X \setminus A\). Then \(X \setminus A\) is open by the
              previous item, which, in turn, this implies that \(A\) closed.
    \end{enumerate}
\end{proof}

\begin{proposition}[Basis criterion for open sets]
    Let \(X\) a topological space and \(\mathcal B\) a base for the topology of
    \(X\). A set \(A \subseteq X\) is open if and only if for all points \(p \in
    A\) there exists a neighbourhood of \(p\), \(B_p \in \mathcal B\), such that
    \(B_p \subseteq A\).
\end{proposition}

\begin{proof}
    Let \(A\) be open, then there exists a collection of elements of \(\mathcal B\)
    whose union is \(A\) --- which implies that there exists, for all \(p \in A\), a
    set \(B \subseteq \mathcal B\) such that \(p \in B\) and \(B \subseteq
    A\). For the contrary, if \(p \in A\) is any point and \(B \in \mathcal B\) is
    the corresponding neighbourhood \(p \in B \subseteq A\), then from
    \cref{prop:equiv-open-and-closed-set} we find that \(A\) is open.
\end{proof}

\begin{proposition}\label{prop: closure equivalent prop}
    The following propositions are equivalent, regarding points on the closure of a
    set \(A\)
    \begin{enumerate}[(a)]\setlength\itemsep{0em}
        \item \(x \in \Cl A\).

        \item Every neighbourhood \(U \subseteq X\) of \(x\) is such that \(U
              \cap A \neq \emptyset\).

        \item There exists a basis \(\mathcal B_x\) at the point \(x\) such that for all
              \(U \in \mathcal B_x\) we have \(U \cap A = \emptyset\).
    \end{enumerate}
\end{proposition}

\begin{proof}
    (a) implies (b): Let \(x \in X\) and suppose that (b) is false for \(x\), so
    that there exists \(U \subseteq X\) neighbourhood of \(x\) such that \(U \cap A
    = \emptyset\). Then \(A \subseteq X \setminus U\), that is, \(A\) is a subset of
    the complement of \(U\). Since \(U\) is open, then \(X \setminus U \in C_A
    \coloneq \{F \subseteq X \colon A \subseteq F,\ F \text{ is closed}\}\). From the
    definition of closure, we have that \(\Cl A \subseteq X \setminus U\) and
    therefore \(x \not\in \Cl A\). (b) implies (c): From the definition of a
    basis at the point \(x\), we know that every \(U \in \mathcal B_x\) is a
    neighbourhood of \(x\), hence if (b) is true for \(x\), proposition (c) follows
    immediately. (c) implies (a): Let \(x \in X\) such that \(x \not\in \Cl
    A\), so that proposition (a) is false for \(x\).  From the definition of
    closure, there exists \(F \subseteq C_A\) such that \(x \not\in F\). Consider
    the open complement \(V = X \setminus F\) so that \(x \in V\) and \(V \cap A =
    \emptyset\). Hence, given any basis at \(x\) there exists a neighbourhood of
    \(x\), say \(U\), such that \(U \subseteq V\) and hence \(U \cap A =
    \emptyset\), which implies that proposition (c) is false for \(x\).
\end{proof}

\begin{corollary}\label{cor: disjoint closure persistence}
    If \(U\) is an open set and \(U \cap A = \emptyset\), then \(U \cap \Cl
    A = \emptyset\). Also, if \(U\) and \(V\) are disjoint sets, then \(U \cap
    \Cl V = \Cl U \cap V = \emptyset\).
\end{corollary}

\begin{proof}
    Suppose \(U \cap A = \emptyset\) and that there exists \(x \in U \cap
    \Cl A\), so that \(x \in \Cl A\). Since \(U\) is open, it
    is a neighbourhood of \(x\), hence from \cref{prop: closure equivalent prop}
    we find that \(U \cap A \neq \emptyset\), which is false, thus \(U \cap
    \Cl A = \emptyset\).
\end{proof}

In order to classify points as being interior, exterior or on the boundary of a
set, we may use the following important proposition.

\begin{proposition}[Interior, exterior and boundary points]
    \label{prop:classification-int-ext-boundary-points}
    Classification of points:
    \begin{enumerate}[(a)]\setlength\itemsep{0em}
        \item \(x \in \Int A\) if and only if there exists \(U \subseteq A\)
              neighbourhood of \(x\).

        \item \(x \in \Ext A\) if and only if there exists \(U \subseteq X \setminus A\)
              neighbourhood of \(x\).

        \item \(x \in \Bd A\) if and only if all neighbourhoods \(U \subseteq X\)
              of \(x\) are such that \(U \cap A \neq \emptyset\) and \(U \cap (X \setminus
              A) \neq \emptyset\).
    \end{enumerate}
\end{proposition}

\begin{proof}
    \begin{enumerate}[(a)]\setlength\itemsep{0em}
        \item Let \(x \in A\) be any point. Suppose \(U \subseteq X\) is neighbourhood
              of \(x\) such that \(U \subseteq A\), then from the fact that \(U\) is open we
              conclude that \(U \subseteq \Int A\) from the definition of the interior
              operator. Suppose \(x \in \Int A\) then from definition there is a
              neighbourhood of \(x\) contained in \(A\).

        \item Suppose exists \(U \subseteq X \setminus A\) neighbourhood of \(x\). From
              \cref{prop: closure equivalent prop} we find \(x \notin \Cl A\) and
              therefore \(x \in \Ext A\). The other side of the implication is trivial.

        \item Since all neighbourhoods of \(x\) contain a point of \(X \setminus A\),
              from the first item of this proposition we find that \(x\) cannot belong to
              the interior of \(A\). Moreover, every neighbourhood contains a point of
              \(A\), then \(x\) cannot be an element of \(\Ext A\). Therefore \(x \in
              \Bd A\).
    \end{enumerate}
\end{proof}

\begin{proposition}[Decomposition of the closure]
    \label{prop:closure-composition}
    \(\Cl A = A \cup \Bd A = \Int A \cup \Bd A\).
\end{proposition}

\begin{proof}
    For the first equality we have
    \begin{align*}
        A \cup \Bd A
         & = A \cup [X \setminus (\Int A \cup \Ext A)]               \\
         & = A \cup [(X \setminus \Int A) \cap (X \setminus \Ext A)] \\
         & = A \cup [(X \setminus \Int A) \cap \Cl A]                \\
         & = [A \cup (X \setminus \Int A)] \cap [A \cup \Cl A]       \\
         & = X \cap \Cl A = \Cl A.
    \end{align*}
    Analogously, for the second equality
    \begin{align*}
        \Int A \cup \Bd A
         & = \Int A \cup [(X \setminus \Int A) \cap \Cl A]               \\
         & = [\Int A \cup (X \setminus \Int A)] \cap [\Int A \cup \Cl A] \\
         & = X \cap \Cl A = \Cl A.
    \end{align*}
    This proves the proposition.
\end{proof}

\begin{proposition}
    Let \(X\) be a topological space and \(A \subseteq X\). Then
    \(\Cl(X \setminus A) = X \setminus \Int A\) and also
    \(\Int(X \setminus A) = X \setminus \Cl A\).
\end{proposition}

\begin{proof}
    We prove the first equality. Let
    \(\mathcal A \coloneq \{U \subseteq A \colon U \text{ is open}\}\), from
    definition we have that \(\Int A = \bigcup_{U \in \mathcal A} U\), moreover,
    \(X \setminus \Int A = X \setminus \bigcup_{U \in \mathcal A} U = \bigcap_{U \in
        \mathcal A} X \setminus U\) from de Morgan's Laws. Notice that obviously
    \(X \setminus U \subseteq X\) and moreover since \(U\) is open, then the
    complement \(X \setminus U\) is closed. This makes
    \(X \setminus \Int(A) = \bigcap_{U \in \mathcal A} X \setminus U = \Cl(X
    \setminus A)\). Now we show the second equality. Define
    \(\mathcal{\widetilde A} \coloneq \{U \subseteq A \colon U \text{ is
        closed}\}\), then \(\Cl A = \bigcap_{U \in \mathcal{\widetilde A}} U\) and
    hence
    \(X \setminus \Cl A = X \setminus \bigcap_{U \in \mathcal{\widetilde A}} U =
    \bigcup_{U \in \widetilde{\mathcal{A}}} X \setminus U\). Notice that
    \(X\setminus U \subseteq X \setminus A\) and since \(U\) is closed, then
    \(X \setminus U\) is open. From this we conclude that
    \(X \setminus \Cl A = \bigcup_{U \in \mathcal{\widetilde A}} X \setminus U =
    \Int(X \setminus A)\).
\end{proof}

\begin{proposition}\label{prop: finite union of closures}
    Let a finite collection of subsets \(\{A_i\}_{i = 1}^n \subseteq 2^X\), where
    \(X\) is a topological space, then we have that
    \[
        \Cl\bigg(\bigcup_{i = 0}^n A_i\bigg) = \bigcup_{i = 0}^n \Cl A_i
    \]
\end{proposition}

\begin{proof}
    It suffices to prove that for \(A, B \subseteq X\) we have \(\Cl(A \cup
    B) = \Cl A \cup \Cl B\). Notice that \(\Cl A,
    \Cl B \subseteq \Cl(A \cup B)\), hence \(\Cl A \cup
    \Cl B \subseteq \Cl(A \cup B)\). Now, since \(A \subseteq
    \Cl A\) and \(B \subseteq \Cl B\), we find \(A \cup B \subseteq
    \Cl A \cup \Cl B\), then \(\Cl(A \cup B) \subseteq
    \Cl(\Cl A \cup \Cl B) = \Cl A \cup \Cl B\).
\end{proof}

\begin{definition}[Locally finite family]
    A collection of subsets \(\{A_i\}_{i \in I} \subseteq 2^X\) of a topological
    space \(X\) is said to be a locally finite family if for every point \(x\)
    there exists a neighbourhood \(U \subseteq X\) for which the collection \(\{i
    \in I \colon U \cap A_i \neq \emptyset\}\) is finite.
\end{definition}

\begin{definition}[Discrete family]
    A collection of subsets \(\{A_i\}_{i \in I} \subseteq 2^X\) of a topological
    space \(X\) is said to be a discrete family if for all \(x \in X\) there
    exists a neighbourhood \(U \subseteq X\) such that there exists at most one
    \(A_i\) in the family such that \(U \cap A \neq \emptyset\). A discrete family
    is also a locally finite family.
\end{definition}

\begin{proposition}
    Given a locally finite family of sets \(\{A_i\}_{i \in I} \subseteq 2^X\),
    where \(X\) is a topological space, we have that
    \[
        \Cl\bigg(\bigcup_{i \in  I} A_i\bigg) = \bigcup_{i \in  I} \Cl A_i.
    \]
\end{proposition}

\begin{proof}
    Notice that clearly \(A_i \subseteq \Cl\big(\bigcup_{i \in I} A_i\big)\) for all
    \(i \in I\), hence
    \(\bigcup_{i \in I} \Cl A_i \subseteq \Cl\big(\bigcup_{i \in I}
    A_i\big)\). Moreover, we have from hypothesis that \(\{A_i\}_{i \in I}\) is a
    locally finite family, thus given \(x \in \Cl\big(\bigcup_{i \in I} A_i\big)\)
    we can find a neighbourhood of \(x\), say \(U \subset X\), such that
    \(I_0 \coloneq \{i \in I \colon U \cap A_i \neq \emptyset\}\) is finite. Notice
    that from \cref{prop: closure equivalent prop} we have that \(x\) cannot be a
    limit point of any of the sets with index \(i \in I \setminus I_0\), hence
    \(x \not\in \Cl\big(\bigcup_{i \in I \setminus I_0} A_i\big)\). On the other
    hand, we have
    \(x \in \Cl\big(\bigcup_{i \in I} A_i\big) = \Cl\big(\bigcup_{i \in I_0}
    A_i\big) \cup \Cl\big(\bigcup_{i \in I \setminus I_0} A_i\big)\), which implies
    that
    \(x \in \Cl\big(\bigcup_{i \in I_0} A_i\big) = \bigcup_{i \in I_0} \Cl A_i
    \subseteq \bigcup_{i \in I} \Cl A_i\) (the equality comes from the fact
    that \(I_0\) is finite and hence \cref{prop: finite union of closures} hold).
\end{proof}

\begin{proposition}
    If \(\{A_i\}_{i \in I}\) is a locally finite (resp.\ discrete) family, then the
    family \(\{\Cl A_i\}_{i \in I}\) is locally finite (resp.\ discrete). Conversely,
    if \(\{\Cl A_i\}_{i \in I}\) is locally finite, then \(\{A_i\}_{i \in I}\) is
    locally finite.
\end{proposition}

\begin{proof}
    Let the locally finite (resp.\ discrete) family \(\{A_i\}_{i \in I}\). Given any
    element \(x \in X\) and a neighbourhood \(U\) of \(x\) such that the indexing
    set \(I_0 \coloneq \{i \in I \colon U \cap A_i \neq \emptyset\}\) is finite
    (resp.\ is either empty or a singleton). Notice that since
    \(U \cap A_i \subseteq U \cap \Cl A_i\), we find that for all
    \(i \in I_0, U \cap \Cl A_i \neq \emptyset\), so that
    \(I_0 = I_0' \coloneq \{i \in I \colon U \cap \Cl A_i \neq \emptyset\}\). Since
    \(U\) is an open set, from \cref{cor: disjoint closure persistence} we find that
    \(U \cap \Cl A_i = U \cap A_i = \emptyset\) for all \(i \in I \setminus
    I_0\). For the converse, if the collection of closures is locally finite, then
    clearly \(\{A_{i}\}_{i \in I}\) is locally finite.
\end{proof}

\subsection{Derived and Dense Sets}

\begin{definition}[Limit point and derived set]
    \label{def:limit-point-derived-set}
    A point \(p \in A\) is called a limit point of \(A\) if every neighbourhood of
    \(p\) has a point of \(A \setminus \{p\}\). We define the set \(A'\) as the
    collection of all limit points of \(A\), which we'll call derived set.
\end{definition}

\begin{proposition}
    \label{prop:equivalent-def-limit-point}
    A point \(p \in A\) is a limit point if and only if \(p \in \Cl(A
    \setminus \{p\})\).
\end{proposition}

\begin{proof}
    If \(p\) is a limit point, then from \cref{prop: closure equivalent prop} we see
    that \(p \in \Cl(A \setminus \{p\})\). Moreover, if \(p \in \Cl(A
    \setminus \{p\})\), then again from \cref{prop: closure equivalent prop} we
    see that there is a basis \(\mathcal{B}_p\) at \(p\) such that for all \(U \in
    \mathcal{B}_p\) we have \(U \cap (A \setminus \{p\}) \neq \emptyset\) --- thus
    \(p\) is a limit point of \(A\).
\end{proof}

\begin{definition}[Isolated points]
    \label{def:isolated-points}
    If \(p \in A \setminus A'\), we say that \(p\) is an isolated point of \(A\).
\end{definition}

\begin{proposition}\label{prop:isolated-point-equivalent-def}
    A point \(p \in A\) is isolated if and only if there exists a neighbourhood of
    \(p\) for which the only point of intersection with \(A\) is \(p\), i.e. \(U
    \subseteq X\) neighbourhood of \(p\) with \(U \cap A = \{p\}\).
\end{proposition}

\begin{proof}
    If \(p\) is isolated then there exists a neighbourhood \(U \subseteq X\) of
    \(p\) such that \(U \cap (A \setminus \{p\}) = \emptyset\) but since \(p \in
    U\), then it follows that \(U \cap A = \{p\}\). Now, suppose there exists such
    \(U\), then in particular \(p\) does not satisfy the condition to be a limit
    point of \(A\), thus \(p \notin A\), hence \(p \in A \setminus A'\) and
    therefore \(p\) is a limit point.
\end{proof}

\begin{proposition}
    \label{prop:closure-set-plus-limit-points}
    The closure of a set is the union of the set with its limit points,
    i.e. \(\Cl A = A \cup A'\).
\end{proposition}

\begin{proof}
    Let \(p \in A'\), then from definition any neighbourhood of \(p\) intersects
    \(A \setminus \{p\}\) --- and, in particular, intersects \(A\) --- therefore, by
    means of \cref{prop: closure equivalent prop} we see that \(p \in
    \Cl A\), thus \(A' \subseteq \Cl A\). It's obvious that \(A
    \subseteq \Cl A\), thus \(A \cup A' \subseteq \Cl A\). On the
    other hand, if \(q \in \Cl A\), then we assume that \(q \notin A\) ---
    since, on the contrary, it's clear that \(q \in A \cup A'\). Again using
    \cref{prop: closure equivalent prop} we get that every neighbourhood of \(q\)
    intersects \(A\), and since \(q \notin A\), then the intersection contains a
    point other than \(q\) --- that is, \(q\) is a limit point of \(A\). Hence
    \(\Cl A \subseteq A \cup A'\), which finishes the proof.
\end{proof}

\begin{corollary}
    \label{cor:closed-limit-points}
    A set \(A\) is closed if and only if \(A\) contains all of its limit points,
    that is, \(A = A'\).
\end{corollary}

\begin{proof}
    If \(A\) is closed, then \(A = \Cl A\) and hence \(A' \subseteq A\) from
    \cref{prop:closure-set-plus-limit-points}. Otherwise, if \(A\) contains all of
    its limit points, then \(A \cup A' = A = \Cl A\) --- that is, \(A\) is
    closed.
\end{proof}

\begin{definition}[Dense set]
    \label{def:dense-set}
    A set \(A \subseteq X\) is said to be dense if \(\Cl A = X\).
\end{definition}

\begin{proposition}
    \label{prop:dense-non-empty-intersects}
    A set \(A\) is dense in \(X\) if and only if for all non-empty open subsets of
    \(X\) contains a point of \(A\).
\end{proposition}

\begin{proof}
    Let \(A\) be dense in \(X\) and let \(U \subseteq X\) be any non-empty
    open subset of the space. Suppose, for the sake of contradiction, that \(U \cap
    A = \emptyset\). Since \(U\) is non-empty, take any \(x \in U\), then from
    \cref{prop: closure equivalent prop} we find that \(x \notin \Cl A\) but
    since \(\Cl A = X\), then \(x \notin X\), which is a contradiction ---
    thus \(U \cap A\) is non-empty. On the other hand, if we suppose that every
    non-empty open set of the space intersects \(A\), then given any \(x \in X\) we
    see that \(x\) is a limit point of \(A\), hence \(x \in \Cl A\), thus \(X
    \subseteq \Cl A\), which proves the proposition since \(\Cl A
    \subseteq X\).
\end{proof}

%%% Local Variables:
%%% TeX-master: "../../deep-dive"
%%% End:
\documentclass[11pt, reqno]{amsart}
\usepackage[english]{babel}

\usepackage{layout}
\usepackage{afterpage}
\usepackage[
  asymmetric,
  textheight     = 673pt,
  marginparsep   = 7pt,
  footskip       = 27pt,
  hoffset        = 0pt,
  paperwidth     = 597pt,
  textwidth      = 452pt,
  marginparwidth = 101pt,
  voffset        = 0pt,
  paperheight    = 845pt,
]{geometry}

\newcommand{\changegeometry}{\newgeometry{includehead,headheight=89pt}%
  \afterpage{\aftergroup\restoregeometry}%
}

% Math stuff: do not mess with the ordering!
\usepackage{mathtools}
\usepackage{amsthm}
\usepackage{amssymb}
\usepackage{stmaryrd}
\usepackage{tikz-cd}
\usepackage{tqft}

% Font
\usepackage[no-math]{newpxtext}
\usepackage{newpxmath}
% Set arrow tip to that of newpxmath
\tikzset{>=Straight Barb, commutative diagrams/arrow style=tikz}

% Utilities
\usepackage{enumerate}
\usepackage{todonotes}
\usepackage{graphicx}

\newcommand{\HRule}{\rule{\linewidth}{0.5mm}} % Horizontal rule
\usepackage{multirow}

% Color
\usepackage{xcolor}
\definecolor{brightmaroon}{rgb}{0.76, 0.13, 0.28}

\usepackage{pdfpages}

% References
\usepackage{hyperref}
\hypersetup{
  colorlinks = true,
  allcolors  = brightmaroon,
}
\usepackage[capitalize,nameinlink]{cleveref}
\usepackage[
  backend = biber,
  style   = alphabetic,
]{biblatex}
\addbibresource{bibliography.bib}


% Table of contents: show subsections
\setcounter{tocdepth}{2}

\linespread{1.05}
\vfuzz=14pt % No more vbox errors all over the place

%%%%%%%%%%%%%%%%%%%%%%%%%%%%%%%%%%%%%%%%%%%%%%%%%%%%%%%%%%%%%%%%%%%%%%%%%%%%%%%
% ** Environments **

\theoremstyle{definition}
\newtheorem{theorem}{Theorem}[section]
\newtheorem{proposition}[theorem]{Proposition}
\newtheorem{lemma}[theorem]{Lemma}
\newtheorem{corollary}[theorem]{Corollary}
\newtheorem{axiom}[theorem]{Axiom}
\newtheorem{definition}[theorem]{Definition}
\newtheorem{remark}[theorem]{Remark}
\newtheorem{example}[theorem]{Example}
\newtheorem{notation}[theorem]{Notation}

%%%%%%%%%%%%%%%%%%%%%%%%%%%%%%%%%%%%%%%%%%%%%%%%%%%%%%%%%%%%%%%%%%%%%%%%%%%%%%%
% ** symbols **

\renewcommand{\qedsymbol}{\(\natural\)}
\renewcommand{\leq}{\leqslant}
\renewcommand{\geq}{\geqslant}
\renewcommand{\setminus}{\smallsetminus}
\renewcommand{\preceq}{\preccurlyeq}

% ':' for maps and '\colon' for relations on collections
\DeclareMathSymbol{:}{\mathpunct}{operators}{"3A}
\let\colon\relax
\DeclareMathSymbol{\colon}{\mathrel}{operators}{"3A}

% Disjoint unions over sets
\newcommand{\disj}{\amalg}     % Disjoint union
\newcommand{\bigdisj}{\coprod} % Indexed disjoint union

\DeclareMathOperator{\Log}{Log}
\newcommand{\img}{\text{i}}

% Constant map
\DeclareMathOperator{\const}{cons}

% Multiplication map
\DeclareMathOperator{\mul}{mul}

%%%%%%%%%%%%%%%%%%%%%%%%%%%%%%%%%%%%%%%%%%%%%%%%%%%%%%%%%%%%%%%%%%%%%%%%%%%%%%%
% ** arrows **

% Alias for Rightarrow
\newcommand{\To}{\Rightarrow}

% Monomorphism arrow
\newcommand{\mono}{\rightarrowtail}

% Epimorphism arrow
\newcommand{\epi}{\twoheadrightarrow}

% Unique morphism
\newcommand{\unique}{\to}%{\dashrightarrow}
\newcommand{\xdashrightarrow}[2][]{\ext@arrow 0359\rightarrowfill@@{#1}{#2}}

% Isomorphism symbol
\newcommand{\iso}{\simeq}

\newcommand{\arrowiso}{\iso}
% Isomorphism arrow
\newcommand{\isoto}{\xrightarrow{\raisebox{-.6ex}[0ex][0ex]{\(\arrowiso\)}}}

% How isomorphisms are depicted in diagrams: either \sim or \simeq
\newcommand{\dis}{\iso}

\newcommand{\isounique}{%
  \xdashrightarrow{\raisebox{-.6ex}[0ex][0ex]{\(\arrowiso\)}}
}%

% Natural transformation arrow
\newcommand{\nat}{\Rightarrow}

% Natural isomorphism
\newcommand{\isonat}{\xRightarrow{\raisebox{-.6ex}[0ex][0ex]{\(\arrowiso\)}}}

% Embedding arrow
\newcommand{\emb}{\hookrightarrow}

% Parallel arrows
\newcommand{\para}{\rightrightarrows}

% Adjoint arrows
\newcommand{\adj}{\rightleftarrows}

% Implication
\renewcommand{\implies}{\Rightarrow}
\renewcommand{\impliedby}{\Leftarrow}

% Limits
\DeclareMathOperator{\Lim}{lim}     % Limit
\DeclareMathOperator{\Colim}{colim} % Colimit
\DeclareMathOperator{\Eq}{eq}       % Equalizer
\DeclareMathOperator{\Coeq}{coeq}   % Coequalizer

%%%%%%%%%%%%%%%%%%%%%%%%%%%%%%%%%%%%%%%%%%%%%%%%%%%%%%%%%%%%%%%%%%%%%%%%%%%%%%%
% ** Common collections **

\newcommand{\Z}{\mathbf{Z}}
\newcommand{\N}{\mathbf{N}}
\newcommand{\Q}{\mathbf{Q}}
\newcommand{\CC}{\mathbf{C}}
\newcommand{\R}{\mathbf{R}}
\newcommand{\FF}{\mathbf{F}}

\renewcommand{\emptyset}{\varnothing}

\newcommand{\Uhs}{\mathbf{H}}  % Upper half space
\newcommand{\Proj}{\mathbf{P}} % Projective space

%%%%%%%%%%%%%%%%%%%%%%%%%%%%%%%%%%%%%%%%%%%%%%%%%%%%%%%%%%%%%%%%%%%%%%%%%%%%%%%
% ** Categories **

% Font for categories
\newcommand{\cat}{\texttt}
\newcommand{\catfont}{\texttt}

% Opposite category
\newcommand{\op}{\mathrm{op}}

% Common categories
\newcommand{\Set}{{\catfont{Set}}}          % Sets
\newcommand{\FinSet}{{\catfont{FinSet}}}    % Finite sets
\newcommand{\pSet}{{\catfont{pSet}}}        % Pointed sets
\newcommand{\tOrd}{{\catfont{tOrd}}}        % Totally ordered sets

\newcommand{\Vect}{{\catfont{Vect}}}        % Vector spaces
\newcommand{\FinVect}{{\catfont{FinVect}}}  % Finite vector spaces

\newcommand{\TVect}{{\catfont{TVect}}}      % Topological vector spaces
\newcommand{\Ban}{{\catfont{Ban}}}          % Banach spaces

\newcommand{\Grp}{{\catfont{Grp}}}          % Groups
\newcommand{\Ab}{{\catfont{Ab}}}            % Abelian groups
\newcommand{\GSet}[1]{{{#1}\text{-}\Set}}   % G-sets
\newcommand{\Grpd}{{\catfont{Grpd}}}        % Groupoids

\newcommand{\Mon}{{\catfont{Mon}}}          % Monoidal category
\newcommand{\coMon}{{\catfont{coMon}}}      % coMonoidal category
\newcommand{\rActMon}{{\catfont{rActMon}}}  % right action category
\newcommand{\lActMon}{{\catfont{lActMon}}}  % left action category
\newcommand{\BrMonCat}{{\catfont{BrMonCat}}} % Cat of braided monoidal cats

\newcommand{\Graph}{{\catfont{Graph}}}      % Graphs
\newcommand{\SimpGraph}{{\catfont{sGraph}}} % Simple graphs
\newcommand{\ProfCol}{{\catfont{Prof}(\Col)}}   % C-profile category


\newcommand{\Rng}{{\catfont{Ring}}}             % Rings
\newcommand{\cRng}{{\catfont{CRing}}}           % Commutative rings
\newcommand{\rMod}[1]{{\texttt{Mod}_{#1}}}      % Right modules
\newcommand{\lMod}[1]{{{}_{#1}\catfont{Mod}}}   % Left modules
\newcommand{\Mod}[1]{{#1\text{-}\catfont{Mod}}} % Modules over comm. ring
\newcommand{\Alg}[1]{{#1\text{-}\catfont{Alg}}} % Algebras
\newcommand{\cAlg}[1]{{#1\text{-}\catfont{CAlg}}} % Commutative algebras

\newcommand{\Cat}{{\catfont{Cat}}}          % Small categories
\newcommand{\CAT}{{\catfont{CAT}}}          % Big categories
\newcommand{\UCat}{{\mathcal{U}\text{-}\catfont{Cat}}} % U-Categories

\newcommand{\Psh}[1]{{\catfont{Psh}({#1})}}   % Category of presheaves
\newcommand{\comma}{\downarrow} % Comma category separator
\DeclareMathOperator{\El}{El}              % Category of elements

\DeclareMathOperator{\Map}{Map}

% Operators
\DeclareMathOperator{\Hom}{Mor}   % Morphisms
\DeclareMathOperator{\Mono}{Mono} % Monomorphisms
\DeclareMathOperator{\Epi}{Epi}   % Epimorphisms
\DeclareMathOperator{\Fct}{Fct}   % Functors
\DeclareMathOperator{\Obj}{Obj}   % Objects
\DeclareMathOperator{\Mor}{Mor}   % Morphisms, again
\DeclareMathOperator{\End}{End}   % Endomorphisms
\DeclareMathOperator{\Aut}{Aut}   % Automorphisms
\DeclareMathOperator{\Id}{id}     % Identity
\DeclareMathOperator{\im}{im}     % Image
\DeclareMathOperator{\dom}{dom}   % Domain
\DeclareMathOperator{\codom}{cod} % Codomain
\DeclareMathOperator{\supp}{supp} % Support

% Yoneda embedding
\newcommand{\yo}{\text{\usefont{U}{min}{m}{n}\symbol{'210}}}
\DeclareFontFamily{U}{min}{}
\DeclareFontShape{U}{min}{m}{n}{<-> udmj30}{}

%%%%%%%%%%%%%%%%%%%%%%%%%%%%%%%%%%%%%%%%%%%%%%%%%%%%%%%%%%%%%%%%%%%%%%%%%%%%%%%
% ** algebra **
\DeclareMathOperator{\rank}{rank}
\DeclareMathOperator{\coker}{coker}
\DeclareMathOperator{\codim}{codim}
\DeclareMathOperator{\Tr}{tr}   % Trace
\DeclareMathOperator{\Sym}{Sym} % Symmetric space
\DeclareMathOperator{\Alt}{Alt} % Alternating space
\DeclareMathOperator{\Ann}{Ann} % Annihilator
\DeclareMathOperator{\Char}{char}
\DeclareMathOperator{\Span}{span}
\DeclareMathOperator{\Inn}{Inn}     % Inner automorphisms
\DeclareMathOperator{\Spec}{Spec}   % Prime spectrum
\DeclareMathOperator{\Specm}{Spec_m} % Maximal spectrum
\newcommand{\lie}[1]{\mathfrak{#1}} % Font for Lie structures
\DeclareMathOperator{\Rees}{Rees}   % Rees algebra
\DeclareMathOperator{\Frac}{Frac}   % Field of fractions

\DeclareMathOperator{\torsion}{\texttt{tor}}   % Torsion

\DeclareMathOperator{\eval}{eval}
\DeclareMathOperator{\sign}{sign}

% Matrices
\DeclareMathOperator{\Mat}{Mat}
\DeclareMathOperator{\GL}{GL}
\DeclareMathOperator{\SL}{SL}
\DeclareMathOperator{\PSL}{PSL}
\DeclareMathOperator{\SO}{SO}
\DeclareMathOperator{\SU}{SU}
\DeclareMathOperator{\Unit}{U}
\DeclareMathOperator{\Orth}{O}

% Symbol for the group of units --- for instance, the group of units of a ring
% \(R\) will be denoted by \(R^{\unit}\).
\newcommand{\unit}{\times}

% Orbit and stabilizer
\DeclareMathOperator{\Orb}{Orb}
\DeclareMathOperator{\Stab}{Stab}

% Left and right group actions
\newcommand{\laction}{\circlearrowright}
\newcommand{\raction}{\circlearrowleft}

% Ring ideals font
\newcommand{\ideal}[1]{\mathfrak{#1}}

%%%%%%%%%%%%%%%%%%%%%%%%%%%%%%%%%%%%%%%%%%%%%%%%%%%%%%%%%%%%%%%%%%%%%%%%%%%%%%%
% ** Topology **

\let\Top\relax
\newcommand{\Top}{{\catfont{Top}}}                       % Topological spaces

\newcommand{\wHTop}{{\catfont{wH}\text{-}\catfont{Top}}} % Weak Hausdorff
\newcommand{\kTop}{{k\text{-}\catfont{Top}}}             % k-spaces
\newcommand{\cgTop}{{\catfont{cgTop}}} % Compactly generated

\newcommand{\pTop}{{\catfont{Top}^{*}}}   % Pointed top spaces
\newcommand{\bpTop}{{\catfont{Top}^{*/}}} % Base point top spaces

\newcommand{\Ho}[1]{{\catfont{Ho}(#1)}}            % Homotopy category
\newcommand{\HoTop}{{\catfont{Ho}(\catfont{Top})}} % Classical Homotopy cat

\newcommand{\Splx}{{\mathbf{\Delta}}}           % Simplex category
\newcommand{\sSet}{{\catfont{sSet}}}            % Simplicial sets
\newcommand{\Simp}[1]{{\catfont{Simp}(#1)}}     % Simplicial category
\newcommand{\CoSimp}[1]{{\catfont{CoSimp}(#1)}} % Cosimplicial category

\newcommand{\splx}{\Delta}
\newcommand{\splxtop}{\Delta_{\text{top}}}

\DeclareMathOperator{\Sing}{Sing} % Singular complex functor
\DeclareMathOperator{\Nondeg}{nd} % Non-degenerate simplices
\newcommand{\disc}{\text{disc}}   % discrete

\DeclareMathOperator{\Sk}{sk} % Skeleton

\DeclareMathOperator{\Cof}{Cof} % Cofibrant replacement
\DeclareMathOperator{\Fib}{Fib} % Fibrant replacement
\newcommand{\operad}{\mathcal}
\newcommand{\Operad}{{\catfont{Operad}}}

% attaching spaces
\newcommand{\att}{\amalg}     % Disjoint union
\newcommand{\bigatt}{\coprod} % Indexed disjoint union

% Homotopy
\newcommand{\simht}{\sim_{\text{h}}}              % Homotopy between maps
\newcommand{\simhtrel}[1]{\sim_{\text{rel }{#1}}} % Relative htpy
\newcommand{\isoht}{\iso_{\text{h}}}              % Homotopy equivalence
\newcommand{\htpy}{\Rightarrow}                   % Htpy arrow
\newcommand{\htpyrel}[1]{\Rightarrow_{\text{rel }{#1}}} % Relative htpy arrow


% Functors on topological spaces
\DeclareMathOperator{\Disc}{Disc}       % Discrete topology: Set -> Top
\DeclareMathOperator{\Cone}{Cone}       % Cone
\DeclareMathOperator{\Cyl}{Cyl}         % Cylinder
\DeclareMathOperator{\Susp}{S}          % Suspension
\DeclareMathOperator{\rSusp}{\Sigma}    % Reduced suspension
\DeclareMathOperator{\Path}{Path}       % Path object
\DeclareMathOperator{\Loop}{\Omega}       % Loop space
\DeclareMathOperator{\Eval}{eval}       % Evaluation map
\DeclareMathOperator{\curry}{curry}     % Currying a map
\DeclareMathOperator{\uncurry}{uncurry} % Currying a map
\DeclareMathOperator{\co}{co}           % Compact open co(K, U)

\newcommand{\lift}{\widehat} % Lifting of path and stuff

% Set operators
\DeclareMathOperator{\Cl}{Cl}       % Closure
\DeclareMathOperator{\Bd}{\partial} % Boundary
\DeclareMathOperator{\Int}{Int}     % Interior
\DeclareMathOperator{\Ext}{Ext}     % Exterior

%%%%%%%%%%%%%%%%%%%%%%%%%%%%%%%%%%%%%%%%%%%%%%%%%%%%%%%%%%%%%%%%%%%%%%%%%%%%%%%
% ** Differentiable structures **

\newcommand{\Bun}{{\catfont{Bun}}} % Bundle category
\newcommand{\VecBun}{{\catfont{VecBun}}} % Vector Bundle category
\newcommand{\Man}{{\catfont{Man}}} % Manifolds

% Norm
\DeclarePairedDelimiter{\norm}{\lVert}{\rVert}

% Differential operators
\newcommand{\diff}{\mathrm{d}}
\newcommand{\Diff}{\mathrm{D}}
\DeclareMathOperator{\grad}{grad} % Gradient
\DeclareMathOperator{\Hess}{Hess} % Hessian
\DeclareMathOperator{\Jac}{Jac}   % Jacobian
\DeclareMathOperator{\Curl}{Curl} % Curl
\DeclareMathOperator{\VecField}{\mathfrak{X}}

\DeclareMathOperator{\Grass}{Grass}    % Grassmann variety
\DeclareMathOperator{\Stie}{Stie}      % Stiefel variety

% Vector bundle
\DeclareMathOperator{\zerosec}{Zero} % Zero section

\newcommand{\trans}{\pitchfork} % Transversality

% Set operators
\DeclareMathOperator{\Vol}{vol}   % Volume
\DeclareMathOperator{\Mesh}{mesh} % Mesh

%%%%%%%%%%%%%%%%%%%%%%%%%%%%%%%%%%%%%%%%%%%%%%%%%%%%%%%%%%%%%%%%%%%%%%%%%%%%%%%
% ** Graphs **

% Colouring
\newcommand{\Col}{\mathfrak{C}}
\newcommand{\prof}[1]{\underline{#1}}

\DeclareMathOperator{\Edge}{Edge}
\DeclareMathOperator{\Vertex}{Vert}
\DeclareMathOperator{\Circ}{circ}
\DeclareMathOperator{\diam}{diam}
\newcommand{\emptygraph}{\varnothing}
\DeclarePairedDelimiterX{\size}[1]{\lVert}{\rVert}{#1}

%%%%%%%%%%%%%%%%%%%%%%%%%%%%%%%%%%%%%%%%%%%%%%%%%%%%%%%%%%%%%%%%%%%%%%%%%%%%%%%
% ** machine learning **

\DeclareMathOperator{\Prob}{\mathbb{P}} % Probability
\DeclareMathOperator{\Expect}{\mathbb{E}} % Expectation
\newcommand{\Mean}{\mu} % Mean
\DeclareMathOperator{\Cov}{Cov} % Covariance
\DeclareMathOperator{\Var}{Var} % Variance
\DeclareMathOperator{\Correlation}{Corr}
\DeclareMathOperator{\stdev}{\sigma} % standard deviation
\DeclareMathOperator{\ndist}{\mathcal{N}} % Normal distribution
\DeclareMathOperator{\precision}{\beta}
\DeclareMathOperator*{\argmax}{arg\,max}
\DeclareMathOperator*{\argmin}{arg\,min}
\DeclareMathOperator{\vcdim}{VCdim}
\DeclareMathOperator{\Growth}{Growth} % Growth function
\DeclareMathOperator{\Quantile}{Quant}


%%%%%%%%%%%%%%%%%%%%%%%%%%%%%%%%%%%%%%%%%%%%%%%%%%%%%%%%%%%%%%%%%%%%%%%%%%%%%%%
% ** MACROS END HERE **
%%%%%%%%%%%%%%%%%%%%%%%%%%%%%%%%%%%%%%%%%%%%%%%%%%%%%%%%%%%%%%%%%%%%%%%%%%%%%%%


\begin{document}

\begin{titlepage}
 \vfill
  \begin{center}
       \textsc{\LARGE \textbf{University of São Paulo}} \\[2.0cm]

       \vskip 0.5cm
       \textsc{\large Luiz Gustavo Mugnaini Anselmo}

       {\normalsize Molecular Sciences Course \\
         Class 30, n\(^{\text{o}}\)
         USP 11809746

       E-mail: \texttt{luizmugnaini@usp.br}}\\[2.0cm]

       \HRule\\
       \vskip 0.5cm
       {\LARGE \textbf{Dendroidal Homotopy Theory \& Operads}}
       \HRule\\[1.5cm]

       \hspace{.45\textwidth}
       \begin{minipage}{.5\textwidth}
       \normalsize \textbf{Advanced Studies Report III}\\[0.5cm]

       \textsc{\large Prof.~Ivan Struchiner}\\
       University of São Paulo \\
       Institute of Mathematics and Statistics \\
       E-mail: \texttt{ivanstru@ime.usp.br}\\[1cm]

       \normalsize São Paulo, July of 2023
       \end{minipage}
  \end{center}
\end{titlepage}

% Title stuff
\title[Dendroidal Homotopy Theory \& Operads]{%
{\footnotesize\sl Advanced Studies Report III} \\ \smallskip
  Dendroidal Homotopy Theory \& Operads
}%

\author{%
  Luiz Gustavo Mugnaini Anselmo and Ivan Struchiner
}%

\address{%
  Institute of Mathematics and Statistics, University of São
  Paulo, Rua do Matão 1010, 05508--090~São Paulo, SP
}%

\email{luizmugnaini@usp.br, ivanstru@ime.usp.br}

\begin{abstract}
This report documents the research advances of the student Luiz G. Mugnaini
A. during his third semester of the advanced cycle of the Molecular Sciences
course. The main topic of research explored this semester was the study of model
categories, operads, and dendroidal sets.
\end{abstract}
\maketitle

% \section{This Semester \&  the Future of the Project}

% With the insurgence of geometric and topological deep learning, we think it
% would be a great opportunity to utilise our knowledge of algebraic topology to
% help advance these fields of research---which currently lack mathematical
% formality and a good framework to build upon. This led us to the decision that
% part of our research group would start focusing on topological deep learning
% using techniques from pure mathematics. With this in mind, we opted to add to my
% semester (2023/1) a course on machine learning (MAC5832). Moreover, we thought
% it would be a great idea to exchange the course on algebraic geometry (MAT5761)
% for one in homological algebra (MAT5997), which would best fit our intentions
% with the current research. All alterations where properly approved by Ivan
% Struchiner.

% \begin{table}[h!]\label{tab:disc}
%   \centering
%   \caption{Modified list of courses attended in the first semester of 2023.}
%   \begin{tabular}{ |c|c|c|c|c| }
%     \hline
%     Year and Semester & Code & Discipline & Type & Credits \\
%     \hline
%     \multirow{4}{*}{2023 / 1}
%                    & CCM0328 & Iniciação à Pesquisa II            & G  & 12 \\
%                    & MAT5997 & Tópicos de Álgebra   & PG & 8 \\
%                    & MAT6684 & Topologia Algébrica                & PG & 8 \\
%                    & MAC5832 & Introdução ao Aprendizado de Máquina & PG & 8 \\
%     \hline
% \end{tabular}
% \end{table}

\section{Operads}

\begin{definition}[Operad]
\label{def:operad}
An \emph{operad} \(\operad P\) is given by a \emph{set} of \emph{colours} \(C\),
and for each \(n \in \N\) and a sequence \(c_1, \dots, c_n, c \in C\) there is a set
\emph{operations} denoted by
\[
\operad P(c_1, \ldots, c_n; c).
\]
We think of these operations as taking \(n\) input colours \(c_1, \dots, c_n\)
and outputting a colour \(c\). The operad also implements the following structures:
\begin{itemize}\setlength\itemsep{0em}
\item For each colour \(c \in C\), there exists a unit map
  \(u_c: * \to \operad P(c; c)\) that maps \(* \mapsto 1_c\).

\item For any sequence \(c_1, \dots, c_n, c \in C\) and any \(n\)-tuple of
  sequences \(d_1^j, \dots, d_{k_j}^j\) for \(1 \leq j \leq n\), there is a
  composition of operations:
  \[
  \operad P(c_1, \dots, c_n; c)
  \times \prod_{j=1}^n \operad P(d_1^j, \dots, d_{k_j}^j; c_j)
  \overset{\gamma}\longrightarrow
  \operad P(d_1^1, \dots, d_{k_1}^1, d_1^2, \dots, d_{k_n}^n; c).
  \]
  Moreover, this composition is unital with respect to each \(1_c\) and
  associative.  The associative property can be seen explicitly as follows:

  Let \(p \in \operad P(c_1, \dots, c_n; c)\), and
  \(q_1, \dots, q_n \in \prod_{j=1}^n \operad P(d_1^1, \dots, d_{k_n}^n; c)\). Then,
  taking operations \(r_1^j, \dots, r_{k_j}^j\) for \(1 \leq j \leq n\) where
  \(r_i^j\) has output colour \(d_i^j\), we obtain
  \[
  (p(q_1, \dots, q_n))(r_1^1, \dots, r_{k_1}^1, r_{k_2}^2, \dots r_{k_n}^n)
  = p(q_1(r_1^1, \dots, r_{k_1}^1), \dots, q_n(r_1^n, \dots, q_{k_n}^n)).
  \]


\item For each permutation \(\sigma\) on \(n\) elements, there exists an induced map
  \[
  \operad P(c_1, \dots, c_n; c)
  \overset{\sigma^{*}}\longrightarrow
  \operad P(c_{\sigma(1)}, \dots, c_{\sigma(n)}; c).
  \]
  Moreover, \(\sigma^{*}\) gives a right action that is compatible with the
  composition given by \(\gamma\).
\end{itemize}
\end{definition}

Given any function of sets \(f: D \to C\) there exists an induced operad
\(f^{*} \operad P\) coloured by \(D\), explicitly given by
\[
f^{*} \operad P(d_1, \dots, d_n; d) = \operad P(f d_1, \dots, f d_n; f d).
\]

Consider operads \(\operad Q\), \(\operad P\) to be coloured by \(D\) and \(C\),
respectively. We define a morphism of operads \(\phi: \operad Q \to \operad P\) to be
a function of sets \(D \to C\) for which we associate a function
\[
f_{d_1, \dots, d_n; d}: \operad Q(d_1, \dots, d_n; d)
\longrightarrow
\operad P(f d_1, \dots, f d_n; d),
\]
for each sequence \(d_1, \dots, d_n, d \in D\), which commutes with all operations
and permutations. This defines a category which we shall denote by \(\Operad\)

\begin{remark}
\label{rem:operad-set-to-sym-mon-cat}
The \cref{def:operad} can be further generalised by replacing \(\Set\) with any
symmetric monoidal category \(\cat E\).
\end{remark}

% \section{Next Semester}

% The following table displays the planned courses that the student will
% participate in the following semester.

% \begin{table}[h!]\label{tab:disc}
%     \centering
%     \caption{Planned list of courses that will be attended in the next semester (second semester of 2023).}
%     \begin{tabular}{ |c|c|c|c|c| }
%         \hline
%         Year and Semester & Code & Discipline & Type & Credits \\
%         \hline
%         \multirow{4}{*}{2023 / 2}
%               & CCM0418 & Iniciação à Pesquisa III     & G  & 12      \\
%               & MAT5008 & Teoria de Homotopia Abstrata & PG & 8 \\
%               & MAC5922 & Combinatória I               & PG & 8 \\
%               & MAC5921 & Deep Learning                & PG & 8 \\
%         \hline
%     \end{tabular}
% \end{table}


\end{document}

%%% Local Variables:
%%% mode: latex
%%% TeX-master: t
%%% End:

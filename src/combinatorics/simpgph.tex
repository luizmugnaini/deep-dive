\section{The Simple Graph --- Looping All Over}

\begin{definition}[Simple Graph]\label{def: simp-graph}
    A simple graph is defined to be a graph \(G = (V, E, d)\) (in the sense of
    \cref{def: general-graph}) where
    \[
        d: E \to V^2/_{(x, y) \sim (y, x)}
        \text{, mapping }
        e \overset d \longmapsto (x, y) = (y, x)
    \]
    That is, \(d\) defines a reflexive and symmetric relation --- that is,
    \(\Edge(x, x)\) is true for all \(x \in V\), and \(\Edge(x, y) = \Edge(y, x)\)
    for all pairs \(x, y \in V\).
\end{definition}

We'll now construct a category for simple graphs using presheaves on the the
following category.

\begin{definition}[Walking simple graph]\label{def: walking-simp-graph}
    A walking simple graph is defined as a category \(\cat S\) consisting of
    \begin{itemize}
        \setlength\itemsep{0.0em}
        \item An object \(V\), called the set of vertices --- containing elements
              \(x\).
        \item An object \(E\), called the set of edges --- containing ordered pairs
              \(e = (x, y)\), where \(x, y \in V\).
        \item Morphisms \(s, t: E \to V\) called source and target, respectively ---
              given an edge \(e = (x, y)\) we have \(e \xmapsto{s} x\) and \(e \xmapsto{t}
              y\). We also impose that the map defined by \(e \mapsto (s(e), t(e))\)
              in \(\End_{\Set}(E)\) is injective.
        \item Identity morphisms \(\Id_V: V \to V\) and \(\Id_E: E \to E\)
              (reflexivity of vertices and edges).
        \item An automorphism \(\sigma: E \isoto E\) called symmetry --- mapping \(e =
              (x, y) \xmapsto\sigma (y, x)\).
    \end{itemize}
\end{definition}

\begin{proposition}[Simple graph is a presheaf]\label{prop: simp-graph-presheaf}
    Let \(\cat S\) be a walking simple graph. A simple graph is a presheaf
    \[
        G: \cat S^\op \to \Set.
    \]
\end{proposition}

\begin{proof}
    The source and target morphisms correspond to \(G(s), G(t): G(E) \to G(V)\) in
    such a way that, given any \(e = (x, y) \in G(E)\), \(e \xmapsto{G(s)} x\) and
    \(e \xmapsto{G(t)} y\) (edges have sources and targets). Moreover, since the
    map \(e \mapsto (s(e), t(e))\) is injective, if \(e, e' \in G(E)\) are edges
    such that \((s(e), t(e)) = (s(e'), t(e'))\), then \(e = e'\) (edges are well
    defined). The automorphism \(\sigma\) corresponds to \(G(\sigma): G(E) \to
    G(E)\) which is again an automorphism and hence defines an equivalence between
    edges \((x, y)\) and \((y, x)\) --- which implies that the source and target
    of an edge are indiscernible from each other.
\end{proof}

\begin{definition}[Morphism of simple graphs]
    We define morphisms of simple graphs in two equivalent ways --- respectively
    following \cref{def: simp-graph} and \cref{prop: simp-graph-presheaf}:
    \begin{itemize}
        \setlength\itemsep{0em}
        \item A morphism \(G \to H\) of simple graphs \(G = (V, E)\) and \(H = (V',
              E')\) is a map \(f: V \to V'\) such that \(\Edge(x, y)\) implies \(\Edge(f(x),
              f(y))\).
        \item Let \(G = (V, E, s, t)\) and \(H = (V', E', s', t')\) be simple
              graphs. A morphism \(G \to H\) is a natural transformation \(\alpha: G \Rightarrow H\)
              \[
                  \begin{tikzcd}
                      \cat S^\op
                      \ar[r, bend left, "G", ""{name=G}]
                      \ar[r, bend right, swap, "H", ""{name=H}]
                      &\Set
                      \ar[Rightarrow, "\alpha", from=G, to=H]
                  \end{tikzcd}
              \]
              Where \(\alpha\) is defined by a pair of morphisms \(\alpha_V: V \to V'\)
              and \(\alpha_E: E \to E'\) such that the following diagrams commute
              \[
                  \begin{tikzcd}
                      E \ar[r, "s"] \ar[d, swap, "\alpha_E"] &V \ar[d, "\alpha_V"]
                      \\
                      E' \ar[r, swap, "s'"] &V'
                  \end{tikzcd}
                  \qquad \qquad \qquad
                  \begin{tikzcd}
                      E \ar[r, "t"] \ar[d, swap, "\alpha_E"] &V \ar[d, "\alpha_V"]
                      \\
                      E' \ar[r, swap, "t'"] &V'
                  \end{tikzcd}
              \]
              from the symmetric and reflective relations in \(\Mor(\cat S)\) we see
              that this is equivalent to the commutativity of the diagram\footnote{
                  Where, for any \(e \in E\), \(e' \in E'\), and \((x, y) \in \langle V
                  \rangle^2\), we define the maps \(s \times t\), \(s' \times t'\) and
                  \(\alpha_V \times \alpha_V\) as
                  \begin{itemize}
                      \setlength\itemsep{0.0em}
                      \item \(e \xmapsto{s \times t} (s(e), t(e)) = (t(e), s(e))\).
                      \item \(e' \xmapsto{s' \times t'} (s'(e'), t'(e')) = (t'(e'),
                            s'(e'))\).
                      \item \((x, y) \xmapsto{\alpha_V \times \alpha_V} (\alpha_V(x),
                            \alpha_V(y)) = (\alpha_V(y), \alpha_V(x))\).
                  \end{itemize}
              }
              \[
                  \begin{tikzcd}
                      E \ar[r, "s \times t"]
                      \ar[d, swap, "\alpha_E"]
                      &V^2/_{(x, y) \sim (y, x)}
                      \ar[d, "\alpha_V \times \alpha_V"]
                      \\
                      E' \ar[r, swap, "s' \times t'"] &V'^2/_{(x', y') \sim (y', x')}
                  \end{tikzcd}
              \]
    \end{itemize}
\end{definition}

\begin{definition}[Simple graphs category]
    The category of simple graphs, denoted by \(\SimpGraph\), consists of
    simple graphs and morphisms between them. That is, \(\SimpGraph = \Set^{\cat
        S^\op}\) --- the category of presheaves on the walking simple graphs category
    \(\cat S\).
\end{definition}

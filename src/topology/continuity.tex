\section{Morphisms of Topological Spaces}

\begin{definition}[Continuous map]\label{def: continuous map}
Let \(X\) and \(Y\) be topological spaces and consider the map \(f : X \to Y\). We
say that \(f\) is continuous if for all \(U \subseteq Y\) open, the preimage
\(f^{-1}(U)\) is open in \(X\).
\end{definition}

\begin{proposition}
A map \(f : X \to Y\) is continuous if and only if the preimage of every closed
subset is closed.
\end{proposition}

\begin{proof}
Suppose that \(f\) satisfies the latter, the given any \(U \subseteq Y\) there
exists a closed set \(C \subseteq Y\) such that \(U = Y \setminus C\), thus
\(f^{-1}(C)\) is closed, then \(f^{-1}(X \setminus C) = f^{-1}(U)\) is open. For
the contrary, the analogous argument is used.
\end{proof}

\begin{proposition}\label{prop: continuous maps properties}
Let \(X, Y, Z\) be topological spaces. The following properties of continuous
maps between topological spaces hold
\begin{enumerate}[(CM1)]
  \item Every constant map \(f: X \to Y\) is continuous.
  \item The identity map is continuous.
  \item If \(f: X \to Y\) is continuous, then for all open set \(U \subseteq X\)
    we have \(f|_U\) continuous.
  \item If \(f: X \to Y\) and \(g: Y \to Z\) are continuous maps, then \(g f : X
    \to Z\).
\end{enumerate}
\end{proposition}

\begin{proof}
\begin{enumerate}[(1)]\setlength\itemsep{0em}
\item Let any point \(a \in Y\) and consider the constant map \(x \xmapsto f a\)
  for all \(x \in X\). Then, for all \(U \subseteq Y \setminus \{a\}\) open, we
  have that \(f^{-1}(U) = \emptyset\), thus have open preimage. On the other
  hand, the fibre \(f^{-1}(a) = X\), hence also open, thus \(f\) is continuous.

\item Notice that if \(U \subseteq Y\) is any open set, then \(\Id_X^{-1}(U) =
  U\) and thus open.

\item Let \(g : X \to Y\) be a continuous map, and \(U \subseteq X\) be any open
  set, then we can take any open \(V \subseteq g(U)\) and conclude that
  \(g^{-1}(V)\) is open (from the hypothesis that \(g\) is continuous); let now
  an open \(H \subseteq Y \setminus g(U)\), then certainly \(g|_U^{-1}(H) =
  \emptyset\), thus open. Hence \(g|_U : U \to Y\) is indeed continuous.

\item Let \(U \subseteq Z\) be open, then \((g f)^{-1}(U) = f^{-1}
  (g^{-1}(U))\). Moreover, since from hypothesis \(g\) is continuous, then
  \(g^{-1}(U)\) is open, now, from the continuity of \(f\) we conclude that
  \(f^{-1}(g^{-1}(U))\) is open, hence \(g f\) is continuous.
\end{enumerate}
\end{proof}

\begin{definition}[Category of topological spaces]
We define the category of topological spaces to be composed of objects named
topological spaces and morphisms being continuous maps between them --- we'll
denote such category by \(\Top\).
\end{definition}

\begin{proposition}[Local criterion for continuity]
\label{prop: local criterion for continuity}
Let \(f: X \to Y\) be a map between topological spaces. Then \(f\) is continuous
if and only if for all points \(x \in X\) there exists a neighbourhood \(U_x
\subseteq X\) of \(x\) such that \(f|_{U_x}\) is continuous.
\end{proposition}

\begin{proof}
If \(f\) is a morphism, then for any \(x \in X\) we can take the whole space
\(X\) as a neighbourhood of \(x\) and the proposition follows. On the other hand
If \(f\) is locally continuous for every point of \(X\), let \(V \subseteq Y\)
be any open set and consider, for every \(x \in f^{-1}(V)\) a neighbourhood
\(U_x\) such that \(f|_{U_x}\) is continuous --- and hence \(f|_{U_x}^{-1}(V)\)
is open for all \(x \in f^{-1}(V)\). Notice that \(f^{-1}(V) \cap U_x =
f|_{U_x}^{-1}(V)\), which has to be open on \(X\). Moreover, from construction
\(f^{-1}(V) = \bigcup_{x \in f^{-1}(V)} f^{-1}(V) \cap U_x\) is the union of
open sets, thus \(f^{-1}(V)\) is open --- and therefore \(f\) is continuous.
\end{proof}

\begin{proposition}[Basis criterion for continuity]
Let \(f: X \to Y\) be a map between topological spaces, and \(\mathcal B\) be
a basis for the topology of \(Y\). Then, \(f\) is continuous (hence a
morphism) if and only if for all \(B \in \mathcal B\) we have \(f^{-1}(B)
\subseteq X\) open.
\end{proposition}

\begin{proof}
(\(\Rightarrow\)) If \(f\) is continuous, then certainly \(f^{-1}(B)\) is
open. (\(\Leftarrow\)) On the other hand, if \(A \subseteq X\) is any open
set, then \(A = \bigcup_{p \in A} B_p\) for \(B_p \in \mathcal B\)
neighbourhood of \(p\). Hence \(f^{-1}(A) = f^{-1} (\bigcup_{p \in A}
B_p) = \bigcup_{p \in A} f^{-1}(B_p)\) is the union of open sets, thus
\(f^{-1}(A)\) is open and therefore \(f\) is continuous.
\end{proof}

\begin{proposition}\label{prop: basis image surjective}
Let \(f: X \to Y\) be a morphism of topological spaces and \(\mathcal B\) be a
basis for the space \(X\). Then \(f(\mathcal B) = \{f(B) \colon B \in \mathcal
B\}\) is a basis for the space \(Y\) if and only if \(f\) is surjective and
open.
\end{proposition}

\begin{proof}
(\(\Rightarrow\)) Suppose \(f(\mathcal B)\) is a basis for \(Y\), then for all
\(V \subseteq Y\) open, there exists an indexing set \(I\) such that \(V =
\bigcup_{i \in  I} U_i\), where \(U_i \in f(\mathcal B)\). This implies in the
existence of \(B \in \mathcal B\) such that \(U_i = f(B)\) hence \(f\) is
open. Consider now any point \(y \in Y\) and any neighbourhood \(V_y\) of
\(y\). From the same argument as above we have \(V_y = \bigcup_{i \in  I_y}
U_i\), where there exists some \(i \in I_y\) such that \(y \in U_i\) and hence
\(y \in U_i = f(B)\) for some \(B \in \mathcal B\).
(\(\Leftarrow\)) Suppose \(f\) is surjective and open. Let \(V \subseteq Y\)
be any open set. Since \(f\) is continuous and surjective, we have that
\(U = f^{-1}(V)\) is a non-empty open set. Since \(\mathcal B\) is a basis, we
can write \(U = \bigcup_{i \in  I} B_i\) where \(B_i \in B\), and then \(f(U)
= f\left( \bigcup_{i \in  I} B_i \right) = \bigcup_{i \in I} f(B_i)\), where
\(f(B_i) \in f(\mathcal B)\) and \(f(U) = V\) from the fact that \(f\) is
surjective. Since \(f\) is open, \(f(B)\) is open for all \(B \in \mathcal
B\). Hence \(f(\mathcal B)\) is a basis for the space \(Y\).
\end{proof}

\begin{definition}[Isomorphism]\label{def: homeomorphism}
Let \(f\) be a morphism of topological spaces. If \(f\) is bijective and has a
continuous inverse, then we say that \(f\) is an isomorphism of topological
spaces --- which can also be called an homeomorphism.
\end{definition}

\begin{definition}[Open \& Closed maps]\label{def: open/closed maps}
Let \(f : X \to Y\) be any set-function between topological spaces.
\begin{itemize}\setlength\itemsep{0em}
  \item We say that \(f\) is an open map if for all \(U \subseteq X\) open the
    image \(f(U) \subseteq Y\) is open.
  \item We say that \(f\) is a closed map if for all \(C \subseteq X\) closed,
    the image \(f(C) \subseteq Y\) is closed.
\end{itemize}
\end{definition}

\begin{proposition}\label{prop:homeomorphism-is-open-closed}
Let \(f: X \to Y\) be a map of topological spaces and consider that \(f\) is an
isomorphism, then \(f\) is an open and a closed map.
\end{proposition}

\begin{proof}
Let \(U \subseteq X\) be an open (resp. closed) set and consider \(V \coloneq f(U)
\subseteq Y\).  Since \(f\) is a bijection, we find that \(f(U) = V\) is open
(resp. closed) by the continuity of \(f^{-1}\). Hence \(f\) is an open (resp.
closed) map.
\end{proof}

\begin{corollary}\label{cor:bij-iso-open-closed}
Let \(f: X \to Y\) be a bijective set-function, where \(X\) and \(Y\) are
topological spaces. The following propositions are equivalent
\begin{enumerate}[(a)]\setlength\itemsep{0em}
\item \(f\) is an isomorphism of topological spaces.
\item \(f\) is open.
\item \(f\) is closed.
\end{enumerate}
\end{corollary}

\begin{proof}
(a) \(\Rightarrow\) (b): Suppose \(f\) is an isomorphism, then since \(f^{-1}\)
is also an isomorphism, it follows that, given any \(U \subseteq X\) the image
\(f(U)\) is open --- hence \(f\) is open. (b) \(\Rightarrow\) (c): Let \(f\) be
open, then given any closed set \(C \subseteq X\), we have that \(f(X \setminus
C) \subseteq Y\) is open, moreover, since \(f\) is injective, \(f(X \setminus C)
= f(X) \setminus f(C)\) and since \(f(X) = Y\) from the surjectivity of \(f\),
we conclude that \(f(C)\) is closed. (c) \(\Rightarrow\) (a) Let \(f\) be
closed, then given any \(V \subseteq Y\) open, we know that \(Y \setminus V\) is
closed and since \(f\) is a bijection, \(f^{-1}(Y \setminus V) = f^{-1}(Y)
\setminus f^{-1}(V) = X \setminus f^{-1}(V) \subseteq X\) is closed, thus
\(f^{-1}(V)\) is open --- which implies in the continuity of \(f\). Moreover,
since \(f\) is a bijection, \(f^{-1}\) is also closed, and therefore continuous
by the same analogous proof, thus \(f\) is an isomorphism.
\end{proof}

\begin{proposition}\label{prop:classification-maps-interior-closure}
Let \(X\) and \(Y\) be topological spaces and \(f: X \to Y\) be a
set-function. We can classify the behaviour of \(f\) by the following conditions
\begin{enumerate}[(a)]\setlength\itemsep{0em}
\item \(f\) is a morphism of topological spaces if and only if \(f(\Cl A)
  \subseteq \Cl(f(A))\) for every set \(A \subseteq X\).
\item \(f\) is a morphism of topological spaces if and only if \(f^{-1}(\Int B)
  \subseteq \Int f^{-1}(B)\) for every set \(B \subseteq Y\).
\item \(f\) is closed if and only if \(f(\Cl A) \supseteq
  \Cl(f(A))\) for every set \(A \subseteq X\).
\item \(f\) is open if and only if \(f^{-1}(\Int B) \supseteq \Int f^{-1}(B)\)
  for every set \(B \subseteq Y\).
\end{enumerate}
\end{proposition}

\begin{proof}
\todo[inline]{prove}
\end{proof}

\begin{definition}[Local isomorphism]
\label{def:local-homeomorphism}
Let \(f: X \to Y\) be a set-function between topological spaces \(X\) and
\(Y\). We say that \(f\) is a local isomorphism of topological spaces if, for
all \(x \in X\), there exists a neighbourhood \(U \subseteq X\) such that \(f(U)
\subseteq Y\) is open and the induced map \(f: U \isoto f(U)\) is an isomorphism
of topological spaces.
\end{definition}

\begin{proposition}\label{prop:properties-local-homeomorphism}
The following are properties pertaining to local isomorphisms
\begin{enumerate}[(a)]\setlength\itemsep{0em}
\item Every isomorphism is a local isomorphism.
\item Every local isomorphism is continuous and open.
\item Every bijective local isomorphism is an isomorphism.
\end{enumerate}
\end{proposition}

\begin{proof}
\begin{enumerate}[(a)]\setlength\itemsep{0em}
\item Let \(f: X \isoto Y\) be an isomorphism, then for all \(x \in X\) we can
  choose the neighbourhood \(X\) and the restriction \(f|_X = f: X \isoto f(X) =
  Y\) is an isomorphism.
\item Let \(g: X \to Y\) be a local isomorphism. Let \(x \in X\) be any element
  and consider \(U \subseteq X\) such that \(g|_U: U \isoto g(U)\) is an
  isomorphism, then in particular \(g|_U\) is continuous --- by \cref{prop:
  local criterion for continuity} we find that \(g\) is continuous. Now let \(U
  \subseteq X\) be any open set, for each \(x \in U\) take \(U_x \subseteq X\)
  neighbourhood of \(x\) such that \(g|_{U_x}: U_x \to g(U_x)\) is an
  isomorphism. Notice that the restriction \(g|_{U \cap U_x}: U \cap U_x \isoto
  f(U \cap U_x)\) is also an isomorphism which implies in \(f(U \cap U_x)
  \subseteq Y\) being open --- moreover, \(V = \bigcup_{x \in U} f(U \cap
  U_{x})\) thus \(V\) is open.
\item Let \(f: X \to V\) be a bijective local isomorphism, then by the last item,
  \(f\) is open. Using \cref{cor:bij-iso-open-closed} we see that \(f\) is an
  isomorphism.
\end{enumerate}
\end{proof}

\begin{definition}[Embedding]\label{def:top-embedding}
Let \(f: X \mono Y\) be an injective morphism of topological spaces. If the
induced morphism \(f: X \to f(X)\) is an isomorphism, then we say that \(f\) is a
topological embedding of \(X\) in \(Y\).
\end{definition}

\subsection{Topology Generated by Mappings}

\begin{proposition}[Topology generated by a collection of mappings]
\label{prop: top generated by collection of maps}
Let \(X\) be a topological space and \(\{Y_i\}_{i \in I}\) be a collection of
topological spaces. Let \(\{f_i : X \to Y_i\}_{i \in I}\) be the collection of
mappings between such topological spaces. Then there exists a initial
topology on \(X\) such that \(f_i\) is continuous, for all \(i \in I\). Such
topology is generated by the base
\[
  \mathcal B = \left\{ \bigcap_{j \in J} f_j^{-1}(U_j) : U_i \subseteq Y_i
  \text{ is open, } J \subseteq I \text{ is finite}\right\}.
\]
We call such topology as the topology generated by the collection of mappings
\(\{f_i\}_{i \in I}\).
\end{proposition}

\begin{proof}
First we show that \(\mathcal B\) is indeed a basis for \(X\). Let \(x \in X\)
be any point, then clearly \(x \in f_i^{-1}(Y_i)\) for all \(i \in I\), hence
\(x \in \bigcap_{j \in  J} f_j^{-1}(Y_j) \in \mathcal B\) for a finite
indexing set \(J \subseteq I\).
Let \(J, S \subseteq I\) be finite indexing sets, then consider the sets
\(A \coloneq \bigcap_{j \in J} f_j^{-1}(U_j), B \coloneq \bigcap_{s \in S} f_s^{-1}(V_s) \in
\mathcal B\) and let any point \(x \in A \cap B\). Then in particular we can
let an non-empty indexing set \(T = J \cap S\) so that \(x \in f_t^{-1}(U_t)\)
for all \(t \in T\) and therefore \(C \coloneq \bigcap_{t \in  T} f_t^{-1}(U_t)\) is
such that \(x \in C \subseteq A \cap B\). This shows that \(\mathcal B\)
indeed satisfies the basis properties.

Now we show the initial topology property. Let \(\tau\) be the topology
generated by \(\mathcal B\) and consider \(\tau'\) to be any other
topology on \(X\) for which the functions \(f_i\) are continuous for all \(i
\in I\). Trivially we must have \(\mathcal B \subseteq \tau'\) and
therefore \(\tau \subseteq \tau'\). This says that \(\tau\)
is coarser than \(\tau'\), which proves the proposition.
\end{proof}

\begin{proposition}
Let \(X\) and \(Y\) be topological spaces, and the topology of \(Y\) be
generated by the collection of maps \(\{f_i : Y \to Y_i\}_{i \in I}\), where
\(\{Y_i\}_{i \in I}\) is a collection of topological spaces. Then a map \(f :
X \to Y\) is continuous if and only if the composition \(f_i f : X \to
Y_i\) is continuous for every \(i \in I\).
\end{proposition}

\begin{proof}
(\(\Leftarrow\)) Suppose \(f_i f\) is continuous for all \(i \in I\),
then since \(f_i\) is continuous on the topology of \(Y\) it follows that for
any given open set \(U \subseteq Y_i\) we have \(f_i^{-1}(U) = V \subseteq Y\)
open. In particular, notice that \((f_i f)^{-1}(U) = f^{-1}(f_i^{-1}(U))
= f^{-1}(V) \subseteq X\) which must be open from the hypothesis of the
continuity of \(f_i f\). (\(\Rightarrow\)) Moreover, if \(f\) is
continuous, then clearly the composition of continuous functions is
continuous.
\end{proof}

%%% Local Variables:
%%% TeX-master: "../../deep-dive"
%%% End:
\subsection{Arithmetic functions}

\begin{proposition}
    If \(n \in \Z\), then \(\exists a, b \in \Z\) such that \(a\)
    is square free for which  \(n = ab^2\).
\end{proposition}

\begin{proof}
    Let \(n := \prod_k p_k^{a_k}\), where \(a_k := 2 b_k + \varepsilon(a_k)\),
    where \(\varepsilon(a_k) := 0\) if \(a_k\) even and \(\varepsilon(a_k) := 1\)
    otherwise. Now, let \(a := \prod_k p_k^{\varepsilon(a_k)}\) and \(b :=
    \prod_k p_k^{b_k}\), thus from construction  \(n = a b^2\) and since
    \(\varepsilon(a_k) \in \{0, 1\}\) we have \(a\) square-free.
\end{proof}

\begin{definition}
    A function \(f\) with \(\mathrm{Dom}(f) = \Z_{>0}\) is said to be
    \emph{multiplicative} if \(\forall a, b \in \Z_{>0}\) such that
    \((a,b) = 1\) then  \(f(ab) = f(a)f(b)\). The function is said to be
    \emph{totally multiplicative} if  \(\forall a, b \in \Z_{>0}\) then
    \(f(ab) = f(a)f(b)\).
\end{definition}

\begin{definition}
    The function \(\sigma_k: \Z \to \Z_{\geqslant 0}\) is defined
    \(\forall k \in \Z_{\geqslant 0}\) as
    \[
        \sigma_k(n) = \sum_{d\mid n} d^k.
    \]
    Where \(\sigma_0\) denotes the number of positive divisors of a given integer
    and \(\sigma_1\) denotes the sum of such divisors.
\end{definition}

\begin{proposition}
    Let \(n \in \Z_{>0}\) and \(n := \prod_{k = 1}^\ell
    p_k^{a_k}\). Then
    \begin{enumerate}[I.]
        \item \(\sigma_0(n) = \prod_{k = 1}^{\ell} (a_k + 1)\).
        \item  \(\sigma_1(n) = \prod_{k=1}^\ell \frac{p_k^{a_k+1} - 1}{p_k - 1}\).
    \end{enumerate}
\end{proposition}

\begin{proof}
    Let \(m := \prod_{k = 1}^\ell p_k^{b_k}\), then the condition for \(m \mid
    n\) is that \(\forall k \in [0, \ell]\) we have satisfied \(0 \leqslant b_k
    \leqslant a_k\), so that every prime gets divided in the unique factorization
    of such integers. Notice then that for every \(b_k\) there are \(a_k + 1\)
    possible values, thus, the total amount of combinations one can get in order
    to make every divisor of \(n\) is simply  \(\prod_{k=1}^\ell (a_k + 1)\),
    which proves item I.

    For item II, notice that \(\sigma_1(n) := \sum_{j = 1}^{v(n)} \left( \prod_{k
        = 1}^{\ell} p_k^{b_k} \right)_j\), then \(\sigma_1(n) =\prod_{k=1}^\ell \left(
    \sum_{b_k = 0}^{a_k} p_k^{b_k}\right)\) but for each sum we have \(\sum_{b_k =
        1}^{a_k} p_k^{b_k} = \frac{p_k^{a_k + 1} - 1}{p_k - 1}\), thus we just shown
    what we intended.
\end{proof}

\subsubsection{Möbius function}

\begin{definition}
    We define the Möbius function, \(\mu: \Z_{>0} \to \Z\) as
    \[
        \mu(n) =
        \begin{cases}
            1,\ \text{ if } n = 1;                              \\
            0,\ \text{ if } \exists a \in \Z_{> 1}: a^2 \mid n; \\
            (-1)^\ell,\ \text{ if } n = \prod_{k = 1}^{\ell} p_k \text{
                distinct primes. }
        \end{cases}
    \]
    is a multiplicative function.
\end{definition}

\begin{proposition}
    For all positive integer \(n\):
    \[
        \sum_{d \mid n} \mu(d) =
        \begin{cases}
            1,\ \text{ if } n = 1; \\
            0,\ \text{ if } n > 1.
        \end{cases}
    \]
\end{proposition}

\begin{proof}
    The case \(n = 1\) comes right from the definition of the function. For \(n
    > 1\), lets just prove the case where \(n = p^k\) for some  \(p\) prime and
    then we can extend this proof by means of the multiplicative property of the
    Möbius function. We have
    \[
        \sum_{d \mid p^k} \mu(d) = \sum_{j = 0}^k \mu(p^j) =
        \underbrace{(1)}_{\mu(1)} + \underbrace{(-1)}_{\mu(p)} = 0.
    \]
    For the general case where \(n = \prod_{k = 1}^\ell p_k^{a_k}\) we would
    simply have  \(\mu(n) = (1 - 1)^\ell = 0\).
\end{proof}

\begin{definition}[Dirichlet Convolution]
    Let functions \(f, g : \Z_{>0}\), we define the \emph{Dirichlet
        convolution} (or \emph{product}) of such functions as
    \[
        f \ast g(n) := \sum_{d\mid n} f(d) g(n/d) = \sum_{d_1d_2=n} f(d_1)g(d_2).
    \]
\end{definition}

Let the functions \(\mathbb{I}(n) := 0\) for \(n > 1\) and  \(\mathbb{I}(1) :=
1\). Then we have
\[
    f \ast \mathbb{I}(n) = \mathbb{I} \ast f(n) = f(n)\mathbb{I}(1) = f(n).
\]
Also, define the function \(I(n) = 1,\ \forall  n \in \Z_{>0}\) then
\[
    f \ast I(n) = I \ast f(n) = \sum_{d \mid n} f(n).
\]

\begin{lemma}\label{dirichletConvLemma}
    Given the functions \(I,\ \mathbb{I}\) just defined above, we have
    \[
        I \ast \mu = \mu \ast I = \mathbb{I}.
    \]
\end{lemma}

\begin{proof}
    For the case \(n = 1\) we simply have  \(I \ast \mu(1) = \mu(1) I(1) = \mu
    \ast I(1) = 1 = \mathbb{I}(1)\). Now, for \(n > 1\) we have  \(I \ast \mu(n)
    = \sum_{d \mid n} \mu(d) = \mu \ast I(n) = 0\) since the divisors \(d\) of
    \(n\) are greater than \(1\).
\end{proof}

\begin{theorem}[Möbius inversion]\label{mobiusInversion}
    Let \(f\) with domain  \(\Z_{>0}\) and \(F(n) := \sum_{d \mid n}
    f(d)\), then, \(\forall n \in \Z_{>0}\)
    \[
        f(n) = \sum_{d\mid n} \mu(d) F(n/d).
    \]
\end{theorem}

\begin{proof}
    Notice that from its definition we have \(F = f \ast I\) and therefore, if we
    use the lemma \ref{dirichletConvLemma} we get \(F \ast \mu = (f \ast I) \ast
    \mu = f (I \ast \mu) = f \ast \mathbb{I}\) and thus, as already shown we have
    \(f \ast \mathbb{I} = f\), then we conclude that  \(f = F \ast \mu\), which
    proves the statement.
\end{proof}

\begin{proposition}
    Consider Euler's totient function (\ref{eulerTotient}). Then we have
    \[
        \sum_{d \mid n} \varphi(d) = n.
    \]
\end{proposition}

\begin{proof}
    Write down the set of rational numbers \(A = \{\frac{1}{n}, \frac{2}{n},
    \dots, \frac{n-1}{n}, 1\}\) and notice that \(\forall a/d \in A\) we have \(d
    \mid n\) and \((a, d) = 1\) (we are considering that \(a/d\) is reduced to
    lowest terms).

    All the divisors of \(n\) of the form  \(d' = d \ell\), where
    \((\ell, n/d) = 1\) and \(1 \leqslant \ell \leqslant n/d\) are such that
    \(a/(d\ell) \in A\) thus \((a, d \ell) = 1 \implies (a,d) = 1\).

    Then, we can find the number of all such \(d'\) divisors, which is simply
    going to be \(\varphi(d)\), by definition (since we are looking for
    numerators relatively prime with \(d\)). Notice also that if we divide the
    set \(A\) into \(A(d) := \{\frac{a}{d\ell} : (\ell,n/d) = 1\text{ and } 1
    \leqslant \ell \leqslant n/d \}\) then these sets are all disjoint and have
    cardinality \(|A(d)| = \varphi(d)\) and thus \(\bigcup_{d \mid n} A(d) = A\),
    which then implies in
    \[
        \sum_{d \mid n} |A(d)| = \sum_{d \mid n} \varphi(d) = n = |A|
    \]
    which turns out to be a really natural result.
\end{proof}

\begin{proposition}
    Let \(n \in \Z\), then
    \[
        \varphi(n) = n \prod_{p \mid n} \left( 1 - \frac{1}{p} \right)
    \]
\end{proposition}

\begin{proof}
    Let \(n = \prod_{k = 1}^\ell p_k^{a_k}\). Since \(n = \sum_{d \mid n}
    \varphi(d)\) then, using \ref{mobiusInversion} we get
    \[
        \varphi(n) = \sum_{d \mid n} \mu(d) \frac{n}{d}
    \]
    Notice that we'll be considering only divisors of the form \(d = \prod_{k =
        1}^{t} p_k\), where \(t \leqslant \ell\), since for those \(\mu(d) =
    (-1)^t\). Then, we have
    \[
        \sum_{d \mid n} \mu(d) \frac{n}{d} =
        n - \sum_{i=1}^\ell \frac{n}{p_i} + \sum_{\substack{i,j \in [1, \ell] \\
                i\neq j}} \frac{n}{p_i p_j} - \dots
        =
        \sum_{k=0}^\ell (-1)^k \sum_{\substack{\{i_j\} \subseteq [1, \ell], \\
                |\{i_j\}| = k}} \frac{n}{\prod_{\{i_j\}} p_{i_j}}
        =
        n \prod_{k = 1}^\ell \left( 1 - \frac{1}{p_k} \right)
    \]
\end{proof}



\section{Simplex Category}

\subsection{Construction}

\begin{definition}[Skeletal simplex category]
\label{def:skeletal-simplex-category}
We denote by \(\Splx\) the category whose objects are natural numbers
\([n] \coloneq \{0 \leq 1 \leq \dots \leq n\}\), and whose morphisms
\(\phi: [n] \to [m]\) are order-preserving --- that is, if \(i \leq j\) then
\(\phi(i) \leq \phi(j)\). The category \(\Splx\) is called the \emph{skeletal
  simplex category}.

Along with the simplex category comes distinguished morphisms:
\begin{itemize}\setlength\itemsep{0em}
\item For each \(0 \leq j \leq n\), we denote by
  \(\delta_j^n: [n - 1] \mono [n]\) the order-preserving \emph{injective}
  morphism skipping the \(j\)-th value:
  \[
  \delta_j^n(i) \coloneq
  \begin{cases}
    i, &\text{if } i < j, \\
    i + 1, &i \geq j.
  \end{cases}
  \]
  This morphism is called \emph{elementary faces}.

\item For each \(0 \leq j \leq n\), we denote by
  \(\sigma_j^n: [n + 1] \epi [n]\) the \emph{surjective} morphism repeating the
  \(j\)-th value twice and every other value only once:
  \[
  \sigma_j^n(i) \coloneq
  \begin{cases}
    i, &\text{if } i \leq j, \\
    i - 1, &\text{if } i > j.
  \end{cases}
  \]
  This morphism is called \emph{elementary degeneracies}.
\end{itemize}
When convenient, we may drop the superscript of these maps and simply refer to
them as \(\delta_i\) and \(\sigma_j\).
\end{definition}

\begin{lemma}[Generating morphisms in \(\Splx\)]
\label{lem:elementary-maps-generate-simplex-maps}
Every morphism of \(\Splx\) can be generated by a composition of elementary
faces and degeneracies.
\end{lemma}

\begin{proof}
Let \(f: [n] \to [m]\) be any morphism of \(\Splx\). We can factor \(f\) through
an injection \(\iota: [n] \mono [k]\) and a surjection \(s: [k] \epi [m]\):
\[
\begin{tikzcd}
{[n]} \ar[rr, "f"] \ar[dr, two heads, "s"'] &&{[m]} \\
&{[k]} \ar[ru, tail, "\iota"'] &
\end{tikzcd}
\]
where \(k \coloneq |\im f|\). Since \(s\) and \(\iota\) must be order-preserving
maps, it follows that they can be written as a finite composition of elementary
degeneracies and elementary faces.
\end{proof}

\begin{corollary}[Cosimplicial identities]
\label{cor:cosimplicial-identities}
Fix any \(n \in \N\) and consider indices \(0 \leq i, j \leq n\). The following
identities correlate elementary faces and degeneracies:
\begin{enumerate}[(1)]\setlength\itemsep{0em}
\item If \(i < j\) then
  \(\delta_j^n \delta_i^{n-1} = \delta_i^n \delta_{j-1}^{n-1}\).

\item If \(i < j\) then
  \(\sigma_i^{n-1} \sigma_j^n = \sigma_{j-1}^{n-1} \sigma_i^n \).

\item If \(i < j\) then
  \(\sigma_i^{n-1} \delta_j^n = \delta_{j-1}^{n-1} \sigma_i^{n-2}\).

\item If \(i = j - 1\) or \(i = j\), then \(\sigma_i^n \delta_j^{n+1} = \Id_n\).

\item If \(i > j\) then
  \(\sigma_i^{n-1} \delta_j^n = \delta_j^{n-1} \sigma_{i-1}^{n-2}\).
\end{enumerate}
These identities can be found in the following four commutative diagrams:
%
\begin{equation*}
  \begin{tikzcd}
  {[n-2]} \ar[r, "\delta_i^{n-1}"] \ar[d, "\delta_{j-1}^{n-1}"']
  &{[n-1]} \ar[d, "\delta_j^n"] \\
  {[n-1]} \ar[r, "\delta_i^n"'] &{[n]}
  \end{tikzcd}
  \qquad
  \qquad
  \begin{tikzcd}
  {[n+1]} \ar[r, "\sigma_j^n"] \ar[d, "\sigma_i^n"']
  &{[n]} \ar[d, "\sigma_i^{n-1}"] \\
  {[n]} \ar[r, "\sigma_{j-1}^{n-1}"'] &{[n-1]}
  \end{tikzcd}
\end{equation*}
%
\begin{equation*}
  \begin{tikzcd}
  {[n - 1]} \ar[r, "\delta_j^n"] \ar[d, "\sigma_i^{n-2}"']
  &{[n]} \ar[d, "\sigma_i^{n-1}"] \\
  {[n-2]} \ar[r, "\delta_{j-1}^{n-1}"'] &{[n-1]}
  \end{tikzcd}
  \qquad
  \qquad
  \begin{tikzcd}
  {[n - 1]} \ar[r, "\delta_j^n"] \ar[d, "\sigma_{i-1}^{n-2}"']
  &{[n]} \ar[d, "\sigma_i^{n-1}"] \\
  {[n-2]} \ar[r, "\delta_j^{n-1}"'] &{[n-1]}
  \end{tikzcd}
\end{equation*}
\end{corollary}

\begin{definition}[Cosimplicial object]
\label{def:cosimp-obj}
Given a category \(\cat C\), we define a \emph{cosimplicial object} in
\(\cat C\) to be a covariant functor
\[
F: \Splx \longrightarrow \cat C.
\]
Any cosimplicial object \(F\) is completely determined by \(F [n]\) for all
\(n \in \N\) and by the maps \(F \delta_i\) and \(F \sigma_j\). The collection
of all cosimplicial objects of \(\cat C\) will be denoted by
\(\CoSimp{\cat C}\).
\end{definition}

\subsection{Limits \& Colimits in \texorpdfstring{\(\Splx\)}{Delta}}

\begin{lemma}[Pushout of injections]
\label{lem:splx-cat-pushout-injections}
Let \(\iota: [k] \emb [n]\) and \(\tau: [k] \emb [m]\) be order-preserving
inclusions where
\[
\iota(j) \coloneq j\ \text{ and }\ \tau(j) \coloneq j + (m - k),
\]
that is, \(\iota\) sends \([k]\) to the initial segment of \([n]\), while
\(\tau\) sends \([k]\) to the terminal segment of \([m]\). There exists a
pushout
\[
\begin{tikzcd}
{[k]} \ar[r, hook, "\iota"] \ar[d, hook, "\tau"']
\ar[rd, "\ulcorner", very near end, phantom]
&{[n]} \ar[d] \\
{[m]} \ar[r] &{[n] \cup_{[k]} [m]}
\end{tikzcd}
\]
in the simplex category \(\Splx\).
\end{lemma}

\begin{proof}
Indeed, if we consider \([m + n - k]\) as a candidate for the pushout, notice
that the following diagram commutes
\[
\begin{tikzcd}
{[k]} \ar[r, hook, "\iota"] \ar[d, hook, "\tau"']
&{[n]} \ar[d, "\tau"] \\
{[m]} \ar[r, "\iota"'] &{[m + n - k]}
\end{tikzcd}
\]
Now, consider any object \([\ell] \in \Splx\) toghether with two morphisms
\(f: [n] \to [\ell]\) and \(g: [m] \to [\ell]\) such that \(f \iota = g
\tau\). Define a map \(\phi: [m + n - k] \to [\ell]\) as follows
\[
\phi(j) \coloneq
\begin{cases}
  g(j),                  &\text{if } j \leq m - k, \\
  g(j) = f(j - (m - k)), &\text{if } m - k \leq j \leq m, \\
  f(j - (m - k)),        &\text{if } m \leq j \leq m + n - k.
\end{cases}
\]
It is easy to see that \(\phi\) is an order-preserving map and is uniquely
defined so that the following diagram commutes
\[
\begin{tikzcd}
{[k]} \ar[r, hook, "\iota"] \ar[d, hook, "\tau"']
&{[n]} \ar[d, "\tau"] \ar[drd, bend left, "f"] & \\
{[m]} \ar[r, "\iota"'] \ar[rrd, bend right, "g"']
&{[m + n - k]} \ar[rd, dashed, "\phi"] & \\
& &{[\ell]}
\end{tikzcd}
\]
Therefore \([m + n - k] = [n] \cup_{[k]} [m]\) since we are in a skeletal
category and isomorphism classes contain a unique representative.
\end{proof}

These pushouts lead to an interesting construction, any object \([n] \in \Splx\)
is the colimit of a diagram consisting of \([0]\)'s and \([1]\)'s, since
\[
\begin{tikzcd}
{[0]} \ar[r, hook, "0"]
\ar[d, hook, "m"'] \ar[rd, phantom, very near end, "\ulcorner"]
&{[n]} \ar[d] \\
{[m]} \ar[r] &{[m + n]}
\end{tikzcd}
\]

\begin{lemma}[Pushout of surjections]
\label{lem:splx-cat-pushout-surjections}
The following properties concern the pushout of pairs of surjective morphisms in
\(\Splx\):
\begin{enumerate}[(a)]\setlength\itemsep{0em}
\item Considering the cosimplicial identity (2) (see
  \cref{cor:cosimplicial-identities}), where \(i < j\), there exists
  \emph{sections} \(\alpha: [n] \to [n+1]\) of \(\sigma_i^n\), and
  \(\beta: [n-1] \to [n]\) of \(\sigma_i^{n-1}\) such that the following diagram
  commutes
  \[
  \begin{tikzcd}
  {[n+1]} \ar[rr, "\sigma_j^n"] \ar[dd, bend left, "\sigma_i^n"]
  &&{[n]} \ar[dd, bend right, "\sigma_i^{n-1}"'] \\ & &\\
  {[n]} \ar[uu, bend left, "\alpha"] \ar[rr, "\sigma_{j-1}^{n-1}"']
  &&{[n-1]} \ar[uu, bend right, "\beta"']
  \end{tikzcd}
  \]
  that is, \(\sigma_j^n \alpha = \beta \sigma_{j-1}^{n-1}\) --- these sections
  are said to be \emph{compatible} with the square. Therefore, the square is an
  \emph{absolute pushout}.
\item Let \(p: [n] \epi [k]\) and \(q: [n] \epi [\ell]\) be \emph{surjections}
  in \(\Splx\). Then the \emph{pushout} of \(p\) and \(q\) exists and is
  \emph{absolute}:
  \[
  \begin{tikzcd}
  {[n]} \ar[r, two heads, "q"]
  \ar[d, two heads, "p"']
  \ar[rd, very near end, phantom, "\ulcorner"]
  &{[\ell]} \ar[d] \\
  {[k]} \ar[r] &{[m]}
  \end{tikzcd}
  \]
\end{enumerate}
\end{lemma}

\begin{proof}
For the proof of item (a), proceed as follows. From the cosimplicial identity
(3) we have \(\delta_i^{n+1} \sigma_i^n = \sigma_i^{n+1} \delta_{i+1}^{n+2}\),
moreover, from (4) we obtain \(\sigma_i^{n+1} \delta_{i+1}^{n+2} = \Id_{n+1}\)
--- thus \(\delta_i^{n+1}\) is a section of \(\sigma_i^n\). Analogously, we find
that \(\delta_i^n\) is a section of \(\sigma_i^{n-1}\). For the
compatibility condition, notice that since \(i < j\), then from (5) we know that
\(\sigma_j^n \delta_i^{n+1} = \delta_i^n \sigma_{j-1}^{n-1}\). Therefore we may
define \(\alpha \coloneq \delta_i^{n+1}\) and \(\beta \coloneq \delta_i^n\).

\todo[inline]{Do absolutes}
\end{proof}

%%% Local Variables:
%%% mode: latex
%%% TeX-master: "../../deep-dive"
%%% End:

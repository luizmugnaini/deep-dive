\section{ZFC}

\begin{theorem}[Cantor]
\label{thm:cantor-power-set-theorem}
Let \(A\) be a set and \(f: A \to 2^A\) be a map, then \(f\) is not surjective
and hence \(|A| < |2^A|\).
\end{theorem}

\begin{proof}
Let \(B \coloneq \{x \in A \colon x \notin f(x)\}\) be a set of \(2^A\) and
suppose, for the sake of contradiction, that \(f\) is surjective. That is,
there must exist some \(y \in A\) for which \(f(y) = B\) --- notice however that
this can't be the case since by the law of excluded middle if \(y \in B\) then
\(y \notin f(y)\) thus \(f(y) \neq B\), while if \(y \notin B\) then \(y \in
f(y)\) and thus \(f(y) \neq B\). Therefore \(f\) cannot be surjective and thus
\(|A| < |2^A|\) since there exists an injective mapping \(x \mapsto \{x\}\).
\end{proof}

\begin{theorem}
\label{thm:no-set-contains-all-sets}
There is no set containing all sets as members.
\end{theorem}

\begin{proof}
For the sake of contradiction, let \(A\) be a set containing every set as a
member. In particular \(2^A \subseteq A\) so that \(|2^A| \leq A\) --- this
can't be the case by \cref{thm:cantor-power-set-theorem}, thus \(A\) cannot be a
set.
\end{proof}

\begin{definition}[Cardinal numbers]
\label{def:ordinal-cardinal}
A \emph{cardinal number} is an isomorphism \emph{class} of sets, and the
\emph{cardinality} of a given set \(S\) is its isomorphism class. The following
are properties pertaining to cardinal numbers:
\begin{enumerate}[(a)]\setlength\itemsep{0em}
\item Every set has a unique cardinal number as its cardinality.
\item Every cardinal number is the cardinality of some set.
\item Two sets have the same cardinality if and only if they are isomorphic as
  sets.
\end{enumerate}
\end{definition}

\begin{definition}[Well ordering]
\label{def:well-ordering}
A \emph{well-ordering} on a set \(S\) is a total ordering such that every
non-empty subset of \(S\) has a \emph{least} element. A set equipped with a
well-order is called a \emph{poset}.
\end{definition}

\begin{definition}[Ordinal]
\label{def:ordinal}
An \emph{ordinal number} is an isomorphism \emph{class} of \emph{well-ordered
  sets}, and the \emph{ordinal rank} of a poset \(S\) is its isomorphism
class. The following are properties satisfied by the ordinal numbers in its
regard to ordinal ranks:
\begin{enumerate}[(a)]\setlength\itemsep{0em}
\item Every poset has a unique ordinal numbers as its ordinal rank.

\item Every ordinal number is the ordinal rank of some poset.

\item Two sets have the same ordinal rank if and only if they are isomorphic as
  posets.
\end{enumerate}
Moreover, ordinals also have the following properties:
\begin{enumerate}[(a)]\setcounter{enumi}{3}\setlength\itemsep{0em}
\item Every ordinal \(\alpha\) has an immediate successor \(\alpha + 1\) ---
  this process entails the addition of an element to the end of a chain of a
  well-ordering of type \(\alpha\).

\item There is a natural well-ordering on the collection of all ordinals. Given
  two ordinals \(\alpha\) and \(\beta\), we say that \(\alpha \leq \beta\) if
  and only if there exists an initial segment of \(\beta\) for which \(\alpha\)
  is isomorphic to.

\item The induced well-ordering on the set \(\{\beta \colon \beta < \alpha\}\)
  is the isomorphism class represented by \(\alpha\) --- therefore, one can
  define an ordinal as a set containing all the smaller ordinals as members.

\item (Equivalent to the Axiom of Choice) Every set is well-orderable, hence
  bijective to some ordinal.
\end{enumerate}

We denote by \(\omega\) the \emph{natural numbers}, so that every ordinal
bijective to \(\omega\) is said to be \emph{countable}.
\end{definition}

The cardinal numbers are well-ordered, and can be indexed by means of
ordinals. We denote the \(\alpha\)-th cardinal number by \(\aleph_{\alpha}\) ---
hence \(\aleph_0 = \omega\), while \(\aleph_1 = \omega_1\) (the first
uncountable ordinal), for instance.

\begin{definition}[Successor \& limit ordinal]
\label{def:successor-limit-ordinal}
Given an ordinal \(\alpha\), we can classify it as a \emph{successor ordinal},
if there exists an ordinal \(\beta\) such that \(\alpha = \beta + 1\), or as a
\emph{limit ordinal} --- in particular, every cardinal is a limit ordinal.
\end{definition}

\begin{definition}[Successor \& limit cardinal]
\label{def:successor-limit-cardinal}
A cardinal is said to be a \emph{successor cardinal} if its corresponding
indexing ordinal is a successor ordinal, otherwise it is a \emph{limit
cardinal}.
\end{definition}


%%% Local Variables:
%%% mode: latex
%%% TeX-master: "../../deep-dive"
%%% End:

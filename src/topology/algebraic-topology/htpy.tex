\section{Homotopy}

\begin{definition}[Left homotopy in \(\Top\)]
    \label{def:left-homotopy-Top}
    Let \(f, g: X \para Y\) be parallel topological morphisms. We define a
    \emph{left homotopy} \(\eta: f \htpy g\) between \(f\) and \(g\) to be a morphism
    \(\eta: X \times I \to Y\) such that the following diagram commutes
    \[
        \begin{tikzcd}
            X \ar[d, rd, "f"'] \ar[r, "i_0"]
            & X \times I \ar[d, "\eta"]
            & X \ar[l, "i_1"'] \ar[ld, "g"]
            \\
            & Y
        \end{tikzcd}
    \]
    where \(i_0, i_1: X \para X \times I\) are parallel morphisms with \(x
    \xmapsto{i_{0}} (x, 0)\) and \(x \xmapsto{i_{1}} (x, 1)\).

    If topological spaces are viewed as categories whose objects are its points and
    morphisms are the identities, then topological morphisms \(f, g: X \para Y\) are
    functors between the categories \(X\) and \(Y\) and a homotopy \(\eta: f \htpy g\)
    is a natural transformation between \(f\) and \(g\), where the following diagram
    commutes in the category \(Y\) for any two points \(x, x' \in X\):
    \[
        \begin{tikzcd}
            {f(x)}  \ar[r, "{\eta(x, -)}"] \ar[d, "f"'] &{g(x)} \ar[d, "g"] \\
            {f(x')} \ar[r, "\eta{(x', -)}"'] &{g(x')}
        \end{tikzcd}
    \]
\end{definition}

To put concretely, left homotopy between \(f\) and \(g\) is a continuous map
\(\eta\) such that \(\eta(-, 0) = f\) and \(\eta(-, 1) = g\)---which can be
visually thought of as a deformation of the morphism \(f\) to \(g\). If \(f\)
and \(g\) are indeed homotopic, we denote this by \(f \simht g\).

\begin{corollary}[Equivalence relation]
    \label{cor:htpy-equivalence-rel}
    Left homotopy induces an \emph{equivalence relation} \(\simht\) on the
    collection of topological morphisms. Moreover, homotopy respects composition of
    morphisms.
\end{corollary}

\begin{proof}
    First we prove that \(\simht\) is an equivalence relation:
    \begin{itemize}\setlength\itemsep{0em}
        \item (Reflexive) Given a continuous map \(f: X \to Y\), we explicitly define a
              homotopy \(f \htpy f\) by mapping \((x, t) \mapsto x\) for all
              \((x, t) \in X \times I\).

        \item (Symmetric) Let \(g: X \to Y\) be another continuous map and suppose that
              there exists a homotopy \(\eta: f \htpy g\). We can define a homotopy \(\eta':
              g \htpy f\) by mapping \((x, t) \mapsto \eta(x, 1 - t)\)---so that \(\eta'(-,
              0) = \eta(-, 1) = g\) and \(\eta'(-, 1) = \eta(-, 0) = f\).

        \item (Transitive) Consider yet another morphism \(h: X \to Y\) and suppose that
              there exists a homotopy \(\sigma: g \htpy h\). In order to construct a homotopy
              \(\delta: f \htpy h\), we define a \(X\)-parametrized path by concatenating
              \(\eta\) and \(\sigma\) appropriately
              \[
                  \delta(x, t) \coloneq
                  \begin{cases}
                      \eta(x, 2 t),       & t \in [0, 1/2], \\
                      \sigma(x, 2 t - 1), & t \in [1/2, 1].
                  \end{cases}
              \]
              Indeed, \(\delta(-, 0) = \eta(-, 0) = f\) while
              \(\delta(-, 1) = \sigma(-, 1) = h\). It remains to prove that \(\delta\) is
              continuous. Notice that the sets \(A \coloneq X \times [0, 1/2]\) and
              \(B \coloneq X \times [1/2, 1]\) are both closed in the product topology and
              cover the whole space \(X \times I\). By
              \cref{prop:continuous-from-covering-subspaces}, since \(\delta\)
              is continuous on both \(A\) and \(B\) then \(\delta\) is continuous on
              \(A \cup B = X \times I\). We say that \(\delta\) is the \emph{product} of the
              homotopies \(\eta\) and \(\sigma\).
    \end{itemize}
    To prove that homotopy preserves composition, consider the following commutative
    diagram in \(\Top\)
    \[
        \begin{tikzcd}
            X \ar[r, "f"]
            &Y \ar[r, shift left, "g"] \ar[r, shift right, "g'"']
            &Z \ar[r, "h"]
            &W
        \end{tikzcd}
    \]
    If there exists a homotopy \(\eta: g \htpy g'\), we want to show that \(h g f\)
    is homotopy equivalent to \(h g' f\). To do that, consider the following
    commutative diagram
    \[
        \begin{tikzcd}
            X \ar[r, "i_0"] \ar[d, "f"']
            &X \times I \ar[d, "f \times \Id_I"]
            &X \ar[l, "i_1"'] \ar[d, "f"]
            \\
            Y \ar[r, "i_0"] \ar[dr, "g"']
            &Y \times I \ar[d, "\eta"]
            &Y \ar[l, "i_1"'] \ar[dl, "g'"]
            \\
            &Z \ar[d, "h"] &
            \\
            &W &
        \end{tikzcd}
    \]
    From the diagram we see that a natural choice of homotopy
    \(\sigma: h g f \htpy h g' f\) is given by \(\sigma = h \eta (f \times \Id_I)\)
    thus indeed \(h g f \simht h g' f\).
\end{proof}

Given any two topological spaces \(X\) and \(Y\), we denote the
collection of continuous maps \(X \to Y\) \emph{up to homotopy equivalence} by
\[
    [X, Y] \coloneq \Hom_{\Top}(X, Y)/{\simht}.
\]
We call a morphism \([f] \in [X, Y]\) a \emph{homotopy class}. Moreover, for any
three spaces \(X, Y, Z \in \Top\), there exists a \emph{unique} compositional
map
\[
    [X, Y] \times [Y, Z] \longrightarrow [X, Z]
\]
such that the following diagram commutes
\[
    \begin{tikzcd}
        \Hom_{\Top}(X, Y) \times \Hom_{\Top}(Y, Z)
        \ar[rr] \ar[dd, two heads]
        &&\Hom_{\Top}(X, Z) \ar[dd, two heads]
        \\ && \\
        {[X, Y] \times [Y, Z]} \ar[rr, dashed]
        && {[X, Z]}
    \end{tikzcd}
\]
That is, given morphisms \(X \xrightarrow{f} Y \xrightarrow{g} Z\), we have a
composition of \([f]\) with \([g]\) uniquely defined as
\[
    [g] \circ [f] \coloneq [g f].
\]

\begin{definition}[Homotopy category]
    \label{def:Ho(Top)}
    We therefore define a category \(\HoTop\) composed of topological
    spaces---which, viewed as an object of \(\HoTop\), is called a \emph{homotopy
        type}---and classes of continuous morphisms between them up to homotopy.
\end{definition}

This quotient operation on the category of topological spaces induces a
canonically defined projective functor
\[
    \kappa: \Top \longrightarrow \HoTop.
\]

\begin{definition}[Homotopy equivalence]
    \label{def:homotopy-equivalence}
    Let \(X\) and \(Y\) be topological spaces. We say that a continuous map
    \(f: X \to Y\) is a \emph{homotopy equivalence} of \(X\) and \(Y\) if there
    exists a continuous map \(g: Y \to X\) and homotopies \(f g \htpy \Id_Y\) and
    \(g f \htpy \Id_X\).

    If there exists such homotopy equivalence, we write that \(X \isoht Y\)---it
    is to be noted that homotopy equivalences are exactly the isomorphisms in the
    homotopy category \(\HoTop\).
\end{definition}

\begin{corollary}
    \label{cor:homeomorphism-is-homotopy-equivalence}
    Every topological isomorphism is a homotopy equivalence.
\end{corollary}

\begin{proof}
    Let \(f: X \isoto Y\) be a topological isomorphism. We consider its image
    under the functor \(\Top \to \HoTop\), which we'll name \([f]: X \to Y\). Then
    since \(f^{-1}: Y \isoto X\) is also a continuous morphism, we can consider its
    class \([f^{-1}]\) and notice that \([f] [f^{-1}] = [f f^{-1}] = [\Id_Y]\) and
    \([f^{-1}] [f] = [f^{-1} f] = [\Id_X]\). Therefore, there exists two homotopies
    \(f f^{-1} \htpy \Id_Y\) and \(f^{-1} f \htpy \Id_X\).
\end{proof}

\begin{corollary}
    \label{cor:htpy-equiv-is-equiv-relation}
    Homotopy equivalence is an equivalence relation on the class of topological
    spaces.
\end{corollary}

\begin{definition}[Relative homotopy]
    \label{def:relative-homotopy}
    Let \(X\) be a space and \(A \subseteq X\) be a subspace. If \(f, g: X \para Y\)
    are parallel continuous maps such that \(f|_A = g|_A\), a \emph{homotopy between
        \(f\) and \(g\) relative to \(A\)}---if existent---is a homotopy
    \(\eta: f \htpyrel{A} g\) such that
    \[
        \eta(a, t) = f(a) = g(a)\quad\text{ for all } a \in A \text{ and } t \in I.
    \]
    If \(f\) and \(g\) are homotopic relative to \(A\), we shall also denote this by
    \(f \simhtrel{A} g\).
\end{definition}

\subsection{Contractibility}

\begin{definition}[Null-homotopy]
    \label{def:null-homotopy}
    A morphism of topological spaces \(X \to Y\) is said to be \emph{null-homotopic}
    if it is homotopic to a \emph{constant map}. A homotopy between a morphism and a
    constant morphism is called a \emph{null-homotopy}. In particular, a null-homotopy of the identity map \(\Id: X \to X\) is a \emph{contraction} of \(X\).
\end{definition}

\begin{proposition}
    \label{prop:composition-null-homotopic-iff-some-is-null-homotopic}
    Let \(f: X \to Y\) and \(g: Y \to Z\) be continuous maps. If either \(f\) or
    \(g\) is null-homotopic, then \(g f: X \to Z\) is null-homotopic.
\end{proposition}

\begin{proof}
    Let \(f\) be null-homotopic and \(\eta: f \htpy c_p\) be a homotopy between \(f\)
    and the constant map \(c_p: X \to Y\) on \(p \in Y\). The composition
    \(g \eta: X \times I \to Z\) is a continuous map such that \(g\eta(-, 0) = g f\)
    and \(g \eta(-, 1) = g c_p\), where \(g c_p = \overline{c}_{g(p)}\) is the
    constant map on \(g(p) \in Z\). Therefore \(g \eta\) is a null-homotopy of
    \(gf\).

    If \(g\) is null-homotopic with a homotopy \(\varepsilon: g \htpy c_z\) for some
    constant map \(c_z: Y \to Z\) on \(z \in Z\). Notice that the map
    \(\varepsilon \circ (f \times \Id_I): X \times I \to Z\) is continuous and for
    all \(x \in X\) we have
    \begin{align*}
        \varepsilon\circ (f \times \Id_I)(x, 0)
         & = \varepsilon(f(x), 0) = gf(x),        \\
        \varepsilon\circ (f \times \Id_I)(x, 1)
         & = \varepsilon(f(x), 1) = c_z f(x) = z. \\
    \end{align*}
    Therefore establishing a null-homotopy \(g f \simht \overline{c}_z\)---where
    \(\overline{c}_z: X \to Z\) is constant on \(z\).
\end{proof}

\begin{proposition}
    \label{prop:map-from-Sn-is-null-homotopic}
    A continuous map \(f: S^n \to Y\) is null-homotopic if and only if there
    exists a continuous map \(F: D^{n+1} \to Y\) such that \(F|_{S^n} = f\).
\end{proposition}

\begin{proof}
    First we prove that \(\Cone(S^n) \iso D^{n+1}\). Let
    \(\phi: \Cone(S^n) \to D^{n+1}\) to be the map given by \([x, t] \mapsto t x\),
    then \(\| t x \| = t \| x \| \leq 1\) and \(\im \phi \subseteq
    D^{n+1}\). Moreover, the map is well defined, since \(\phi[x, 0] = 0\)---which
    is the only uncertain region of the cone. Since the quotient of a compact space
    is compact, the space \(\Cone(S^n)\) is compact---moreover, \(D^{n+1}\) is
    Hausdorff. Notice that \(\phi\) establishes a bijection between a compact space
    and a Hausdorff space, thus \(\phi\) is an isomorphism.

    Suppose \(f\) is null-homotopic to a continuous map \(c_p: S_n \to Y\), for some
    point \(p \in Y\). Let \(\eta: c_p \htpy f\) be an inverse null-homotopy for
    \(f\). Define \(F\) via the isomorphism \(\Cone(S^n) \iso D^{n+1}\) as \(F(t x)
    \coloneq \eta(x, t)\), where \(x \in S^n\) and \(t \in I\). From this, for every
    \(s \in S^n\) we get \(F(s) = \eta(s, 1) = f(s)\), and \(F(0) = p\).

    For the converse, let \(F: D^{n+1} \to Y\) be an extension of \(f\). Define a
    map \(\eta: X \times I \to Y\) given by a collection of maps
    \((\eta_t: S^n \to Y)_{t \in I}\), where \(\eta_t(x) \coloneq F(t x)\)---which
    is a continuous map---so that the following diagram commutes in \(\Top\) for all
    \(t \in I\):
    \[
        \begin{tikzcd}
            S^n \ar[r, "\pi_t"] \ar[rd, "\eta_t"']
            &\Cone(S^n) \ar[r, "\dis"]
            &D^{n+1} \ar[ld, "F"] \\
            &Y &
        \end{tikzcd}
    \]
    where \(\pi_t\) is the continuous map \(x \mapsto [x, t]\). From this we find
    that \(\eta_0(x) = F(0)\) is constant and \(\eta_1(x) = F(x) = f(x)\) since
    \(x \in S^n\).

    It just remains to be shown that \(\eta\) is continuous. If \(U \subseteq Y\) is
    any open set, let \((x_0, t_0) \in \eta^{-1}(U)\) be any pair. Since \(F\) is
    continuous, then there exists \(V \subseteq F^{-1}(U)\) neighbourhood of the
    point \(t_0 x_0 \in D^{n+1}\). Since \(\phi\) is an isomorphism, there exists an
    open set \(P' \times T \subseteq \Cone(S^n)\) that is a neighbourhood of the
    class \([x_0, t_0]\), and \(P' \times T \subseteq \phi^{-1}(V)\). In particular,
    we know that \(\eta_{t_0}\) is continuous, therefore there exists a
    neighbourhood \(P \subseteq \eta_{t_0}^{-1}(P' \times T)\) of \(x_0\) in
    \(S^n\).

    If \((x, t) \in P \times T\) is any point, then from construction we have
    \(F(t x) \in U\)---thus \(\eta(x, t) \in U\), which shows that
    \(P \times T \subseteq \eta^{-1}(U)\) is a neighbourhood of \((x_0, t_0)\),
    hence \(\eta^{-1}(U)\) is open.
\end{proof}

\begin{definition}[Contractible space]
    \label{def:contractible-space}
    A space \(X\) is said to be \emph{contractible} if the unique continuous map
    \(X \to *\) is a homotopy equivalence. From its construction, contractible
    spaces are the terminal objects of \(\HoTop\). A null-homotopy of \(\Id_X\) is
    said to be a \emph{contraction} of \(X\).
\end{definition}

\begin{example}[A ball is contractible]
    \label{exp:ball-contractible}
    Consider the open (or closed) ball \(B^n \subseteq \R^n\). Let \(f: B^n \to *\)
    be the unique continuous map from the ball to the point space, and
    \(\iota: * \to B^n\) be the map \(* \mapsto 0\). We define a homotopy
    \(\eta: \iota f \htpy \Id_{B^n}\) given by \((p, t) \mapsto t p\), and a homotopy
    \(\sigma: f \iota \htpy \Id_{*}\) mapping \((*, t) \mapsto *\), since
    \(f \iota = \Id_{*}\). Therefore \(B^n\) is contractible.

    Furthermore, since \(B^n \iso \R^n\) in \(\Top\), it follows that \(\R^n \isoht
    B^n\) and thus \(\R^n \isoht *\) ---the euclidean space is contractible.
\end{example}

\begin{proposition}
    \label{prop:contractible-iff-id-is-htpy-const}
    A space \(X\) is contractible if and only if \(\Id_X\) is null-homotopic.
\end{proposition}

\begin{proof}
    Suppose \(X\) is contractible, then \(f: X \to *\) is a homotopy
    equivalence. Let \(g: * \to X\) be a homotopy inverse of \(f\), and let \(x_0
    \in X\) be the image of \(g\)---that is \(g(*) = x_0\) and hence \(g f: X \to
    X\) is a constant map \(g f(x) = x_0\). Therefore \(\Id_X\) is homotopic to the
    constant map \(g f\)---and the choice of \(x_0\) was arbitrary.

    For the converse, let \(x_0 \in X\) be any point and let \(\eta: \Id_X \htpy
    c_{x_0}\) be a homotopy---where \(c_{x_0}: X \to X\) is the constant map \(x
    \mapsto x_0\). Consider the unique continuous map \(f: X \to *\) and let \(g: *
    \to X\) be given by \(g(*) = x_0\), then \(g f = c_{x_0} \simht \Id_X\) via
    \(\eta\) and \(f g = * = \Id_{*}\). Therefore \(f\) is a homotopy equivalence.
\end{proof}

\begin{proposition}
    \label{prop:contractible-then-path-connected}
    If \(X\) is contractible, then it is path connected.
\end{proposition}

\begin{proof}
    Let \(x, y \in X\) be any two points. Since \(X\) is contractible, then
    \(\Id_X\) is null-homotopic---hence there exists a constant map \(c_p: X \to X\)
    and a homotopy \(\eta: \Id_X \htpy c_p\). Notice that by definition \(\eta(x, 0)
    = x\) and \(\eta(x, 1) = p\), therefore \(\eta(x, -): I \to X\) is a path from
    \(x\) to \(p\)---and analogously, \(\eta(y, -): I \to X\) is a path from \(y\)
    to \(p\). To construct a path from \(x\) to \(y\) we can simply concatenate the
    path \(\eta(x, -)\) with the inverse path \(\eta^{-1}(y, -)\), which is given by
    \(\eta^{-1}(y, t) = \eta(y, 1 - t)\). Concretely, let \(\gamma: I \to X\) be
    defined by
    \[
        \gamma(t) \coloneq
        \begin{cases}
            \eta(x, 2t),      & \text{if } t \in [0, 1/2] \\
            \eta(y, 2 - 2 t), & \text{if } t \in [1/2, 1]
        \end{cases}
    \]
    which is continuous by \cref{prop:continuous-from-covering-subspaces}. Then
    \(\gamma\) is a path from \(x\) to \(y\).
\end{proof}

\begin{remark}
    \label{rem:sphere-path-conn-but-not-contractible}
    The converse of \cref{prop:contractible-then-path-connected} is not true in
    general, for instance, the sphere \(S^2\) is path connected, but not
    contractible.
\end{remark}

\begin{corollary}
    \label{cor:contractible-to-path-connected-homotopic-maps}
    If \(X\) is contractible and \(Y\) is path connected, then every pair of
    parallel morphisms \(X \para Y\) is homotopic, in short, \([X, Y]\) has a single
    point.
\end{corollary}

\begin{proof}
    Let \(f, g: X \para Y\) be topological morphisms. Since \(X\) is contractible,
    let \(\eta: \Id_X \htpy c_p\) be a homotopy from the identity to the constant
    function \(c_p: X \to X\) at the point \(p \in X\). Notice that
    \(f \eta(-, 0) = f \Id_X = f\) and \(f \eta(-, 1) = f c_p = c_{f(p)}\), where
    \(\overline{c}_{f(p)}: X \to Y\) is the map constant on \(f(p)\). Given any point
    \(y \in Y\), the collection of paths
    \(\Path_Y(f(p), y)\) is non-empty, thus one can define a collection
    \((\gamma_x)_{x \in X}\) of paths \(\gamma_x \in \Path_Y(f(p), g(x))\). Define a
    map \(\varepsilon: X \times I \to Y\) given by
    \[
        \varepsilon(x, t) \coloneq
        \begin{cases}
            f \eta(x, t),     & \text{if } t \in [0, 1/2] \\
            \gamma_x(2t - 1), & \text{if } t \in [1/2, 1]
        \end{cases}
    \]
    then \(\varepsilon\) is continuous by
    \cref{prop:continuous-from-covering-subspaces} and defines a homotopy
    \(\varepsilon: f \htpy g\).
\end{proof}

\begin{corollary}
    \label{cor:contractible-then-maps-to-space-are-homotopic}
    Let \(Y\) be a contractible space. Then for every topological space \(X\) any
    pair of parallel morphisms \(X \para Y\) are homotopic, that is, \([X, Y]\) has
    a single point.
\end{corollary}

\begin{proof}
    Let \(f, g: X \para Y\) be any two parallel morphisms. Since \(Y\) is
    contractible, there exists \(p \in Y\) such that \(\eta: \Id_Y \htpy c_p\), where
    \(c_p: Y \to Y\) is the constant map on \(p\). Notice that the composition
    \(\eta \circ (f \times \Id_I)\) is a continuous map such that, for all
    \(x \in X\) we have
    \begin{align*}
        \eta \circ (f \times \Id_I)(x, 0) & = \eta(f(x), 0) = \Id_Y(f(x)) = f(x), \\
        \eta \circ (f \times \Id_I)(x, 1) & = \eta(f(x), 1) = c_p(f(x)) = p.
    \end{align*}
    Therefore \(\eta \circ (f \times \Id_I): X \times I \to Y\) defines a homotopy
    \(f \simht c_p f\) where \(c_p f: X \to Y\) is merely the constant map
    \(x \mapsto p\), which we'll call \(\overline{c}_p: X \to Y\). Now, taking the
    inverse homotopy \((\eta)^{-1}(y, t) = \eta(y, 1 - t)\) we obtain a homotopy
    \(c_p \simht \Id_Y\). We can then consider the continuous map
    \(\eta^{-1} \circ (g \times \Id_I): X \times I \to Y\)
    \begin{align*}
        \eta^{-1} \circ (g \times \Id_I)(x, 0) & = \eta^{-1}(g(x), 0) = c_p(g(x)) = p,      \\
        \eta^{-1} \circ (g \times \Id_I)(x, 1) & = \eta^{-1}(g(x), 1) = \Id_Y(g(x)) = g(x).
    \end{align*}
    therefore \(\eta^{-1} \circ (g \times \Id_I)\) is a homotopy \(c_p g =
    \overline{c}_p \simht \Id_Y\). Therefore the concatenation of homotopies
    \(\varepsilon: X \times I \to Y\) given by
    \[
        \varepsilon(x, t) \coloneq
        \begin{cases}
            \eta \circ (f \times \Id_I)(x, 2 t),          & \text{if } t \in [0, 1/2] \\
            \eta^{-1} \circ (g \times \Id_I)(x, 2 t - 1), & \text{if } t \in [1/2, 1]
        \end{cases}
    \]
    defines a homotopy \(f \simht \overline{c}_p \simht g\) as wanted.
\end{proof}

\section{Mapping Spaces}

\subsection{Compact-Open Topology}

\begin{definition}[Evaluation map]
    \label{def:evaluation-map}
    Let \(X\) and \(Y\) be topological spaces. We define the \emph{evaluation map}
    on \(X\) and \(Y\) to be the set-function:
    \[
        \Eval: \Hom_{\Top}(X, Y) \times X \longrightarrow Y
        \quad\text{ mapping }\quad
        (f, x) \longmapsto f(x).
    \]
\end{definition}

\begin{definition}[Admissible topology]
    \label{def:admissible-topology}
    A topology on the set of continuous maps \(\Hom_{\Top}(X, Y)\) is said to be
    \emph{admissible} if the evaluation map
    \(\Eval: \Hom_{\Top}(X, Y) \times X \to Y\) is \emph{continuous}.
\end{definition}

\begin{definition}[Compact-open topology]
    \label{def:compact-open-topology}
    Let \(X\) and \(Y\) be topological spaces. A pair \((K, U)\)---composed of a
    compact set \(K \subseteq X\) and an open set \(U \subseteq Y\)---defines a
    a collection of continuous maps
    \[
        \co(K, U) \coloneq \{f \in \Hom_{\Top}(X, Y) \colon f(K) \subseteq U\}
    \]
    on the space \(\Hom_{\Top}(X, Y)\). We define the \emph{compact-open topology}
    on \(\Hom_{\Top}(X, Y)\) to be the topology whose subbasis is the collection of
    sets \(\co(K, U)\) for each \(K \subseteq X\) compact and \(U \subseteq Y\)
    open. We'll denote the compact-open topology of \(\Hom_{\Top}(X, Y)\) by
    \(\co(X, Y)\).
\end{definition}

\begin{proposition}
    \label{prop:compact-open-is-coarser}
    The compact-open topology in \(\Hom_{\Top}(X, Y)\) is the coarsest between the
    admissible topologies in the mapping space.
\end{proposition}

\begin{proof}
    Suppose \(\tau\) is an admissible topology for \(\Hom_{\Top}(X, Y)\). We show
    that for any \(\co(K, U) \in \co(X, Y)\), we have \(\co(K, U) \in \tau\). Let
    \(k \in K\) be any point and \(f \in \co(K, U)\) be any map with
    \(f(K) \subseteq U\). From definition \(\Eval(f, k) = f(k) \in U\)---moreover,
    since \(\Eval\) is continuous, then
    \(\Eval^{-1}(U) \subseteq \Hom_{\Top}(X, Y) \times X\) is open---there must
    exist a neighbourhood \(V_k \subseteq (\Hom_{\Top}(X, Y), \tau)\) of \(f\) and a
    neighbourhood \(W_k \subseteq K\) of \(k\) such that
    \[
        \Eval(V_k \times W_k) \subseteq U.
    \]
    The collection \((W_k)_{k \in K}\) is a cover for \(K\)---and since \(K\) is
    compact, let \(\{W_{k_1}, \dots, W_{k_n}\}\) be a finite subcover. Therefore,
    for all \(1 \leq j \leq n\) we have
    \(\Eval(V_{k_j} \times W_{k_j}) \subseteq U\). Since open sets are closed under
    finite intersections, consider the neighbourhood
    \(V \coloneq V_{k_1} \cap \dots \cap V_{k_n}\) of \(f\). Let
    \(k \in K\) be any point, then there must exist \(1 \leq j \leq n\) such that
    \(k \in W_{k_j}\). Therefore if \(g \in V\) is any map, we have
    \[
        g(k) = \Eval(g, k) \in \Eval(V \times W_{k_j}) \subseteq
        \Eval(V_{k_j} \times W_{k_j}) \subseteq U,
    \]
    hence \(g(K) \subseteq U\)---thus we conclude that \(g \in \co(K, U)\) and \(V
    \subseteq \co(K, U)\) is an open subset containing \(f\). This shows that
    \(\co(K, U)\) is open in \((\Hom_{\Top}(X, Y), \tau)\), therefore \(\co(K, U)
    \in \tau\), which proves the proposition.
\end{proof}

\begin{proposition}
    \label{prop:cpct-open-is-admissible}
    If \(X\) is a \emph{locally compact Hausdorff} space, then the compact-open
    topology on \(\Hom_{\Top}(X, Y)\) is \emph{admissible}.
\end{proposition}

\begin{proof}
    Let \(U \subseteq Y\) be any open set, if \(\eval^{-1}(U)\) is non-empty, let
    \((f, x) \in \eval^{-1}(U)\) be any pair. Since \(f\) is continuous, then
    \(f^{-1}(U)\) is an open set and there must exist a neighbourhood
    \(W \subseteq X\) of \(x\) such that \(W \subseteq f^{-1}(U)\)---thus
    \(f(W) \subseteq U\). Since \(X\) is locally compact Hausdorff, then \(W\) is
    also locally compact Hausdorff and therefore there must exist a relatively
    compact \(V \subseteq X\) neighbourhood of \(x\) such that
    \(V \subseteq \Cl(V) \subseteq W\)---so that
    \(L \coloneq \co(\Cl(V), U) \times V\) is a neighbourhood of the pair
    \((f, x)\). In fact, if \((g, y) \in L\) is any pair, then \(g(y) \in U\),
    therefore \((g, y) \in \eval^{-1}(U)\). This implies in
    \(L \subseteq \eval^{-1}(U)\) and proves that \(\eval^{-1}(U)\) is open in
    \(\Hom_{\Top}(X, Y) \times X\).
\end{proof}

\begin{corollary}
    \label{cor:cpct-open-is-coarsest-admissible}
    When \(X\) is locally compact Hausdorff, the compact-open topology is the
    coarsest admissible topology on \(\Hom_{\Top}(X, Y)\).
\end{corollary}

\subsection{Exponential Objects}

\begin{definition}[Adjoint map]
    \label{def:adjoint-map-top}
    Let \(f: X \times Y \to Z\) be a topological morphism. We define the
    \emph{adjoint map} of \(f\) to be the continuous map \(f^{\wedge}: X \to
    \Hom_{\Top}(Y, Z)\) given by \(f^{\wedge}(x)(y) \coloneq f(x, y)\)---this
    adjoint map is the \emph{currying of \(f\)}.
\end{definition}

\begin{proposition}
    \label{prop:adjoint-map-is-continuous}
    Given a continuous map \(f: X \times Y \to Z\), the adjoint map
    \(f^{\wedge}: X \to \Hom_{\Top}(Y, Z)\) is continuous and, for every
    \(x \in X\), the map \(f^{\wedge}(x): Y \to Z\) is continuous.
\end{proposition}

\begin{proof}
    Let \(\co(K, U) \subseteq \Hom_{\Top}(Y, Z)\) be an open set in the compact-open
    topology. Let \(f^{\wedge}(x) \in \co(K, U)\) for some \(x \in X\)---hence
    \(f(\{x\} \times K) \subseteq U\). From tube lemma (see \cref{lem:tube-lemma})
    there exists \(V \subseteq X\) neighbourhood of \(x\) such that \(V \times K
    \subseteq f^{-1}(U)\). Therefore \(f^{\wedge}(V) \subseteq \co(K, U)\), which
    shows that \(f^{\wedge}(\co(K, U)) \subseteq X\) is an open set. Since this is the
    case for any element of the subbasis of \(\Hom_{\Top}(Y, Z)\), it follows that
    \(f^{\wedge}\) is continuous.

    The last assertion is trivial since \(f^{\wedge}(x) = f \iota_x\), where
    \(\iota_x: Y \emb X \times Y\) is the injective map \(y \mapsto (x, y)\), which
    is continuous---therefore \(f^{\wedge}(x)\) is a continuous map.
\end{proof}

The notion of an adjoint map associated with the space \(\Hom_{\Top}(X \times Y,
Z)\) gives rise to a set-function
\begin{align*}
    \curry: \Hom_{\Top}(X \times Y, Z)
      & \xrightarrow{\hspace*{2cm}} \Hom_{\Top}(X, \Hom_{\Top}(Y, Z)) \\
    f & \xmapsto{\hspace*{2cm}} f^{\wedge}.
\end{align*}
Dual to \emph{currying} is the notion of \emph{uncurrying}, which is a
set-function
\begin{align*}
    \uncurry: \Hom_{\Top}(X, \Hom_{\Top}(Y, Z))
         & \xrightarrow{\hspace*{2cm}} \Hom_{\Top}(X \times Y, Z) \\
    \phi & \xmapsto{\hspace*{2cm}}
    \phi^{\vee} \coloneq \eval_{Y, Z} \circ (\phi \times \Id_Y),
\end{align*}
that is, \(\phi^{\vee}(x, y) \coloneq \phi(x)(y)\).

\begin{proposition}
    \label{prop:uncurrying-continuous}
    Let \(X\), and \(Z\) be any two spaces and \(Y\) be a locally compact Hausdorff
    space. Given a continuous map \(g: X \to \Hom_{\Top}(Y, Z)\), the uncurried
    map \(g^{\vee}: X \times Y \to Z\) is continuous.
\end{proposition}

\begin{proof}
    Let \(U \subseteq Z\) be an open set, and let \((x, y) \in (g^{\vee})^{-1}(U)\)
    be any point. Since \(g(x) \in \Hom_{\Top}(Y, Z)\) and \(y \in (g(x))^{-1}(U)\),
    there must exist a neighbourhood \(W \subseteq Y\) of \(y\) such that
    \(W \subseteq (g(x))^{-1}(U)\). Since \(Y\) is locally compact Hausdorff, the
    open set \(W\) is also locally compact Hausdorff and therefore there exists a
    relatively compact neighbourhood \(V \subseteq W\) of \(y\) with
    \(V \subseteq \Cl(V) \subseteq W\). Thus \(g(x)(\Cl(V)) \subseteq U\), therefore
    \(g(x) \in \co(\Cl(V), U)\)---that is \(g(x)\) is \emph{open} in
    \(\Hom_{\Top}(Y, Z)\).

    Notice that since \(g\) is continuous, there exists a neighbourhood
    \(Q \subseteq X\) of \(x\) such that \(g(Q) \subseteq \co(\Cl(V), U)\). Moreover
    the neighbourhood \(T \times V \subseteq X \times Y\) of \((x, y)\) is contained
    in \((g^{\vee})^{-1}(U)\), since for any \((a, b) \in T \times V\) we have
    \(g^{\vee}(a, b) = g(a)(b) \in U\). Therefore we've shown that
    \((g^{\vee})^{-1}(U) \subseteq X \times Y\) is an open set, which proves that
    \(g^{\vee}\) is continuous.
\end{proof}

Using \cref{prop:adjoint-map-is-continuous} and
\cref{prop:uncurrying-continuous}, we have proven the following corollary.

\begin{corollary}
    \label{cor:bijection-uncurried-to-curried}
    Let \(X\), \(Y\) and \(Z\) be topological spaces. If \(Y\) is a locally compact
    Hausdorff space, then currying is a \emph{set bijection}
    \(\Hom_{\Top}(X \times Y, Z) \iso \Hom_{\Top}(X, \Hom_{\Top}(Y, Z))\).
\end{corollary}

\begin{theorem}
    \label{thm:currying-top-iso}
    Let \(X\), \(Y\) and \(Z\) be topological spaces. If both \(X\) and \(Y\) are
    Hausdorff and \(Y\) is also locally compact, then
    \[
        \Hom_{\Top}(X \times Y, Z)
        \xrightarrow[ \ \curry\ ]{\iso} \Hom_{\Top}(X, \Hom_{\Top}(Y, Z)).
    \]
    is a \emph{topological isomorphism}.
\end{theorem}

\begin{proof}
    By \cref{cor:bijection-uncurried-to-curried} it is sufficient to prove that
    \(\curry\) and \(\uncurry\) are continuous maps. Under the compact-open topology
    we know that an open set of the \emph{subbasis} of
    \(\Hom_{\Top}(X, \Hom_{\Top}(Y, Z))\) is of the form \(\co(K, V)\), for
    \(K \subseteq X\) compact and \(V \subseteq \Hom_{\Top}(Y, Z)\) open---that is,
    there exists an open set \(U \subseteq Z\) and compact subset \(L \subseteq Y\)
    such that \(V = \co(L, U)\). Therefore \(\co(K, V) = \co(K, \co(L, U))\).

    Consider any such open set
    \(W \coloneq \co(K, \co(L, U)) \subseteq \Hom_{\Top}(X, \Hom_{\Top}(Y, Z))\).
    Since \(K\) and \(L\) are compact, then \(K \times L \subseteq X \times Y\) is a
    compact set. If \(f \in \curry^{-1}(W)\), then
    \(f(K \times L) = f^{\wedge}(K)(L) \subseteq U\). This implies that
    \(\co(K \times L, U) \subseteq \curry^{-1}(W)\) is a neighbourhood of
    \(f\)---therefore \(\curry^{-1}(W)\) is open in \(\Hom_{\Top}(X \times Y,
    Z)\). From this we conclude that \(\curry\) is a continuous map.

    Consider now any open set \(\co(Q, U)\) of the subbasis of
    \(\Hom_{\Top}(X \times Y, Z)\). Define \(K \coloneq \pi_X(Q)\) and
    \(L \coloneq \pi_Y(Q)\), with \(Q \subseteq K \times L\)---which are compact
    sets since the image of a compact set under a continuous map is compact. Let
    \(g \in \uncurry^{-1}(\co(Q, U))\) be a curried map. If \((x, y) \in Q\), then
    \(g(x)(y) = g^{\vee}(x, y) \in U\)---which implies that \(\co(K, \co(L, U))
    \subseteq \uncurry^{-1}(\co(Q, U))\) is a neighbourhood of \(g\).
\end{proof}

\begin{proposition}
    \label{prop:prod-quot-map-id-of-loc-cpct-is-quot-map}
    Let \(Z\) be a locally compact topological space, and \(p: X \to Y\) be a
    quotient map. Then the product morphism
    \[
        p \times \Id_Z: X \times Z \to Y \times Z
    \]
    is a quotient map.
\end{proposition}

\begin{proof}
    Let \(h: Y \times Z \to W\) be a set-function---where \(W\) is a topological
    space---such that the composition \(h \circ (p \times \Id_Z)\) is a continuous
    map. From \cref{prop:adjoint-map-is-continuous} we find that the adjoint
    \((h \circ (p \times \Id_Z))^{\wedge} = h^{\wedge} p\) is continuous. Since
    \(p\) is a quotient map, then from the universal property \(h^{\wedge}\) is
    continuous. Now, since \(Z\) is locally compact, the uncurried \(h\) is
    continuous by \cref{prop:uncurrying-continuous}.
\end{proof}

\subsection{Linear Homotopy}

\begin{definition}[Star-shaped spaces]
    \label{def:star-shaped-space}
    A subspace \(A \subseteq \R^n\) is said to be \emph{star-shaped} with respect to
    a point \(x \in A\) if, for all \(y \in A\), the line path \((1 - t)x + t y\) is
    contained in \(A\) for all \(t \in [0, 1]\). A set \(C \subseteq \R^n\) is said
    to be \emph{convex} if and only if it is star-shaped with respect to each of its
    points.
\end{definition}

\begin{proposition}
    \label{prop:homotopy-from-quotient-map}
    Let \(p: X \to Y\) be a quotient morphism of topological spaces. If
    \(\eta: Y \times I \to Z\) is a set-function between topological spaces such
    that the map \(\varepsilon: X \times I \to Z\) given by
    \((x, t) \mapsto \eta(p(x), t)\) is a homotopy, then \(\eta\) is a homotopy.
\end{proposition}

\begin{proof}
    Since \(p\) is a quotient map and \(I\) is compact (hence locally compact), we
    know from \cref{prop:prod-quot-map-id-of-loc-cpct-is-quot-map} that \(p \times
    \Id_I: X \times I \to Y \times I\) is a quotient map. Therefore, since \(\eta
    \circ (p \times \Id)\) is continuous---since \(\varepsilon\) is a homotopy--- we
    find that \(\eta\) is a continuous map, thus a homotopy.
\end{proof}

\begin{definition}
    \label{def:linear-homotopy}
    Let \(X\) be a space and \(Y\) be a star-shaped space. We say that a homotopy
    \(\ell: X \times I \to Y\) is \emph{linear} if there exists continuous maps
    \(f, g: X \para Y\) such that
    \[
        \ell(x, t) = (1 - t) f(x) + t g(x).
    \]
\end{definition}

\section{Retractions \& Deformations}

\subsection{Retractions \& Cofibrations}

\begin{definition}[Retract]
    \label{def:retract}
    Let \(X\) be a topological space. A subspace \(A \subseteq X\) is said to be a
    \emph{retract} of \(X\) if the inclusion map \(\iota: A \emb X\) is a split
    monomorphism in \(\Top\)---that is, there exists a continuous map
    \(r: X \to A\) such that \(r \iota = \Id_A\). We call \(r\) a \emph{retraction
        of \(X\) to \(A\)}.

    A subspace \(B \subseteq X\) is said to be a \emph{weak retract} of \(X\) if
    \(\iota: B \emb X\) is a split monomorphism in \(\HoTop\)---that is, there
    exists a continuous map \(r: X \to A\) such that \(r \iota \simht \Id_A\).
\end{definition}

\begin{definition}[Homotopy extension property]
    \label{def:homotopy-extension-property}
    Let \(X\) and \(Y\) be topological spaces, and \(A \subseteq X\) be a
    subspace. The pair \((X, A)\) is said to have the \emph{homotopy extension
        property with respect to \(Y\)} if, given continuous maps \(g: X \to Y\) and
    \(\varepsilon: A \times I \to Y\)---such that \(g(a) = \varepsilon(a, 0)\) for
    all \(a \in A\)---there exists a continuous map \(\eta: X \times I \to Y\) such
    that \(\eta(x, 0) = g(x)\) for all \(x \in X\), and
    \(\eta|_{A \times I} = \varepsilon\)---that is, \(\eta\) is an extension of
    \(\varepsilon\). This can all be summarized by the following commutative diagram
    in \(\Top\):
    \[
        \begin{tikzcd}
            A \times 0 \ar[rr, hook] \ar[dd, hook]
            & &A \times I \ar[dd, hook] \ar[ld, "\varepsilon"'] \\
            &Y & \\
            X \times 0 \ar[rr, hook] \ar[ru, "g"]
            & &X \times I \ar[lu, "\eta"']
        \end{tikzcd}
    \]
\end{definition}

\begin{corollary}[Extending morphisms]
    \label{cor:homotopic-extension-property}
    Let \((X, A)\) have the homotopy extension property with respect to \(Y\), and
    consider parallel morphisms \(f, f': A \para Y\). If \(f\) is homotopic to
    \(f'\), and \(f\) has an \emph{extension} to \(X\), then \(f'\) also admits an
    extension to \(X\).
\end{corollary}

\begin{proof}
    Let \(g: X \to Y\) be an extension of \(f\)---that is, \(g|_A = f\)--- and
    \(\varepsilon: f \htpy f'\) be a homotopy then, from the homotopy extension
    property of the pair \((X, A)\), there exists a continuous map
    \(\eta: X \times I \to Y\) which is an \emph{extension} of
    \(\varepsilon\). From this we conclude that \(\eta(-, 1): X \to Y\) is an
    extension of \(f'\).
\end{proof}

\begin{definition}[Cofibration]
    \label{def:cofibration}
    Let \(f: Z \to X\) be a morphism between topological spaces. We say that \(f\)
    is a \emph{cofibration} if, for any topological space \(Y\), and continuous maps
    \(g: X \to Y\) and \(\varepsilon: Z \times I \to Y\) such that
    \(g(f(z)) = \varepsilon(z, 0)\), for all \(z \in Z\), there exists a continuous
    map \(\eta: X \times I \to Y\) such that \(\eta(x, 0) = g(x)\) for all
    \(x \in X\), and \(\eta(f(z), t) = \varepsilon(z, t)\) for all \(z \in Z\) and
    \(t \in I\). This is summarized by the following commutative diagram in
    \(\Top\):
    \[
        \begin{tikzcd}
            Z \times 0 \ar[rr, hook] \ar[dd, "f \times \Id_0"']
            & &Z \times I \ar[dd, "f \times \Id_I"] \ar[ld, "\varepsilon"'] \\
            &Y & \\
            X \times 0 \ar[rr, hook] \ar[ru, "g"]
            & &X \times I \ar[lu, "\eta"']
        \end{tikzcd}
    \]
\end{definition}

\begin{example}
    \label{exp:inclusion-cofibration}
    The inclusion map \(A \emb X\) is a cofibration if and only if the pair
    \((X, A)\) has the homotopy extension property every space.
\end{example}

\begin{theorem}[Weakness of retracts]
    \label{thm:weak-retract-iff-retract}
    Let \(X\) be a topological space and \(A \subseteq X\) be a subspace. If the
    pair \((X, A)\) has the \emph{homotopy extension property with respect to
        \(A\)}, then \(A\) is a \emph{weak retract} of \(X\) if and only if \(A\) is a
    \emph{retract} of \(X\).
\end{theorem}

\begin{proof}
    We know that if \(A\) is a retract of \(X\) then in particular \(A\) is a weak
    retract. We now prove the converse. Let \(r: X \to A\) be a weak retraction of
    \(X\) to \(A\). Since \(r \iota \simht \Id_A\), let
    \(\varepsilon: r \iota \htpy \Id_A\) be a homotopy. Since \((X, A)\) has the
    homotopy extension property with respect to \(A\), it follows that there exists
    a map \(\eta: X \times I \to A\) extending \(\varepsilon\)---that is, since
    \(\varepsilon(a, 0) = r(a)\) for all \(a \in A\), then \(\eta(x, 0) = r(x)\) for
    all \(x \in X\). If we define \(r': X \to A\) to be the map
    \(r' \coloneq \eta(-, 1)\), then \(r'\) is a retraction of \(X\) to
    \(A\)---since \(r'|_A = \Id_A\)--- and \(\eta\) establishes a homotopy
    \(r \simht r'\).
\end{proof}

\subsection{Deformations}

\begin{definition}[Deformation]
    \label{def:deformation}
    Let \(X\) be a space and \(A \subseteq X\) be a subspace. We define a
    \emph{deformation of \(A\) in \(X\)} to be a homotopy
    \[
        \delta: A \times I \longrightarrow X
        \quad\text{ such that }\quad
        \delta(-, 0) = \Id_A.
    \]

    If \(\delta(A \times 1) \subseteq B\), for some subspace \(B\) of \(X\), then
    \(\delta\) is said to be a \emph{deformation of \(A\) into \(B\)}---and \(A\) is
    said to be \emph{deformable} in \(X\) into \(B\).

    In particular, the space \(X\) is called \emph{deformable} if there exists a
    subspace \(A \subseteq X\) such that \(X\) is deformable in itself into
    \(A\)---hence, contractibility of \(X\) is equivalent to \(X\) being deformable
    into a point.
\end{definition}

\begin{lemma}
    \label{lem:deformable-iff-right-inverse}
    A space \(X\) is \emph{deformable} into a subspace \(A\) if and only if the
    inclusion map \(A \emb X\) admits a \emph{right homotopy inverse}.
\end{lemma}

\begin{proof}
    Suppose that \(\iota: A \emb X\) has a right homotopy inverse \(f: X \to
    A\). Let \(\eta: \Id_X \htpy \iota f\) be a homotopy, then \(\eta(-, 0) =
    \Id_X\) and \(\eta(X, 1) = \iota f(X) \subseteq A\)---therefore \(\eta\) is a
    deformation of \(X\) into \(A\).

    For the converse, suppose \(X\) is deformable into \(A\), and let \(\delta: X
    \times I \to X\) be such a deformation. Define a map \(f: X \to A\) to be such
    that \(\iota f = \delta(-, 1)\), then \(f\) is continuous and \(\delta\)
    establishes a homotopy \(\Id_X \simht \iota f\)---thus \(f\) is a right homotopy
    inverse of \(\iota\).
\end{proof}

\begin{definition}[Deformation retract]
    \label{def:deformation-retract}
    Let \(X\) be a space, and \(A \subseteq X\) a subspace. We define the following
    concepts:
    \begin{enumerate}[(a)]\setlength\itemsep{0em}
        \item The subspace \(A\) is said to be a \emph{weak deformation
                  retract} of \(X\) if the canonical inclusion \(A \emb X\) is a
              \emph{homotopy equivalence}.

        \item The subspace \(A\) is said to be a \emph{deformation retract} of \(X\) if
              there exists a retraction \(r: X \to A\) of \(\iota\)---that is,
              \(r \iota = \Id_X\)---such that there exists a homotopy
              \(\iota r \simht \Id_X\). A homotopy \(\eta: \Id_X \htpy \iota r\) is called a
              \emph{deformation retraction of \(X\) to \(A\)}.

        \item The subspace \(A\) is said to be a \emph{strong deformation retract} of
              \(X\) if there is a retraction \(r: X \to A\) of \(\iota\)---that is,
              \(r \iota = \Id_X\)---such that there exists a \emph{relative} homotopy
              \(\iota r \simhtrel{A} \Id_X\). A homotopy \(\eta: \Id_X \htpyrel{A} \iota r\)
              is called a \emph{strong deformation retraction of \(X\) to \(A\)}.
    \end{enumerate}
\end{definition}

\begin{lemma}
    \label{lem:weak-deformation-retract-iff-weak-retract-and-deformable}
    Let \(A \subseteq X\) be a subspace of a topological space \(X\). Then \(A\) is
    a \emph{weak deformation retract} of \(X\) if and only if \(A\) is a \emph{weak
        retract} of \(X\) and \(X\) is deformable into \(A\).
\end{lemma}

\begin{proof}
    From \cref{lem:deformable-iff-right-inverse}, there exists a left homotopy
    inverse of the inclusion \(\iota: A \emb X\) if and only if \(X\) is deformable
    into \(A\). On the other hand, \(A\) is a weak retract of \(X\) if and only if
    \(\iota\) admits a right homotopy inverse. Therefore the proposition follows.
\end{proof}

\begin{example}
    \label{exp:Sn-is-strong-deformation-retract}
    The \(n\)-th sphere \(S^n\) is a strong deformation retraction of the punctured
    euclidean space \(\R^{n+1} \setminus 0\). This is realized by the linear
    homotopy \(\delta: (\R^{n+1} \setminus 0) \times I \to \R^{n+1} \setminus 0\)
    given by
    \[
        \delta(x, t) \coloneq (1 - t) x + t \frac{x}{\| x \|}.
    \]
\end{example}

\begin{lemma}
    \label{lem:deformable-into-retract}
    Let \(X\) be a space, and \(A \subseteq X\) be a \emph{retract} of \(X\). If
    \(X\) is \emph{deformable} into the retract \(A\), then \(A\) is a
    \emph{deformation retract} of \(X\).
\end{lemma}

\begin{proof}
    Since \(A\) is a retract, there exists a left homotopy inverse \(r: X \to A\)
    of the inclusion \(\iota: A \emb X\). From hypothesis, \(X\) is deformable into
    \(A\), then there exists a right homotopy inverse \(f: X \to A\) of \(\iota\)
    (by \cref{lem:deformable-iff-right-inverse}), notice however that
    \[
        f = \Id_A f = (r \iota) f = r (\iota f) = r \Id_X = r.
    \]
    Therefore we have \(\Id_X \simht \iota r\)---thus \(A\) is a deformation retract
    of \(X\).
\end{proof}

\begin{corollary}[Removing weaknesses via homotopy extensions]
    \label{cor:htpy-extension-prop-weak-iff-deformation}
    If the pair \((X, A)\) has the homotopy extension property with respect to the
    subspace \(A\), then \(A\) is a \emph{weak deformation retract} of \(X\) if and
    only if \(A\) is a \emph{deformation retract} of \(X\).
\end{corollary}

\begin{proof}
    If \(A\) is a weak deformation retract of \(X\), then by
    \cref{lem:weak-deformation-retract-iff-weak-retract-and-deformable} we know that
    \(A\) is a weak retract of \(X\) and \(X\) is deformable into \(A\). From
    \cref{thm:weak-retract-iff-retract}, since \((X, A)\) has the homotopy extension
    property, then \(A\) is a weak retract if and only if it is a
    retract. Therefore, since \(A\) is a retract of \(X\) and \(X\) is deformable
    into \(A\), by \cref{lem:deformable-into-retract} we conclude that \(A\) is a
    deformation retract of \(X\). The converse is clearly true since a deformation
    retract is always a weak deformation retract.
\end{proof}

\begin{theorem}[Strengthening deformation retracts]
    \label{thm:deformation-retract-iff-strong-deformation-retract}
    Let \(X\) be a space and \(A \subseteq X\) be a subspace. Define a subspace
    \[
        Y \coloneq (X \times 0) \cup (A \times I) \cup (X \times 1)
    \]
    of the cylinder \(X \times I\). Then, if \((X \times I, Y)\) has the
    \emph{homotopy extension property with respect to \(X\)} and \(A\) is
    \emph{closed} in \(X\), then \(A\) is a \emph{deformation retract} of \(X\) if
    and only if \(A\) is a \emph{strong deformation retract} of \(X\).
\end{theorem}

\begin{proof}
    Since strong deformation retracts are always deformation retracts, we prove the
    other side of the equivalence. Assume the hypothesis, and let \(A\) be a
    deformation retract of \(X\). Let \(\delta: \Id_X \htpy \iota r\) be a
    deformation retraction of \(X\) to \(A\)---where, as usual, \(\iota: A \emb X\)
    is the inclusion and \(r: X \to A\) is the retraction. Define a map
    \(\varepsilon: Y \times I \to X\) as follows
    \[
        \varepsilon((x, t), s) \coloneq
        \begin{cases}
            x,                   & \text{if } x \in X \text{ and } t = 0,   \\
            \delta(x, (1-s)t),   & \text{if } x \in A \text{ and } t \in I, \\
            \delta(r(x), 1 - s), & \text{if } x \in X \text{ and } t = 1.
        \end{cases}
    \]
    The map is indeed well defined since, given any point \(a \in A\), one has
    \(\varepsilon((a, 0), s) = a = \delta(a, 0)\)---since \(\delta\) is a
    deformation retraction. Moreover, the compatibility of the last two equalities
    is met because
    \[
        \varepsilon((a, 1), 1 - s) = \delta(a, 1 - s) = \delta(r(a), 1 - s),
    \]
    since \(r \iota = \Id_A\). For the continuity of \(\varepsilon\), we use the
    fact that the sets \((X \times 0) \times I\), \((A \times I) \times I\) and
    \((X \times 1) \times I\) form a \emph{closed} cover of
    \(Y \times I\)---moreover, since \(\varepsilon\) is continuous in each of the
    sets of such closed cover---by \cref{prop:continuous-from-covering-subspaces} we
    conclude that \(\varepsilon\) is continuous.

    If \((x, t) \in Y\) is any pair, since \(\delta(x, 0) = x\) and
    \[
        \delta(r(x), 1) = \iota r(r(x)) = r(x) = \delta(x, 1),
    \]
    then in general \(\varepsilon((x, t), 0) = \delta(x, t)\). Using the homotopy
    extension property of the pair \((X \times I, Y)\) with respect to \(X\), there
    exists an \emph{extension}
    \[
        \chi: (X \times I) \times I \to X
    \]
    for which \(\chi((x, t), 0) = \delta(x, t)\) for all \((x, t) \in X \times I\),
    and \(\chi|_{Y \times I} = \varepsilon\). Now define a map \(\eta: X \times I
    \to X\) given by \(\eta(x, t) \coloneq \chi((x, t), 1)\), then we obtain
    \[
        \eta(x, t) =
        \begin{cases}
            \varepsilon((x, 0), 1) = x,
                             & \text{if } x \in X \text{ and } t = 0,
            \\
            \varepsilon((x, t), 1) = \delta(x, 0) = x,
                             & \text{if } x \in A \text{ and } t \in I,
            \\
            \chi((x, t), 1), & \text{if } x \in X \text{ and } t \in I,
            \\
            \varepsilon((x, 1), 1)
            = \delta(r(x), 0) = \iota r(x),
                             & \text{if } x \in X \text{ and } t = 1,
        \end{cases}
    \]
    therefore \(\eta(-, 0) = \Id_X\) and \(\eta(-, 1) = \iota r\), while \(\eta(a,
    t) = a\) for all \((a, t) \in A \times I\). Therefore \(\eta\) establishes a
    homotopy \(\Id_X \simhtrel{A} \iota r\), which shows that \(A\) is a strong
    deformation retract of \(X\).
\end{proof}

\subsection{Mapping Cylinder \& Cofibrations}

\begin{definition}[Mapping cylinder]
    \label{def:mapping-cylinder}
    Let \(f: X \to Y\) be a topological morphism. We define the \emph{mapping
        cylinder} of \(f\) to be the pushout
    \[
        \begin{tikzcd}
            X \ar[r, "f"] \ar[d, "i_0"']
            \ar[rd, phantom, very near end, "\ulcorner"]
            &Y \ar[d] \\
            X \times I \ar[r] &\Cyl(f)
        \end{tikzcd}
    \]
    in \(\Top\), where \(i_0: X \to X \times I\) is the morphism
    \(x \mapsto (x, 0)\)---the definition can be ``isomorphically'' given by
    replacing \(i_0\) by \(i_1: X \to X \times I\) mapping \(x \mapsto (x, 1)\), the
    resulting pushouts are isomorphic topological spaces.

    Put more concretely, the cylinder of \(f\) is the topological space
    \[
        \Cyl(f) = Y \cup_f (X \times I),
    \]
    where one identifies \((x, 0) \sim f(x)\) for all \(x \in X\).

    Together with the mapping cylinder, we have two distinguished \emph{embedding}
    morphisms \(\iota_X: X \emb \Cyl(f)\) with \(x \mapsto [x, 0]\), and
    \(\iota_Y: Y \emb \Cyl(f)\) mapping \(y \mapsto [y]\). Moreover, one has a
    \emph{retraction}  \(r: \Cyl(f) \to Y\) given by \([x, t] \mapsto f(x)\) for
    \((x, t) \in X \times I\), while \([y] \mapsto y\) for \(y \in Y\).
\end{definition}

\begin{theorem}
    \label{thm:factorization-morphism-embedding-retraction}
    Given a topological morphism \(f: X \to Y\), there exists morphisms
    \(\iota_X: X \to \Cyl(f)\) and \(r: \Cyl(f) \to Y\) such that the following
    holds:
    \begin{enumerate}[(a)]\setlength\itemsep{0em}
        \item The diagram
              \[
                  \begin{tikzcd}
                      X \ar[rr, "\iota_X"] \ar[dr, "f"'] & &\Cyl(f) \ar[ld, "r"] \\
                      &Y &
                  \end{tikzcd}
              \]
              \emph{commutes} in \(\Top\). That is, every continuous map can be factored
              through an embedding and a retraction of its mapping cylinder.

        \item There exists a \emph{relative homotopy}
              \[
                  \Id_{\Cyl(f)} \simhtrel{Y} \iota_Y r.
              \]
              Thus \(\iota_Y\) and \(r\) are homotopy equivalences.

        \item The embedding \(\iota_X\) of \(X\) into the cylinder of \(f\) is a
              \emph{cofibration}.
    \end{enumerate}
\end{theorem}

\begin{proof}
    Item (a) follows directly from definition: \(r \iota_X(x) = r[x, 0] = f(x)\) for
    any \(x \in X\). For item (b), consider the map \(\eta: \Cyl(f) \times I \to
    \Cyl(f)\) given by
    \[
        \eta(p, s) \coloneq
        \begin{cases}
            [x, (1 - s) t], & \text{if } p = [x, t]
            \text{ for } (x, t) \in X \times I,                        \\
            [y],            & \text{if } p = [y] \text{ for } y \in Y.
        \end{cases}
    \]
    Then \(\eta\) is continuous in both \((X \times I) \times I\) and
    \(Y \times I\), which forms a cover of \(\Cyl(f) \times I\)---therefore \(\eta\)
    is continuous. Moreover, for any \(y \in Y\) we have \(\eta([y], -) = [y]\),
    hence \(\eta\) is a homotopy relative to \(Y\). Since
    \(\eta([x, t], 0) = [x, t]\) and \(\eta([y], 0) = [y]\) then
    \(\eta(-, 0) = \Id_{\Cyl(f)}\). On the other hand,
    \(\eta([x, t], 1) = [x, 0] = [f(x)]\) and \(\eta([y], 1) = [y]\), that is,
    \(\eta(-, 0) = \iota_Y r\). Therefore we can conclude that \(\eta\) establishes
    a relative homotopy \(\Id_{\Cyl(f)} \simhtrel{Y} \iota_Y r\).

    In order to prove item (c), let \(W\) be any topological space and consider
    continuous maps \(g: \Cyl(f) \to W\) and \(\varepsilon: X \times I \to W\) such
    that \(g \iota_X(x) = \varepsilon(x, 0)\) for all \(x \in X\). Define a map
    \(\delta: \Cyl(f) \times I \to W\) given by
    \begin{align*}
        \delta([y], s)    & \coloneq g[y], \\
        \delta([x, t], s) & \coloneq
        \begin{cases}
            g\big[x, \frac{2 t - s}{2 - s}\big],
             & \text{if } s \leq 2 t,
            \\
            \varepsilon\big(x, \frac{s - 2 t}{1 - t}\big),
             & \text{if } 2 t \leq s,
        \end{cases}
    \end{align*}
    where \(y \in Y\) and \(x \in X\). Therefore, one has
    \(\delta([x, t], 0) = g[x, t]\) while \(\delta([y], 0) = g[y]\). Moreover,
    \(\delta\) is an extension of \(\varepsilon\) since
    \(\delta|_{X \times I} = \varepsilon\). This shows that \(\iota_X\) is a
    cofibration.
\end{proof}

\begin{lemma}
    \label{lem:htpy-equivalence-iff-weak-def-retract}
    A continuous map \(f: X \to Y\) is a \emph{homotopy equivalence} if and only if
    \(X\) is a \emph{weak deformation retract} of the mapping cylinder \(\Cyl(f)\).
\end{lemma}

\begin{proof}
    From the factorization of \cref{thm:factorization-morphism-embedding-retraction}
    we have \(f = r \iota_X\). Since \(\iota_Y r \simht \Id_{\Cyl(f)}\), then
    \(\iota_X \simht \iota_Y f\). If \(f\) is a homotopy equivalence, then let
    \(g: Y \to X\) be its homotopy inverse. It follows that the map
    \(g r: \Cyl(f) \to X\) is a homotopy inverse of \(\iota_X\). Therefore \(X\) is
    a weak deformation retract of \(\Cyl(f)\)

    Conversely, if \(X\) is a weak deformation retract of \(\Cyl(f)\), let
    \(k: \Cyl(f) \to X\) be an homotopy inverse of \(\iota_X\). Then the map \(k
    \iota_Y: Y \to X\) is a homotopy inverse of \(f\).
\end{proof}

\begin{theorem}
    \label{thm:same-htpy-type-iff-emb-weak-deformation-retract}
    Two topological spaces \(X\) and \(Y\) have the \emph{same homotopy type} if and
    only if both can be embedded as a \emph{weak deformation retract} of a common
    space \(Z\).
\end{theorem}

\begin{proof}
    If \(f: X \to Y\) is a homotopy equivalence, then by
    \cref{lem:htpy-equivalence-iff-weak-def-retract} we find that \(X\) is a weak
    deformation retract of \(\Cyl(f)\). Since \(Y\) is also a weak deformation
    retract of \(\Cyl(f)\), the statement follows. For the converse, if \(i_X: X \to
    Z\) and \(i_Y: Y \to Z\) are embeddings, that are homotopy equivalences, then
    let \(r: Z \to Y\) be the retract of \(i_Y\) and define \(f \coloneq r i_X: X
    \to Y\), then \(f\) is a homotopy equivalence \(X \isoht Y\).
\end{proof}

\section{Fundamental Groupoid \& The Fundamental Group}

\subsection{Paths}

\begin{notation}[Family of paths]
    \label{not:family-of-paths}
    Let \(X\) be a space and \(x, y \in X\) be any two points. We'll denote by
    \(\Path_X(x, y)\) the family of paths \(\gamma: I \to X\) with \(\gamma(0) = x\) and \(\gamma(1) = y\).
\end{notation}

\begin{definition}[Operations on paths]
    \label{def:operations-paths}
    Let \(X\) be a topological space. We define the following operations on the
    space of paths of \(X\):
    \begin{itemize}\setlength\itemsep{0em}
        \item Given a path \(\gamma: I \to X\) we define the \emph{reverse} path of
              \(\gamma\) to be a path \(\gamma^{-1}: I \to X\) given by
              \(\gamma^{-1}(t) \coloneq \gamma(1 - t)\).

        \item If \(p \in \Path_X(x, y)\) and \(q \in \Path_X(y, z)\) are paths in \(X\),
              we define the \emph{concatenation} of \(p\) with \(q\) to be a path \(q \cdot
              p: I \to X\) given by
              \[
                  (q \cdot p)(t) \coloneq
                  \begin{cases}
                      p(2 t),     & t \in [0, 1/2], \\
                      q(2 t - 1), & t \in [1/2, 1].
                  \end{cases}
              \]
              Yielding a path \(q \cdot p \in \Path_X(x, z)\).

        \item Moreover, we define \(\const_x: I \to X\) to be the unique constant path
              on \(x\)---that is, \(\const_x(t) = x\) for all \(t \in I\).
    \end{itemize}
\end{definition}

\begin{proposition}
    \label{prop:path-connected-fundamental-groupoid-trivial}
    If \(X\) is a path connected topological space, then any two paths on \(X\) are
    homotopic.
\end{proposition}

\begin{proof}
    Let \(f, g: I \para X\) be any two paths. Define, for each \(t \in I\), a path
    \(\gamma_t \in
    \Path_X(f(t), g(t))\)---which exists since \(X\) is path connected. Define a map
    \(\eta: I \times I \to X\) given by \(\eta(t, s) \coloneq \gamma_t(s)\), then it
    is clear that \(\eta\) is a homotopy between \(f\) and \(g\).
\end{proof}

\begin{definition}[Homotopy relative boundary]
    \label{def:htpy-relative-boundary}
    Let \(X\) be a space, and \(\gamma, \gamma': I \para X\) be paths with common
    endpoints:
    \begin{align*}
        \gamma(0) = \gamma'(0) \eqqcolon p_0\quad \text{ and }\quad
        \gamma(1) = \gamma'(1) \eqqcolon p_1.
    \end{align*}
    We define a \emph{homotopy relative boundary} between \(\gamma\) and \(\gamma'\)
    to be a homotopy \(\eta: \gamma \htpy \gamma'\) such that \(\eta\) is constant on
    the endpoints \(p_0\) and \(p_1\)---that is,
    \begin{align*}
        \eta(0, -) = \const_{p_0}\quad \text{ and }\quad \eta(1, -) = \const_{p_1}.
    \end{align*}
\end{definition}

\begin{proposition}
    \label{prop:htpy-rel-boundary-is-equiv-relation}
    Homotopy relative boundary is an equivalence relation on the family of paths of
    a topological space.
\end{proposition}

\begin{proof}
    Let \(X\) be a topological space and \(x, y \in X\) be any two points---we'll
    consider the family
    \(\Path_X(x, y)\). The constant homotopy \(\gamma \htpy \gamma\) is clearly a homotopy relative boundary, thus the relation is reflexive.

    If \(\delta \in \Path_X(x, y)\) is another path, and \(\eta: \gamma \htpy
    \delta\) is a homotopy relative boundary, then we can construct the a reverse
    homotopy \(\sigma: \delta \htpy \gamma\) as \(\sigma(-, t) \coloneq \eta(-, 1 -
    t)\). Notice that \(\sigma(0, t) = \eta(0, 1 - t) = \const_x(1 - t)\) is
    constant on \(x\), while \(\sigma(1, t) = \eta(1, 1 - t) = \const_y(1 - t)\) is
    constant on \(y\). Therefore the relation is symmetric.

    Consider yet another path \(p \in
    \Path_X(x, y)\) and a homotopy relative boundary \(\eta: \delta \htpy p\). We
    define a map \(\varepsilon: I \times I \to X\) given by
    \[
        \varepsilon(-, t) \coloneq
        \begin{cases}
            \sigma(-, 2t),    & t \in [0, 1/2], \\
            \eta(-, 2 t - 1), & t \in [1/2, 1].
        \end{cases}
    \]
    This map is continuous by the same argument used in
    \cref{cor:htpy-equivalence-rel} and thus establishes a homotopy relative boundary
    \(\varepsilon: \gamma \htpy p\).
\end{proof}

\subsection{The Fundamental Groupoid}

\begin{definition}[Fundamental groupoid]
    \label{def:fundamental-groupoid}
    Given a topological space \(X\), we define the \emph{fundamental groupoid} of
    \(X\) to be the category \(\Pi_1(X)\) whose objects are the points of \(X\), and
    whose morphisms are paths between those points up to homotopy relative boundary.
    That is, given \(x, y \in X\) we have
    \[
        \Hom_{\Pi_1(X)}(x, y) = \Path_X(x, y)/{\sim_{\text{hrb}}},
    \]
    where \(\sim_{\text{hrb}}\) is the equivalence relation on the family of paths
    \(x \to y\) given by homotopy relative boundary.

    Composition of morphisms in \(\Pi_1(X)\) is naturally defined by the
    concatenation of paths---in other words, given points \(x, y, z \in X\) we
    have
    \[
        \begin{tikzcd}
            {\Path_X(x, y) \times \Path_X(y, z)}
            \ar[d, two heads] \ar[rr]
            &&{\Path_X(x, z)} \ar[d, two heads]
            \\
            {\Hom_{\Pi_1(X)}(x, y) \times \Hom_{\Pi_1(X)}(y, z)}
            \ar[rr, dashed]
            &&{\Hom_{\Pi_1(X)}(x, z)}
        \end{tikzcd}
    \]
    where, for any paths \(\gamma \in \Path_X(x, y)\) and \(\delta \in \Path_X(y,
    z)\), the concatenation of paths \((\gamma, \delta) \mapsto \delta \cdot \gamma\) induces
    a concatenation operation on the respective class paths
    \[
        ([\gamma], [\delta]) \longmapsto
        [\delta] \cdot [\gamma] \coloneq [\delta \cdot \gamma].
    \]
\end{definition}

\begin{corollary}
    \label{cor:Pi1-is-groupoid}
    \(\Pi_1(X)\) is a groupoid.
\end{corollary}

\begin{proof}
    We show that \(\Pi_1(X)\) is a category whose morphisms are isomorphisms.
    \begin{itemize}\setlength\itemsep{0em}
        \item Given any point \(x \in X\) one has an identity map
              \([\const_x] \in \Hom_{\Pi_1(X)}(x, x)\).

        \item Concatenation of class paths is unital with respect to constant paths:
              given any path class \([\gamma] \in \Hom_{\Pi_1(X)}(x, y)\), we have
              \[
                  [\const_y] \cdot [\gamma]
                  = [\const_y \cdot \gamma]
                  = [\gamma]
                  = [\gamma \cdot \const_x]
                  = [\gamma] \cdot [\const_x].
              \]

        \item Concatenation is associative. Let \(x, y, z, w \in X\) be any four points
              and consider class paths \([\alpha]: x \to y\), \([\beta]: y \to z\), and
              \([\gamma]: z \to w\).

              First we have to show that \(\gamma \cdot (\beta \cdot \alpha)\) and
              \((\gamma \cdot \beta) \cdot \alpha\) are relative boundary homotopic. To that
              end, define a continuous map \(\tau: I \to I\) by
              \[
                  \tau(t) \coloneq
                  \begin{cases}
                      2 t,                       & t \in [0, 1/4],   \\
                      t + \frac{1}{4},           & t \in [1/4, 1/2], \\
                      \frac{t}{2} + \frac{1}{2}, & t \in [1/2, 1].
                  \end{cases}
              \]
              Then, one sees right away that
              \[
                  (\gamma \cdot (\beta \cdot \alpha))(\tau(t))
                  = ((\gamma \cdot \beta) \cdot \alpha)(t)
              \]
              for all \(t \in I\)---hence \(\gamma \cdot (\beta \cdot \alpha)
              \sim_{\text{hrb}} (\gamma \cdot \beta) \cdot \alpha\). From this, we finally
              obtain the associativity of the path classes,
              \[
                  [\gamma] \cdot ([\beta] \cdot [\alpha])
                  = [\gamma \cdot (\beta \cdot \alpha)]
                  = [(\gamma \cdot \beta) \cdot \alpha]
                  = ([\gamma] \cdot [\beta]) \cdot [\alpha].
              \]

        \item Every path class \([\gamma] \in \Hom_{\Pi_1(X)}(x, y)\) is an isomorphism,
              since it has an inverse \([\gamma^{-1}] \in \Hom_{\Pi_1(X)}(y, x)\). Indeed,
              one has
              \[
                  [\gamma] \cdot [\gamma^{-1}] = [\gamma \cdot \gamma^{-1}] = [\const_y]
                  \quad\text{ and }\quad
                  [\gamma^{-1}] \cdot [\gamma] = [\gamma^{-1} \cdot \gamma] = [\const_x].
              \]
    \end{itemize}
\end{proof}

\begin{definition}[The category \(\bpTop\)]
    \label{def:base-point-preserving-Top-cat}
    A \emph{pointed topological space} is a pair \((X, x)\) consisting of a
    topological space \(X\) together with a base-point \(x \in X\). We define a
    category \(\bpTop\) whose objects are pointed topological spaces, and whose
    morphisms \(f: (X, x) \to (Y, y)\), for any \((X, x), (Y, y) \in \bpTop\), are
    continuous maps \(f: X \to Y\) such that \(f(x) = y\). We say that the morphisms
    of \(\bpTop\) preserve base-points.

    In \(\bpTop\), we define a \emph{pointed homotopy} \(\eta: f \htpy g\) between
    parallel morphisms \(f, g: (X, x) \para (Y, y)\) to be a homotopy preserving
    base-points, that is,
    \[
        \eta(x, -) = \const_y.
    \]
    Analogously to homotopies in \(\Top\), pointed homotopies define an equivalence
    relation \(\sim_{\text{ph}}\) in \(\bpTop\). We define the homotopy category of
    \(\bpTop\) to be the category \(\Ho{\bpTop}\) composed of pointed topological
    spaces and morphisms
    \[
        \Hom_{\Ho{\bpTop}}((X, x), (Y, y))
        \coloneq \Hom_{\bpTop}((X, x), (Y, y))/{\sim_{\text{ph}}}.
    \]
    This quotient induces a natural projective functor
    \[
        \kappa^{*/}: \bpTop \longrightarrow \Ho{\bpTop}.
    \]
\end{definition}

\subsection{The Fundamental Group}

\begin{definition}[Fundamental group]
    \label{def:fundamental-group}
    Let \(X\) be a topological space and \(x \in X\) be any point. We define
    the \emph{fundamental group} of \(X\) at the base-point \(x\) as the family of
    \emph{loops}
    \[
        \pi_1(X, x) \coloneq \Aut_{\Pi_1(X)}(x),
    \]
    endowed with the operation of concatenation of paths. Therefore, we see that the
    fundamental group is a functor
    \[
        \pi_1: \bpTop \longrightarrow \Grp.
    \]
\end{definition}

\begin{proposition}
    \label{prop:pi1-equivalent-to-Pi1-for-connected-space}
    Let \(X\) be a path connected space. For every point \(x \in X\) the inclusion
    functor \(\pi_1(X, x) \to \Pi_1 X\) is an \emph{equivalence of categories}.
\end{proposition}

\begin{proof}
    Indeed, notice that \(\pi_1(X, x) = \Aut_{\Pi_1 X}(x)\) and since \(X\) is a
    connected space, \(\Pi_1 X\) is a connected groupoid---that is, for every pair
    of points \(x, y \in \Pi_1 X\), there exists an isomorphism \(x \iso y\) in
    \(\Pi_1 X\). It follows that \(\Aut_{\Pi_1 X} (x) = \Sk (\Pi_1 X)\), and
    therefore the inclusion functor is an equivalence of categories (see
    \cref{exp:skeletal-functor}).
\end{proof}

\begin{definition}[Pushforwards in \(\pi_1\)]
    \label{def:pushforward-pi1}
    Let \(f: (X, x) \to (Y, y)\) be a morphism of pointed topological spaces. There
    exists an induced pushforward
    \[
        f_{*}: \pi_1(X, x) \longrightarrow \pi_1(Y, y)
    \]
    mapping \([\gamma] \mapsto [f \gamma]\), which establishes a group morphism
    between the fundamental groups of the initial pointed topological spaces.
\end{definition}

\begin{proposition}
    \label{prop:pointed-htpy-pushforward-agree}
    Let \(f, g: (X, x) \para (Y, y)\) be morphisms in \(\bpTop\). If there exists a
    \emph{pointed homotopy} \(\eta: f \htpy g\), then \(f_{*} = g_{*}\).
\end{proposition}

\begin{proof}
    Let \(\gamma\) be a loop at \(x\) representing some class of \(\pi_1(X,
    x)\). The pointed homotopy \(\eta\) naturally induces a homotopy
    \(f \gamma \htpy g \gamma\)---thus \([f \gamma] = [g \gamma]\) in
    \(\pi_1(Y, y)\)---therefore \(f_{*} = g_{*}\).
\end{proof}

By \cref{prop:pointed-htpy-pushforward-agree} we obtain a factorization of the
fundamental group functor through the homotopy category of pointed topological
spaces. To put briefly, the following diagram is quasi-commutative
\[
    \begin{tikzcd}
        \bpTop \ar[r, "\pi_1"] \ar[d, "\kappa^{*/}"'] &\Grp \\
        \Ho{\bpTop} \ar[ru, bend right]
    \end{tikzcd}
\]

\begin{corollary}[Preserving isomorphisms]
    \label{cor:pushforward-preserve-isomorphism}
    If \(f: (X, x) \isoto (Y, y)\) is a \emph{homotopy equivalence} in \(\bpTop\),
    then \(f_{*}: \pi_1(X, x) \isoto \pi_1(Y, y)\) is an \emph{isomorphism} in
    \(\Grp\).
\end{corollary}

\begin{proof}
    If \(g\) is the homotopy inverse of \(f\), then for any
    \([\gamma] \in \pi_1(X, x)\) we have
    \(g_{*}f_{*} [\gamma] = g_{*}[f \gamma] = [g f \gamma] = [\gamma]\), therefore
    \(g_{*} f_{*} = \Id_{\pi_1(X, x)}\). Analogously we have
    \(f_{*} g_{*} = \Id_{\pi_1(Y, y)}\). Therefore \(f_{*}\) is an isomorphism of
    groups.
\end{proof}

\subsection{Simply Connected Spaces}

\begin{definition}[Simply connected space]
    \label{def:simply-connected}
    A topological space \(X\) is said to be \emph{simply connected} if it is path
    connected and its fundamental group is
    trivial, \(\pi_1(X, x) \iso 1\) for any base-point \(x \in X\).
\end{definition}

\todo[inline]{Continue on semi-locally simply connected spaces}

\subsection{Examples of Fundamental Groups}

\begin{example}[\(\pi_1\) of the euclidean space]
    \label{exp:euclidean-space-pi1-is-trivial}
    Let \(x \in \R^n\) be any point. Given any loop \(\ell: x \to x\), one can
    define a homotopy relative boundary \(\eta: \ell \htpy \const_x\) given by
    \(\eta(s, t) \coloneq (1 - t) \ell(s) + t x\). Therefore the fundamental group
    of the \(n\)-dimensional euclidean space is trivial,
    \[
        \pi_1(X, x) = *.
    \]
\end{example}

\begin{proposition}
    \label{prop:pi1-manifold-is-countable}
    The fundamental group of a topological manifold is countable.
\end{proposition}

\begin{proof}
    Let \(\mathcal{B}\) be a countable open cover of \(M\) consisting of coordinate
    balls. Since \(M\) has countably many connected components (see
    \cref{prop:connectivity-manifolds}), those connected components are also path
    components. This implies that any two \(B, B' \in \mathcal{B}\) are such that
    the intersection \(B \cap B'\) are composed of at most countably many path
    components (since \(M\) is locally path connected). Let \(\mathcal{X}\) be a
    collection containing a single point from each \(B \cap B'\) for any pair
    \((B, B') \in \mathcal{B} \times \mathcal{B}\). For every \(B \in \mathcal{B}\)
    and points \(x, y \in \mathcal{X}\) with \(x, y \in B\), fix a path
    \(\gamma_{(x, y)}^B \in \Path_B(x, y)\).

    Clearly, the fundamental groups of a path connected component of \(M\) based at
    any two points are always isomorphic. Since \(\mathcal{X}\) contains at least
    one point of each path component of \(M\), we may choose a base-point
    \(p \in \mathcal{X}\). Let \(\Gamma_p\) denote the collection of loops based at
    \(p\) that are equal to a finite concatenation of paths of the form
    \(\gamma_{(x, y)}^B\), for some \(B \in \mathcal{B}\). Since \(\mathcal{B}\) is
    countable, so is \(\Gamma_p\). We'll settle to prove that every element of
    \(\pi_1(M, p)\) can be represented by a loop in \(\Gamma_p\).

    Let \(f\) be any loop based at \(p\). Consider the collection
    \((f^{-1}(B))_{B \in \mathcal{B}}\), which is an open cover of \(I\). Since
    \(I\) is compact, there exists a finite subcover out of such collection. The
    finite subcover gives rise to a finite collection of numbers
    \[
        0 \eqqcolon a_0 < a_1 < \dots < a_k \coloneq 1
    \]
    for which the closed interval \([a_{j-1}, a_j]\) is contained in some
    \(f^{-1}(B)\).

    For each \(0 < j \leq k\), define \(f_j: I \to M\) to be the path given by
    \[
        f_j(t) \coloneq f((1 - t) a_{j-1} + t a_j),
    \]
    that is, \(f_j\) is a path \(f(a_{j-1}) \to f(a_j)\)---and let
    \(B_j \in \mathcal{B}\) be such that \(\im f_j \subseteq B_j\). From
    construction we have that \(f(a_j) \in B_j \cap B_{j+1}\) for each
    \(0 \leq j < k\).

    For each \(0 \leq j < k\) let \(g_j \in \Path_{B_j \cap B_{j+1}}(x_j, f(a_j))\),
    where \(x_j \in \mathcal{X}\) as in the first paragraph---and \(x_0, x_k
    \coloneq p\), with constant paths \(g_0, g_k \coloneq \const_p\). Notice that
    \begin{align*}
        f & \simht f_k \cdots f_1                                                               \\
          & \simht g_k^{-1} f_k (g_{k-1} g_{k-1}^{-1}) \cdots (g_2 g_2^{-1}) f_2 (g_1 g_1^{-1})
        f_1 g_0                                                                                 \\
          & \simht (g_k^{-1} f_k  g_{k-1}^{-1}) \cdots (g_1^{-1} f_1 g_0),                      \\
    \end{align*}
    where \(g_j^{-1} f_j g_{j-1} \in \Path_{B_j}(x_{j-1}, x_j)\) for all
    \(0 < j \leq k\). Since each \(B_j\) is simply connected, then
    \(g_j^{-1} f_j g_{j-1}\) is relative boundary homotopic to chosen path
    \(\gamma_{(x_{j-1}, x_j)}^B\). This shows that \(f\) is homotopic to a path in
    \(\Gamma_p\)---therefore \(\pi_1(M, p)\) is countable.
\end{proof}

%%% Local Variables:
%%% mode: latex
%%% TeX-master: "../../deep-dive"
%%% End:


\section{Mappings \texorpdfstring{\(S^1 \longrightarrow S^1\)}{S1 -> S1}}

We shall study the circle via the inclusion \(S^1 \emb \CC\) mapping
\((\cos(2 \pi t), \sin(2 \pi t)) \mapsto e^{2 \pi \img t}\) and the
identification \(q: I \to S^1\) given by \(t \mapsto e^{2 \pi \img t}\). In
general, we shall also make use of the map \(\R \to S^1\) given by
\(x \mapsto e^{2 \pi \img x}\).

Each path \(\phi: (I, 0) \to (\R, 0)\) between pointed spaces, can be
\emph{lifted} to a unique path \(\widehat \phi: (S^1, 1) \to (S^1, 1)\) such
that the diagram
\[
    \begin{tikzcd}
        I \ar[r, "\phi_n"] \ar[d] &\R \ar[d] \\
        S^1 \ar[r, dashed, "\widehat \phi_n"'] &S^1
    \end{tikzcd}
\]
commutes in \(\Top\). Explicitly, we have
\[
    \widehat \phi_n(e^{2 \pi \img t}) = e^{2 \pi \img \phi_n(t)}.
\]

\begin{proposition}[Unwinding pointed maps]
    \label{prop:unwinding-pointed-maps}
    Let \(f: (S^1, 1) \to (S^1, 1)\) be a \emph{pointed} continuous map. Then there
    exists a \emph{unique pointed morphism} \(\phi: (I, 0) \to (\R, 0)\) such that
    \(f = \widehat \phi\).
\end{proposition}

\begin{proof}
    Define \(h: I \to S^1\) to be the morphism \(h \coloneq f q\)---which is
    uniformly continuous since \(I\) is a compact space. From the latter properties,
    one  can find a finite partition
    \[
        0 \eqqcolon t_0 < t_1 < \dots < t_k \coloneq 1
    \]
    of \(I\), such that \(|h(t) - h(t_j)| < 2\) for all \(t \in [t_j, t_{j+1}]\),
    for each \(0 \leq j < k\). This is done in order to ensure that
    \(h(t) \neq -h(t_j)\)---so that the complex logarithm\footnote{ If
        \(z = r e^{\img \theta} \in \CC\) is a complex number, in polar form,
        satisfying \(r > 0\) and \(\theta \in (-\pi, \pi)\), then the \emph{complex
            logarithm} of \(z\) is \(\Log(z) = \log(r) + \img \theta\).
    } of the quotient \(h(t)/h(t_j)\) is
    well defined. With these conditions being satisfied, we can define a map
    \(\phi: I \to \R\) given by
    \[
        \phi(t) \coloneq
        \frac{1}{2 \pi \img} \bigg(
        \Log\Big( \frac{h(t_1)}{h(t_0)} \Big)
        + \dots
        + \Log\Big( \frac{h(t_k)}{h(t_{k-1})} \Big)
        + \Log\Big( \frac{h(t)}{h(t_j)} \Big)
        \bigg)
    \]
    for all \(t \in [t_j, t_{j+1}]\). Therefore one has
    \begin{align*}
        q \phi(t)
         & = e^{2 \pi \img \phi(t)}
        \\
         & = e^{
                \Log\big( \frac{h(t_1)}{h(t_0)} \big)
                + \dots
                + \Log\big( \frac{h(t_j)}{h(t_{j-1})} \big)
                + \Log\big( \frac{h(t)}{h(t_j)} \big)
            }
        \\
         & = e^{\Log\big( \frac{h(t_1)}{h(t_0)} \big)}
        \cdots
        e^{\Log\big( \frac{h(t_j)}{h(t_{j-1})} \big)}
        e^{\Log\big( \frac{h(t)}{h(t_k)} \big)}
        \\
         & = \frac{h(t_{1})}{h(t_0)} \cdots
        \frac{h(t_{j})}{h(t_{j-1})}
        \cdot
        \frac{h(t)}{h(t_j)}
        \\
         & = \frac{h(t)}{h(t_0)}.
    \end{align*}
    Now, from construction we know that \(h(t_0) = h(0) = 1\), thus
    \(q \phi(t) = h(t)\). Moreover \(\phi(0) = 0\), which implies that \(\phi\) is a
    continuous pointed map of the form \((I, 0) \to (\R, 0)\) such that
    \[
        \widehat \phi(t) = f(e^{2 \pi \img t}).
    \]

    For the uniqueness of \(\phi\), suppose that \(\psi: (I, 0) \to (\R, 0)\) is
    another pointed morphism such that \(f = \widehat \psi\). Then in particular
    \(\widehat \phi = \widehat \psi\), which implies in
    \(e^{2 \pi \img \phi(t)} = e^{2 \pi \img \psi(t)}\) for each \(t \in I\). This can
    only be the case if \(\phi(t) - \psi(t) \in \Z\). Therefore, since the map
    \(\phi - \psi\) is a continuous map of the form \((I, 0) \to (\Z, 0)\), it
    follows that \(\phi = \psi\)---since \(I\) is connected and \(\Z\) is discrete.
\end{proof}

\begin{theorem}[Unwinding maps]
    \label{thm:unwinding-maps}
    Let \(f: S^1 \to S^1\) be any morphism, then there exists a \emph{unique pointed
        morphism} \(\phi: (I, 0) \to (\R, 0)\) such that \(f = f(1) \widehat \phi\).
\end{theorem}

\begin{proof}
    From \(f\) we can define a \emph{pointed} morphism \(g: (S^1, 1) \to (S^1, 1)\)
    by \(g \coloneq f(1)^{-1} f\), so that indeed \(g(1) = 1\). From
    \cref{prop:unwinding-pointed-maps} there exists a unique pointed morphism
    \(\phi: (I, 0) \to (\R, 0)\) such that \(g = \widehat \phi\). Therefore, from
    the definition of \(g\) we find that \(f = f(1) \widehat \phi\).
\end{proof}

\begin{lemma}
    \label{lem:path-to-int-is-htpy-turns-on-circle}
    Let \(n \in \Z\) and \(\phi \in \Path_{\R}(0, n)\). If we consider the linear
    path \(\phi_n: I \to \R\) given by \(\phi_n(t) \coloneq t n\), then there
    exists a relative homotopy \(\widehat \phi \simhtrel{1} \widehat \phi_n\)
    between the lifted paths in the circle.
\end{lemma}

\begin{proof}
    There exists a relative linear homotopy \(\ell: \phi \htpyrel{\Bd I} \phi_n\)
    given by
    \[
        \ell(s, t) \coloneq (1 - t) \phi(s) + t \phi_n(s).
    \]
    Notice that this homotopy can be lifted to
    \(\widehat \ell: S^1 \times I \to S^1\) mapping
    \[
        \widehat\ell(e^{2 \pi \img s}, t)
        = e^{2 \pi \img \big( (1-t)\phi(s) + t \phi_n(s) \big)}
        = e^{2 \pi \img (1-t)\phi(s)} e^{2 \pi \img t \phi_n(s)}
    \]
    which is a continuous map such that
    \(\widehat \ell(e^{2 \pi \img s}, 0) = e^{2 \pi \img \phi(s)} = \widehat
    \phi(e^{2 \pi \img s})\) while on the other end
    \(\widehat \ell(e^{2 \pi \img s}, 1) = e^{2 \pi \img \phi_n(s)} = \widehat
    \phi_n(e^{2 \pi \img s})\). That is, \(\widehat \ell\) establishes a relative
    homotopy \(\widehat \phi \simhtrel{1} \widehat \phi_n\) as we desired.
\end{proof}

\begin{proposition}
    \label{prop:circle-morphism-is-winding}
    Let \(f: S^1 \to S^1\) be a morphism. Then there exists a unique integer
    \(n \in \Z\) for which \(f \simht \widehat \phi_n\), where \(\phi_n: I \to \R\)
    is the linear path \(\phi_n(t) \coloneq t n\).
\end{proposition}

\begin{proof}
    From \cref{thm:unwinding-maps} we find that \(f = f(1) \widehat \phi\) for a
    unique pointed map \(\phi: (I, 0) \to (\R, 0)\). Since
    \(\widehat \phi: (S^1, 1) \to (S^1, 1)\) is a pointed continuous map, it follows
    that
    \[
        \widehat\phi(1) = \widehat \phi(e^{2 \pi \img}) = e^{2 \pi \img \phi(1)} = 1,
    \]
    which implies in \(\phi(1) \coloneq n \in \Z\). From
    \cref{lem:path-to-int-is-htpy-turns-on-circle} we get
    \(\widehat \phi \simhtrel{1} \widehat \phi_n\).

    Notice that if \(\zeta_0 \coloneq e^{2\pi \img s_0} \in S^1\) is any fixed
    point, the multiplication map \(\mul: S^1 \to S^1\) given by
    \(\mul(\zeta) \coloneq \zeta_0 \zeta\) is a rotation of the point \(\zeta\) on
    the circle---we'll show that \(\mul \simht \Id_{S^1}\). Let
    \(\delta: S^1 \times I \to S^1\) be the map
    \(\delta(e^{2 \pi \img s}, t) \coloneq e^{2 \pi \img ((1 - t) s_0 + s)}\), so
    that \(\delta(-, 0) = \mul\) and \(\delta(-, 1) = \Id_{S^1}\).

    From the last paragraph we find that
    \[
        f = f(1) \cdot \widehat \phi
        \simht \Id_{S_1} \widehat \phi
        = \widehat \phi
        \simht \widehat \phi_n,
    \]
    therefore \(f \simht \widehat \phi_n\) as wanted.
\end{proof}

\begin{definition}[Degree]
    \label{def:degree-S1-map}
    Given an endomorphism \(f: S^1 \to S^1\), let \(\phi: I \to \R\) be the unique
    path such that \(f = f(1) \widehat \phi\)---then we define the \emph{degree} of
    \(f\) to be \(\deg f \coloneq \phi(1) \in \Z\).
\end{definition}

\begin{lemma}
    \label{lem:homotopic-S1-maps-have-equal-degree}
    Let \(f, g: S^1 \para S^1\) be endomorphisms of the circle. Then \(f \simht g\)
    if and only if \(\deg f = \deg g\).
\end{lemma}

\begin{proof}
    (\(\implies\)) Let \(\eta: f \htpy g\) be a homotopy. For each \(s \in I\) there
    exists an endomorphism \(\eta_s \coloneq \eta(-, s): S^1 \to S^1\) and, from
    \cref{thm:unwinding-maps}, there is a unique pointed continuous map
    \(\phi_s: (I, 0) \to (\R, 0)\) with \(\phi_s(1) \in \Z\) for which
    \begin{equation}\label{eq:etas-from-widehat-phis}
        \eta_s = \eta_s(1) \widehat \phi_s.
    \end{equation}
    We shall construct an explicit equation for \(\phi_s\) and prove that the
    mapping \(\Phi: I \times I \to \R\) given by \(\Phi(t, s) \coloneq \phi_s(t)\)
    is a homotopy.

    We proceed as in the proof of \cref{prop:unwinding-pointed-maps}: define a map
    \(\varepsilon: I \times I \to S^1\) by
    \[
        \varepsilon(t, s) \coloneq \eta_s(e^{2 \pi \img t}) = \eta_s q(t).
    \]
    Since \(I \times I\) is compact, \(\varepsilon\) is uniformly
    continuous---therefore one can choose a partition
    \[
        0 \eqqcolon t_0 < t_1 < \dots < t_k \coloneq 1
    \]
    of \(I\) such that \(|\varepsilon(t, s) - \varepsilon(t_j, s)| < 2\) for all
    \(s \in I\) and \(t \in [t_j, t_{j+1}]\)---this ensures that the complex
    logarithm of \(\varepsilon(t, s)/\varepsilon(t_j, s) \neq -1\) is well
    defined. For each \(s \in I\), we construct a map \(\psi_s: I \to S^1\) by
    \begin{equation}\label{eq:explicit-form-psis-actually-phis}
        \psi_s(t) \coloneq \frac{1}{2 \pi \img} \bigg(
        \Log \Big( \frac{\varepsilon(t_1, s)}{\varepsilon(t_0, s)} \Big)
        + \dots +
        \Log \Big( \frac{\varepsilon(t_j, s)}{\varepsilon(t_{j-1}, s)} \Big)
        + \Log \Big( \frac{\varepsilon(t, s)}{\varepsilon(t_j, s)} \Big)
        \bigg)
    \end{equation}
    for all \(t \in [t_j, t_{j+1}]\). Then for every \(s \in I\) one has
    \begin{align*}
        q \psi_s(t) & = e^{2 \pi \img \psi_s(t)}                                             \\
                    & = e^{
                \Log \big( \frac{\varepsilon(t_1, s)}{\varepsilon(t_0, s)} \big)
                + \dots +
                \Log \big( \frac{\varepsilon(t_j, s)}{\varepsilon(t_{j-1}, s)} \big)
                + \Log \big( \frac{\varepsilon(t, s)}{\varepsilon(t_j, s)} \big)
        }                                                                                    \\
                    & = e^{\Log \big( \frac{\varepsilon(t_1, s)}{\varepsilon(t_0, s)} \big)}
        \dots
        e^{\Log \big( \frac{\varepsilon(t_j, s)}{\varepsilon(t_{j-1}, s)} \big)}
        e^{\Log \big( \frac{\varepsilon(t, s)}{\varepsilon(t_j, s)} \big)}                   \\
                    & = \frac{\varepsilon(t_1, s)}{\varepsilon(t_0, s)}
        \cdots
        \frac{\varepsilon(t_j, s)}{\varepsilon(t_{j-1}, s)}
        \cdot
        \frac{\varepsilon(t, s)}{\varepsilon(t_j, s)}                                        \\
                    & = \frac{\varepsilon(t, s)}{\varepsilon(t_0, s)}                        \\
                    & = \frac{\eta_s(e^{2 \pi \img t})}{\eta_s(1)}.
    \end{align*}
    Therefore \(\eta_s = \eta_s(1) \widehat \psi_s\), which shows that
    \(\psi_s = \phi_s\) (from \cref{eq:etas-from-widehat-phis}). Looking at
    \cref{eq:explicit-form-psis-actually-phis} we see that the mapping \(\Phi\)
    above-mentioned is continuous, hence a homotopy.

    Considering the continuous map \(\Phi(1, -): I \to \R\) given by
    \(s \mapsto \phi_s(1)\), we know that \(\phi_s(1) \in \Z\) from earlier
    considerations---therefore using the fact that \(I\) is connected and \(\Z\) is
    discrete, it must bae the case that the induced map \(\Phi(1, -): I \to \Z\) is
    \emph{constant}. In particular, we shall have \(\phi_0(1) = \phi_1(1)\)---but
    from definition we have \(f = f(1) \widehat \phi_0\) and
    \(g = g(1) \widehat \phi_1\), so that \(\deg f = \phi_0(1)\) and
    \(\deg g = \phi_1(1)\). From these considerations we can finally conclude that
    \(\deg f = \deg g\).

    (\(\impliedby\)) For the converse, suppose that \(\deg f = \deg g \coloneq
    n\). Therefore there exists two unique pointed morphisms
    \(\phi, \psi \in \Path_{\R}(0, n)\) such that \(f = f(1) \widehat \phi\) and
    \(g = g(1) \widehat \psi\). Recall that since \(f(1), g(1) \in S^1\) are unitary
    complex numbers, multiplying a circle point by them amounts to a rotation of the
    initial point through the circle---which was already shown to be homotopic to
    the identity map of the circle. Therefore we conclude that
    \(\phi \simht \phi_n \psi\), where \(\phi_n \in \Path_{\R}(0, n)\) is the linear
    path \(\phi_n(t) \coloneq t n\). Thus
    \[
        f \simht \widehat \phi
        \simht \widehat \phi_n
        \simht \widehat \psi
        \simht g,
    \]
    proving that \(f\) and \(g\) are homotopic.
\end{proof}

\begin{lemma}
    \label{lem:degree-nth-power-map-is-n}
    For each \(n \in \Z\) the map \(e_n: S^1 \to S^1\) given by \(\zeta \mapsto
    \zeta^n\) is a continuous map of degree
    \[
        \deg e_n = n.
    \]
\end{lemma}

\begin{proof}
    Notice that from definition we have \(e_n(e^{2 \pi \img t}) = e^{2 \pi \img (n
            t)} = q \phi_n(t)\), where \(\phi_n \in \Path_{\R}(0, n)\) is the linear path
    \(\phi_n(t) \coloneq t n\). Therefore \(\deg e_n = \phi_n(1) = n\).
\end{proof}

\begin{theorem}[\(\deg\) is a ring isomorphism]
    \label{thm:deg-is-ring-isomorphis}
    The map \(\deg: [S^1, S^1] \to \Z\) given by \([f] \mapsto \deg f\) is a
    \emph{ring isomorphism}. In other words, the degree map is an isomorphism of the
    \emph{first cohomology group of \(S^1\)} with \(\Z\):
    \[
        H^1(S^1) \iso \Z.
    \]
\end{theorem}

\begin{proof}
    From \cref{lem:homotopic-S1-maps-have-equal-degree} we know that \(\deg\) is a
    bijective map, we need to show that it's also a ring morphism. Consider any two
    endomorphism classes \([f], [g] \in [S^1, S^1]\) with degrees \(\deg [f]
    \coloneq n\) and \(\deg [g] \coloneq m\), then:
    \begin{itemize}\setlength\itemsep{0em}
        \item (Additive structure). From \cref{lem:degree-nth-power-map-is-n} we know
              that \(f \simht e_n\) and \(g \simht e_m\), moreover given any
              \(\zeta \in S^1\) one has
              \[
                  (e_n \cdot e_m)(\zeta) = e_n(\zeta) e_m(\zeta) = \zeta^{n + m}.
              \]
              Therefore \(e_n \cdot e_m = e_{n + m}\) and, consequently
              \[
                  \deg([f] \cdot [g])
                  = \deg([e_n] \cdot [e_m])
                  = \deg [e_n \cdot e_m]
                  = \deg [e_{n+m}]
                  = n + m
                  = \deg [f] + \deg [g].
              \]

        \item (Multiplicative structure). Given any \(\zeta \in S^1\) we have
              \[
                  e_n e_m(\zeta) = e_n(\zeta^m) = \zeta^{m n} = \zeta^{n m},
              \]
              therefore \(e_n e_m = e_{n m}\). From this we find that
              \[
                  \deg([f] \circ [g])
                  = \deg([e_n] \circ [e_m])
                  = \deg [e_n \circ e_m]
                  = \deg [e_{n m}]
                  = n m
                  = \deg [f] \cdot \deg [g].
              \]
    \end{itemize}
    This finishes the proof that \(\deg\) is an isomorphism of rings
    \[
        [S^1, S^1] \iso \Z.
    \]
\end{proof}

\begin{corollary}[Automorphisms of \(S^1\)]
    \label{cor:automorphism-of-circle-is-reflection-or-identity}
    If \(f \in \Aut_{\Top}(S^1)\) is an automorphism, then \(\deg f = \pm 1\). As a
    consequence, either \(f \simht \Id_{S^1}\) or \(f \simht \rho\), where \(\rho\)
    represents the reflection by complex conjugation.
\end{corollary}

\begin{proof}
    If \(f\) is an automorphism, let \(f^{-1}: S^1 \to S^1\) be its
    inverse. Therefore \(f f^{-1} = \Id_{S^1}\), implying in
    \[
        1 = \deg \Id_{S^1} = \deg(f f^{-1}) = \deg f \cdot \deg f^{-1},
    \]
    which can only be the case if \(\deg f = \deg f^{-1} = \pm 1\). Notice that the
    identity morphism is \(\Id_{S^1} = e_1\), therefore \(\deg \Id_{S^1} = 1\). On
    the other hand, the reflection \(\rho: S^1 \to S^1\) maps
    \(e^{2 \pi \img t} \mapsto e^{-2 \pi \img t}\)---thus \(\rho = e_{-1}\),
    implying in \(\deg \rho = -1\). From the latter two considerations, the last
    proposition follows.
\end{proof}

\begin{example}
    \label{exp:null-homotopic-has-degree-zero}
    A null-homotopic map \(f: S^1 \to S^1\) has \(\deg f = 0\), since it's homotopic
    to a constant map.
\end{example}

\begin{corollary}
    \label{cor:circle-not-contractible}
    The circle \(S^1\) is not contractible.
\end{corollary}

\begin{proof}
    Since \(\deg \Id_{S^1} = 1\), from our considerations from
    \cref{exp:null-homotopic-has-degree-zero} we conclude that \(\Id_{S^1}\) isn't
    null-homotopic and therefore there exists no contraction of the circle.
\end{proof}

\begin{corollary}
    \label{cor:no-retraction-from-disk-to-circle}
    There exists no retraction \(D^2 \to S^1\).
\end{corollary}

\begin{proof}
    Suppose, for the sake of contradiction, that there exists a retraction \(r: D^2
    \to S^1\) such that \(r \iota = \Id_{S^1}\), where \(\iota: S^1 \emb D^2\) is
    the canonical inclusion. Since \(D^2\) is a contractible space, then \(r\) is
    null-homotopic. This would imply that \(r \iota\) was null-homotopic and thus
    would be \(\Id_{S^1}\), which contradicts
    \cref{cor:circle-not-contractible}. Therefore \(r\) cannot exist.
\end{proof}

\subsection{Applications of \texorpdfstring{\(\pi_1(S^1)\)}{pi1(S1)}}

\begin{theorem}[Brouwer's fixed point]
    \label{thm:brouwer-fixed-point}
    Every continuous map \(D^2 \to D^2\) has a fixed point.
\end{theorem}

\begin{proof}
    Let \(f: D^2 \to D^2\) be any continuous map. Suppose, for the sake of
    contradiction, that \(f\) admits no fixed point. We construct a map
    \(r: D^2 \to S^1\) as follows: for each \(x \in D^2\), since \(f x \neq x\), we
    can define a point \(r x \in S^1\) given by
    \(r x \coloneq \frac{f x - x}{\| fx - x \|}\). Notice \(r\) defines a retraction
    of the disk to the sphere, contradicting the result of
    \cref{cor:no-retraction-from-disk-to-circle}, therefore \(f\) must admit at
    least one fixed point.
\end{proof}

\begin{theorem}[Fundamental Theorem of Algebra]
    \label{thm:fundamental-theorem-of-algebra-pi1-S1}
    Every non-constant polynomial with coefficients in \(\CC\) has root in \(\CC\).
\end{theorem}

\begin{proof}
    Let \(p(z) = z^n + a_1 z^{n-1} + \dots + a_n\) be any polynomial---note that the
    restriction to the case of monomials does not create any loss of
    generality. Suppose, for the sake of contradiction, that \(p\) admits no roots
    in \(\CC\). Then for any real number \(r \geq 0\)
    \todo[inline]{Finish this proof (look Hatcher's book)}
\end{proof}

\section{Normed Vector Spaces}

\todo[inline]{Define normed vector space and stuff\dots}

\subsection{Properties of Normed Vector Spaces}

\begin{proposition}
\label{prop:continous-linear-thus-bounded}
Let \(f: X \to Y\) be an \(\R\)-linear map between normed \(\R\)-vector spaces
\(X\) and \(Y\). Then, \(f\) is continous if and only if there exists a scalar
\(C > 0\), called \emph{bound}, for which \(\norm{f(x)}_Y \leq C \norm{x}_{X}\)
for all \(x \in X\).
\end{proposition}

\begin{proof}
If \(f\) is bounded by \(C\), let \(B\) be a basis for \(X\) and consider any
element \(x \coloneq \sum_{v \in B} a_{v} v\). From linearity we have \(f(x) =
\sum_{v \in B} a_v f(v)\), thus
\[
  \norm{f(x)}_Y = \norm{\sum_{v \in B} a_v f(v)}_{Y}
  \leq \sum_{v \in B} \norm{a_v}_{\R} \norm{f(v)}_{Y}
  \leq C \norm{x}_{X} \sum_{v \in B} \norm{a_v}_{\R}.
\]
This boild down to \(f = O(\Id_{X})\) --- which implies in \(f(x - x_0) = f(x) -
f(x_0) \to 0\) as \(x \to x_0\), where \(x_0 \in X\) is any point, that is,
\(f\) is continous at any point of \(X\). Even better than that, we can show
that \(f\) is uniformly continous (I won't carry it out since it's equivalent to
what we wrote in \cref{prop: linear-continuous}).

For the opposite, suppose \(f\) is continous at \(0\), then there will surely
exist \(\delta > 0\) for which \(\norm{x}_X < \delta\) implies in
\(\norm{f(x)}_Y < 1\). Therefore, for any choice of non-zero \(x \in X\), we
find
\[
  \norm{f\left( \frac{\delta}{\norm{x}_X} x \right)}_Y
  = \frac{\delta}{\norm{x}_X} \norm{f(x)}_{Y} < 1,
\]
therefore \(\norm{f(x)} < \frac{\norm{x}_{X}}{\delta}\), thus \(f\) is indeed
bounded.
\end{proof}

\begin{proposition}
\label{prop:composition-banach-morphisms}
Let \(E, F\) and \(G\) be normed vector spaces, and let \(u: E \to F\) and \(v:
F \to G\) be continous linear maps. Then, \(v u: E \to G\) is also a continuous
linear map, and
\[
  \norm{v u} \leq \norm{v} \norm{u},
\]
where \(\norm{f} \coloneq \sup_{x \in X} \norm{f(x)}\) for a continous linear
map \(f: X \to Y\) between normed vector spaces\footnote{Beware! This is not the
norm we shall adopt for our studies on banachable topological vector spaces. For
the latter, see \cref{def:norm-morphism-TopVect}}.
\end{proposition}

\begin{proof}
The first assertion is trivial, since the composition of continous maps is
continous, and the same is true for linear maps. Let \(x \in E\) be any element,
notice that
\[
  \norm{vu(x)}_G
  \leq \norm{v} \norm{u(x)}_F
  \leq \norm{v} \norm{u} \norm{x}_{E},
\]
thus the inequality holds.
\end{proof}

\begin{proposition}
\label{prop:multilinear-continous-iff-bounded}
A multilinear map \(\phi: \prod_{j=1}^n E_j \to F\) between normed vector spaces
\(E_1, \dots, E_n\), and \(F\), is continous if and only if there exists a bound
\(C > 0\) such that, for every \(x \in \prod_{j = 1}^n E_j\),
\[
  \norm{\phi(x)}_{F} \leq C \prod_{j=1}^n \norm{x_j}_{E_j}.
\]
\end{proposition}

\begin{proposition}
\label{prop:canonical-iso-banach-multilinear}
Let \(E_1, \dots, E_r\), and \(F\) be normed vector spaces. There exists a
canonical map from repeated continous linear maps to the continous multilinear
maps, which is a continuous linear isomorphism, and is norm-preserving --- that
is, the canonical map
\[
  \Phi: L(E_1, L(E_2, \dots, L(E_r, F), \dots)) \isoto L^n(E_1, \dots, E_n; F)
\]
is a \emph{Banach isomorphism}.
\end{proposition}

\begin{proof}
We define \(\Phi\) by the following: if \(\lambda \in L(E_1, L(E_2, \dots,
L(E_n, F) \dots))\) is given by
\[
  \lambda(x_1) = \lambda_2,\ \text{ where }\ \lambda_2(x_2) = \lambda_3, \
  \dots, \ \lambda_n(x_n) = y \in F,
\]
we define \(\Phi(\lambda) \coloneq \overline{\lambda} \in L(E_1, \dots, E_n;
F)\) by the mapping
\[
  \overline\lambda(x_1, \dots, x_n) \coloneq \lambda(x_1)(x_2) \dots (x_n),
\]
where \(\lambda_j(x_j)(x_{j+1}) \dots (x_n) \coloneq \lambda_{j-1}(x_{j-1})
\dots (x_n)\) for every \(1 \leq j \leq n\) --- where \(\lambda_1 \coloneq
\lambda\).

Given \(\lambda \in L(E_1, L(E_2, \dots, L(E_n, F), \dots))\), the map
\(\overline\lambda\) is surely multilinear since each of the recursive arguments
are linear. Moreover, notice that, for any \(x \in \prod_{j=1}^n E_j\), we have
\[
  \norm{\overline\lambda(x)}_{F}
  \leq \norm{\lambda(x_1) (x_2) \dots (x_n)}_{F}
  \leq \norm{\lambda} \prod_{j=1}^n \norm{x_{j}}_{E_j},
\]
thus \(\norm{\overline\lambda} \leq \norm{\lambda}\).

On the other hand, given \(\overline\phi \in L(E_1, \dots, E_n; F)\), define the
map \(\phi = \Phi^{-1}(\overline\phi)\) by
\[
  \phi(x_1)(x_2) \dots (x_n) \coloneq \overline\phi(x_1, \dots, x_n).
\]
Therefore
\[
  \norm{\phi(x_1)(x_2) \dots (x_n)}_{F}
  \leq \norm{\overline\phi} \prod_{j=1}^n \norm{x_j}_{E_j},
\]
which shows that \(\norm{\phi} \leq \norm{\overline\phi}\). We conclude that
\(\Phi(\lambda) = \overline\lambda\) for all repeating map \(\lambda\).
\end{proof}

\begin{theorem}[Hahn-Banach]
\label{thm:Hahn-Banach}
Let \(E\) be a normed \(\R\)-vector space, and \(F \subseteq E\) be a
subspace. Let \(\lambda \in F^{*}\) be a functional with bound \(C > 0\). Then
there exist an extension of \(\lambda\) to a functional on \(E\) with the same
bound \(C\) --- that is, a map \(\overline \lambda: E \to \R\) such that
\(\overline\lambda|_F = \lambda\) and \(\norm{\overline\lambda(x)}_\R \leq C
\norm{x}_{E}\) for all \(x \in E\).
\end{theorem}

\begin{corollary}[Hahn-Banach]
\label{cor:Hahn-Banach}
Let \(E\) be a Banach space and \(x \in E\) be a non-zero element. There exists
a continuous linear map \(\phi \in E^{*}\) such that \(\phi(x) \neq 0\).
\end{corollary}

\subsection{Properties of Banach Spaces}

\begin{definition}
\label{def:banach-isomorphism}
We define a \emph{Banach isomorphism} to be a continuous linear map \(u: E \to
F\), between Banach spaces \(E\) and \(F\), that is both invertible (there
exists a continous linear map \(u^{-1}: F \to E\) that is the two-sided inverse
of \(u\)), and norm preserving --- that is, given any \(x \in E\), we have
\(\norm{u(x)}_F = \norm{x}_E\). Banach isomorphisms may also be refered to
isometries in the literature.
\end{definition}

\begin{proposition}[Bijections are isomorphisms]
\label{prop:continuous-bijective-linear-is-isomorphism}
Every continuous bijective \(\R\)-linear map between topological vector spaces
is an isomorphism.
\end{proposition}

\begin{proposition}[Splitting]
\label{prop:banach-split}
Let \(E\) be a Banach space, and \(F\) and \(G\) be complementary closed
subspaces of \(E\) --- that is, \(E = F + G\) and \(F \cap G = 0\). Then the
morphism \(F \times G \to E\) given by \((f, g) \mapsto f + g\) is a continuous
linear isomorphism.
\end{proposition}

\todo[inline]{Continue, there are tons of things to study on Banach spaces\dots}
%%% Local Variables:
%%% mode: latex
%%% TeX-master: "../../../deep-dive"
%%% End:

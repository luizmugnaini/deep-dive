\subsection{Distribution of primes}

\begin{definition}
   We define the function \(\pi: \mathbb{Z}_{>1} \to
   \mathbb{Z}_{\geqslant 0}\) for a given \(n \in \mathbb{Z}_{\geqslant 0}\) to
   be the number of prime numbers \(1 < p \leqslant n\).
\end{definition}

\begin{theorem}
   A weak lower bound: \(\pi(x) \geqslant \log(\log(x))\) for \(x \geqslant 2\).
\end{theorem}

\begin{proof}
   Since \(\prod_{j=1}^n p_j + 1\) is not divisible by any of the primes in
   \(\{p_1, \dots, p_n\} \) then the next prime needs to satisfy \(p_{n+1}
   \leqslant \prod_{j=1}^n p_j + 1\) because there is always a prime (less than
   or equal to the number, dividing it). (Proof by induction that \(p_n <
   2^{2^n}\)) If by hypothesis of induction \(p_n < 2^{2^n}\) then  
   \[
      p_{n+1} < \prod_{j=1}^n 2^{2^j} + 1 = 2^{\sum_{j=1}^n 2^j} + 1 =
      2^{2^{n+1} - 2}+1 < 2^{2^{n+1}}.
   \]
   Thus, \(p^n < 2^{2^n}\) and therefore \(\pi(2^{2^n}) \geqslant n\). Now, let
   \(x > e\) and let \(n \in \mathbb{Z}\) be such that \(e^{e^{n-1}} < x
   \leqslant e^{e^n}\). For \(n > 3\) we have \(e^{n-1} > 2^n\) thus 
   \[
      \pi(x) \geqslant  \pi(e^{e^{n-1}}) \geqslant  \pi(e^{2^n}) \geqslant n
      \geqslant \log(\log(x)).
   \] 
   which is a proof for \(x > e^e\). Now, for the case where \(x \leqslant e^e\)
   
\end{proof}

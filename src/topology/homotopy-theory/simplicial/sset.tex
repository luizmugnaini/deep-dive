\section{Simplex Category}

\subsection{Construction}

\begin{definition}[Skeletal simplex category]
\label{def:skeletal-simplex-category}
We denote by \(\Splx\) the category whose objects are natural numbers
\([n] \coloneq \{0 \leq 1 \leq \dots \leq n\}\), and whose morphisms
\(\phi: [n] \to [m]\) are order-preserving --- that is, if \(i \leq j\) then
\(\phi(i) \leq \phi(j)\). The category \(\Splx\) is called the \emph{skeletal
  simplex category}.

Along with the simplex category comes distinguished morphisms:
\begin{itemize}\setlength\itemsep{0em}
\item For each \(0 \leq j \leq n\), we denote by
  \(\delta_j^n: [n - 1] \mono [n]\) the order-preserving \emph{injective}
  morphism skipping the \(j\)-th value:
  \[
  \delta_j^n(i) \coloneq
  \begin{cases}
    i, &\text{if } i < j, \\
    i + 1, &i \geq j.
  \end{cases}
  \]
  This morphism is called \emph{elementary faces}.

\item For each \(0 \leq j \leq n\), we denote by
  \(\sigma_j^n: [n + 1] \epi [n]\) the \emph{surjective} morphism repeating the
  \(j\)-th value twice and every other value only once:
  \[
  \sigma_j^n(i) \coloneq
  \begin{cases}
    i, &\text{if } i \leq j, \\
    i - 1, &\text{if } i > j.
  \end{cases}
  \]
  This morphism is called \emph{elementary degeneracies}.
\end{itemize}
When convenient, we may drop the superscript of these maps and simply refer to
them as \(\delta_i\) and \(\sigma_j\).
\end{definition}

\begin{lemma}[Generating morphisms in \(\Splx\)]
\label{lem:elementary-maps-generate-simplex-maps}
Every morphism of \(\Splx\) can be generated by a composition of elementary
faces and degeneracies.
\end{lemma}

\begin{proof}
Let \(f: [n] \to [m]\) be any morphism of \(\Splx\). We can factor \(f\) through
an injection \(\iota: [n] \mono [k]\) and a surjection \(s: [k] \epi [m]\):
\[
\begin{tikzcd}
{[n]} \ar[rr, "f"] \ar[dr, two heads, "s"'] &&{[m]} \\
&{[k]} \ar[ru, tail, "\iota"'] &
\end{tikzcd}
\]
where \(k \coloneq |\im f|\). Since \(s\) and \(\iota\) must be order-preserving
maps, it follows that they can be written as a finite composition of elementary
degeneracies and elementary faces.
\end{proof}

\begin{corollary}
\label{cor:mono-and-epi-in-Splx-are-split-mono-and-epi}
Every monomorphism of \(\Splx\) is split (admits a retraction),
while every epimorphism of \(\Splx\) is split (admits a section).
\end{corollary}

\begin{proof}
Given an elementary face \(\delta_j: [n] \mono [n+1]\) the elementary degeneracy
\(\sigma_j: [n+1] \epi [n]\) is a retract of \(\delta_j\), and \(\delta_j\) is a
section of \(\sigma_j\).
\end{proof}

\begin{corollary}[Cosimplicial identities]
\label{cor:cosimplicial-identities}
Fix any \(n \in \N\) and consider indices \(0 \leq i, j \leq n\). The following
identities correlate elementary faces and degeneracies:
\begin{enumerate}[(1)]\setlength\itemsep{0em}
\item If \(i < j\) then
  \(\delta_j^n \delta_i^{n-1} = \delta_i^n \delta_{j-1}^{n-1}\).

\item If \(i < j\) then
  \(\sigma_i^{n-1} \sigma_j^n = \sigma_{j-1}^{n-1} \sigma_i^n \).

\item If \(i < j\) then
  \(\sigma_i^{n-1} \delta_j^n = \delta_{j-1}^{n-1} \sigma_i^{n-2}\).

\item If \(i = j - 1\) or \(i = j\), then \(\sigma_i^n \delta_j^{n+1} = \Id_n\).

\item If \(i > j\) then
  \(\sigma_i^{n-1} \delta_j^n = \delta_j^{n-1} \sigma_{i-1}^{n-2}\).
\end{enumerate}
These identities can be found in the following four commutative diagrams:
%
\begin{equation*}
  \begin{tikzcd}
  {[n-2]} \ar[r, "\delta_i^{n-1}"] \ar[d, "\delta_{j-1}^{n-1}"']
  &{[n-1]} \ar[d, "\delta_j^n"] \\
  {[n-1]} \ar[r, "\delta_i^n"'] &{[n]}
  \end{tikzcd}
  \qquad
  \qquad
  \begin{tikzcd}
  {[n+1]} \ar[r, "\sigma_j^n"] \ar[d, "\sigma_i^n"']
  &{[n]} \ar[d, "\sigma_i^{n-1}"] \\
  {[n]} \ar[r, "\sigma_{j-1}^{n-1}"'] &{[n-1]}
  \end{tikzcd}
\end{equation*}
%
\begin{equation*}
  \begin{tikzcd}
  {[n - 1]} \ar[r, "\delta_j^n"] \ar[d, "\sigma_i^{n-2}"']
  &{[n]} \ar[d, "\sigma_i^{n-1}"] \\
  {[n-2]} \ar[r, "\delta_{j-1}^{n-1}"'] &{[n-1]}
  \end{tikzcd}
  \qquad
  \qquad
  \begin{tikzcd}
  {[n - 1]} \ar[r, "\delta_j^n"] \ar[d, "\sigma_{i-1}^{n-2}"']
  &{[n]} \ar[d, "\sigma_i^{n-1}"] \\
  {[n-2]} \ar[r, "\delta_j^{n-1}"'] &{[n-1]}
  \end{tikzcd}
\end{equation*}
\end{corollary}

\begin{definition}[Cosimplicial object]
\label{def:cosimp-obj}
Given a category \(\cat C\), we define a \emph{cosimplicial object} in
\(\cat C\) to be a covariant functor
\[
F: \Splx \longrightarrow \cat C.
\]
Any cosimplicial object \(F\) is completely determined by \(F [n]\) for all
\(n \in \N\) and by the maps \(F \delta_i\) and \(F \sigma_j\). The collection
of all cosimplicial objects of \(\cat C\) will be denoted by
\(\CoSimp{\cat C}\).
\end{definition}

\subsection{Limits \& Colimits in \texorpdfstring{\(\Splx\)}{Delta}}

\begin{lemma}[Pushout of injections]
\label{lem:splx-cat-pushout-injections}
Let \(\iota: [k] \emb [n]\) and \(\tau: [k] \emb [m]\) be order-preserving
inclusions where
\[
\iota(j) \coloneq j\ \text{ and }\ \tau(j) \coloneq j + (m - k),
\]
that is, \(\iota\) sends \([k]\) to the initial segment of \([n]\), while
\(\tau\) sends \([k]\) to the terminal segment of \([m]\). There exists a
pushout
\[
\begin{tikzcd}
{[k]} \ar[r, hook, "\iota"] \ar[d, hook, "\tau"']
\ar[rd, "\ulcorner", very near end, phantom]
&{[n]} \ar[d] \\
{[m]} \ar[r] &{[n] \cup_{[k]} [m]}
\end{tikzcd}
\]
in the simplex category \(\Splx\).
\end{lemma}

\begin{proof}
Indeed, if we consider \([m + n - k]\) as a candidate for the pushout, notice
that the following diagram commutes
\[
\begin{tikzcd}
{[k]} \ar[r, hook, "\iota"] \ar[d, hook, "\tau"']
&{[n]} \ar[d, "\tau"] \\
{[m]} \ar[r, "\iota"'] &{[m + n - k]}
\end{tikzcd}
\]
Now, consider any object \([\ell] \in \Splx\) together with two morphisms
\(f: [n] \to [\ell]\) and \(g: [m] \to [\ell]\) such that \(f \iota = g
\tau\). Define a map \(\phi: [m + n - k] \to [\ell]\) as follows
\[
\phi(j) \coloneq
\begin{cases}
  g(j),                  &\text{if } j \leq m - k, \\
  g(j) = f(j - (m - k)), &\text{if } m - k \leq j \leq m, \\
  f(j - (m - k)),        &\text{if } m \leq j \leq m + n - k.
\end{cases}
\]
It is easy to see that \(\phi\) is an order-preserving map and is uniquely
defined so that the following diagram commutes
\[
\begin{tikzcd}
{[k]} \ar[r, hook, "\iota"] \ar[d, hook, "\tau"']
&{[n]} \ar[d, "\tau"] \ar[drd, bend left, "f"] & \\
{[m]} \ar[r, "\iota"'] \ar[rrd, bend right, "g"']
&{[m + n - k]} \ar[rd, dashed, "\phi"] & \\
& &{[\ell]}
\end{tikzcd}
\]
Therefore \([m + n - k] = [n] \cup_{[k]} [m]\) since we are in a skeletal
category and isomorphism classes contain a unique representative.
\end{proof}

These pushouts lead to an interesting construction, any object \([n] \in \Splx\)
is the colimit of a diagram consisting of \([0]\)'s and \([1]\)'s, since
\[
\begin{tikzcd}
{[0]} \ar[r, hook, "0"]
\ar[d, hook, "m"'] \ar[rd, phantom, very near end, "\ulcorner"]
&{[n]} \ar[d] \\
{[m]} \ar[r] &{[m + n]}
\end{tikzcd}
\]

\begin{lemma}[Pushout of surjections]
\label{lem:splx-cat-pushout-surjections}
The following properties concern the pushout of pairs of surjective morphisms in
\(\Splx\):
\begin{enumerate}[(a)]\setlength\itemsep{0em}
\item Considering the cosimplicial identity (2) (see
  \cref{cor:cosimplicial-identities}), where \(i < j\), there exists
  \emph{sections} \(\alpha: [n] \to [n+1]\) of \(\sigma_i^n\), and
  \(\beta: [n-1] \to [n]\) of \(\sigma_i^{n-1}\) such that the following diagram
  commutes
  \[
  \begin{tikzcd}
  {[n+1]} \ar[rr, "\sigma_j^n"] \ar[dd, bend left, "\sigma_i^n"]
  &&{[n]} \ar[dd, bend right, "\sigma_i^{n-1}"'] \\ & &\\
  {[n]} \ar[uu, bend left, "\alpha"] \ar[rr, "\sigma_{j-1}^{n-1}"']
  &&{[n-1]} \ar[uu, bend right, "\beta"']
  \end{tikzcd}
  \]
  that is, \(\sigma_j^n \alpha = \beta \sigma_{j-1}^{n-1}\) --- these sections
  are said to be \emph{compatible} with the square. Therefore, the square is an
  \emph{absolute pushout}.
\item Let \(p: [n] \epi [k]\) and \(q: [n] \epi [\ell]\) be \emph{surjections}
  in \(\Splx\). Then the \emph{pushout} of \(p\) and \(q\) exists and is
  \emph{absolute}:
  \[
  \begin{tikzcd}
  {[n]} \ar[r, two heads, "q"]
  \ar[d, two heads, "p"']
  \ar[rd, very near end, phantom, "\ulcorner"]
  &{[\ell]} \ar[d] \\
  {[k]} \ar[r] &{[m]}
  \end{tikzcd}
  \]
\end{enumerate}
\end{lemma}

\begin{proof}
For the proof of item (a), proceed as follows. From the cosimplicial identity
(3) we have \(\delta_i^{n+1} \sigma_i^n = \sigma_i^{n+1} \delta_{i+1}^{n+2}\),
moreover, from (4) we obtain \(\sigma_i^{n+1} \delta_{i+1}^{n+2} = \Id_{n+1}\)
--- thus \(\delta_i^{n+1}\) is a section of \(\sigma_i^n\). Analogously, we find
that \(\delta_i^n\) is a section of \(\sigma_i^{n-1}\). For the
compatibility condition, notice that since \(i < j\), then from (5) we know that
\(\sigma_j^n \delta_i^{n+1} = \delta_i^n \sigma_{j-1}^{n-1}\). Therefore we may
define \(\alpha \coloneq \delta_i^{n+1}\) and \(\beta \coloneq \delta_i^n\).

For item (b), since surjections are finite compositions of degeneracies, via
item (a) and \cref{prop:absolute-pushout}, we conclude that the square in (b) is
an absolute pushout.
\end{proof}

\begin{lemma}
\label{lem:splx-cat-pullback-mono-pushout-epi}
The following are properties of pullbacks along monomorphisms and pushouts along
epimorphisms in the simplex category \(\Splx\):
\begin{enumerate}[(a)]\setlength\itemsep{0em}
\item If \(f: [m] \mono [n]\) is a \emph{monomorphism}, then if \(g: [k] \to
  [n]\) is any morphism such that \(\im g \cap \im f \neq \emptyset\), then the
  square
  \[
  \begin{tikzcd}
  {[a]}  \ar[r] \ar[d] \ar[rd, phantom, very near start, "\lrcorner"]
  &{[k]} \ar[d, "g"] \\
  {[m]} \ar[r, tail, "f"'] &{[n]}
  \end{tikzcd}
  \]
  is a \emph{pullback} in \(\Splx\).
\item If \(f: [m] \epi [n]\) is an \emph{epimorphism}, then for any morphism
  \(g: [m] \to [k]\), then square
  \[
  \begin{tikzcd}
  {[m]} \ar[r, two heads, "f"] \ar[d, "g"']
  \ar[dr, very near end, phantom, "\ulcorner"]
  &{[n]} \ar[d] \\
  {[k]} \ar[r] &{[b]}
  \end{tikzcd}
  \]
  is a \emph{pushout} in \(\Splx\).
\end{enumerate}
\end{lemma}

% \begin{proof}
% For item (a), since \(\im g \cap \im f \neq \emptyset\), then the preimage
% \(g^{-1}(\im f) \subseteq [k]\) is a non-empty set

% For item (b), since \(f\) is a surjection, it can be decomposed into
% degeneracies. Thus it is sufficient to prove the proposition for an elementary
% degeneracy \(\sigma_j: [m] \to [m-1]\). The pushout of the pair \((\sigma_j,
% g)\) is given by a map
% \end{proof}
\todo[inline]{I did not understand how to prove this, the proof presented on the
book is a bit mysterious}

\subsection{Simplicial Sets}

\begin{definition}[Simplicial object]
\label{def:simplicial-object}
For any category \(\cat C\), a \emph{simplicial object} in \(\cat C\) is a
contravariant functor \(\Splx^{\op} \to \cat C\). We define \(\Simp{\cat C}\) to
be the category whose objects are simplicial objects and natural transformations
between them---such category is commonly referred to as the \emph{simplicial
  category of \(\cat C\)}.

Less compactly, a simplicial object \(X \in \Simp{\cat C}\) consists of a
collection of objects \((X_n)_{n \in \N}\), for
\(X_n \coloneq X [n] \in \cat C\), and arrows \(\alpha^{*}: X_n \to X_m\) in
\(\cat C\) for each map \(\alpha: [m] \to [n]\) in \(\Splx\). The nature of
these arrows are as discussed in \cref{rem:yoneda-f*-notation}.

Since every map \(\alpha\) in \(\Splx\) may be decomposed into elementary face
and degeneracies, the the collection \(\delta_i^{*}\) and \(\sigma_j^{*}\) also
generate the morphisms \(\alpha^{*}\) between the objects of
\((X_n)_{n \in \N}\) of \(\cat C\). In the context of the simplicial category,
we denote them by
\begin{align*}
  d_i^n &\coloneq (\delta_i^n)^{*}: X_n \longrightarrow X_{n-1}, \\
  s_j^n &\coloneq (\sigma_j^n)^{*}: X_n \longrightarrow X_{n+1},
\end{align*}
for all \(0 \leq i, j \leq n\). The arrows \(d_i\) are called \emph{face maps}
(or \emph{cofaces}), while the arrows \(s_j\) are called \emph{degeneracy maps}
(or \emph{codegeneracies}) of the simplicial object \(X\).

A map \(\eta: X \nat Y\) is a natural transformation between simplicial objects
if and only if it is compatible with the face and degeneracy maps, that is, the
following two diagrams commute in \(\cat C\) for all \(n \in \N\),
\(0 \leq i, j\leq n\):
\[
\begin{tikzcd}
X_n \ar[r, "\eta_n"] \ar[d, "d_i^n"'] &Y_n \ar[d, "d_i^n"] \\
X_{n-1} \ar[r, "\eta_{n-1}"'] &Y_{n-1}
\end{tikzcd}
\qquad
\qquad
\begin{tikzcd}
X_n \ar[r, "\eta_n"] \ar[d, "s_j^n"'] &Y_n \ar[d, "s_j^n"] \\
X_{n+1} \ar[r, "\eta_{n+1}"'] &Y_{n+1}
\end{tikzcd}
\]

In particular, the most important case we'll study for the time being is the
simplicial category of sets, which we denote by \(\sSet\). The points of a
given set \(X_n\) will be referred to as the \emph{\(n\)-simplices} of
\(X \in \sSet\).
\end{definition}

\begin{corollary}[Simplicial identities]
\label{cor:simplicial-identities}
Since the face and degeneracy maps \(d_i\) and \(s_j\) are \emph{dual} to the
elementary face and degeneracy maps \(\delta_i\) and \(\sigma_j\), they satisfy
the following identities---which are dual to \cref{cor:cosimplicial-identities}:
\begin{enumerate}[(1)]\setlength\itemsep{0em}
\item \(d_i d_j = d_{j-1} d_i\), for \(i < j\).
\item \(s_j s_i = s_i s_{j-1}\), for \(i < j\).
\item \(d_j s_i = s_i d_{j-1}\), for \(i < j-1\).
\item \(d_j s_i = \Id\), if \(i = j\) or \(i = j-1\).
\item \(d_j s_i = s_{i-1} d_j\), for \(i > j\).
\end{enumerate}
\end{corollary}

\subsubsection{Geometric Realization of the Simplex Category}

We define, for each \(n \in \N\), a corresponding \emph{standard topological
  \(n\)-simplex} \(\Delta^n\) given by
\[
\Delta^n \coloneq \bigg\{(t_0, \dots, t_n) \in \R^{n+1} \colon \sum_{j=0}^n t_j = 1
\text{ and } t_j \geq 0 \text{ for all } j
\bigg\}.
\]
Each \(\Delta^n\) is composed of \(n+1\) vertices
\(v_j \coloneq (\delta_{ij})_{i=0}^n\).

From a categorical point of view, standard simplices are nothing more than a functor
\[
\Delta^{\bullet}: \Splx \longrightarrow \Top,
\]
mapping objects \(\Delta^{\bullet}[n] \coloneq \Delta^n\)---where \(\Delta^n\)
is endowed with the standard euclidean topology---and for each
morphism \(f: [n] \to [m]\) in \(\Splx\), we map
\(\Delta^{\bullet} f \coloneq f_{*}\), where \(f_{*}: \Delta^n \to \Delta^m\) is
a uniquely determined continuous map such that
\(f_{*}(v_j) \coloneq v_{f(j)}\). From this definition we obtain
\[
f_{*}(t_0, \dots, t_n) = (s_0, \dots, s_m) \text{,\ \ where \ }
s_j = \sum_{f(i) = j} t_i,
\]
that is, \(s_j\) is the sum of the points that are collapsed to the \(j\)-th
coordinate.

This functor gives us a geometric visualization of the action of the elementary
face and degeneracies:
\begin{itemize}\setlength\itemsep{0em}
\item Given an elementary \emph{face} map \(\delta_j: [n-1] \mono [n]\), for any
  \(0 \leq j \leq n\), the corresponding map
  \((\delta_j)_{*}: \Delta^{n-1} \emb \Delta^n\) is given by
  \[
  (\delta_j)_{*} v_i =
  \begin{cases}
    v_i, &\text{if } i < j, \\
    v_{i+1}, &\text{if } i \geq j.
  \end{cases}
  \]
  That is, \((\delta_j)_{*}\) embeds \(\Delta^{n-1}\) as a face of \(\Delta^n\)
  opposite to the \(j\)-th vertex.

\item An elementary \emph{degeneracy} map \(\sigma_j: [n+1] \epi [n]\), for any
  \(0 \leq j \leq n\), has a map \((\sigma_j)_{*}: \Delta^{n+1} \epi \Delta^n\)
  mapping the vertices as follows
  \[
  (\sigma_j)_{*} v_i =
  \begin{cases}
    v_i, &\text{if } i \leq j, \\
    v_{i-1}, &\text{if } i > j.
  \end{cases}
  \]
  Geometrically, the degeneracy map makes \(\Delta^{n+1}\) into \(\Delta^n\) by
  removing a face of dimension \(1\)---through the projection parallel to the
  line connecting \(v_j\) and \(v_{j+1}\).
\end{itemize}

\subsubsection{Geometric Realization of a Simplicial Set}

Given a simplicial set \(X: \Splx^{\op} \to \Set\), we consider the topological
space
\[
\coprod_{n \in \N} X_n \times \Delta^n
\]
and construct in this space a minimal equivalence relation \(\sim_{\text{gr}}\)
for which points \((x, t) \in X_n \times \Delta^n\) and
\((x', t') \in X_m \times \Delta^m\) are
\emph{equivalent}---\((x, t) \sim_{\text{gr}} (x', t')\)---if and only if there
exists a morphism \(\alpha: [m] \to [n]\) in \(\Splx\) such that
\(x' = \alpha^{*} x\) and \(t = \alpha_{*} t'\). In an equivalent manner, we may
summarize this equivalence relation as gluing points of the form
\[
(x, \alpha_{*} t) \sim_{\text{gr}} (\alpha^{*} x, t).
\]
The points of the resulting quotient space
\((\coprod_{n \in \N} X_n \times \Delta^n)/{\sim_{\text{gr}}}\) are denoted by
\(x \otimes t\)---corresponding to the class of a pair \((x, t)\).

\todo[inline]{Study why this construction is related to a tensor product of the
  form \(X \otimes_{\Splx} \Delta^{\bullet}\).}

\begin{definition}[Geometric realization functor]
\label{def:geometric-realization-functor}
We define the \emph{geometric realization} of the category of simplicial sets to
be a functor
\[
|-|: \sSet \longrightarrow \Top,
\]
mapping
\(|X| \coloneq (\coprod_{n \in \N} X_n \times \Delta^n)/{\sim_{\text{gr}}}\) and
for each natural transformation \(\eta: X \to Y\) we have a topological morphism
\(|\eta|: |X| \to |Y|\) given by \(x \otimes t \mapsto \eta_n x \otimes t\),
for \((x, t) \in X_n \times \Delta^n\).

For any simplicial set \(X\), each \(n\)-simplex \(x \in X_n\) induces
\emph{topological} morphism
\[
\widehat{x}: \Delta^n \longrightarrow |X|\text{, \ mapping }\
t \longmapsto x \otimes t.
\]
From construction, given a morphism \(\alpha: [n] \to [m]\) in \(\Splx\) and a
point \(y \in X_m\) such that \(y = \alpha^{*} x\), the diagram
\begin{equation}\label{eq:commutativity-n-simplexes}
\begin{tikzcd}
\Delta^m \ar[rr, "\alpha_{*}"] \ar[rd, "\widehat y"']
& &\Delta^n \ar[ld, "\widehat x"] \\
&{|X|} &
\end{tikzcd}
\end{equation}
commutes in \(\Top\).

\todo[inline]{This looks like there exists an induced slice over category
  \(\cat C/|X|\)---where \(\cat C\) is a subcategory of \(\Top\) consisting of
  standard simplices and morphisms between them---whose objects are
  \(\widehat x\) and morphisms \(\phi: \widehat x \to \widehat y\) are maps
  \(\phi_{*}: \Delta^m \to \Delta^n\) for some \(\phi: [n] \to [m]\) in
  \(\Splx\).}
\end{definition}

Given any category \(\cat C\) and a functor \(F: \Splx \to \cat C\), we can
define a \emph{simplicial set} induced by any object \(C \in \cat C\) given by
\[
\Hom_{\cat C}(F(-), C): \Splx^{\op} \longrightarrow \Set.
\]
This simplicial set maps each \([n] \in \Splx\) to the set of morphisms
\(\Hom_{\cat C}(F n, C)\), and each morphism \(\alpha: [m] \to [n]\) of
\(\Splx\) to the set-function
\(\alpha^{*}: \Hom_{\cat C}(F n, C) \to \Hom_{\cat C}(F m, C)\).


\begin{definition}[Singular complex]
\label{def:singular-complex-functor}
Let \(\cat C\) be a category and \(F: \Splx \to \cat C\) be a functor. We define
the \emph{singular complex} of \(\cat C\) to be the functor
\[
\Sing_F: \cat C \longrightarrow \sSet
\]
mapping each object \(C \in \cat C\) to the simplicial set
\[
\Sing_F(C) \coloneq \Hom_{\cat C}(F(-), C): \Splx^{\op} \longrightarrow \Set,
\]
and each map \(\alpha: [m] \to [n]\) to the set-function
\[
\Sing_F(\alpha) \coloneq \alpha^{*}:
\Hom_{\cat C}(F n, C) \longrightarrow \Hom_{\cat C}(F m, C).
\]

In particular, the standard simplices functor \(\Delta^{\bullet}: \Splx \to
\Top\) induces a singular complex on each topological space. Since we shall be
mostly interested in this particular case for the time being, we shall reserve
the notation
\[
\Sing \coloneq \Sing_{\Delta^{\bullet}}: \Top \longrightarrow \sSet,
\]
with no subscripts, for the standard simplices. In this case, given a
topological space \(T\), we shall denote by \(\Sing(T)_n\) the image of
\([n] \in \Splx\) under the simplicial set \(\Sing(T)\).
\end{definition}

Given a simplicial set \(X\), the collection of simplexes
\((\widehat{x}_n)_{n \in \N}\), where \(x_n \in X_n\), covers the whole
topological space \(|X|\)---in the sense that collection of images forms a cover
of \(|X|\). Therefore, given any topological morphism \(\phi: |X| \to T\), this
map is completely defined by the family of compositions
\((\phi \widehat{x}_n: \Delta^n \to T)_{n \in \N}\). Given a morphism
\(\alpha: [n] \to [n]\) in \(\Splx\), by \cref{eq:commutativity-n-simplexes},
the diagram
\[
\begin{tikzcd}
\Delta^m \ar[rr, "\alpha_{*}"]
\ar[rd, "\widehat y_m"']
\ar[drd, bend right, "\phi \widehat y_m"']
& &\Delta^n \ar[ld, "\widehat x_n"]
\ar[dld, bend left, "\phi \widehat x_n"] \\
&{|X|} \ar[d, "\phi"] & \\
&T &
\end{tikzcd}
\]
commutes in \(\Top\). This construction induces unique a collection of maps
\[
(\phi_n: X_n \longrightarrow \Sing(T)_n)_{n \in \N},
\]
where \(\phi_n(x) \mapsto \phi \widehat x\). Notice that this family of arrows
is nothing more than a natural transformation \(\phi: X \nat \Sing(T)\) between
simplicial sets. From this we conclude that there exists a natural bijection
\[
\Hom_{\Top}(|X|, T) \iso \Hom_{\sSet}(X, \Sing(T)),
\]
thus the singular complex functor is \emph{right adjoint} to the geometric
realization,
\[
\begin{tikzcd}
\sSet \ar[rr, shift left, "{|-|}"] &&\Top \ar[ll, shift left, "\Sing"]
\end{tikzcd}
\]

\subsection{Geometric Realization as a CW-complex}

\begin{definition}[Degenerate \(n\)-simplex]
\label{def:degenerate-n-simplex}
Given a simplicial set \(X\), we say that an \(n\)-simplex \(x \in X_n\) is
\emph{degenerate} if \(x \in s_j X_{n-1}\) for some \emph{codegeneracy} map
\(s_j: X_{n-1} \to X_n\), where \(0 \leq j \leq n-1\).

Equivalently, \(x\) is denenerate if there exists a \emph{surjective} map
\(\alpha: [n] \epi [m]\) and \(m\)-simplex \(y \in X_m\) for which
\(x = \alpha^{*} y\).
\end{definition}

\begin{lemma}[Eilenberg-Zilber]
\label{lem:Eilenberg-Zilber}
Let \(x\) be a \emph{degenerate} \(n\)-simplex of a given simplicial set
\(X\). There \emph{exists a unique} pair \((\alpha, y)\) such that
\(\alpha: [n] \epi [k]\) is a \emph{surjective} map and \(y\) is a
\emph{non-degenerate} \(k\)-simplex of \(X\) satisfying \(\alpha^{*} y = x\).
\end{lemma}

\begin{proof}
The existence of the pair \((\alpha, y)\) comes straight from definition. Now
suppose \((\beta, z)\) is another pair satisfying the said property, where \(z\)
is a non-degenerate \(\ell\)-simplex of \(X\). Since pushouts in \(\Splx\) are
absolute (see \cref{lem:splx-cat-pushout-surjections}), the pushout of the pair
\((\alpha, \beta)\):
\[
\begin{tikzcd}
{[n]} \ar[r, two heads, "\alpha"]
\ar[d, two heads, "\beta"']
\ar[dr, phantom, "\ulcorner", very near end]
&{[k]} \ar[d, "\gamma"] \\
{[\ell]} \ar[r, "\omega"'] &{[s]}
\end{tikzcd}
\]
is turned into a pullback by the simplicial set \(X: \Splx^{\op} \to \Set\),
that is:
\[
\begin{tikzcd}
X_n \ar[rd, phantom, very near start, "\lrcorner"]
&X_k \ar[l, "\alpha^{*}"']
\\
X_{\ell} \ar[u, "\beta^{*}"]
&X_s \ar[u, "\gamma^{*}"'] \ar[l, "\omega^{*}"]
\end{tikzcd}
\]
From the pushout property, \(\gamma\) and \(\omega\) are epimorphisms, thus
split, \(X\) ensures that \(\gamma^{*}\) and \(\omega^{*}\) are
split-epimorphisms in \(\Set\). Therefore there exists \(s\)-simplices
\(a, b \in X_s\) such that \(\gamma^{*} a = y\) and \(\omega^{*} b = z\). Notice
however that we assumed \(y\) and \(z\) to be both non-degenerate, hence it must
be the case that \(\gamma\) and \(\omega\) are \emph{identities}. This implies
in \(\beta = \alpha\) and \(y = z\).
\end{proof}

%%% Local Variables:
%%% mode: latex
%%% TeX-master: "../../deep-dive"
%%% End:

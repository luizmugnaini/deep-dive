\section{Connected Spaces}

\begin{definition}[Connected space]
\label{def:connected-space}
Let \(X\) be a topological space. We say that \(X\) is \emph{connected} if and
only if one of the following conditions hold:
\begin{enumerate}[(a)]\setlength\itemsep{0em}
\item The space \(X\) cannot be expressed as the union of two disjoint non-empty
  open sets.
\item Every morphism \(f: X \to \{0, 1\}\) is constant --- where \(\{0, 1\}\) is a
  space endowed with the discrete topology.
\end{enumerate}
\end{definition}

\begin{corollary}
The two conditions in \cref{def:connected-space} are equivalent.
\end{corollary}

\begin{proof}
Assume \(X\) satisfies the condition (a), and let \(x \in X\) be any
point. Suppose \(f(x) = 0\) (or \(f(x) = 1\), we do not loose generality by
choosing a point of the domain), and consider \(f^{-1}(0) \subseteq X\), which
must be open from the continuity of \(f\), thus \(f^{-1}(0)\) is a neighbourhood
of \(x\) in \(X\). If we suppose, for the sake of contradiction, that there
exists \(y \in X\) such that \(f(y) = 1\), then \(f^{-1}(1)\) is also open and
is a neighbourhood of \(y\) --- notice however that \(f^{-1}(0) \cap f^{-1}(1) =
\emptyset\), thus we arrive at a contradiction, there must be no point \(y\)
with image different than zero.

We prove the counter positive: not (b) implies not (a). Suppose \(f: X
\to \{0, 1\}\) is not constant, so that there exists two points \(x, y \in X\)
such that \(f(x) = 0\) and \(f(y) = 1\) (and therefore \(f\) is surjective),
notice however that, since \(f\) is continous,
\end{proof}

\begin{definition}[Connected components]
\label{def:connected-components}
Let \(X\) be a topological space. Define an equivalence relation \(\sim\) by,
given \(x, y \in X\), we have \(x \sim y\) if and only if there exists a
connected subspace of \(X\) containing both \(x\) and \(y\). The collection of
equivalence classes \(X/{\sim}\) is called \emph{connected components} of \(X\).
\end{definition}

\begin{notation}[Standard interval]
From now on, when talking about \emph{paths} and \emph{homotopies}, we shall
reserve the symbol \(I\) to denote the \emph{standard topological interval},
which is defined by
\[
  I \coloneq [0, 1] \subseteq \R.
\]
\end{notation}

\begin{definition}[Paths \& loops]
\label{def:path-loop}
A \emph{path} in a topological space \(X\) is any continuous map \(\gamma: I \to
X\). A \emph{loop} in \(X\) is a continuous map \(\ell: I \to X\) such that
\(\ell(0) = \ell(1)\).
\end{definition}

\begin{definition}[Path connected space]
\label{def:path-connected}
A topological space \(X\) is said to be \emph{path connected} if and only if
for all \(x, y \in X\) there exists a \emph{path} connecting \(x\) and \(y\).
\end{definition}

\begin{proposition}[Path connected equivalence relation]
\label{prop:path-connected-equiv-relation}
There exists an equivalence relation \(\sim\) on the topological \(X\) defined
by: given \(x, y \in X\), we have \(x \sim y\) if and only if there exists a
path in \(X\) connecting \(x\) and \(y\).
\end{proposition}

\begin{proof}
The constant path \(x: I \to X\) given by \(t \mapsto x\) is a path on
\(X\), thus \(x \sim x\). Let \(x \sim y\) and \(\gamma\) be a path from
\(\gamma(0) = x\) to \(\gamma(1) = y\), then we can define a map \(\lambda: [0,
1] \to X\) given by \(\lambda(t) \coloneq \gamma(1 - t)\), so that \(\lambda\)
is both continuous, and \(\lambda(0) = y\) while \(\lambda(1) = x\) --- thus
\(\lambda\) is a path between \(y\) and \(x\), therefore \(y \sim x\). Suppose
now \(y \sim z\) and let \(\eta\) be a path connecting \(y\) to \(z\). We
define a map \(\phi: I \to X\) given by
\[
  \phi(t) \coloneq
  \begin{cases}
    \lambda(2t), &\text{ for } t \in [0, 1/2] \\
    \eta(2t - 1), &\text{ for } t \in [1/2, 1]
  \end{cases}
\]
which is surely continuous and connects both \(x\) and \(z\) --- thus \(x \sim
z\).
\end{proof}

\begin{definition}[Path connected components]
\label{def:path-connected-components}
Let \(X\) be a path connected topological space, and \(\sim\) be the equivalence
relation described in \cref{prop:path-connected-equiv-relation}. The collection
\(X/{\sim}\) is called the \emph{path connected components} of the space
\(X\). We denote the collection of all path components of \(X\) by \(\pi_0(X)\)
--- the collection of homotopy classes between maps \(* \to X\).
\end{definition}

\begin{definition}[\(\pi_0\) functor]
\label{def:pi0-functor}
The concept of connected components of a space induce a covariant functor
\(\pi_0: \Top \to \Set\) defined by mapping objects \(X \mapsto \pi_0 X\), and
morphisms \(f: X \to Y\) to \(\pi_0 f: \pi_0 X \to \pi_0 Y\) --- which is a well
defined map since, given a path component \(P\) of \(X\), the set \(f(X)
\subseteq Y\) is connected and therefore contained in a unique path component of
\(Y\).
\end{definition}

\begin{theorem}[Morphisms preserve connectivity]
\label{thm:morphisms-preserve-connectivity}
Let \(X\) be a (path) connected space and \(f: X \to Y\) be a topological
morphism. Then \(f(X) \subseteq Y\) is (path) connected.
\end{theorem}

\begin{proof}
Suppose that \(f(X)\) is not connected, and let \(g: f(X) \to \{0, 1\}\) be a
non-constant morphism --- in particular, this is equivalent to the condition of
\(gf: X \to \{0, 1\}\) being a non-constant morphism, thus implying in the
non-connectedness of \(X\).

On the other hand, assume now that \(X\) is path connected and let \(u, v \in
f(X)\) --- define \(x, y \in X\) so that \(f(x) = u\) and \(f(y) = v\). Let
\(\gamma: I \to X\) be a path connecting \(x\) and \(y\) --- then \(f\gamma: I
\to f(X)\) is a path connecting \(u\) and \(v\), which proves the proposition.
\end{proof}

\begin{corollary}
\label{cor:connectedness-top-property}
Connectedness and path connectedness are both topological properties.
\end{corollary}

\begin{corollary}
\label{cor:quotient-path-connected-is-path-connected}
The quotient of a (path) connected topological space is (path) connected.
\end{corollary}

\begin{proposition}
\label{prop:connectivity-quotients}
Let \(f: X \epi Y\) be a morphism of topological spaces \(X\) and \(Y\). If
\(Y\) is connected and, for all \(y \in Y\), the fiber \(f^{-1}(y)\) is
connected, then \(X\) is connected.
\end{proposition}

\begin{proof}
Let \(g: X \to \{0, 1\}\) be a morphism. From the connectedness condition on the
fibers of \(f\), it follows that \(g\) must be constant throughout the fibers of
\(f\) --- this implies in the existence of a morphism \(g^{*}: Y \to \{0, 1\}\)
such that \(g = g^{*} f\). Since \(Y\) is connected, \(g^{*}\) must be constant,
therefore the composition \(g^{*} f\) is constant and so is \(g\) --- that is,
\(X\) is connected.
\end{proof}

\begin{proposition}
\label{prop:union-path-connected}
Let \(J\) be an indexing set associated to a collection \(\{X_{j}\}_{j \in J}\)
of (path) connected topological spaces. Define the space \(X \coloneq \bigcup_{j
\in J} X_j\). If the intersection \(\bigcap_{j \in J} X_j\) is non-empty, then
\(X\) is (path) connected.
\end{proposition}

\begin{proof}
Suppose the intersection is non empty and let \(x \in \bigcup_{j \in J}
X_j\) be any point. We split the proposition into the two given cases.

Suppose \(X_j\) is connected for all \(j \in J\). Let \(f: X \to \{0, 1\}\) be a
morphism --- we want to show that it has to be constant. Since \(X_j\) is
connected, it follows tha the restriction \(f|_{X_j}\) has to be constant, and,
since \(x \in X_j\), then \(f(x) = f(y)\) for all \(y \in X_j\). The fact that
this must be true for all \(X_j\) shows that \(f\) must be constant throughout
the whole set \(X\).

Suppose \(X_j\) is path connected for all \(j \in J\) --- that is, from
hypothesis, \(X_j\) is a path connected component of \(X\). We now prove that
the intersection is connected to each of the path connected components. For each
\(j \in J\), choose any point \(y \in X_j\). Since \(x \in X_j\), then \(x \sim
y\) and therefore every point of \(X\) is connected by a path.
\end{proof}

\todo[inline]{Continue on connected spaces}

\section{Compact Spaces}

\begin{definition}[Compact space]
\label{def:compact-space}
A topological space \(X\) is said to be \emph{compact} if for \emph{every} open
cover there exists a \emph{finite subcover}.
\end{definition}

\begin{definition}[Sequentially compact space]
\label{def:sequentially-compact}
A topological space \(X\) is said to be \emph{sequentially compact} if every
sequence of points \((x_j)_j\) in \(X\), there exists a subsequence \((x_j')_j
\subseteq (x_j)_j\) that converges in \(X\).
\end{definition}

\begin{proposition}[Metric space compactness]
\label{prop:metric-space-compactness-equivalences}
Compactness, limit point compactness, and sequential compactness are
\emph{equivalent} in \emph{metric spaces}.
\end{proposition}

\todo[inline]{Prove metric compactness and expand section}

\begin{proposition}
\label{prop:hausdorff-compact-implies-closed}
Let \(X\) be a Hausdorff space. If \(A \subseteq X\) is compact then \(A\) is
closed in \(X\).
\end{proposition}

\begin{proof}
Let \(x_0 \in A'\) be any limit point of \(A\). We suppose, for the sake of
contradiction, that \(x_0 \notin A\). Since \(X\) is Hausdorff, for every \(x
\in A\) we let \(U_x\) be a neighbourhood of \(x\) such that there exists a
neighbourhood of \(x_0\) that is disjoint from \(U_x\). Define now the
collection \(\mathcal{U} \coloneq \{U_x : x \in A\}\) --- thus clearly
\(\mathcal{U}\) covers \(A\) since it contains every point of \(A\), this being
possible because we assumed that \(x_0\) does not belong to \(A\). Since \(A\)
is compact, we let \(\mathcal{C} \coloneq \{U_{x_1}, \dots, U_{x_n}\} \subseteq
\mathcal{U}\) be a finite subcover. Define \(\mathcal{O} \coloneq \{O_1, \dots,
O_n\}\) of neighbourhoods of \(x_0\) such that \(U_{x_j} \cap O_j = \emptyset\)
for all \(1 \leq j \leq n\) --- thus \(U_{x_j} \cap \bigcap_{j=1}^n O_j =
\emptyset\), and therefore \(A \cap \bigcap_{j=1}^n O_j = \emptyset\) which by
\cref{def:limit-point-derived-set} implies that \(x_0\) cannot be a limit point
of \(A\), thus we arrived at a contradiction. Therefore, for \(x_0\) to be a
limit point of \(A\), it must be the case that \(x_0 \in A\), that is, \(A\) is
closed.
\end{proof}

\begin{corollary}
\label{cor:metric-space-compact-implies-bounded-closed}
In any metric space \(X\), if \(A \subseteq X\) is \emph{compact} then \(A\) is
both \emph{bounded} and \emph{closed} in \(X\).
\end{corollary}

\begin{proof}
Since every metric space is Hausdorff, \(A\) is closed by
\cref{prop:hausdorff-compact-implies-closed}. Moreover, if \(x \in A\) is any
point, we find that the collection of open balls \(\mathcal{B}_x \coloneq
\{B_x(n)\}_{n \in \N}\) forms an open cover of \(A\) and, since \(A\) is
compact, there exists a finite subcover of \(\mathcal{B}_x\). Therefore there
exists a maximal \(m \in \N\) for which \(A \subseteq B_x(m)\) --- thus \(A\) is
bounded.
\end{proof}

\begin{remark}
\label{rem:closed-and-bounded-not-compact}
Notice that a bounded and closed set in a metric space does \emph{not} need to
be compact, for instance, consider the space \(\ell^2(\N)\) (see
\cref{exp:p-norms}) and the subset \(A \coloneq \{f_n\}_{n \in \N}\) composed of
sequences \(f_n \in \ell^2(\N)\) such that \(f_n(j) \coloneq \delta_{n j}\) ---
that is, sequences where the only nonzero term is the \(n\)-th one, which equals
\(1\). Clearly \(A\) is bounded since \(\| f_{n} \|_2 = 1\), moreover, \(A\) is
closed because it has no limit points. However, since no subset of \(A\)
contains a limit point of \(A\), we find that \(A\) is not limit point compact
--- thus not compact (see \cref{prop:metric-space-compactness-equivalences})
\end{remark}

\begin{proposition}
\label{prop:convergent-subsequence-implies-compact}
Let \(X\) be a metric space and \(A \subseteq X\) be a subset. Then, \(A\) is
compact if and only if every sequence of points in \(A\) has a convergent
subsequence in \(A\).
\end{proposition}

\begin{proof}
\todo[inline]{write proof}
\end{proof}



%%% Local Variables:
%%% mode: latex
%%% TeX-master: "../../../deep-dive"
%%% End:

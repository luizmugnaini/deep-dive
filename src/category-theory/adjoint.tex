\section{Reflections}

\begin{definition}[Reflection along a functor]
\label{def:reflection-functor}
Let \(F: \cat C \to \cat D\) be any functor and \(D \in \cat D\). We define a
\emph{reflection of \(D\) along \(F\)} to be a pair \((R_D, \eta_D)\) where
\(R_D \in \cat C\) is an object, and \(\eta_D: D \to F R_D\) is a morphism of
\(\cat D\) such that: if \(C \in \cat C\) is any object and
\(\delta: D \to F C\) is a morphism of \(\cat D\), then there exists a
\emph{unique morphism} \(\varepsilon: R_D \to C\) in \(\cat C\) such that the
diagram
\[
\begin{tikzcd}
D \ar[rd, "\eta_D"'] \ar[rr, "\delta"] & &F C \\
&F R_D \ar[ru, dashed, "F \varepsilon"'] &
\end{tikzcd}
\]
commutes in the category \(\cat D\).
\end{definition}

\begin{proposition}[Uniqueness of reflections]
\label{prop:reflection-unique-up-to-iso}
Let \(F: \cat C \to \cat D\) be a functor and \(D \in \cat D\) be an object. If
the reflection of \(D\) along \(F\) exists, then it is \emph{unique up to
  isomorphism}.
\end{proposition}

\begin{proof}
Suppose the reflection of \(D\) along \(F\) exists and let \((R_D, \eta_D)\) and
\((R_D', \eta_D')\) be two such reflections. From definition:
\begin{itemize}\setlength\itemsep{0em}
\item Choose the object \(R_D \in \cat C\) and the morphism
  \(\eta_D: D \to F R_D\) in \(\cat D\). Since \((R_D', \eta_D')\) is a reflection,
  there exists a unique \(\varepsilon': R_D' \to R_D\) such that \(F(\varepsilon') \eta_D' = \eta_D\).

\item Analogously, choose the object \(R_D' \in \cat C\) and the morphism
  \(\eta_D': D \to F R_D'\) in \(\cat D\). Since \((R_D, \eta_D)\) is a reflection,
  there exists a unique \(\varepsilon: R_D \to R_D'\) such that \(F(\varepsilon) \eta_D = \eta_D'\).
\end{itemize}
With \(\varepsilon\) and \(\varepsilon'\) in hands, notice that
\[
F(\varepsilon \varepsilon') \eta_D' = F(\varepsilon) F(\varepsilon') \eta_D' = F(\varepsilon) \eta_D = \eta_D'
= \Id_{F R_D'} \eta_D' = F(\Id_{R_D'}) \eta_D'
\]
and since the composition \(\varepsilon \varepsilon'\) is unique satisfying such equation, it must
be the case that \(\varepsilon \varepsilon' = \Id_{R_D'}\). Analogously we obtain that
\(\varepsilon' \varepsilon = \Id_{R_D}\), showing that \(R_D \iso R_D'\).
\end{proof}

\begin{proposition}
\label{prop:reflection-as-func-and-unit-as-nat-tranf}
Let \(F: \cat C \to \cat D\) be a functor and assume that for any \(D \in \cat D\)
the reflection of \(D\) along \(F\) exists and is \((R_D, \eta_D)\). Then there
exists a \emph{unique} functor \(R: \cat D \to \cat C\) such that the following
two properties are satisfied:
\begin{enumerate}[(a)]\setlength\itemsep{0em}
\item For any \(D \in \cat D\) we have \(R D = R_D\).

\item The map \(\eta: \Id_{\cat D} \nat F R\) given by
  \((\eta_D: D \to F R D)_{D \in \cat D}\) is a natural transformation.
\end{enumerate}
\end{proposition}

\begin{proof}
Let \(R: \cat D \to \cat C\) be defined as above---we want to prove that \(R\) is
indeed a functor. Lets define how \(R\) acts on \(\Mor(\cat D)\). Consider a
morphism \(\delta: D \to D'\) in \(\cat D\). The reflection pair
\((R_D, \eta_D)\) says that there exists a unique morphism
\(\varepsilon: R_D \to R_{D'}\) such that the following diagram commutes
\[
\begin{tikzcd}
D \ar[r, "\delta"] \ar[rd, "\eta_D"']
&D' \ar[r, "\eta_{D'}"]
&F R_{D'}
\\
& F R_D \ar[ru, "F \varepsilon"']
&
\end{tikzcd}
\]
We shall define \(R \delta \coloneq \varepsilon\). If \(\delta': D' \to D''\) is yet another morphism in
\(\cat D\), then from construction one has
\begin{align*}
  F(R(\delta') R(\delta)) \eta_D
  = F R(\delta') F R(\delta) \eta_D
  = F(\varepsilon') F(\varepsilon) \eta_D
  = F(\varepsilon') \eta_{D'} \delta
  = \eta_{D''} \delta' \delta.
\end{align*}
On the other hand we have
\[
F R(\delta' \delta) \eta_D = \eta_{D''} (\delta' \delta),
\]
therefore, from uniqueness we obtain \(R(\delta') R(\delta) = R(\delta' \delta)\).
\end{proof}

\begin{definition}[Left adjoint]
\label{def:left-adjoint}
A functor \(R: \cat D \to \cat C\) is a \emph{left adjoint} of a functor
\(F: \cat C \to \cat D\) if there exists a natural transformation
\(\eta: \Id_{\cat D} \nat F R\) for which \((R D, \eta_D)\) is a reflection of
\(D\) along \(F\) for any \(D \in \cat D\).
\end{definition}

\begin{definition}[Coreflection]
\label{def:coreflection-along-functor}
Let \(F: \cat C \to \cat D\) be a functor and \(D \in \cat D\) be any object. A pair
\((R_D, \nu_D)\) is said to be a \emph{coreflection} of \(D\) along \(F\) if \(R_D
\in \cat C\) and \(\nu_D: F R_D \to D\) is a morphism of \(\cat D\) such that: for any
\(C \in \cat C\) and morphism \(\delta: F C \to D\) in \(\cat D\) there exists a
\emph{unique} morphism \(\varepsilon: C \to R_D\) for which the triangle
\[
\begin{tikzcd}
F C \ar[rr, "\delta"] \ar[rd, "F \varepsilon"']
& &D
\\
&F R_D
\ar[ru, "\nu_D"']
&
\end{tikzcd}
\]
commutes in \(\cat D\).
\end{definition}

\begin{definition}[Right adjoint]
\label{def:right-adjoint}
A functor \(R: \cat D \to \cat C\) is a \emph{right adjoint} of a functor
\(F: \cat C \to \cat D\) if there exists a natural transformation
\(\nu: F R \nat \Id_{\cat D}\) such that: for each \(D \in \cat D\) the pair
\((R D, \nu_D)\) is a coreflection of \(D\) along \(F\).
\end{definition}

\begin{theorem}[Equivalent definitions]
\label{thm:equivalent-defs-adjoint-pair}
Let \(F \colon \cat C \adj \cat D \colon G\) be two functors. The following
properties are equivalent:
\begin{enumerate}[(a)]\setlength\itemsep{0em}
\item The functor \(G\) is left adjoint to \(F\).
\item There exists a pair of natural transformations
  \(\eta: \Id_{\cat D} \nat F G\) and \(\nu: G F \nat \Id_{\cat C}\) for which
  \[
  (F \nu) (\eta F) = \Id_F,\,
  \text{ and }\,
  (\nu G) (G \eta) = \Id_G.
  \]

\item For every pair of objects \(C \in \cat C\) and \(D \in \cat D\) there exists a
  natural bijection
  \[
  \theta_{C D}: \Mor_{\cat C}(G D, C) \isoto \Mor_{\cat D}(D, F C).
  \]
  The naturality of \(\theta\) over the choice of \(C\) and \(D\) can be expressed as
  follows: for any two other objects \(C' \in \cat C\) and \(D' \in \cat D\)
  together with morphisms \(c \in \Mor_{\cat C}(C', C)\) and \(d \in \Mor_{\cat
    D}(D', D)\) the following two squares commute
  \[
  \begin{tikzcd}
  \Mor_{\cat C}(G D, C) \ar[r, "c_{*}"] \ar[d, "\theta_{C D}"']
  &\Mor_{\cat C}(G D, C') \ar[d, "\theta_{C' D}"]
  \\
  \Mor_{\cat D}(D, F C) \ar[r, "{(F c)_{*}}"']
  &\Mor_{\cat D}(D, F C')
  \end{tikzcd}
  \qquad
  \qquad
  \begin{tikzcd}
  \Mor_{\cat C}(G D, C) \ar[r, "(G d)^{*}"]
  \ar[d, "\theta_{C D}"']
  &\Mor_{\cat C}(G D', C) \ar[d, "\theta_{C D'}"]
  \\
  \Mor_{\cat D}(D, F C) \ar[r, "{d^{*}}"']
  &\Mor_{\cat D}(D', F C)
  \end{tikzcd}
  \]

\item The functor \(F\) is right adjoint to \(G\).
\end{enumerate}
\end{theorem}

\begin{notation}
\label{not:adjoint-functor}
Let \(F \colon \cat C \adj \cat D \colon G\) be functors forming a pair
\((G, F)\) of adjoint functors. We shall denote that \(G\) is a left adjoint of
\(F\) (or, which is equivalent, \(F\) is right adjoint of \(G\)) by
\(G \Adj F\).
\end{notation}

% \section{Adjoint Pairs}

% \begin{definition}
% \label{def:adjoint-pair-functors}
% Let \(F: \cat C \to \cat D\) and \(G: \cat D \to \cat C\) be covariant functors. We
% say that \((F, G)\) is an \emph{adjoint pair} of functors if for any two \(C \in
% \cat C\) and \(D \in \cat D\) there exists a natural bijection in \(\Set\):
% \[
% \tau_{(C, D)}: \Mor_{\cat D}(F C, D) \isoto \Mor_{\cat C}(C, G D).
% \]
% In other words, if \(C' \in \cat C\) and \(D' \in \cat D\) are two other objects,
% and we consider morphisms \(f \in \Mor_{\cat C}(C', C)\) and \(g \in \Mor_{\cat
%   D}(D', D)\), then the following two squares commute:
% \[
% \begin{tikzcd}
% \Mor_{\cat D}(F C, D)
% \ar[r, "(F f)^{*}"]
% \ar[d, "\tau_{(C, D)}"', "\dis"]
% &\Mor_{\cat D}(F C', D)
% \ar[d, "\tau_{(C', D)}", "\dis"']
% \\
% \Mor_{\cat C}(C, G D)
% \ar[r, "f^{*}"']
% &\Mor_{\cat C}(C', G D)
% \end{tikzcd}
% \qquad
% \qquad
% \begin{tikzcd}
% \Mor_{\cat D}(F C, D)
% \ar[r, "g_{*}"]
% \ar[d, "\tau_{(C, D)}"', "\dis"]
% &\Mor_{\cat D}(F C, D')
% \ar[d, "\tau_{(C, D')}", "\dis"']
% \\
% \Mor_{\cat C}(C, G D)
% \ar[r, "{G g}_{*}"']
% &\Mor_{\cat C}(C, G D')
% \end{tikzcd}
% \]
% \end{definition}

% \begin{remark}
% \label{rem:adjoint-ordered-pair}
% If \((F, G)\) is an adjoint pair of functors, \emph{it does not follow} that
% \((G, F)\) is an adjoint pair---that is, the order of the functors matter.
% \end{remark}

% \begin{definition}
% \label{def:right-left-adjoint-functors}
% Let \(F \colon \cat C \adj \cat D \colon G\) be functors with \((F, G)\) being an
% adjoint pair. We say that \(F\) is a \emph{right adjoint} of \(G\), and that
% \(G\) is a \emph{left adjoint} of \(F\).
% \end{definition}

% \begin{definition}[Unit]
% \label{def:adjoint-pair-unit}
% Given an adjoint pair of functors \(F \colon \cat C \adj \cat D \colon G\), we
% define the \emph{unit} of the pair \((F, G)\) to be the natural transformation
% \[
% u: \Id_{\cat C} \longrightarrow G F
% \]
% where for each \(C \in \cat C\) we define \(u_C: C \to G F C\) in \(\cat C\) to be
% the map \(u_C \coloneq \tau_{(C, F C)}(\Id_{F C})\), where \(\tau_{(C, F C)}\) is the natural
% bijection \(\End_{\cat D}(F C) \iso \Mor_{\cat C}(C, G F C)\) yielded by the
% adjoint pair.
% \end{definition}

%%% Local Variables:
%%% mode: latex
%%% TeX-master: "../../deep-dive"
%%% End:

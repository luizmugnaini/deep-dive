\subsection{\texorpdfstring{\(\sum_p 1/p\)}{1/p} diverges}

\begin{theorem}
    The sum over the inverse of all positive primes on \(\mathbb{Z}\), \(\sum_{p
        \in \mathbb{Z}_{>0}}\frac{1}{p}\), diverges.
\end{theorem}

\begin{proof}
    Let \(p_{\ell(n)}\) be the greatest prime \(p_{\ell(n)} \leqslant n\). We
    define a function
    \[
        \lambda(n) := \prod_{k = 1}^{\ell(n)} \left( 1 - \frac{1}{p_k}\right)^{-1}.
    \]
    Now, notice that we can also rewrite all terms as the infinite sum of the
    geometric series of ratio \(1/p_k\):
    \[
        1 - \frac{1}{p_k} = \sum_{a_k = 0}^{\infty}\frac{1}{p_k^{a_k}} =
        \frac{1}{1 - \frac{1}{p_k}}.
    \]
    Thus we can substitute for the \(\lambda\) function
    \begin{align*}
        \lambda(n)
         & = \prod_{k=1}^{\ell(n)} \sum_{a_k = 0}^\infty \frac{1}{p_k^{a_k}}
        \\
         & = \left( 1 + \frac{1}{p_1} + \frac{1}{p_1^2} + \dots \right)  \dots
        \left( 1 + \frac{1}{p_j} + \frac{1}{p_j^2} + \dots \right) \dots \left( 1
        + \frac{1}{p_{\ell(n)}} + \frac{1}{p_{\ell(n)}^2} + \dots \right)
        \\
         & = \sum_{(a_k)_{k=1}^{\ell(n)}} \prod_{k = 1}^{\ell(n)} \frac{1}{p_k^{a_k}}
    \end{align*}
    where \(\left( a_k \right)_{k = 1}^{\ell(n)}\), with \(a_k \in
    \mathbb{Z}_{\geqslant 0}\), are all the possible tuples of coefficients for
    \(\left( p_k \right)_{k = 1}^{\ell(n)}\). It is clear that \(\sum_{k=1}^n
    k^{-1} < \lambda(n)\) and also we know that \(\lim_{n \to \infty}
    \sum_{k=1}^n k^{-1} = 0\) diverges, thus \(\lim_{n \to \infty}
    \lambda(n) = \infty\) diverges. This result will be important for the
    last step of the proof, bear it in mind.

    Now we consider the \(\log(\lambda(n))\) in order to find some way of extract
    the sum \(\sum_p 1/p\) from the function. It goes as follows:
    \[
        \log(\lambda(n))
        = \log \left( \prod_{k=1}^{\ell(n)} \left(1 - \frac{1}{p_k}\right)^{-1} \right)
        = - \sum_{k=1}^{\ell(n)} \log \left( 1 - \frac{1}{p_k} \right)
    \]
    If now we take a Taylor series on \(\log(1 - p_k^{-1})\) we get
    \(\log(1-p_k^{-1}) = -\sum_{m=1}^\infty \frac{1}{m p_k^m}\) thus, substituting
    in the equation above we find out that
    \[
        \log(\lambda(n)) = \sum_{k=1}^{\ell(n)} \sum_{m = 1}^\infty \frac{1}{mp_k^m}
        = \sum_{k = 1}^{\ell(n)} \frac{1}{p_k} + \sum_{m = 1}^{\ell(n)}
        \sum_{m=2}^\infty \frac{1}{mp_k^{m}}
    \]
    Let's analyse this second term of the last equation, in fact, we have
    \begin{gather*}
        \sum_{m=2}^\infty \frac{1}{mp_k^{m}} < \sum_{m=2}^\infty \frac{1}{p_k^m} =
        p_k^{-2} \underbrace{\frac{p_k}{p_k - 1}}_{\in (1, 2]} \leqslant 2 p_k^{-2}
        \\
        \text{Thus } \log(\lambda(n)) < \sum_{k=1}^{\ell(n)} p_k^{-1} + 2
        \sum_{k=1}^{\ell(n)} p_k^{-2}
    \end{gather*}
    If we accept the fact that \(\sum_{k = 1}^\infty n^{-2}\) converges
    (actually, it converges to \(\pi^2/6\) but we aren't going to prove it right
    now), it's easy to accept that \(\sum_{k = 1}^\infty p_k^{-2}\) shows the
    same behaviour and indeed converges just as the previous sum. Now, if the
    first term,  \(\sum_{k=1}^\infty p_k^{-1}\) also converged, then there would
    exist \(N \in \mathbb{Z}\) such that \(\lim_{n\to\infty}\log((\lambda(n))) <
    N\) and thus \(\lambda(n) < e^N\). Notice now that this is a false claim
    since we have shown that \(\lim_{n \to \infty} \lambda(n) = \infty\)
    diverges. Thus, it is necessary for the first term to diverge, then
    \[
        \lim_{n \to \infty} \sum_{k=1}^{\ell(n)} \frac{1}{p_k} = \infty.
    \]
\end{proof}

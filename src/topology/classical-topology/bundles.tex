\documentclass[../../../deep-dive]{subfile}

\begin{document}

\section{Bundle}

\subsection{Initial Construction}

\begin{definition}[Bundle]
\label{def:bundle}
A \emph{bundle} is defined to be a triple \((E, p, B)\), where \(E\) and \(B\)
are \emph{spaces} and \(p: E \to B\) is a \emph{morphism}. We refer to \(B\) as
the \emph{base space}, while \(E\) is the \emph{total space}, and \(p\) is the
\emph{projection} of the bundle. As usual, given any \(b \in B\), we name the
object \(p^{-1} b\) the \emph{fibre} of the bundle over \(b\).

A \emph{subbundle} of \((E, p, B)\) is a bundle \((E', p', B')\) such that
\(E'\) and \(B'\) are \emph{subspaces} of \(E\) and \(B\), respectively, and
\(p' = p|_{E'}: E' \to B'\).
\end{definition}

\begin{definition}[Product bundle]
\label{def:product-bundle}
A \emph{product bundle over \(B\) with fibre \(F\)} is a triple
\((B \times F, p, B)\) where \(p(x, y) \coloneq x\) is the first projection.
\end{definition}

\begin{definition}[Cross section]
\label{def:cross-section-bundle}
Given a bundle \((E, p, B)\), a \emph{cross section} of the bundle is a
\emph{section} \(s: B \to E\) of \(p\)---that is, \(p s = \Id_B\). As
an immediate consequence of this definition, if \((E', p', B')\) is a subbundle
of \((E, p, B)\), then \(s\) is a cross section of \((E', p', B')\) if and only
if \(s B \subseteq E'\).
\end{definition}

\begin{lemma}[Product bundle cross section]
\label{lem:cross-section-of-product-bundle}
Given a product bundle \((B \times F, p, B)\), a cross section
\(s: B \to B \times F\) will always have the form \(s = \Id_B \times f\), where
\(f: B \to F\) is a uniquely defined morphism. Therefore the collection of cross
sections of product bundles is in bijection with the collection of maps
\(B \to F\).
\end{lemma}

\begin{proof}
Let \(s\) be any cross section of \((B \times F, p, B)\), from the definition of
\(s\), there exists unique morphisms \(s': B \to B\) and \(f: F \to B\) such
that \(s = s' \times f\)---it remains to be shown that \(s'\) is the identity on
\(B\). From definition of a product bundle, we know that \(p s = s'\)---since
\(p\) is the projection of the first factor---moreover, from the definition of a
cross section, \(p s = \Id_B\), therefore \(s' = \Id_B\) as wanted.
\end{proof}

\begin{definition}[Stiefel variety]
\label{def:stiefel-variety}
We define the \emph{Stiefel variety of orthonormal \(k\)-frames\footnote{ A
    \(k\)-frame in an \(n\)-dimensional vector space is an ordered collection of
    \(k\) linearly independent vectors.  }} in \(\R^n\) to be the \emph{compact
  subspace} \(\Stie_k \R^n \subseteq (S^{n-1})^k\) for which
\((v_1, \dots, v_k) \in \Stie_k \R^n\) if and only if
\(\langle v_i, v_j \rangle = \delta_{ij}\)---where \(\langle -, - \rangle\) is
the standard euclidean inner product.
\end{definition}

\begin{definition}[Grassmann variety]
\label{def:grassmann-variety}
The real \emph{\(k\)-Grassmann variety} is defined to be the topological space
\(\Grass_k \R^n\) whose points are \(k\)-dimensional subspaces of \(\R^n\), and
endowed with the quotient topology generated by the map
\(\Stie_k \R^n \epi \Grass_k \R^n\) given by
\((v_1, \dots, v_k) \mapsto \langle v_1, \dots, v_k \rangle\). Since
\(\Stie_k \R^n\) is compact, it follows that \(\Grass_k \R^n\) is also compact.
\end{definition}

\begin{example}
\label{exp:grassmanian-projective-space}
Notice that \(\Stie_1 \R^n = S^{n-1}\) and by the construction of the
Grassmannian variety, we see that \(\Grass_1 \R^n = \R \Proj^{n-1}\).
\end{example}

\begin{definition}[Bundle morphism]
\label{def:bundle-morphism}
Given two bundles \((E, p, B)\) and \((E', p', B')\), a \emph{morphism of
  bundles} \((E, p, B) \to (E', p', B')\) is a pair \((u, f)\) of morphisms
\(u: E \to E'\) and \(f: B \to B'\) such that the diagram
\[
\begin{tikzcd}
E \ar[r, "u"] \ar[d, "p"'] &E' \ar[d, "p'"] \\
B \ar[r, "f"'] &B'
\end{tikzcd}
\]
commutes in \(\Top\)---equivalently, \(u (p^{-1} B) \subseteq p'^{-1}(f
B)\). The special case where the base spaces coincide, we define a \emph{bundle
  morphism over \(B\)} (also refered to as \(B\)-morphism)
\((E, p, B) \to (E', p', B)\) to be a morphism \(u: E \to E'\) such that the
triangle
\[
\begin{tikzcd}
E \ar[rr, "u"] \ar[dr, "p"'] & &E' \ar[ld, "p'"] \\
&B &
\end{tikzcd}
\]
commutes in \(\Top\), which can equivalently be expressed as the condition
\(u (p^{-1} B) \subseteq p'^{-1} B\).

Given any two bundle morphisms \((u, f): (E, p, B) \to (E', p', B')\) and \((v,
g): (E', p', B') \to (E'', p'', B'')\), we define the \emph{composition} of
those morphisms to be the pair
\[
(v, g) \circ (u, f) \coloneq (v u, g f):
(E, p, B) \longrightarrow (E'', p'', B''),
\]
which is again a bundle morphism, since
\[
\begin{tikzcd}
E \ar[r, "u"] \ar[d, "p"']
&E' \ar[d, "p'"] \ar[r, "v"]
&E'' \ar[d, "p''"]
\\
B \ar[r, "f"']
&B' \ar[r, "g"']
&B''
\end{tikzcd}
\]
is a commutative diagram in \(\Top\).
\end{definition}

\begin{example}[Cross section as bundle morphism]
\label{exp:cross-section-is-bundle-morphism}
Notice that a cross section is nothing more than a bundle morphism over \(B\) of
the form \(s: (B, \Id_B, B) \to (E, p, B)\).
\end{example}

\begin{definition}[Category of bundles]
\label{def:category-of-bundles}
We denote by \(\Bun\) the category composed of bundles and bundle
morphisms. Given a space \(B\), we can also define a full subcategory \(\Bun_B\)
of \(\Bun\), whose objects are bundles with base space \(B\) and bundle
morphisms over \(B\).
\end{definition}

\begin{definition}[Fibre of a bundle]
\label{def:fibre-of-a-bundle}
We say that a space \(F\) is \emph{the fibre} of a bundle \((E, p, B)\) if there
exists a topological isomorphism \(p^{-1} b \iso F\) for every \(b \in B\). A
bundle \((E, p, B)\) is said to be \emph{trivial with fibre \(F\)} if there
exists a bundle \(B\)-isomorphism \((E, p, B) \iso (B \times F, p, B)\).
\end{definition}

\subsection{Universal Properties}

\begin{proposition}[Products in \(\Bun\)]
\label{prop:bundle-products}
Given a family of bundles \((E_j, p_j, B_j)_{j \in J}\), we define the
\emph{product} of this family of bundles to be the bundle
\[
\Big( \prod_{j \in J} E_j, \prod_{j \in J} p_j, \prod_{j \in J} B_j \Big),
\]
which is defines a product in the category \(\Bun\).
\end{proposition}

\begin{proposition}[Pullbacks in \(\Bun_B\)]
\label{prop:fibre-product-over-B}
Given two bundles \(\xi = (E, p, B)\) and \(\xi' = (E', p', B)\), define
\[
E \oplus E' \coloneq \{(x, x') \in E \times E' \colon p x = p' x'\},
\]
and \(q: E \oplus E' \to B\) to be the morphism
\(q(x, x') \coloneq p x = p' x'\). Then the triple
\[
\xi \oplus \xi' \coloneq (E \oplus E', q, B),
\]
called \emph{fibre product over \(B\)} of \(\xi\) and \(\xi'\), is the
\emph{pullback} of the pair \((\xi, \xi')\) in the category \(\Bun_B\).
\end{proposition}

\subsection{Induced Bundle}

\begin{definition}[Induced bundle]
\label{def:induced-bundle}
Let \(\xi = (E, p, B)\) be a bundle, and \(f: B_0 \to B\) be a continuous
map. There exists an \emph{induced bundle} of \(\xi\) under \(f\), denoted
\(f^{*} \xi\), with base space \(B_0\) and total space \(E_0\) defined as the
pullback:
\[
\begin{tikzcd}
E_0
\ar[rd, phantom, very near start, "\lrcorner"]
\ar[r, "f_{\xi}"]
\ar[d, "p_0"']
&E
\ar[d, "p"]
\\
B_0 \ar[r, "f"']
&B
\end{tikzcd}
\]
Explicitly, \(E_0\) consists of pairs \((b_0, x) \in B_0 \times E\) such that
\(f b_0 = p x\). The projection of the bundle \(f^{*} \xi\) is the map
\(p_0: E_0 \to B_0\) given by \((b_0, x) \mapsto b_0\).

The mapping \(f_{\xi}: E_0 \to E\) given by \((b_0, x) \mapsto x\) induces a
morphism of bundles \((f_{\xi}, f): f^{*} \xi \to \xi\), the so called
\emph{canonical morphism of an induced bundle}.
\end{definition}

\begin{proposition}
\label{prop:induced-bundle-associated-isomorphisms}
Let \(\xi = (E, p, B)\) be a bundle and \(f: B_0 \to B\) be a continuous
map. Considering the canonical morphism \((f_{\xi}, f): f^{*} \xi \to \xi\),
for each \(b_0 \in B_0\) the restricted map
\[
f_{\xi}: p_0^{-1} b_0 \isoto p^{-1}(f b_0)
\]
is a \emph{topological isomorphism}. Furthermore, given a bundle
\(\eta = (E', p', B_0)\) and a morphism \((v, f): \eta \to \xi\), there
\emph{exists a unique \(B_0\)-morphism} \(w: \eta \to f^{*} \xi\) such that
\(f_{\xi} w = v\). In other words, the following diagram
\[
\begin{tikzcd}
E' \ar[r, "v"] \ar[rd, dashed, "w"'] &E \\
&E_0 \ar[u, "f_{\xi}"']
\end{tikzcd}
\]
commutes in \(\Top\).
\end{proposition}

\begin{proof}
For the first part of the proposition, recalling the definition one has that the
fibre \(p_0^{-1} b_0\) is composed of pairs \((b_0, x) \in b_0 \times E\) such
that \(p x = f b_0\)---that is, \(p_0^{-1} b_0 = b_0 \times p^{-1} (f
b_0)\). Since \(f_{\xi}(b_0, x) = x\), then its restriction is a topological
isomorphism with local inverse \(x \mapsto (b_0, x)\).

To prove the second part, define a map \(w: E' \to E_0\) by
\(w \coloneq (p', v)\), therefore
\[
\begin{tikzcd}
E' \ar[rr, "w"] \ar[rd, "p'"'] & &E_0 \ar[dl, "p_0"]  \\
&B_0 &
\end{tikzcd}
\]
commutes, showing that \(w\) is a \(B_0\)-morphism. Moreover, we have for any
\(y \in E'\) that
\[
f_{\xi} w y = f_{\xi}(p' y, v y) = v y,
\]
therefore \(f_{\xi} w = v\) as wanted. For the uniqueness of \(w\), suppose
\(\ell: E' \to E_0\) satisfies both \(p_0 \ell = p'\) and \(f_{\xi} \ell = v\),
then since \(\ell = (p_0 \ell, f_{\xi} \ell) = (p', v)\), this shows that
\(\ell = w\).
\end{proof}

\begin{proposition}
\label{prop:induced-bundle-is-functorial}
Given any continuous map \(f: B_0 \to B\), the induced bundle construction via
\(f\) is a functor
\[
f^{*}: \Bun_B \to \Bun_{B_0}.
\]
Moreover, given any morphism \(u: \xi \to \eta\) in \(\Bun_B\), the diagram
\[
\begin{tikzcd}
&E(f^{*} \eta)
\ar[dd]
\ar[rr, "f_{\eta}"]
&
& E \eta \ar[dd]
\\
E (f^{*} \xi)
\ar[ur, "f^{*} u"]
\ar[rd]
\ar[rr, "f_{\xi}" description, crossing over, near end]
&
&E \xi \ar[ur, "u"] \ar[rd]
&
\\
&B_0 \ar[rr, "f"']
&
&B
\end{tikzcd}
\]
\end{proposition}

\begin{proof}
For the functoriallity, given any \(B\)-bundle morphism \(u: \xi \to \eta\), the
associated map \(f^{*} u: f^{*} \xi \to f^{*} \eta\) is canonically given by the
mapping \((b_0, x) \mapsto (b_0, u x)\), which is a \(B_0\)-morphism of
bundles. Moreover, if we consider the identity morphism \(\Id_{\xi}: \xi \to
\xi\) one has
\[
f^{*}(\Id_{\xi})(b_0, x) = (b_0, x) = \Id_{f^{*} \xi}(b_0, x),
\]
therefore \(f^{*} \Id_{\xi} = \Id_{f^{*} \xi}\). Also, if \(v: \eta \to \zeta\)
is any other bundle morphism, we have
\[
f^{*}(v u)(b_0, x)
= (b_0, v u x)
= f^{*}(v)(b_0, u x)
= f^{*}(v)(f^{*}(u)(b_0, x)),
\]
that is, \(f^{*}(v u) = f^{*} v \circ f^{*} u\). This finishes the proof that
\(f^{*}\) is indeed a functor.

The only additional information the diagram brings is that \(u f_{\xi}\) should
equal \(f_{\eta} f^{*} u\), and this is what we'll show. Let \((b_0, x) \in
E(f^{*} \xi)\) be any point, then
\[
u f_{\xi} (b_0, x)
= u x
= f_{\eta}(b_0, u x)
= f_{\eta} (f^{*} u) (b_0, x),
\]
which proves the commutativity of the diagram.
\end{proof}

\begin{proposition}[Functorial transitivity of the induced bundle]
\label{prop:induced-bundle-functorial-transitivity}
Consider continuous maps \(B_1 \xrightarrow g B_0 \xrightarrow f B\), and a
bundle \(\xi = (E, p, B)\). Then the following are properties conserning the
induced bundles over \(\xi\):
\begin{enumerate}[(a)]\setlength\itemsep{0em}
\item Considering the identity map \(\Id: B \to B\), there exists a bundle
  \(B\)-isomorphism
  \[
  \Id^{*} \xi \iso \xi.
  \]

\item There exists a \(B_1\)-isomorphism of bundles
  \[
  g^{*} f^{*} \xi \iso (f g)^{*} \xi.
  \]
\end{enumerate}
\end{proposition}

\begin{proof}
Notice that the morphisms of bundles \(\xi \to \Id^{*} \xi\) given by
\(x \mapsto (p x, x)\) has an inverse \((b, x) \mapsto x\), proving the first
isomorphism. For the second item, define
\(u: g^{*} f^{*} \xi \to (f g)^{*} \xi\) by the mapping
\(u(b_1, (b_0, x)) \coloneq (b_1, x)\) then by the fact that
\((b_1, (b_0, x)) \in E (g^{*} f^{*} \xi)\) if and only if
\(g b_1 = p_1(b_0, x) = b_0\), we can conclude that \(u\) is an isomorphism of
\(B_1\)-bundles.
\end{proof}

\section{Fibre Bundles}

\begin{definition}[Bundle projection]
\label{def:bundle-projection}
Let \(X\), \(B\), and \(F\) be Hausdorff spaces. We say that a continuous map
\(p: X \to B\) is a \emph{bundle projection} with \emph{fibre} \(F\) if for each
\(b \in B\) there exists a neighbourhood \(U \subseteq B\) of \(b\) such that
there is a \emph{topological isomorphism}
\[
\phi: U \times F \longrightarrow p^{-1} U,
\quad\text{such that}\quad
p \phi(x, y) = x
\]
for all \(x \in U\) and \(y \in F\)---the map \(\phi\) is called a
\emph{trivialization of the bundle over \(U\)}. This means that on the set
\(p^{-1} U\), the map \(p\) is a \emph{projection} of the type
\(U \times F \epi U\).
\end{definition}

\begin{definition}[Fibre bundle]
\label{def:fibre-bundle}
Let \(G\) be a topological group acting \emph{effectively} on a Hausdorff space
\(F\)---seen as a group of topological isomorphisms. Let \(X\) and \(B\) be
Hausdorff spaces. We define a \emph{fibre bundle} (or simply \emph{bundle}) over
the \emph{base space} \(B\) with \emph{total space} \(X\), \emph{fibre} \(F\),
and \emph{structure group} \(G\), to be a pair \((p, \Phi)\) where
\(p: X \to B\) is a \emph{bundle projection} and \(\Phi\) is a collection of
trivializations of \(p\) (as described in \cref{def:bundle-projection})---the
members of \(\Phi\) will be called \emph{charts} over \(U\)---such that:
\begin{itemize}\setlength\itemsep{0em}
\item For each \(b \in B\) there exists a neighbourhood \(U \subseteq B\) of
  \(b\) and a chart \(\phi \in \Phi\) of the form
  \(\phi: U \times F \to p^{-1} U\).

\item Given a chart \(\phi: U \times F \to p^{-1} U\), member of \(\Phi\), then
  any subset \(V \subseteq U\) is such that the restriction
  \(\phi|_{V \times F}\) belongs to the family \(\Phi\).

\item Given any pair of charts \(\phi, \psi \in \Phi\) over a common open set
  \(U\), there exists a continuous map \(\theta: U \to G\) such that
  \[
  \psi(u, y) = \phi(u, \theta(u)(y)).
  \]

\item The family \(\Phi\) is \emph{maximal} among the collections satisfying the
  previous properties.
\end{itemize}
The fibre bundle is said to be \emph{smooth} if each object above is a
\emph{smooth manifold} and all maps are \emph{smooth morphisms}.
\end{definition}

\section{Vector Bundle}

\subsection{First definitions}

\begin{definition}[Vector bundle]
\label{def:vector-bundle}
A (topological) \emph{vector bundle} is a fibre bundle with a fibre \(\R^n\) and
structure group contained in \(\GL_n(\R)\). Given a vector bundle \(\xi\), we
denote its total space by \(E \xi\) and base space by \(B \xi\).
\end{definition}

\begin{notation}
\label{not:fibre-of-vector-bundle}
Given a vector bundle \(\xi = (E, p, B)\), we denote by
\(\xi_b \coloneq p^{-1} b\) (which can also be denoted by \(E_b\)) the fibre of
\(b \in B\) over \(p\).
\end{notation}

\begin{definition}[Morphism of vector bundles]
\label{def:morphism-vector-bundles}
If \(\xi = (E, p, B)\) and \(\xi' = (E', p', B')\) are any two vector bundles,
we define a \emph{bundle morphism} \(\xi \to \xi'\) is a pair \((u, f)\) of
continuous maps \(u: E \to E'\) and \(f: B \to B'\) such that the diagram
\[
\begin{tikzcd}
E \ar[d, "p"'] \ar[r, "u"] &E' \ar[d, "p'"] \\
B \ar[r, "f"'] &B'
\end{tikzcd}
\]
commutes and the restriction \(u_b: \xi_b \to \xi'_{f b}\) is
\emph{\(\R\)-linear} for every \(b \in B\).
\end{definition}

\begin{definition}
\label{def:vector-bundle-category}
We denote by \(\VecBun\) the category of vector bundles and morphisms between
them. Furthermore, given a base space \(B\), we also define a full subcategory
\(\VecBun_B\) of vector bundles with base \(B\) and \(B\)-morphisms.
\end{definition}

\subsection{Charts and Atlases}

\begin{definition}[Vector bundle chart \& atlas]
\label{def:vector-bundle-chart-and-atlas}
Given a vector bundle \((E, p, B)\), we define an \(n\)-dimensional vector
bundle chart \((U, \phi)\), for some open set \(U \subseteq B\), to be a
topological isomorphism
\[
\phi: p^{-1} U \overset{\iso}\longrightarrow U \times \R^n
\]
such that the diagram
\[
\begin{tikzcd}
p^{-1} U \ar[r, "\phi"]
\ar[d, "p"']
&U \times \R^n \ar[ld, bend left, "\pi_1"] \\
U &
\end{tikzcd}
\]
commutes in \(\Top\)---and \(\pi_1\) is the projection of the first factor. The
isomorphism \(\phi\) induces a collection
\((\phi_x: p^{-1} x \isoto \R^n)_{x \in U}\) of isomorphisms given by the
composition
\[
\begin{tikzcd}
p^{-1} x \ar[r, "\phi"', "\dis"]
\ar[rr, "\phi_x", bend left]
&x \times \R^n
\ar[r, "\dis", "\pi_2"']
&\R^n
\end{tikzcd}
\]
Therefore, given any \(y \in p^{-1} U\), if \(y \in p^{-1} x\), then
\(\phi y = (x, \phi_x y)\).
\end{definition}

\begin{definition}
\label{def:vector-bundle-atlas}
A family \(\Phi \coloneq (U_j, \phi_j)_{j \in J}\) of vector bundle charts on
\((E, p, B)\) with domain covering \(B\) and values in \(\R^n\) is said to form
a \emph{vector bundle atlas} for \((E, p, B)\) if for any two vector bundle
charts \((U, \phi)\) and \((V, \psi)\) of \(\Phi\) we have:
\begin{enumerate}[(a)]\setlength\itemsep{0em}
\item For every \(x \in U \cap V\), the transition map
  \[
  \psi_x \phi_x^{-1}: \R^n \overset{\iso}\longrightarrow \R^n
  \]
  is an \emph{\(\R\)-linear topological isomorphism}.

\item The map \(g: U \cap V \to \GL_n(\R)\) sending
  \(x \mapsto \psi_x \phi_x^{-1}\) is \emph{continuous}.
\end{enumerate}
If such conditions are satisfied, by the requirement of item (b), the atlas
\(\Phi\) induces a collection of continuous maps
\[
(g_{i j}: U_i \cap U_j \longrightarrow \GL_n(\R))_{(i, j) \in J \times J},
\]
called \emph{cocycle} of \(\Phi\). For any three \(i, j, k \in J\), and
\(x \in U_i \cap U_j \cap U_k\), one has that
\[
g_{ij}(x) g_{j k}(x)
= ((\phi_i)_x (\phi_j^{-1})_x) ((\phi_j)_x (\phi_k^{-1})_x)
= (\phi_i)_x (\phi_k^{-1})_{x}
= g_{i k} x.
\]
Moreover, for any \(j \in J\) we find \(g_{i i} x = \Id_{\R^n}\). The tuple
\((E, p, B, \Phi)\), is said to be a \emph{vector bundle with \(n\)-dimensional
  fibre}. Furthermore an atlas \emph{for} \((E, p, B)\) is a subatlas of
\(\Phi\).

For every \(x \in B\), we can endow the fibre \(E_x\) with the structure of an
\(\R\)-vector space for which \(\phi_x: E_x \isoto \R^n\) is an \(\R\)-linear
isomorphism, independently of the choice of \((\phi, U) \in \Phi\). We'll thus
call \(E\) an \emph{\(n\)-plane bundle}
\end{definition}

\begin{definition}[Zero section]
\label{def:zero-section}
Given a vector bundle \(\xi\), we shall denote by \(\zerosec: B \xi \to E \xi\)
the \emph{zero section} of \(\xi\), that is, the mapping \(x \mapsto 0 \in E_x\).
\end{definition}

\begin{definition}[Trivial vector bundle]
\label{def:trivial-vector-bundle}
The \emph{\(n\)-dimensional trivial vector bundle} is the vector bundle
\[
\varepsilon_B^n \coloneq (B \times \R^n, p, B, \Phi),
\]
where \(p\) is the projection of the first component, and \(\Phi\) is the unique
maximal vector bundle atlas on \(\varepsilon_B^n\) containing identity maps for
each open set of \(B \times \R^n\). A vector bundle over \(B\) is said to be
trivial if it is isomorphic---such isomorphism is said to be a
\emph{trivialization}---to \(\varepsilon_B^n\) for some \(n \in \N\).
\end{definition}

\begin{definition}[Smooth vector bundle]
\label{def:smooth-bundles}
A vector bundle \(\xi\) is said to be smoth if its associated spaces are smooth
manifolds and the projection is a \(C^{\infty}\)-morphism.
\end{definition}

\begin{example}[Tangent bundle]
\label{exp:tangent-bundle}
Given a smooth \(n\)-manifold \(M\), we define the \emph{tangent bundle} (see
the discussion at \cref{sub:tangent-bundle}) of \(M\) to be the vector bundle
\((T M, \pi, M)\). For each chart \(\phi: U \to \R^n\) we define a vector bundle
chart \(\pi^{-1} U \to U \times \R^n\) mapping tangent vectors
\(X \mapsto (x, \phi_{*\, x} X)\).

Given any \(C^{\infty}\)-morphism \(f: M \to N\) of manifolds, there is an
induced vector bundle morphism \(T f: T M \to T N\).
\end{example}

\begin{theorem}[Isomorphism of vector bundles]
\label{thm:isomorphism-of-vector-bundles}
Let \(u: \xi \to \eta\) be a \(B\)-morphism of \(n\)-dimensional vector
bundles. Then \(u\) is a \(B\)-isomorphism of vector bundles if and only if
\(u_b: \xi_b \to \eta_b\) is an \(\R\)-linear isomorphism for each \(b \in B\).
\end{theorem}

\begin{proof}
(\(\implies\)) If \(u\) is a \(B\)-isomorphsim of vector bundles, then the
restriction mapping \(u^{-1}|_{\eta_b}: \eta_b \to \xi_b\) is an inverse for
\(u_b\).

(\(\impliedby\)) For the converse, suppose that the restriction
\(u_b: \xi_b \isoto \eta_b\) is a linear isomorphism for all \(b \in
B\). Construct a set-function \(v: E \eta \to E \xi\) such that
\(v|_{\eta_b} \coloneq u_b^{-1}\)---we must show that \(v\) is continuous. Let
\(U \subseteq B\) be any open set, and \(\phi: \xi^{-1} U \isoto U \times \R^n\)
and \(\psi: \eta^{-1} U \isoto U \times \R^n\) be vector bundle charts for
\(\xi\) and \(\eta\), respectively. Then if we consider the map
\(\psi u \phi^{-1}\), we find that it has the form
\((b, x) \mapsto (b, f_b x)\)---where \(b \mapsto f_b\) is a mapping
\(U \to \GL_n(\R)\). On the other hand, the map \(\phi v \psi^{-1}\) is of the
form \((b, x) \mapsto (b, f_b^{-1} x)\)---where \(b \mapsto f\) is again a map
\(U \to \GL_n(\R)\).
\end{proof}

\begin{definition}[Whitney sum]
\label{def:whitney-sum}
Given vector bundles \(\xi, \eta \in \VecBun_B\), we define the \emph{Whitney
  sum} of \(\xi\) and \(\eta\) to be the fibre product
\(\xi \oplus \eta \in \VecBun_B\), where
\((\xi \oplus \eta)_b = \xi_b \oplus \eta_b\) has the structure of the direct
sum of vector spaces. Given charts \(\phi: U \times \R^n \to \xi^{-1} U\) and
\(\psi: U \times \R^m \to \eta^{-1} U\) for the respective vector bundles, we
define an induced vector bundle chart for \(\xi \oplus \eta\) to be
\[
\phi \oplus \psi: U \times \R^{n+m} \to \xi^{-1} U \oplus \eta^{-1} U.
\]

\end{definition}

\subsection{Induced Vector Bundles}

\begin{proposition}[Induced vector bundle]
\label{prop:induced-vector-bundle}
Let \(\xi = (E, p, B)\) be an \(n\)-dimensional vector bundle,
and\(f: B_0 \to B\) be a continuous map. Then the induced bundle
\(f^{*} \xi = (E_0, p_0, B_0)\) admits a unique vector bundle structure, and the
canonical morphism \((f_{\xi}, f): f^{*} \xi \to \xi\) is a vector bundle
morphism. Moreover, \(f_{\xi}: p_0^{-1} b_0 \to \xi^{-1} b\) is a a linear
isomorphism (refer to \cref{def:induced-bundle}).
\end{proposition}

\begin{proof}
From \cref{def:induced-bundle} we know that
\(p_0^{-1} b_0 = b_0 \times p^{-1} (f b_0)\). Consider a pair of fibre points
\((b_0, x), (b_0, y) \in p_0^{-1} b_0\) and define vector space structures
\((b_0, x) + (b_0, y) \coloneq (b_0, x + y)\), and
\(\lambda (b_0, x) = (b_0, \lambda x)\) for any \(\lambda \in \R\). We know that
the restriction \(f_{\xi}: p_0^{-1} b_0 \to p^{-1}(f b_0)\) is a topological
isomorphism (see \cref{prop:induced-bundle-associated-isomorphisms}), but via
the linear structure just defined in \(p_0^{-1} b_0\), we see that \(f_{\xi}\)
is also a linear map, thus it's an isomorphism of vector spaces. This shows that
\(p_0^{-1} b_0\) is restrained to this linear structure, which proves uniqueness.

For a local trivialization of \(f^{*} \xi\), consider, for any open set \(U
\subseteq B\), a vector bundle chart \(h: U \times \R^n \to p^{-1} U\) for
\(\xi\). We can construct a map \(h': f^{-1} U \times \R^n \to
p_0^{-1}(f^{-1} U)\) mapping \((b_0, x) \mapsto (b_0, h(f b_0, x))\), which
forms a vector bundle chart for \(f^{*} \xi\) over the open set \(f^{-1} U\).
\end{proof}

\subsection{Homotopy Properties of Vector Bundles}

\begin{lemma}[Local trivialization]
\label{lem:local-trivialization}
Let \(\xi = (E, p, B \times I)\) be a \(C^r\) vector bundle for some
\(0 \leq r \leq \infty\). Then every point \(b \in B\) admits a neighbourhood
\(V \subseteq B\) such that \(\xi|_{V \times I}\) is \emph{trivial}.
\end{lemma}

\begin{proof}
Since \(\xi\) is locally trivial and \(I\) is compact, then given \(b \in B\)
consider a neighbourhood \(V_j \subseteq B\) of \(b\) and a partition
\(0 = t_0 < \dots < t_m = 1\) of \(I\) such that \(\xi\) is trivial in a
neihbourhood of \(V_j \times I_j \coloneq V_j \times [t_{j-1}, t_j]\) for each
\(0 < j \leq m\). Define \(V \coloneq \bigcap_{j=1}^m V_j\) and let
\((U_j)_{j=1}^m\) be a collection where \(U_j \subseteq I\) is a neighbourhood
of \(I_j\) such that \(\xi|_{V \times U_j}\) is trivial.

We do induction on \(m\). For the base case \(m = 1\) it follows by construction
that \(\xi|_{V \times U_1}\) is trivial and since \(I_1 = I\) then \(U_1 =
I\). Now if \(m > 1\) we can proceed assuming that the case is true for all
\(n < m\): for each \(1 \leq j \leq m\) we find a neighbourhood
\(J \subseteq I\) of the interval \(I_1 \cup \dots \cup I_j = [0, t_j]\) for
which \(\xi|_{V \times J}\) is trivial. This hints at the fact that it is
sufficient to prove this construction for the case \(m = 2\), and this is what
we'll set out to do.

Take two subintervals \(U_1 \coloneq [0, b]\) and \(U_2 \coloneq [a, 1]\) where
\(0 < a < b < 1\). Let
\(\phi_j: p^{-1}(V \times U_j) \to (V \times U_j) \times \R^n\), for
\(j \in \{1, 2\}\), be \(C^r\)-charts (where we assumed \(\xi\) to be
\(n\)-dimensional). Associated with these maps is the cocycle mapping
\(g_{2 1}: V \times (U_1 \cap U_2) \to \GL_n(\R)\) mapping
\(x \mapsto \phi_{1\, x} \phi_{2\, x}^{-1}\). Let \(a < c < b\) and consider a
\(C^r\)-map \(\lambda: U_2 \to U_1 \cap U_2\) for which
\(\lambda|_{[a, c]} = \Id_{[a, c]}\). Defining \(\mu \coloneq \Id_V \times
\lambda: V \times [a, 1] \to V \times [a, b]\), we can construct a map
\[
h \coloneq g \mu: V \times [a, 1] \longrightarrow \GL_n(\R)
\]
such that \(h|_{V \times [a, c]} = g|_{V \times [a, c]}\). This allows us to
construct a trivialization \(\psi\) where we define, for each \(x \in V \times
I\), a map \(\psi_x: \xi_x \to \R^n\), and
\[
\psi_x \coloneq
\begin{cases}
  \phi_{1\, x}, &\text{if } x \in V \times [0, c] \\
  h(x) \phi_{2\, x}, &\text{if } x \in V \times [a, 1]
\end{cases}
\]
so that by construction for any \(x \in [a, c]\) one has
\(h(x) \phi_{2\, x} = (\phi_{1\, x} \phi_{2 x}^{-1}) \phi_{2\, x} = \phi_{1\,
  x}\), showing that \(\psi_x\) is well defined. This defines a trivialization
\(\psi: p^{-1}(V \times I) \to (V \times I) \times \R^n\) as wanted.
\end{proof}

\begin{corollary}
\label{cor:smooth-vector-bundle-over-interval-is-trivial}
Any \(C^r\) vector bundle, with \(0 \leq r \leq \infty\), over an
\emph{interval} is \emph{trivial}
\end{corollary}

\subsection{Oriented Vector Bundles}

\begin{definition}
\label{def:orientation-vector-bundle}
Let \(\xi\) be a vector bundle. We define an \emph{orientation} for \(\xi\) to
be a family \(\omega = (\omega_x)_{x \in B \xi}\) where \(\omega_x\) is an
orientation for the vector space fibre \(\xi_x\) such that \(\xi\) has an atlas
for which: every chart \(\phi: \xi|_U \to \R^n\) in the atlas \(\Phi\) of
\(\xi\) has an \emph{orientation preserving} map
\(\phi_x: (\xi_x, \omega_x) \to (\R^n, \omega_n)\)\footnote{The use of
  \(\omega_n\) denotes the standard orientation for the euclidean space \(\R^n\)
  (see \cref{def:standard-orientation-euclidean}).}. If this is the case,
\(\omega\) is said to be a \emph{coherent} family of orientations of the fibres
and \(\Phi\) is an \emph{oriented atlas} of \(\xi\).
\end{definition}

\begin{lemma}
\label{lem:unique-orientation-from-vector-bundle-isomorphism}
Let \(f: \eta \isoto \xi\) be an isomorphism in \(\VecBun\), and let \(\omega\)
be an orientation for \(\xi\). Then there exists a unique orientation \(\theta\)
for \(\eta\) for which \(f\) preserves the orientations of the fibres.
\end{lemma}

\begin{proposition}
\label{thm:simply-connected-manifold-admits-orientation}
Every vector bundle over a simply connected manifold admits an orientation.
\end{proposition}

\begin{proposition}
\label{prop:vecbun-orientable-iff-loop-preserves-orientation}
A vector bundle \(\xi\) over a manifold \(M\) is \emph{orientable} if and only
if every loop \(\gamma \in \Loop M\) preserves the orientation of
\(\xi_{\gamma(0)}\).
\end{proposition}

\begin{corollary}
\label{cor:orientable-vecbun-connected-manifold-2-orientations}
An orientable vector bundle over a connected manifold has only two orientations.
\end{corollary}

\end{document}

%%% Local Variables:
%%% mode: latex
%%% TeX-master: "../../../deep-dive"
%%% End:

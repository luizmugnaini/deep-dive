\section{Singular Homology}

\subsection{Life in Topological Spaces: Singular Simplices}

\begin{definition}[Convexity]
    \label{def:convexity}
    The following are definitions concerning convexity of spaces in \(\R^n\):
    \begin{enumerate}[(a)]\setlength\itemsep{0em}
        \item Given any two points \(x, y \in \R^n\), we define the \emph{segment} from
              \(x\) to \(y\) to be the set of points
              \(\{(1-t) x + t y \colon t \in [0, 1]\}\).

        \item A set \(C \subseteq \R^n\) is said to be \emph{convex} if for any pair of
              points \(x, y \in C\) the segment from \(x\) to \(y\) is contained in \(C\).

        \item Given a set \(A \subseteq \R^n\), we define the \emph{convex hull} of
              \(A\) to be the intersection of all convex sets of \(\R^n\) containing \(A\).

        \item An \emph{\(m\)-simplex} \(\sigma\) in \(\R^n\) is defined to be the
              convex-hull of a collection of \(m+1\) distinct points \(\{x_0, \dots, x_m\}\)
              such that the set \(\{x_j - x_0 \colon 1 \leq j \leq m\}\) is linearly
              independent. The points \(x_j\) are called the \emph{vertices} of
              \(\sigma\). If an \emph{order} is assigned to the collection of vertices of
              \(\sigma\), then we obtain an ordered simplex---the ordered simplex is
              commonly written as \(\sigma = [x_0, \dots, x_m]\).
    \end{enumerate}
\end{definition}

\begin{lemma}
    \label{lem:linear-independence-simplex}
    Let \(\{x_0, \dots, x_m\} \subseteq \R^n\) be a collection of \(m+1\) distinct
    points. Then the set \(L = \{x_j - x_0 \colon 1 \leq j \leq m\}\) is linearly
    independent if and only if for any two sequences of parameters \((s_j)_{j=0}^m\)
    and \((t_j)_{j=0}^m\) satisfying both \(\sum_j s_j x_j = \sum_j t_j x_j\) and
    \(\sum_j s_j = \sum_j t_j\) implies that \(s_j = t_j\) for each \(0 \leq j \leq m\).
\end{lemma}

\begin{proof}
    Suppose that \(L\) is linearly independent and the two conditions hold for a given
    pair of sequences of parameters. Then
    \begin{align*}
        0 & = \sum_{j=0}^m (s_j - t_j) x_j                                          \\
          & = \sum_{j=0}^m (s_j - t_j) x_j - \Big( \sum_{j=0}^m s_j - t_j \Big) x_0 \\
          & = \sum_{j=1}^{m} (s_j - t_j) (x_j - x_0)
    \end{align*}
    but since \(L\) is linearly independent, then it must be the case that
    \(s_j - t_j = 0\). For the second case, assume only that the two conditions for
    any pair sequences are met: then if \(\sum_{j=1}^m a_j (x_j - x_0) = 0\) then
    \(\sum_j a_j x_j = \sum_j a_j x_0\) hence \(a_j = 0\) for each
    \(1 \leq j \leq m\)---proving that \(L\) is linearly independent.
\end{proof}

\begin{corollary}
    \label{cor:barycentric-coordinates-simplex}
    Let \(\sigma\) be the \(m\)-simplex given by the convex hull of the set of
    points \(\{x_0, \dots, x_m\}\). Then every point contained in \(\sigma\) is
    uniquely represented by \(\sum_{j=0}^m t_j x_j\) where \(t_j \geq 0\) and
    \(\sum_j t_j = 1\). The tuple \((t_0, \dots, t_m)\) is called the
    \emph{barycentric coordinates} of the point in question.
\end{corollary}

\begin{corollary}
    \label{cor:ordered-simplex-iso-standard-simplex}
    Let \(\sigma = [x_0, \dots, x_m]\) be an ordered \(m\)-simplex. Then the  map
    \(\splxtop^m \to \sigma\), from the standard topological \(m\)-simplex, given by
    \((t_0, \dots, t_m) \mapsto \sum_{j=0}^m t_j x_j\) is an isomorphism of
    topological spaces.
\end{corollary}

\begin{definition}[Singular simplex]
    \label{def:singular-simplex}
    Let \(X\) be a topological space. We define a \emph{singular \(m\)-simplex} in
    \(X\) to be a morphism of topological spaces \(\phi: \splxtop^m \to X\).

    Following the construction of the singular complex functor (see
    \cref{def:singular-complex-functor}) we define a structure of a \emph{free
        abelian group} to the set
    \[
        \Sing_m X \coloneq \Sing_{\splxtop^m} X = \Hom_{\Top}(\splxtop^m, X).
    \]
    An element of \(\Sing_m X\) is called a \emph{singular \(m\)-chain} of \(X\) and
    assumes the form \(\sum_{\phi} n_{\phi} \phi\) for finitely many non-zero
    integers \(n_{\phi}\) associated with singular \(m\)-simplices \(\phi\).

    Together with the singular simplex we also define, for each \(0 \leq j \leq m\)
    a morphism of abelian groups
    \[
        \face_m^{(j)}: \Sing_m X \longrightarrow \Sing_{m-1} X
    \]
    called \emph{\(j\)-th face map}, which is explicitly given by
    \[
        \face_m^{(j)} \phi(t_0, \dots, t_{m-1})
        \coloneq \phi(t_0, \dots, t_{j-1}, 0, t_j, \dots, t_{m-1}).
    \]
    That is, the \(j\)-th face map embeds \(\splxtop^{m-1}\) into \(\splxtop^m\)
    face opposite to the \(j\)-th vertex and then map it to \(X\) again via
    \(\phi\). More compactly, we can also write
    \[
        \face_m^{(j)} \phi \eqqcolon \phi|_{[0, \dots, \widehat j, \dots, m]}.
    \]

    We shall also define the \emph{boundary operator} to be the morphism
    of abelian groups
    \[
        \face_m: \Sing_m X \longrightarrow \Sing_{m-1} X
    \]
    to be given by the alternating sum of face maps
    \[
        \face_m \coloneq \sum_{j=0}^m (-1)^j \face_m^{(j)}.
    \]
\end{definition}

\begin{proposition}
    \label{prop:boundary-squared-is-zero}
    The composition of the maps
    \(\Sing_m X \xrightarrow{\face_m} \Sing_{m-1} X \xrightarrow{\face_{m-1}}
    \Sing_{m-2} X\) is identically zero. That is, the boundary of an \(m\)-chain is
    an \((m-1)\)-chain with empty boundary.
\end{proposition}

\begin{proof}
    With no loss of generality, consider merely one singular \(m\)-simplex
    \(\phi\). One has
    \begin{align*}
        \face_{m-1} \face_m \phi
         & = \sum_{i=0}^m (-1)^i
        \Big( \sum_{j=0}^{m-1} (-1)^j \partial_j \partial_i \phi \Big) \\
         & = \sum_{i=0}^m (-1)^i
        \Big(
        \sum_{i > j} (-1)^j
        \phi|_{[v_0, \dots, \widehat v_j, \dots, \widehat v_i, \dots, v_{m-2}]}
        + \sum_{i < j} (-1)^{j-1}
        \phi|_{[v_0, \dots, \widehat v_i, \dots, \widehat v_j, \dots, v_{m-2}]}
        \Big)                                                          \\
    \end{align*}
    Notice that the first term accounts for the removal of \(j\) then of \(i\), with
    \(i > j\), yielding a factor \((-1)^{i + j}\) for each newly generated
    simplex. The second term deals with the removal of \(i\) first and then \(j\),
    with \(i < j\), thus \(j\) must be updated to a lower index by \(1\), hence the
    factor \((-1)^{i + j - 1}\). Notice that the pair of sums has different signs
    and the same simplices---therefore they pairwise cancel, yielding zero.
\end{proof}

\subsection{Cycles \& Boundaries: Homology}

\begin{definition}
    \label{def:cycle-boundary-homology}
    Let \(X\) be a topological space. We define the following:
    \begin{enumerate}[(a)]\setlength\itemsep{0em}
        \item A singular \(n\)-chain \(c \in \Sing_n X\) is said to be an
              \emph{\(n\)-cycle} if \(\face c = 0\). The collection of \(n\)-cycles of
              \(\Sing_n X\) forms a subgroup \(\Cycle_n(X)\).

        \item A singular \(n\)-chain \(b \in \Sing_n X\) is called an
              \emph{\(n\)-boundary} if there exists \(s \in \Sing_{n+1} X\) with
              \(\face s = b\). The collection of such \(n\)-boundaries forms a subgroup of
              \(\Sing_n X\) named \(\Bdry_n(X)\)

        \item From \cref{prop:boundary-squared-is-zero} we obtain
              \(\Bdry_n(X) \subseteq \Cycle_n(X)\) and thus we may define the \emph{\(n\)-th
                  singular homology group of \(X\)} as the quotient group
              \[
                  \Hmlg_n(X) \coloneq \Cycle_n(X) / \Bdry_n(X).
              \]
    \end{enumerate}
\end{definition}

\begin{definition}[Chain complex]
    \label{def:chain-complex}
    We define a \emph{chain complex} to be a \(\Z\)-graded abelian group
    \(C = (C_n)_{n \in \Z}\) together with a graded endomorphism \(\face: C \to C\)
    of degree \(-1\) such that \(\face \circ \face = 0\). Given chain complexes
    \((C, \face)\) and \((C', \face')\) we define a \emph{morphism of chain
        complexes} (or \emph{chain map}) \(\Phi: C \to C'\) to be a morphism of graded
    abelian groups with \emph{zero degree} satisfying the commutativity of the
    square
    \[
        \begin{tikzcd}
            C_n \ar[d, "\Phi_n"'] \ar[r, "\face_n"] &C_{n-1} \ar[d, "\Phi_{n-1}"] \\
            C_n' \ar[r, "\face_n'"'] &C_{n-1}'
        \end{tikzcd}
    \]
    for every \(n \in \Z\).

    Considering the \(\Z\)-graded abelian groups \(\Cycle_{\bullet}(C) = \ker
    \face\) and \(\Bdry_{\bullet}(C) = \im \face\) we obtain the following
    \(\Z\)-graded quotient abelian group
    \[
        \Hmlg_{\bullet}(C) = \Cycle_{\bullet}(C)/\Bdry_{\bullet}(C).
    \]
    Moreover, given a chain map \(\Phi: C \to C'\) since
    \(\Phi(\Cycle_{\bullet}(C)) \subseteq \Cycle_{\bullet}(C')\) and
    \(\Phi(\Bdry_{\bullet}(C)) \subseteq \Bdry_{\bullet}(C)\) then \(\Phi\) induces
    a morphism of graded groups between the homology groups associated with \(C\)
    and \(C'\):
    \[
        \Phi_{*}: \Hmlg_{\bullet}(C) \longrightarrow \Hmlg_{\bullet}(C').
    \]

    Given a topological space \(X\), its associated chain complex is given by
    \(\Sing_{\bullet} X\) together with the boundary operator \(\face\) and the
    homology of \(X\) is the homology of the chain complex \((\Sing_{\bullet} X, \face)\).
\end{definition}

\begin{corollary}
    \label{cor:homology-zero-iff-seq-is-exact}
    Given a topological space \(X\), the homology \(\Hmlg_{\bullet}(X)\) is zero if
    and only if the sequence \((\Sing_{\bullet} X, \face_{\bullet})\) is
    exact. In other words, \(\Hmlg_{\bullet}(X)\) measures how exact is the chain
    complex of \(X\).
\end{corollary}

\begin{proposition}
    \label{prop:induced-chain-map-and-homology-map}
    Let \(f: X \to Y\) be a morphism of topological spaces. For each \(n \in \Z\)
    and singular \(n\)-simplex \(\phi \in \Sing_n X\) we obtain an induced singular
    \(n\)-simplex in \(Y\) given by the \emph{pushforward}
    \(f_{\#\, n}(\phi) = f \phi \in \Sing_n Y\). From this we obtain a morphism of
    abelian groups \(f_{\#\, n}: \Sing_n X \to \Sing_n Y\). This can be extended to
    a chain map
    \[
        f_{\#}: \Sing_{\bullet} X \longrightarrow \Sing_{\bullet} Y.
    \]
    Moreover, this map also induces a graded morphism of zero degree between the
    homology groups of \(X\) and \(Y\):
    \[
        f_{*}: \Hmlg_{\bullet}(X) \longrightarrow \Hmlg_{\bullet}(Y).
    \]
    Both \(f_{\#}\) and \(f_{*}\) have a functorial nature.
\end{proposition}

\begin{proof}
    To prove that \(f_{\#}\) is a chain map, it must be the case that
    \[
        \begin{tikzcd}
            \Sing_n X \ar[d, "f_{\#\, n}"'] \ar[r, "\face_n"]
            &\Sing_{n-1} X \ar[d, "f_{\#\, n-1}"]
            \\
            \Sing_n Y \ar[r, "\face_n'"']
            &\Sing_{n-1} Y
        \end{tikzcd}
    \]
    commutes for any \(n \in \Z\). Let \(\phi \in \Sing_n X\) be any element and
    notice that for any \(0 \leq j \leq n\) we have
    \[
        f_{\#\, n-1}(\partial_n^{(j)} \phi)
        = f \phi|_{[0, \dots, \widehat j, \dots, n]},
    \]
    moreover
    \[
        {\face_n'}^{(j)} f_{\# n}(\phi)
        = (f_{\# n} (\phi))|_{[0, \dots, \widehat j, \dots, n]}
        = (f \phi)|_{[0, \dots, \widehat j, \dots, n]}
        = f \phi|_{[0, \dots, \widehat j, \dots, n]}
    \]
    therefore the diagram indeed commutes.
\end{proof}

\begin{proposition}
    \label{prop:direct-sum-of-chain-complexes}
    Let \((C^{\lambda}, \face^{\lambda})_{\lambda \in \Lambda}\) be a collection of chain
    complexes and define a chain complex \(C \coloneq \bigoplus_{\lambda \in
        \Lambda} C^{\lambda}\) as \(C_p = \bigoplus_{\lambda \in \Lambda}
    C_p^{\lambda}\) and the natural boundary maps. Then there exists a natural isomorphism
    \[
        \Hmlg_{\bullet}(C) \iso \bigoplus_{\lambda \in \Lambda} \Hmlg_{\bullet}(C^{\lambda})
    \]
\end{proposition}

\begin{proof}
    From construction we have
    \(\Sing_{\bullet} C = \bigoplus_{\lambda \in \Lambda} \Sing_{\bullet}
    C^{\lambda}\). Therefore for any \(p \in \Z\) we have
    \begin{align*}
        \Hmlg_p(C)
         & = \Cycle_p(C)/\Bdry_p(C)                                                        \\
         & = \frac{\bigoplus_{\lambda \in
                \Lambda}\Cycle_p(C^{\lambda})}{\bigoplus_{\lambda \in \Lambda}\Bdry_p(C^{\lambda})}
        \\
         & \iso \bigoplus_{\lambda \in \Lambda} \Cycle_p(C^{\lambda})/\Bdry_p(C^{\lambda}) \\
         & = \bigoplus_{\lambda \in \Lambda} \Hmlg_p(C^{\lambda}).
    \end{align*}
\end{proof}

\begin{proposition}
    \label{prop:coprod-space-decomposition-homology-group}
    Let \(X \iso \bigdisj_{\lambda \in \Lambda} X_{\lambda}\) be a topological
    space, where \(X_{\lambda}\) are path-connected components of \(X\), then
    \[
        \Hmlg_{\bullet}(X) \iso \bigoplus_{\lambda \in \Lambda} \Hmlg_{\bullet}(X_{\lambda}).
    \]
\end{proposition}

\begin{proof}
    For each \(p \in \Z\) consider the morphism of groups
    \(\Theta_p: \bigoplus_{\lambda} \Sing_p X_{\lambda} \to \Sing_p X\) given by
    \[
        \Big( \sum_{\phi_{\lambda}} n_{\lambda, \phi_{\lambda}} \phi \Big)_{\lambda}
        \longmapsto
        \sum_{\lambda} \sum_{\phi_{\lambda}} n_{\lambda, \phi_{\lambda}} \phi_{\lambda}.
    \]
    The \(\Z\)-graded abelian group morphism
    \(\Theta: \bigoplus_{\lambda} \Sing_{\bullet} X_{\lambda} \to \Sing_{\bullet}
    X\) given by \((\Theta_p)_{p \in \Z}\) is a morphism between free abelian
    groups, therefore \(\Theta\) is injective. Let \(\phi \in \Sing_p X\) be any
    singular \(p\)-simplex of \(X\). Since \(\splxtop^p\) is path-connected, its
    image \(\phi(\splxtop^p) \subseteq X\) must be contained in some unique
    path-connected component \(X_{\lambda}\) of \(X\). Therefore we might as well
    restrict the target of \(\phi\) and obtain a singular \(p\)-simplex
    \(\phi|^{X_{\lambda}} \in \Sing_p X_{\lambda}\). This shows that \(\Theta_p(\phi|^{X_{\lambda}}) = \phi\) and
    therefore \(\Theta\) is a surjective map. Therefore \(\Theta\) is an isomorphism. Using
    \cref{prop:direct-sum-of-chain-complexes} we obtain the desired isomorphism
    between homology groups.
\end{proof}

\section{Calculating The Homology: Some Examples}

\begin{proposition}
    \label{prop:path-connected-homology0-is-Z}
    Let \(X\) be path-connected, then \(\Hmlg_0(X) \iso \Z\).
\end{proposition}

\begin{proof}
    Define the morphism of groups \(\varepsilon: \Sing_0 X \to \Z\) to be given by
    \(\sum_j n_j \sigma_j \mapsto \sum_j n_j\)---which is clearly surjective. Notice
    that if \(\sigma \in \Sing_1 X\) is any singular \(1\)-simplex, then
    \(\face_1 \sigma = \sigma_{[\widehat 0, 1]} - \sigma_{[0, \widehat
                1]}\). Therefore \(\varepsilon \sigma = 1 - 1 = 0\) and thus
    \(\im \face_1 \subseteq \ker \varepsilon\). For the contrary, let
    \(\sum_j n_j \sigma_j \in \ker \varepsilon\)---that is, \(\sum_j n_j = 0\). Let
    \(x_0 \in X\) be any point and, since \(X\) is path-connected, take for each
    \(j\) a singular \(1\)-simplex \(\tau_j: \splxtop^1 \to X\) (which is a path in
    \(X\) since \(\splxtop^1 = I\)) connecting \(x_0\) with \(\im \sigma_j\)---which
    is a unique point in \(X\) the domain of \(\sigma_j\) is a single point
    \(\splxtop^0\). Then
    \[
        \face_1 \Big( \sum_j n_j \tau_j \Big) = \sum_j n_j (\sigma_j - x_0)
        = \sum_j n_j \sigma_j - \Big(\sum_j n_j\Big) x_0
        = \sum_j n_j \sigma_j,
    \]
    which shows that \(\ker \varepsilon \subseteq \im \face_1\). Therefore
    \[
        \begin{tikzcd}
            \Sing_0 X \ar[r, "\varepsilon"] \ar[d, two heads]
            &\Z \\
            \frac{\Sing_0 X}{\ker \varepsilon}
            \ar[ru, dashed, "\dis"', bend right] &
        \end{tikzcd}
    \]
    induces a unique isomorphism \(\Hmlg_0(X) \iso \Z\) since
    \[
        \Sing_0(X)/\ker\varepsilon = \Sing_0(X)/\im \face_1 = \Hmlg_0(X).
    \]
\end{proof}

\begin{example}[Single point space homology]
    \label{exp:homology-of-point-space}
    Let \(X = \{*\}\) be the discrete space with a single point. For any \(p \in
    \N\) we have a unique singular \(p\)-simplex \(\phi_p: \splxtop^p \to X\), and
    thus \(\face_p^{(j)} \phi_p = \phi_{p-1}\)---where \(\phi_{p-1}\) is the unique
    singular \(p-1\)-simplex of \(X\). This shows that each \(\Sing_p X\) is a
    cyclic group generated by \(\phi_p\). Moreover, for any \(p \in \N\) we have
    \[
        \face \phi_p = \sum_{j=0}^p (-1)^j \face_j \phi_p
        = \sum_{j=0}^p (-1)^j \phi_{p-1}
        =
        \begin{cases}
            \phi_{p-1}, & \text{if } p \text{ is even and } p > 0 \\
            0,          & \text{otherwise}
        \end{cases}
    \]
    Therefore one has \(\Cycle_p(X) = \Bdry_p(X)\) for each \(p > 0\), and
    \(\Cycle_0(X) = \Sing_0 X\) while \(\Bdry_0(X) = 0\). Hence we conclude that the
    homology groups associated to \(X\) are
    \[
        \Hmlg_p(X) =
        \begin{cases}
            \Z, & \text{if } p = 0 \\
            0,  & \text{if } p > 0
        \end{cases}
    \]
    where \(\Hmlg_0(X) = \Z\) comes from the fact that \(X\) is path-connected and
    \cref{prop:path-connected-homology0-is-Z}.
\end{example}

\begin{theorem}
    \label{thm:convex-space-Rn-zero-homology}
    Let \(X \subseteq \R^n\) be a convex subspace. Then for any \(p > 0\) we have
    \[
        \Hmlg_p(X) = 0.
    \]
\end{theorem}

\begin{proof}
    If \(X = \emptyset\) then we are done, now assume that \(X\) is non-empty. Let \(x \in X\)
    be any point and consider a singular \(p\)-simplex \(\phi \in \Sing_p X\) for some
    \(p \geq 0\). Since \(X\) is convex, we can define a singular \((p+1)\)-simplex \(\theta
    \in \Sing_{p+1} X\) given by
    \[
        \theta(t_0, \dots, t_{p+1}) \coloneq
        \begin{cases}
            (1 - t_0) \Big(\phi\Big( \frac{t_{1}}{1 - t_{0}}, \dots, \frac{t_{p+1}}{1 - t_0}
            \Big)\Big) + t_0 x, & \text{for } t_0 < 1  \\
            x,                  & \text{for } t_0 = 1.
        \end{cases}
    \]
    We have to check that \(\theta\) is indeed continuous at \(x\). Notice that
    \begin{align*}
        \lim_{t_0 \to 1} \| \theta(t_0, \dots, t_{p+1}) - x \|
         & = \lim_{t_0 \to 1} \Big\| (1 - t_0)
        \Big(\phi\Big( \frac{t_{1}}{1 - t_{0}}, \dots, \frac{t_{p+1}}{1 - t_0} \Big)\Big)
        - (1 - t_0) x
        \Big\|                                          \\
         & \leq \lim_{t_0 \to 1} (1 - t_0) \Big( \Big\|
        \phi\Big( \frac{t_{1}}{1 - t_{0}}, \dots, \frac{t_{p+1}}{1 - t_0} \Big)
        \Big\|
        + \| x \|
        \Big).
    \end{align*}
    Using the fact that the image \(\phi(\splxtop^p) \subseteq X\) is compact, it follows that
    the second term of the right-hand side of the above equation is bound, therefore
    \[
        \lim_{t_0 \to 1} \| \theta(t_{0}, \dots, t_{p+1}) - x \| = 0.
    \]

    This construction induces a morphism \(T_p: \Sing_p X \to \Sing_{p+1} X\) given by
    \(\phi \mapsto \theta\) for every \(p \geq 0\). Moreover, notice that
    \(\face_{p+1}^0 T_p \phi = \face_{p+1}^0 \theta = \phi\) and
    \(T_p \face_{p+1}^0 \theta = T_p \phi = \theta\), therefore the induced morphism
    \(T: \Sing_{\bullet} X \to \Sing_{\bullet} X\) is an isomorphism of
    \(\Z\)-graded abelian groups with inverse \(\face_{\bullet}^0\). Hence
    \(\Sing_p X \iso \Sing_{p + 1} X\) for any \(p \geq 0\).

    We now check that \(T\) is a chain map. For each
    \(1 \leq j \leq p+1\) one has
    \begin{align*}
        \face_{p+1}^j (T_p(\phi))(t_0, \dots, t_p)
         & = T_p(\phi)(t_0, \dots, t_{j-1}, 0, t_j, \dots, t_p) \\
         & = (1 - t_0) \Big(
        \phi\Big(
            \frac{t_{1}}{1 - t_{0}}, \dots, \frac{t_{j-1}}{1 - t_0}, 0,
            \frac{t_{j}}{1 - t_0}, \dots, \frac{t_{p}}{1 - t_{p}}
            \Big)
        \Big) + t_0 x
    \end{align*}
    and for \(\face_p^{j-1}\) we have
    \begin{align*}
        T_{p-1}(\face_p^{j-1} \phi)(t_0, \dots, t_{p})
         & = (1 - t_0) \face_p^{j-1} \phi\Big(
        \frac{t_{1}}{1 - t_0}, \dots, \frac{t_{p}}{1 - t_{0}}
        \Big) + t_0 x                          \\
         & = (1 - t_0) \phi\Big(
        \frac{t_{1}}{1 - t_{0}}, \dots, \frac{t_{j-1}}{1 - t_{0}}, 0,
        \frac{t_j}{1 - t_0}, \dots, \frac{t_{p}}{1 - t_0}
        \Big) + t_0 x.
    \end{align*}
    With this we conclude that
    \[
        \face_{p+1}^j T_p = T_{p-1} \face_p^{j-1}.
    \]

    For any \(\phi \in \Sing_p X\) we have
    \begin{align*}
        \face_{p+1} T_p \phi
         & = \face_{p+1}^0 T_p \phi + \sum_{j=1}^{p+1} (-1)^j \face_{p+1}^j T_p \phi \\
         & = \face_{p+1}^0 T_p \phi + \sum_{j=1}^{p+1} (-1)^j \face_{p+1}^j T_p \phi
        - \Big(
        \sum_{j=1}^{p+1} (-1)^j T_{p-1} \face_p^{j-1} \phi
        + \sum_{j=0}^p (-1)^j T_{p-1} \face_p^j \phi
        \Big)                                                                        \\
         & = \face_{p+1}^0 T_p \phi + \Big(
        \sum_{j=1}^{p+1} (-1)^j \face_{p+1}^j T_p \phi
        - \sum_{j=1}^{p+1} (-1)^j T_{p-1} \face_p^{j-1} \phi
        \Big) + \sum_{j=0}^p (-1)^j T_{p-1} \face_p^j \phi                           \\
         & = \face_{p+1}^0 T_p \phi + \sum_{j=0}^p (-1)^j T_{p-1} \face_p^j \phi     \\
         & = \phi - T \face_p \phi
    \end{align*}
    Therefore we conclude that for any \(p \geq 1\) we have
    \[
        \face_{p+1} T_p + T_{p-1} \face_p = \Id_{\Sing_p X}.
    \]

    Consider a cycle \(z \in \Cycle_p(X)\) for \(p > 0\) and notice that
    \(\face_{p+1} T_p z + T_{p-1} \face_p z = z\) implies in
    \(\face_{p+1} T_p z = z\) since \(\face_p z = 0\). Therefore from definition we
    conclude that \(z \in \Bdry_p(X)\), which implies that \(\Hmlg_p(X) = 0\) for each
    \(p > 0\).
\end{proof}

\section{Chain Homotopies}

Let \(T: (C, \face) \to (C', \face')\) be a morphism of
graded groups of degree \(1\). The morphism of graded groups
\[
    \face' T + T \face: C \longrightarrow C'
\]
is then of degree \(0\) since both \(\face\) and \(\face'\) have degree
\(-1\). Moreover, \(\face' T + T \face\) is in fact a chain map since
\[
    \face'(\face' T + T \face)
    = \face' \face' T + \face' T \face
    = \face' T \face
    = \face' T \face + T \face \face
    = (\face' T + T \face) \face
\]
shows that it satisfies the needed commutativity condition. Notice that for any
cycle \(z \in \Cycle_p(C)\) we have
\[
    (\face' T + T \face) z = \face' T z + T \face z = \face' T z \in \Bdry_p(C').
\]
Therefore the induced morphism of homology groups
\[
    (\face' T + T \face)_{*}: \Hmlg_{\bullet}(C) \longrightarrow \Hmlg_{\bullet}(C')
\]
is identically zero.

\begin{definition}
    \label{def:chain-homotopic-maps}
    Let \(f, g: (C, \face) \to (C', \face')\) be parallel chain
    maps. We say that \(f\) and \(g\) are \emph{chain-homotopic} if there exists a
    morphism of graded groups \(T: C \to C'\) with degree \(1\)
    such that
    \[
        \face' T + T \face = f - g.
    \]
\end{definition}

\begin{proposition}
    \label{prop:homotopic-chain-maps-equal-induced-homology-morphisms}
    Let \(f, g: C \para C'\) be chain-homotopic chain morphisms, then we have the
    equality
    \[
        f_{*} = g_{*}: \Hmlg_{\bullet}(C) \longrightarrow \Hmlg_{\bullet}(C')
    \]
    of the induced morphisms of homology groups.
\end{proposition}

\begin{proof}
    Let \(T: C \to C'\) be a chain-homotopy between \(f\) and \(g\), thus \(f - g =
    \face' T - T \face\) and we have
    \[
        f_{*} - g_{*} = (f - g)_{*} = (\face' T - T \face)_{*} = 0.
    \]
\end{proof}

\begin{theorem}
    \label{thm:homotopic-maps-equal-induced-homology-morphism}
    Let \(f, g: X \para Y\) be homotopic topological morphisms. Then we have the
    equality
    \[
        f_{*} = g_{*}: \Hmlg_{\bullet}(X) \longrightarrow \Hmlg_{\bullet}(Y)
    \]
    of the induced morphisms of homology groups.
\end{theorem}

\begin{proof}
    In view of \cref{prop:homotopic-chain-maps-equal-induced-homology-morphisms} we
    shall merely prove that the maps \(f_{\#}\) and \(g_{\#}\) are
    chain-homotopic. Let \(\eta: f \htpy g\) be a homotopy and consider the
    inclusions \(i_0, i_1: X \para X \times I\) mapping, respectively
    \(x \mapsto (x, 0)\) and \(x \mapsto (x, 1)\).

    We now discuss why it is sufficient for \(i_{0\, \#}\) and \(i_{1\, \#}\) to be
    chain-homotopic in order for \(f_{\#}\) and \(g_{\#}\) to be
    chain-homotopic. Suppose that \(i_{0\, \#}\) and \(i_{1\, \#}\) are indeed
    chain-homotopic and let \(T: \Sing_{*} X \to \Sing_{*}(X \times I)\) be a degree
    \(1\) morphism of graded abelian groups such that
    \(\face_{X \times I} T + T \face_X = i_{0\, \#} - i_{1\, \#}\). If we now consider
    the induced map \(\eta_{\#}: \Sing_{*}(X \times I) \to \Sing_{*} X\) we see that
    \begin{align*}
        \eta_{\#} (\face_{X \times I} T + T \face_X)
         & = \eta_{\#} (i_{0\, \#} - i_{1\, \#})
        \\
        \face_X (\eta_{\#} T) + (\eta_{\#} T) \face_X
         & = f_{\#} - g_{\#}.
    \end{align*}
    Thus we see that if \(i_{0\, \#}\) and \(i_{1\, \#}\) are chain-homotopic it
    follows that \(f_{\#}\) and \(g_{\#}\) are also chain-homotopic, as wanted.

    We shall prove the existence of the chain-homotopy
    \(T: \Sing_{\bullet} X \to \Sing_{\bullet}(X \times I)\) via induction. Let
    \(X\) denote any topological space and suppose there exists \(n > 0\) such that
    for any integer \(j < n\) there exists a morphism of abelian groups \((T_X)_j:
    \Sing_j X \to \Sing_{j+1}(X \times I)\) such that
    \begin{equation}\label{eq:chain-homotopy-condition-proof}
        (\face_{X \times I})_{j+1} (T_X)_j + (T_X)_{j-1} (\face_X)_j
        = (i_{0\, \#})_j - (i_{1\, \#})_j
    \end{equation}
    and that for any topological morphism \(h: X \to Z\) the following diagram
    commutes for any \(j < n\):
    \begin{equation}\label{eq:naturality-chain-homotopy-condition-proof}
        \begin{tikzcd}
            \Sing_j X \ar[r, "(T_X)_j"] \ar[d, "h_{\#}"']
            &\Sing_{j+1}(X \times I) \ar[d, "(h \times \Id_I)_{\#}"] \\
            \Sing_j Z \ar[r, "(T_Z)_j"']
            &\Sing_{j+1}(Z \times I)
        \end{tikzcd}
    \end{equation}
    We shall now define a map \((T_X)_n: \Sing_n X \to \Sing_{n+1}(X \times
    I)\). Let \(\phi \in \Sing_n X\) be any singular \(n\)-simplex and consider the
    identity singular \(n\)-simplex \(\Id_n \in \Sing_n \splxtop^n\). Notice that
    since \(\phi_{\#}(\Id_n) = \phi \Id_n = \phi\), then if we define
    \((T_{\splxtop^n})_n: \Sing_n\splxtop^n \to \Sing_{n+1}(\splxtop^n \times I)\)
    following the naturality requirement presented in
    \cref{eq:naturality-chain-homotopy-condition-proof} then we may ask for
    \begin{equation}\label{eq:T-X-n-phi-chain-homotopy}
        T_X\phi = T_X \phi_{\#}\Id_n
        = (\phi \times \Id_I)_{\#}(T_{\splxtop^n} \Id_n).
    \end{equation}
    Hence in order to define \((T_X)_n\) it is sufficient to define
    \((T_{\splxtop^n})_n\).

    For the sake of convenience, we shall adopt the notation
    \(\face \coloneq \face_{\splxtop^n}\) and
    \(\face' \coloneq \face_{\splxtop^n \times I}\). Let
    \(d \in \Sing_n \splxtop^n\) be any singular \(n\)-simplex and consider the
    chain
    \[
        c \coloneq i_{0\, \#} d - i_{1\, \#} d -
        T_{\splxtop^n} \face_n d \in \Sing_n(\splxtop^n \times I).
    \]
    Notice that \(c\) is in fact a cycle, since
    \begin{align*}
        \face'_n c
         & = \face'_n i_{0\, \#} d -
        \face'_n i_{1\, \#} d -
        \face'_n T_{\splxtop^n} (\face_n d) \\
         & = i_{0\, \#} \face_n d -
        i_{1\, \#} \face_n d -
        [
        i_{0\, \#} \face_n d - i_{1\, \#} \face_n d -
        T_{\splxtop^n} \face_{n-1} \face_n d
        ]                                   \\
         & = 0
    \end{align*}
    where we used \cref{eq:chain-homotopy-condition-proof} in order to expand the
    \(\face'_n (T_{\splxtop^n})_{n-1}(\face_n d)\) term. Hence we conclude that
    \(c \in \Cycle_n(\splxtop^n \times I)\). Moreover, from
    \cref{thm:convex-space-Rn-zero-homology} we see that
    \(\Hmlg_n(\splxtop^n \times I) = 0\) thus \(c \in \Bdry_n(\splxtop^n \times I)\) and
    there exists \(b \in \Sing_{n+1}(\splxtop^n \times I)\) such that
    \(\face_{n+1}' b = c\). We shall define
    \[
        T_{\splxtop^n} d \coloneq b.
    \]
    Therefore \(\face_{n+1}' T_{\splxtop^n} d = \face_{n+1}' b = c\) and hence
    \[
        \face'_{n+1} T_{\splxtop^n} d + T_{\splxtop^n} \face_n d
        = i_{0\, \#} d - i_{1\, \#} d.
    \]
    This finishes the final inductive step and proves that
    \cref{eq:T-X-n-phi-chain-homotopy} can be constructed:
    \[
        T_X: \Sing_n X \longrightarrow \Sing_{n+1}(X \times I).
    \]

    From the naturality expressed in
    \cref{eq:naturality-chain-homotopy-condition-proof} we find that for any \(\phi
    \in \Sing_n X\) the maps \(i_{0\, \#}\) and \(i_{1\, \#}\) induce
    \begin{align*}
        i_{0\, \#} \phi
         & = i_{0\, \#} \phi_{\#} \Id_n
        = (\phi \times \Id_I) i_{0\, \#} \Id_n \\
        i_{1\, \#} \phi
         & = i_{1\, \#} \phi_{\#} \Id_n
        = (\phi \times \Id_I) i_{1\, \#} \Id_n \\
    \end{align*}
    Moreover, notice that
    \begin{align*}
        \face_{X \times I} T_X \phi + T_X \face_X \phi
         & = \face_{X \times I} T_X \phi_{\#} \Id_n
        + T_X \face_X \phi_{\#} \Id_n                                                   \\
         & = \face_{X \times I} (\phi \times \Id_I)_{\#} T_X \Id_n
        + T_X \phi_{\#} \face_{\splxtop^n} \Id_n                                        \\
         & = (\phi \times \Id_I)_{\#} \face_{\splxtop^n \times I} T_{\splxtop^n} \Id_n
        + (\phi \times \Id_I)_{\#} T_{\splxtop^n} \face_{\splxtop^n} \Id_n              \\
         & = (\phi \times \Id_I)_{\#} (\face_{\splxtop^n \times I} T_{\splxtop^n} \Id_n
        + T_{\splxtop^n} \face_{\splxtop^n} \Id_n)                                      \\
         & = (\phi \times \Id_I)_{\#} (i_{0\, \#} \Id_n - i_{1\, \#} \Id_n)             \\
         & = i_{0\, \#} \phi - i_{1\, \#} \phi.
    \end{align*}
    Moreover, naturality can be proved as follows:
    \begin{align*}
        T_X \face_X \phi
         & = T_X \face_X \phi_{\#} \Id_n                                      \\
         & = T_X \phi_{\#} \face_{\splxtop^n} \Id_n                           \\
         & = (\phi \times \Id_I)_{\#} T_{\splxtop^n} \face_{\splxtop^n} \Id_n \\
         & = \face_{X \times I} (\phi \times \Id_I)_{\#} T_{\splxtop^n} \Id_n \\
         & = \face_{X \times I} T_X \phi_{\#} \Id_n                           \\
         & = \face_{X \times I} T_X \phi.
    \end{align*}
    This finishes the proof that \(T_X\) is indeed a chain-homotopy between
    \(i_{0\, \#}\) and \(i_{1\, \#}\) as wanted.
\end{proof}

\section{Homotopy Invariance}

\begin{corollary}
    \label{cor:htpy-equiv-iso-homology}
    Let \(f: X \isotoht Y\) be a \emph{homotopy equivalence}, then the induced
    homology map \(f_{*}: \Hmlg_{\bullet}(X) \to \Hmlg_{\bullet}(Y)\) is an
    \emph{isomorphism} of \(\Z\)-graded abelian groups:
    \[
        \Hmlg_{\bullet}(X) \iso \Hmlg_{\bullet}(Y).
    \]
\end{corollary}

\begin{proof}
    Let \(g: Y \to X\) be a homotopy inverse of \(f\). Thus \(g f \simht \Id_X\) and
    \(f g \simht \Id_Y\), thus via
    \cref{thm:homotopic-maps-equal-induced-homology-morphism} we obtain \((g f)_{*}
    = g_{*} f_{*} = \Id_{\Sing_{\bullet} X}\) and \((f g)_{*} = f_{*} g_{*} =
    \Id_{\Sing_{\bullet} Y}\). This shows that \(f_{*}\) and \(g_{*}\) are inverses
    of each other and thus \(f_{*}\) is an isomorphism.
\end{proof}

\begin{corollary}
    \label{cor:retract-monic-homology}
    Let \(A\) be a \emph{retract} of \(X\) and \(\iota: A \emb X\) be the canonical
    inclusion, then \(\iota_{*}: \Hmlg_{\bullet}(A) \to \Hmlg_{\bullet}(X)\) is a
    \emph{split monomorphism} onto a direct summand. If \(A\) is a \emph{deformation
        retract} of \(X\), then \(\iota_{*}\) is an \emph{isomorphism}
\end{corollary}

\begin{proof}
    Since \(\iota\) admits a left inverse \(r: X \to A\) then
    \(r_{*} \iota_{*} = \Id_{\Hmlg_{\bullet}(A)}\) shows that \(\iota_{*}\) is a
    split monomorphism. We now show that \(\iota_{*}\) maps \(\Hmlg_{\bullet}(A)\)
    onto a direct summand of \(\Hmlg_{\bullet}(X)\). Let
    \(G_1 \coloneq \im \iota_{*}\) and \(G_2 \coloneq \ker r_{*}\) be two subgroups
    of \(\Hmlg_{\bullet}(X)\). Let \(\alpha \in \Hmlg_{\bullet}(X)\) be any element
    and notice that we can write
    \(\alpha = \iota_{*} r_{*} \alpha + (\alpha - \iota_{*} r_{*} \alpha)\), where
    \(\iota_{*} r_{*} \alpha \in \im \iota_{*}\) and since \(r_{*}\) is a left
    inverse of \(\iota_{*}\) then
    \(\alpha - \iota_{*} r_{*} \alpha \in \ker r_{*}\). Therefore we have
    \(\Hmlg_{\bullet}(X) = G_1 + G_2\), furthermore, if \(\beta \in G_1 \cap G_2\)
    we have that there exists \(\gamma \in \Hmlg_{\bullet}(A)\) such that
    \(\iota_{*} \gamma = \beta\) and \(r_{*} \beta = 0\), therefore
    \[
        \gamma = r_{*} i_{*} \gamma = r_{*} \beta = 0
    \]
    hence \(\beta = 0\). This shows that \(G_1 \cap G_2 = 0\), thus
    \(G_1 + G_2 = G_1 \oplus G_2\) and \(\Hmlg_{\bullet}(X) = G_1 \oplus G_2\).
\end{proof}

\section{Long Exact Sequence Theorem}

\begin{theorem}
    \label{thm:long-exact-sequence}
    Given a short exact sequence of chain complexes with degree zero chain maps:
    \[
        \begin{tikzcd}
            0 \ar[r]
            &A \ar[r, tail, "i"]
            &B \ar[r, two heads, "j"]
            &C \ar[r]
            &0
        \end{tikzcd}
    \]
    the following long sequence of homology groups is exact:
    \[
        \begin{tikzcd}
            \cdots \ar[r, "i_{*}"]
            &\Hmlg_p(B) \ar[r, "j_{*}"]
            &\Hmlg_p(C) \ar[r, "\delta"]
            &\Hmlg_{p-1}(A) \ar[r, "i_{*}"]
            &\cdots
        \end{tikzcd}
    \]
\end{theorem}

\begin{proof}
    We shall define \(\delta: \Hmlg_\bullet(C) \to \Hmlg_{\bullet}(A)\) with degree
    \(-1\). Let \(c \in \Cycle_p(C)\) and since \(j_p\) is surjective, there exists
    \(b \in B_p\) such that \(j_p b = c\). Furthermore,
    \[
        j_{p-1} \face_p^B b = \face_p^C j_p b = \face_p^C c = 0,
    \]
    from the fact that \(j\) is a chain map---therefore
    \(\face_p^B b \in \ker j_{p-1} = \im i_{p-1}\). Since
    \(\face_p^B b \in \im i_{p-1}\) then there exists a unique \(a \in A_{p-1}\) such
    that \(i_{p-1} a = \face_p^B b\)---the uniqueness of \(a\) comes from the fact
    that \(i\) is an injective map. Define the map
    \(\delta_p: \Hmlg_p(C) \to \Hmlg_{p-1}(A)\) to take
    \(\delta_p[c] \coloneq [a]\). We need to show that \(\delta_p\) is well defined, which requires
    us to prove that the mapping \([c] \mapsto [a]\) is independent of the choice of
    \(b\) and \(c\):
    \begin{itemize}\setlength\itemsep{0em}
        \item Let \(b' \in B_p\) be such that \(j_p b' = c\), thus \(b' - b \in \ker j_p\)
              and since \(\ker j_p = \im i_p\) it follows that there exists \(a' \in A_p\) for
              which \(i_p a' = b' - b\). Therefore
              \[
                  \face_p^B b'
                  = \face_p^B b + \face_p^B i_p a'
                  = i_p a + i_p \face_p^A a'
                  = i_p(a + \face_p^A a'),
              \]
              which shows that \([a + \face_p^A a'] = [a]\), therefore the construction is
              independent of the choice of \(b\).

        \item Let \([c'] = [c]\) so that \(c' - c \in \Bdry_p(C)\), and choose
              \(\alpha \in C_{p+1}\) such that \(c' - c = \face_{p+1}^C \alpha\). By the surjectiveness
              of \(j\), choose \(b' \in B_{p+1}\) for which \(\alpha = j_{p+1} b'\). Therefore
              \[
                  c' = c + \face_{p+1}^C \alpha = c + \face_{p+1}^C j_{p+1} b'
                  = j_p(b) + j_p(\face_{p+1}^B b')
                  = j_p(b + \face_{p+1}^B b').
              \]
              Moreover, one also has
              \[
                  \face_p^B(b + \face_{p+1}^B b')
                  = \face_p^B b + \face_p^B \face_{p+1}^B b'
                  = \face_p^{B} b,
              \]
              therefore \(i_{p-1} a = \face_p^B b = \face_p^B(b + \face_{p+1}^B b')\).  This
              shows us that the construction is agnostic of the choice of representative of
              the class \([c]\).
    \end{itemize}
    It remains for us to prove that the resulting sequence of homology groups is
    exact:
    \begin{itemize}\setlength\itemsep{0em}
        \item (\(\im i_{*} = \ker j_{*}\)) Notice that by functoriality we have
              \(j_{*} i_{*} = (j i)_{*} = 0\) since \(\im i \subseteq \ker j\)---this shows that
              \(\im i_{*} \subseteq \ker j_{*}\). On the other hand, if
              \([b] \in \ker j_{*\, p}\) is any element, then \(j_p b \in \Bdry_p(C)\) and there
              exists \(c \in C_{p+1}\) such that \(j_p b = \face_{p+1}^C c\). Since \(j\) is
              surjective, there also exists \(b' \in B_{p+1}\) such that \(j_{p+1} b' =
              c\). Notice that
              \[
                  j_p\face_{p+1}^B b' = \face_p^B j_{p+1} b' = \face_p^B c = j_p b,
              \]
              therefore \(b - \face_{p+1}^B b' \in \ker j_p = \im i_p\). Since \(i\) is
              injective it follows that there exists a unique \(a \in A_p\) such that
              \(i a = b - \face_{p+1}^B b'\). Notice that since \([b] \in \Hmlg_p(B)\) then in
              particular \(b \in \Cycle_p(B)\), therefore
              \[
                  i_{p-1} \face_p^A a
                  = \face_p^B i_p a
                  =  \face_p^B(b - \face_{p+1}^B b')
                  = \face_p^B b - \face_p^B \face_{p+1}^B b'
                  = 0.
              \]
              From the injectivity of \(i\) we conclude that \(\face_p^A a = 0\), therefore
              \(a \in Z_p(A)\) and hence
              \[
                  [b] = [b - \face_{p+1}^B b'] = [i_p a] = i_{*\, p}[a],
              \]
              which proves that \(\ker j_{*\, p} \subseteq \im i_{*\, p}\) and in general
              \(\ker j_{*} \subseteq \im i_{*}\).

        \item (\(\im j_{*} = \ker \delta\)) By the definition of \(\delta\) we have
              \(\delta j_{*} = 0\) thus \(\im j_{*} \subseteq \ker \delta\). Let
              \([c] \in \ker \delta_p\) then \(\delta_p[c] = [a] = 0\), which means that
              \(a \in \Bdry_{p-1}(A)\) and therefore there exists \(a' \in A_p\) such that
              \(a = \face_p^A a'\). Since \(j\) is surjective, let \(b \in B_p\) be such that
              \(j_p b = c\), then
              \[
                  j_p(b - i_p a') = j_p b - j_p i_p a' = j_p b = c,
              \]
              moreover we also have
              \[
                  \face_p^B(b - i_p a')
                  = \face_p^B b - \face_p^B i_p a'
                  = \face_p^B b - i_{p-1} \face_p^A a'
                  = \face_p^B b - i_{p-1} a
                  = 0
              \]
              therefore \(b - i_p a' \in \Cycle_p(B)\) and hence
              \[
                  j_{*\, p}[b - i_p a'] = [j_p(b - i_p a')] = [c],
              \]
              showing that \(\ker \delta \subseteq \im j_{*}\).

        \item (\(\im \delta = \ker i_{*}\))  By the definition of \(\delta\), if \(a \in \im \delta_p\)
              with \(\delta[c] = [a]\), and both \(j_p b = c\) and \(i_{p-1} a = \face_p^B b\) then
              \[
                  i_{*\, p-1}[a] = [i_{p-1} a] = [\face_p^B b] = 0,
              \]
              thus \(\im \delta \subseteq \ker i_{*}\). Moreover, if \([a'] \in \ker i_{*\, p-1}\), let
              \(i_{p-1} a = \face_p^B b\) for some \(b \in B_p\). Since
              \[
                  \face_p^C j_p b = j_{p-1} \face_p^B b = j_{p-1} i_{p-1} a = 0,
              \]
              then \(j_p b \in \Cycle_p(C)\) and hence \(\delta[j b] = [a]\). Therefore
              \(\ker i_{*} \subseteq \im \delta\).
    \end{itemize}
\end{proof}

\begin{proposition}
    \label{prop:long-exact-sequence-construction-is-natural}
    The long exact sequence induced by short exact sequences of chain complexes is
    natural. In other terms, given a commutative diagram with exact rows:
    \[
        \begin{tikzcd}
            0 \ar[r]
            &C \ar[r, "f", tail] \ar[d, "\alpha"']
            &D \ar[r, "g", two heads] \ar[d, "\beta"]
            &E \ar[r] \ar[d, "\gamma"]
            &0
            \\
            0 \ar[r]
            &C' \ar[r, "f'"', tail]
            &D' \ar[r, "g'"', two heads]
            &E' \ar[r]
            &0
        \end{tikzcd}
    \]
    then the following diagram with long exact rows commutes:
    \[
        \begin{tikzcd}
            \cdots \ar[r]
            &\Hmlg_n(D) \ar[r, "g_{*}", tail] \ar[d, "\beta_{*}"']
            &\Hmlg_n(E) \ar[r, "\delta_n", two heads] \ar[d, "\gamma_{*}"]
            &\Hmlg_{n-1}(C) \ar[r, "f_{*}"] \ar[d, "\alpha_{*}"]
            &\Hmlg_{n-1}(D) \ar[r] \ar[d, "\beta_{*}"]
            &\cdots
            \\
            \cdots \ar[r]
            &\Hmlg_n(D') \ar[r, "g_{*}'"', tail]
            &\Hmlg_n(E') \ar[r, "\delta_n"', two heads]
            &\Hmlg_{n-1}(C') \ar[r, "f_{*}'"']
            &\Hmlg_{n-1}(D') \ar[r]
            &\cdots
        \end{tikzcd}
    \]
\end{proposition}


\section{Reduced Homology}

\begin{definition}[Reduced homology]
    \label{def:reduced-homology}
    Let \(X\) be a non-empty topological space, and define a surjective morphism
    \(\varepsilon: \Sing_0 X \epi \Z\) given by
    \(\sum_j n_j \phi_j \mapsto \sum_j n_j\). Notice that \(\im \face_1 \subseteq \ker \varepsilon\) since given any
    \(\tau \in \Sing_1 X\) we have
    \[
        \varepsilon \face_1 \tau = \varepsilon(\tau(1) - \tau(0)) = 1 - 1 = 0.
    \]
    Therefore the sequence
    \[
        \begin{tikzcd}
            \cdots \ar[r]
            &\Sing_2 X \ar[r, "\face_2"]
            &\Sing_1 X \ar[r, "\face_1"]
            &\Sing_0 X \ar[r, two heads, "\varepsilon"]
            &\Z
            \ar[r]
            &0
        \end{tikzcd}
    \]
    is a chain complex---in fact, we call such complex the \emph{augmented chain
        complex} of \(X\). It is to be noticed that from the isomorphism theorem for
    groups there exists a unique morphism \(\varepsilon_{*}: \Hmlg_0(X) \to \Z\) such that the
    following diagram commutes
    \[
        \begin{tikzcd}
            \Sing_0 X \ar[rr, "\varepsilon"] \ar[d, two heads]
            &&\Z \\
            \Hmlg_0(X) = \frac{\Sing_0 X}{\im \face_1}
            \ar[rru, dashed, bend right, "\varepsilon_{*}"']
            &&
        \end{tikzcd}
    \]
    If we now define a \(\Z\)-graded group \(\rHmlg_{\bullet}(X)\) as follows: for each
    \(p > 0\) let \(\rHmlg_p(X) \coloneq \Hmlg_p(X)\), while
    \(\rHmlg_0(X) \coloneq \ker \varepsilon/\im \face_1\). The group \(\rHmlg_p(X)\) is called the
    \emph{\(p\)-th reduced homology group} of \(X\).

    In particular we know that
    \(\ker \varepsilon_{*} = \ker \varepsilon/\im \face_1\), therefore we may define a choice-dependent
    isomorphism
    \[
        \rHmlg_0(X) \oplus \Z \iso \Hmlg_0(X)
    \]
    as follows: given \(\psi \in \Sing_0 X\), define the mappings
    \(([\sum_j n_j \phi_j], m) \mapsto [\sum_j n_j \phi_j + m \psi]\) together with an inverse
    \([\sum_j n_j \phi_j] \mapsto ([\sum_j n_j \phi_j - \sum_j n_j \psi], \sum_j n_j)\). Another way to see the
    existence of this isomorphism is to notice that since \(\im \varepsilon_{*} = \Z\) and
    \(\ker \varepsilon_{*} = \rHmlg_0(X)\) then there exists a short exact sequence
    \[
        \begin{tikzcd}
            0 \ar[r]
            &\rHmlg_0(X) \ar[r, hook]
            &\Hmlg_0(X) \ar[r, two heads, "\varepsilon"]
            &\Z \ar[r]
            &0
        \end{tikzcd}
    \]
    and since \(\Z\) is free then the sequence splits and there exists an
    isomorphism \(\Hmlg_0(X) \iso \rHmlg_0(X) \oplus \Z\) as expected.
\end{definition}

\section{Mayer-Vietoris Sequence}

\subsection{Barycentric Subdivision}

\todo[inline]{To be added}

\subsection{Mayer-Vietoris Theorem}

\begin{definition}
    \label{def:cover-singular-simplicial-complex}
    Let \(X\) be a topological space and \(\mathcal{U}\) be an open cover of \(X\). Denote by
    \(\Sing_{\bullet}^{\mathcal{U}} X\) the \(\Z\)-graded abelian group given by groups
    \(\Sing_n^{\mathcal{U}} X\) composed of singular \(n\)-simplices\footnote{A simplex with
        such property is called \(\mathcal{U}\)-small.} \(\phi: \splxtop^n \to X\) such that
    \(\im \phi \subseteq U\) for some \(U \in \mathcal{U}\)---for each \(n \in \Z\). Since
    \(\im \face^j \phi \subseteq \im \phi\), the boundary map
    \[
        \face: \Sing_{\bullet}^{\mathcal{U}} X \to \Sing_{\bullet}^{\mathcal{U}} X
    \]
    is well defined and has degree \(-1\).

    Given another topological space \(Y\) together with an open covering
    \(\mathcal{V}\) and a continuous map \(f: X \to Y\) for which each
    \(U \in \mathcal{U}\) has image \(f U \subseteq V\) for some
    \(V \in \mathcal{V}\), we have an induced chain map
    \(f_{\#}: \Sing_{\bullet}^{\mathcal{U}} X \to \Sing_{\bullet}^{\mathcal{V}} Y\).
\end{definition}

\begin{theorem}
    \label{thm:iso-hmlg-groups-cover}
    Let \(\mathcal{U}\) be a collection of subsets of \(X\) such that the interior of
    \(\mathcal{U}\) is an open cover for \(X\), then the canonical inclusion map
    \(\iota: \Sing_{\bullet}^{\mathcal{U}} X \to \Sing_{\bullet} X\) induces an isomorphism
    \[
        \iota_{*}: \Hmlg_{\bullet}(\Sing_{\bullet}^{\mathcal{U}} X) \isoto \Hmlg_{\bullet}(X).
    \]
\end{theorem}

\begin{theorem}
    \label{thm:mayer-vietoris}
    Let \(X\) be a space and \(U, V \subseteq X\) be subsets such that
    \(\Int U \cup \Int V = X\) with inclusion maps:
    \[
        \begin{tikzcd}
            &U \ar[rd, "k"] & \\
            U \cap V \ar[ur, "i"] \ar[rd, "j"'] & &U \cup V = X \\
            &V \ar[ru, "\ell"']&
        \end{tikzcd}
    \]
    Then there exists a long exact sequence of homology
    groups, called the \emph{Mayer-Vietoris sequence}, of the form
        {\small
            \[
                \begin{tikzcd}
                    \cdots \ar[r]
                    &\Hmlg_n(U \cap V) \ar[r, "i_* \oplus j_{*}"]
                    &\Hmlg_n(U) \oplus \Hmlg_n(V) \ar[r, "k_{*} - \ell_{*}"]
                    &\Hmlg_n(X) \ar[r, "\delta"]
                    &\Hmlg_{n-1}(U \cap V) \ar[r]
                    &\cdots
                \end{tikzcd}
            \]
        }

    Moreover, this sequence is natural---that is, given subsets
    \(U', V' \subseteq X'\) with \(\Int U' \cup \Int V' = X\) and a continuous map
    \(f: X \to X'\) for which \(f U \subseteq U'\) and \(f V \subseteq V'\) then the following diagram
    commutes:
    {\small
    \[
        \begin{tikzcd}
            \cdots \ar[r]
            &\Hmlg_n(U \cap V) \ar[r]
            \ar[d, "f_{*}"]
            &\Hmlg_n(U) \oplus \Hmlg_n(V) \ar[r]
            \ar[d, "f_{*} \oplus f_{*}"]
            &\Hmlg_n(X) \ar[r]
            \ar[d, "f_{*}"]
            &\Hmlg_{n-1}(U \cap V) \ar[r]
            \ar[d, "f_{*}"]
            &\cdots
            \\
            \cdots \ar[r]
            &\Hmlg_n(U' \cap V') \ar[r]
            &\Hmlg_n(U') \oplus \Hmlg_n(V') \ar[r]
            &\Hmlg_n(X') \ar[r]
            &\Hmlg_{n-1}(U \cap V) \ar[r]
            &\cdots
        \end{tikzcd}
    \]
    }
\end{theorem}

\begin{proof}
    For convenience, let \(\mathcal{U} \coloneq \{U, V\}\). Let \(S_U\) and \(S_V\) be the sets
    consisting of all singular \(n\)-simplices in \(U\) and \(V\), respectively. If
    \(\operatorname F: \Set \to \Ab\) denotes the free abelian group functor, then it
    follows that
    \[
        \Sing_n (U \cap V) = \operatorname F(S_U \cap S_V)
        \quad
        \text{ and }
        \quad
        \Sing_n^{\mathcal{U}} X = \operatorname F(S_U \cup S_V).
    \]
    Define a morphism of groups
    \[
        h: \operatorname F(S_U) \oplus \operatorname F(S_V) \epi \operatorname F(S_U \cup S_V)
    \]
    to be given by \((\phi, \psi) \mapsto \phi - \psi\), which gives an epimorphism. Further,
    define a monomorphism of groups
    \[
        g: \operatorname F(S_U \cap S_V) \mono \operatorname F(S_U) \oplus \operatorname F(S_V)
    \]
    mapping \(\tau \mapsto (\tau, \tau)\). Notice that from construction we have
    \(h g = 0\), hence \(\im g \subseteq \ker h\).

    On the other hand, let
    \((\Phi, \Psi) \coloneq (\sum_{\phi \in S_U} n_{\phi} \phi, \sum_{\psi \in S_{V}} m_{\psi} \psi) \in \ker h\), which is the
    case if and only if for each \(\phi \in S_U\) associated with a non-zero coefficient
    \(n_{\phi}\) there exists \(\psi \in S_V\) with \(m_{\psi} = n_{\phi}\) and
    \(\psi = \phi\)---this construction should be bijective in the singular simplices that
    have non-zero coefficient in order to obtain \(\Phi + \Psi = 0\). From this
    consideration, we conclude that for \((\Phi, \Psi)\) to be an element of the kernel of
    \(h\) it must be the case that each non-zero simplex lies in the intersection
    \(S_U \cap S_V\), therefore it is clear that \(\ker h \subseteq \im g\). This shows us that
    \begin{equation}\label{eq:chain-cplx-short-exact-seq-mv-seq}
        \begin{tikzcd}
            0 \ar[r]
            &\Sing_n(U \cap V)
            \ar[r, tail, "g"]
            &\Sing_n U \oplus \Sing_n V
            \ar[r, two heads, "h"]
            &\Sing_n(U \cup V)
            \ar[r]
            &0
        \end{tikzcd}
    \end{equation}
    is a short exact sequence, which can be extended as a short exact sequence of
    chain complexes and degree zero chain maps \(g\) and \(h\).

    Using the long exact sequence theorem on
    \cref{eq:chain-cplx-short-exact-seq-mv-seq} we obtain an exact sequence of
    homology groups
        {\small
            \[
                \begin{tikzcd}
                    \cdots \ar[r]
                    &\Hmlg_n(U \cap V) \ar[r, "g_{*}"]
                    &\Hmlg_n(U) \oplus \Hmlg_n(V) \ar[r, "h_{*}"]
                    &\Hmlg_n^{\mathcal{U}}(X) \ar[r, "\delta"]
                    &\Hmlg_{n-1}(U \cap V) \ar[r]
                    &\cdots
                \end{tikzcd}
            \]
        }
    Now using the isomorphism from \cref{thm:iso-hmlg-groups-cover} we obtain the
    desired long exact sequence.
\end{proof}

\begin{corollary}
    \label{cor:mayer-vietoris-for-reduced-homology}
    Let \(X\) be a non-empty topological space, and \(U, V \subseteq X\) be intersecting
    subsets such that \(\Int U \cup \Int V = X\). There exists an exact sequence
    \[
        \begin{tikzcd}[column sep=small]
            \cdots \ar[r]
            &\rHmlg_p(U \cap V) \ar[r]
            &\rHmlg_p(U) \oplus \rHmlg_p(V) \ar[r]
            &\rHmlg_p(X) \ar[r]
            &\rHmlg_{p-1}(U \cap V) \ar[r]
            &\cdots
        \end{tikzcd}
    \]
    and the reduced homology makes this sequence end:
    \[
        \begin{tikzcd}
            \cdots \ar[r]
            &\rHmlg_0(U) \oplus \rHmlg_0(V) \ar[r, two heads]
            &\rHmlg_0(X) \ar[r]
            &0
        \end{tikzcd}
    \]
\end{corollary}

\subsection{Applications of The Mayer-Vietoris Sequence}

\begin{example}[The circle]
    \label{exp:homology-of-the-circle}
    Consider the circle \(S^1\) and denote by \(s\) and \(n\) its south and north
    poles, respectively. Define subsets \(U \coloneq S^1 \setminus s\) and
    \(V \coloneq S^1 \setminus n\). First let us calculate the first homology group of
    \(S^1\): to that end, we can use the Mayer-Vietoris sequence for the triple
    \((S^1, U, V)\) to obtain an exact sequence
    \[
        \begin{tikzcd}
            \Hmlg_1(U) \oplus \Hmlg_1(V) \ar[r, "k_{*} - \ell_{*}"]
            &\Hmlg_1(S^1) \ar[r, "\delta"]
            &\Hmlg_0(U \cap V) \ar[r, "i_{*} \oplus j_{*}"]
            &\Hmlg_0(U) \oplus \Hmlg_0(V)
        \end{tikzcd}
    \]
    Notice that both \(U\) and \(V\) are contractible spaces, therefore they have
    the same homotopy as the single point space, that is
    \[
        \Hmlg_p(U) = \Hmlg_p(V) =
        \begin{cases}
            \Z, & \text{if } p = 0 \\
            0,  & \text{otherwise}
        \end{cases}
    \]
    Thus \(\Hmlg_1(U) \oplus \Hmlg_1(V) = 0\)---implying that \(\delta\) is a monomorphism---and
    we also have \(\Hmlg_0(U) \oplus \Hmlg_0(V) \iso \Z \oplus \Z\). On the other hand, the
    set \(U \cap V\) is homotopic to a space consisting of two points---that is,
    \(U \cap V \isoht \{*\} \disj \{*\}\)---thus in particular we have an isomorphism
    \(\Hmlg_0(U \cap V) \iso \Z \oplus \Z\) by mapping
    \(a \phi + b \psi \mapsto (a, b)\). In summary, we obtain the following exact
    sequence
    \[
        \begin{tikzcd}
            0 \ar[r]
            &\Hmlg_1(S^1) \ar[r, "\delta", tail]
            &\Z \oplus \Z \ar[r, "i_{*}' \oplus j_{*}'"]
            &\Z \oplus \Z
        \end{tikzcd}
    \]
    where both \(i_{*}', j_{*}': \Z \oplus \Z \para \Z\) map
    \((a, b) \mapsto a + b\), since \(i_{*}\) and \(j_{*}\) are inclusions.  Given an
    element \(a \phi + b \psi \in \Hmlg_0(U \cap V)\), we find that
    \(a \phi + b \psi\) corresponds to an element of the kernel of
    \(i_{*}' \oplus j_{*}'\) if and only if \(a = -b\). Hence
    \(\ker(i_*' \oplus j_{*}')\) is a subgroup of \(\Z \oplus \Z\) generated by
    elements of the form \((a, -a)\)---that is,
    \[
        \ker(i_{*}' \oplus j_{*}') = \im \delta \iso \Z.
    \]
    Since \(\delta\) is injective, then \(\Hmlg_1(S^1) \iso \im \delta\), therefore
    \(\Hmlg_1(S^1) \iso \Z\).

    Moreover, for the case of \(p > 1\) we have an exact sequence
    \[
        \begin{tikzcd}
            \Hmlg_p(U) \oplus \Hmlg_p(V) \ar[r, "k_{*} - \ell_{*}"]
            &\Hmlg_p(S^1) \ar[r, "\delta"]
            &\Hmlg_{p-1}(U \cap V)
        \end{tikzcd}
    \]
    however from previous considerations we know that for \(p > 1\) we have both
    \(\Hmlg_p(U) = \Hmlg_p(V) = 0\) and \(\Hmlg_{p-1}(U \cap V) = 0\). This shows that
    \(\ker \delta = \Hmlg_p(S^1)\) and since \(\im(k_{*} - \ell_{*}) = 0\) it follows from
    exactness that \(\Hmlg_p(S^1) = 0\). This results shows us that:
    \[
        \Hmlg_p(S^1) =
        \begin{cases}
            \Z, & \text{if } p = 1 \\
            0,  & \text{otherwise}
        \end{cases}
    \]
\end{example}

\begin{example}[\(n\)-spheres]
    \label{exp:homology-of-sphere}
    Let's consider the general case of an \(n\)-sphere \(S^n \subseteq \R^{n+1}\), where
    \(n > 0\). Let \(n, s \in S^n\) be the north and south poles, respectively, and
    define subsets \(U \coloneq S^n \setminus n\) and
    \(V \coloneq S^n \setminus s\). Via the stereographic projection we know that there are
    topological isomorphisms
    \[
        U \iso \R^n \iso V \quad \text{ and } \quad U \cap V \iso \R^n \setminus 0,
    \]
    furthermore, \(S^{n-1}\) is a deformation retract of \(\R^n \setminus 0\), hence
    \(U \cap V \isoht S^{n-1}\). Therefore we have, for any \(p\):
    \[
        \begin{tikzcd}
            \Hmlg_p(U) \oplus \Hmlg_p(V) \ar[r, "k_{*} - \ell_{*}"]
            \ar[d, "\dis"]
            &\Hmlg_p(S^n) \ar[r, "\delta"]
            \ar[d, equals]
            &\Hmlg_{p-1}(U \cap V) \ar[r, "i_{*} \oplus j_{*}"]
            \ar[d, "\dis"]
            &\Hmlg_{p-1}(U) \oplus \Hmlg_{p-1}(V)
            \ar[d, "\dis"]
            \\
            \Hmlg_p(\R^n) \oplus \Hmlg_p(\R^n) \ar[r, "h_{*}"']
            &\Hmlg_p(S^n) \ar[r, "d_{*}"']
            &\Hmlg_{p-1}(S^{n-1}) \ar[r, "g_{*}"']
            &\Hmlg_{p-1}(\R^n) \oplus \Hmlg_{p-1}(\R^n)
        \end{tikzcd}
    \]
    Since \(\R^n\) is contractible then for \(p > 0\) we have \(\Hmlg_p(\R^n) = 0\),
    therefore both end terms of the above sequence are zero. By exactness of the
    sequence, this implies that for \(p > 1\) the map \(\delta\) is an isomorphism
    \(\Hmlg_p(S^n) \iso \Hmlg_{p-1}(S^{n-1})\). Therefore, for \(p = n\) we find
    recursively that \(\Hmlg_n(S^n) \iso \Hmlg_1(S^1) \iso \Z\). In the case of
    \(p = 1\) and \(n > 1\) we have that both \(d_{*}\) and \(g_{*}\) are
    monomorphisms, thus \(\Hmlg_1(S^n) \iso 0\)---this also follows from Hurewicz's
    theorem. Otherwise, if \(p > 1\) and \(p \neq n\) then \(\Hmlg_p(S^n)\). This
    settles the following: for \(n > 0\)
    \[
        \Hmlg_p(S^n) =
        \begin{cases}
            \Z, & \text{if } p \in \{0, n\} \\
            0,  & \text{otherwise}
        \end{cases}
    \]
    for the case \(n = 0\), \(S^0\) consists of two points---therefore has two path
    connected components, hence \(\Hmlg_0(S^0) \iso \Z \oplus \Z\), thus
    \[
        \Hmlg_p(S^0) =
        \begin{cases}
            \Z \oplus \Z, & \text{if } p = 0 \\
            0,            & \text{otherwise}
        \end{cases}
    \]
\end{example}

\begin{example}[Torus]
    \label{exp:torus-homology}
    Consider the torus \(T\). We shall calculate the homology of \(T\) using two
    different coverings \(U\) and \(V\):
    \begin{enumerate}[(a)]\setlength\itemsep{0em}
        \item Let \(U\) and \(V\) be intersecting subsets of \(T\), where \(U\) is one
              half of the torus and \(V\) is the other half---therefore:
              \[
                  U \iso V \isoht S^1 \quad \text{ and } \quad U \cap V = S^1 \disj S^1.
              \]
              Since \(T\) is path-connected, we know that \(\Hmlg_0(T) = \Z\).
              \begin{itemize}\setlength\itemsep{0em}
                  \item For the case where \(p > 2\) we have
                        \[
                            \begin{tikzcd}
                                \Hmlg_p(U) \oplus \Hmlg_p(V) \ar[d, "\dis"]
                                \ar[r]
                                &\Hmlg_p(T) \ar[d, equals]
                                \ar[r]
                                &\Hmlg_{p-1}(U \cap V) \ar[d, "\dis"]
                                \\
                                0 \ar[r] &\Hmlg_p(T) \ar[r] &0
                            \end{tikzcd}
                        \]
                        hence from exactness we obtain \(\Hmlg_p(T) = 0\).

                  \item For the case \(p = 2\) we have
                        \[
                            \begin{tikzcd}
                                \Hmlg_2(U) \oplus \Hmlg_2(V)
                                \ar[r, "k_{*} - \ell_{*}"]
                                \ar[d, equals]
                                &\Hmlg_2(T)
                                \ar[r, "\delta"]
                                \ar[d, equals]
                                &\Hmlg_1(U \cap V)
                                \ar[d, "\dis"]
                                \ar[r, "i_{*} \oplus j_{*}"]
                                &\Hmlg_1(U) \oplus \Hmlg_1(V)
                                \ar[d, "\dis"]
                                \\
                                0 \ar[r] &\Hmlg_2(T) \ar[r] &\Z \oplus \Z \ar[r] &\Z \oplus \Z
                            \end{tikzcd}
                        \]
                        Since the sequence is exact, then \(\Hmlg_2(T) \mono \Z \oplus \Z\) is a
                        monomorphism. Let's analyse the map \(i_{*} \oplus j_{*}\): let
                        \(u, v \in Z_1(U \cap V)\) be equators, where \(u\) sits in one of the cylinders
                        while \(v\) in the other. Notice however that
                        \[
                            (i_{*} \oplus j_{*})[\alpha]_{U \cap V} = (\alpha, \beta) = (i_{*} \oplus j_{*})[\beta]_{U \cap V}
                        \]
                        since the classes of \(\alpha\) and \(\beta\) are equal in both \(\Hmlg_1(U)\) and
                        \(\Hmlg_1(V)\). From injectivity of \(\delta\) we know that
                        \(\Hmlg_2(T) \iso \im \delta\), moreover exactness implies in
                        \(\im \delta = \ker(i_{*} \oplus j_{*})\). Notice that
                        \[
                            \ker(i_{*} \oplus j_{*}) = \Z[\alpha - \beta]_{U \cap V} \iso \Z,
                        \]
                        therefore \(\Hmlg_2(T) \iso \Z\).

                  \item For the case where \(p = 1\) we have {\small
                                \[
                                    \begin{tikzcd}
                                        \Hmlg_1(U \cap V) \ar[r, "{(i_{*} \oplus j_*)_1}"] \ar[d, "\dis"]
                                        &{\Hmlg_1(U) \oplus \Hmlg_1(V)} \ar[r, "{k_{*} - \ell_{*}}"] \ar[d, "\dis"]
                                        &\Hmlg_1(T) \ar[r, "\delta"] \ar[d, equals]
                                        &\Hmlg_0(U \cap V) \ar[r, "{(i_* \oplus j_{*})_0}"] \ar[d, "\dis"]
                                        &\Hmlg_0(U) \oplus \Hmlg_0(V) \ar[d, "\dis"]
                                        \\
                                        \Z \oplus \Z \ar[r]
                                        &\Z \oplus \Z \ar[r]
                                        &\Hmlg_1(T) \ar[r]
                                        &\Z \oplus \Z \ar[r]
                                        &\Z \oplus \Z
                                    \end{tikzcd}
                                \]
                            } In order to analyse the map \((i_* \oplus j_{*})_0\) we can take distinct points
                        \(u, v \in U \cap V\)---each contained in one of the two connected components of
                        \(U \cap V\)---so that \([u]_U\) generates \(\Hmlg_0(U)\) while \([v]_V\) generates
                        \(\Hmlg_0(V)\). It follows that
                        \[
                            \Hmlg_0(U \cap V) \iso \Z [u]_{U \cap V} \oplus \Z [v]_{U \cap V}.
                        \]
                        Notice that since \([u]_U = [v]_U\) and \([u]_V = [v]_V\) then
                        \[
                            (i_{*} \oplus j_{*})_0 [u]_{U \cap V} = (i_{*} \oplus j_{*})_0 [v]_{U \cap V},
                        \]
                        therefore, \(\ker(i_{*} \oplus j_{*})_0 = \Z[u - v]_{U \cap V} \iso \Z\) and from
                        exactness we have
                        \[
                            \im \delta = \ker(i_{*} \oplus j_{*})_0 \iso \Z.
                        \]
                        It should be noted that the long sequence can be split into short exact
                        sequences, for instance, we are interested in the following induced short
                        sequence:
                        \[
                            \begin{tikzcd}
                                0 \ar[r]
                                &\im(k_{*} - \ell_{*})_1 \ar[r] \ar[d, equals]
                                &\Hmlg_1(T) \ar[r] \ar[d, equals]
                                &\im \delta \ar[d, equals] \ar[r]
                                &0
                                \\
                                0 \ar[r]
                                &\im(k_{*} - \ell_{*})_1 \ar[r]
                                &\Hmlg_1(T) \ar[r]
                                &\ker(i_{*} \oplus j_{*})_0
                                \ar[r]
                                &0
                            \end{tikzcd}
                        \]
                        Moreover, we have
                        \[
                            \im (k_{*} - \ell_{*}) \iso \frac{\Z \oplus \Z}{\ker (k_{*} - \ell_{*})}
                            \iso \frac{\Z \oplus \Z}{\im(i_{*} \oplus j_{*})_1}
                            \iso \frac{\Z \oplus \Z}{\Z}
                            \iso \Z,
                        \]
                        therefore the short exact sequence above becomes the split sequence
                        \[
                            \begin{tikzcd}
                                0 \ar[r] &\Z \ar[r] &\Hmlg_1(T) \ar[r] &\Z \ar[r] &0
                            \end{tikzcd}
                        \]
                        which implies in \(\Hmlg_1(T) \iso \Z \oplus \Z\).
              \end{itemize}
              This can be summarised as
              \[
                  \Hmlg_p(T) =
                  \begin{cases}
                      \Z,           & \text{if } p = 0 \\
                      \Z \oplus \Z, & \text{if } p = 1 \\
                      \Z,           & \text{if } p = 2 \\
                      0,            & \text{otherwise}
                  \end{cases}
              \]

        \item Let \(p \in T\) be any point. Consider the cover composed of
              \(U \coloneq T \setminus p\) and \(V\) being a disk around \(p\). Notice that
              \(V\) is topologically isomorphic to a disk in \(\R^2\), which is a convex
              euclidean set, thus
              \[
                  \Hmlg_p(V) =
                  \begin{cases}
                      \Z, & \text{if } p = 0 \\
                      0,  & \text{otherwise}
                  \end{cases}
              \]
              Also notice that \(U \cap V = V \setminus p\), which is isomorphic to \(S^1\), hence
              \(\Hmlg_p(U \cap V) \iso \Hmlg_p(S^1)\). The reader should now draw for himself
              the gluing diagram of \(T\) without a point, noticing that we can
              homotopically remove the inside face of the diagram, remaining only the gluing
              boundaries, which after being glued we see that
              \(T \setminus p \isoht S^1 \vee S^1\), hence
              \(\Hmlg_p(U) \iso \Hmlg_p(S^1 \vee S^1)\). Using the the Hurewicz's theorem we
              obtain:
              \[
                  \Hmlg_1(U)
                  \iso \Hmlg_1(S^1 \vee S^1)
                  \iso \pi_1^{\text{Ab}}(S^1 \vee S^1)
                  = (\Z * \Z)^{\text{Ab}}
                  \iso \Z \oplus \Z.
              \]
              Notice that for \(p \geq 3\) we have
              \[
                  \begin{tikzcd}
                      0 = \Hmlg_p(U) \oplus \Hmlg_p(V) \ar[r]
                      &\Hmlg_p(T) \ar[r]
                      &\Hmlg_{p-1}(U \cap V) = 0
                  \end{tikzcd}
              \]
              therefore from exactness we get \(\Hmlg_p(T) = 0\). For the case where
              \(p = 2\) we have:
              \[
                  \begin{tikzcd}[column sep=small]
                      \Hmlg_2(U) \oplus \Hmlg_2(V)
                      \ar[r] \ar[dd, equals]
                      &\Hmlg_2(T)
                      \ar[r] \ar[dd, equals]
                      &\Hmlg_1(U \cap V)
                      \ar[r] \ar[dd, "\dis"]
                      &\Hmlg_1(U) \oplus \Hmlg_1(V)
                      \ar[d, "\dis"]
                      \\
                      &&
                      &(\Z \oplus \Z) \oplus 0
                      \ar[d, "\dis"]
                      \\
                      0
                      \ar[r]
                      &\Hmlg_2(T)
                      \ar[r]
                      &\Z
                      \ar[r]
                      &\Z \oplus \Z
                  \end{tikzcd}
              \]
              Since \(H_1(V)\) is zero, then in particular the map
              \(\Hmlg_1(U \cap V) \to \Hmlg_1(U) \oplus \Hmlg_1(V)\) sends
              \([\phi] \mapsto (0, 0)\), therefore the induced map \(\Z \to \Z \oplus \Z\) is zero, that is,
              has kernel \(\Z\). Since the sequence is exact, the image of
              \(\Hmlg_2(T) \to \Z\) equals \(\Z\), therefore being both an injection and
              surjection, proving that \(\Hmlg_2(T) \iso \Z\).

              Now for \(p = 1\):
              \[
                  \begin{tikzcd}[column sep=small]
                      \Hmlg_1(U \cap V)
                      \ar[r] \ar[dd, "\dis"]
                      &\Hmlg_1(U) \oplus \Hmlg_1(V)
                      \ar[r] \ar[d, "\dis"]
                      &\Hmlg_1(T) \ar[r]
                      \ar[dd, equals]
                      &\Hmlg_0(U \cap V)
                      \ar[dd, "\dis"]
                      \ar[r]
                      &\Hmlg_0(U) \oplus \Hmlg_0(V)
                      \ar[dd, equals]
                      \\
                      &(\Z \oplus \Z) \oplus 0
                      \ar[d, "\dis"]
                      &&&
                      \\
                      \Z
                      \ar[r]
                      &\Z \oplus \Z
                      \ar[r]
                      &\Hmlg_1(T)
                      \ar[r]
                      &\Z \ar[r]
                      &\Z \oplus \Z
                  \end{tikzcd}
              \]
              Furthermore, from exactness we know that \(\Z \oplus \Z \to \Hmlg_1(T)\) has null
              kernel, hence a monomorphism, since the image of \(\Z \to \Z \oplus \Z\) is zero. On
              the other hand, since \(\Hmlg_0(U \cap V) \to \Hmlg_0(U) \oplus \Hmlg_0(V)\) induces an
              injective mapping \(1 \mapsto (1, 1)\) of the form \(\Z \to \Z \oplus \Z\), then by
              exactness it follows that the image of \(\Hmlg_1(T) \to \Hmlg_0(U \cap V)\) is
              identically zero. This proves that \(\Z \oplus \Z \to \Hmlg_1(T)\) is also an
              epimorphism, which shows the existence of an isomorphism
              \(\Hmlg_1(T) \iso \Z \oplus \Z\). In summary we have:
              \[
                  \Hmlg_p(T) =
                  \begin{cases}
                      \Z,           & \text{if } p = 0 \\
                      \Z \oplus \Z, & \text{if } p = 1 \\
                      \Z,           & \text{if } p = 2 \\
                      0,            & \text{otherwise}
                  \end{cases}
              \]
              where \(\Hmlg_0(T) = \Z\) comes from the fact that \(T\) is path-connected.
    \end{enumerate}
\end{example}

\begin{example}[Klein bottle]
    \label{exp:klein-bottle-homology}
    Let \(K\) denote the Klein bottle and let both \(M\) and \(M'\) denote copies of
    a M\"{o}bius band. If the relation \(\sim\) denoted the gluing of \(M\) and
    \(M'\) along their boundary, then \(K \iso (M \disj M')/{\sim}\). Moreover, we know
    that \(M = M' \isoht S^1\) and \(M \cap M' \isoht S^1\). From this we know that for
    each \(p > 2\) we have
    \[
        \begin{tikzcd}
            \Hmlg_p(M) \oplus \Hmlg_p(M') \ar[r] \ar[d, equals]
            &\Hmlg_p(K) \ar[r] \ar[d, equals]
            &\Hmlg_{p-1}(M \cap M') \ar[d, equals]
            \\
            0 \ar[r]
            &\Hmlg_p(K) \ar[r]
            &0
        \end{tikzcd}
    \]
    therefore \(\Hmlg_p(K) = 0\) from exactness. Let's consider the case where
    \(p = 2\):
    \[
        \begin{tikzcd}
            \Hmlg_2(M) \oplus \Hmlg_2(M') \ar[r] \ar[d, equals]
            &\Hmlg_2(K)
            \ar[r] \ar[d, equals]
            &\Hmlg_1(M \cap M')
            \ar[r] \ar[d, equals]
            &\Hmlg_1(M) \oplus \Hmlg_1(M') \ar[d, equals]
            \\
            0 \ar[r]
            &\Hmlg_2(K)
            \ar[r]
            &\Z
            \ar[r]
            &\Z \oplus \Z
        \end{tikzcd}
    \]
    Consider the circle \(\sigma \in Z_1(M \cap M')\), which generates all
    \(1\)-cycles of the intersection, then when mapped back to either \(M\) or
    \(M'\) we obtain a cycle looping two times around the boundary of the strips,
    therefore the induced mapping \(\Z \to \Z \oplus \Z\) takes
    \(1 \mapsto (2, 2)\), showing that the map is injective---hence has a null kernel. Since
    \(\Hmlg_2(K) \mono \Z\) is also injective, then \(\Hmlg_2(K)\) is isomorphic to
    its image in \(\Z\)---which on the other hand is equal to the kernel of
    \(\Z \to \Z \oplus \Z\) by exactness of the sequence---hence \(\Hmlg_2(K) = 0\).

    For the case where \(p=1\) we have:
    \[
        \begin{tikzcd}[column sep=small]
            \Hmlg_1(M) \oplus \Hmlg_1(M') \ar[d, equals] \ar[r]
            &\Hmlg_1(K) \ar[d, equals] \ar[r]
            &\Hmlg_0(M \cap M') \ar[d, equals] \ar[r]
            &\Hmlg_0(M) \oplus \Hmlg_0(M') \ar[r] \ar[d, equals]
            &\Hmlg_0(K) \ar[d, equals]
            \\
            \Z \oplus \Z
            \ar[r]
            &\Hmlg_1(K)
            \ar[r]
            &\Z \ar[r]
            &\Z \oplus \Z \ar[r]
            &\Z
        \end{tikzcd}
    \]
    Notice that \(\Z \to \Z \oplus \Z\) maps \(1 \mapsto (1, 1)\) since it maps the generator of
    the circle \(M \cap M'\) to the generators of the disjoint circles \(M\) and
    \(M'\). This mapping is injective, therefore by
    exactness it follows that the map \(\Hmlg_1(K) \to \Z\) is identically zero. From
    this we can extract the following short exact sequence:
    \[
        \begin{tikzcd}
            0 \ar[r]
            &\Hmlg_1(M \cap M') \ar[r, tail] \ar[d, equals]
            &\Hmlg_1(M) \oplus \Hmlg_1(M) \ar[r, two heads] \ar[d, equals]
            &\Hmlg_1(K) \ar[r] \ar[d, equals]
            &0
            \\
            0 \ar[r]
            &\Z \ar[r, tail]
            &\Z \oplus \Z \ar[r, two heads]
            &\Hmlg_1(K) \ar[r]
            &0
        \end{tikzcd}
    \]
    therefore from the isomorphism theorem for groups we have
    \[
        \Hmlg_1(K)
        \iso \frac{\Z \oplus \Z}{\im(\Z \to \Z \oplus \Z)}
        = \frac{\Z \oplus \Z}{\{(2x, 2x) \colon x \in \Z\}}
        \iso \Z \oplus (\Z/2\Z).
    \]
    Indeed, the last isomorphism is given by the mapping
    \([a, b] \mapsto (a + b, [b])\): surjectivity is trivial, and on the other hand if
    \((a - b, [b]) = (0, [0])\) then \(a = b\) and \(b = 2c\) for some \(c \in \Z\)
    therefore \([a, b] = [2c, 2c] = 0\), proving injectivity---which shows the last
    isomorphism. In summary we obtain:
    \[
        \Hmlg_p(K) =
        \begin{cases}
            \Z,                 & \text{if } p = 0 \\
            \Z \oplus (\Z/2\Z), & \text{if } p = 1 \\
            0,                  & \text{otherwise}
        \end{cases}
    \]
\end{example}

\begin{example}[Projective plane]
    \label{exp:rp2-homology}
    Consider \(\R \Proj^2\) as the gluing of the M\"{o}bius strip \(M\) with the
    disk \(D\) along the boundary. Let \(p \in \R \Proj^2\) be any point, define
    \(U\) to be a disk around \(p\), and \(V \coloneq \R \Proj^2 \setminus p \isoht M \isoht
    S^1\)---then we also have \(U \cap V = U \setminus p \isoht S^1\). For any \(p > 2\) we have
    \[
        \begin{tikzcd}
            \Hmlg_p(U) \oplus \Hmlg_p(V) \ar[r] \ar[d, equals]
            &\Hmlg_p(\R \Proj^2) \ar[r] \ar[d, equals]
            &\Hmlg_{p-1}(U \cap V) \ar[d, equals]
            \\
            0 \ar[r]
            &\Hmlg_p(\R \Proj^2) \ar[r]
            &0
        \end{tikzcd}
    \]
    and from exactness we obtain \(\Hmlg_p(\R\Proj^2) = 0\). For the case \(p = 2\)
    we have
    \[
        \begin{tikzcd}[row sep=small]
            \Hmlg_2(U) \oplus \Hmlg_2(V) \ar[r] \ar[dd, equals]
            &\Hmlg_2(\R\Proj^2) \ar[r, "\delta"] \ar[dd, equals]
            &\Hmlg_1(U \cap V) \ar[dd, equals] \ar[r, "i_* \oplus j_{*}"]
            &\Hmlg_1(U) \oplus \Hmlg_1(V) \ar[d, equals]
            \\
            &&&0 \oplus \Z \ar[d, equals]
            \\
            0 \ar[r]
            &\Hmlg_2(\R\Proj^2) \ar[r, tail]
            &\Z \ar[r]
            &\Z
        \end{tikzcd}
    \]
    From injectivity of the map \(\delta\) we find
    \(\Hmlg_2(\R\Proj^2) \iso \im \delta\) and from exactness we have
    \(\im \delta = \ker(i_{*} \oplus j_{*})\). Notice however that the mapping
    \(\Z \to \Z\) must send \(1 \mapsto 2\) since the generator of
    \(\Hmlg_1(U \cap V)\) is a loop through the circle, which is certainly a boundary
    in \(\Hmlg_1(U)\) since \(U\) is a disk, while \(j\) will map such loop to a
    double loop around the boundary of the M\"{o}bius strip. This shows that
    \(i_{*} \oplus j_{*}\) is injective and hence has null kernel, proving that
    \(\Hmlg_2(\R\Proj^2) = 0\). For the case where \(p = 1\) one has, in the reduced
    homology case, that:
    \[
        \begin{tikzcd}[row sep=small]
            \Hmlg_1(U \cap V) \ar[r, "i_{*} \oplus j_{*}"] \ar[dd, equals]
            &\Hmlg_1(U) \oplus \Hmlg_1(V) \ar[r, "k_{*} - \ell_{*}"] \ar[dd, equals]
            &\Hmlg_1(\R\Proj^2) \ar[r] \ar[dd, equals]
            &\rHmlg_0(U \cap V) \ar[d, equals]
            \\
            &&&\rHmlg_0(S^1) \ar[d, equals]
            \\
            \Z \ar[r] &\Z \ar[r, two heads] &\Hmlg_1(\R\Proj^2) \ar[r] &0
        \end{tikzcd}
    \]
    Since \(\Z \epi \Hmlg_1(\R\Proj^2)\) is surjective, there exists an isomorphism
    \[
        \Hmlg_1(\R\Proj^2) \iso \Z/\ker(k_{*} - \ell_{*}) = \Z/\im(i_{*} \oplus j_{*}) = \Z/2\Z.
    \]
    Moreover, since \(\R\Proj^2\) is path-connected then
    \(\Hmlg_0(\R\Proj^2) \iso \Z\). In summary we obtained:
    \[
        \Hmlg_p(\R \Proj^2) =
        \begin{cases}
            \Z,     & \text{if } p = 0 \\
            \Z/2\Z, & \text{if } p = 1 \\
            0,      & \text{otherwise}
        \end{cases}
    \]
\end{example}

%%% Local Variables:
%%% mode: latex
%%% TeX-master: "../../../deep-dive"
%%% End:

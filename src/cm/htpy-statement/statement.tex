\documentclass[11pt,reqno]{amsart}
\usepackage[brazilian]{babel}

% Page geometry
\usepackage{fullpage}

% Math stuff: do not mess with the ordering!
\usepackage{mathtools}
\usepackage{amsthm}
\usepackage{amssymb}
\usepackage{stmaryrd}
\usepackage{tikz}
\usepackage{tikz-cd}

% Font
\usepackage[no-math]{newpxtext} % last: mathpazo
\usepackage{newpxmath}
% Set arrow tip to that of newpxmath
\tikzset{>=Straight Barb, commutative diagrams/arrow style=tikz}

% Utilities
\usepackage{enumerate}
\usepackage{todonotes}

% Color
\usepackage{xcolor}
\definecolor{brightmaroon}{rgb}{0.76, 0.13, 0.28}

% References
\usepackage{hyperref}
\hypersetup{
  colorlinks = true,
  allcolors  = brightmaroon,
}
\usepackage{cleveref}
\usepackage[
backend=biber,
style=alphabetic,
]{biblatex}
\addbibresource{../src/bibliography.bib}

% this is NEEDED in order to use babel with the brazilian module!!
\tikzset{
  every picture/.prefix style={
    execute at begin picture=\shorthandoff{"}
  }
}

\linespread{1.05}
\vfuzz=14pt % No more vbox errors all over the place

\newcommand{\HRule}{\rule{\linewidth}{0.5mm}} % Horizontal rule
\usepackage{multirow}

% Environments
\theoremstyle{definition}
\newtheorem{theorem}{Theorem}[section]
\newtheorem{proposition}[theorem]{Proposition}
\newtheorem{lemma}[theorem]{Lemma}
\newtheorem{corollary}[theorem]{Corollary}
\newtheorem{axiom}[theorem]{Axiom}
\newtheorem{definition}[theorem]{Definition}
\newtheorem{remark}[theorem]{Remark}
\newtheorem{example}[theorem]{Example}
\newtheorem{notation}[theorem]{Notation}

% symbols
\renewcommand{\qedsymbol}{\(\natural\)}% blacklozenge
\renewcommand{\leq}{\leqslant}
\renewcommand{\geq}{\geqslant}
\renewcommand{\setminus}{\smallsetminus}
% \newcommand{\coloneq}{\coloneqq}

% ':' for maps and '\colon' for relations on collections
\DeclareMathSymbol{:}{\mathpunct}{operators}{"3A}
\let\colon\relax
\DeclareMathSymbol{\colon}{\mathrel}{operators}{"3A}

% General
\DeclareMathOperator{\Hom}{Mor}
\DeclareMathOperator{\Fct}{Fct}
\DeclareMathOperator{\Obj}{Obj}
\DeclareMathOperator{\Mor}{Mor}
\DeclareMathOperator{\End}{End}
\DeclareMathOperator{\Aut}{Aut}
\DeclareMathOperator{\Id}{id}
\DeclareMathOperator{\im}{im}
\DeclareMathOperator{\dom}{dom}
\DeclareMathOperator{\codom}{cod}
\DeclareMathOperator{\Sing}{Sing}

% arrows
\newcommand{\op}{\mathrm{op}}
\newcommand{\isoto}{\xrightarrow{\raisebox{-.6ex}[0ex][0ex]{$\sim$}}}
\newcommand{\isonat}{\xRightarrow{\raisebox{-.8ex}[0ex][0ex]{$\sim$}}}
\newcommand{\To}{\Rightarrow}
\newcommand{\mono}{\rightarrowtail}
\newcommand{\epi}{\twoheadrightarrow}
\newcommand{\iso}{\simeq} % changed from simeq
\newcommand{\dis}{\iso} % diagram isomorphism
\newcommand{\nat}{\Rightarrow}
\newcommand{\emb}{\hookrightarrow}
\newcommand{\cat}{\mathcal}
\renewcommand{\implies}{\Rightarrow}

% Common collections
\newcommand{\Z}{\mathbf{Z}}
\newcommand{\N}{\mathbf{N}}
\newcommand{\Q}{\mathbf{Q}}
\newcommand{\CC}{\mathbf{C}}
\newcommand{\R}{\mathbf{R}}
\renewcommand{\emptyset}{\varnothing}

% Common categories
\newcommand{\Set}{{\textbf{Set}}}
\newcommand{\sSet}{{\textbf{sSet}}}
\newcommand{\Vect}{{\textbf{Vec}}}
\let\Top\relax
\newcommand{\Top}{{\textbf{Top}}}
\newcommand{\HoTop}{{\textbf{Ho}(\textbf{Top})}}
\newcommand{\Trees}{{\mathbf{\Omega}}}
\newcommand{\Grp}{{\textbf{Grp}}}
\newcommand{\Ab}{{\textbf{Ab}}}
\newcommand{\Graph}{{\textbf{Graph}}}

\begin{document}
\begin{titlepage}
 \vfill
  \begin{center}
       \textsc{\LARGE \textbf{Universidade de São Paulo}} \\[2.0cm]

       \vskip 0.5cm
       \textsc{\large Luiz Gustavo Mugnaini Anselmo}

       {\normalsize Curso de Ciências Moleculares \\
         Turma 30, n\(^{\text{o}}\)
         USP 11809746

       E-mail: \texttt{luizmugnaini@usp.br}}\\[2.0cm]

       \HRule\\
       \vskip 0.5cm
       {\LARGE \textbf{Teoria de Homotopia Dendroidal \& Operads}}
       \HRule\\[1.5cm]

       \hspace{.45\textwidth}
       \begin{minipage}{.5\textwidth}
         \normalsize \textbf{Projeto de Iniciação à Pesquisa:
           Ciclo Avançado do Curso de Ciências
           Moleculares}\\[0.5cm]

       \textsc{\large Prof.~Ivan Struchiner}\\
       Universidade de São Paulo \\
       Instituto de Matemática e Estatística \\
       E-mail: \texttt{ivanstru@ime.usp.br}\\[1cm]

       \normalsize São Paulo, Julho de 2022
       \end{minipage}
  \end{center}
\end{titlepage}

% Title stuff
\title[Teoria de Homotopia Dendroidal \& Operads]{%
{\footnotesize\sl Projeto para o Ciclo Avançado} \\ \smallskip
  Teoria de Homotopia Dendroidal \& Operads
}%

\author{%
  Luiz Gustavo Mugnaini Anselmo e Ivan Struchiner
}%

\address{%
  Instituto de Matemática e Estatística, Universidade de São
  Paulo, Rua do Matão 1010, 05508--090~São Paulo, SP
}%

\email{luizmugnaini@usp.br, ivanstru@ime.usp.br}

\begin{abstract}
Este é o projeto de pesquisa para o Ciclo Avançado de Luiz Gustavo Mugnaini
Anselmo, a ser desenvolvido sob a supervisão de Ivan Struchiner, no Instituto de
Matemática e Estatística, USP, no período de agosto de 2022 a junho de
2024. Tendo como objetivo estudar espaços topológicos por meio de métodos
combinatórios, o projeto será desenvolvido por meio do estudo de teoria de
homotopia simplicial e dendroidal, operads e teoria de categorias. Um dos
objetivos do projeto é assegurar uma boa preparação do aluno para o ingresso em
um programa de pós-graduação.
\end{abstract}
\maketitle


\section{Introdução}\label{sec:intro}

O projeto para o Ciclo Avançado, do Curso de Ciências Moleculares, de Luiz
Gustavo Mugnaini Anselmo terá como principais grandes áreas o estudo de teoria
de homotopia e teoria de categorias. Objetiva-se um estudo coerente da
literatura central básica das áreas supracitadas, possibilitando ao aluno uma
sólida formação fundamental em tais áreas, favorecendo o seu ingresso em um
futuro programa de pós-graduação na área que se destina o atual projeto.

A teoria de homotopia se baseia no estudo de estruturas equipadas com a noção de
\emph{deformações} entre morfismos --- o que leva o nome de \emph{homotopia}. No
contexto da categoria \(\Top\), dados espaços topológicos \(X\) e \(Y\) e
morfismos topológicos \(f, g: X \rightrightarrows Y\), definimos uma homotopia
\(\eta: f \To g\) entre \(f\) e \(g\) como um morfismo topológico entre o
cilindro de \(X\) em \(Y\), isto é \(\eta: X \times I \to Y\) --- onde \(I
\coloneq [0, 1] \emb \R\) é o intervalo unitário --- tal que o seguinte diagrama
comute
\[
\begin{tikzcd}
  X \ar[r, "\Id_X \times\, \delta_0"] \ar[dr, swap, "f"]
  &X \times I \ar[d, "\eta"]
  &X \ar[l, swap, "\Id_X \times\, \delta_1"] \ar[dl, "g"] \\
  &Y &
\end{tikzcd}
\]
onde \(\delta_0, \delta_1: X \rightrightarrows I\) são os mapas \(\delta_{0}(x)
\coloneq 0\) e \(\delta_1(x) \coloneq 1\) para todo \(x \in X\). A noção de
homotopia gera uma relação de equivalência, a qual podemos denotar por \(\sim\),
entre morfismos --- possibilitando a criação de uma categoria homotópica,
constituída de espaços topológicos e classes de morfismos topológicos
homotópicos, isto é, \(\Mor(\Top)/{\sim}\). Os isomorfismos dessa categoria,
chamados \emph{equivalências homotópicas}, foram a primeira tentativa de buscar
uma relação de equivalência entre espaços topológicos que substituísse
isomorfismos em \(\Top\). Equivalências homotópicas, todavia, não foram
suficientes para a viabilização do programa de classificação de espaços
topológicos.

A fim de se enfraquecer ainda mais a noção de equivalência entre espaços
topológicos, surgem as equivalências \emph{fracas} homotópicas, consistindo de
morfismos topológicos \(f: X \to Y\) tais que \(f\) induz uma bijeção entre as
componentes conexas por caminhos, \(\pi_0(X) \isoto \pi_0(Y)\), de \(X\) e
\(Y\), e isomorfismos de grupo \(\pi_n(X, x) \isoto \pi_n(Y, f(x))\) para todo
\(x \in X\) e \(n \in \Z_{\geq 1}\) entre os grupos the homotopia de \(X\) e
\(Y\). Tais morfismos, entretanto, não possuem inversas bem definidas, portanto,
a fim de construir uma categorias onde isomorfismos são equivalências fracas,
precisamos \emph{adicionar} em \(\Top\) inversas \emph{formais} para estes
morfismos --- este processo leva o nome de \emph{localização} de \(\Top\) com
relação às equivalências fracas homotópicas, e a categoria resultante é
comumente denotada por \(\HoTop\). Em 1949, J.H.C. Whitehead apresentou um
teorema que diz que equivalências fracas homotópicas entre espaços CW são
equivalências homotópicas --- fortalecendo a visão de que equivalências fracas
são uma ferramenta fundamental para a classificação de espaços topológicos.

A \emph{abstração} de tal construção, feita inicialmente no contexto de espaços
topológicos, nos possibilita estudar contextos mais abrangentes, tal como o
estudo de \emph{categorias modelo} --- introduzida pelos trabalhos de
D. Quillen. Uma categoria modelo é axiomatizada como possuindo três classes de
morfismos, sendo estas: \emph{equivalências fracas}, \emph{fibrações} e
\emph{cofibrações}. O objetivo da construção é de obter um ambiente
\emph{``modelo''} para se trabalhar com teorias de homotopia.

A fim de se estudar teoria de homotopia por métodos \emph{combinatórios}, temos
a clássica abordagem por meio de \emph{conjuntos simpliciais} e, mais
recentemente, \emph{conjuntos dendroidais}. A categoria simplicial
\(\mathbf{\Delta}\) é definida por objetos que são \emph{ordinais finitos}
não-vazios e morfismos preservando \emph{ordem} --- enquanto isso, conjuntos
simpliciais são pré-feixes \(\mathbf{\Delta}^{\op} \to \Set\), que dão origem à
categoria de conjuntos simpliciais \(\sSet\). A partir de um conjunto
simplicial, podemos construir um modelo combinatório de espaços topológicos pelo
processo de colagem de faces de simplices --- isto leva o nome de
\emph{realização geométrica} do conjunto simplicial, que é definido como um
funtor covariante \(|-|: \sSet \to \Top\). Um funtor \emph{adjunto} à direita de
\(|-|\) é o funtor de \emph{complexo singular} \(\Sing: \Top \to \sSet\). O par
\(|-|\) e \(\Sing\) define uma \emph{equivalência de Quillen} entre as estruras
modelo destas categorias.

Toda categoria pequena \(\cat C\) induz um funtor \emph{nervo} \(N: \cat C \to
\sSet\) --- possibilitanto, por meio do funtor de realização geométrica \(|-|\),
a formação de um espaço topológico \(B\cat{C}\), chamado espaço de
\emph{classificação} de \(\cat C\). O espaço de classificação, todavia, é uma
ferramenta rudimentar para o estudo de \(\cat C\), uma vez que a direção dos
morfismos são \emph{esquecidas} --- fazendo com que diversas categorias tenham
realizações homotopicamente equivalentes, comportamento muitas vezes indesejável.

\emph{Operads} são uma generalização da noção de uma categoria, onde
anteriormente interpretavamos morfismos como setas de \emph{um} objeto a outro,
substituímos por \emph{operações} --- o \textit{input} de mapas são coleções
finitas de objetos e o \textit{output} é um único objeto, portanto, a
\emph{composição} de mapas no contexto de operads pode ser interpretada como
\emph{árvores com raíz}. A fim de construir uma teoria para operads que se
assemelhe à relação entre categorias e conjuntos simpliciais, surge a abordagem
\emph{dendroidal}. Neste novo contexto, construímos a categoria \(\Trees\) de
árvores, onde, em oposição à \(\mathbf{\Delta}\), a coleção de ordinais é
substituída por \emph{árvores} e morfismos entre árvores são mapas entre operads
livremente gerados por tais árvores. Dessa categoria, definimos um
\emph{conjunto dendroidal} como um pré-feixe \(\Trees^{\op} \to \Set\). Uma
vantagem do tratamento dendroidal é a \emph{generalização natural} da abordagem
simplicial de \emph{\(\infty\)-categorias}, por meio do uso da teoria de
\emph{\(\infty\)-operads} --- isto é, a teoria de conjuntos dendroidais
\emph{extende} a teoria de conjuntos simpliciais, principalmente se valendo da
presença de uma grande quantidade de \emph{automorfismos} entre árvores,
propriedade essa não contemplada pela abordagem simplicial.

\section{Literatura}

O tópico do projeto em questão requer uma vasta compreensão da literatura, sendo
ambicioso cobrí-la no período de 4~semestres --- dessa forma, diversas das obras
fundamentais para o desenvolvimento atual de teoria de homotopia não serão
incluídas na literatura base que será utilizada durante os estudos do ciclo
avançado. Os recursos de estudo serão divididos em duas partes:
\begin{itemize}\setlength\itemsep{0em}
\item Literatura elementar, constituída pelas obras que serão utilizadas na
  formação do conhecimento básico necessário para o estudo do projeto. O
  acompanhamento desta literatura será feita nos primeiros momentos do ciclo
  avançado --- progredindo para tópicos mais específicos posteriormente.

\item Literatura específica, englobando principais duas obras essenciais ao
  contexto do projeto --- as quais assumem o conhecimento do leitor dos tópicos
  abordados na literatura elementar.
\end{itemize}

\subsection{Literatura Elementar}\label{sub:lit-elem}

\subsubsection{\textbf{Topology} --- J. Munkres~\cite{Mun00}}

O conhecimento de topologia \textit{point-set} é essencial para a motivação das
contruções envolvidas no projeto em questão.

\subsubsection{\textbf{Algebra} --- S. Lang~\cite{Lang93}}

O livro de Lang é a referência clássica no estudo elementar de álgebra geral,
cobrindo áreas como: teoria de grupos, teoria de anéis, módulos e corpos. O
livro de P. Aluffi~\cite{Aluf09} também será utilizado como referência
complementar.

\subsubsection{%
  \textbf{Categories and Sheaves} --- M. Kashiwara \& P. Shapira~\cite{Shap06}
}%

Os tópicos abordados por M. Kashiwara e P. Shapira são base da linguagem
empregada em todas as áreas da matemática moderna --- sendo, portanto, uma
leitura essencial. Outros textos de apoio também são~\cite{Rie16}
e~\cite{MacLane78}.

\subsubsection{
  \textbf{Commutative Algebra with a View Towards Algebraic Geometry}
  --- D. Eisenbud~\cite{Eisen95}
}

Álgebra comutativa é uma ferramenta essencial para o estudo das mais diversas
áreas da matemática, tornando-se imprescindível o seu domínio, o livro de
D. Eisenbud é um detalhado e extenso texto básico na área. Outra referência
clássica que será utilizada como apoio será o livro de I.G. MacDonald e
M.F. Atyah~\cite{MacDAty69}

\subsubsection{%
  \textbf{An Introduction to Homological Algebra} --- C.A. Weibel~\cite{Wei95}
}%

Álgebra homológica é uma ferramenta básica para o estudo de topologia algébrica
e diversas outras áreas da matemática, portanto, se torna imprescindível seu
conhecimento.

\subsubsection{%
  \textbf{Algebraic Topology} --- T.T. Dieck~\cite{Die08}
}%

Uma introdução à área de topologia algébrica, homologia e teoria de homotopia em
espaços topológicos. Como referências adicionais, serão utilizados os livros de
P. May~\cite{May99Concise} e~\cite{MayPonto12More}, esse último sendo escrito em
colaboração com K. Ponto.

\subsection{Literatura Específica}\label{sub:lit-esp}

\subsubsection{%
  \textbf{Simplicial and Dendroidal Homotopy Theory}
  --- G. Heuts \& I. Moerdijk~\cite{HeuMoer22}
}

O livro de Heuts e Moerdijk será a referência \emph{principal} para o estudo do
projeto, contando com uma exposição da teoria simplicial e dendroidal de
homotopia e suas conexões com operads topológicos e \textit{infinite loop
spaces}. Os autores se utilizam do formalismo proporcionado pelas categorias
modelo de Quillen, estrutura modelo de Kan-Quillen e a estrutura modelo de Joyal
para a contrução de teorias de homotopia para \(\infty\)-categorias.

\subsubsection{%
  \textbf{Simplicial Homotopy Theory}
  --- P.G. Goerss \& J.F. Jardine~\cite{GoeJar09}
}

O livro de Goerss e Jardine é exposição moderna da teoria básica envolvendo a
vasta área de métodos simpliciais em teoria de homotopia. Ele servirá como
referência complementar, sendo utilizado de forma paralela aos estudos do
livro de Heuts e Moerdijk.

\subsubsection{%
  \textbf{Algebraic Operads}
  --- J.L Loday \& B. Vallete~\cite{LodayVall12}
}

O livro de Loday e Vallete servirá como referência complementar ao livro
principal no estudo de teorias algébricas por meio de operads, tendo também
aplicações à teoria de homotopia.

\subsubsection{%
  \textbf{Modules Over Operads and Functors}
  --- B. Fresse~\cite{Fresse09}
}

O livro de Fresse tem o objetivo de estudar generalizações de teorias algébricas
por meio do estudo de álgebras sobre operads. Para o objetivo deste projeto,
este livro será utilizado como uma referência complementar à obra principal.

\appendix

\section{Planejamento de grade de disciplinas}\label{sec:disciplinas}

\begin{table}[h!]\label{tab:disc}
  \centering
  \caption{Proposta de grade de disciplinas para o ciclo avançado. Os tipos são
    divididos em graduação (G) e pós-graduação (PG).}
  \begin{tabular}{ |c|c|c|c|c| }
    \hline
    Ano e semestre & Código & Disciplina & Tipo & Créditos \\
    \hline

    \multirow{4}{*}{2022} & CCM0318 & Iniciação à Pesquisa I & G & 12 \\
                   & MAT6680 & Tópicos de Anéis e Módulos & PG & 8 \\
                   & MAT0431 & Introdução à Topologia Algébrica & G & 4 \\
                   & MAT0463 & Tópicos de Matemática IV & G & 4 \\
    \hline

    \multirow{4}{*}{2023 / 1} & CCM0328 & Iniciação à Pesquisa II & G & 12 \\
                   & MAT5797  & Tópicos em Álgebra & PG & 8 \\
                   & MAT6684 & Topologia Algébrica & PG & 8 \\
    \hline

    \multirow{4}{*}{2023 / 2} & CCM0418 & Iniciação à Pesquisa III & G & 12 \\
                   & MAT5809 & Teoria de Homotopia & PG & 8 \\
                   & MAE0227 & Probabilidade II & G & 6 \\
                   & MAC5921 & Deep Learning & PG & 8 \\
    \hline

    \multirow{3}{*}{2024} & CCM0428 & Iniciação à Pesquisa IV & G & 12 \\
                   & MAT5771 & Geometria Riemanniana & PG & 10 \\
                   & MAT5721 & Introdução à Análise Funcional & PG & 10 \\
    \hline

    \multicolumn{4}{|c|}{Total de créditos} & 126 \\
    \hline
\end{tabular}
\end{table}

\subsection{Outras Disciplinas}
Além disso, a depender do desenvolvimento do projeto e do oferecimento de
disciplinas, poderão ser cursadas algumas das seguintes disciplinas:
\begin{enumerate}[1.]
\item Introdução à Geometria Algébrica (MAT5761 --- pós-graduação).
\item Topologia Algébrica II (MAT5866 --- pós-graduação).
\item Teoria dos Toposes e Aplicações (MAT6668 --- pós-graduação).
\item Variedades Diferenciáveis e Grupos de Lie (MAT5799 --- pós-graduação).
\item Introdução às Álgebras de Hopf (MAT6016 --- pós-graduação).
\item Introdução à Teoria dos Conjuntos (MAT6202 --- pós-graduação).
\item Teoria de Galois (MAT6643 --- pós-graduação).
\item Representações de Grupos (MAT6681 --- pós-graduação).
\end{enumerate}

\printbibliography

\end{document}
%%% Local Variables:
%%% mode: latex
%%% TeX-master: t
%%% End:

\section{Bases and Countability}

\begin{definition}[Base]\label{def: base}
  Let \(X\) be a topological space. A collection \(\mathcal B \subseteq 2^X\) is
  said to be a basis for the topology of \(X\) if it satisfies the following
  \begin{enumerate}[(B1)]
    \item Every element of \(\mathcal B\) is an open set of \(X\).
    \item Every open subset \(U \subseteq X\) can be written as a union of
      elements of \(B\), that is, exists \(\{B_i\}_{i \in I} \subseteq \mathcal
      B\) for which \(U = \bigcup_{i \in  I} B_i\).
  \end{enumerate}
\end{definition}

\begin{proposition}[Necessary and sufficient condition for a basis]
  Let \(X\) be a set and \(\mathcal B \subseteq 2^X\). Then \(\mathcal B\) is a
  basis for some topology of \(X\) if and only if it satisfies
  \begin{enumerate}[(1)]
    \item \(X = \bigcup_{B \in \mathcal B} B\).
    \item If \(x \in A \cap B\), where \(A, B \in \mathcal B\), then there
      exists \(C \in \mathcal B\) such that \(x \in C \subseteq A \cap B\).
  \end{enumerate}
\end{proposition}

\todo[inline]{write proof}

\begin{definition}[Subbase]\label{def: subbase}
  Let \((X, \mathcal T)\) be a topological space. A collection \(\mathcal S
  \subseteq \mathcal T\) is called a subbase for \((X, \mathcal T)\) if the
  collection of all finite intersections \(U_1 \cap \dots \cap U_n\), where
  \(U_i \in \mathcal S\), is a base for \((X, \mathcal T)\).
\end{definition}

\begin{definition}[Weight]\label{def: weight}
  Let \(X\) be a topological space and \(\mathfrak B\) be the collection of
  all bases for the topology of \(X\). We define the weight of \(X\) as
  \[
    w(X) = \min_{\mathcal B \in \mathfrak B} |\mathcal B|.
  \]
\end{definition}

\begin{proposition}[Basis criterion for open sets]
  Let \(X\) a topological space and \(\mathcal B\) a base for the topology of
  \(X\). A set \(A \subseteq X\) is open if and only if for all points \(p \in
  A\) there exists a neighbourhood of \(p\), \(B_p \in \mathcal B\), such that
  \(B_p \subseteq A\).
\end{proposition}

\begin{proposition}[Basis criterion for continuity]
  Let \(f: X \to Y\) be a map between topological spaces, and \(\mathcal B\) be
  a basis for the topology of \(Y\). Then, \(f\) is continuous (hence a
  morphism) if and only if for all \(B \in \mathcal B\) we have \(f^{-1}(B)
  \subseteq X\) open.
\end{proposition}

\begin{proof}
  (\(\Rightarrow\)) If \(f\) is continuous, then certainly \(f^{-1}(B)\) is
  open. (\(\Leftarrow\)) On the other hand, if \(A \subseteq X\) is any open
  set, then \(A = \bigcup_{p \in A} B_p\) for \(B_p \in \mathcal B\)
  neighbourhood of \(p\). Hence \(f^{-1}(A) = f^{-1} \left( \bigcup_{p \in A}
  B_p \right) = \bigcup_{p \in A} f^{-1}(B_p)\) is the union of open sets, thus
  \(f^{-1}(A)\) is open and therefore \(f\) is continuous.
\end{proof}

\begin{definition}[Basis at a point]\label{def: basis at a point}
  Let \(X\) be a topological space and \(p \in X\) be any fixed point. We define
  the collection \(\mathcal B_p \subseteq 2^X\) of neighbourhoods of \(p\) to be
  the \emph{neighbourhood basis for the topology of \(X\) at \(p\)} if for any
  neighbourhood \(U_p \subseteq X\), there exists \(B \in \mathcal B_p\) such
  that \(B \subseteq U_p\).
\end{definition}

\begin{proposition}\label{prop: basis image surjective}
  Let \(f: X \to Y\) be a morphism of topological spaces and \(\mathcal B\) be a
  basis for the space \(X\). Then \(f(\mathcal B) = \{f(B): B \in \mathcal B\}\)
  is a basis for the space \(Y\) if and only if \(f\) is surjective and open.
\end{proposition}

\begin{proof}
  (\(\Rightarrow\)) Suppose \(f(\mathcal B)\) is a basis for \(Y\), then for all
  \(V \subseteq Y\) open, there exists an indexing set \(I\) such that \(V =
  \bigcup_{i \in  I} U_i\), where \(U_i \in f(\mathcal B)\). This implies in the
  existence of \(B \in \mathcal B\) such that \(U_i = f(B)\) hence \(f\) is
  open. Consider now any point \(y \in Y\) and any neighbourhood \(V_y\) of
  \(y\). From the same argument as above we have \(V_y = \bigcup_{i \in  I_y}
  U_i\), where there exists some \(i \in I_y\) such that \(y \in U_i\) and hence
  \(y \in U_i = f(B)\) for some \(B \in \mathcal B\).
  (\(\Leftarrow\)) Suppose \(f\) is surjective and open. Let \(V \subseteq Y\)
  be any open set. Since \(f\) is continuous and surjective, we have that
  \(U = f^{-1}(V)\) is a nonempty open set. Since \(\mathcal B\) is a basis, we
  can write \(U = \bigcup_{i \in  I} B_i\) where \(B_i \in B\), and then \(f(U)
  = f\left( \bigcup_{i \in  I} B_i \right) = \bigcup_{i \in I} f(B_i)\), where
  \(f(B_i) \in f(\mathcal B)\) and \(f(U) = V\) from the fact that \(f\) is
  surjective. Since \(f\) is open, \(f(B)\) is open for all \(B \in \mathcal
  B\). Hence \(f(\mathcal B)\) is a basis for the space \(Y\).
\end{proof}

\begin{definition}\label{def: character}
  Let \((X, \mathcal T)\) be a topological space. Let \(x \in X\) be any point
  and consider \(\mathfrak B_x\) the collection of all bases at \(x\). Then we
  define the character of \(X\) at the point \(x\) as
  \[
    \chi(x, (X, \mathcal T)) = \min_{\mathcal B_x \in \mathfrak B_x} |\mathcal
    B_x|
  \]
\end{definition}

\begin{definition}[First countable]\label{def: first countable}
  Let \(X\) be a topological space. We say that \(X\) is first countable if for
  all points \(p \in X\) there exists a countable basis of neighbourhoods at
  \(p\). Equivalently \(\chi(x, X) \leq \aleph_0\).
\end{definition}

\subsection{Convergence on First Countable Spaces}

\begin{proposition}[Sufficient condition for Hausdorff]
  Let \(X\) be a first countable space. Then, \(X\) is Hausdorff if and only if
  every sequence has at most one limit
\end{proposition}

\begin{proof}
  Let \(x, y \in X\) be distinct points, and \(\mathcal B\) be a neighbourhood
  basis at \(x\) and \(\mathcal A\) be a neighbourhood basis at \(y\). If \(X\)
  is not Hausdorff, then for all \(n \in \N\), choose a point \(x_n \in
  B_n \cap A_n\), where \(B_n \in \mathcal B\) and \(A_n \in \mathcal A\) and
  consider the sequence \(\{x_n\}_{n \in \N}\). Then there exists
  sub-sequences \(\{x_k\}\) and \(\{x_j\}\) for which \(x_k \to x\) and \(x_j \to
  y\).
\end{proof}

\begin{definition}[Nested neighbourhood basis]
  Let \(X\) be a topological space and a point \(p \in X\). The infinite sequence
  of neighbourhoods of \(p\), namely \((U_i)_{i \in \N}\) is said to be
  a nested neighbourhood basis at \(p\) if for all \(i \in \N, U_{i+1}
  \subseteq U_i\) and for all neighbourhood \(V\) of \(p\), there exists \(i \in
  \N\) for which \(U_i \subseteq V\).
\end{definition}

\begin{lemma}
  Let \(X\) be a first countable topological space. Then, for all points \(p \in
  X\) there exists a nested neighbourhood basis at \(p\).
\end{lemma}

\begin{proof}
  Since \(X\) is first countable, let \(\mathcal V\) be a countable basis for the
  topology of \(X\) at \(p\). If \(|\mathcal V| < \infty\) then define \(U_i :=
  V_1 \cap \dots \cap V_{|\mathcal V|}\) for all \(i \in \N\). If
  \(|\mathcal V|\) is infinite, then define for all \(i \in \N\) the
  set \(U_i := V_1 \cap \dots \cap V_i\). For both cases, the sequence
  \((U_i)_{i \in \N}\) is a nested neighbourhood basis.
\end{proof}

\begin{definition}[Eventually in]
  Let \(X\) be a topological space, and a sequence \((x_i)_{i \in \N}
  \subseteq X\), and a set \(A \subseteq X\). We say that \((x_i)_{i \in
  \N}\) is eventually in \(A\) if \(x_i \in A\) for all but finitely
  many \(i \in \N\).
\end{definition}

\begin{lemma}[Sequence lemma]\label{lem: sequence lemma}
  Let \(X\) be a first countable space, and a set \(A \subseteq X\), and a point
  \(p \in X\). Then
  \begin{enumerate}[(a)]
    \item \(p \in \overline A\) if and only if \(p\) is a limit of points of
      \(A\).
    \item \(A\) is closed in \(X\) if and only if \(A\) contains every limit
      point of sequences in \(A\).
    \item \(p \in \Int(A)\) if and only if all sequences that converge to
      \(p\) are eventually in \(A\).
    \item \(A\) is open in \(X\) if and only if every sequence in \(X\)
      converging to a point of \(A\) is eventually in \(A\).
  \end{enumerate}
\end{lemma}

\begin{proof}
  (a) (\(\Rightarrow\)) Let \(p \in \overline A\), then for all neighbourhoods
  \(V \subseteq A\) of \(p\), the set \(V \cap (A \setminus \{p\})\) is
  nonempty. Since \(X\) is first countable, consider \((U_i)_{i \in
  \N}\) a nested neighbourhood basis at \(p\) and construct the sequence
  \(x : \N \to \bigcup_{i \in \N} U_i \cap (A \setminus \{p\})
  \) defined as \(i \mapsto x_i \in U_i \cap (A \setminus \{p\})\). We'll show
  that \(x_i \to p\). Consider \(V \subseteq X\) any neighbourhood of \(p\),
  since \((U_i)_{i \in \N}\) is a basis then there exists an indexing set
  \(I_{V} \subseteq \N\) for which \(V = \bigcup_{i \in  I_V} U_i\).
  Consider the index \(n := \min(I_V)\), then, from the definition of the nested
  basis, we have \(\forall i \geq n, x_i \in V \cap A\), hence \(x_i \to p\).
  (\(\Leftarrow\)) Suppose that \((x_i)_{i \in \N}\) is a sequence of
  points in \(A\) such that \(x_i \to p\). From definition, for all
  neighbourhood \(V \subseteq X\) of \(p\), there exists \(n \in \N\)
  such that \(\forall i \geq N, x_i \in V\), moreover, since \(x_i \in A\) then
  \(x_i \in V \cap A\) which implies that \(p \not\in \Ext(A)\), hence \(p \in
  \overline A\).

  (b) (\(\Rightarrow\)) Suppose that \(A\) is closed, then \(A = \overline A\).
  Consider any sequence \((x_i)_{i \in \N}\) of points in \(A\) and let
  \(x_i \to p\). From item (a) we have that \(p \in A\). (\(\Leftarrow\))
  Suppose the contrary, then given any \(p \in \overline A\) we have \(p \in
  A\), which implies that \(A\) is closed.

  (c) (\(\Rightarrow\)) Suppose \(p \in \Int(A)\). Let \((x_i)_{i \in
  \N} \subseteq X\) be a sequence such that \(x_i \to p\). Consider any
  neighbourhood \(V \subseteq A\) of \(p\), then from the definition of
  convergence, exists \(N \in \N\) such that \(\forall i \geq N, x_i \in
  V_p\), then there exists at most \(N - 1\) points of \((x_i)_{i \in
  \N}\) outside \(A\), hence the sequence is eventually in \(A\).
  (\(\Leftarrow\)) Suppose \((x_i)_{i \in \N}\) is not-eventually in
  \(A\) and \(x_i \to p\). For the sake of contradiction, suppose that \(p \in
  \Int(A)\), then given a neighbourhood \(V \subseteq A\) of \(p\), there must
  exist \(N \in \N\) such that \(\forall i \geq N, x_i \in V \subseteq
  A\), which is a contradiction to the fact that \((x_i)_{i \in \N}\) is
  not-eventually in \(A\), hence \(p \not\in \Int(A)\).

  (d) (\(\Rightarrow\)) Let \(A\) be open, then \(A = \Int(A)\). Given any \(p
  \in A\) we have from item (c) that \(p\) is the limit of all of its converging
  sequences are eventually in \(A\). (\(\Leftarrow\)) Suppose the contrary, and
  let \(p \in A\) be any point. From item (c) we see that \(p \in \Int(A)\),
  which implies that \(A \subseteq \Int(A)\) and hence \(A\) is open.
\end{proof}

\subsection{Second Countable Spaces and Covers}

\begin{definition}[Second countable]\label{def: second countable}
  A topological space \(X\) is second countable if it admits a countable basis
  for its topology. Equivalently \(w(X) \leq \aleph_0\).
\end{definition}

\begin{definition}[Cover]\label{def: cover}
  Let \(X\) be a topological space and \(\mathcal U \subseteq 2^X\). We say that
  \(\mathcal U\) is a cover of \(X\) if for all points \(x \in X\) there exists
  \(U \in \mathcal U\) such that \(x \in U\).
\end{definition}

\begin{definition}[Open and closed covers]
  Let \(\mathcal U\) be a cover of a space \(X\). If all elements \(U \in
  \mathcal U\) are opens in \(X\), then \(\mathcal U\) is said to be open. On
  the other hand, if all \(U \in \mathcal U\) are closed in \(X\), then
  \(\mathcal U\) is said to be closed.
\end{definition}

\begin{definition}[Subcover]
  Let \(\mathcal U\) be a cover of a space \(X\). If \(\mathcal U' \subseteq
  \mathcal U\) is a cover for \(X\), then we call it a subcover of \(\mathcal
  U\).
\end{definition}

\begin{definition}[Lindelöf space]
  A space \(X\) is said to be Lindelöf if every open cover of \(X\) has a
  countable subcover.
\end{definition}

\begin{definition}[Separable space]
  A space \(X\) is said to be separable if it contains a countable dense subset.
\end{definition}

\begin{proposition}[Properties of second countable spaces]
  \label{prop: second countable properties}
  The following properties hold
  \begin{enumerate}[(SC1)]
    \item A second countable space is first countable.
    \item A second countable space is separable.
    \item A second countable space is Lindelöf.
  \end{enumerate}
\end{proposition}

\begin{proof}
  Let \(\mathcal B\) be a countable basis for \(X\).
  (SC1) Let a point \(p \in X\), then the set \(\mathcal B_p \subseteq \mathcal
  B\) of neighbourhoods of \(p\) is a countable basis for \(X\) at \(p\).

  (SC2) Let \(f: I \to \bigcup_{B \in \mathcal B} B\) with \(n \mapsto x_n\),
  where \(|I| = |B|\) is an indexing set, and \(x \in X\) be any point, and any
  neighbourhood of \(V_x \subseteq X\) of \(x\). Since \(B\) is a basis, there
  exists an indexing set \(I_{V_x}\) such that \(V_x = \bigcup_{i \in  I_{V_x}}
  B_i\), hence \(V_x \cap A\) is nonempty. Moreover, define a sequence
  \((x_i)_{i \in I_{V_x}}\) such that \(x_i \in B_i\). Notice that clearly \(x_i
  \to x\), from the fact that \(V_x\) is a neighbourhood of \(x\), and \(x_i \in
  A\), from \cref{lem: sequence lemma} we have that \(x \in \overline{\im(f)}\).
  Hence we conclude that \(X = \overline{\im(f)}\) and thus \(\im(f)\) is a
  countable dense subset of \(X\).

  (SC3) Let \(\mathcal U\) be a cover for \(X\) and define \(\mathcal B' := \{B
  \in \mathcal B : B \subseteq U (U \in \mathcal U)\}\), and \(\mathcal U' :=
  \{U \in \mathcal U : B \subseteq U (B \in \mathcal B')\}\). We'll show that
  \(\mathcal U'\) is a subcover of \(\mathcal U\). Let \(x \in X\) be any point.
  since \(\mathcal U\) covers \(X\) then there exists \(U_x \in \mathcal U\)
  such that \(x \in U_x\). Moreover, since \(\mathcal B\) is a basis for the
  topology of \(X\), then there exists \(B_x \in \mathcal B\) such that \(B_x
  \subseteq U_x\), so that \(B_x \in \mathcal B'\) and \(U_x \in \mathcal U'\)
  from the construction of the collections. Hence \(\mathcal U'\) covers \(X\).
\end{proof}

\begin{proposition}\label{prop: metric space properties}
  Given a metric space \(M\), the following properties are equivalent
  \begin{enumerate}[(MS1)]
    \item \(M\) is second countable.
    \item \(M\) is separable.
    \item \(M\) is Lindelöf.
  \end{enumerate}
\end{proposition}

\begin{proof}
  (Separable \(\Rightarrow\) Lindelöf) Suppose \(M\) is separable and let \(A\)
  be a countable dense set in \(M\). Let \(\mathcal U\) be a cover of \(M\) and
  define the collection \(\mathcal U' := \{U \in \mathcal U : U \supseteq B(a,
  r),\ (a, r) \in A \times \Q\}\), where \(B(a,r)\) is the open ball of
  radius \(r\) around \(a\). We show that \(\mathcal U'\) is a subcover of
  \(M\).
  Let \(x \in M\) be any point. Since \(\mathcal U\) covers \(M\) then there
  exists \(U_x \in \mathcal U\) such that \(x \in U_x\). Since \(M\) is a metric
  space, there exists an open ball \(B(x, \ell) \subseteq U_x\) for some \(\ell
  \in \R\).
  Since \(A\) is dense in \(M\), there exists a point \(a \in A\) such that \(a
  \in B(x, \ell/2)\). Now, from the fact that \(\Q\) is dense in
  \(\R\) we conclude that there exists an \(r \in \Q\) such that
  \(d(x, a) < r < \ell/2\). Notice that \(B(a, r) \subseteq B(x, r) \subseteq
  U_x\), hence \(U_x \in \mathcal U'\). Therefore \(\mathcal U'\) is a subcover
  of \(\mathcal U\), and since \(|\mathcal U'| = |A \times \Q|\), we
  find that \(\mathcal U'\) is countable.

  (Lindelöf \(\Rightarrow\) second countable) Suppose \(M\) is Lindelöf, then
  given a cover \(\mathcal U := \{B(x, \frac{1}{n}) \subseteq M : x \in M, n \in
  \N\}\) of \(M\) there exists a countable subcover \(\mathcal U'\). Let
  \(U \subseteq M\) be any open set and let \(x \in U\) be any point. Since
  \(M\) is a metric space, there exists \(r > 0\) such that \(B(x, r) \subseteq
  U\). Define now \(n \in N\) such that \(\frac 1 n < \frac r 2\). Since
  \(\mathcal U'\) is a cover of \(M\), there exists a \(B(p, \frac 1 n) \in
  \mathcal U'\) such that \(x \in B(p, \frac 1 n)\). We now show that \(B(p,
  \frac 1 n) \subseteq B(x, r)\). Let \(y \in B(p, \frac 1 n)\) be any point,
  then \(d(p, y) < \frac 1 n\), moreover \(d(x, p) < \frac 1 n\), hence we
  conclude that \(d(x, y) \leq d(x, p) + d(p, y) < \frac 2 n < r\). This implies
  in particular that \(y \in B(x, r)\) and in general that \(B(p, \frac 1 n)
  \subseteq B(x, r) \subseteq U\). Hence we conclude that \(U\) can be written
  as a union of elements of \(\mathcal U'\), which implies that \(\mathcal U'\)
  is a countable basis for \(M\).

  By means of \cref{prop: second countable properties} we conclude the
  equivalence chain.
\end{proof}

\subsection{Weights and Cardinality}

\begin{proposition}\label{prop: weight m space subset union}
  Let \(X\) be a topological space and \(w(X) \leq \mathbf m\). Then for every
  collection \(\{U_i\}_{i \in I} \subseteq 2^X\) of open sets, there exists
  \(I_0 \subseteq I\) such that \(|I_0| \leq \mathbf m\) and \(\bigcup_{i \in
  I_0} U_i = \bigcup_{i \in  I} U_i\).
\end{proposition}

\begin{proof}
  Since \(I_0 \subseteq I\) then clearly \(\bigcup_{i \in  I_0} U_i \subseteq
  \bigcup_{i \in  I} U_i\). Let \(\mathcal B\) be a base for \(X\) such that
  \(|\mathcal B| \leq \mathbf m\) and define the collection \(\mathcal B_0 :=
  \{U \in \mathcal B : U \subseteq U_i (i \in I)\}\). Define a function \(f:
  \mathcal B_0 \to I\) such that for all \(U \in \mathcal B_0\) we have \(U
  \subseteq  U_{f(U)} \in \{U_i\}_{i \in I}\). Define the indexing set \(I_0 :=
  f(\mathcal B_0) \subseteq I\). Notice that \(|I_0| \leq |\mathcal B| \leq
  \mathbf m\). For any point \(x \in \bigcup_{i \in  I} U_i\) there exists \(i
  \in S\) such that \(x \in U_i\), and hence exists \(U \in \mathcal B\) such
  that \(x \in U \subseteq U_i\), from the fact that \(\mathcal B\) is a basis.
  From the definition of \(\mathcal B_0\) it follows that \(U \in \mathcal B_0\)
  and hence \(f(U) \in I_0\). Therefore, from the construction of \(f\) it
  follows that \(x \in U \subseteq U_{f(U)} \subseteq \bigcup_{i \in  I_0}
  U_i\). From this we conclude that \(\bigcup_{i \in  I} U_i \subseteq
  \bigcup_{i \in  I_0} U_i\).
\end{proof}

\begin{proposition}
  Let \(X\) be a topological space. If \(w(X) \leq \mathbf m\), then for every base
  \(\mathcal B\) for the topology of \(X\) there exists a base \(\mathcal B_0\)
  such that \(|\mathcal B_0| \leq \mathbf m\) and \(\mathcal B_0 \subseteq
  \mathcal B\).
\end{proposition}

\begin{proof}
  Suppose \(\mathbf m \geq \aleph_0\). Let \(\mathcal B := \{U_i\}_{i \in I}\)
  be any base for the space \(X\). Define a base \(\mathcal B_1 := \{W_t\}_{t
  \in T}\) for \(X\) such that \(|T| \leq \mathbf m\).

  For all \(t \in T\) define the set \(I(t) := \{i \in I : U_i \subseteq
  W_t\}\). From the fact that \(W_t\) is open and \(\mathcal B\) is a base for
  \(X\), we have that \(W_t = \bigcup_{i \in  I(t)} U_i\).  From \cref{prop:
  weight m space subset union} we find that there exists \(I_0(t) \subseteq
  I(t)\) for which \(|I_0(t)| \leq \mathbf m\) and
  \begin{equation}\label{eq: W_t I_0(t)}
    W_t = \bigcup_{i \in  I(t)} U_i = \bigcup_{i \in  I_0(t)} U_i
  \end{equation}

  Now, define the set \(\mathcal B_0 := \{U_i\}_{i \in I(t), t \in T} \subseteq
  \mathcal B\). Since \(|T|, |S_0(t)| \leq \mathbf m\) and the fact that
  \(\mathbf m^2 = \mathbf m\), we find that \(|\mathcal B_0| \leq \mathbf m\).
  We now show that \(\mathcal B_0\) is a base for the space \(X\). Let a any
  point \(x \in X\) and a neighbourhood \(U \subseteq X\) of \(x\). From the
  hypothesis that \(\mathcal B_1\) is a basis, there exists \(t \in T\) such
  that \(W_t \subseteq U\). On the other hand \cref{eq: W_t I_0(t)} ensures the
  existence of an \(i \in I_0(t)\) such that \(U_i \subseteq W_t \subseteq U\).
  Since \(U_i \in \mathcal B_0\), we find that \(\mathcal B_0\) is indeed a
  basis for \(X\).
\end{proof}

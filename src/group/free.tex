\section{Free Groups}

\begin{definition}
  \label{def:group-generated}
  Let \(S\) be a set and \(G\) be a group. If every element of \(G\) can be
  written as the product of finitely many powers of the elements of \(S\), then
  we say that \(G\) is generated by \(S\). Moreover, if \(S\) is finite, we
  naturally say that \(G\) is finitely generated. Sometimes we denote this by
  \(G = \langle S \rangle\).

  If \(\varphi: S \to G\) is a set-function, \(G\) is said to be generated by \(\varphi\) if
  the image \(\im \varphi\) generates the group \(G\).
\end{definition}

Let \(S\) be a set. We define the category \(\cat C\) be a category composed of
objects that are set-functions \(f: S \to G\), denoted \((f, G) \in \cat C\), where
\(G \in \Grp\). Given another object \((g, H) \in \cat C\), a morphism between \(f \to
g\) is a morphisms of groups \(\phi: G \to H\) such that the following diagram
commutes
\[
  \begin{tikzcd}
    &S \ar[dr, "g"] \ar[dl, swap, "f"] & \\
    G \ar[rr, "\phi"] & &H
  \end{tikzcd}
\]

\begin{definition}[Free group]
  \label{def:free-group}
  Given a set \(S\) we define the free group of \(S\) to be the initial object
  \((\iota, F(S))\) in the category \(\cat C\) defined above. In other words,
  \(F(S)\) is said to be the free group of \(S\) if there exists a set-functions
  \(\iota\) such that for all \((f, G) \in \cat C\) there exists a unique morphism \(f
  \to \iota\) given by a morphism of groups \(\phi: F(S) \to G\) such that the diagram
  commmutes
  \[
    \begin{tikzcd}
      S \ar[d, swap, "\iota"] \ar[r, "f"] & G \\
      F(S) \ar[ur, dashed, swap, "\phi"] &
    \end{tikzcd}
  \]
\end{definition}

\begin{lemma}
  \label{lem:isomorphism-indexing-set-group}
  There exists an indexing set \(I\) and a corresponding collection of groups
  \(\{G_{i}\}_{i \in I}\) such that if \(f: S \to G\) is a set-function and \(G\) is
  generated by \(f\) then \(G\) is isomorphic to some \(G_i \in \{G_{i}\}_{i \in I}\).
\end{lemma}

\begin{proof}
  Since \(G = \left\langle \im f \right\rangle\) then if \(S\) is a finite set, it follows
  that \(G\) is enumerable, since \(\im f\) will a finite set of elements of
  \(G\). On the other hand, if \(S\) is infinite, then \(\im f\) can be either
  finite or infinite, on the first case we find that \(G\) is enumerable, on the
  second case \(G\) will be infinite --- hence in general we can assert that \(|G|
  \leq |S|\).

  Define \(X\) to be an enumerable set if \(S\) is finite, and \(|X| = |S|\) if
  \(S\) is infinite. Let \(A \subseteq X\) be a non empty subset and define \(\Gamma(A)\) to
  be the collection of binary relations \(\gamma: A \times A \to A\) such that \((A, \gamma)\) is
  a group. Define \(\mathcal{A} = \{(A, \gamma): A \subseteq X, \gamma \in \Gamma(A)\}\), the set of groups on
  subsets of the set \(X\). The collection \(\mathcal{A}\) suffices the condition
  required.

  This is a rather strange affirmation, but I'll carry an explanation of why it
  makes sense. The set \(X\) is defined so that every possible product of
  elements of \(S\) can be injectively assigned to an element of \(X\), that is,
  for every collection \(B\) of finite sequences of elements of \(S\), there
  exists an injective set-function \(i: B \mono X\). Since \(\left\langle \im f
  \right\rangle\) can be seen set-wise as a collection of finite sequences of elements
  of \(S\), then there exists a bijection \(j: \left\langle \im f \right\rangle \isoto B\) ---
  for some \(B\) as defined above. Since there exists an injection \(i: B \mono
  X\), the restriction map \(i: B \isoto \im i \subseteq X\) is a bijection and hence
  \(ij: \left\langle \im f \right\rangle \isoto \im i\) is also bijective. Then, given \(\gamma \in
  \Gamma(\im i)\) the group \((\im i, \gamma)\) is isomorphic to \(\left\langle \im f \right\rangle =
  G\).
\end{proof}

\begin{proposition}[Every set has a free group]
  \label{prop:universal-free-group}
  Let \(S\) be any set. Then there exists a free group \((\iota, F(S))\), unique up
  to isomorphism, such that \(\iota\) is injective and \(F = \left\langle \im \iota \right\rangle\).
\end{proposition}

\begin{proof}
  Let \(I\) be an indexing set and \(\{G_{i}\}_{i \in I}\) be an indexed
  collection of groups. Define
  \[
    F_0 = \prod_{i \in I} \prod_{\ell \in \Hom_{\Set}(S, G_i)} G_i \times \{\ell\},
  \]
  and consider the set-function \(f_0: S \to F_0\) as the mapping \(s \mapsto (\ell(s),
  \ell)\) for some set-function \(\ell \in \bigcup_{i \in I} \Hom_{\Set}(S, G_i)\).

  Consider \(G\) to be a group and let \(g: S \to G\) generate \(G\). From
  \cref{lem:isomorphism-indexing-set-group} we find that there exists a group
  \(G_j \in \{G_{i}\}_{i \in I}\) such that there is an isomorphism \(\phi: G \isoto
  G_{j}\). Moreover, the set-function \(\psi = \phi g: S \to G_j\) is an element of
  \(\bigcup_{i \in I} \Hom_{\Set}(S, G_i)\). Define the projection map \(\pi_{j, \psi}: F_0 \to
  G_j \times \{\psi\}\). We can now define a map \(\psi_{*} = \phi^{-1} \pi_{j, \psi}: F_0 \to G\)
  such that the following diagram commutes
  \[
    \begin{tikzcd}
      S \ar[r, "f_0"] \ar[d, swap, "g"] &F_0 \ar[d, "\pi_{j, \psi}"]
      \ar[dl, swap, "\psi_{*}"] \\
      G \ar[r, swap, "\phi"] &G_j \times \{\psi\}
    \end{tikzcd}
  \]

  Define the subgroup \(F = \left\langle \im f_{0} \right\rangle\) of \(F_0\), define the
  set-function \(f: S \to F\) to be the equivalent of \(f_0\) --- which makes \(f\)
  an injective function. Also, let \(g_{*}: F \to G\) be the restriction \(g_{*} =
  \psi_{*}|_{F}\). The given maps satisfy \(g = g_{*} f\) and since \(f\) is an
  injective function, it restricts the image of \(g_{*}\) so that it is
  necessarily unique. Summarizing, we may view this construction as the following
  diagram
  \[
    \begin{tikzcd}
      S \ar[r, hook, "f"] \ar[d, swap, "g"] &F \ar[dl, dashed, "g_{*}"] \\
      G &
    \end{tikzcd}
  \]
  This is equivalent to our definition of the free group, thus \((F, f)\) is the
  free group of \(S\) (up to isomorphism).
\end{proof}


%%% Local Variables:
%%% mode: latex
%%% TeX-master: "../../algebra"
%%% End:

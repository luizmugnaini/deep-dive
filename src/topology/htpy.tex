\section{Homotopy}

\begin{definition}[Left homotopy in \(\Top\)]
\label{def:left-homotopy-Top}
Let \(f, g: X \para Y\) be parallel topological morphisms. We define a
\emph{left homotopy} \(\eta: f \nat g\) between \(f\) and \(g\) to be a morphism
\(\eta: X \times I \to Y\) such that the following diagram commutes
\[
\begin{tikzcd}
X \ar[d, rd, "f"'] \ar[r, "i_0"]
& X \times I \ar[d, "\eta"]
& X \ar[l, "i_1"'] \ar[ld, "g"]
\\
& Y
\end{tikzcd}
\]
where \(i_0, i_1: X \to X \times I\) are parallel morphisms with \(x
\xmapsto{i_{0}} (x, 0)\) and \(x \xmapsto{i_{1}} (x, 1)\).

If topological spaces are viewed as categories whose objects are its points and
morphisms are the identities, then topological morphisms \(f, g: X \para Y\) are
functors between the cagories \(X\) and \(Y\) and a homotopy \(\eta: f \nat g\)
is a natural transformation between \(f\) and \(g\), where the following diagram
commutes in the category \(Y\) for any two points \(x, x' \in X\):
\[
\begin{tikzcd}
{f(x)}  \ar[r, "{\eta(x, -)}"] \ar[d, "f"'] &{g(x)} \ar[d, "g"] \\
{f(x')} \ar[r, "\eta{(x', -)}"'] &{g(x')}
\end{tikzcd}
\]
\end{definition}

To put concretely, left homotopy between \(f\) and \(g\) is a continuous map
\(\eta\) such that \(\eta(-, 0) = f\) and \(\eta(-, 1) = g\) --- which can be
visually thought of as a deformation of the morphism \(f\) to \(g\). If \(f\)
and \(g\) are indeed homotopic, we denote this by \(f \simht g\).

\begin{corollary}[Equivalence relation]
\label{cor:htpy-equivalence-rel}
Left homotopy induces an \emph{equivalence relation} \(\simht\) on the
collection of topological morphisms. Moreover, homotopy respects composition of
morphisms.
\end{corollary}

\begin{proof}
First we prove that \(\simht\) is an equivalence relation:
\begin{itemize}\setlength\itemsep{0em}
\item (Reflexive) Given a continuous map \(f: X \to Y\), we explicitly define a
  homotopy \(f \nat f\) by mapping \((x, t) \mapsto x\) for all
  \((x, t) \in X \times I\).

\item (Symmetric) Let \(g: X \to Y\) be another continous map and suppose that
  there exists a homotopy \(\eta: f \nat g\). We can define a homotopy \(\eta':
  g \nat f\) by mapping \((x, t) \mapsto \eta(x, 1 - t)\) --- so that \(\eta'(-,
  0) = \eta(-, 1) = g\) and \(\eta'(-, 1) = \eta(-, 0) = f\).

\item (Transitive) Consider yet another morphism \(h: X \to Y\) and suppose that
  there exists a homotopy \(\sigma: g \nat h\). In order to construct a homotopy
  \(\delta: f \nat h\), we define a \(X\)-parametrized path by concatenating
  \(\eta\) and \(\sigma\) appropriately
  \[
  \delta(x, t) \coloneq
  \begin{cases}
    \eta(x, 2 t), & t \in [0, 1/2], \\
    \sigma(x, 2 t - 1), & t \in [1/2, 1].
  \end{cases}
  \]
  Indeed, \(\delta(-, 0) = \eta(-, 0) = f\) while
  \(\delta(-, 1) = \sigma(-, 1) = h\). It remains to prove that \(\delta\) is
  continuous. Notice that the sets \(A \coloneq X \times [0, 1/2]\) and
  \(B \coloneq X \times [1/2, 1]\) are both closed in the product topology and
  cover the whole space \(X \times I\). By
  \cref{prop:continuous-from-covering-subspaces}, since \(\delta\)
  is continuous on both \(A\) and \(B\) then \(\delta\) is continuous on
  \(A \cup B = X \times I\).
\end{itemize}
To prove that homotopy preserves composition, consider the following commutative
diagram in \(\Top\)
\[
\begin{tikzcd}
X \ar[r, "f"]
&Y \ar[r, shift left, "g"] \ar[r, shift right, "g'"']
&Z \ar[r, "h"]
&W
\end{tikzcd}
\]
If there exists a homotopy \(\eta: g \nat g'\), we want to show that \(h g f\)
is homotopy equivalent to \(h g' f\). To do that, consider the following
commutative diagram
\[
\begin{tikzcd}
X \ar[r, "i_0"] \ar[d, "f"']
&X \times I \ar[d, "f \times \Id_I"]
&X \ar[l, "i_1"'] \ar[d, "f"]
\\
Y \ar[r, "i_0"] \ar[dr, "g"']
&Y \times I \ar[d, "\eta"]
&Y \ar[l, "i_1"'] \ar[dl, "g'"]
\\
&Z \ar[d, "h"] &
\\
&W &
\end{tikzcd}
\]
From the diagram we see that a natural choice of homotopy
\(\sigma: h g f \nat h g' f\) is given by \(\sigma = h \eta (f \times \Id_I)\)
thus indeed \(h g f \simht h g' f\).
\end{proof}

Given any two topological spaces \(X\) and \(Y\), we denote the
collection of continuous maps \(X \to Y\) \emph{up to homotopy equivalence} by
\[
[X, Y] \coloneq \Hom_{\Top}(X, Y)/{\simht}.
\]
We call a morphism \([f] \in [X, Y]\) a \emph{homotopy class}. Moreover, for any
three spaces \(X, Y, Z \in \Top\), there exists a \emph{unique} compositional
map
\[
[X, Y] \times [Y, Z] \longrightarrow [X, Z]
\]
such that the following diagram commutes
\[
\begin{tikzcd}
\Hom_{\Top}(X, Y) \times \Hom_{\Top}(Y, Z)
\ar[rr] \ar[dd, two heads]
&&\Hom_{\Top}(X, Z) \ar[dd, two heads]
\\ && \\
{[X, Y] \times [Y, Z]} \ar[rr, dashed]
&& {[X, Z]}
\end{tikzcd}
\]
That is, given morphisms \(X \xrightarrow{f} Y \xrightarrow{g} Z\), we have a
composition of \([f]\) with \([g]\) uniquely defined as
\[
[g] \circ [f] \coloneq [g f].
\]

\begin{definition}[Homotopy category]
\label{def:Ho(Top)}
We therefore define a category \(\HoTop\) composed of topological spaces ---
which, viewed as an object of \(\HoTop\), is called a \emph{homotopy type} ---
and classes of continuous morphisms between them up to homotopy.
\end{definition}

This quotienting operation on the category of topological spaces induces a
canonically defined projective functor
\[
\kappa: \Top \longrightarrow \HoTop.
\]

\begin{definition}[Homotopy equivalence]
\label{def:homotopy-equivalence}
Let \(X\) and \(Y\) be topological spaces. We say that a continuous map
\(f: X \to Y\) is a \emph{homotopy equivalence} of \(X\) and \(Y\) if there
exists a continuous map \(g: Y \to X\) and homotopies \(f g \nat \Id_Y\) and
\(g f \nat \Id_X\).

If there exists such homotopy equivalence, we write that \(X \isoht Y\) --- it
is to be noted that homotopy equivalences are exactly the isomorphisms in the
homotopy category \(\HoTop\).
\end{definition}

\begin{corollary}
\label{cor:homeomorphism-is-homotopy-equivalence}
Every topological isomorphism is a homotopy equivalence.
\end{corollary}

\begin{proof}
Let \(f: X \isoto Y\) be a topological isomorphism. We consider its image
under the functor \(\Top \to \HoTop\), which we'll name \([f]: X \to Y\). Then
since \(f^{-1}: Y \isoto X\) is also a continuous morphism, we can consider its
class \([f^{-1}]\) and notice that \([f] [f^{-1}] = [f f^{-1}] = [\Id_Y]\) and
\([f^{-1}] [f] = [f^{-1} f] = [\Id_X]\). Therefore, there exists two homotopies
\(f f^{-1} \nat \Id_Y\) and \(f^{-1} f \nat \Id_X\).
\end{proof}

\begin{corollary}
\label{cor:htpy-equiv-is-equiv-relation}
Homotopy equivalence is an equivalence relation on the class of topological
spaces.
\end{corollary}

\subsection{Contractibility}

\begin{definition}[Contractible space]
\label{def:contractible-space}
A space \(X\) is said to be \emph{contractible} if the unique continuous map
\(X \to *\) is a homotopy equivalence. From its construction, contractible
spaces are the terminal objects of \(\HoTop\).
\end{definition}

\begin{example}[A ball is contractible]
\label{exp:ball-contractible}
Consider the open (or closed) ball \(B^n \subseteq \R^n\). Let \(f: B^n \to *\)
be the unique continous map from the ball to the point space, and
\(\iota: * \to B^n\) be the map \(* \mapsto 0\). We define a homotopy
\(\eta: \iota f \nat \Id_{B^n}\) given by \((p, t) \mapsto t p\), and a homotopy
\(\sigma: f \iota \nat \Id_{*}\) simply given by \((*, t) \mapsto *\) --- since
\(f \iota = \Id_{*}\). Therefore \(B^n\) is contractible.

Furthermore, since \(B^n \iso \R^n\) in \(\Top\), it follows that \(\R^n \isoht
B^n\) and thus \(\R^n \isoht *\) --- the euclidean space is contractible.
\end{example}

\section{Fundamental Groupoid \& The Fundamental Group}

\subsection{Paths}

\begin{notation}[Family of paths]
\label{not:family-of-paths}
Let \(X\) be a space and \(x, y \in X\) be any two points. We'll denote by
\(\Path_X(x, y)\) the family of paths \(\gamma: I \to X\) with \(\gamma(0) = x\) and \(\gamma(1) = y\).
\end{notation}

\begin{definition}[Operations on paths]
\label{def:operations-paths}
Let \(X\) be a topological space. We define the following operations on the
space of paths of \(X\):
\begin{itemize}\setlength\itemsep{0em}
\item Given a path \(\gamma: I \to X\) we define the \emph{reverse} path of
  \(\gamma\) to be a path \(\gamma^{-1}: I \to X\) given by
  \(\gamma^{-1}(t) \coloneq \gamma(1 - t)\).

\item If \(p \in \Path_X(x, y)\) and \(q \in \Path_X(y, z)\) are paths in \(X\),
  we define the \emph{concatenation} of \(p\) with \(q\) to be a path \(q \cdot
  p: I \to X\) given by
  \[
  (q \cdot p)(t) \coloneq
  \begin{cases}
    p(2 t), &t \in [0, 1/2], \\
    q(2 t - 1), &t \in [1/2, 1].
  \end{cases}
  \]
  Yielding a path \(q \cdot p \in \Path_X(x, z)\).

\item Moreover, we define \(\const_x: I \to X\) to be the unique constant path
  on \(x\) --- that is, \(\const_x(t) = x\) for all \(t \in I\).
\end{itemize}
\end{definition}

\begin{definition}[Homotopy relative boundary]
\label{def:htpy-relative-boundary}
Let \(X\) be a space, and \(\gamma, \gamma': I \para X\) be paths with common
endpoints:
\begin{align*}
 \gamma(0) = \gamma'(0) \eqqcolon p_0\quad \text{ and }\quad
 \gamma(1) = \gamma'(1) \eqqcolon p_1.
\end{align*}
We define a \emph{homotopy relative boundary} between \(\gamma\) and \(\gamma'\)
to be a homotopy \(\eta: \gamma \nat \gamma'\) such that \(\eta\) is constant on
the endpoints \(p_0\) and \(p_1\) --- that is,
\begin{align*}
  \eta(0, -) = \const_{p_0}\quad \text{ and }\quad \eta(1, -) = \const_{p_1}.
\end{align*}
\end{definition}

\begin{proposition}
\label{prop:htpy-rel-boundary-is-equiv-relation}
Homotopy relative boundary is an equivalence relation on the family of paths of
a topological space.
\end{proposition}

\begin{proof}
Let \(X\) be a topological space and \(x, y \in X\) be any two points --- we'll
consider the family \(\Path_X(x, y)\). The constant homotopy \(\gamma \nat
\gamma\) is clearly a homotopy relative boundary, thus the relation is
reflexive.

If \(\delta \in \Path_X(x, y)\) is another path, and \(\eta: \gamma \nat
\delta\) is a homotopy relative boundary, then we can construct the a reverse
homotopy \(\sigma: \delta \nat \gamma\) as \(\sigma(-, t) \coloneq \eta(-, 1 -
t)\). Notice that \(\sigma(0, t) = \eta(0, 1 - t) = \const_x(1 - t)\) is
constant on \(x\), while \(\sigma(1, t) = \eta(1, 1 - t) = \const_y(1 - t)\) is
constant on \(y\). Therefore the relation is symmetric.

Consider yet another path \(p \in \Path_X(x, y)\) and a homotopy relative
boundary \(\eta: \delta \nat p\). We define a map \(\varepsilon: I \times I \to
X\) given by
\[
\varepsilon(-, t) \coloneq
\begin{cases}
  \sigma(-, 2t), &t \in [0, 1/2], \\
  \eta(-, 2 t - 1), &t \in [1/2, 1].
\end{cases}
\]
This map is continous by the same argument used in
\cref{cor:htpy-equivalence-rel} and thus stablishes a homotopy relative boundary
\(\varepsilon: \gamma \nat p\).
\end{proof}

\subsection{The Fundamental Groupoid}

\begin{definition}[Fundamental groupoid]
\label{def:fundamental-groupoid}
Given a topological space \(X\), we define the \emph{fundamental groupoid} of
\(X\) to be the category \(\Pi_1(X)\) whose objects are the points of \(X\), and
whose morphisms are paths between those points up to homotopy relative boundary.
That is, given \(x, y \in X\) we have
\[
\Hom_{\Pi_1(X)}(x, y) = \Path_X(x, y)/{\sim_{\text{hrb}}},
\]
where \(\sim_{\text{hrb}}\) is the equivalence relation on the family of paths
\(x \to y\) given by homotopy relative boundary.

Composition of morphisms in \(\Pi_1(X)\) is naturally defined by the
concatenation of paths --- in other words, given points \(x, y, z \in X\) we
have
\[
\begin{tikzcd}
{\Path_X(x, y) \times \Path_X(y, z)}
\ar[d, two heads] \ar[rr]
&&{\Path_X(x, z)} \ar[d, two heads]
\\
{\Hom_{\Pi_1(X)}(x, y) \times \Hom_{\Pi_1(X)}(y, z)}
\ar[rr, dashed]
&&{\Hom_{\Pi_1(X)}(x, z)}
\end{tikzcd}
\]
where, for any paths \(\gamma \in \Path_X(x, y)\) and \(\delta \in \Path_X(y,
z)\), the concatenation of paths \((\gamma, \delta) \mapsto \delta \cdot \gamma\) induces
a concatenation operation on the respective class paths
\[
([\gamma], [\delta]) \longmapsto
[\delta] \cdot [\gamma] \coloneq [\delta \cdot \gamma].
\]
\end{definition}

\begin{corollary}
\label{cor:Pi1-is-groupoid}
\(\Pi_1(X)\) is a groupoid.
\end{corollary}

\begin{proof}
We show that \(\Pi_1(X)\) is a category whose morphisms are isomorphisms.
\begin{itemize}\setlength\itemsep{0em}
\item Given any point \(x \in X\) one has an identity map
  \([\const_x] \in \Hom_{\Pi_1(X)}(x, x)\).

\item Concatenation of class paths is unital with respect to constant paths:
  given any path class \([\gamma] \in \Hom_{\Pi_1(X)}(x, y)\), we have
  \[
  [\const_y] \cdot [\gamma]
  = [\const_y \cdot \gamma]
  = [\gamma]
  = [\gamma \cdot \const_x]
  = [\gamma] \cdot [\const_x].
  \]

\item Concatenation is associative. Let \(x, y, z, w \in X\) be any four points
  and consider class paths \([\alpha]: x \to y\), \([\beta]: y \to z\), and
  \([\gamma]: z \to w\).

  First we have to show that \(\gamma \cdot (\beta \cdot \alpha)\) and
  \((\gamma \cdot \beta) \cdot \alpha\) are relative boundary homotopic. To that
  end, define a continuous map \(\tau: I \to I\) by
  \[
  \tau(t) \coloneq
  \begin{cases}
    2 t, &t \in [0, 1/4], \\
    t + \frac{1}{4}, &t \in [1/4, 1/2], \\
    \frac{t}{2} + \frac{1}{2}, &t \in [1/2, 1].
  \end{cases}
  \]
  Then, one sees right away that
  \[
  (\gamma \cdot (\beta \cdot \alpha))(\tau(t))
  = ((\gamma \cdot \beta) \cdot \alpha)(t)
  \]
  for all \(t \in I\) --- hence \(\gamma \cdot (\beta \cdot \alpha)
  \sim_{\text{hrb}} (\gamma \cdot \beta) \cdot \alpha\). From this, we finally
  obtain the associativity of the path classes,
  \[
  [\gamma] \cdot ([\beta] \cdot [\alpha])
  = [\gamma \cdot (\beta \cdot \alpha)]
  = [(\gamma \cdot \beta) \cdot \alpha]
  = ([\gamma] \cdot [\beta]) \cdot [\alpha].
  \]

\item Every path class \([\gamma] \in \Hom_{\Pi_1(X)}(x, y)\) is an isomorphism,
  since it has an inverse \([\gamma^{-1}] \in \Hom_{\Pi_1(X)}(y, x)\). Indeed,
  one has
  \[
  [\gamma] \cdot [\gamma^{-1}] = [\gamma \cdot \gamma^{-1}] = [\const_y]
  \quad\text{ and }\quad
  [\gamma^{-1}] \cdot [\gamma] = [\gamma^{-1} \cdot \gamma] = [\const_x].
  \]
\end{itemize}
\end{proof}

\begin{definition}[The category \(\bpTop\)]
\label{def:base-point-preserving-Top-cat}
A \emph{pointed topological space} is a pair \((X, x)\) consisting of a
topological space \(X\) together with a base-point \(x \in X\). We define a
category \(\bpTop\) whose objects are pointed topological spaces, and whose
morphisms \(f: (X, x) \to (Y, y)\), for any \((X, x), (Y, y) \in \bpTop\), are
continuous maps \(f: X \to Y\) such that \(f(x) = y\). We say that the morphisms
of \(\bpTop\) preserve base-points.

In \(\bpTop\), we define a \emph{pointed homotopy} \(\eta: f \nat g\) between
parallel morphisms \(f, g: (X, x) \para (Y, y)\) to be a homotopy preserving
base-points, that is,
\[
\eta(x, -) = \const_y.
\]
Analogously to homotopies in \(\Top\), pointed homotopies define an equivalence
relation \(\sim_{\text{ph}}\) in \(\bpTop\). We define the homotopy category of
\(\bpTop\) to be the category \(\Ho{\bpTop}\) composed of pointed topological
spaces and morphisms
\[
\Hom_{\Ho{\bpTop}}((X, x), (Y, y))
\coloneq \Hom_{\bpTop}((X, x), (Y, y))/{\sim_{\text{ph}}}.
\]
This quotienting induces a natural projective functor
\[
\kappa^{*/}: \bpTop \longrightarrow \Ho{\bpTop}.
\]
\end{definition}

\subsection{The Fundamental Group}

\begin{definition}[Fundamental group]
\label{def:fundamental-group}
Let \(X\) be a topological space and \(x \in X\) be any point. We define
the \emph{fundamental group} of \(X\) at the base-point \(x\) as the family of
\emph{loops}
\[
\pi_1(X, x) \coloneq \Aut_{\Pi_1(X)}(x),
\]
endowed with the operation of concatenation of paths. Therefore, we see that the
fundamental group is a functor
\[
\pi_1: \bpTop \longrightarrow \Grp.
\]
\end{definition}

\begin{definition}[Pushforwards in \(\pi_1\)]
\label{def:pushforward-pi1}
Let \(f: (X, x) \to (Y, y)\) be a morphism of pointed topological spaces. There
exists a naturally defined pushforward
\[
f_{*}: \pi_1(X, x) \longrightarrow \pi_1(Y, y)
\]
mapping \([\gamma] \mapsto [f \gamma]\), which stablishes a group morphism
between the fundamental groups of the inicial pointed topological spaces.
\end{definition}


\begin{proposition}
\label{prop:pointed-htpy-pushforward-agree}
Let \(f, g: (X, x) \para (Y, y)\) be morphisms in \(\bpTop\). If there exists a
\emph{pointed homotopy} \(\eta: f \nat g\), then \(f_{*} = g_{*}\).
\end{proposition}

\begin{proof}
Let \(\gamma\) be a loop at \(x\) representing some class of \(\pi_1(X,
x)\). The pointed homotopy \(\eta\) naturally induces a homotopy
\(f \gamma \nat g \gamma\) --- thus \([f \gamma] = [g \gamma]\) in
\(\pi_1(Y, y)\) --- therefore \(f_{*} = g_{*}\).
\end{proof}

By \cref{prop:pointed-htpy-pushforward-agree} we obtain a factorization of the
fundamental group functor through the homotopy category of pointed topological
spaces. To put briefly, the following diagram is quasi-commutative
\[
\begin{tikzcd}
\bpTop \ar[r, "\pi_1"] \ar[d, "\kappa^{*/}"'] &\Grp \\
\Ho{\bpTop} \ar[ru, bend right]
\end{tikzcd}
\]

\begin{corollary}[Preserving isomorphisms]
\label{cor:pushforward-preserve-isomorphism}
If \(f: (X, x) \isoto (Y, y)\) is a \emph{homotopy equivalence} in \(\bpTop\),
then \(f_{*}: \pi_1(X, x) \isoto \pi_1(Y, y)\) is an \emph{isomorphism} in
\(\Grp\).
\end{corollary}

\begin{proof}
If \(g\) is the homotopy inverse of \(f\), then for any
\([\gamma] \in \pi_1(X, x)\) we have
\(g_{*}f_{*} [\gamma] = g_{*}[f \gamma] = [g f \gamma] = [\gamma]\), therefore
\(g_{*} f_{*} = \Id_{\pi_1(X, x)}\). Analogously we have
\(f_{*} g_{*} = \Id_{\pi_1(Y, y)}\). Therefore \(f_{*}\) is an isomorphism of
groups.
\end{proof}

\subsection{Simply Connected Spaces}

\begin{definition}[Simply connected space]
\label{def:simply-connected}
A topological space \(X\) is said to be \emph{simply connected} if it is path
connected --- that is, \(\pi_0(X) = *\) --- and its fundamental group is
trivial, \(\pi_1(X, x) \iso 1\) for any base-point \(x \in X\).
\end{definition}

\todo[inline]{Continue on semi-locally simply connected spaces}

\subsection{Examples of Fundamental Groups}

\begin{example}[\(\pi_1\) of the euclidean space]
\label{exp:euclidean-space-pi1-is-trivial}
Let \(x \in \R^n\) be any point. Given any loop \(\ell: x \to x\), one can
define a homotopy relative boundary \(\eta: \ell \nat \const_x\) given by
\(\eta(s, t) \coloneq (1 - t) \ell(s) + t x\). Therefore the fundamental group
of the \(n\)-dimensional euclidean space is trivial,
\[
\pi_1(X, x) = *.
\]
\end{example}

\begin{proposition}
\label{prop:pi1-manifold-is-countable}
The fundamental group of a topological manifold is countable.
\end{proposition}

\begin{proof}
Let \(\mathcal{B}\) be a countable open cover of \(M\) consisting of coordinate
balls. Since \(M\) has countably many connected components (see
\cref{prop:connectivity-manifolds}), those connected components are also path
components. This implies that any two \(B, B' \in \mathcal{B}\) are such that
the intersection \(B \cap B'\) are composed of at most countably many path
components (since \(M\) is locally path connected). Let \(\mathcal{X}\) be a
collection containing a single point from each \(B \cap B'\) for any pair
\((B, B') \in \mathcal{B} \times \mathcal{B}\). For every \(B \in \mathcal{B}\)
and points \(x, y \in \mathcal{X}\) with \(x, y \in B\), fix a path
\(\gamma_{(x, y)}^B \in \Path_B(x, y)\).

Clearly, the fundamental groups of a path connected component of \(M\) based at
any two points are always isomorphic. Since \(\mathcal{X}\) contains at least
one point of each path component of \(M\), we may choose a base-point
\(p \in \mathcal{X}\). Let \(\Gamma_p\) denote the collection of loops based at
\(p\) that are equal to a finite concatenation of paths of the form
\(\gamma_{(x, y)}^B\), for some \(B \in \mathcal{B}\). Since \(\mathcal{B}\) is
countable, so is \(\Gamma_p\). We'll settle to prove that every element of
\(\pi_1(M, p)\) can be represented by a loop in \(\Gamma_p\).

Let \(f\) be any loop based at \(p\). Consider the collection
\((f^{-1}(B))_{B \in \mathcal{B}}\), which is an open cover of \(I\). Since
\(I\) is compact, there exists a finite subcover out of such collection. The
finite subcover gives rise to a finite collection of numbers
\[
0 \eqqcolon a_0 < a_1 < \dots < a_k \coloneq 1
\]
for which the closed interval \([a_{j-1}, a_j]\) is contained in some
\(f^{-1}(B)\).

For each \(0 < j \leq k\), define \(f_j: I \to M\) to be the path given by
\[
f_j(t) \coloneq f((1 - t) a_{j-1} + t a_j),
\]
that is, \(f_j\) is a path \(f(a_{j-1}) \to f(a_j)\) --- and let
\(B_j \in \mathcal{B}\) be such that \(\im f_j \subseteq B_j\). From
construction we have that \(f(a_j) \in B_j \cap B_{j+1}\) for each
\(0 \leq j < k\).

For each \(0 \leq j < k\) let \(g_j \in \Path_{B_j \cap B_{j+1}}(x_j, f(a_j))\),
where \(x_j \in \mathcal{X}\) as in the first paragraph --- and \(x_0, x_k
\coloneq p\), with constant paths \(g_0, g_k \coloneq \const_p\). Notice that
\begin{align*}
f &\simht f_k \dots f_1 \\
  &\simht g_k^{-1} f_k (g_{k-1} g_{k-1}^{-1}) \dots (g_2 g_2^{-1}) f_2 (g_1 g_1^{-1})
    f_1 g_0 \\
  &\simht (g_k^{-1} f_k  g_{k-1}^{-1}) \dots (g_1^{-1} f_1 g_0), \\
\end{align*}
where \(g_j^{-1} f_j g_{j-1} \in \Path_{B_j}(x_{j-1}, x_j)\) for all
\(0 < j \leq k\). Since each \(B_j\) is simply connected, then
\(g_j^{-1} f_j g_{j-1}\) is relative boundary homotopic to chosen path
\(\gamma_{(x_{j-1}, x_j)}^B\). This shows that \(f\) is homotopic to a path in
\(\Gamma_p\) --- therefore \(\pi_1(M, p)\) is countable.
\end{proof}

%%% Local Variables:
%%% mode: latex
%%% TeX-master: "../../deep-dive"
%%% End:

\section{Universal Properties}

\subsection{Initial and final objects}

\begin{definition}[Initial and Final]\label{def: initial and final objects}
   Given a category \(\cat C\), an object \(I \in \cat C\) is said to be an
   \emph{initial object} in the category \(\cat{C}\) if for all \(A \in
   \cat C\), there exists exactly one morphism \(f \in \Hom(I, A)\) so that
   \(\Hom(I, A) = f\). Conversely, an object \(F \in \cat C\) is said to be an
   \emph{final object} in \(\cat C\) if there is exactly one morphism \(g \in
   \Hom(A, F)\) for all given \(A \in \cat C\) and thus \(\Hom(A, F) = g\).
\end{definition}

\begin{proposition}
   Let a category \(\cat{C}\), then initial and final objects are said to be
   unique up to a unique isomorphism:
   \begin{enumerate}[I.]
      \item If \(I',I'' \in \Obj(\cat{C})\) are initial objects of the
         category, then \(I' \iso I''\), where the isomorphism \(\varphi_I: I'
         \to I''\) is unique. 
      \item If \(F',F'' \in \Obj(\cat{C})\) are initial objects of the
         category, then \(F' \iso F''\), where the isomorphism \(\varphi_F: F'
         \to F''\) is unique. 
   \end{enumerate}
\end{proposition}

\begin{proof}
   Since \(I', I''\) are both initial objects of  \(\cat{C}\) then exists a
   unique morphism \(f \in \Hom(I', I'')\) and \(\exists! g \in \Hom(I'', I')\)
   from which we can compose and conclude that \(f g = \Id_{I''}\) and \(gf =
   \Id_{I'}\) and thus \(f\) and \(g\) are isomorphisms and also \(f^{-1} = g\)
   and \(g^{-1} = f\).  For final objects the same reasoning works and thus the
   proof will be omitted.
\end{proof}

We normally say that a given construction satisfies a universal property if such
construction is a terminal object of the category.

\subsection{Quotients}

Denote \(\sim\) an equivalence relation between elements of a given object
\(A\) in the category \(\Set\). We say that the quotient \(A/\sim\) is
universal with respect to the property that the image of equivalent elements
under morphism is equal. 

Moreover, the objects of the category we'll be dealing will be objects
\(\varphi \in \Hom(A, Z)\) and the morphisms between objects \(\varphi : A \to
Z\) and \(\varphi' : A \to Z'\) is defined to be the commutative diagram
\[
  \begin{tikzcd}
      &A \ar[dl, swap, "\varphi"] \ar[dr, "\varphi'"] & \\
    Z \ar[rr, "\psi"] & &Z'
  \end{tikzcd}
\]

\begin{proposition}
   The canonical projection \(\pi : A \to A/\sim\) is an initial object of the
   category.
\end{proposition}

\begin{proof}
   Let a morphism \(\varphi: A \to Z\). The claim says that there exists a
   unique morphism \(\tilde{\varphi} : A/\sim \to Z\) such that the diagram
   commutes
    \[
      \begin{tikzcd}
        A \ar[rr, "\varphi"] \ar[dr, swap, "\pi"] 
          & 
            & Z \\
          & A/\sim \ar[ur, swap, "\widetilde \varphi"]
      \end{tikzcd}
   \] 
   From the imposition that such diagram commutes, we are inclined to define the
   morphism \(\tilde\varphi\) such that for every element \(a \in A\) it
   satisfies \(\tilde\varphi(\overline{a}) = \varphi(a)\), so that
   \(\tilde\varphi \circ \pi (a) = \varphi(a)\). Since the projection and
   \(\varphi\) are essentially unique, it is clear that if \(\tilde\varphi\)
   exists, then it should also be unique. We now show that we can construct the
   morphism \(\tilde\varphi\). Notice that if \(\overline{a} = \overline{a'}\) 
   then \(\tilde\varphi(\overline{a}) = \varphi(a)= \varphi(a') =
   \tilde\varphi(\overline{a'})\) thus indeed the morphism is well defined.
\end{proof}

\subsection{Products}

\begin{proposition}\label{universal property for products}
   Let objects \(X, Y\) in a given category \(\cat{C}\) and projections \(\pi_X
   \in \Hom(X \times Y, X)\) and \(\pi_Y \in \Hom(X \times Y, Y)\). Let any
   object \(Z\) in the category and set \(\varphi \in \Hom(Z, X)\) and \(\psi
   \in \Hom(Z, Y)\). Then there exists a unique morphism \(\ell \in \Hom(Z, X
   \times Y)\) such that the following diagram commutes
   \[
      \begin{tikzcd}
          & Z 
          \ar[d, dashed, "\ell"] 
          \ar[ddl, bend right, swap, "\varphi"] 
          \ar[ddr, bend left, "\psi"]
            & \\
          & X \times Y \ar[dl, "\pi_X"] \ar[dr, "\pi_Y"] \\
        X 
          & 
            & Y
      \end{tikzcd}
   \] 
   this is the so called \emph{universal property for products}.
\end{proposition}

\begin{proof}
   Define the morphism \(\ell(z) := (\varphi(z), \psi(z))\) for any  \(z \in
   Z\), then \(\pi_X \circ \ell (z) = \pi_X(\varphi(z), \psi(z)) = \varphi(z)\) 
   and the same being true for the left branch. Notice that trivially there is a
   unique mapping with such property because it need to imitate the image of
   both \(\varphi\) and \(\psi\).
\end{proof}

\begin{proposition}[Fibered products]
   Let \(X, Y, Z\) objects of a category \(\cat C\) and morphisms \(\varphi :X
   \to Z\) and \(\psi : Y \to Z\). Define \(P\) to be an object such that the
   morphism  \(\pi_X : P \to X\) and  \(\pi_Y : P \to Y\) are such that, for any
   morphisms \(f : L \to X\) and  \(g : L \to Y\), where \(L\) is another object
   of the category \(\cat C\), there exists a unique morphism \(\ell: L \to P\) 
   for which the diagram commutes
   \[
     \begin{tikzcd}
       L
       \ar[dr, dashed, "\ell"]
       \ar[drr, bend left, "g"]
       \ar[ddr, bend right, swap, "f"]
        &
          & \\
        &P
        \ar[r, "\pi_Y"]
        \ar[d, swap, "\pi_X"]
          &Y
          \ar[d, "\psi"]
          \\
        &X
        \ar[r, swap, "\varphi"]
          &Z
     \end{tikzcd}
   \] 
   We call the object \(P\) the \emph{fiber product of \(X\) and \(Y\) over \(Z\)}.
\end{proposition}

\begin{proof}
   Indeed we can construct such object \(P\), for instance, notice that firstly
   the morphisms \(\varphi \circ \pi_X = \psi \circ \pi_Y\). Thus for any
   given element \(p \in P\), we need \(\pi_X(p) = x \in X\) such that
   \(\varphi(x) = \psi(\pi_Y(p))\), and also the converse should be true.
   Therefore, we simply need to define the object \(P\) as 
   \[
      P := \{(a, b) \in X \times Y : \varphi(a) = \psi(b)\} 
   \] 
   and the morphisms as simply projections of such set into the respective
   object \(X\) or \(Y\).

   With this description, we can see that \(\ell\) should be constructed in such
   a way that for any given morphism \(f, g\) shown in the diagram the following
   holds: \(\pi_X \circ \ell = \pi_Y \circ \ell\) thus, it is simply necessary
   to impose \(\ell(u) = (f(u), g(u))\) for all \(u \in L\).
\end{proof}

\subsection{Coproduct}

Since products are seen as final objects in the category of interest, we see
coproducts (or disjoint unions) as initial objects of the category.

\begin{proposition}
   Let objects \(X, Y\) in a category \(\cat C\) and let the morphisms
   \(\iota_X : X \to X \amalg Y\) and \(\iota_Y : Y \to X \amalg Y\). Let now
   another object \(Z\) and assume the morphisms \(\varphi : X \to Z\) and
   \(\psi : Y \to Z\). Then there exists a unique morphism \(\ell : X \amalg Y
   \to Z\) such that the diagram commutes:
    \[
      \begin{tikzcd}
        X
        \ar[ddr, bend right, swap,  "\varphi"]
        \ar[dr, "\iota_X"]
          &
            &Y
            \ar[dl, swap, "\iota_Y"]
            \ar[ddl, bend left, "\psi"]
            \\
          &X \amalg Y
          \ar[d, dashed, "\ell"]
            &\\
          &Z
      \end{tikzcd}
   \] 
\end{proposition}

\begin{proof}
   \todo[inline]{Coproduct universal property}  
\end{proof}
   
\begin{proposition}
   The disjoint union is a coproduct in \(\Set\).
\end{proposition}

\begin{proof}
    Since the disjoint union of sets \(X, Y\) is said to be the union of,
    respectively, isomorphic copies \(X', Y'\) we can proceed by defining  \(X'
    := \{0\} \times X\) and \(Y' := \{1\} \times Y\), which will do the work
    for us. Next we simply define the inclusions \(\iota_X(a) := (0, a)\) and
    \(\iota_Y(b) := (1, b)\) for all \(a \in X\) and \(b \in Y\). For the final
    step, we define the unique function \(\ell\) as \(\ell((0,a)) := \varphi(a)
    \in Z\) for \(a \in X\) and \(\ell((1, b)) := \psi(b) \in Z\)  for \(b \in
    Y\). All defined functions are clearly well defined and \(\ell\) needs to be
    unique because it imitates the image of \(\varphi\) and \(\psi\).
\end{proof}

\begin{proposition}[Cofibered Product]
   Let objects \(X, Y, Z\) in a category \(\cat C\) and morphisms \(\varphi : Z
   \to X\) and \(\psi : Z \to Y\). Let the inclusions \(\iota_A : A \to A \amalg
   B\) and  \(\iota_B: B \to A \amalg B\) and define the object 
    \[
       P := A \amalg B / (\iota \circ \varphi(z) \sim \iota \circ \psi(z),\
       \forall z \in Z).
   \] 
   With maps \(j_A : A \xrightarrow{\iota_A} A \amalg B \to P\) and  \(j_B : B
   \xrightarrow{\iota_B} A \amalg B \to P\). Then there exists a unique morphism
    \(\ell : P \to L\) for which the diagram commutes
     \[
       \begin{tikzcd}
         Z
         \ar[r, "\psi"]
         \ar[d, swap, "\varphi"]
           &Y
           \ar[d, "j_Y"]
           \ar[ddr, bend left, "g"]
             &\\
         X
         \ar[r, "j_X"]
         \ar[drr, bend right, swap, "f"]
           &P
           \ar[dr, dashed, "\ell"]
             &\\
           &
             &L
       \end{tikzcd}
    \] 
    We call \(P\) the fiber coproduct of \(X\) and \(Y\) over \(Z\)
\end{proposition}

\begin{proof}
   Notice that we are imposing \(\ell \circ j_X = f\) and \(\ell \circ j_Y =
   g\), thus  
   \todo[inline]{Cofibered product universal property}
\end{proof}

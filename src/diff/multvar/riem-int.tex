\section{Riemann Integration of Real Valued Maps}

\subsection{Primary Definitions}

The main setting we are going to be working in this section, which will
encompass the study of multiple Riemann integrals, is the standard euclidean
space \(\R^n\) and the \(n\)-dimensional closed intervals \(I = [a, b] = \{x \in
\R^n \colon a_j \leq x_j \leq b_j \text{ for } 1 \leq j \leq n\}\). If it seems fit,
we can denote that a point \(x \in \R^n\) lies in the interval generated by
given points \(a, b \in \R^n\) by simply saying that \(a \leq x \leq
b\). Another terminology we are going to adopt is that, the interval \(I\) is
non-degenerate if \(a_j < b_j\) for all \(1 \leq j \leq n\).

\begin{notation}
In \emph{this} section we denote by \(\mathcal I^n\) the collection of all
\emph{closed} intervals of \(\R^n\).
\end{notation}

\begin{definition}[Interval measure]
\label{def:interval-measure}
We define the map \(\Vol: \mathcal I^n \to \R\) as the \emph{measure} (or
volume) of the \(n\)-dimensional closed intervals of \(\R^n\), it's defined as
the product of the interval sides, that is
\[
  \Vol I \coloneq \prod_{j=1}^n b_j - a_j,
  \ \text{ for }\ I = [a, b] \in \mathcal I^n.
\]
\end{definition}

\begin{corollary}[Measure of intervals in \(\R^n\)]
\label{cor:interval-measure-properties}
Let \(I \coloneq [a, b] \subseteq \R^n\) be an \(n\)-dimensional closed
interval, then the following properties are satisfied concerning the measure
\(\Vol\):
\begin{enumerate}[(a)]\setlength\itemsep{0em}
\item (Homogeneity) Let \(\gamma \geq 0\) be a scalar and define the
  multiplication of the interval by \(\gamma\) as \(\gamma I \coloneq [\gamma
  a, \gamma b]\). Then we have that
  \[
    \Vol(\lambda I) = \lambda^n \Vol I.
  \]
\item (Additivity) Given a finite collection of closed intervals \(\{I_{j}
  \subseteq \R^n\}_{j = 1}^p\), we have that
  \[
    \Vol \bigcup_{j=1}^p I_j = \sum_{j=1}^p \Vol I_j.
  \]
\item (Cover inequality) Given a finite closed cover \(\{I_{j}\}^{p}\), by
  \(n\)-dimensional closed intervals, of \(I\) --- that is \(I \subseteq
  \bigcup_{j=1}^p I_j\) --- then
  \[
    \Vol I \leq \sum_{j=1}^p \Vol I_j.
  \]
\end{enumerate}
\end{corollary}

\begin{definition}[Partition]
\label{def:interval-partition}
Let \(I \subseteq \R^n\) be a closed interval. A \emph{partition} on \(I\) is a
\emph{finite} collection of closed intervals \(\{I_{j}\}_{j=1}^p\) such that \(I
= \bigcup_{j=1}^p I_j\). The intervals pertaining to the partition are said to
be \emph{finer} than \(I\).
\end{definition}

\begin{definition}[Partition mesh]
\label{def:partition-mesh}
Given a partition \(P \in 2^{\mathcal I^n}\), we define the \emph{mesh} of \(P\)
as the maximum diameter (recall \cref{def:Rn-diameter}) of the elements of the
partition. That is, \(\Mesh: 2^{\mathcal I^n} \to \R\) is a map defined by
\[
  \Mesh(P) \coloneq \max_{I \in P} d(I).
\]
\end{definition}

\begin{definition}[Distinguished points]
\label{def:distinguished-points}
Given a partition \(P = \{I_{j}\}_{j=1}^p \in 2^{\mathcal I^n}\), we define a
collection of \emph{distinguished points} of the partition as a collection of
points \(\xi \coloneq \{\xi_j \in I_{j}\}_{j=1}^p\). The partition \(P\)
together with the distinguished points \(\xi\) will be denoted as the pair \((P,
\xi)\) --- the collection of pairs \((P, \xi)\) will be denoted by \(\mathcal
P\).
\end{definition}

An important filter base \(\mathcal B \subseteq 2^{\mathcal P}\) is
defined as the collection of sets \(B_d\), where \(d > 0\) is a scalar, such
that \(B_d \coloneq \{(P, \xi) \in \mathcal P \colon \Mesh(P) < d\}\). We'll commonly
denote \(\mathcal B\) by \(\Mesh(P) \to 0\).

\subsection{Riemann Sums and Integrals}

\begin{definition}[Riemann sum]
\label{def:riemann-sum}
Let \(f: I \to \R\) be a map where \(I \in \mathcal I^n\). Consider the
partition together with distinguished points \((P, \xi) \in \mathcal P\), then,
we define the \emph{Riemann sum} \(\sigma: \R^I \times \mathcal P \to \R\) by
\[
  \sigma(f, P, \xi) \coloneq \sum_{j=1}^p f(\xi_j) \Vol(I_j),
\]
where \(P \coloneq \{I_{j}\}_{j=1}^p\). We say that \(\sigma(f, P, \xi)\) is the
Riemann sum of the map \(f\) with respect to the partition \(P\) and
distinguished points \(\xi\).
\end{definition}

\begin{definition}[Riemann integrable maps]
\label{def:riemann-integrable}
A map \(f: I \to \R\) is said to be \emph{Riemann integrable} if the limit
\[
  \lim_{\Mesh(P) \to 0} \sigma(f, P, \xi)
\]
exists in \(\R\). We'll denote the \(\R\)-vector space of Riemann integrable
maps with a given domain \(E \subseteq \R^n\) by \(\mathcal R(E)\)\footnote{For
the time being, we have only defined the case where \(E\) is an interval, but
I'm already generalizing the notation for its uses in the following
subsections.}.
\end{definition}

\begin{proposition}[Boundness of Riemann integrable maps]
\label{prop:riemann-integral-map-bounded}
Let \(f: I \to \R\) be a Riemann integrable map. Then, \(f\) is bounded on \(I\).
\end{proposition}

\begin{proof}
We prove the contrapositive proposition. Suppose that \(f\) is unbounded on
\(I\) and let \(P\) be any partition of the interval \(I\). In particular, since
\(P\) covers \(I\), then there exists an interval \(I_{k} \in P\) for which
\(f\) is unbounded. Let \(\xi\) be any collection of distinguished points of
\(P\) and define \(\xi'\) as the collection of distinguished points \(\xi_j'
\coloneq \xi_j\) for \(j \neq m\), and \(\xi_k' \in I_k\) to be such that
\(\xi_k' \neq \xi_k\). Then, from construction, it follows that \(\sigma(f, P,
\xi) - \sigma(f, P, \xi') = (f(\xi_k) - f(\xi_k')) \Vol(I_k)\). Since from
hypothesis \(f\) is unbounded in \(\xi_k\), for every \(M > 0\), there exists
\(\xi_k' \in I_k\) such that \(\norm{f(\xi_k) - f(\xi_k')} > M\) --- that is,
\(\norm{f(\xi_k) - f(\xi_k')}\) is obviously unbounded, which implies in the
divergence of the Riemann sums, hence \(f\) is non-Riemann integrable.
\end{proof}

\begin{definition}[Riemann integral]
\label{def:riemann-integral}
The Riemann integral of real valued maps is an \(\R\)-linear map \(\int:
\mathcal R \to \R\) defined by mapping any \(f \in \mathcal R(I)\) to
\[
  \int_I f(x)\, \diff x \coloneq \lim_{\Mesh(P) \to 0} \sigma(f, P, \xi).
\]
\end{definition}

\subsection{Sets of Lebesgue Measure Zero}

\begin{definition}[Set of Lebesgue measure zero]
\label{def:measure-zero-set}
A set \(E \subseteq \R^n\) is said be of Lebesgue \emph{measure zero} if for
every \(\varepsilon > 0\) there exists a \emph{countable open cover} \(\mathcal
U\) of \(E\) by \(n\)-dimensional \emph{open} intervals whose total volume
\(\sum_{I \in \mathcal U} \Vol \Cl I\) does \emph{not} exceed
\(\varepsilon\).
\end{definition}

\begin{corollary}
\label{cor:compact-finite-cover-infinitesimal-volume}
A compact subset \(E \subseteq \R^n\) is of measure zero if and only if, for all
\(\varepsilon\), there exists a \emph{finite} open cover \(\mathcal U\) of \(E\)
by open \(n\)-dimensional intervals such that \(\sum_{I \in \mathcal U} \Vol
\Cl I \leq \varepsilon\).
\end{corollary}

\begin{proof}
If \(E\) satisfies the last property, then clearly \(E\) is a set of measure
zero. On the other hand, if we assume that \(E\) is of measure zero, given any
\(\varepsilon > 0\), let \(\mathcal C\) be a \emph{countable} cover of \(E\) for
which \(\sum_{I \in \mathcal C} \Vol \Cl I \leq \varepsilon\). Since
\(E\) is compact, there exists a finite subcover \(\mathcal U \subseteq \mathcal
C\) of \(E\), and since \(\sum_{I \in \mathcal U} \Vol \Cl I \leq \sum_{I
\in \mathcal C} \Vol \Cl I \leq \varepsilon\), thus \(\mathcal U\) is the
wanted finite cover.
\end{proof}

\begin{lemma}
\label{lem:measure-zero-properties}
The following are properties of sets of measure zero:
\begin{enumerate}[(a)]\setlength\itemsep{0em}
\item A subset of a set of measure zero is of measure zero.
\item The countable union of sets of measure zero is of measure zero.
\item A countable set is of measure zero.
\item A non-degenerate interval is \emph{not} a set of measure zero.
\end{enumerate}
\end{lemma}

\begin{proof}
\begin{enumerate}[(a)]\setlength\itemsep{0em}
\item Let \(E \subseteq \R^n\) be a set of measure zero and \(A \subseteq E\) be
  a subset. If \(\mathcal U\) is a closed cover by intervals satisfying the
  measure zero condition, then in particular \(\mathcal U\) covers \(A\)
  therefore \(A\) is of zero measure.

\item Let \(\{E_{j} \subseteq \R^{n}\}_{j \in J}\) be a countable collection of
  sets of measure zero, and let \(\{\mathcal U_{j}\}_{j \in J}\) be a collection
  where \(\mathcal U_j\) is the corresponding closed cover by intervals for
  \(E_j\). Notice that the countable union \(E \coloneq \bigcup_{j \in J} E_j\)
  can be covered by \(\mathcal U \coloneq \bigcup_{j \in J} \mathcal U_j\),
  therefore, since the union of countable collections is countable, it follows
  that \(\mathcal U\) is a countable cover for \(E\) which satisfies the wanted
  property.

\item We initially consider a single point in space. Notice that, for any given
  \(\varepsilon\), there exists a closed interval (for instance, one could
  choose an interval of equal sides containing the point, whose sides have
  length less than \(\varepsilon^{1/n}\)), whose volume is less than
  \(\varepsilon\), contaning the given point --- that is, this one interval is
  sufficient to cover the point. We conclude that a singleton is of measure
  zero. Using the last item, we find that a countable set is of measure zero.

\item Let \(I = [a, b] \subseteq \R^n\) be a non-degenerate interval. Since
  \(\R^n\) is Lindel\"{o}f, every cover of \(I\) has a finite subcover so, we can
  proceed by induction on the cardinality \(m \in \N\) of the open cover. For
  \(m = 1\), let \((\alpha, \beta) \subseteq \R^n\) be an open interval covering
  \(I\). Notice that every \(x \in I\) is such that \(a_j \leq x_j \leq b_j\),
  for all \(1 \leq j \leq n\), then, since \(x\) must lie at \((\alpha,
  \beta)\), we necessarily have \(\alpha_j < a \leq x_j \leq b < \beta_j\) so
  that \(b_j - a_j < \beta_j - \alpha_j\) and hence \(\Vol I < \Vol [\alpha,
  \beta]\). For the hypothesis of induction, suppose the proposition holds for a
  cover of cardinality \(n - 1 \in \N_{> 1}\). Let \(\{(\alpha^{i},
  \beta^i) \subseteq \R^n\}_{i = 1}^m\) be a cover of \(I\) by open
  intervals. Let \(1 \leq k \leq n\) be such that \(a \in (\alpha^k, \beta^k)\),
  that is, \(\alpha_k < a_j < \beta_{j}^{k}\).  If for some index \(1 \leq j_0
  \leq n\) we have \(b_{j_0} > \beta_{j_0}^k\), we define the point \(\beta'\) to
  be such that, if \(\beta_j^k < b_j\) then \(\beta_j' = \beta_j^k\), otherwise,
  if \(\beta_j^k \geq b_j\), we let \(a_j < \beta_j' < b_j\) --- that is, we
  constructed a point so that the closed interval \([\beta', b]\) is
  non-degenerate. From the hypothesis of induction, every cover of \([\beta',
  b]\) with cardinality \(m - 1\) has total volume strictly greater than \(\Vol
  [\beta', b]\). In particular, since \([\beta', b] \subset [a, b]\) then
  \(\{(\alpha^i, \beta^i)\}_{i=1}^m\) is a cover of \([\beta', b]\), notice that
  \([\beta', b] \cap (\alpha^k, \beta^k) = \emptyset\), thus the cover
  \(\{(\alpha^{i}, \beta^i)\}_{i = 1, i \neq k}^m\) is a cover of \([\beta', b]\)
  with cardinality \(n - 1\), hence \(\Vol [\beta', b] < \sum_{i=1, i \neq k}^m
  \Vol [\alpha^i, \beta^i]\). Then we find that
  \begin{align*}
    b_j - a_j
    &< (b_j - \beta'_j) + (\beta_j' - a_j) \\
    &\leq (b_j - \beta_j') + (\beta_j^k - a_j) \\
    &< (b_j - \beta_j') + (\beta_j^k - \alpha_j^k).
  \end{align*}
  Therefore we conclude that
  \begin{align*}
    \Vol [a, b]
    &< \Vol [\beta', b] + \Vol [\alpha^k, \beta^k] \\
    &< \sum_{i=1, i \neq k}^{m} \Vol [\alpha^i, \beta^{i}] + \Vol [\alpha^k,
    \beta^k] \\
    &= \sum_{i=1}^m \Vol [\alpha^i, \beta^i],
  \end{align*}
  which proves that the proposition is true for all \(m \in \N\).
\end{enumerate}
\end{proof}

\begin{example}
\label{exp:graph-lebesgue-measure-zero}
Let \(f: I \to \R\) be a continuous map on the interval \(I \subseteq
\R^{n-1}\). The graph of \(f\) is a \(n\)-dimensional set of Lebesgue measure
zero.

\begin{proof}
Let \(\Gamma\) denote the graph of \(f\) over \(I\). To see this, notice that
since \(I\) is closed, then \(f\) is uniformly continuous on \(I\). For any
\(\varepsilon > 0\), let \(\delta > 0\) be such that \(\norm{f(x) - f(y)}_{\R} <
\varepsilon\) for every \(x, y \in I\) such that \(\norm{x - y}_{\R^{n-1}} <
\delta\). Let \(P\) be a partition of \(I\) with \(\Mesh P < \delta\). For any
point \(x_0 \in R\) of each \(R \in P\), we have an induced interval \(R'
\coloneq R \times [f(x_0) - \varepsilon, f(x_0) + \varepsilon]\) which is such
that \(\Gamma \subseteq R'\), since we have an oscillation \(\omega(f, R) <
\varepsilon\) from the construction of \(P\). Let \(P'\) be the collection of
induced intervals \(R'\) of the partition \(P\). From the last observation we
have that \(P'\) is a closed cover of \(\Gamma\) by closed \(n\)-dimensional
intervals. Moreover, \(\sum_{R' \in P'} \Vol R' = \sum_{R \in P} 2 \varepsilon
\Vol R = 2 \varepsilon \Vol I\) --- therefore \(\Gamma\) indeed is of Lebesgue
measure zero
\end{proof}
\end{example}

\begin{notation}
Given a set \(X\) and a property \(P\), we say that \(P\) holds \emph{almost
everywhere} on \(X\) if the subset \(A \subseteq X\), such that \(P\) is not
true, is a set of measure zero.
\end{notation}

\begin{theorem}[Lebesgue's criterion]
\label{thm:lebesgue-criterion-integrable}
A map \(f: I \to \R\) is Riemann integrable if and only if \(f\) is
\emph{bounded} on \(I\) and \(f\) is \emph{continuous almost everywhere} on
\(I\).
\end{theorem}

\begin{proof}
(\emph{Necessity}) Let \(f\) be Riemann integrable, then, from
\cref{prop:riemann-integral-map-bounded}, \(f\) is bounded on \(I\). For the
sake of contradiction, let \(E \subseteq I\) be the set composed of the points
of discontinuity of \(f\), we'll suppose that \(E\) doesn't have measure
zero. Notice that if \(x \in E\), then there exists \(n \in \N\) for which
\(\omega(f, x) \geq 1/n\) --- that is, \(f\) does not converge to a value in
\(x\). We can then define \(E_n \coloneq \{x \in I \colon \omega(f, x) \geq 1/n\}\)
for every \(n \in \N\) so that \(E_n \subseteq E\) and thus \(E = \bigcup_{n \in
\N} E_n\). Since \(E\) isn't of measure zero from assumption, it follows that
there necessarily exists at least one \(n_0 \in \N\) such that \(E_{n_0}\) isn't
of measure zero.

Let \(P\) be a partition of \(I\), we'll consider two subsets of this partition:
\[
  A \coloneq \{R \in P \colon R \cap E_{n_{0}} \neq \emptyset
  \text{ and } \omega(f, R) \geq 1/(2n_0)\},
\]
and \(B \coloneq P \setminus A\). Since \(P\) partitions the interval, for any
\(x \in E_{n_0}\), there exists \(R \in P\) such that \(x \in \Int R\) or \(x
\in \partial R\) --- we now analyse both cases:
\begin{itemize}\setlength\itemsep{0em}
\item In the case where \(x\) is an interior point, since \(\omega(f, x) \geq
  1/n_0\), it follows that \(R \in A\).
\item Otherwise, if \(x\) is a boundary point, then there actually exists at
  least another \(R' \in P\) such that \(x \in \partial R \cap \partial
  R'\). Suppose, for the sake of contradiction, that every interval \(R' \in P\)
  containing \(x\) as a boundary point is such that \(f\) has an oscillation
  \(\omega(f, R') < 1/(2n_0)\), then, if we take any ball \(B_x(r) \cap I\),
  neighbourhood of \(x\), we find that, \(\omega(f, B_x(r)) < 1/(2n_0) +
  1/(2n_0) = 1/n_0\), that is, the limit \(\omega(f, x) < 1/n_0\), which is a
  contradiction to the assumption that \(x \in E_n\) --- thus there must exist
  \(R' \in P\) with \(x \in \partial R'\) such that \(\omega(f, R') \geq
  1/(2n_0)\) so that \(R' \in A\).
\end{itemize}
This implies that \(A\) covers the interval \(E_{n_0}\) by closed intervals and,
by assumption, \(\sum_{R \in A} \Vol R > \Vol E_{n_0}\).

We are now going to consider any two distinct choices of distinguished points
\(\xi\) and \(\xi'\) of \(P\) such that, if \(\xi_j\) and\(\xi_j'\) are elements
of a common interval of \(B\) then \(\xi_j = \xi'_j\), and if \(\xi_j\) and
\(\xi_j'\) are elements common to an interval of \(A\), we choose \(\xi_j\) and
\(\xi_j\) to be any points such that \(f(\xi_j) - f(\xi_j') > 1/(3n_0)\) ---
which is always possible from the construction of \(A\). Notice that we have
\begin{align*}
  \norm{\sigma(f, P, \xi) - \sigma(f, P, \xi')}
  &= \norm{\sum_{R_{j} \in A} (f(\xi_j) - f(\xi_j')) \Vol R_j} \\
  &> \frac 1 {3 n_0} \sum_{R_j \in A} \Vol R_j \\
  &> \frac 1 {3 n_0} \Vol E_{n_0} > 0,
\end{align*}
and therefore the limit \(\lim_{\Mesh(P) \to 0} \sigma(f, P, \xi)\) does not
converge whenever \(E_{n_0}\) is of measure zero --- that is, if the set of
discontinuities of \(f\) isn't of measure zero, then \(f\) is non-Riemann
integrable.

(\emph{Sufficiency}) Suppose that \(f\) is bounded and continuous almost
everywhere on \(I\). Let \(\varepsilon > 0\) be any constant and consider the
subset \(E_{\varepsilon} \coloneq \{x \in I \colon \omega(f, x) \geq \varepsilon\}\)
of discontinuous points of \(f\). If \(E\) is the set of points of discontinuity
of \(f\), then by hypothesis \(E\) is of measure zero --- since
\(E_{\varepsilon} \subseteq E\), then \(E_{\varepsilon}\) is also of measure
zero. Moreover, \(E_{\varepsilon}\) is necessarily closed, which implies in
\(E_{\varepsilon}\) compact. Let \(\{I_{j}\}_{j=1}^k\) be a \emph{finite} open
cover of \(E_{\varepsilon}\) by open intervals \(I_j \subseteq \R^n\) such that
\(\sum_{i=1}^k \Vol \Cl I < \varepsilon\) (recall that this is possible
because of \cref{cor:compact-finite-cover-infinitesimal-volume}).

We now define the \(\lambda\)-dilations of intervals of the cover
\(\{I_{j}\}_{j=1}^k\) but with centre unchanged --- that is, if \(I_j = (x,
y)\), then the dilation \(I_{j}^{\lambda} \coloneq (z, w)\) must be such that
\(w_i - z_i = \lambda^{1/n} (y_i - x_i)\) and \(w_i + z_i = y_i + x_i\), for all
\(1 \leq i \leq n\). Solving such system we obtain \(z_i = \frac 1 2 [x_i(1 +
\lambda^{1/n}) + y_i(1 - \lambda^{1/n})]\) and \(w_i = \frac 1 2 [x_i(1 -
\lambda^{1/n}) + y_i(1 + \lambda^{1/n})]\). This way we have
\[
  \sum_{j=1}^k \Vol \Cl I_j^{\lambda}
  = \lambda^n \sum_{j=1}^k \Vol \Cl I_j
  < \lambda^n \varepsilon.
\]
We'll define the sets \(C_{\lambda} \coloneq (\bigcup_{j=1}^k I_j^{\lambda})
\cap I\) --- moreover, \(d\) will be the minimum distance between the boundaries
of \(C_2\) and \(C_3\).

From construction, we have \(E_{\varepsilon} \subseteq \Int C_2\), now if we
consider the compact set \(K \coloneq I \setminus (\Int C_2)\), then, for every
\(x \in K\), we have \(\omega(f, x) < \varepsilon\). Using
\cref{thm:heine-cantor}, we find that there exists a \(\delta > 0\) such that
\(\norm{f(x) - f(y)} < 2 \varepsilon\), for all \(x, y \in K\) with \(\norm{x -
y} < \delta\). We'll now finally show the convergence of the Riemann sums. Let
\(P' \coloneq {\{R'_{\alpha}\}}_{\alpha \in A}\) and \(P'' \coloneq
{\{R''_{\beta}\}}_{\beta \in B}\) be any two partitions of \(I\) with \(\Mesh(P'),
\Mesh(P'') < \min(d, \delta)\). Define \(P \coloneq \{R_{(\alpha, \beta)}
\coloneq R'_{\alpha} \cap R''_{\beta}\}_{(\alpha, \beta) \in A \times B}\),
which clearly is a partition of \(I\). Since we are dealing with partitions,
for every \(R'_{\alpha} \in P'\) we have \(R'_{\alpha} = \bigcup_{\beta \in
B} R_{(\alpha, \beta)}\) --- thus \(\Vol R'_{\alpha} = \sum_{\beta \in B} \Vol
R_{(\alpha, \beta)}\). Let \(\xi'\) and \(\xi\) be any two distinguished points
of the partitions \(P'\) and \(P\), respectively, then
\begin{align*}
  \norm{\sigma(f, P', \xi') - \sigma(f, P, \xi)}
  &= \norm{\sum_{(\alpha, \beta) \in A \times B} (f(\xi'_{\alpha}) -
    f(\xi_{(\alpha, \beta)})) \Vol R_{(\alpha, \beta)}} \\
  &\leq \sum_{(\alpha, \beta) \in A' \times B'}
    \norm{f(\xi'_{\alpha}) - f(\xi_{(\alpha, \beta)})} \Vol R_{(\alpha, \beta)} \\
  &\qquad+  \sum_{(\alpha, \beta) (A \times B) \setminus (A' \times B')}
  \norm{f(\xi'_{\alpha}) - f(\xi_{(\alpha, \beta)})} \Vol R_{(\alpha, \beta)},
\end{align*}
where we define
\[
  A' \times B' \coloneq
  \{(\alpha, \beta) \in A \times B : R'_{\alpha} \subseteq C_3\}.
\]
Moreover, since the diameter \(d(R'_{\alpha}) < d\), then for every
\(R'_{\alpha}\) not entirely contained in \(C_3\), the intersection with any
\(R''_{\beta}\), namely \(R_{(\alpha, \beta)}\), cannot lie in the interior of
\(C_2\) --- that is, for all \((\alpha, \beta) \in (A \times B) \setminus (A'
\times B')\) we'll have \(R_{(\alpha, \beta)} \subseteq K\) and surely
\(\xi'_{\alpha}, \xi_{(\alpha, \beta)} \in K\), which satisfy
\(\norm{f(\xi'_{\alpha}) - f(\xi_{(\alpha, \beta)})} < 2 \varepsilon\) since
\(\Mesh P' < \delta\). Assuming \(f\) is bounded by \(M > 0\) in \(I\), that is
\(\norm{f} \leq M\), then \(\norm{f(\xi'_{\alpha}) - f(\xi_{(\alpha, \beta)})} <
2 M\) and thus
\begin{align*}
  \norm{\sigma(f, P', \xi') - \sigma(f, P, \xi)}
  &\leq  \sum_{(\alpha, \beta) \in A' \times B'} 2 M \Vol R_{(\alpha, \beta)}
    + \sum_{(\alpha, \beta) \in (A \times B) \setminus (A' \times B')}
    2 \varepsilon \Vol R_{(\alpha, \beta)} \\
  &\leq 2 M (3^n \varepsilon) + 2 \varepsilon \Vol I \\
  &= 2 \varepsilon(3^n M + \Vol I).
\end{align*}
Therefore, since the same construction is applicable for \(P''\), that is,
\(\norm{\sigma(f, P'', \xi''') - \sigma(f, P, \xi)} \leq 2 \varepsilon (3^n M +
\Vol I)\), thus
\[
  \norm{\sigma(f, P', \xi') - \sigma(f, P'', \xi'')}
  \leq 4 \varepsilon (3^n M + \Vol I).
\]
Since \(P'\) and \(P''\) where chosen arbitrarily, it follows that the sequence
of Riemann sums converge by the Cauchy criterion and thus \(f\) is Riemann
integrable from definition.
\end{proof}

\subsection{Integrating Over Sets}

\begin{definition}
\label{def:admissible-set}
A set \(E \subseteq \R^n\) is said to be \emph{admissible} if it is bounded and
has measure zero boundary.
\end{definition}

\begin{corollary}[Operations on admissible sets]
\label{cor:operations-admissible-sets}
The finite union or intersection of admissible sets is admissible, and the
difference of admissible sets is admissible.
\end{corollary}

\begin{notation}
\label{not:characteristic-set}
We denote the characteristic map of a given set \(E\) to be \(\chi_E: E \to \{0,
1\}\), where \(\chi_E(x) \coloneq 1\) if \(x \in E\), and \(\chi_E(x) \coloneq
0\) if \(x \not\in E\). Moreover, given any map \(f: E \to \R\), we define the
map \(f_{\chi_E}: \R^n \to \R\) as \(f_{\chi_{E}}|_{E} \coloneq f\) and
\(f_{\chi_{E}}\) is zero everywhere else.
\end{notation}

\begin{definition}[Riemann integral over a set]
\label{def:riemann-integral-over-set}
Let \(E \subseteq \R^n\) be an admissible set and \(f: I \to \R\) for some \(I
\supseteq E\), then we define the integral of \(f\) over the set \(E\) as
\[
  \int_E f(x)\, \diff x \coloneq \int_I f_{\chi_E}(x)\, \diff x.
\]
\end{definition}

\begin{lemma}[The integral is well-defined]
\label{lem:integral-well-defined}
Let \(f: E \to \R\) be a map. The integral of \(f\) over \(E\) is independent of
the choice of the interval containing \(E\).
\end{lemma}

\begin{proof}
Let \(I, I' \subseteq \R^n\) be any two intervals containing \(E\). Define an
auxiliary interval \(I_0 \coloneq I \cap I'\). It follows from construction that
every point of discontinuity of \(f_{\chi_E}\) is contained in \(I_0\). From
\cref{thm:lebesgue-criterion-integrable} we see that, if the collection of
points of discontinuity of \(f_{\chi_E}\) is not of measure zero, then both
integrals of \(f_{\chi_E}\) over \(I\) and \(I'\) fail to exist --- while, if
the collection is of measure zero, both integrals exist simultaneously.

Suppose that the points of discontinuity form a set of measure zero. Given any
\(\varepsilon > 0\), we consider a partition \(P_0\) of \(I\) with \(\Mesh P_0 <
\varepsilon\). Define partitions \(P\) and \(P'\) of \(I\) and \(I'\),
respectively, such that \(P_0 \subseteq P \cap P'\) --- that is, inside \(I_0\),
they share the exact same collection of intervals as \(P_0\). Given any
distinguished points \(\xi\) and \(\xi'\) of \(P\) and \(P'\), respectively,
we have that \(f_{\chi_{E}}(x) = 0\) for every \(x \in \xi \cup \xi'\) such that
\(x \not\in I_0\), therefore the Riemann sums of \(f_{\chi_E}\) under the
partitions \(P\) and \(P'\) are always reduced to Riemann sums of \(f_{\chi_E}\)
under the partition \(P_0\) and the corresponding distinguished points. This
implies that the limit of the Riemann sums are equal for both partitions and
therefore we have
\[
  \int_I f_{\chi_E}(x)\, \diff x = \int_{I'} f_{\chi_E}(x)\, \diff x.
\]
\end{proof}

\begin{corollary}
\label{cor:condidion-riemann-integrable-admissible-set}
Let \(E \subseteq \R^n\) be an admissible set and \(f: E \to \R\). Then, \(f\)
is Riemann integrable if and only if is bounded and continuous at almost all
points of \(E\).
\end{corollary}

\begin{proof}
Notice that the discontinuities of the corresponding map \(f_{\chi_E}\) are
those of \(f\) and perhaps a collection of points of \(\partial E\), where
\(\chi_E\) changes its value. Since \(\partial E\) is of measure zero, it doesn't
interfere in the use of Lebesgue's criterion and thus the proposition follows.
\end{proof}

\subsection{Jordan Measure}

\begin{definition}[Jordan measure]
\label{def:jordan-measure}
Given a bounded set \(E \subseteq \R\), we define the Jordan measure of \(E\) as
the map
\[
  \mu(E) \coloneq \int_E \diff x.
\]
Notice that the integral over \(E\) only exists for admissible sets, thus the
Jordan measure \(\mu\) is only defined for admissible sets.
\end{definition}

\begin{definition}[Jordan's measure zero sets]
\label{def:jordan-measure-zero}
A set \(E \subseteq \R^n\) is said to be of Jordan measure zero (or of content
zero) if for every \(\varepsilon > 0\) there exists a \emph{finite} open cover
by intervals whose total volume is less than \(\varepsilon\).
\end{definition}

\begin{definition}[Jordan-measurable sets]
\label{def:jordan-measurable}
A set \(E \subseteq \R^n\) is said to be Jordan-measurable if it is bounded and
\(\partial E\) is a set of Jordan measure zero.
\end{definition}

\subsection{Riemann Integral Properties}

\begin{corollary}
The collection of Riemann integrable maps \(E \to \R\) for an \(\R\)-vector
space. Moreover, the Riemann integral is an \(\R\)-linear functional \(\int_E:
\mathcal R(E) \to \R\) of the dual space \(\mathcal R(E)^{*}\).
\end{corollary}

\begin{proposition}
\label{prop:domain-relations-riemann-integral}
Let \(E, S \subseteq \R^n\) be admissible sets and \(f: E \cup S \to \R\). Then
\begin{enumerate}[(a)]\setlength\itemsep{0em}
\item \(f\) is Riemann integrable over \(E \cup S\) if and only if it's
  integrable over both \(E\) and \(S\) simultaneously. Moreover, if such
  condition is met, the \(f\) is also Riemann integrable over \(E \cap S\).
\item If \(f\) is Riemann integrable over \(E \cup S\) \emph{and} \(\mu(E \cap
  S) = 0\), then
  \[
    \int_{E \cup S} f(x)\, \diff x = \int_E f(x)\, \diff x + \int_S f(x)\, \diff x.
  \]
\end{enumerate}
\end{proposition}

\begin{proof}
\begin{enumerate}[(a)]\setlength\itemsep{0em}
\item If \(f\) is Riemann integrable over \(E \cup S\), then \(f\) is bounded
  and continuous at almost all points of \(E \cup S\), which implies that \(f\)
  is Riemann integrable over both \(E\) and \(S\). Since \(E \cap S \subseteq E
  \cup S\) then \(f\) is also Riemann integrable over the intersection.
\item Notice that \(\chi_{E \cup S}(x) = \chi_{E}(x) + \chi_S(x) - \chi_{E \cap
  S}(x)\), therefore by definition --- if \(I \supseteq E \cup S\) is an
  interval, then
  \begin{align*}
    \int_{E \cup S} f(x)\, \diff x
    &= \int_{I} f_{\chi_{E \cup S}}(x)\, \diff x \\
    &= \int_I f_{\chi_E}(x) + f_{\chi_S}(x) - f_{\chi_{E \cap S}}(x)\, \diff x \\
    &= \int_I f_{\chi_E}(x)\, \diff x
      + \int_{I} f_{\chi_S}(x)\, \diff x
      - \int_I f_{\chi_{E \cap S}}(x)\, \diff x \\
    &= \int_I f_{\chi_E}(x)\, \diff x
      + \int_{I} f_{\chi_S}(x)\, \diff x
  \end{align*}
  where we used the hypothesis that \(\mu(E \cap S) = 0\) to conclude that
  \[
    \int_I f_{\chi_{E \cap S}}(x)\, \diff x
    = \int_{E \cap S} f(x)\, \diff x = 0.
  \]
\end{enumerate}
\end{proof}

\begin{proposition}[Estimate]
\label{prop:norm-of-integral-less-integral-of-norm}
Let \(f \in \mathcal R(E)\), where \(E \subseteq \R^{n}\) is an admissible set,
then \(\norm{f} \in \mathcal R(E)\) and also
\[
  \norm{\int_E f(x)\, \diff x} \leq \int_E \norm{f(x)}\, \diff x.
\]
\end{proposition}

\begin{proof}
Notice that if \(f\) is Riemann integrable, then \(f\) is bounded and and
continuous at almost all points of \(E\), thus surely \(\norm f\) is bounded and
continuous --- hence Riemann integrable. Moreover, since \(f(x) \leq
\norm{f(x)}\) for all \(x \in E\), it follows that, for any given partition
\(P\) and distinguished points \(\chi\), we have the inequality \(\sigma(f, P,
\chi) \leq \sigma(\norm{f}, P, \chi)\), thus the integral inequality holds by
taking the limit \(\Mesh P \to 0\).
\end{proof}

\begin{proposition}[Non-negative real valued maps]
\label{prop:non-negative-real-valued-integral-non-negative}
Let \(f: E \to \R\) be a Riemann integrable map over the admissible set \(E
\subseteq \R^{n}\). If \(f\) is non-negative over \(E\), that is, \(f(x) \geq
0\) for all \(x \in E\), then
\[
  \int_E f(x)\, \diff x \geq 0.
\]
\end{proposition}

\begin{proof}
Just observe that for any partition \(P\) and distinguished points \(\chi\), we
have \(\sigma(f, P, \chi) \geq 0\), thus \(\lim_{\Mesh P \to 0} \sigma(f, P,
\chi) \geq 0\).
\end{proof}

\begin{corollary}
\label{cor:inequality-maps-inequality-riemann-integral}
Let \(E \subseteq R^n\) be an admissible set and \(f, g: E \to \R\) be two given
maps. The following are miscellaneous immediate implications of
\cref{cor:inequality-maps-inequality-riemann-integral}:
\begin{enumerate}[(a)]\setlength\itemsep{0em}
\item If both \(f\) and \(g\) are Riemann integrable, and \(f(x) \leq g(x)\) for
  all \(x \in E\), then
  \[
    \int_E f(x)\, \diff x \leq \int_E g(x)\, \diff x.
  \]

\item If \(f\) is Riemann integrable over \(E\) and, for some constants \(m, M
  \in \R\) we have \(m \leq f(x) \leq M\) for all \(x \in E\), then
  \[
    m \mu(E) \leq \int_E f(x)\, \diff x \leq M \mu(E).
  \]

\item If \(f\) is Riemann integrable over \(E\), define constants \(m
  \coloneq \inf_{x \in E} f(x)\) and \(M \coloneq \sup_{x \in E} f(x)\). Then,
  there exist a constant \(m \leq \omega \leq M\) for which
  \[
    \int_E f(x)\, \diff x = \theta \mu(E).
  \]

\item If additionally \(E\) is connected and \(f\) is continuous, then \(f\) is
  Riemann integrable over \(E\) and there exists \(y \in E\) for which
  \[
    \int_E f(x)\, \diff x = f(y) \mu(E).
  \]

\item If \(f\) and \(g\) are Riemann integrable over \(E\) and, for some
  constants \(m, M \in \R\) we have \(m \leq f(x) \leq M\) for all \(x \in E\),
  and \(g(x) \geq 0\) for all \(x \in E\), then
  \[
    m \int_E g(x)\, \diff x \leq \int_E f(x) g(x)\, \diff x
    \leq M \int_E g(x)\, \diff x.
  \]
\end{enumerate}
\end{corollary}

\begin{lemma}
\label{lem:zero-integral-jordan-measure-zero}
Let \(f: E \to \R\) be a non-negative map that is Riemann integrable over
\(E\), where \(E\) is an admissible set. If \(\int_E f(x)\, \diff x = 0\), then
\(f(x) = 0\) at almost all points of \(E\).
\end{lemma}

\begin{proof}
Let \(x_0 \in E\) be any point of continuity of \(f\) and, for the sake of
contradiction, assume that \(f(x_0) > 0\) and consider a constant \(c > 0\) with
\(f(x_0) \geq c > 0\). Let \(U \subseteq E\) be any neighbourhood of \(x_0\) for
which \(f(x) \geq c > 0\) for every \(x \in U\). Then we obtain the following
contradiction --- where \(I \supseteq E\) is an interval:
\[
  \int_E f(x)\, \diff x
  = \int_I f_{\chi_{E}}(x)\, \diff x
  = \int_U f_{\chi_E}(x)\, \diff x
  + \int_{I \setminus U} f_{\chi_{E}}(x)\, \diff x
  \geq \int_U f_{\chi_{E}}(x)\, \diff x
  \geq c \mu(U) > 0.
\]
This cannot be true, thus \(f(x_0) = 0\) for every point \(x_0 \in E\) where
\(f\) is continuous. Since \(f\) is continuous almost everywhere, it follows
that \(f\) is zero almost everywhere.
\end{proof}

\begin{corollary}
\label{cor:norm-riemann-integral}
Let \(\sim\) be the equivalence relation on \(\mathcal R(E)\) as follows: \(f
\sim g\) if and only if the collection of points where the map \(f - g\) is
non-zero forms a set of Lebesgue measure zero. Then the map \(\norm{-}:
\mathcal R(E)/{\sim} \to \R\) defined by
\[
  \norm{f} \coloneq \int_E \norm{f(x)}_{\R^n}\, \diff x
\]
is a norm on the vector space \(\mathcal R(E)/{\sim}\).
\end{corollary}

\section{Fubini's Theorem}

\subsubsection{Upper and Lower Darboux Integrals}

In order to proceed to the theorem of Fubini type, we first need to define the
concept of Darboux lower and upper integrals.

\begin{definition}[Darboux sums]
\label{def:darboux-sums}
Let \(f: I \to \R\) be a map defined on the interval \(I \subseteq \R^n\). If
\(P \coloneq \{I_{j}\}_{j \in J}\) is a partition of \(I\) by intervals, then we
define the following
\begin{itemize}\setlength\itemsep{0em}
\item The lower Darboux sum of \(f\) over the interval \(I\) corresponding to
  the partition \(P\) is
  \[
    s(f, P) \coloneq \sum_{j \in J} \Vol(I_j) \inf_{x \in I_j} f(x).
  \]
\item The upper Darboux sum of \(f\) over the interval \(I\) corresponding to
  the partition \(P\) is
  \[
    S(f, P) \coloneq \sum_{j \in J} \Vol(I_j) \sup_{x \in I_j} f(x).
  \]
\end{itemize}
\end{definition}

\begin{definition}[Upper and lower Darboux integrals]
\label{def:upper-lower-darboux}
Let \(f: I \to \R\) be a map over the interval \(I \subseteq \R^n\). We define:
\begin{itemize}\setlength\itemsep{0em}
\item The lower Darboux integral of \(f\) is \(\underline{\mathcal{J}} \coloneq
  \sup_P s(f, P)\), where the \(P\) goes through all partitions of \(I\).

\item The upper Darboux integral of \(f\) is \(\underline{\mathcal{J}} \coloneq
  \inf_P S(f, P)\), where the \(P\) goes through all partitions of \(I\).
\end{itemize}
\end{definition}

\begin{theorem}[Darboux criterion]
\label{thm:darboux-criterion}
A map \(f: I \to \R\), where \(I \subseteq \R^n\) is an interval, is integrable
over \(I\) if and only if \(f\) is bounded on \(I\) and the upper and lower
Darboux integrals agree over \(I\).
\end{theorem}

\todo[inline]{Add proof of Darboux criterion later}

\subsubsection{Fubini}

\begin{theorem}
\label{thm:fubini}
Let \(X \times Y \subseteq \R^{m + n}\) be an interval, and \(f: X \times Y \to
\R\) be a Riemann integrable map over \(X \times Y\). Then we have the following
equality:
\begin{equation}
\label{eq:fubini}
  \int_{X \times Y} f(x, y)\, \diff x\, \diff y
  = \int_X \left( \int_Y f(x, y)\, \diff y\right)\, \diff x
  = \int_Y \left( \int_X f(x, y)\, \diff x\right)\, \diff y.
\end{equation}
\end{theorem}

\begin{proof}
In order to prove the theorem, we first construct the following maps:
\begin{itemize}\setlength\itemsep{0em}
\item \(F_X: X \to \R\) is defined as \(F_{X}(x_0) \coloneq \int_Y f(x_0, y)\,
  \diff x\) whenever such integral exists, in case not, we let
  \(\underline{\mathcal{J}_{x_0}} \leq F_{X}(x_0) \leq
  \overline{\mathcal{J}_{x_0}}\) be any element between the Darboux integrals of
  the map \(f(x_{0}, -): Y \to \R\) given by \(y \mapsto f(x_{0}, y)\).

\item \(F_Y: Y \to \R\) is defined as \(F_{Y}(y_{0}) \coloneq \int_x f(x,
  y_{0})\, \diff x\) whenever such integral exists, in case not, we let
  \(\underline{\mathcal{J}_{y_0}} \leq F_{Y}(y_{0}) \leq
  \overline{\mathcal{J}_{y_0}}\) be any element between the Darboux integrals of
  the map \(f(-, y_0): X \to \R\) given by \(x \mapsto f(x, y_{0})\).
\end{itemize}
We'll carry the proof for the first equality, but, knowing the construction of
\(F_Y\), the idea for the second equality is the exact analogue.

Let \(P \coloneq \{X_{i}\}_{i \in I} \times \{Y_{j}\}_{j \in J}\) be any
partition of the interval \(X \times Y\). Notice that
\begin{align*}
  s(f, P)
  &= \sum_{(i, j) \in I \times J}
    \inf_{(x, y) \in X_i \times y_j}f(x, y) \Vol(X_i \times Y_j) \\
  &= \sum_{(i, j) \in I \times J}
    \inf_{(x, y) \in X_i \times y_j}f(x, y) \Vol(X_i) \Vol(Y_j) \\
  &\leq \sum_{i \in I}
    \inf_{x \in X_i} \left[ \sum_{j \in J} f(x, y) \Vol(Y_j) \right] \Vol X_i  \\
  &\leq \sum_{i \in I} \inf_{x \in X_i} \left[ \underline{\int_{Y}} f(x, y)\,
    \diff y \right] \Vol X_i \\
  &\leq \sum_{i \in I} \inf_{x \in X_i} F_X(x) \Vol X_i \\
  &\leq \sum_{i \in I} \sup_{x \in X_i} F_X(x) \Vol X_i \\
  &\leq \sum_{i \in I} \sup_{x \in X_i} \left[ \overline{\int_X} f(x, y)\, \diff
  x \right] \Vol X_i \\
  &\leq \sum_{i \in I} \sup_{x \in X_i} \left[ \sum_{j \in J} f(x, y) \Vol Y_j
    \right] \Vol X_i \\
  &\leq \sum_{(i, j) \in I \times J} f(x, y) \Vol(X_i \times Y_j) \\
  &= S(f, P).
\end{align*}
By hypothesis, \(f\) is Riemann integrable over \(X \times Y\), thus
\[
  \lim_{\Mesh P \to 0} s(f, P)
  = \lim_{\Mesh P \to 0} S(f, P)
  = \int_{X \times Y} f(x, y)\, \diff x\, \diff y,
\]
therefore \(F_X\) is Riemann integrable over
\(X\) and
\[
  \int_{X \times Y} f(x, y)\, \diff x\, \diff y = \int_X F_X(x)\, \diff x.
\]
\end{proof}

\begin{corollary}
\label{cor:riemann-int-XxY-almost-int-X-and-Y}
Let \(f: X \times Y \to \R\) be Riemann integrable. Then for almost all points
\(x_0 \in X\) and almost all \(y_0 \in Y\) the integrals \(\int_Y f(x_0, y)\,
\diff y\) and \(\int_X f(x, y_0)\, \diff x\) exist.
\end{corollary}

\begin{proof}
Since \(s(f, P) \leq S(f, P)\) for all partitions \(P\) of \(X \times Y\), then
\(\underline{\int_Y} f(x_0, y)\, \diff y \leq \overline{\int_X} f(x_0, y)\,
\diff y\) for every \(x_0 \in X\). From \cref{thm:fubini} we see that
\[
  \int_X \left( \overline{\int_Y} f(x, y)\, \diff y
    - \underline{\int_Y} f(x, y)\, \diff y\right)\, \diff x = 0,
\]
therefore by \cref{lem:zero-integral-jordan-measure-zero} we conclude that the
map \(x_0 \mapsto \overline{\int_Y} f(x_0, y)\, \diff y - \underline{\int_Y}
f(x_0, y)\, \diff y\) equals zero for almost all \(x_0 \in X\) --- that is,
\(\int_Y f(x_0, y)\, \diff y\) exists for almost all \(x_0 \in X\). The exact
same argument can be used to show this for \(\int_X f(x, y_0)\, \diff x\).
\end{proof}

\begin{corollary}[Iterated Fubini's theorem]
\label{cor:iterated-fubini-theorem}
Let \(I \coloneq \prod_{j = 1}^n [a_j, b_j] \subseteq \R^n\) be an interval and
\(f: I \to \R\) be a Riemann integrable map over \(I\). Then
\[
  \int_I f(x)\, \diff x = \int_{a_n}^{b_n} \left(
    \int_{a_{n-1}}^{b_{n-1}} \left(
      \dots \left(
        \int_{a_1}^{b_1} f(x, y)\, \diff x_1
      \right)
      \dots \right)\, \diff x_{n-1} \right)\, \diff x_n.
\]
Moreover, the result of the integral is invariant under any permutation of the
order of the sub-intervals \([a_j, b_j]\) of integration.
\end{corollary}

\begin{proof}
This is just a direct application of \cref{thm:fubini}, moreover, the second
part of the proposition can be obtained by seeing that the second equality of
\cref{eq:fubini} is nothing but a transposition on the set \(\{X, Y\}\) ---
generalizing this for transpositions on \(\{[a_j, b_j]\}_{j=1}^n\) imply that
any permutation of the collection satisfies the integral equality.
\end{proof}

\subsubsection{Measuring Volumes Between Graphs of Maps}

\begin{corollary}
\label{cor:integral-over-graph-maps}
Define \(D \subseteq \R^{n-1}\) to be a bounded set and let maps \(g_{1}, g_{2}:
D \to \R\) such that \(g_1 \leq g_2\). Let \(f: E \to \R\) be a Riemann
integrable map over \(E \coloneq \{(x, y) \in \R^{n - 1} \times \R \colon x \in D
\text{ and } g_1(x) \leq y \leq g_2(x)\}\). Then
\[
  \int_E f(x, y)\, \diff x\, \diff y
  = \int_D \left( \int_{g_1(x)}^{g_2(x)} f(x, y)\, \diff y \right)\, \diff x.
\]
\end{corollary}

\begin{proof}
For each \(x \in D\) define the set \(E_x \coloneq \{y \in \R \colon g_{1}(x) \leq y
\leq g_2(x)\}\), and observe that, from construction, \(\chi_E(x, y) = \chi_D(x)
\chi_{E_x}(y)\). Given any interval \(I \coloneq I_x \times I_y \supseteq E\),
we have
\begin{align*}
  \int_E f(x, y)\, \diff x\, \diff y
  = \int_I f_{\chi_E}(x, y)\, \diff x\, \diff y
  = \int_{I_x} \left( \int_{I_y} f_{\chi_E}(x, y)\, \diff y  \right)\, \diff x
\end{align*}
\end{proof}

\begin{corollary}[Measuring the set between the graph of continuous maps]
\label{cor:measure-set-between-graphs-of-continuous-maps}
Let \(D \subseteq \R^{n-1}\) be a Jordan-measurable set and \(g_1, g_2: D \to
\R\) be continuous maps such that \(g_1 \leq g_2\). Then the set \(E \coloneq
\{(x, y) \in \R^{n-1} \times \R \colon x \in D \text{ and } g_1(x) \leq y \leq
g_2(x)\} \subseteq \R^n\) is Jordan-measurable and its Jordan measure is
\[
  \mu(E) = \int_D g_2(x) - g_1(x)\, \diff x.
\]
\end{corollary}

\begin{proof}
Denote by \(\Gamma_{g_1}\) and \(\Gamma_{g_2}\) the graphs of \(g_1\) and
\(g_2\), respectively. By \cref{exp:graph-lebesgue-measure-zero},
\(\Gamma_{g_1}\) and \(\Gamma_{g_2}\) are sets of Lebesgue measure
zero. Define a set
\[
Z \coloneq \partial D \times
[\inf_{x \in D} g_1(x), \sup_{x \in D} g_2(x)] \subseteq \R^n
\] and the length \(\ell \coloneq \sup_{x
\in d} g_2(x) - \inf_{x \in D} g_1(x)\). Since \(D\) is a Jordan-measurable set,
it follows that \(\partial D\) can be covered by a finite collection of open
sets with total volume arbitrarily small --- in particular, smaller than
\(\varepsilon / \ell\). In particular, it follows that \(Z\) is a set of Jordan
measure zero. Notice that \(\partial E \subseteq \Gamma_{g_1} \cup
\Gamma_{g_2} \cup Z\) and thus \(\partial E\) is a set of Jordan measure zero
--- thus \(E\) is Jordan-measurable. The measure formula for \(E\) follows
immediately by \cref{cor:integral-over-graph-maps}
\end{proof}

\begin{corollary}
\label{cor:slicing-measurable-sets-and-integrating}
Let \(E \subseteq \R^n\) be a Jordan-measurable set and \(I \coloneq I_x \times
I_y \subseteq \R^{n-1} \times \R\) be an interval containing \(E\). For each
\(y_0 \in I_y\), define the section \(E_{y_0} \coloneq \{(x, y_{0})\} \subseteq
E\). Then, for almost all \(y_0 \in I_y\) the set \(E_{y_0}\) is
Jordan-measurable, and
\[
  \mu(E) = \int_{I_y} F_{\mu}(y)\, \diff y,
\]
where the map \(F_{\mu}: I_y \to \R\) is defined as \(F_{\mu}(y) \coloneq
\mu(E_y)\) if \(E_y\) happens to be Jordan-measurable, otherwise, we let
\(\underline{\int_{E_y}} \diff y \leq F_{\mu}(y) \leq \overline{\int_{E_y}}
\diff y\) be any value between the lower and upper Darboux integrals of the
measure of \(E_y\).
\end{corollary}

\begin{corollary}[Cavalieri's principle]
\label{cor:cavalieri-principle}
Let \(A\) and \(B\) be two Jordan-measurable solids in \(\R^3\). For every \(z_0
\in \R\), define the sets \(A_{z_0} \coloneq \{(x, y, z_{0}) \in A\}\) and
\(B_{z_0} \coloneq \{(x, y, z_{0}) \in B\}\). If for every \(z_0 \in \R\) the
sets \(A_{z_0}\) and \(B_{z_0}\) have the same area, that is, \(\mu(A_{z_0}) =
\mu(B_{z_0})\), then the solids have the same volumes --- \(\mu(A) = \mu(B)\).
\end{corollary}

\section{Change of Variables}

\begin{remark}
Throughout this section we'll use always the variable names \(D_t\) and \(D_x\)
for \emph{domains} --- recall \cref{def: domain}.
\end{remark}

\begin{definition}[Support of a real valued map]
\label{def:support-real-valued-map}
Given a map \(f: D \to \R\), where \(D \subseteq \R^n\) is a domain, the support
of \(f\), denoted by \(\supp f\), is the closure of the collection of points of
\(D\) with non-zero image under \(f\), that is
\[
  \supp f \coloneq \Cl(\{x \in D \colon f(x) \neq 0\}).
\]
\end{definition}

The goal of this section will be to prove the theorem on the change of variables
in the context of multiple integrals. Our approach will be to prove that such
theorem holds for elementary \(C^1\)-isomorphisms and also for the compositions
of those. We'll decompose the general Riemann integrable map \(f: D_x \to \R\)
into the composition of such elementary \(C^1\)-isomorphisms and hence prove the
theorem. For the proof of the theorem, head to
\cref{subsub:change-variables-proof}.

\begin{theorem}[Change of variables]
\label{thm:change-variables}
Let \(D_x, D_t \subseteq \R^n\) be bounded open domains, and \(\phi: D_t \isoto
D_x\) a \(C^1\)-isomorphism. Let \(f: D_x \to \R\) be a Riemann integrable map
over \(D_x\) for which \(\supp f\) is compact on \(D_x\). Then the map given by
\(t \mapsto f \phi(t) \norm{\det(\Jac \phi(t))}\) is Riemann integrable on
\(D_t\) and
\[
  \int_{D_x} f(x)\, \diff x
  = \int_{D_t} f \phi(t) \norm{\det(\Jac \phi(t))}\, \diff t.
\]
\end{theorem}

\subsection{Measurable Sets and Smooth Mappings}

\begin{lemma}
\label{lem:C1-isomorphisms-preserve-measure-properties}
Let \(\phi: D_t \isoto D_x\) be a \(C^1\)-isomorphism. Then the following is
true:
\begin{enumerate}[(a)]\setlength\itemsep{0em}
\item Given any subset \(E_t \subseteq D_t\) of Lebesgue measure zero, the image
  \(\phi(E_t) \subseteq D_x\) is also a set of Lebesgue measure zero.

\item Let \(E_t \subseteq D_t\) be any open subset with closure \(\Cl E_t
  \subseteq D_t\) of Jordan measure zero. Then its image \(\phi(E_t)\) and the
  closure \(\Cl(\phi(E_t))\) are both contained in \(D_x\) and the closure
  is a set of Jordan measure zero.

\item Let \(E_t \subseteq D_t\) be a Jordan-measurable set, and \(\Cl E_t
  \subseteq D_t\). The image \(\phi(E_t)\) is Jordan measurable and
  \(\Cl(\phi(E_t)) \subseteq D_x\).
\end{enumerate}
\end{lemma}

\begin{proof}
\begin{enumerate}[(a)]\setlength\itemsep{0em}
\item Let \(I \subseteq D_t\) be a closed interval with \(E_t \subseteq
I\). Since \(\phi\) is continuously differentiable, there exists \(M > 0\) such
that \(\norm{\diff \phi(t)(x)} \leq M\) for all \(x \in I\). Using \cref{thm:
several-mvt-Rn-val}, we obtain that for all pairs \(x, y \in I\), we have the
relation \(\norm{f(x) - f(y)}_{\R^n} \leq M \norm{x - y}_{\R^n}\). Given any
\(\varepsilon > 0\), since \(E_t\) is of Lebesgue measure zero, let
\(\{I_{j}\}_{j \in J}\) be a countable collection of open intervals covering
\(E_t\) with \(\sum_{j \in J} \Vol I_j < \varepsilon\) and we may assume that
\(I_j \subseteq I\) for all \(j \in J\). Consider the countable collection
\(\{\phi(I_{j})\}_{j \in J}\), which forms a cover for \(\phi(E_t)\). Let \(t_j
\in I_j\) be the centre point of the interval corresponding to the index \(j \in
J\) --- notice that, since the image of pairs of points of \(D_t\) under
\(\phi\) is bounded by the product of their distance and \(M\) --- defining
\(x_j \coloneq \phi(t_j)\), we can consider the open interval \(I_j' \subseteq
E_t\) with \(x_j\) being its centre point, with dimension \(M\) times of
\(I_j\), then clearly \(I_j'\) is able to cover the whole set \(\phi(I_j)\). Let
\(\{I_{j}'\}_j\) be the collection of those intervals we constructed and notice
that the image \(\phi(E_t)\) is completely encompassed by the union \(\bigcup_{j
\in J} I_j'\), that is, \(\{I_{j}'\}_{j \in J}\) covers \(\phi(E_t)\). Moreover,
we the volume of such collection was constructed so that \(\sum_{j \in J} \Vol
I_j' = \sum_{j \in J} M^n \Vol I_j < M^n \varepsilon\), that is, \(\phi(E_t)\)
can be covered by a collection of open intervals whose total volume can be
chosen to be arbitrarily small volume --- thus \(\phi(E_t)\) is of Lebesgue
measure zero.

\item Since \(\Cl E_t\) is a set of Jordan measure zero, it follows it
  follows that \(\Cl E_t\) is also of Lebesgue measure zero. Notice that,
  by the above item, the image \(\phi(\Cl E_t)\) will also be of Lebesgue
  measure zero. Notice that \(\Cl(\phi(\Cl E_t))\) is a compact set
  of \(\R^n\) and therefore we can apply
  \cref{cor:compact-finite-cover-infinitesimal-volume} to conclude that
  \(\Cl(\phi(\Cl E_t))\) is a set of Jordan measure zero. Since any
  subset of a set of Jordan measure zero also has measure zero, it follows that
  \(\phi(\Cl E_t)\) is of measure zero.

\item Since \(E_t\) is Jordan-measurable, the boundary \(\partial E_t\) is of
  Jordan measure zero, therefore so is its image \(\phi(\partial E_t)\) by the
  use of the last item. Since \(\phi\) is a \(C^1\)-isomorphism, given any \(p \in
  \Int E_t\), we have \(\phi(p) \in \Int \phi(E_t)\), therefore \(\partial
  \phi(E_t) = \phi(\partial E_t)\) --- thus \(\phi(E_t)\) has boundary of Jordan
  measure zero and therefore is Jordan-measurable.
\end{enumerate}
\end{proof}

\begin{corollary}
\label{cor:existence-right-int-change-variables}
In the context of \cref{thm:change-variables}, the integral
\[
\int_{D_t} f\phi(t) \norm{\det(\Jac \phi(t))}\, \diff t
\]
exists.
\end{corollary}

\begin{proof}
Since \(\phi\) is a \(C^1\)-isomorphism, its Jacobian \(\Jac \phi(t)\) is invertible
for every \(t \in D_t\), thus \(\det(\Jac \phi(t)) \neq 0\). If \(g: D_t \to \R\)
is defined as \(g(t) \coloneq f \phi(t) \norm{\det(\Jac \phi(t))}\), then for
every \(t \in D_t\) such that \(\phi(t) \not\in \supp f\) we have \(t \not\in
\supp g\) --- thus \(\supp g = \supp(f \phi \phi^{-1}(\supp f))\), and \(\supp
g\) is a compact subset of \(D_t\) since it is closed. From this we can conclude
that the map \(g_{\chi_{D_t}}\) has points of discontinuity in the fibers
\(\phi^{-1}(x)\), where \(x \in D_x\) is a point of discontinuity of
\(f\). Since \(f\) is Riemann integrable in \(D_x\), such points form a
Lebesgue measure zero set --- thus the points of discontinuity of
\(g_{\chi_{D_t}}\) are also of Lebesgue measure zero since \(\phi\) preserves
Lebesgue measure zero sets by
\cref{lem:C1-isomorphisms-preserve-measure-properties}. Using the Lebesgue's
criterion, we conclude that \(g_{\chi_{D_t}}\) is Riemann integrable over
any closed interval \(I_t\) such that \(D_t \subseteq I_t\) and thus the said
integral does exist.
\end{proof}

\subsection{One-Dimensional Change of Variables}

\begin{lemma}[Changing variables in one dimension]
\label{lem:change-variables-one-dimensional}
Let \(\phi: I_t \isoto I_x\) be a \(C^1\)-isomorphism over the \(1\)-dimensional
intervals \(I_t, I_x \subseteq \R\) and let \(f: I_x \to \R\) be a Riemann
integrable map over \(I_x\). Then the map \(I_t \ni t \mapsto f \phi(t)
\norm{\phi'(t)}\) is a Riemann integrable map over \(I_t\) and the following
relation holds
\[
  \int_{I_x} f(x)\, \diff x = \int_{I_t} f \phi(t) \norm{\phi'(t)}\, \diff t.
\]
\end{lemma}

\begin{proof}
All maps \(f, \phi\) and \(\phi'\) are bounded, thus the map \(g: I_t \to \R\)
given by \(g(t) \coloneq f \phi(t) \norm{\phi'(t)}\) is bounded. As pointed out
in the proof of \cref{cor:existence-right-int-change-variables}, the points of
discontinuity of \(g\) are only those lying in the fibers \(\phi^{-1}(x)\),
where \(x \in I_x\) is a point of discontinuity of \(f\) --- since \(f\) is
Riemann integrable and \(\phi^{-1}\) is a \(C^1\)-isomorphism, the points of
discontinuity of \(g\) form a set of Lebesgue measure zero, thus \(g\) is
Riemann integrable over \(I_t\). Let \(P_x \coloneq \{[x_{j-1}, x_{j}]\}_{j \in
J}\) be a partition of \(I_x\) and define \(P_t \coloneq \{\phi^{-1}([x_{j-1},
x_j])\}_{j \in J}\) as the partition of \(I_t\). Since \(\phi\) and
\(\phi^{-1}\) are both continuous functions over compact sets, it follows that
they are both uniformly continuous --- thus, taking a sequence of partitions
of \(I_x\) and the induced partitions of \(I_t\), we have \(\Mesh P_x \to 0\) if
and only if \(\Mesh P_t \to 0\). Let \(\xi\) be a collection of distinguished
points of \(P_x\) and \(\tau \coloneq \{\phi^{-1}(\xi_{j})\}_{j \in J}\) be the
corresponding collection of distinguishes points of \(P_t\). Notice that,
from construction, we have the following
\[
  \sum_{j \in J} f(\xi_{j}) \norm{x_j - x_{j-1}}
  = \sum_{j \in J} f \phi(\tau_j) \norm{\phi(t_j) - \phi(t_{j-1})}.
\]
Since both \(f\) and \(g\) are Riemann integrable (over their respective
domains), we find that the choice of distinguished points is arbitrary, thus we
may choose \(\xi\) so that, when applying the mean value theorem to \(\phi\), we
have \(\phi'(\tau_j) = \frac{\phi(t_j) - \phi(t_{j-1})}{t_j - t_{j-1}}\) ---
where, as before, \(\tau_j \coloneq \phi^{-1}(\xi_j)\). This way we may rewrite
the equation as
\[
  \sum_{j \in J} f(\xi_{j}) \norm{x_j - x_{j-1}}
  = \sum_{j \in J} f \phi(\tau_{j}) \norm{\phi'(\tau_{j})}\,
  \norm{t_j - t_{j - 1}}.
\]
Now, taking the limit \(\Mesh P_x \to 0\), which implies \(\Mesh P_t \to 0\), we
find the integral equality just wanted.
\end{proof}

\begin{corollary}
\label{cor:C1-isomorphism-intervals-upper-lower-integral}
Let \(\phi: I_t \isoto I_x\) be a \(C^1\)-isomorphism between closed
\(1\)-dimensional intervals \(I_t, I_x \subseteq \R\), and \(f: I_x \to \R\) be
a Riemann integrable map over \(I_{x}\). Then we have the following relations
\[
  \overline{\int_{I_x}} f(x)\, \diff x
  = \overline{\int_{I_t}} f \phi(t) \norm{\phi'(t)}\, \diff t
  \ \text{ and }\
  \underline{\int_{I_x}} f(x)\, \diff x
  = \underline{\int_{I_t}} f \phi(t) \norm{\phi'(t)}\, \diff t.
\]
\end{corollary}

\begin{proof}
We'll verify the first relation, the second may be obtained with an analogous
proof. Construct partitions \(P_x\) and \(P_t\) for \(I_x\) and \(I_t\),
respectively, just as we've done for the proof of
\cref{lem:change-variables-one-dimensional}. Assume for the time being that
\(f\) is a non-negative map (we'll take hold of the general case just in a
bit) and let \(M > 0\) be a constant bounding \(f\). Define \(\varepsilon
\coloneq \sup_{j \in J} \omega(\phi, [t_{j-1}, t_j])\), that is, the supremum
oscillation of \(\phi\) over each of the intervals of the partition
\(P_t\). Notice that we have
{\footnotesize%
\begin{align*}
  \sum_{j \in J} \sup_{x \in [x_{j-1}, x_j]} f(x) \norm{x_j - x_{j-1}}
  &\leq \sum_{j \in J} \sup_{t \in [t_{j-1}, t_j]} f \phi(t)
    \sup_{t \in [t_{j-1}, t_j]} \norm{\phi'(t)}\, \norm{t_j - t_{j-1}} \\
  &\leq \sum_{j \in J} \sup_{t \in [t_{j-1}, t_j]}
    \left[
    f \phi(t) \sup_{t \in [t_{j-1}, t_j]} \norm{\phi'(t)}
    \right]
    \norm{t_j - t_{j-1}} \\
  &\leq \sum_{j \in J} \sup_{t \in [t_{j-1}, t_j]}
    f \phi(t) \left( \norm{\phi'(t)} + \varepsilon \right)
    \norm{t_j - t_{j-1}} \\
  &\leq \sum_{j \in J} \sup_{t \in [t_{j-1}, t_j]}
    f \phi(t) \norm{\phi'(t)}\, \norm{t_j - t_{j-1}}
    + \varepsilon \sum_{j \in J} \sup_{t \in [t_{j-1}, t_j]}
    f\phi(t) \norm{t_j - t_{j-1}} \\
  &\leq \sum_{j \in J} \sup_{t \in [t_{j-1}, t_j]}
    f \phi(t) \norm{\phi'(t)}\, \norm{t_j - t_{j-1}}
    + \varepsilon M \Vol I_t
\end{align*}
}%
where \(\Vol I_t\) happens to be the length of the interval, since \(I_t
\subseteq \R\). Since \(\phi\) is uniformly continuous, we find, from the last
inequalities, that in the limit \(\Mesh P_t \to 0\) (which implies in \(\Mesh
P_x \to 0\)):
\[
  \overline{\int_{I_x}} f(x)\, \diff x
  \leq \overline{\int_{I_t}} f \phi(t) \norm{\phi'(t)}\, \diff t.
\]

Notice that we can use the same analogous proof for the \(C^1\)-isomorphism
\(\phi^{-1}\) and the map \(I_t \ni t \mapsto f \phi(t) \norm{\phi'(t)}\), we
find that
\[
  \overline{\int_{I_t}} f \phi(t) \norm{\phi'(t)}\, \diff t
  \leq \overline{\int_{I_x}} f(x)\, \diff x,
\]
thus the equality for the first relation has been established for non-negative
\(f\). Notice, however, that \(f = \max(f, 0) - \max(-f, 0)\), for any map
\(f\), thus the equality holds for the general case.
\end{proof}

\subsection{Change of Variables on Elementary
\texorpdfstring{\(C^1\)}{C1}-Isomorphisms}

\begin{definition}[Elementary \(C^1\)-isomorphism in \(\R^n\)]
\label{def:elementary-C1-isomorphism-Rn}
Let \(\{t_{j}\}_{j=1}^n\) and \(\{x_{j}\}_{j = 1}^n\) be basis for the euclidean
space \(\R^n\). Given any \(1 \leq k \leq n\), a \(C^1\)-isomorphism \(\phi: D_t
\isoto D_x\) such that \(\phi_j(t_j) = t_j\) for every \(j \neq k\), and
\(\phi_k(t_{k}) = x_k\), is said to be a \(k\)-elementary \(C^1\)-isomorphism ---
that is, the only coordinate of \(\R^n\) changed under the mapping of \(\phi\)
is the \(k\)-th.
\end{definition}

\begin{lemma}
\label{lem:change-variables-elementary-C1-isomorphism}
In the context of \cref{thm:change-variables}, if \(\phi\) is an elementary
\(C^1\)-isomorphism, then the proposition is valid.
\end{lemma}

\begin{proof}
Let \(\phi\) be an elementary \(C^1\)-isomorphism on the \(k\)-th
coordinate. Given points \(x, t \in \R^n\), we define \(x' \coloneq (x_j)_{j=1,
j \neq k}^n \in \R^{n-1}\) and \(t' \coloneq (t_j)_{j=1, j \neq k}^n \in
\R^{n-1}\). Moreover, for every \(x', t' \in \R^{n-1}\), we define
\(1\)-dimensional sections of the domains \(D_x\) and \(D_t\), respectively, as
\begin{gather*}
  D_{x_k}(x') \coloneq
  \{p \in D_x \colon p_k = x_k \text{ and } p_j = x'_j \text{ for } j \neq k\}, \\
  D_{t_k}(t') \coloneq
  \{p \in D_t \colon p_k = t_k \text{ and } t_j = t'_j \text{ for } j \neq k\}. \\
\end{gather*}

Let \(I_x \subseteq \R^n\) be a closed interval containing \(D_x\) and we let
\(I_{x'}^0 \times I_{x_k} \times I_{x'}^1 \coloneq I_x\) be the representation
of \(I_x\) by closed intervals \(I_{x'}^0 \subseteq \R^{k-1}\), \(I_{x_k}
\subseteq \R\) and \(I_{x'}^1 \subseteq \R^{n-k}\). Analogously, let \(I_t
\subseteq \R^n\) be a closed interval containing \(D_t\) and let \(I_{t'}^0
\times I_{t_k} \times I_{t'}^1 \coloneq I_t\) where, as before, \(I_{t'}^0
\subseteq \R^{k-1}\), \(I_{t_k} \subseteq \R\) and \(I_{t'}^1 \subseteq
\R^{n-k}\) are all closed intervals. Thus, we have
{\small
\begin{align}
  \nonumber
  \int_{D_x} f(x)\, \diff x
  &= \int_{I_x} f_{\chi_{D_x}}(x)\, \diff x \\
  \label{eq:elem-diff-cv-fubini}
  &= \int_{I_{x'}^0 \times I_{x'}^1} \left(
  \int_{I_{x_k}} f_{\chi_{D_x}}(x')\, \diff x_k
  \right)\, \diff x' \\
  \label{eq:elem-diff-cv-def}
  &= \int_{I_{x'}^0 \times I_{x'}^1} \left(
    \int_{D_{x_k}(x')} f(x)\, \diff x
    \right)\, \diff x_0' \\
  \label{eq:elem-diff-cv-one-dim}
  &= \int_{I_{t'}^0 \times I_{t'}^1} \left(
    \int_{I_{t_k}} f\phi_{\chi_{D_t}}(t_1', \dots, t_k, \dots, t_n')
    \norm{\det\left( \Jac \phi_{\chi_{D_t}}(t_1', \dots, t_k, \dots, t_n')
    \right)}\, \diff t_k \right)\, \diff t' \\
  \label{eq:elem-diff-cv-fubini-2}
  &= \int_{I_{t'}^0 \times I_{t_k} \times I_{t'}^1}
    f \phi_{\chi_{D_t}}(t) \norm{\det(\Jac \phi_{\chi_{D_t}}(t))}\, \diff t
  = \int_{I_t}
    f \phi_{\chi_{D_t}}(t) \norm{\det(\Jac \phi_{\chi_{D_t}}(t))}\, \diff t
  \\
  \label{eq:elem-diff-cv-def-2}
  &= \int_{D_t} f \phi(t) \norm{\det(\Jac \phi(t))}\, \diff t
\end{align}
}
where we have the following use of theorems and definitions for each of the
equations: \cref{eq:elem-diff-cv-fubini} used \cref{thm:fubini},
\cref{eq:elem-diff-cv-def} used \cref{def:riemann-integral-over-set},
\cref{eq:elem-diff-cv-one-dim} used \cref{lem:change-variables-one-dimensional},
\cref{eq:elem-diff-cv-fubini-2} used \cref{thm:fubini}, and, finally,
\cref{eq:elem-diff-cv-def-2} used \cref{def:riemann-integral-over-set}. This
proves the lemma.
\end{proof}

\begin{proposition}
\label{prop:decomposition-elementary-C1}
Let \(f: E \to \R^n\) be a \(C^1\) map on the open set \(E \subseteq \R^n\),
with \(0 \in E\) --- moreover, we impose that \(f(0) \coloneq 0\) and \(\diff
f(0): T_0\R^n \to T_0\R^n\) is an \(\R\)-linear isomorphism. Then there exists a
neighbourhood \(U \subseteq E\) of \(0 \in \R^n\), with mappings:
\begin{itemize}\setlength\itemsep{0em}
\item \(\{\phi_{j}: U \to \R^n\}_{j=1}^n\) of elementary \(C^1\)-isomorphisms,
  such that \(\phi_j(0) = 0\) and \(\diff \phi(0)\) is an \(\R\)-linear
  isomorphism.
\item We define, for each \(1 \leq k \leq n\), the map \(\tau_k: \R^n \to \R^n\)
  as \(\tau_k(\sum_{j=1}^n x_j e_j) \coloneq \sum_{j=1}^n x_{\tau(j)} e_{j}\),
  where \(\tau \in \Aut_{\Grp}([n])\) is a transposition.
\end{itemize}
These maps allow us to write \(f|_U\) as the composition of elementary
\(C^1\)-isomorphisms and transpositions
\[
  f(x) = \tau_1 \dots \tau_{n-1} g_n \dots g_1(x)\text{, for all } x \in U.
\]
\end{proposition}

\begin{proof}
Define, for every \(0 \leq k \leq n\), the map \(p_k: \R^n \to \R^n\) by
\(p_k(\sum_{j=1}^n x_j e_j) \coloneq \sum_{j=1}^k x_j e_j\).

We do induction in the following proposition, where \(1 \leq m \leq n - 1\)
\begin{itemize}\setlength\itemsep{0em}
\item There exists a neighbourhood \(V_m \subseteq \R^n\) of zero, a map \(f_m
  \in C^1(V_m)\) --- with \(f_m(0) = 0\) and \(\diff f_m(0)\) linear
  isomorphism (we define \(f_1 \coloneq f\))--- for which
  \[
    p_{m-1} f_m(x) = p_{m-1}(x)\text{, for all } x \in V_m.
  \]
\end{itemize}
Notice that the proposition is clearly true for \(m = 1\), now, we assume
it's true for \(1 < m < n - 1\) as our hypothesis of induction.

Notice that \(f_m\) is a map that does only change the last \(x_m, \dots,
x_{n}\) variables of its input, which allows us to write it as
\[
  f_{m}(x) = p_{m-1}(x) + \sum_{j=m}^n \alpha_j(x) e_j,
\]
where \(\alpha_j: V_m \to \R\) is a \(C^1(V_m)\) map for all \(m \leq j \leq
n\), moreover, this implies in
\[
  \diff f_m(0)(e_m) = \sum_{j = m}^n \partial_{m} \alpha_j(0) e_{j}.
\]
Since \(\diff f_{m}(0)\) is an isomorphism, it cannot be the zero-map, thus
there must exist some \(m \leq k_0 \leq n\) for which \(\partial_m
\alpha_{k_0}(0) \neq 0\). Define \(\tau_m\) to be the transposition interchanging
the \(m\)-th and \(k_0\)-th values of its input. Define a map \(\phi_m: V_m \to
\R^n\) as
\[
  \phi_m(x) \coloneq x + (\alpha_{k_0}(x) - x_m) e_m,
\]
so that \(\phi_{m} \in C^1(V_m)\), and only changes the \(m\)-th coordinate of
its input. Moreover \(\diff \phi_m(0)\) is an isomorphism, thus there must
exist a neighbourhood \(U_m \subseteq V_m\) of zero for which the induced map
\(\phi_m: U_m \to \phi_m(U_m)\) is a \(C^1\)-isomorphism (which, in fact, is
elementary). We define \(V_{m + 1} \coloneq \phi_m(U_m)\), and also
\[
  f_{m+1}(y) \coloneq \tau_m f_m\phi^{-1}_m(y)\text{, for all } y \in V_{m+1},
\]
which makes \(f_{m+1} \in C^{1}(V_{m+1})\), also \(f_{m+1}(0) = 0\), and \(\diff
f_{m+1}(0)\) is an isomorphism. Notice that, for all \(x \in U_m\) we have
\begin{align*}
  p_m f_{m+1}(\phi_m(x))
  &= p_m \tau_m f_m \phi_m^{-1}(\phi_m(x)) \\
  &= p_m \tau_m f(x) \\
  &= p_m \tau_m \left[ p_{m-1}(x) + \sum_{j=m}^n \alpha_j(x) e_j \right] \\
  &= p_{m} \left[
    p_{m-1}(x) + \alpha_{k_0}(x) e_m + \dots
    + \alpha_m(x) e_{k_0} + \dots + \alpha_n(x) e_n
    \right] \\
  &= p_{m-1}(x) + \alpha_{k_0}(x) e_m \\
  &= p_m \phi_m(x).
\end{align*}
That is, since \(\phi_m\) is a bijection in \(V_{m+1}\), for any \(y \in
V_{m+1}\) we have the equality
\[
  p_m f_{m+1}(y) = p_m(y).
\]
This concludes the proof by induction of our initial statement. We now must show
that it, in fact, implies the proposition we set out to prove. For that we
recall the definition of \(f_{m+1}\) and notice that, by taking \(\tau_m\) from
both sides we obtain \(\tau_m f_{m+1}(y) = \tau_{m} \tau_m f_m \phi_m^{-1}(y) =
f_m \phi_m^{-1}(y)\) for every \(y \in V_{m+1}\) --- which allow us to write,
taking \(y = \phi_m(x)\),
\[
  f_m(x) = \tau_m f_{m+1}(\phi_m(x)) \text{, for every } x \in U_m.
\]
This recursive definition implies in (recalling that we set \(f_1 \coloneq f\)):
\[
  f = \tau_1 f_2 \phi_1
  = \dots
  = \tau_1 \dots \tau_{n-1} f_n \phi_{n-1} \dots \phi_{1},
\]
which is what we wanted since \(f_n\) is an elementary \(C^{1}\)-isomorphism for
some neighbourhood of zero.
\end{proof}

\subsection{Proof for the Theorem on Change of Variables}
\label{subsub:change-variables-proof}

We now go for our (finally) last lemma before we can get to the proof of the
main theorem of the section.

\begin{lemma}
\label{lem:composition-diffeo-change-variables}
Let \(D_s \xrightarrow{\psi} D_t \xrightarrow{\phi} D_x\) be
\(C^1\)-isomorphisms. Moreover, let \(f: D_x \to \R\) be a Riemann integrable
map over \(D_x\). If \cref{thm:change-variables} holds for both \(\psi\), and
for \(\phi\), then it is also valid for the composition \(\phi \psi: D_s \isoto
D_x\).
\end{lemma}

\begin{proof}
Notice that, for every \(s \in D_s\), \(\Jac (\phi \psi)(s) = \Jac
\phi(\psi(s)) \Jac \psi(s)\), then
\[
  \det(\Jac {(\phi \psi)}(s)) = \det(\Jac \phi(\psi(s))) \det(\Jac \psi(s)).
\]
Therefore we can write
{\small
\begin{align*}
  \int_{D_x} f(x)\, \diff x
  = \int_{D_t} f \phi(t) \norm{\det(\Jac \phi(t))}\, \diff t
  &= \int_{D_s} f \phi \psi(s) \norm{\det(\Jac \phi(\psi(s)))}
  \norm{\det(\Jac \psi(s))}\, \diff s \\
  &= \int_{D_s} f \phi \psi(s) \norm{\det(\Jac{(\phi \psi)}(s))}\, \diff s,
\end{align*}
}
which proves the proposition.
\end{proof}

Now we are ready for the proof of \cref{thm:change-variables}.

\begin{proof}
Define the compact set \(K_t \coloneq \supp(t \mapsto f \phi(t) \norm{\det(\Jac
\phi(t))})\), and, for every \(t \in K_t\), let \(U(t) \subseteq D_t\) be a
neighbourhood of \(t\) with diameter less than \(\delta(t)\), for some
\(\delta(t) > 0\) --- for which \(\phi|_{U(t)}\) can be decomposed into
elementary \(C^1\)-isomorphisms. For each \(t \in K_t\), let \(U'(t) \subseteq
U(t)\) be a neighbourhood of \(t\) with diameter less than \(\delta(t)/2\) ---
with this, the collection \(\{U'(t)\}_{t \in K_t}\) is a cover for \(K_t\), more
than that, since \(K_t\) is compact, there exists a finite collection of points
\(\{t_{j}\}_{j=1}^k\) such that \(\{U'(t_j)\}_{j=1}^k\) is a cover for
\(K_t\). Define \(\delta \coloneq \frac 1 2 \min(\delta(t_j)_{j=1}^k)\) so that,
for any set \(A \subseteq D_t\) with closure of diameter less than \(\delta\)
and non-empty intersection with the support, \(A \cap K_t \neq \emptyset\), the
given set must be contained in some \(U'(t_m)\) for some \(1 \leq m \leq k\).

Let \(I \subseteq \R^n\) be an interval containing \(D_t\), and \(P \coloneq
\{I_{s}\}_{s \in S}\) a partition of \(I\) with \(\Mesh P < \min(\delta, d)\)
--- where \(d \coloneq \inf_{(a, b) \in \partial K_t \times \partial D_t}
\norm{b - a}\) is the minimal distance between the boundaries of \(K_t\) and
\(D_{t}\). Let \(S'\) be the index set such that \(I_s \cap K_t \neq \emptyset\)
if and only if \(s \in S'\) --- the set indices of intervals of the partition
\(P\) with non-empty intersection with the support. Then
\begin{align*}
  \int_{D_t} f\phi(t) \norm{\det(\Jac \phi(t))}\, \diff t
  &= \int_I f \phi_{\chi_{D_t}}(t)
  \norm{\det(\Jac \phi_{\chi_{D_t}}(t))}\, \diff t \\
  &= \sum_{s' \in S'} \int_{I_{s'}} f \phi(t) \norm{\det(\Jac \phi(t))}\, \diff t.
\end{align*}

Since \(I_s \in P\) are Jordan-measurable sets, so is their image \(\phi(I_s)
\subseteq D_x\) --- this follows from
\cref{lem:C1-isomorphisms-preserve-measure-properties}. Define the
Jordan-measurable set \(E \coloneq \bigcup_{s' \in S'} I_{s'}\) so that, by our
construction, given any \(x \in \supp f\), let \(t \coloneq \phi^{-1}(x)\), then
\(f\phi(t) \norm{\det(\Jac \phi(t))} \neq 0\) thus \(t \in K_t\) --- therefore,
\(t \in \bigcup_{s' \in S'} I_{s'}\), which implies in \(x \in E\), that is,
\(\supp f \subseteq E\). Notice that, if \(I_x \subseteq \R^n\) is an interval
containing \(D_x\), then
\begin{align}
  \nonumber
  \int_{D_x} f(x)\, \diff x
  &= \int_{I_x} f_{\chi_{D_x}}(x)\, \diff x
  = \underbrace{\int_{I_x \setminus E} f_{\chi_{D_x}}(x)\, \diff x}_0
  + \int_E f_{\chi_{D_x}}(x)\, \diff x \\
  \label{eq:cv-diff-1}
  &= \int_E f(x)\, \diff x
  = \sum_{s' \in S'} \int_{\phi(I_{s'})} f(x)\, \diff x.
\end{align}
Since \(\phi\) is decomposed into elementary \(C^1\)-isomorphisms in
any neighbourhood \(U'(t)\), for \(t \in K_t\), it follows that \(\phi\)
decomposes into elementary \(C^1\)-isomorphisms in every interval \(I_{s'}\) for
\(s' \in S'\) --- hence, by means of
\cref{lem:change-variables-elementary-C1-isomorphism} we conclude that, for
every \(s' \in S'\)
\begin{equation}
\label{eq:cv-diff-2}
  \int_{\phi(I_{s'})} f(x)\, \diff x = \int_{I_{s'}} f\phi(t) \norm{\det(\Jac
    \phi(t))}\, \diff t.
\end{equation}
Thus, merging both \cref{eq:cv-diff-1} and \cref{eq:cv-diff-2}, we obtain the
desired equality.
\end{proof}


%%% Local Variables:
%%% mode: latex
%%% TeX-master: "../../../deep-dive"
%%% End:

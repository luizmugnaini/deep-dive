\section{Subspaces}

\begin{definition}[Subspace topology]
  \label{def: subspace topology}
  Given a topological space \((X, \mathcal T)\) and a subset \(S \subseteq X\),
  we define the space \((S, \mathcal T_S)\) as a subspace of \(X\) if
  \[
    \mathcal T_S = \{U \subseteq S: U = S \cap V,\ V \in \mathcal T\}.
  \] 
  If so, we call \(\mathcal T_S\) the relative topology or subspace topology.
\end{definition}

\begin{proposition}\label{prop: closed in subspace}
  Let \(X\) be a topological space and \(S\) be a subspace of \(X\). A set \(A
  \subseteq S\) is closed in \(S\) if and only if \(A = S \cap F\) for some
  closed set \(F \subseteq X\) with respect to \(X\). The closure of \(A\) with
  respect to \(S\), denoted \(\widetilde A\), is such that \(\widetilde A = S
  \cap \overline A\).
\end{proposition}

\begin{proof}
  (\(\Rightarrow\)) Suppose \(A\) is a closed set in \(S\), then \(S \setminus
  S = S \cap U\) is open and hence \(U\) is some open set of \(X\). Then 
  \[
    A = S \setminus (S \setminus A) = S \setminus (S \cap U)
    = S \cap (X \setminus U)
  \] 
  where \(X \setminus U\) is closed, hence \(A\) equals to the intersection of
  \(S\) with a closed set of \(X\).
  (\(\Leftarrow\)) Suppose that \(A = S \cap F\) for a closed set \(F\) in
  \(X\). Then we have
  \[
    S \setminus A = S \setminus (S \cap F) = S \cap (X \setminus F)
  \]
  Since \(X\setminus F\) is open, then \(S \setminus A\) is open. We conclude
  that \(A\) is closed in \(S\).

  For the last proposition, notice that \(\widetilde A = \bigcap \{F \subseteq S
  : F \supseteq A, F \text{ is closed in } S\}\), that is, the intersection of
  sets \(S \cap C\) such that \(C\) is closed in \(X\) and \(C \supseteq A\), so
  that \(\widetilde A = S \cap \overline A\).
\end{proof}

\begin{proposition}\label{prop: relative open to open}
  Let \(X\) be a topological space and \(S\) be a subspace. Then
  \begin{enumerate}[(a)]
    \item If \(U \subseteq S\) is open (resp. closed) and \(S\) is an open
      (resp. closed) subset of \(X\), then \(U\) is open (resp. closed) in
      \(X\). 
    \item If \(U \subseteq S\) and \(U\) is open (resp. closed) in \(X\), then
      it is open (resp. closed) in \(S\).
  \end{enumerate}
\end{proposition}

\begin{proof}
  (a) Suppose that \(U\) is open in \(S\) and \(S\) is open in \(X\). From the
  definition of the relative topology, there exists an open \(V \subseteq X\) in
  \(X\) such that \(U = S \cap V\). Since \(S\) and \(V\) are both open in
  \(X\), then \(U\) is open in \(X\). From \cref{prop: closed in subspace} we
  find that if \(U\) is closed in \(S\) then \(B = S \cap F\) for some closed
  set \(F \subseteq X\). Moreover, if \(S\) is closed in \(X\), then clearly
  \(B\) is closed in \(X\).

  (b) If \(B \subseteq S\) then clearly \(B \cap S = B\), hence if \(B\) is open
  in \(X\), so it is in \(S\). Moreover, if \(B\) is closed in \(X\), from
  \cref{prop: closed in subspace} \(B\) is closed in \(S\).
\end{proof}

\begin{definition}[Second definition of the subspace topology]
  \label{def: second def subspace top}
  Let \(X\) be a topological space and \(S \subseteq X\) be any set. The
  subspace topology on \(S\) is the coarsest topology on the set such that the
  inclusion \(\iota_S: S \to X\) is continuous.
\end{definition}

\begin{corollary}
  Definitions \ref{def: subspace topology} and \ref{def: second def subspace
  top} are equivalent.
\end{corollary}

\begin{proof}
  Suppose \(\mathcal T_{\iota_S}\) is the coarsest topology such that
  \(\iota_S\) is continuous and \(\mathcal T_S\) be the subspace topology. Our
  goal is to show that they are, in fact, equal. Given any \(U = S \cap V \in
  \mathcal T_S\) we have \(\iota_S^{-1}(V) = S \cap V = U\), since \(V\) is
  open, we find that \(U\) is open in \(\mathcal T_{\iota_S}\). This implies in
  \(\mathcal T_S \subseteq \mathcal T_{\iota_S}\). Moreover, let \(O \in
  \mathcal T_{\iota_S}\), then there must exist some \(A \subseteq X\) open set
  for which \(\iota_S^{-1}(A) = A \cap S = O\) (this comes directly from the
  fact that \(\mathcal T_{\iota_S}\) was solely constructed for the purpose of
  making \(\iota_S\) continuous), since \(A\) is open, then \(O = A \cap S \in
  \mathcal T_S\). This implies that \(\mathcal T_{\iota_S} \subseteq \mathcal
  T_S\). We conclude finally that both definitions are indeed equivalent.
\end{proof}

\begin{theorem}[Universal property of the subspace topology]
  Let \(X\) be a topological space and \(S\) be a subspace. Given any
  topological space \(Y\), a map \(f: Y \to S\) is continuous if and only if
  \(\iota_S \circ f : Y \to X\) is continuous, where \(\iota_S: S
  \hookrightarrow X\) is the inclusion map. Hence the following diagram commutes
  \[
    \begin{tikzcd}
      Y \ar[r, "f"] \ar[rd, swap, "\iota_S \circ f"]
        &S \ar[d, "\iota_S"] \\
        &X
    \end{tikzcd}
  \] 
  Moreover, on the converse, if \(S \subseteq X\) is a topological space such
  that the above property holds, then it is equipped with the subspace topology.
\end{theorem}

\begin{proof}
  First we show that if \((S, \mathcal T_S)\) is a subspace of \((X, \mathcal
  T)\) then it satisfies the universal property. Suppose then that \(S\) a
  subspace of \(X\). (\(\Rightarrow\)) Let \(f\) be continuous. If \(V \subseteq
  X\) is any open subset, we have that
  \[
    (\iota_S \circ f)^{-1}(V) = f^{-1}(\iota^{-1}(V)) = f^{-1}(S \cap V)
  \] 
  since \(V\) is said to be open, then \(S \cap V\) is open in \(S\), which
  implies that \(f^{-1}(S \cap V) = (\iota_S \circ f)^{-1}(V)\) is open. This
  shows that the map \(\iota_S \circ f\) is open. (\(\Leftarrow\)) Let \(\iota_S
  \circ f\) be continuous. Consider \(U = S \cap V = \iota_S^{-1}(V)\) to be any
  open set of the subspace \(S\) (that is \(V\) is an open of the space \(X\)).
  Then we have
  \[
    f^{-1}(U) = f^{-1}(\iota_S^{-1}(V)) = (\iota_S \circ f)^{-1}(V)
  \] 
  since \(\iota_S \circ f\) is continuous, then \(f^{-1}(U)\) is open, therefore
  \(f\) is continuous.

  We now show that if an object satisfies such the property, then it is the
  subspace. Let \((S, \mathcal T')\) be a space satisfying the universal
  property. In particular we can take the subspace \((S, \mathcal T_S)\) of
  \((X, \mathcal T)\) and the identity map \(\Id_S: (S, \mathcal T_S) \to (S,
  \mathcal T')\). Since \((S, \mathcal T')\) satisfies the universal property,
  we have that \(\Id_S\) is continuous if and only if \(\iota_S \circ \Id_S =
  \iota_S\) is continuous. That is
  \[
    \begin{tikzcd}
      (S, \mathcal T_S)
      \ar[r, "\Id_S"] 
      \ar[rd, swap, "\iota_S \circ \Id_S = \iota_S"]
        &(S, \mathcal T') \ar[d, "\iota_S"] \\
        &(X, \mathcal T)
    \end{tikzcd}
  \] 
  We know from \cref{def: second def subspace top} that \(\iota_S \circ \Id_S =
  \iota_S\) is continuous for the subspace topology \(\mathcal T_S\), hence the
  universal property implies that \(\Id_S\) is continuous. In particular, this
  says that \(\mathcal T' \subseteq \mathcal T_S\). In order to show the other
  side of the equality, consider now the space \((S, \mathcal T')\) and the
  identity map \(\Id_S': (S, \mathcal T') \to (S, \mathcal T')\) so that from the
  universal property of \((S, \mathcal T')\) the map \(\Id_S'\) is continuous if
  and only if \(\iota_S \circ \Id_S' = \iota_S\) is continuous. That is, the
  following diagram commutes
  \[
    \begin{tikzcd}
      (S, \mathcal T') \ar[r, "\Id_S'"] 
      \ar[rd, swap, "\iota_S \circ \Id_S' = \iota_S"]
        &(S, \mathcal T') \ar[d, "\iota_S"] \\
        &(X, \mathcal T)
    \end{tikzcd}
  \] 
  Notice that since \(\Id_S'\) is continuous (see \cref{prop: continuous maps
  properties}) then from the universal property the map \(\iota_S\) is
  continuous on \(\mathcal T'\). Since \(\mathcal T_S\) is the coarsest topology
  such that \(\iota_S\) is continuous (see \cref{def: second def subspace top}),
  then clearly \(\mathcal T_S \subseteq \mathcal T'\). This finishes the proof
  that \(\mathcal T' = \mathcal T_S\) and hence the space \((S, \mathcal T') =
  (S, \mathcal T_S)\) is the subspace of \((X, \mathcal T)\).
\end{proof}

\begin{corollary}\label{cor: subspace maps properties}
  Let \(X, Y\) be topological spaces and \(f: X \to Y\) be a continuous map.
  Then the following hold
  \begin{enumerate}[(a)]
    \item (Domain restriction) Let \(S\) be a subspace of \(X\). Then \(f|_S\)
      is continuous.
    \item (Codomain restriction) Let \(T\) be a subspace of \(Y\) such that
      \(f(X) \subseteq T\). Then \(f: X \to T\) is continuous.
    \item (Codomain expansion) Let \(Y\) be a subspace of \(Z\). Then the map
      \(f : X \to Z\) is continuous.
  \end{enumerate}
\end{corollary}

\begin{proof}
  (a) Notice that \(f|_S = f \circ \iota_S\). Applying the universal property,
  we find that since \(f\) is continuous, so is \(f|_S\). (b) From the universal
  property we have that \(\iota_T \circ f = f\) is continuous, so is \(f: X \to
  T\). (c) Notice that from the universal property we have \(\iota_Y \circ f: X
  \to Z\) continuous, since \(f\) is continuous. The following are the universal
  property diagrams for items (b) and (c):
  \[
    \begin{tikzcd}
      X \rar["f"] \ar[rd, swap, "\iota_T \circ f = f"] &T \dar["\iota_T"] \\ &Y
    \end{tikzcd} 
    \qquad
    \begin{tikzcd}
      X \rar["f"] \ar[rd, swap, "\iota_Y \circ f"] &Y \dar["\iota_Y"] \\ &Z
    \end{tikzcd}
  \]
\end{proof}

\begin{proposition}[Subspace propeties]\label{prop: subspace properties}
  Let \((X, \mathcal T)\) be a topological space and \((S, \mathcal T_S)\) be a
  subspace of \(X\). The following are properties concerning the subspace
  topology
  \begin{enumerate}[(SP1)]
    \item\label{prop: subspace transitivity}
      Let \(T\) be a subspace of \(S\). Then \(T\) is a subspace of \(X\).  \item\label{prop: basis for subspace} Let \(\mathcal B\) be a basis for \(X\). Then the collection \(\mathcal B_S = \{B \cap S : B \in \mathcal B\}\) is a basis for \(S\).
    \item\label{prop: convergence subspace}
      Let \((p_i)_{i \in \mathbb{N}} \subseteq S\) be a sequence and \(p \in
      S\). Then \(p_i \to p\) in \(S\) if and only if \(p_i \to p\) in \(X\).
    \item\label{prop: Hausdorff implies Hausdorff subspace}
      A subspace of a Hausdorff space is Hausdorff.
    \item\label{prop: firs count implies first count subspace}
      A subspace of a first countable space is first countable.
    \item\label{prop: sec count implies sec count subspace}
      A subspace of a second countable space is second countable.
  \end{enumerate}
\end{proposition}

\begin{proof}
  (SP1) Let \(Z\) be some topological space, we can choose maps \(f: Z \to T\)
  and \(g = \iota_T \circ f : Z \to S\) and apply the universal property on both
  \(T\) and \(S\) in order to get
  \[
    \begin{tikzcd}
      Z \ar[r,"f"] 
      \ar[rd, swap, "g"]
      \ar[rdd, swap, bend right,
      "\iota_S \circ g = \iota_T' \circ f"]
        &T \ar[d, hook, "\iota_T"] 
        \ar[dd, hook, bend left = 60, "\iota_S \circ \iota_T := \iota_T'"]\\
        &S \ar[d, hook, "\iota_S"] \\
        &X
    \end{tikzcd}
    \qquad \Leftrightarrow \qquad
    \begin{tikzcd}
      Z \ar[dr, swap, "\iota_T' \circ f"] \ar[r, "f"] 
        &T \ar[d, hook, "\iota_T'"] \\
        &X
    \end{tikzcd}
  \] 
  Moreover, if \(f\) is continuous, from the universal property of \(T\) we find
  that \(g\) is continuous, but using the universal property of \(S\) that tells
  us that \(\iota_S \circ g = \iota_T' \circ f\) is continuous. The converse is
  true by using the same argumentation. Hence \(T\) satisfies the universal
  property and therefore is a subspace of \(X\).

  (SP2) First notice that since \(B \in \mathcal B\) is open in \(X\), then
  \(\mathcal B_S \subseteq \mathcal T_S\). Let \(s \in S\) be any element, then
  in particular we have \(s \in B_s \in \mathcal B\) for some element of the
  basis. Then \(s \in S \cap B_s \in \mathcal B_S\). Let \(A = S \cap B_1, B = S
  \cap B_2 \in \mathcal B_S\) and \(x \in A \cap B\). Then in particular \(x \in
  B_1 \cap B_2\). Since \(\mathcal B\) is a basis for \(X\), then exists \(B_3
  \in \mathcal B\) such that \(x \in B_3 \subseteq B_1 \cap B_2\). Hence the
  corresponding set \(C = S \cap B_3 \in \mathcal B_S\) is such that \(x \in C
  \subseteq A \cap B\). This proves that \(\mathcal B_S\) is a basis.

  (SP3) (\(\Rightarrow\)) Suppose \(p_i \to p\) in \(S\), that is, for all \(U_p
  = S \cap V_p \in \mathcal T_S\), where \(V_p \in \mathcal T\), we have some
  \(N \in \mathbb{N}\) such that \(\forall n \geq N, p_n \in U_p\). In
  particular, this implies that \(\forall n \geq N, p_n \in V_p\), hence \(p_i
  \to p\) in \(X\).
  (\(\Leftarrow\)) Suppose that \(p_i \to p\) in \(X\). Let \(V_p \subseteq X\)
  be a neighbourhood of \(p\), then since \(p \in S\) we have that \(V_p \cap S
  \neq \emptyset\) is an element of \(\mathcal T_S\). Let \(U_p \in
  \mathcal T_S\) be any neighbourhood of \(p\), then there exists \(V \in
  \mathcal T\) such that \(U_p = S \cap V\) (from the definition of the subspace
  topology) and also \(p \in V\), so that \(V\) is a neighbourhood of \(p\).
  Therefore there exists \(M \in \mathbb{N}\) such that \(\forall n \geq M, p_n
  \in S \cap V = U_p \subseteq V\). This implies that \(p_i \to p\) in \(S\).

  (SP4) Let \(X\) be Hausdorff. Let \(x, y \in S\) be distinct points. In
  particular, there exits \(A, B \in X\) neighbourhoods of \(x\) and \(y\),
  respectively, such that \(A \cap B = \emptyset\). Hence the sets \(U = S \cap
  A, V = S \cap B \in \mathcal T_S\) are neighbourhoods of \(x\) and \(y\)
  respectively and since \(U \subseteq A\) and \(V \subseteq B\) we find that
  \(U \cap V = \emptyset\). Hence \(S\) is Hausdorff.

  (SP5) Let \(X\) be a first countable space. Let \(p \in S\) be any point.
  Define a countable base \(\mathcal B_p\) of neighbourhoods of \(p\) for \(X\).
  Define the, clearly countable, collection \(\mathcal B_p' = \{S \cap B_p: B_p
  \in \mathcal B_p\} \subseteq 2^S\). From \cref{prop: basis for subspace} we
  have that \(\mathcal B_p'\) is a basis for \(S\). Moreover, every element of
  \(\mathcal B_p'\) is clearly a neighbourhood of \(p\). This proves the
  property.

  (SP6) Let \(X\) be second countable and \(\mathcal B\) be a countable basis
  for the topology of \(X\). Define the, clearly countable, collection
  \(\mathcal B' := \{S \cap B : B \in \mathcal B\}\). Since \(\mathcal B\) is a
  basis for \(X\), we use \cref{prop: basis for subspace} to conclude that
  \(\mathcal B'\) is a basis for \(S\).
\end{proof}

\subsection{Topological Embeddings}

\begin{definition}[Embedding]\label{def: topological embedding}
  Let \(f: Y \to X\) be a continuous injective map of topological spaces. We
  call \(f\) an embedding when \(f': Y \isoto f(Y)\) is a homeomorphism.
\end{definition}

\begin{example}
  Let \(X\) be a topological space and \(S\) be a subspace of \(X\). We show
  that \(\iota_S: S \hookrightarrow X\) is an embedding. Notice that
  \(\iota_S(S) = S\) and hence \(\iota_S': S \to \iota(S) = S\) is equal to the
  identity map \(\Id_S\), which clearly establishes a homeomorphism.
\end{example}

\begin{proposition}
  Let \(f\) be a continuous injective map between topological spaces. If \(f\)
  is either open or closed, then \(f\) is an embedding.
\end{proposition}

\begin{proof}
  Let \(f: X \to Y\). Since \(f\) is injective, the codomain restricted map
  \(f': X \xrightarrow f f(X)\) is a bijection and continuous by \cref{cor:
  subspace maps properties}. Suppose \(f\) is open (resp. closed) and consider
  any open (resp. closed) set \(U \subseteq X\), then \(f(U) \subseteq f(X)\) is
  open (resp. closed) in \(f(X)\) this shows that \(f'\) is a homeomorphism.
\end{proof}

\begin{proposition}
  A surjective embedding is a homeomorphism.
\end{proposition}

\begin{proof}
  Let \(f: X \to Y\) be a surjective embedding. Notice that in a surjective map
  we have \(f(X) = Y\), hence \(f' = f : X \isoto f(X) = Y\).
\end{proof}

\begin{example}[2-torus surface]
  Consider first a circle \((x - d)^2 + z^2 = r^2\), with center at \((d, 0)\).
  Let \(\phi\) be the angle going up from the \(x\) to the \(z\) axis, then we
  can parametrize such circle as \(x = r \cos(\phi) + d,\ z = r\sin(\phi)\).
  Lets now consider the revolution of such circle around the \(z\) axis. If
  \(\theta\) is the angle going up from the \(x\) axis to the \(y\) axis, we
  find the new equation \((\sqrt{x^2 + y^2} - d)^2 + z^2 = r^2\) has a
  parametrization given by
  \[
    (x, y, z) = ((r \cos(\phi) + d)\cos(\theta), (r\cos(\phi) + d)\sin(\theta),
    r\sin(\phi)).
  \]
  This mapping is clearly continuous. Although not injective, we can choose any
  point \(p = (x, y)\) and find a neighbourhood of \(p\) such that the mapping
  is injective. 
\end{example}

\begin{lemma}[Gluing]
  Let \(X\) and \(Y\) be topological spaces and consider \(\mathcal U =
  \{U_i\}_i\) an open cover (or finite closed cover) of \(X\). Let \(\{f_i: U_i
  \to Y\}_i\) be a collection of continuous maps such that \(f_i|_{U_i \cap U_j}
  = f_j|_{U_i \cap U_j}\) for all indices \(i\) and \(j\). Then there exists a
  unique continuous map \(f: X \to Y\) such that \(f|_{U_i} = f_i\) for all \(i\).
\end{lemma}

\begin{proof}
  Let \(\mathcal U\) be open, then given any point \(p \in X\), \(f|_{U_p} =
  f_p\) is continuous for some neighbourhood \(U_p \in \mathcal U\) of \(p\),
  which implies that \(f\) is continuous. On the other hand, if \(\mathcal U\)
  is a finite closed cover of \(X\), then consider \(C \subseteq Y\) to be any
  closed set, then we have \(f_i^{-1}(C) = f^{-1}(V) \cap U_i\) is closed in
  \(U_i\) (since \(f_i\) is continuous). Since \(U_i\) is closed in \(X\), we
  find that \(f_i^{-1}(C)\) is closed in \(X\) (see \cref{prop: relative open to
  open}).  Notice that, if \(|\mathcal U| = n\), then \(f^{-1}(C) =
  \bigcup_{i=1}^n f_i^{-1}(C)\) is the finite union of closed sets in \(X\),
  which implies that \(f^{-1}(C)\) itself is closed. This shows us that the
  inverse of \(f\) maps closed sets to closed sets, which implies that \(f\) is
  continuous.  Since \(\mathcal U\) is a cover of \(X\), it is clear that \(f\)
  is unique. \(\mathbb{Z}\)
\end{proof}

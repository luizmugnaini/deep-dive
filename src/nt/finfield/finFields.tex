\section{Some Results on Finite Fields}

\begin{definition}[Multiplicative Group]
   Let \(k\) be a finite field. We define \(k^\ast = k \setminus \{0\}\),
   composed with the invertible elements of \(k\) equipped with multiplication,
   to be the \emph{multiplicative group} of \(k\). Therefore, from the
   definition, \(\forall \alpha \in k^\ast\) is a root of the polynomial
   \(x^{|k|-1} - 1 \in k[x]\), also, \(\forall \alpha' \in k\) is a root of the
   polynomial \(x^{|k|} - x \in k[x]\).
\end{definition}

\begin{proposition}
   Let \(k\) be a finite field with \(|k| := q\). Then
   \[
      x^{q} - x = \prod_{\alpha \in k} (x-\alpha).
   \] 
\end{proposition}

\begin{proof}
   Notice that both polynomials are monic of degree \(q\). Since every root
   coincides from the definition of the finite field \(k\), the equality
   follows.
\end{proof}

\begin{corollary}
   Let \(K\) be a field and \(k \subseteq K\) be a finite subfield with \(q\)
   elements. Let \(\alpha \in K\), then 
   \[
      \alpha \in k \Leftrightarrow \alpha^q = \alpha.
   \] 
\end{corollary}

\begin{proof}
   (\(\Rightarrow\)) Let \(\alpha \in k\), then \(\alpha\) is a root of \(x^q -
   x\), thus \(\alpha^q = \alpha\). (\(\Leftarrow\)) Let \(\alpha \in K\) such
   that \(\alpha^q = \alpha\), then \(\alpha\) is a root of \(x^q - x\) and thus
   is an element of \(k\).
\end{proof}

\begin{corollary}
   Let \(f, x^q - x \in k[x]\), where \(|k| = q\). If \(f(x) \mid x^q - x\) then
   \(f\) has \(\deg(f)\) distinct roots.
\end{corollary}

\begin{proof}
   If \(f\) divides  \(x^q - x\) then there exists \(g \in k[x]\) such that
   \(x^q - x = f(x)g(x)\) and therefore \(\deg(g) = \deg(f) - q\). Suppose then
   that \(f\) had less than \(\deg(f)\) roots, then this implies in  \(x^q - x\)
   having less than \(q\) roots, which is definitely false from the
   construction of \(k\). Therefore \(f\) has \(\deg(f)\) distinct roots.
\end{proof}

\begin{theorem}
   Let \(k\) be a finite group. The multiplicative group \(k^\ast\) is cyclic.
\end{theorem}

\begin{proof}
   Let \(|k| := q\) and \(d \mid q-1\), then exists \(a\) such that \(da = q-1\)
   and thus \(x^{q-1} - 1 = (x^d - 1) ((x^d)^{a-1} + \dots + 1)\) and thus
   \(x^d - 1 \mid x^{q-1} - 1\). From last corollary this implies in \(x^d - 1\)
   having \(d\) distinct roots. We now consider the subgroup of \(k^\ast\)
   composed of elements such that \(x^d = 1\) (they have order \(d\)). Define
   now the function \(\psi\) to be such that \(d \mapsto |\{\alpha \in k^\ast :
   \alpha^d = 1\}|\) we have already shown in \ref{unitCyclic} that \(\sum_{c
   \mid d} \psi(c) = d\) and thus we can apply the Möbius inversion theorem to
   obtain  \(\psi(d) = \sum_{c\mid d} \mu(c) (d/c) = \varphi(d)\). We conclude
   that \(\psi(d-1) = \varphi(d-1) \geqslant 1\) for any \(d\), showing the
   existence of an element of order \(d-1\) and showing that the group is indeed
   cyclic.
\end{proof}

\begin{proposition}
   Let \(k\) be a finite field of \(q\) elements and \(\alpha \in k^\ast\). Then
   \(x^n = \alpha\) is solvable if and only if \(\alpha^{\frac{q-1}{d}} = 1\),
   where \(d := (n, q-1)\). Also, if there exists a solution, there are exactly
    \(d\) solutions.
\end{proposition}

\begin{proof}
   Let since \(k^\ast\) is cyclic, let \(\langle \omega \rangle = k^\ast\).
   Define now elements \(0 \leqslant u,v \leqslant q-1\) such that \(\omega^{u}
   = \alpha\) and \(\omega^{v} = x\), thus we have \(x^n = \omega^{nv} =
   \omega^u = \alpha \Leftrightarrow nv \equiv u \pmod{q-1}\) from the
   definition of the generator \(\omega\), the last congruence has solution if
   and only if, defining \(d = (n, q-1)\), \(d \mid u\). Therefore we can write
   \(u=da\) for some \(a\) and it follows that \(\alpha^{\frac{q-1}{d}} =
   \omega^{u \frac{q-1}{d}} = \omega^{da \frac{q-1}{d}} = 1\) as wanted. 
\end{proof}

\begin{proposition}
   Let \(k\) be a finite field. The integer multiples of the identity form a
   subfield \(k' \subseteq k\) such that \(k' \simeq \mathbb{F}_p\) for some
   prime \(p\).
\end{proposition}

\begin{proof}
   Let the morphism \(\psi: \mathbb{Z} \to k\) mapping \(n \mapsto n 1_k\).
   Notice that \(\psi(\mathbb{Z}) := k' \subseteq k\) forms an integral domain,
   also, the \(\ker(\psi)\) is non-zero (from the fact that \(k\) is finite) 
   prime ideal (that is \(\ker(\psi) \neq k\) also, given \(a, b \in k\) then
   \(ab \in \ker(\psi) \Rightarrow a \in \ker(\psi)\) or \(b \in \ker(\psi)\)).
   Notice that since \(\ker(\psi)\) is a prime ideal, then we can see that there
   is an isomorphism \(k' \xrightarrow{\sim} \mathbb{F}_p\), which concludes the
   proof.
\end{proof}

\begin{proposition}
   The number of elements in a finite field is a power of a prime.
\end{proposition}

\begin{proof}
   Since there exists an inclusion morphism \(\mathbb{F}_p \hookrightarrow k\),
   we can see \(k\) as an \(\mathbb{F}_p\)-space. Let now
   \(\dim_{\mathbb{F}_p}(k) = n\) and define \(\{\omega_1, \dots, \omega_n\}\)
   as its basis. Then, we can see that every element consists of a linear
   combination of the \(n\) basis vectors with an \(n\)-tuple \((a_i)_{i=1}^n\)
   of elements \(a_i \in \mathbb{F}_p\), thus, since there are \(p\) elements
   for each index \(i\), the number of possible linear combinations is \(p^n\).
\end{proof}

\begin{proposition}
   If \(\mathrm{char}(k) = p\), then, \(\forall \alpha, \beta \in k\) and \(\forall d
   \in \mathbb{Z}_{>0}\) then
   \[
      (\alpha + \beta)^{p^d} = \alpha^{p^d} + \beta^{p^d}.
   \] 
\end{proposition}

\begin{proof}
   Let \(d = 1\) then trivially  \((\alpha + \beta)^p = \alpha^p + \sum_{1
   \leqslant k \leqslant p-1} \binom{p}{k} \alpha^{p-k}\beta^k + \beta^p =
   \alpha^p + \beta^p\). Suppose the proposition holds for \(d > 1\). Then,
   since \((\alpha + \beta)^{p^d} = \alpha^{p^d} + \beta^{p^d}\) then
   \((\alpha + \beta)^{p^{d+1}} = (\alpha^{p^d} + \beta^{p^d})^p
   = \alpha^{p^{d+1}} + \beta^{p^{d+1}}\), which completes the induction.
\end{proof}

\begin{proposition}
   Let \(k\) be a field. We have that \(x^\ell - 1 \mid x^m - 1\), polynomials in
   \(k[x]\), if and only if \(\ell \mid m\).
\end{proposition}

\begin{proof}
   (\(\Rightarrow\)) Suppose \(m = q\ell + r\) and  \(0 \leqslant r < \ell\),
   thus
   \[
      \frac{x^m - 1}{x^\ell - 1} = \frac{x^{q\ell + r} - 1}{x^\ell - 1} 
      = \frac{(x^{q \ell + r} - x^r) + (x^r - 1)}{x^\ell - 1} 
      = x^r\frac{x^{q\ell} - 1}{x^\ell - 1} + \frac{x^r - 1}{x^\ell - 1}
   \] 
   We now wish to analyse the second term vanishes, since \(\frac{x^{q\ell} -
   1}{x^\ell - 1} \in k[x]\) (just a matter of factorization of the numerator).
   Now, since \(r < \ell\) then it follows that \(\frac{x^r - 1}{x^\ell - 1} \in
   k[x] \Leftrightarrow r = 0\) so that \(x^r - 1 = 0\). (\(\Leftarrow\))
   Trivial.
\end{proof}

\begin{proposition}
   Let \(k\) be a finite field of dimension \(n\) over \(\mathbb{F}_p\), so that
   \(|k| = p^n\). The subfields of \(k\) are in a bijection with the divisors of
   \(n\).
\end{proposition}

\begin{proof}
   (\(\Rightarrow\)) Let \(E \subseteq k\) be a subfield and let
   \(\dim_{\mathbb{F}_p}(E) = d\) our initial goal is the following: show that
   \(d \mid n\). Notice that \(|E| = p^d\) and thus \(|E^\ast| = p^d - 1\).
   Then, we have that every element \(\alpha \in E^\ast\) is a root of both
   \(x^{|E^\ast|} - 1 = x^{(p^d - 1)}\) and \(x^{|k|} - 1 = x^{p^n - 1}\) (since
   \(\alpha \in k^\ast\)) and therefore \(x^{p^d - 1}- 1 \mid x^{p^n - 1}\)
   (because all roots of the first are also roots of the second) iff \(p^d - 1
   \mid p^n-1 \Leftrightarrow d \mid n\).

   (\(\Leftarrow\)) Let \(d \mid n\) and define \(E := \{\alpha \in k :
   \alpha^{p^d} = \alpha\} \subseteq k\). We now check that indeed \(E\) is a
   field: 
   \begin{enumerate}[i.]
      \item \((\alpha + \beta)^{p^d} = \alpha^{p^d} + \beta^{p^d} = \alpha +
         \beta\).
      \item \((\alpha\beta)^{p^d} = \alpha\beta\).
      \item \((\alpha^{-1})^{p^d} = \alpha^{-1},\ \alpha \neq 0\).
   \end{enumerate}
   Since from hypothesis \(d \mid n\) then \(p^d - 1 \mid p^n - 1\) and thus
   \(x^{p^d-1}- 1 \mid x^{p^n-1}-1\) and therefore obviously \(x^{p^d} - x \mid
   x^{p^n} - x\) then \(x^{p^d} - x\) has \(p^d\) distinct roots, which implies
   that \(|E| = p^d \Rightarrow \dim_{\mathbb{F}_p}(E) = d\). Notice that the
   bijection can now be established, since subfields of \(k\) with dimension
   \(d\) over \(\mathbb{F}_p\) are equal to \(E\).
\end{proof}



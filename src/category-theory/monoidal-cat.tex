\documentclass[../../deep-dive]{subfiles}

\begin{document}

\section{Monoidal Categories}

\begin{definition}
\label{def:monoidal-category}
A \emph{monoidal category} is a tuple
\((\cat M, \otimes, 1, \alpha, \lambda, \rho)\) consisting of:
\begin{itemize}\setlength\itemsep{0em}
\item A \emph{category} \(\cat M\).

\item A bifunctor \(\otimes: \cat M \times \cat M \to \cat M\)

\item A distinguished object \(1 \in \cat M\) that is \emph{unitary} with
  respect to \(\otimes\), that is:
  \[
  m \otimes 1 = m = 1 \otimes m
  \]
  for any object \(m \in \cat M\).

\item A \emph{natural isomorphism}
  \[
  \alpha: (- \otimes (- \otimes -))
  \overset{\iso}\Longrightarrow ((- \otimes -) \otimes -).
  \]
  called \emph{associator}, in the sense that given any triple of objects \((a,
  b, c)\) of \(\cat M\), the image
  \[
  \begin{tikzcd}
  a \otimes (b \otimes c)
  \ar[r, "{\alpha(a, b, c)}"', "\dis"]
  &(a \otimes b) \otimes c
  \end{tikzcd}
  \]
  is an isomorphism in \(\cat M\).

\item Two \emph{natural isomorphisms}
  \[
  \lambda: (1 \otimes -)
  \overset{\iso}\Longrightarrow (-)
  \quad
  \text{ and }
  \quad
  \rho: (- \otimes 1)
  \overset{\iso}\Longrightarrow (-)
  \]
  called \emph{left and right unitors}, respectively. In other words, given any
  object \(a \in \cat M\) the morphisms \(\lambda a: 1 \otimes a \isoto a\) and
  \(\rho a: a \otimes 1 \isoto a\) are \emph{isomorphisms} in \(\cat M\).
\end{itemize}
This data should satisfy the following two conditions:
\begin{itemize}\setlength\itemsep{0em}
\item (Triangle identity) Given any pair \((a, b)\) of objects in \(\cat M\),
  the diagram
  \[
  \begin{tikzcd}
  a \otimes (1 \otimes b) \ar[rr, "{\alpha(a, 1, b)}"]
  \ar[rd, "\Id_a \otimes \rho b"']
  & &(a \otimes 1) \otimes b \ar[ld, "\lambda a \otimes \Id_b"]
  \\
  &a \otimes b &
  \end{tikzcd}
  \]
  commutes in \(\cat M\).

\item (Pentagon identity) Given any tuple \((a, b, c, d)\) of objects in
  \(\cat M\), the diagram
  \[
  \begin{tikzcd}
  &
  &(a \otimes b) \otimes (c \otimes d)
  \ar[rdd, "{\alpha(a \otimes b, c, d)}"]
  &
  \\
  & & &
  \\
  a \otimes (b \otimes (c \otimes d))
  \ar[dd, "{\Id_a \otimes \alpha(b, c, d)}"']
  \ar[rruu, "{\alpha(a, b, c \otimes d)}"]
  &
  &
  &((a \otimes b) \otimes c) \otimes d
  \\
  & & &
  \\
  % empty
  a \otimes ((b \otimes c) \otimes d)
  \ar[rrr, "{\alpha(a, b \otimes c, d)}"']
  &
  &
  &(a \otimes (b \otimes c)) \otimes d
  \ar[uu, "{\alpha(a, b, c) \otimes \Id_d}"']
  \end{tikzcd}
  \]
  is commutative in \(\cat M\).
\end{itemize}
\end{definition}
The tuple \((\cat M, \otimes, 1, \alpha, \lambda, \rho)\) is said to be a
\emph{strict monoidal category} if the three natural isomorphisms are naturally
isomorphic to the identity---if this is the case, we shall refer to the category
simply by the triple \((\cat M, \otimes, 1)\).

\begin{definition}[Monoidal functor]
\label{def:monoidal-functor}
Let \((\cat M, \otimes, 1, \alpha, \lambda, \rho)\) and \((\cat N,
\widehat\otimes, \widehat 1, \widehat \alpha, \widehat \lambda, \widehat \rho)\)
be two (strict) monoidal categories. We say that a functor \(F: \cat M \to \cat
N\) is a (\emph{strict}) \emph{monoidal functor} if it preserves the actions of
the natural isomorphisms. To put concretely, we have:
\begin{itemize}\setlength\itemsep{0em}
\item The unit of \(\cat M\) is mapped to the unit of \(\cat N\), that is,
  \(F e = \widehat e\).

\item For any \(a \in \cat M\) one has
  \(F (\lambda a) = \widehat \lambda (F a)\) and
  \(F (\rho a) = \widehat \rho(F a)\).

\item For any pair \((a, b)\) of objects in \(\cat M\) there exists an
  isomorphism \(F(a \otimes b) \iso F a \widehat \otimes F b\) in \(\cat
  N\)---in the strict case, the isomorphism is replaced by an equality.

\item For any triple \((a, b, c)\) of objects in \(\cat M\) we have
  \(F \alpha(a, b, c) = \widehat \alpha (F a, F b, F c)\).

\item For every two maps \(f\) and \(g\) in \(\cat M\) there exists an
  isomorphism \(F(f \otimes g) \iso F f \widehat \otimes F g\) in
  \(\cat N\)---in the strict case, the isomorphism is replaced by an equality.
\end{itemize}
\end{definition}

\begin{definition}[Monoidal natural transformation]
\label{def:monoidal-natural-transformation}
Let \((\cat M, \otimes, 1, \alpha, \lambda, \rho)\) and
\((\cat N, \widehat\otimes, \widehat 1, \widehat \alpha, \widehat \lambda,
\widehat \rho)\) be two (strict) monoidal categories, and consider a pair of
parallel (strict) functors \(F, G: \cat M \para \cat N\). A natural
transformation \(\eta: F \nat G\) is said to be \emph{monoidal} if
\(\eta_1 = \widehat 1\), and for any pair of objects \(a, b \in \cat M\) the
diagram
\[
\begin{tikzcd}
F(a \otimes b) \ar[d, "\dis"']
\ar[r, "\eta_{a \otimes b}"]
&G(a \otimes b) \ar[d, "\dis"] \\
F a \widehat\otimes F b \ar[r, "\eta_a \widehat\otimes \eta_b"']
&G a \widehat\otimes G b
\end{tikzcd}
\]
commutes in the monoidal category \(\cat N\).
\end{definition}

\begin{theorem}[Strictification of monoidal categories]
\label{thm:strictification-mon-cat}
Every monoidal category is \emph{monoidally equivalent} to a \emph{strict}
monoidal category.
\end{theorem}

\begin{proof}
Let \((\cat M, \otimes, 1, \alpha, \lambda, \rho)\) be a monoidal
category. We shall construct a strict monoidal category out of \(\cat M\). To
that end, define a category \(\cat N\) where:
\begin{itemize}\setlength\itemsep{0em}
\item The objects of \(\cat N\) are pairs \((F, \eta)\) where \(F\) is an
  \emph{endofunctor} of \(\cat M\) and
  \[
  \eta: F(- \otimes -) \isonat F(-) \otimes (-)
  \]
  is a \emph{natural isomorphism} such that, for any triple \((a, b, c)\) of
  objects of \(\cat M\), the pentagonal diagram
  \[
  \begin{tikzcd}
  &
  &(F(a) \otimes b) \otimes c
  &
  \\
  & & &
  \\
  F(a \otimes b) \otimes c
  \ar[rruu, "\eta_{(a, b)} \otimes \Id_c"]
  &
  &
  & F(a) \otimes (b \otimes c)
  \ar[luu, "{\alpha(F a, b, c)}"']
  \\
  & & &
  \\
  F((a \otimes b) \otimes c)
  \ar[uu, "\eta_{(a \otimes b, c)}"]
  &
  &
  &F(a \otimes (b \otimes c))
  \ar[lll, "{F \alpha(a, b, c)}"]
  \ar[uu, "\eta_{(a, b \otimes c)}"']
  \end{tikzcd}
  \]

\item A morphism \(\varepsilon: (F, \eta) \to (F', \eta')\) is a natural
  transformation \(\varepsilon: F \nat F'\) such that, given any pair \((a, b)\)
  of objects of \(\cat M\), the diagram
  \begin{equation}\label{eq:coherence-morphism-cat-N}
  \begin{tikzcd}
  F(a \otimes b) \ar[d, "\eta_{(a, b)}"']
  \ar[r, "\varepsilon_{a \otimes b}"]
  &F'(a \otimes b) \ar[d, "\eta'_{(a, b)}"]
  \\
  F(a) \otimes b \ar[r, "\varepsilon_a \otimes \Id_b"']
  &F'(a) \otimes b
  \end{tikzcd}
  \end{equation}
  commutes in \(\cat M\). Moreover, we define the composition of morphisms in
  \(\cat N\) to be given by the vertical composition of natural transformations.

\item Define a bifunctor \(\widehat\otimes: \cat N \times \cat N \to \cat N\) as
  \((F, \eta) \widehat\otimes (F', \eta') \coloneq (F F', \widehat\eta)\), where
  \[
  \widehat\eta: F F'(- \otimes -) \nat F F'(-) \otimes (-)
  \]
  is the natural transformation given by the composition
  \[
  \begin{tikzcd}
  F F'(a \otimes b)
  \ar[rr, "F \eta'_{(a, b)}"']
  \ar[rrrr, bend left, "\widehat\eta_{(a, b)}"]
  &
  &F (F'(a) \otimes b)
  \ar[rr, "\eta_{(F' a, b)}"']
  &
  &F F'(a) \otimes b
  \end{tikzcd}
  \]
  for any pair of objects \((a, b)\) of \(\cat M\).
\end{itemize}
From this construction we find that the triple
\((\cat N, \widehat\otimes, (\Id_{\cat M}, I))\)---where the natural
transformation \(I: (- \otimes -) \isonat (- \otimes -)\) is the identity
morphism \(I_{(a, b)} \coloneq \Id_{a \otimes b}\) in \(\cat M\) for any two
\(a, b \in \cat M\)---is a \emph{strict monoidal category}, since:
\begin{itemize}\setlength\itemsep{0em}
\item The bifunctor \(\widehat\otimes\) satisfies \emph{equality} for both left
  and right unitors: given an object \((F, \eta) \in \cat N\), consider any two
  objects \(a, b \in \cat N\) then by the definition of \((F, \eta)
  \widehat\otimes (\Id_{\cat M}, I) = (F, \widehat\eta)\) and \((\Id_{\cat M},
  I) \widehat\otimes (F, \widehat\eta')\) one has
  \[
  \begin{tikzcd}
  {F(a \otimes b)} \ar[rr, "F \Id_{a \otimes b} = \Id_{F(a \otimes b)}"']
  \ar[rrrr, bend left, "\widehat\eta_{(a, b)}"]
  &&{F(a \otimes b)} \ar[rr, "\eta_{(a, b)}"']
  &&{F(a) \otimes b}
  \end{tikzcd}
  \]
  \[
  \begin{tikzcd}
  {F(a \otimes b)} \ar[rr, "\Id_{\cat M} \eta_{(a, b)} = \eta_{(a, b)}"]
  \ar[rrrr, bend right, "\widehat\eta'_{(a, b)}"']
  &&{F(a) \otimes b} \ar[rr, "I_{(Fa, b)} = \Id_{F(a) \otimes b}"]
  &&{F(a) \otimes b}
  \end{tikzcd}
  \]
  therefore \(\widehat\eta = \eta = \widehat\eta'\). Moreover, this also shows
  that the triangle identity is satisfied.

\item Associativity follows from the associativity of morphisms and functors.
\end{itemize}

We now prove that \(\cat M\) and \(\cat N\) are equivalent categories. In order
to do that, define a functor \(E: \cat M \to \cat N\) mapping objects
\(a \mapsto (a \otimes (-), \alpha(a, -, -))\) and morphisms
\(f \mapsto f \otimes (-)\). We now show that \(E\) is an equivalence of
categories:
\begin{itemize}\setlength\itemsep{0em}
\item (Essentially surjective) Notice that, given any object
  \((F, \eta) \in \cat N\), we can define a morphism
  \[
  \varepsilon: (F 1 \otimes (-), \alpha(F1, -, -))
  \longrightarrow (F, \eta)
  \]
  by constructing a natural transformation
  \(\varepsilon: F 1 \otimes (-) \nat F\) where
  \(\varepsilon_a \coloneq \lambda_{a} \eta_{(1, a)}^{-1}\), which is an
  isomorphism \(F(1) \otimes a \iso F a\) for any \(a \in \cat M\)---showing
  that \(\varepsilon\) is a natural isomorphism, defining an isomorphism
  \(E(F 1) \iso (F, \eta)\).

\item (Full) Let \(a, b \in \cat M\) be any two objects, and
  \(\varepsilon: E a \to E b\) be any morphism of \(\cat N\)---that is, a
  natural transformation \(\varepsilon: (a \otimes -) \nat (b \otimes -)\)
  satisfying the coherence diagram \cref{eq:coherence-morphism-cat-N}. Define
  \(f: a \to b\) to be the morphism in \(\cat M\) given by
  \(f \coloneq (\lambda b) \circ \varepsilon_1 \circ (\lambda^{-1} a)\). By the
  definition of \(E\), one has \(E f = f \otimes (-)\)---we wish to show that
  this agrees with \(\varepsilon\). Given any \(c \in \cat M\) the diagram
  \[
  \begin{tikzcd}
  a \otimes c \ar[rr, "{\Id_a \otimes \rho(c)^{-1}}"]
  \ar[d, "\varepsilon_c"']
  && a \otimes (1  \otimes c) \ar[rr, "{\alpha(a, 1, c)}"]
  \ar[d, "\varepsilon_{e \otimes c}"']
  \ar[rrrr, bend left=30, "{\Id_a \otimes \rho c}"]
  && (a \otimes 1) \otimes c \ar[rr, "{\lambda a \otimes \Id_c}"]
  \ar[d, "\varepsilon_1 \otimes \Id_c"]
  && a \otimes c \ar[d, "f \otimes \Id_c"]
  \\
  b \otimes c \ar[rr, "{\Id_b \otimes \rho(c)^{-1}}"']
  && b \otimes (1 \otimes c) \ar[rr, "{\alpha(b, 1, c)}"']
  \ar[rrrr, bend right=30, "{\Id_b \otimes \rho c}"']
  && (b \otimes 1) \otimes c \ar[rr, "{\lambda b \otimes \Id_c}"']
  && b \otimes c
  \end{tikzcd}
  \]
  is commutative in \(\cat M\): the left and center squares commute by the
  naturallity of \(\varepsilon\), the up and down wings commute by the triangle
  identities, the right square commutes by the definition of \(f\). It follows
  from commutativity that \(\varepsilon_c = f \otimes \Id_c\), therefore
  \(E f = \varepsilon\).

\item (Faithful) Let \(f\) and \(g\) be morphisms of \(\cat M\) such that \(E f
  = E g\), so that in particular \(f \otimes \Id_1 = g \otimes \Id_1\)---hence
  \(f = g\), proving injectivity on the morphism collections of \(\cat M\) and
  \(\cat N\).

\item (Monoidal) First, it is clear that
  \(E 1 = (1 \otimes (-), \alpha(1, -, -)) \iso (\Id_{\cat M}, I)\). Moreover,
  for any pair of morphisms \(f\) and \(g\) of \(\cat M\) one has
  \[
  E(f \otimes g) = (f \otimes g) \otimes (-)
  \iso f \otimes (g \otimes -)
  = E f \widehat\otimes E g.
  \]
  Given any two \(a, b \in \cat M\), from definition:
  \[
  E (a \otimes b) = ((a \otimes b) \otimes (-), \alpha(a \otimes b, -, -))
  \iso (a \otimes (b \otimes -), \alpha(a \otimes b, -, -)),
  \]
  also we know that if
  \[
  (a \otimes (b \otimes -), \beta)
  \coloneq (a, \alpha(a, -, -)) \widehat\otimes (b, \alpha(b, -, -))
  = E a \widehat\otimes E b,
  \]
  then \(\beta\) is defined so that the up wing of the diagram
  \[
  \begin{tikzcd}
  a \otimes (b \otimes (c \otimes d))
  \ar[rr, "{a \otimes \alpha(b, c, d)}"']
  \ar[rrrr, bend left, "\beta_{(c, d)}"]
  \ar[d, "\alpha{(a, b, c \otimes d)}"']
  &&a \otimes ((b \otimes c) \otimes d)
  \ar[rr, "{\alpha(a, b \otimes c, d)}"']
  &&(a \otimes (b \otimes c)) \otimes d
  \ar[d, "\alpha{(a, b, c)} \otimes \Id_d"]
  \\
  (a \otimes b) \otimes (c \otimes d)
  \ar[rrrr, "\alpha{(a \otimes b, c, d)}"']
  && &&((a \otimes b) \otimes c) \otimes d
  \end{tikzcd}
  \]
  commutes in \(\cat M\) for any two \(c, d \in \cat M\)---the square commutes
  by the pentagon identity. This shows that
  \[
  \alpha(a, b, -): (a \otimes (b \otimes -), \beta)
  \isoto (a \otimes (b \otimes -), \alpha(a \otimes b, -, -))
  \]
  is an isomorphsim in \(\cat N\). Therefore
  \(E(a \otimes b) \iso E a \widehat \otimes E b\). For the left and right
  unitor isomorphisms \(E (\lambda a) \iso \widehat \lambda(E a)\) and
  \(E(\rho a) \iso \widehat \rho(E a)\) we shall simply argue that they both
  come straight from the triangle identity of \(\cat M\). Similarly,
  \[
  E(\alpha(a, b, c)) \iso \widehat \alpha(E a, E b, E c)
  \]
  works via a reduction to the pentagon identity in \(\cat M\).
\end{itemize}
This proves that \(E: \cat M \to \cat N\) is indeed a monoidal equivalence of
categories.
\end{proof}

From now on, in view of the equivalence given by
\cref{thm:strictification-mon-cat}, whenever possible we shall adress any
monoidal category as a strict monoidal category.

\begin{definition}[(Co)monoid objects in \(\Mon\)]
\label{def:(co)monoids}
Let \((\cat M, \otimes, 1, \alpha, \lambda, \rho)\) be a monoidal category. We
define the following objects:
\begin{enumerate}[(a)]\setlength\itemsep{0em}
\item A \emph{monoid} in \(\cat M\) is a triple \((m, \mu, \theta)\)---where
  \(m\) is an object of \(\cat M\), a bifunctor \(\mu: m \otimes m \to m\)
  refered to as a \emph{multiplication}, and a functor \(\theta: 1 \to m\)
  called \emph{unit}---such that both diagrams
  \[
  \begin{tikzcd}
  m \otimes (m \otimes m) \ar[d, "\Id_m \otimes \mu"']
  \ar[r, "{\alpha(m, m, m)}"]
  &(m \otimes m) \otimes m
  \ar[r, "\mu \otimes \Id_m"]
  &m \otimes m \ar[d, "\mu"] \\
  m \otimes m \ar[rr, "\mu"']
  &&m
  \end{tikzcd}
  \]
  \[
  \begin{tikzcd}
  1 \otimes m \ar[r, "\theta \otimes \Id_m"]
  \ar[rd, "\rho"']
  & m \otimes m \ar[d, "\mu"]
  &m \otimes 1
  \ar[l, "\Id_m \otimes \theta"']
  \ar[ld, "\lambda"] \\
  &m &
  \end{tikzcd}
  \]
  commute in \(\cat M\). A morphism of monoids
  \(\phi: (m, \mu, \theta) \to (m', \mu', \theta')\) is a morphism
  \(\phi: m \to m'\) in \(\cat M\) satisfying
  \(\phi \mu = \mu'(\phi \otimes \phi)\), and \(\phi \theta = \theta'\). We then
  define the subcategory \(\Mon_{\cat M}\) of \(\cat M\) composed of monoidal
  objects in \(\cat M\).

\item A \emph{comonoid} in \(\cat M\) is a triple \((c, \kappa, \sigma)\)
  \(c\) is an object of \(\cat M\), a bifunctor \(\kappa: c \to c \otimes c\)
  refered to as a \emph{comultiplication}, and a functor \(\sigma: c \to 1\)
  called \emph{counit}---such that both diagrams
  \[
  \begin{tikzcd}
  c \otimes (c \otimes c)
  &(c \otimes c) \otimes c
  \ar[l, "{\alpha(c, c, c)^{-1}}"']
  &c \otimes c
  \ar[l, "\kappa \otimes \Id_c"']
  \\
  c \otimes c
  \ar[u, "\Id_c \otimes \kappa"]
  &&c \ar[ll, "\kappa"] \ar[u, "\kappa"']
  \end{tikzcd}
  \]
  \[
  \begin{tikzcd}
  1 \otimes c
  & c \otimes c
  \ar[l, "\sigma \otimes \Id_c"']
  \ar[r, "\Id_c \otimes \sigma"]
  &c \otimes 1
  \\
  &c \ar[u, "\kappa"'] \ar[lu, "\rho^{-1}"] \ar[ru, "\lambda^{-1}"'] &
  \end{tikzcd}
  \]
  commute in \(\cat M\). A morphism of comonoids
  \(\psi: (c, \kappa, \sigma) \to (c', \kappa', \sigma')\) is a morphism
  \(\psi: c \to c'\) in \(\cat M\) satisfying
  \(\kappa' \psi = (\psi \otimes \psi) \kappa\), and \(\sigma = \sigma'
  \psi\). We then define the subcategory \(\coMon_{\cat M}\) of \(\cat M\)
  composed of comonoidal objects in \(\cat M\).
\end{enumerate}
\end{definition}

\begin{definition}[Monoid actions]
\label{def:monoid-actions}
Let \((\cat M, \otimes, 1)\) be a monoidal category, and
\((m, \mu, \theta) \in \Mon_{\cat M}\). A \emph{left-action} of the monoid
\((m, \mu, \theta)\) on an object \(a \in \cat M\) is a bifunctor
\(\sigma: m \otimes a \to a\) such that
\[
\begin{tikzcd}
m \otimes (m \otimes a)
\ar[r, "{\alpha(m, m, a)}"]
\ar[d, "\Id_m \otimes \sigma"']
&(m \otimes m) \otimes a
\ar[r, "\mu \otimes \Id_a"]
&m \otimes a
\ar[d, "\sigma"']
&1 \otimes a
\ar[l, "\theta \otimes \Id_a"']
\ar[ld, "\lambda"]
\\
m \otimes a \ar[rr, "\sigma"']
&
&a
&
\end{tikzcd}
\]
commutes in \(\cat M\). Right-actions are defined analogously.

Given any two left-actions \(\sigma: m \otimes a \to a\) and
\(\lambda: m \otimes b \to b\), we define a \emph{morphism of left-actions}
\(\phi: \sigma \to \lambda\) to be a morphism \(\phi: a \to b\) in \(\cat M\)
such that
\[
\begin{tikzcd}
m \otimes a
\ar[d, "\sigma"']
\ar[rr, "\Id_m \otimes \phi"]
&&m \otimes b \ar[d, "\lambda"]
\\
a \ar[rr, "\phi"']
&&b
\end{tikzcd}
\]
is a commutative diagram in \(\cat M\). With these notions we are able to define
two categories \(\rActMon_{(\cat M, m)}\) and \(\lActMon_{(\cat M, m)}\),
composed of right and left actions of \(m\) on objects of \(\cat M\),
respectively, and morphisms between them.
\end{definition}

\section{Braided \& Symmetric Monoidal Categories}

\begin{definition}[Braiding]
\label{def:braiding}
Given a monoidal category \((\cat M, \otimes, 1, \alpha, \lambda, \rho)\), we
define a \emph{braiding} of \(\cat M\) to be a natural isomorphism\footnote{That
  is, for any two \(a, b \in \cat M\) one has an isomorphism
  \(\gamma_{(a, b)}: a \otimes b \isoto b \otimes a\).}
\[
\gamma: (- \otimes -') \isonat (-' \otimes -),
\]
that is coherent with associativity and unitors of \(\cat M\), in the sense that
the diagrams
\[
\begin{tikzcd}
(a \otimes b) \otimes c
\ar[rr, "\gamma_{(a \otimes b, c)}"]
\ar[dd, "{\alpha^{-1}(a, b, c)}"']
&&c \otimes (a \otimes b)
\ar[dd, "{\alpha(c, a, b)}"]
\\
&&
\\
a \otimes (b \otimes c)
\ar[dd, "\Id_a \otimes \gamma_{(b, c)}"']
&&(c \otimes a) \otimes b
\ar[dd, "\gamma_{(c, a)} \otimes \Id_b"]
\\
&&
\\
a \otimes (c \otimes b)
\ar[rr, "\alpha{(a, c, b)}"']
&&(a \otimes c) \otimes b
\end{tikzcd}
\qquad
\qquad
\begin{tikzcd}
a \otimes (b \otimes c)
\ar[rr, "\gamma_{(a, b \otimes c)}"]
\ar[dd, "{\alpha(a, b, c)}"']
&&(b \otimes c) \otimes a
\ar[dd, "{\alpha(b, c, a)^{-1}}"]
\\
&&
\\
(a \otimes b) \otimes c
\ar[dd, "\gamma_{(a, b)} \otimes \Id_c"']
&&b \otimes (c \otimes a)
\ar[dd, "\gamma_{(c, a)} \otimes \Id_b"]
\\
&&
\\
(b \otimes a) \otimes c
\ar[rr, "\alpha{(b, a, c)}"']
&&b \otimes (a \otimes c)
\end{tikzcd}
\]
\[
\begin{tikzcd}
a \otimes 1 \ar[rr, "\gamma_{(a, 1)}"]
\ar[rd, "\lambda"']
&&1 \otimes a \ar[ld, "\rho"]
\\
&a &
\end{tikzcd}
\]
should commute for all triples \((a, b, c)\) of objects of \(\cat M\).
\end{definition}

\begin{definition}[Braided monoidal category]
\label{def:braided-monoidal-category}
A monoidal category \((\cat M, \otimes, 1)\) is said to be \emph{braided} if it
is endowed with a braiding \(\gamma\). We shall denote this data shortly by
\((\cat M, \gamma)\).
\end{definition}

\begin{corollary}
\label{cor:braiding-morphisms}
For any given pair of morphisms \(f: a \to b\) and \(g: c \to d\) in a braided
monoidal category \(\cat M\), we have
\[
(g \otimes f) \gamma_{(a, c)} \iso \gamma_{(b, d)} (f \otimes g).
\]
\end{corollary}

\begin{definition}[Braided monoidal functor]
\label{def:braided-monoidal-functor}
A monoidal functor \(F: (\cat A, \gamma) \to (\cat B, \widehat \gamma)\) between
braided monoidal categories is said to be a \emph{braided monoidal functor} if
for every pair of objects \(a, b \in \cat A\) the braiding coherence square
\[
\begin{tikzcd}
F a \otimes F b \ar[r, "\widehat\gamma"] \ar[d, "\dis"']
&F b \otimes F a \ar[d, "\dis"] \\
F (a \otimes b) \ar[r, "F \gamma"'] &F (b \otimes a)
\end{tikzcd}
\]
commutes in \(\cat B\).
\end{definition}

\begin{corollary}
\label{cor:composition-of-braided-monoidal-functors}
The composition of braided monoidal functors is again a braided monoidal
functor.
\end{corollary}

\begin{proof}
Indeed, if
\((\cat A, \gamma) \xrightarrow F (\cat B, \widehat \gamma) \xrightarrow G (\cat
C, \widetilde \gamma)\) are braided monoidal functors, then for every pair
\(a, b \in \cat A\) one has the following commutative diagram in \(\cat C\):
\[
\begin{tikzcd}
G F a \otimes G F b
\ar[r, "\widetilde \gamma"]
\ar[d, "\dis"']
&GF b \otimes GF a
\ar[d, "\dis"]
\\
G(F a \otimes F b)
\ar[r, "G \widehat \gamma"]
\ar[d, "\dis"']
&G (F b \otimes F a)
\ar[d, "\dis"]
\\
GF (a \otimes b)
\ar[r, "G F \gamma"']
&GF (b \otimes a)
\end{tikzcd}
\]
\end{proof}

\begin{definition}
\label{def:category-of-braided-monoidal}
We define a category \(\BrMonCat\) composed of braided monoidal categories and
braided monoidal functors between them.
\end{definition}

\begin{definition}[Symmetric monoidal category]
\label{def:symmetric-monoidal-category}
A braided monoidal category \((\cat M, \gamma)\) is said to be \emph{symmetric}
if for any two \(a, b \in \cat M\) the triangle
\[
\begin{tikzcd}
a \otimes b \ar[rr, "\Id_{a \otimes b}"]
& &a \otimes b \ar[ld, "\gamma_{(a, b)}"] \\
&b \otimes a \ar[ru, "\gamma_{(b, a)}"] &
\end{tikzcd}
\]
commutes in \(\cat M\). A morphism between symmetric monoidal categories is a
braided monoidal functor between them.
\end{definition}

\end{document}
%%% Local Variables:
%%% mode: latex
%%% TeX-master: "../../deep-dive"
%%% End:

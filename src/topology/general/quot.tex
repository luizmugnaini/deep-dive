\section{Quotient Space}

The motivation for the construction of the quotient topology is the study of
surjective set-functions \(\pi: X \epi S\) between topological spaces \(X\) and
sets \(S\), which induce an equivalence relation on the initial topological
space by means of arranging the points of \(X\) in classes where \(x \sim y\)
for \(x, y \in X\) if and only if \(\pi(x) = \pi(y)\), that is, \(x\) and \(y\)
are common points of a fiber \(\pi^{-1}(s)\) for some \(s \in S\).

The geometrically appealing version of such construction would be the idea of
gluing every element \(x \in \pi^{-1}(s)\) into a unique point --- this way we
loose some initial information about the topological space \(X\). However, we
would like to make this gluing process compatible with the theory of topology so
far constructed. To achieve that, one may make the point that we should make
\(S\) into a topological space for which its topology makes \(\pi\) a continuous
map. The first idea that may come to mind is that we can simply force the
continuity of \(\pi\) by defining a topology \(\mathcal T\) on \(S\) so that \(U
\subseteq S\) is an open set if and only if \(\pi^{-1}(U)\) is open --- which is
exactly the definition of a continuous map. Lets give this a formal definition.

\begin{definition}\label{def:quotient-topology}
Let \(X\) be a topological space and \(S\) be any set. Let also \(\pi: X \epi S\)
be a set-function. The quotient topology on \(S\) induced by the map \(\pi\) is
the finest topology such that \(\pi\) is a continuous map.
\end{definition}

\begin{proposition}
The quotient topology is a topology.
\end{proposition}

\begin{proof}
Let \(X /{\sim}\) be a topological space with the quotient topology induced by
the map \(\pi: X \epi X /{\sim}\). Consider an arbitrary collection
\({\{U_{j}\}}_j\) of open sets of \(X /{\sim}\). Notice that since
\(\pi^{-1}(\bigcup_{j} U_{j}) = \bigcup_j \pi^{-1}(U_j)\) then since
\(\pi^{-1}(U_j)\) is open for all index \(j\), we conclude that \(\bigcup_j
\pi^{-1}(U_{j}) \subseteq X\) is open, hence the set \(\bigcup_j U_j \subseteq X
/{\sim}\) is necessarily open --- since \(\pi\) is continuous. Let now \(A, B
\subseteq X /{\sim}\) be any open sets, then \(\pi^{-1}(A \cap B) = \pi^{-1}(A)
\cap \pi^{-1}(B) \subseteq X\), which is open and therefore \(A \cap B\) is open
in \(X /{\sim}\). We conclude that the quotient topology indeed satisfies the
properties of a topology.
\end{proof}

In other terms, let \(T_S\) be the collection of all topologies on \(S\) such
that \(\pi\) is a continuous map. The above definition simply says that the
quotient topology \(\mathcal T\) on the set \(S\) is the intersection \(\mathcal
T = \bigcap_{T \in T_S}T\). Even better than that definition is the fact that we
can determine the quotient topology by the following universal property.

\begin{theorem}[Universal property of the quotient topology]
\label{thm:universal-property-quotient-topology}
Let \(X\) be a topological space and \(S\) be any set, together with a
surjective set-function \(\pi: X \epi S\). The quotient topology \(\mathcal T\)
on \(S\) induced by \(\pi\) is such that, for all topological spaces \(Z\) and
every morphism \(g: X \to Z\) for which all \(x, y \in X\) such that \(\pi(x) =
\pi(y)\) then \(g(x) = g(y)\) --- i.e. \(g\) is constant on the fibers of
\(\pi\) ---, there exists a unique morphism \(f: (S, \mathcal T) \to Z\) such
that the diagram
\[
  \begin{tikzcd}
    X \ar[r, "g"] \ar[d, two heads, swap, "\pi"] &Z\\
    S \ar[ur, swap, bend right, dashed, "f"]
  \end{tikzcd}
\]
commutes in \(\Top\). Moreover, if \(\mathcal T'\) is a topology on \(S\) such
that the diagram commutes, then necessarily \(\mathcal T' = \mathcal T\).
\end{theorem}

\begin{proof}
Since \(\pi\) is surjective, we can completely define a map \(f: S \to Z\)
sending \(s \mapsto g(x)\) such that \(x \in \pi^{-1}(s)\), which is well
defined because, for all \(x, y \in \pi^{-1}(s)\), we have \(g(x) = g(y)\), so
that the image of each \(s \in S\) under the map \(f\) is uniquely defined. We
now show that \(f\) is, in fact, continuous --- and hence a morphism. Let \(U
\subseteq Z\) be any open set of \(Z\), then, since \(g^{-1}(U)\) is open and
\((f \pi)^{-1}(U) = \pi^{-1} f^{-1}(U) = g^{-1}(U)\), \(f^{-1}(U)\) cannot be a
closed set --- in fact, it needs to be an open set, because
\(\pi^{-1}(f^{-1}(U)) \subseteq X\) must be open, since \(\pi\) is continuous.
It follows that \(f\) is continuous and thus the said morphism indeed exists.

For the uniqueness, let \(f\) and \(f'\) be two morphisms such that \(f\pi = g\)
and \(f'\pi = g\). Let \(s \in S\) be any point. Since \(\pi\) is surjective,
there exists \(x \in X\) for which \(x \in \pi^{-1}(s)\), therefore, \(f \pi(x)
= f(s) = f'\pi(x) = f'(s)\) for every element of their domain --- hence \(f =
f'\).

Suppose now that both \((S, \mathcal T)\) and \((S, \mathcal T')\) satisfy the
universal property, that is, the following diagrams commutes for unique
morphisms \(f\) and \(h\)
\[
  \begin{tikzcd}
    &Z & \\
    (S, \mathcal T) \ar[ur, dashed, bend left, "f"]
    &X \ar[l, two heads, "\pi"] \ar[r, swap, two heads, "\pi"] \ar[u, "g"]
    &(S, \mathcal T') \ar[ul, dashed, bend right, swap, "f'"]
  \end{tikzcd}
\]
Since the diagram commutes for all \(Z\), let \(Z = (S, \mathcal T)\), and
consider the map \(f' = \Id': (S, \mathcal T') \to (S, \mathcal T)\). Then,
given any \(U \in \mathcal T\) we find that since \(\Id'\) is continuous that
\(\Id'^{-1}(U) = U \subseteq (S, \mathcal T')\) is open, hence \(\mathcal T
\subseteq \mathcal T'\). Analogously, let \(Z = (S, \mathcal T')\) and consider
\(f = \Id: (S, \mathcal T) \to (S, \mathcal T')\). Let \(U' \in \mathcal T'\)
then from the continuity of \(\Id\) we find that \(\Id^{-1}(U') = U' \subseteq
(S, \mathcal T)\) is open, therefore \(\mathcal T' \subseteq \mathcal T\). Thus
indeed \(\mathcal T = \mathcal T'\) as wanted.
\end{proof}

To ease the way in which we refer to quotients and surjective morphisms that
induce quotients between topological spaces, we define the following terminology.

\begin{definition}[Quotient morphism]
\label{def:quotient-morphism}
A surjective morphism \(\pi: X \to Y\) of topological spaces \(X\) and \(Y\) is
said to be a quotient morphism if \(\pi\) induces the universal property of
quotients --- in other words, open sets of \(Y\) are exactly those that have
open preimage on \(\pi\), that is, \(V \subseteq Y\) is open if and only if
\(\pi^{-1}(V) \subseteq X\) is open.
\end{definition}

\subsection{Some Examples And Applications}

Many important spaces can be obtained with the inclusion of the quotient
topology to our toolkit, I'll now briefly discuss some of those, which will most
probably come up further into this text.

\begin{example}[Projective space]\label{exp:real-projective-space}
Let \(\sim\) be the equivalence relation for which \(x \sim y\) if and only if
\(x = \gamma y\), where \(x, y \in \R^{n+1} \ \{0\}\) and \(\gamma \in \R\). We
define the \(n\)-dimensional real projective space as the quotient \((\R^{n+1}
\setminus \{0\})/{\sim}\), which is denoted by \(\R \Proj^{n}\).
\end{example}

\begin{example}[Cone]\label{exp:cone}
Let \(X\) be any topological space and \(I\) be the standard interval. The
topological space \(X \times I\) is known as the cylinder on \(X\). Via a
quotienting operation, we can collapse regions of this cylinder. For instance,
we can create a cone by collapsing one of the sides of the cylinder, such as
\((X \times I) / (X \times \{1\})\). Such object is denoted \(\Cone X\), the
cone on \(X\).
\end{example}

\begin{example}
\label{exp:wedge-sum-space}
Let \(J\) be an indexing set and \(\{X_j\}_{j \in J}\) be a collection of
non-empty topological spaces. For each \(j \in J\), choose any \(p_j \in X_j\)
as a base point. We define the wedge sum of the collection \(\{X_{j}\}_{j \in
J}\) with respect to the base points \(\{p_{j}\}_{j \in J}\) as the topological
space
\[
  \bigvee_{j \in J} X_j = \coprod_{j \in J} X_j / \{p_{j}\}_{j \in J}
\]
\end{example}

An interesting fact about wedge sums of topological spaces preserve the
Hausdorff property if every component is Hausdorff.

\begin{proposition}
\label{prop:hausdorff-wedge-sum}
Let \(\{X_{j}\}_{j \in J}\) be an indexed collection of Hausdorff topological
spaces. Then the wedge sum \(\bigvee_{j \in J} X_j\) with respect to any choice
of base points is Hausdorff.
\end{proposition}

\begin{proof}
Let \(\{p_j \in X_j\}_{j \in J}\) be any choice of base points for the given
collection. Let \(x, y \in \bigvee_{j \in J} X_j\) be any distinct points in the
wedge sum space. If \(x, y \in X_j\) for some \(j \in J\), then it is clear that
there exists non-intersecting neighbourhoods of \(x\) and \(y\) on \(X_j\) ---
of which we can take their intersection with the disjoint union \(\coprod_{j \in
J} X_j\) and the proposition will hold for \(\bigvee_{j \in J} X_j\). On the
other hand, if \(i, j \in J\) are distinct indices and \(x \in X_i\) while \(y
\in X_j\), then there exists \(U_x \subseteq X_i\) and \(U_y \subseteq X_j\)
neighbours of \(x\) and \(y\), respectively, such that \(U_x \cap X_j =
\emptyset\) and \(U_y \cap X_i = \emptyset\), which in particular imply in \(U_x
\cap U_y= \emptyset\). For our end, we just need to consider the neighbourhoods
\(U_x' = U_x \cap \coprod_{j \in J} X_j\) and \(U_y' = U_y \cap \coprod_{j \in J}
X_j\) so that \(U_x' = U_y'\).
\end{proof}

\begin{proposition}
\label{prop:quot-second-count-locall-euclidean}
Let \(X\) be a second countable space and \(M = X / {\sim}\) be a quotient. If
\(M\) is locally Euclidean, then \(M\) is second countable.
\end{proposition}

\begin{proof}
Let \(\pi: X \epi M\) be the quotienting map that induces the equivalence
relation \(\sim\) in \(X\). If we assume that \(M\) is locally euclidean, we can
let \(\mathcal{C}\) be a cover of \(M\) composed of coordinate balls. Since
\(\pi\) is surjective, \(\mathcal H \coloneq \{\pi^{-1}(U): U \in
\mathcal{C}\}\) is a cover for the space \(X\). Moreover, since \(X\) is second
countable, any cover of \(X\) contains a countable subcover --- in particular,
let \(\mathcal U \subseteq \mathcal H\) be a countable subcover. Define the
countable set \(\mathcal{C}' = \{U \in \mathcal{C} : \pi^{-1}(U) \in
\mathcal{U}\}\). Since \(\mathcal{U}\) covers \(X\) and \(\pi\) is surjective,
it follows that \(\mathcal{C}' \subseteq \mathcal{C}\) is a countable subcover
of \(M\) composed of coordinate balls --- that is, \(M\) is Lindelöf. Better
than that, since coordinate balls are second countable (see
\cref{prop:coordinate-ball-second-countable}) we can apply
\cref{cor:second-countable-out-of-cover} to see that \(M\) is second
countable.
\end{proof}

\begin{corollary}[Manifold from a quotient]
\label{cor:manifold-from-quotient}
In the context of the preceeding proposition, if \(M\) is both locally Euclidean
and Hausdorff, then \(M\) is a topological manifold.
\end{corollary}

\begin{proposition}[Hausdorff from open quotientings]
\label{prop:open-quotient-hausdorff}
Let \(\pi: X \epi Y\) be a surjective morphism of topological spaces \(X\) and
\(Y\). Then \(Y\) is Hausdorff if and only if the collection of pairs of points
with common fiber, \(C \coloneqq \{(p, q) \in X \times X : \pi(p) = \pi(q)\}\),
is closed in \(X \times X\).
\end{proposition}

\begin{proof}
Let \(Y\) be Hausdorff, then, given any \((p, q) \in X \setminus C\), there are
neighbourhoods \(V_p, V_q \subseteq Y\) of \(\pi(p)\) and \(\pi(q)\),
respectively, such that \(V_p \cap V_q = \emptyset\). Since these neighbourhoods
are disjoint, then in particular the collection fibers \(\pi^{-1}(V_p) \times
\pi^{-1}(V_q)\) is contained in \(X \times X \setminus C\), that is, \(X \times
X \setminus C\) is open --- hence \(C\) is closed.

Let \(C\) be closed, then given any distinct points \(a, b \in Y\), the
surjectivity of \(\pi\) implies that there exists \(p, q \in X\) such that
\(\pi(p) = a\) and \(\pi(q) = b\) --- in particular \((p, q) \in X \times X
\setminus C\) and since \(C\) is closed, there exists a neighbourhood \(U_p
\times U_q \subseteq X \times X\) of \((p, q)\) such that \(U_p \times U_q
\subseteq X \times X \setminus C\), that is, \(\pi(U_p), \pi(U_q) \subseteq Y\)
are non intersecting open sets (from the fact that \(\pi\) is open) that are
neighbourhoods of \(a\) and \(b\), respectively --- thus \(Y\) is Hausdorff.
\end{proof}

\subsection{Quotient Morphisms In More Depth}

So far we've been studying the construction of quotients out of surjective
set-functions, but what about being able to classifying a surjective morphisms
between given topological spaces as inducing the universal property of the
quotient space? This will be our goal with this subsection --- identifying
quotienting morphisms. For that end, we shall profit from the main idea behind
quotients: fibers. For that, we define a set given by fibers of \(f\) as being
saturated.

\begin{definition}[Saturated set]
\label{def:saturated-fiber}
Let \(f: X \to Y\) be a set-function. We say that a set \(U \subseteq X\) is
saturated with respect to \(f\) if there exists \(V \subseteq Y\) such that
\(U = f^{-1}(V)\).
\end{definition}

\begin{proposition}[Equivalences for saturated sets]
\label{prop:equivalences-saturated-fiber}
Let \(f: X \to Y\) be a set-function and \(U \subseteq X\) be any subset. The
following propositions are equivalent
\begin{enumerate}[(a)]\setlength\itemsep{0em}
\item The set \(U\) is saturated with respect to \(f\).
\item \(U = f^{-1}(f(U))\).
\item Let \(p \in U\) be any point, \(U\) contains every element \(x \in X\)
  with common fiber to \(p\) --- that is, \(f(x) = f(p)\) implies \(x \in U\).
\end{enumerate}
\end{proposition}

\begin{proof}
(c) \(\Rightarrow\) (b): Suppose \(U\) satisfies proposition (c), it is clear
that \(U \subseteq f^{-1}(f(U))\), on the other hand, given \(x \in
f^{-1}(f(U))\), it follows that \(x\) has a common fiber with some point of
\(U\), which implies that \(x \in U\). (b) \(\Rightarrow\) (a): Trivial from the
definition. (a) \(\Rightarrow\) (c): Let \(V \subseteq Y\) be such that \(U =
f^{-1}(V)\), then, given any \(p \in f^{-1}(V)\), it is clear that \(f(p) \in
V\), hence every point \(x \in X\) such that \(f(x) = f(p) \in V\) then \(x \in
U\), which finishes the equivalence chain.
\end{proof}

\begin{proposition}[Classification of surjective morphisms]
\label{prop:surjective-saturated-is-quotient}
Let \(\pi: X \epi Y\) be a surjective morphism of topological spaces. The map
\(\pi\) is a quotient morphism --- that is, induces the universal property of
quotients for \(X\) and \(Y\) --- if and only if every saturated open (or
closed) set of \(X\) has an open (or closed) image in \(Y\).
\end{proposition}

\begin{proof}
Let \(\pi\) be any surjective morphism taking saturated open sets to open
images. Let \(V \subseteq Y\) be an open set. Since \(\pi\) is surjective and
continuous, then \(\pi^{-1}(V) \subseteq X\) is open. On the other hand, let \(V
\subseteq Y\) be any set of \(Y\) (not necessarily open), such that
\(\pi^{-1}(V) \coloneq U \subseteq X\) is open. This implies directly that \(U\)
is saturated with respect to \(\pi\) and from our initial hypothesis, \(\pi(U) =
V \subseteq Y\) is open. Thus \(\pi\) is a quotient morphism.

For the contrary, let \(\pi\) be a quotient morphism. Then, given any \(U
\subseteq X\) open set, saturated with respect to \(\pi\), define \(V \subseteq
Y\) such that \(U = \pi^{-1}(V)\). Since \(\pi\) is a quotient morphism, it
follows that \(V\) is necessarily open in \(Y\), thus \(\pi(U) = V\) is open.

The proof for the closed set case is completely analogous.
\end{proof}

\begin{proposition}[Properties of quotient morphisms]
\label{prop:properties-quotient-morphism}
The following properties pertain to quotient morphisms between topological
spaces.
\begin{enumerate}[(a)]\setlength\itemsep{0em}
\item The composition of quotient morphisms is a quotient morphism.
\item Injective quotient morphisms are isomorphisms.
\item Let \(\pi: X \epi Y\) be a quotient morphism. Then, \(C \subseteq Y\) is
  closed if and only if \(\pi^{-1}(C) \subseteq X\) is closed.
\item Let \(\pi: X \epi Y\) be a quotient morphism and \(U \subseteq X\) be any
  saturated set (open or closed) with respect to \(\pi\). Then, the restriction
  \(\pi|_U: U \epi \pi(U)\) is a quotient map.
\item Let \(J\) be an indexing set and \(\{\pi_j: X_j \epi Y_j\}_{j \in J}\) be
  an indexed collection of quotient morphisms. The map \(\pi: \coprod_{j \in J}
  X_j \epi \coprod_{j \in J} Y_j\) defined by the restrictions \(\pi(x_{j}) =
  \pi_{j}(x_j)\) for every \(x_j \in X_{j} \cap \coprod_{j \in J} X_j\).
\end{enumerate}
\end{proposition}

\todo[inline]{Prove properties of the quotient morphism}

\begin{example}[Cones]
\label{exp:remove-section-cone-isomorphism}
An application of the last proposition takes us back to \cref{exp:cone}, where
we defined the cone \(\Cone X\) of a topological space \(X\) as \((X \times I) /
(X \times \{1\})\) --- the collapse the top of the cylinder. Notice that, given
any point \((x, t) \in (X \times I) \setminus (X \times \{1\})\), there exists a
neighbourhood \(U \subseteq (X \times I) \setminus (X \times \{1\})\) of \((x,
t)\), therefore \(X \times \{1\}\) is closed in \(X \times I\). If we consider
the quotient morphism of the cone \(\pi: X \times I \epi \Cone X\), any element
of the bottom \(p \in X \times \{1\}\) is mapped to a single element of the
cone, say \(p \xmapsto \pi * \in \Cone X\) --- and since \(U\) is a closed
saturated set of \(X\) with respect to \(\pi\), its image \(\pi(U) = * \subseteq
\Cone X\) is closed. Notice that \(X \iso X \times \{1\}\) by the natural
identification \(x \xmapsto \sim (x, 1)\), thus we can define a quotient map
\(q: X \isoto X \times \{0\} \epi *\) is
\todo[inline]{\(q\) is injective??}
\end{example}


\subsubsection{Sufficient But Not Necessary Conditions}

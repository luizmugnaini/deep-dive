\section{Bases and Dimensions}

\begin{definition}
  Let \(V\) a \(k\)-vector space and \(S \subseteq V\) a subset. By the
  universal property of free vector spaces, the inclusion \(S \hookrightarrow
  V\) induces the unique \(k\)-linear morphism \(f : k^{\oplus S} \to V\). Then
  we can classify the set \(S\) in terms of \(f\) as:
  \begin{enumerate}[I.]
    \item Linearly independent, in the case where \(f\) is injective, and
      linearly dependent otherwise.
    \item Generating (or spanning) if \(f\) is surjective.
    \item Basis if \(f\) is an isomorphism.
  \end{enumerate}
\end{definition}

\begin{proposition}
  A set \(S \subseteq V\) is linearly independent if and only if for all subset
  \(\{v_i\}_i \subseteq S\) and scalars \(\{a_i\}_i \subseteq k\), the equation
  \(\sum_i a_i v_i = 0\) implies \(a_i = 0\) for all index \(i\).
\end{proposition}

\begin{definition}[Subspace spanned]
  Let \(V\) be a \(k\)-vector space. Given a set \(S \subseteq V\), we define
  the subspace spanned by \(S\) to be
  \[
    \mathrm{span}(S) = \im(k^{\oplus S} \to V)
  \]
\end{definition}

\begin{proposition}
  Let \(V\) be a \(k\)-vector space and the set \(S \subseteq V\). Then \(S\) is
  a basis for \(V\) if and only if for all \(v \in V\) there exists a unique
  tuple \((a_s)_{s \in S} \subseteq k\) such that there are only finitely many
  \(a_s \neq 0\) and \(\sum_{s \in S} a_s s = v\).
\end{proposition}

\begin{lemma}[Zorn's Lemma]
  Let \(X\) be a set and consider the partial order relation \((\vdash)
  \subseteq X \times X\), so that \(\vdash\) is reflexive for all \(x \in X\),
  antisymmetric for all \(x, y \in X\), and transitive for all \(x, y, z \in
  X\). Moreover, if \(S \subseteq X\) is such that only one of the propositions:
  \(x \vdash y\) or \(y \vdash x\) is true for every \(x, y \in S\); then there
  exists an element \(z \in X\) for which \(x \vdash z\) for all choices of \(x
  \in X\). Then there exists an element \(m \in X\) for which \(m \vdash x\) if
  and only if \(x = m\).
\end{lemma}

\begin{theorem}
  Every vector space has a basis.
\end{theorem}

\begin{proof}
  Our initial goal is to construct a set \(\mathcal A\) together with a partial
  order and then a set \(\mathcal B\) subset of the first for which the last
  condition of Zorn's lemma is satisfied. Let then \(\mathcal A := \{X \subseteq
  V : X \text{ linearly independent}\}\), partially order under inclusion.
  Moreover, let \(\mathcal B \subseteq \mathcal A\) so that for every two
  elements \(X, Y \in \mathcal B\) we have \(X \subseteq Y\) or (exclusive) \(Y
  \subseteq X\).

  To show that indeed \(\mathcal B\) satisfies the last condition, define \(Z\)
  to be the inclusion of all sets in \(\mathcal B\). \(Z\) is linearly
  independent so that \(Z \in \mathcal A\). We prove such statement: given any
  finite subset \(\{v_1, \dots, v_n\} \subseteq Z\) then let the sets \(X_i \in
  \mathcal C\) be such that \(v_i \in X_i\) and then label the vectors and
  corresponding sets such that \(X_i \subseteq X_{i+1}\) and therefore \(X_1
  \subseteq  X_2 \subseteq  \dots \subseteq X_n\). Then, we have \(\{v_1, \dots,
  v_n\} \subseteq X_n \in \mathcal C\), which implies in \(\{v_1, \dots, v_n\}
  \subseteq Z\) being linearly independent regardless of the choice of such
  vectors. This finishes the proof that \(Z \in \mathcal A\) and \(X \subseteq
  Z\) for every \(X \in \mathcal C\), which shows that \(Z\) satisfies the last
  condition.

  Then by means of Zorn's lemma we conclude that there exists a set \(S \in
  \mathcal A\) such that \(S \subseteq X \in \mathcal A\), which implies in \(S
  = X\).  Since \(S \in \mathcal A\) then \(k^{\oplus S} \to V\) is injective.
  Now we show that it is also surjective. Suppose not, so that there exists a
  vector \(v \in V\) such that \(v \not\in \mathrm{Span}(S)\), then \(S \cup
  \{v\} \in \mathcal A\) (linealy independent) but also \(S \subseteq S \cup
  \{v\}\) but \(S \neq S \cup \{v\}\), which is a contradiction obtained by
  supposing that \(S\) does not span \(V\). Thus we conclude that
  \(\mathrm{Span}(S) = V\) and moreover \(k^{\oplus S} \to V\) is an
  isomorphism, so that \(S\) is a basis.
\end{proof}

\begin{corollary}
  Every vector space is isomorphic to \(k^{\oplus S}\) for some set \(S\).
\end{corollary}

\begin{proposition}
  Let \(B\) be a basis of the \(k\)-vector space \(V\). Then for every
  \(k\)-vector space \(W\) there is a natural isomorphism of vector spaces
  \[
    \Hom_{\Vect_k}(V, W) \iso \prod_{b \in B} W
  \]
\end{proposition}

\begin{proof}
  We show that the map
  \[
    (f: V \to W) \longmapsto (f(b))_{b \in B}
  \]
  is an isomorphism. First, it is injective: let \(f, g \in \Hom(V, W)\);
  suppose that \(f = g\) so that obviously \((f(b))_{b \in B} = (g(b))_{b \in
  B}\) since \(f(b) = g(b)\) for all \(b \in B\); suppose now that \((f(b))_{b
  \in B} = (g(b))_{b \in B}\), since \(B\) is a basis on \(V\) then, given any
  \(v \in V\), we can write it as \(v = \sum_{b \in B} a_b b\) where \(a_b \neq
  0\) only finitely many and \(a_b \in k\), then
  \[
    f(v) = f\bigg(\sum_{b \in B} a_b b\bigg) = \sum_{b \in B} a_b f(b)
    = \sum_{b \in B} a_b g(b) = g\bigg(\sum_{b \in B} a_b b\bigg) = g(v)
  \]
  thus indeed the mapping is injective. Now we show that it is surjective:
  notice that \(|W|^{|B|} = \left| \prod_{b \in B} W \right|\) and since the
  morphism of \(\Hom(V, W)\) are completely determined by its image under \(B
  \subseteq V\) then we conclude that \(|\Hom(V, W)| = |W|^{|B|}\), thus we can
  create a mapping which satisfies the surjectivity.
\end{proof}

\begin{proposition}
  Let \(\alpha : S \to T\) be a map between sets. Then the induced map
  \(\alpha_\ast : k^{\oplus S} \to k^{\oplus T}\) is such that
  \begin{enumerate}[i.]
    \item \(\alpha_\ast\) is an monomorphism if \(\alpha\) is injective.
    \item \(\alpha_\ast\) is an epimorphism if \(\alpha\) is surjective.
    \item \(\alpha_\ast\) is an isomorphism if \(\alpha\) is bijective.
  \end{enumerate}
\end{proposition}

\begin{proof}
  For the first, let \(\alpha\) be injective, then given \(f \in k^{\oplus S}\)
  and define the finite set \(S_f := S \setminus \ker f\). Now we construct a
  unique function \(g \in k^{\oplus T}\) defined such that \(g(t) \neq 0\) iff
  \(t \in \alpha(S_f)\) (well defined since \(\alpha(S_f)\) is finite by the
  injectivity of \(\alpha\)), and moreover the condition \(f(s) =
  g(\alpha(s))\) (uniqueness), then our mapping \(\alpha_{\ast}\) can be
  defined as \(f \mapsto g\), which is surely a monomorphism and well defined.

  For the second proposition, let \(\alpha\) be surjective, then from the same
  map as before, given any function \(g \in k^{\oplus T}\), let  \(T_g := T
  \setminus \ker g\) then for every \(t \in T_g\) we chose one \(s \in
  \alpha^{-1}(t)\) so that for some function \(f \in k^{\oplus T}\) defined by
  \(f(s) = g(\alpha(s)) = g(t)\) and moreover \(f(s) \neq 0\) iff \(s \in
  \alpha^{-1}(t)\) for some \(t \in T_g\) (thus \(f\) is well defined, since
  \(T_g\) is finite and we are taking only one corresponding \(s\) for each \(t
  \in T_g\)). Thus indeed \(f \xmapsto{\alpha_\ast} g\) is surjective.

  The last proposition comes trivially from the last two.
\end{proof}

\begin{proposition}
  Let the surjective \(k\)-linear morphism \(f: V \hookrightarrow W\). There
  exists an injective \(k\)-linear morphism \(g: W \twoheadrightarrow V\) such
  that \(f \circ g = \Id_{W}\).
\end{proposition}

\begin{proof}
  Just take the mapping \(w \overset g \longmapsto v \in f^{-1}(w) \in V\), then
  \(w \xmapsto g v \xmapsto f w\) and thus \(f \circ g = \Id_W\).
\end{proof}

\begin{proposition}
  Let the injective \(k\)-linear morphism \(f: V \hookrightarrow W\). There
  exists a surjective morphism \(g : W \twoheadrightarrow V\) such that \(g
  \circ f = \Id_V\).
\end{proposition}

\begin{proof}
  Given any \(v \in V\), we want \(g(f(v)) = g(w) = v\), but since \(f^{-1}(w)\)
  is a singleton, we can simply make the well defined mapping \(w \xmapsto g v\)
  for all \(w \in \im(f)\). With this unique condition we already have the
  wanted \(g \circ f = \Id_V\). Moreover, given any \(v \in V\), there must
  exist \(w \in W\) (in fact we could specify \(w \in \im(f)\), but here we can
  be more general) such that \(g(w) = v\), thus \(g\) is surjective.
\end{proof}

\begin{proposition}
  Let \(k\)-vector spaces \(V\) and \(W\). There exists injective linear
  morphism \(V \hookrightarrow W\) if and only if there exists a surjective
  morphism \(W \twoheadrightarrow V\).
\end{proposition}

\begin{proof}
  The last two propositions.
\end{proof}

\begin{theorem}\label{thm: iso of basis}
  Let sets \(S\) and \(T\). The following propositions are equivalent:
  \begin{enumerate}[I.]
    \item There exists injective linear morphism \(k^{\oplus S} \hookrightarrow
      k^{\oplus T}\).
    \item There exists surjective linear morphism \(k^{\oplus T}
      \twoheadrightarrow k^{\oplus S}\).
    \item There exists an injection \(S \hookrightarrow T\).
    \item There exists a surjective map \(T \twoheadrightarrow S\) or \(S =
      \varnothing\).
  \end{enumerate}
\end{theorem}

\begin{lemma}
  Let \(V\) a \(k\)-vector space, \(B \subseteq V\) a basis and \(S \subseteq
  V\) a spanning set. Then for all \(b \in B\) there exists \(a \in A\) such
  that \((B \setminus \{b\}) \cup \{a\}\) is a basis.
\end{lemma}

\begin{proof}
  Let \(b \in B\) be any element. Suppose, for the sake of contradiction, that
  \(A \subseteq \mathrm{span}(B \setminus \{b\})\) then \(V = \mathrm{span}(A)
  \subseteq \mathrm{span}(B \setminus \{b\})\), which is a contradiction because
  \(B\) is said to be a basis for \(V\); thus \(A \not\subseteq
  \mathrm{span}(B\setminus \{b\})\). Let \(a \in A \setminus \mathrm{span}(B
  \setminus \{b\})\), then consider the set \((B \setminus \{b\}) \setminus
  \{a\}\). Let then \(a = cb + c_1 b_1 + \dots + c_n b_n\) with \(c \neq 0\)
  from the construction of \(a\) (that is, \(a \not\in \mathrm{span}(B
  \setminus \{b\})\)). This way we can make
  \[
    b = a/c - c_1/c b_1 - \dots - c_n/c b_n\
    \text{ therefore }\ b \in \mathrm{span}((B \setminus \{b\}) \cup \{a\})
  \]
  which makes \((B \setminus \{b\}) \cup \{a\}\) a basis for \(V\).
\end{proof}

\begin{lemma}
  Let \(B\) and \(B'\) be bases the \(k\)-vector space \(V\), then there exists
  a bijection \(B \iso B\), and thus \(|B| = |B'|\).
\end{lemma}

\begin{proof}
  From theorem \cref{thm: iso of basis} we see that since exists injective
  \(k\)-linear morphism \(V \hookrightarrow V\), then exists injective map \(B
  \hookrightarrow B'\) and also surjective map \(B \twoheadrightarrow B'\). From
  Cantor-Schröder-Bernstein theorem we see that there exists bijection \(B \iso
  B'\).
\end{proof}

\begin{proposition}\label{prop: li to basis}
  Let \(V\) be a \(k\)-vector space and \(S \subseteq V\) be any linearly
  independent set. There exists a set \(B \supseteq S\) such that \(B\) is a
  basis of \(V\).
\end{proposition}

\todo[inline]{Prove: linearly independent \(\to\) basis}

\subsection{Matrices and changes of base}

\begin{definition}[Change of basis operator]
  \label{def: change of basis matrix}
  Let \(\mathcal C : V \to V\) be a linear operator of a finite dimensional
  \(k\)-vector space, \(V \iso k^n\), defined by the mapping \([v]_{B_1}
  \xmapsto{\mathcal C} [v]_{B_2}\), where \([v]_{B_1}\) and \([v]_{B_2}\) are the
  representations of the vector \(v\) in the basis \(B_1\) and \(B_2\) of \(V\).
  We define the matrix representation \(C_{B_1, B_2} : k^n \to k^n\) of the
  linear operator \(\mathcal C\) to be the \emph{change of basis matrix}. Then
  if \(B_1 = \{e_i\}_{i=1}^n\), and \(B_2 = \{e'_j\}_{j=1}^n\), and \(C_{B_1,
  B_2} = [c_{i, j}]\), then the coefficients \(c_{i, j}\) of \(C_{B_1, B_2}\)
  must be such that
  \[
    v = \sum_{i=1}^n a_i e_i = \sum_{j=1}^n b_j e'_j
    = \sum_{j=1}^n b_j \left( \sum_{i=1}^n c_{i, j} e_i \right)
    = \sum_{i=1}^n \left( \sum_{j=1}^n c_{i, j} b_j \right) e_i
  \]
\end{definition}

\begin{proposition}[Change of basis, linear morphism]
  \label{prop: change of basis, linear morphism}
  Let \(f : V \to W\) be a \(k\)-linear morphism of finite dimensional spaces,
  \(V \iso k^n\) and \(W \iso k^m\). Let \(M_{B, S} : k^n \to k^m\) and \(M_{B',
  S'}\) be the matrix representation of \(f\) in the basis \(B\) and \(S\),
  where \(B\) is a basis of \(V\) and \(S\) a basis of \(W\). Let now \(B'\) and
  \(S'\) be basis for \(V\) and \(S\). Then, the matrix representation of \(f\)
  in the basis \(B', S'\), that is, \(M_{B', S'} : k^n \to k^m\), is given by
  the conjugation
  \[
    M_{B', S'} = C_{S, S'}^{-1} M_{B, S} C_{B, B'}
  \]
  where \(C_{S, S'}\) and \(C_{B, B'}\) are the change of basis matrix
  representations defined in \cref{def: change of basis matrix}.
\end{proposition}

\begin{proof}
  Consider the function \(\mathcal{C}_{S, S'}^{-1} \circ f \circ \mathcal{C}_{B,
  B'} : V \to W\), then we have the mapping
  \[
    [v]_B \xmapsto{\mathcal{C}_{B, B'}} [v]_{B'}
    \xmapsto f [f([v]_{B'})]_S
    \xmapsto{\mathcal{C}_{S, S'}} [f([v]_{B'})]_{S'}
  \]
  hence the statement follows directly from \cref{prop: matrix of the
  composition}.
\end{proof}

\begin{corollary}\label{cor: change of basis, linear operator}
  In particular, if \(f : V \to V\) is a \(k\)-linear operator, and \(B, B'\)
  are basis of \(V\), then given the matrix representation of \(f\) in base
  \(B\), namely, \(M_B\), we have that the matrix representation of \(f\) in
  base \(B'\) is given by
  \[
    M_{B'} = C^{-1} M_B C
  \]
  where \(C\) is the matrix representation of the change of basis operator from
  base \(B\) to \(B'\).
\end{corollary}

\subsection{Dimension and Rank plus Nullity Theorem}

\begin{definition}
  Let \(V\) be a \(k\)-vector space. We define the dimension of \(V\) as
  \[
    \dim_k(V) = |B|
  \]
  for \(B \subseteq V\) basis of \(V\). In particular, if \(\dim_k(V) < \infty\)
  we say that \(V\) is a finite dimensional \(k\)-vector space.
\end{definition}

\begin{lemma}\label{lem: rank plus nullity lemma}
  Let \(V\) be a \(k\)-vector space and \(W \subseteq V\) a \(k\)-subspace, then
  \[
    \dim_k(V) = \dim_k(W) + \dim_k(V/W).
  \]
\end{lemma}

\begin{proof}
  Let \(B_W = \{w_i\}_{i \in I}\) be a basis for \(W\) and  \(B_{V/W} =
  \{q_j\}_{j \in J}\) a basis for \(V/W\). Then \(\forall j \in J, \exists v_j
  \in V : q_j = [v_j] \in V/W\). Define the set \(A := \{v_j : j \in J\}\),
  then the map \(\varphi : J \to A\) with the mapping \(j \mapsto v_j\) is
  surely surjective; moreover, if \(j, t \in J\) and  \(j \neq t\), then \(q_j
  \neq q_t\) since \(B_{V/W}\) is a basis, hence \([v_j] \neq [v_t]\) and the
  map \(\varphi\) is injective. Then \(\varphi\) is a bijection and thus \(|A| =
  |J|\). Now we show that \(A\) is a linearly independent set: let \(J_t
  \subseteq J\) be a finite subset of \(J\) with \(t\) elements and \(\alpha_1,
  \dots, \alpha_t \in k\) be such that \(\sum_{\ell=1}^t \alpha_\ell v_{j_\ell}
  = 0 \in V\) then \(\sum_{\ell=1}^t \alpha_\ell [v_{j_\ell}] = \big[
  \sum_{\ell=1}^t \alpha_\ell v_{j_\ell} \big] = [0] \in V/W\) but notice that
  the set of classes \([v_j] : v_j \in A\) is linearly independent (from the
  fact that this corresponds to the basis \(B_{V/W}\)), hence we must have
  \(\alpha_\ell = 0\) for all \(1 \leq \ell \leq t\), thus \(A\) is linearly
  independent.

  Consider now the set \(A \cup B_W\), we'll show that this is a basis for
  \(V\). First we show that \(A \cup B_W\) is linearly independent: let \(I_s
  \subseteq I\) be the finite set with \(s\) indices, and \(J_t \subseteq J\) be
  the finite set with \(t\) indices, and \(\alpha_1, \dots \alpha_s, \beta_1,
  \dots, \beta_t \in k\), be such that \(\sum_{\ell=1}^s \alpha_\ell w_{i_\ell}
  + \sum_{r=1}^t \beta_r v_{j_r} = 0 \in V\) if we now take the module with
  respect to \(W\) (as we did before with \(A\)) we see that
  \[
    [0] = \sum_{\ell=1}^s\alpha_\ell [w_{i_\ell}]
    + \sum_{r=1}^t \beta_r [v_{j_r}] = \bigg[\sum_{r=1}^t \beta_r v_{j_r}\bigg]
    \in V/W
  \]
  since \(w_{i_\ell} \in W\) and thus \([w_{i_\ell}] = [0]\); since \(B_{V/W}\)
  (and \([v_{j_r}] = q_{j_r}\)) is linearly independent, we see that
  necessarily \(\beta_r = 0\) for all \(1 \leq r \leq t\). Therefore, we
  conclude that \(\sum_{\ell=1}^s \alpha_\ell w_{i_\ell} = 0\) and since \(B_W\)
  is linearly independent, we conclude that \(\alpha_{\ell} = 0\) for all  \(1
  \leq \ell \leq s\). For the final part, we need to show that \(V =
  \mathrm{span}(A \cup B_{W})\): let \(v \in V\) be any element, then \([v] \in
  V/W = \mathrm{span}(B_{V/W})\) then \([v] = \sum_{\ell=1}^t \alpha_\ell
  [v_{j_\ell}]\) for \(J_t \subseteq J\) and \(\alpha_1, \dots, \alpha_t \in
  k\); this implies that \([v] - \big[ \sum_{\ell=1}^t \alpha_\ell v_{j_\ell}
  \big] = [0]\) and thus \(v - \sum_{\ell=1}^t \alpha_\ell v_{j_\ell} \in W =
  \mathrm{span}(B_{W})\); hence for some \(\beta_1, \dots, \beta_s \in k\) and
  \(I_s \subseteq I\) we have \(v - \sum_{\ell=1}^t \alpha_\ell v_{j_\ell} =
  \sum_{r=1}^s \beta_r w_{i_r}\) hence \(v \in \mathrm{span}(A \cup B_{W})\) and
  thus \(A \cup B_W\) is a basis for \(V\).

  We'll now show that \(A \cap B_W = \emptyset\): suppose \(u \in A \cap B_W\),
  then \(\exists i \in I, j \in J : u = w_i \text{ and } u = v_j\) thus we have
  \(q_j = [v_j] = [w_i] = [0] \in V/W\) which cannot be true since \(B_{V/W}\)
  is a basis. Therefore \(|A \cup B_W| = |A| + |B_W| = |J| + |B_W| =  |B_{V/W}|
  + |B_W| = \dim_k(V)\) as wanted.
\end{proof}

\begin{theorem}[Rank plus nullity]\label{thm: rank plus nullity}
  Let \(L : V \to W\) be a \(k\)-linear morphism, then
  \[
    \dim_k(V) = \dim_k(\ker(L)) + \rank(L).
  \]
\end{theorem}

\begin{proof}
  Notice that from the first isomorphism theorem we have \(V/\ker(L) \iso
  \im(L)\) then it follows that \(\dim_k(V/\ker(L)) = \dim_k(\im(L)) =
  \rank(L)\) from the fact that \(\ker(L)\) is a \(k\)-subspace of \(V\), we
  conclude from \cref{lem: rank plus nullity lemma} that \(\dim_k(V) =
  \dim_k(\ker(L)) + \dim_k(V/\ker(L)) = \dim_k(\ker(L)) + \rank(L)\).
\end{proof}

\begin{corollary}
  Let \(V\) and \(W\) be \(k\)-vector spaces and \(L : V \to W\) be a
  \(k\)-linear morphism. Then we have
  \begin{enumerate}[(a).]
    \item If \(L\) is injective, then \(\dim_k(V) \leq \dim_k(W)\).
    \item If \(L\) is surjective, then \(\dim_k(V) \geq \dim_k(W)\).
    \item If \(L\) is an isomorphism then \(\dim_k(V) = \dim_k(W)\).
  \end{enumerate}
\end{corollary}

\begin{proof}
  (a) If \(L\) is injective, then \(\ker(L) = 0\) and thus \(\dim_k(\ker(L)) =
  0\) and from \cref{thm: rank plus nullity} we have \(\dim_k(V) = \rank(L)\) and
  since \(\rank(L) \leq \dim_k(W)\) since \(\im(L) \subseteq W\) then we
  conclude that the proposition is true. (b) Since \(L\) is surjective, then
  \(\im(L) = W\) and in particular \(\rank(L) = \dim_k(W)\), thus \cref{thm: rank
  plus nullity} implies that  \(\dim_k(V) = \dim_k(\ker(L)) + \dim_k(W) \geq
  \dim_k(W)\). (c) Already proved before, but also comes from the last two
  propositions.
\end{proof}

\begin{corollary}\label{cor: equal dim - iso conditions}
  Let \(L : V \to W\) be a \(k\)-linear morphism of finite vector spaces \(V\)
  and \(W\) such that \(\dim_k(V) = \dim_k(W)\). Then the following propositions
  are equivalent:
  \begin{enumerate}[(a).]
    \item \(L\) is an isomorphism.
    \item \(L\) is injective.
    \item \(L\) is surjective.
  \end{enumerate}
\end{corollary}

\begin{proof}
  (b \(\Rightarrow\) c) Suppose \(L\) injective, then from \cref{thm: rank plus
  nullity} we have \(\dim_k(V) = \rank(L)\) since \(\dim_k(\ker(L)) = 0\) then
  we have \(\dim_k(W) = \rank(L)\) but then \(\im(L) = W\) and therefore \(L\)
  is a surjective linear map. (c \(\Rightarrow\) a) Suppose that \(L\) is
  surjective, from from \cref{thm: rank plus nullity} we have \(\dim_k(V) =
  \dim_k(\ker(L)) + \rank(L) = \dim_k(\ker(L)) + \dim_k(W)\), but from
  hypothesis we have \(\dim_k(V) = \dim_k(W)\), then, if the given vector spaces
  \(V\) and \(W\) are finite dimensional we have \(\dim_k(\ker(L)) = 0\) and
  thus \(\ker(L)\) is the zero space, which implies that \(L\) is injective,
  hence an isomorphism.
\end{proof}


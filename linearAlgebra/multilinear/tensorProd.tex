\todo[inline]{Add content about bilinear maps}

\section{Multilinear Maps and Tensor Products}

\begin{definition}[Multilinear map]\label{def: multilinear map}
  Let \(\{V_i\}_{i=1}^n\) be a collection of \(k\)-vector spaces and \(W\) be a
  \(k\)-vector space. We say that a map
  \[
    \varphi: \prod_{i=1}^n V_i \to W
  \] 
  is a multilinear map if for a given \(1 \leq t \leq n\) we have that for all
  collections \(\{v_i : v_i \in V_i, i \neq t\}\) the map
  \[
    \varphi(v_1, \dots, v_{t-1}, -, v_{t+1}, \dots, v_n): V_t \to W
  \] 
  is a \(k\)-linear morphism.
\end{definition}

\begin{definition}[Space \(\mathcal M\)]\label{def: space M}
  Given a finite collection of \(k\)-vector spaces \(\{V_i\}_{i \in I}\), we
  define the \(k\)-vector space \(\mathcal M = k^{\oplus \prod_{i \in I} V_i}\),
  that is, the space of set-theoretic mappings \(\prod_{i \in I} V_i \to k\)
  with finite support. Since the collection \(\{(v_i)_{i \in I} \in \prod_{i \in
  I} V_i\}\) form a basis for \(\mathcal M\) way we can say that \(\mathcal M\)
  consists of formal linear combinations
  \[
    \mathcal M = 
    \left\{
      \sum a_{(v_i)_i} (v_i)_i :
      (v_i)_i \in \prod_{i \in I} V_i,\
      a_{(v_i)_i} \in k
    \right\}.
  \] 
  Thus, if \(\mathrm{char}(k) = 0\) and exists \(i \in I : \dim_k(V_i) > 0\),
  then \(\mathcal M\) is an infinite-dimensional vector space.
\end{definition}

\begin{definition}[Subspace \(\mathcal M_0\)]\label{def: subspace M_0}
  We define the subspace \(\mathcal M_0 \subseteq \mathcal M\) as
  \[
    \mathcal M_0 = \mathrm{span}
    \big(
      (v_i, \dots, a(v'_j + v''_j), \dots, v_n) - a(v_i, \dots, v'_j +
      v''_j, \dots, v_n) : j \in I,\ a \in k,\ |I| = n 
    \big)
  \]
  where the generating elements belong to \(\mathcal M\) (that is, \(v_i \in
  V_i\) for each index).
\end{definition}

\begin{definition}[Tensor product]\label{def: tensor product}
  We define the tensor product of the finite collection of \(k\)-vector spaces
  \(\{V_i\}_{i \in I}\) as the quotient space
  \[
    \bigotimes_{i \in I} V_i = \mathcal M / \mathcal M_0
  \] 
  where the elements of \(\bigotimes_{i \in I} V_i\) are called tensors, and in
  special, the elements of the form \(\otimes_{i \in I} v_i = (v_i)_{i \in I}
  + \mathcal M_0 \in \bigotimes_{i \in I} V_i\) are called factorizable tensors,
\end{definition}

\begin{lemma}\label{lem: tensor map}
  Let \(\{V_i\}_{i \in I}\) be a finite collection of \(k\)-vector spaces. The
  canonical map to the factorizable tensors
  \[
    t: \prod_{i \in I} V_i \to \bigotimes_{i \in I} V_i
    \text{, mapping }
    (v_i)_{i \in I} \longmapsto \otimes_{i \in I} v_i
  \] 
  is multilinear.
\end{lemma}

\begin{proof}
  Let \(|I| = n\), then
  \begin{align*}
    t(v_1, \dots, v_j + v'_j, \dots, v_n) 
    &= v_1 \otimes \dots \otimes (v_j + v'_j) \otimes \dots \otimes v_n
    \\
    &= (v_1, \dots, v_j + v'_j, \dots, v_n) + \mathcal M_0
    \\
    &= [(v_1, \dots, v_j, \dots, v_n) + \mathcal M_0] 
    + [(v_1, \dots, v'_j, \dots, v_n) + \mathcal M_0]
    \\
    &= v_1 \otimes \dots \otimes v_j \otimes \dots \otimes v_n
    + v_1 \otimes \dots \otimes v'_j \otimes \dots \otimes v_n
    \\
    &= t(v_1, \dots, v_j, \dots, v_n) + t(v_1, \dots, v'_j, \dots, v_n)
  \end{align*}
  where the spliting of the class was possible because \((v_1, \dots, v_{j -1},
  0, v_{j+1}, \dots, v_n) \in \mathcal M_0\). Moreover, if \(a \in k\), then
  \begin{align*}
    t(v_1, \dots, a v_j, \dots, v_n)
    &= v_1 \otimes \dots \otimes a v_j \otimes \dots v_n
    \\
    &= (v_1, \dots, a v_j, \dots, v_n) + \mathcal M_0
    \\
    &= a[(v_1, \dots, v_j, \dots, v_n) + \mathcal M_0]
    \\
    &= a (v_1 \otimes \dots \otimes v_j \otimes \dots \otimes v_n) 
    \\
    &= a t(v_1, \dots, v_j, \dots, v_n)
  \end{align*}
  where the spliting of the classes was possible for the same reason as before.
\end{proof}

\begin{theorem}[Universal property of tensor products]
  \label{thm: universal property of tensor products}
  Let \(\{V_i\}_{i \in I}\) be a finite collection of \(k\)-vector spaces, and
  \(L\) be any \(k\)-vector space, and \(f: \prod_{i \in I} V_i \to L\) be any
  multilinear map. Then there exists a unique \(k\)-linear morphism \(\ell :
  \bigotimes_{i \in I} V_i \to L\) such that the following diagram commutes
  \[
    \begin{tikzcd}
      \prod_{i \in I} V_i 
      \ar[r, "f"]
      \ar[d, swap, "t"] 
        &L \\
      \bigotimes_{i \in I} V_i
      \ar[ur, swap, dashed, "\ell"]
    \end{tikzcd}
  \] 
\end{theorem}

\begin{proof}
  Let \(|I| = n\). (Uniqueness) Suppose that for any \(L\) and \(f\), the
  morphism \(\ell\) exists, so that \(f = \ell \circ t\) and therefore
  \[
    \ell \circ t(v_1, \dots, v_n) = \ell(v_1 \otimes \dots \otimes v_n)
    = f(v_1, \dots, v_n)
  \] 
  then certainly \(\ell\) is uniquely defined by the pointwise image of \(f\),
  since \(\{(v_1, \dots, v_n) \in \prod_{i=1}^n V_i\}\) generates
  \(\bigotimes_{i=1}^n V_i\).

  (Existence) Let \(g : \mathcal M \to L\) be defined as \(g(v_1, \dots, v_n) =
  f(v_1, \dots, v_n)\) so that \(f\) completely determines the image of \(g\).
  Let now \((v_1, \dots, v_{j-1}, 0, v_{j+1}, \dots, v_n) \in \mathcal M_0\),
  then
  \[
    g(v_1, \dots, v_{j-1}, 0, v_{j+1}, \dots, v_n)
    = f(v_1, \dots, v_{j-1}, 0, v_{j+1}, \dots, v_n) = 0
  \] 
  because \(f\) is a multilinear map, then \((v_1,, \dots, v_{j-1}, 0, v_{j+1},
  \dots, v_n) \in \ker(g)\). Moreover, for the second type of element of
  \(\mathcal M_0\)
  \begin{align*}
    g\big(a(v_1, \dots, v_{j-1}, 0, v_{j+1}, \dots, v_n)\big)
    &= f\big(a(v_1, \dots, v_{j-1}, 0, v_{j+1}, \dots, v_n)\big) \\
    &= f\big(a(v_1, \dots, v_{j-1}, 0, v_{j+1}, \dots, v_n)\big) \\
    &= a f(v_1, \dots, v_{j-1}, 0, v_{j+1}, \dots, v_n) \\
    &= 0
  \end{align*}
  thus \(a(v_1, \dots, v_{j-1}, 0, v_{j+1}, \dots, v_n) \in \ker(g)\) and
  hence \(\mathcal M_0 \subseteq \ker(g)\). From the universal property of
  quotients \cref{thm: universal property for quotients} we have that \(g\) 
  induces the uniquely defined \(k\)-linear morphism \(\ell: \mathcal M /
  \mathcal M_0 \to L\) such that the diagram commutes
  \[
    \begin{tikzcd}
      \mathcal M
      \ar[d, two heads, "\pi"]
      \ar[r, "g"]
        &L \\
      \mathcal M / \mathcal M_0
      \ar[ur, swap, dashed, "\ell"]
        &
    \end{tikzcd}
  \]
  then \(g = \ell \circ \pi\) and therefore for all \((v_1, \dots, v_n) \in
  \prod_{i=1}^n V_i\) we have
  \[
    f(v_1, \dots, v_n) 
    = g(v_1, \dots, v_n) 
    = \ell \circ \pi (v_1, \dots, v_n)
    = \ell(v_1 \otimes \dots \otimes v_n)
  \] 
  then indeed \(\ell \circ t = f\) as wanted.
\end{proof}


\begin{corollary}[Multilinear maps are isomorphic to linear maps]
  \label{cor: multilinear maps are isomorphic to linear maps}
  Let \(\{V_i\}_{i=1}^p\) be a finite collection of \(k\)-vector spaces, and
  \(\Hom(V_1, \dots, V_p;\, L)\) denote the space of multilinear maps
  \(\prod_{i=1}^p V_i \to L\), where \(L\) is a given \(k\)-vector space. The
  \(k\)-linear morphism
  \[
    \psi :
    \Hom\left(V_1, \dots, V_p;\, L\right) \longrightarrow
    \Hom\left( \bigotimes_{i=1}^p V_i, L \right),\
    f \xmapsto \psi \ell
  \]
  where \(f = \ell \circ t\) (as in \cref{thm: universal property of tensor
  products}), is an isomorphism, so that 
  \[
    \Hom\left(V_1, \dots, V_p;\, L\right) \iso
    \Hom\left( \bigotimes_{i=1}^p V_i, L \right).
  \]
\end{corollary}

\begin{proof}
  Let \(\ell \in \Hom(\bigotimes_{i=1}^p V_i, L)\), then \(f = \ell \circ t
  \in \psi^{-1}(\ell)\) thus \(\psi\) is surjective. On the other hand, if
  \(f \neq 0\), then \(\ell \circ t \neq 0\) and in particular \(\ell \neq 0\)
  hence \(\psi(f) = \ell \neq 0\), therefore \(\ker(\psi) = 0\). Since \(\psi\)
  is linear, then \(\psi\) is injective.
\end{proof}

\subsection{Dimensions and Bases of Tensor Products}

\begin{proposition}
  Let \(\{V_i\}_{i \in I}\) be a finite collection of \(k\)-vector spaces. If
  exists \(j \in I : V_j = 0\) is the null space then \(\bigotimes_{i \in I}
  V_i = 0\), that is, the tensor product is the null space.
\end{proposition}

\begin{proof}
  Let \(V_j = 0\) for some \(j \in I\). Consider any \(f \in \Hom(\prod_{i
  \in I} V_i,\ L)\), where \(L\) is a \(k\)-vector space. Since \(f\) is linear
  on \(V_j\), consider any element \((v_i)_{i \in I} \in \prod_{i \in I} V_i\)
  then necessarily \(v_j = 0\) and hence \(f((v_i)_{i \in I}) = 0\), since
  \(V_j\) has only one element, every element of the domain of \(f\) is mapped
  to zero, which implies in \(f = 0\). Consider in particular the mapping \(t
  \in \Hom(\prod_{i \in I} V_i ,\ \bigotimes_{i \in I} V_i)\) (\cref{lem: tensor
  map}) then from the latter discussion we have \(t = 0\), since \(\bigotimes_{i
  \in I} V_i = \mathrm{span}(\im(t))\), then we conclude that \(\bigotimes_{i
  \in I} V_i = 0\).
\end{proof}

\begin{proposition}[Tensor product dimension]
  \label{prop: tensor product dimension}
  The dimension of the tensor product of a finite collection of finite
  \(k\)-vector spaces \(\{V_i\}_{i \in I}\) is equal to the product of the
  dimensions of the vector spaces, that is
  \[
    \dim_k \bigoplus_{i \in I} V_i = \prod_{i \in I} \dim_k V_i.
  \] 
\end{proposition}

\begin{proof}
  Let \(|I| = n\). If exists a null vector space in the collection, then the
  dimension of the tensor product is null, thus the proposition follows. Suppose
  there is no such null vector space in the collection, then since
  \(\bigotimes_{i \in I} V_i \iso \left( \bigotimes_{i \in I} V_i \right)^\ast =
  \Hom(\bigotimes_{i \in I} V_i, k)\). From \cref{cor: multilinear maps are
  isomorphic to linear maps} we find that \(\left( \bigotimes_{i \in I} V_i
  \right)^\ast \iso \left( \prod_{i \in I} V_i \right)^\ast\). Let \(B_i =
  \{e_j^{(i)}\}_{j=1}^{\dim_k V_i}\) be a basis for \(V_i\) for every \(i \in
  I\), so that for every element \(v_i \in V_i\) we can write it as a linear
  combination \(v_i = \sum_{j=1}^{|B_1|} x_j^{(i)} e_j^{(i)}\) where
  \(x_j^{(i)} \in k\). Then, for any multilinear map \(f \in \left( \prod_{i \in
  I} V_i \right)^\ast\) we have
  \begin{align*}
    f(v_1, \dots, v_n) 
    = f\left( \sum_{j_1=1}^{|B_1|} x_{j_1}^{(1)} e_{j_1}^{(1)}, \dots,
    \sum_{j_n=1}^{|B_n|} x_{j_n}^{(n)} e_{j_n}^{(n)} \right) 
    = \sum_{\substack{1 \leq j_i \leq |B_i|, \\ 1 \leq i \leq n}} 
    x_{j_1}^{(1)} \dots x_{j_n}^{(n)} f\left(e_{j_1}^{(1)}, \dots,
    e_{j_n}^{(n)}\right)
  \end{align*}
  so that the collection \(\prod_{i \in I} B_i\) forms a base for \(\left(
  \prod_{i \in I} V_i \right)^\ast\) and therefore 
  \[
    \dim_k \left( \prod_{i \in I} V_i \right)^\ast = \prod_{i \in I} |B_i| =
    \prod_{i \in I} \dim_k V_i
  \]
  Hence, since \(\bigoplus_{i \in I} V_i \iso \left( \prod_{i \in I} V_i
  \right)\), we conclude that \(\dim_k \bigoplus_{i \in I} V_i = \prod_{i \in I}
  \dim_k V_i\).
\end{proof}

\begin{lemma}[Tensor basis]\label{lem: tensor basis}
  Given a finite collection of \(k\)-vector spaces \(\{V_i\}_{i \in I}\) and
  suppose none of them are null. The basis for the tensor product
  \(\bigotimes_{i \in I} V_i\) is given by the collection of factorizable
  tensors \(\{\otimes_{i \in I} e_{j_i}^{(i)}\}\), where \(e_{j_i}^{(i)} \in
  B_i\) and \(B_i\) is a basis for \(V_i\).
\end{lemma}

\todo[inline]{What about infinite dimensional vector spaces and infinite tensor
products?}

\subsection{Tensor Product of Function Spaces}

\begin{proposition}
  Let \(\{S_i\}_{i \in I}\) be a finite collection of sets. There exists a
  canonical isomorphism
  \[
    k^{\oplus \left( \prod_{i \in I} S_i \right)} \iso \bigotimes_{i \in I}
    k^{\oplus S_i}.
  \] 
\end{proposition}

\begin{proof}
  We show that the \(k\)-linear morphism \(\phi: k^{\oplus \left( \prod_{i \in
  I} S_i \right)} \to \bigoplus_{i \in I} k^{\oplus S_i}\) mapping
  \(\delta_{(s_i)_{i \in I}} \mapsto \otimes_{i \in I} \delta_{s_i}\), where
  \(\delta\) is the Kronecker delta function, is an isomorphism. First, from
  \cref{prop: universal property free vs} we find that \(\{\delta_{(s_i)_{i \in
  I}} : (s_i)_{i \in I} \in \prod_{i \in I} S_i\}\) is a base for the space
  \(k^{\oplus \left( \prod_{i \in I} S_i \right)}\) and, in particular,
  \(\{\delta_{s_i} : s_i \in S_i\}\) is a base for the space \(k^{\oplus S_i}\).
  From \cref{lem: tensor basis} we find that \(\{\otimes_{i \in I} \delta_{s_i}
  : s_i \in S_i\}\) is a basis for \(\bigotimes_{i \in I} k^{\oplus S_i}\).
\end{proof}

\section{Canonical Isomorphisms and Tensor Products}

\subsection{Associativity and Commutativity}

\begin{proposition}[Associativity]\label{prop: associativity tensor prod}
  Let \(k\)-vector spaces \(V, W, L\), the map 
  \[
    (V \otimes W) \otimes L \to V \otimes (W \otimes L)\ \text{ mapping }\ (v
    \otimes w) \otimes \ell \mapsto v \otimes (w \otimes \ell)
  \]
  is a canonical isomorphism. Hence for any collection of \(k\)-vector spaces
  \(\{V_i\}_{i=1}^p\) we can write any arrangement of parenthesis for their
  tensor product as canonically isomorphic to \(V_1 \otimes \dots \otimes V_p\).
\end{proposition}

\begin{proof}
  Let \(B_V, B_W, B_L\) be basis for the vector spaces \(V, W, L\) respectively.
  Notice that \(\{(v \otimes w) \otimes \ell : (v, w, \ell) \in B_V \times B_W
  \times L\}\) is a basis for \((V \otimes W) \otimes L\) and \(\{v \otimes (w
  \otimes \ell) : (v, w, \ell) \in B_V \times B_W \times B_L\}\) is a basis for
  \(V \otimes (W \otimes L)\) (from \cref{lem: tensor basis}). Therefore, the
  map \((v \otimes w) \otimes \ell \mapsto v \otimes (w \otimes \ell)\)
  transforms one base into another, which implies that it is a canonical
  isomorphism of the considered spaces.
\end{proof}

\begin{proposition}[Commutativity]\label{prop: commutativity tensor prod}
  Let \(\{V_i\}_{i=1}^p\) be a collection of \(k\)-vector spaces, and \(\sigma\)
  be any permutation of the numbers \(\{1, \dots, p\}\). Define the \(k\)-linear
  morphism
  \[
    f_\sigma : \bigotimes_{i=1}^p V_i \to \bigotimes_{i=1}^p V_{\sigma(i)}\
    \text{ mapping }\ v_1 \otimes \dots \otimes v_p \mapsto v_{\sigma(1)}
    \otimes \dots \otimes v_{\sigma(p)}
  \] 
  where \(f_{\tau\sigma} = f_\tau \circ f_\sigma\), for any permutation \(\tau\) 
  on \(\{1, \dots, p\}\). Then \(f_\sigma\) is a canonical isomorphism.
\end{proposition}

\begin{proof}
  Let the map \(g_\sigma: \prod_{i=1}^p V_i \to \bigotimes_{i=1}^p
  V_{\sigma(i)}\) such that \((v_1, \dots, v_p) \mapsto v_{\sigma(1)} \otimes
  \dots \otimes v_{\sigma(p)}\). Then the following diagram commutes
  \[
    \begin{tikzcd}
      \prod_{i=1}^p V_i \ar[r, "g_\sigma"] \ar[d, swap, "t"]
        &\bigotimes_{i=1}^p V_{\sigma(i)}\\
      \bigotimes_{i=1}^p V_i \ar[ur, swap, "f_\sigma"]
    \end{tikzcd}
  \]
  Now, by means of \cref{thm: universal property of tensor products}, we find
  that the morphism \(f_\sigma\) is unique. Notice that \(f_\sigma\) maps the
  base of \(\bigotimes_{i=1}^p V_i\) to the base of \(\bigotimes_{i=1}^p
  V_{\sigma(i)}\), hence \(f_\sigma\) is an isomorphism.
\end{proof}

\subsection{Duality}

\begin{proposition}\label{prop: dual tensor isomorphism}
  Let \(\{V_i\}_{i=1}^p\) be a collection of finite dimensional \(k\)-vector
  spaces. Then the \(k\)-linear morphism
  \[
    \bigoplus_{i=1}^p V_i^* \to \left( \bigoplus_{i=1}^p V_i \right)^* :
    f_1 \otimes \dots \otimes f_p \mapsto (v_1 \otimes \dots \otimes v_p \mapsto
    f_1(v_1) \dots f_p(v_p))
  \] 
  is a natural isomorphism.
\end{proposition}

\begin{proof}
  Since \(V_i\) is finite dimensional for all \(i\), then \(V_i \iso V_i^*\),
  which in particular yield \(\dim_k(V_i) = \dim_k(V_i^*)\) (\cref{prop: finite
  dim vs iso dual vs}), then \(\dim_k(\bigoplus_i V_i^*) = \dim_k(\bigoplus_i
  V_i)\). Now, since \(\bigotimes_i V_i\) is finite, then \(\bigotimes_i V_i
  \iso (\bigotimes_i V_i)^*\), which implies in \(\dim_k(\bigotimes_i V_i) =
  \dim_k(\bigotimes_i V_i)^*\). From this we find that \(\dim_k(\bigoplus V_i^*)
  = \dim_k(\bigoplus_i V_i)^*\). From \cref{cor: equal dim - iso conditions} it
  suffices to show that the map is surjective or injective. Let \(f_1 \otimes
  \dots \otimes f_p \neq 0\), then its image maps to \(f_1(v_1) \dots
  f_p(v_p)\), which cannot be zero if \(v_1 \otimes \dots \otimes v_p \neq 0\).
  This implies that the kernel of the \(k\)-linear map shown is zero, hence it's
  an isomorphism.
\end{proof}

\begin{proposition}\label{prop: tensor with dual iso hom}
  Let \(V, L\) be finite dimensional \(k\)-vector spaces. Then the \(k\)-linear
  morphism
  \[
    V^* \otimes L \to \Hom(V, L) : f \otimes \ell \mapsto (v \mapsto f(v)\ell)
  \] 
  is a canonical isomorphism.
\end{proposition}

\begin{proof}
  Name the above morphism \(\phi\). Let \(\dim_k(V) = n\) and \(\dim_k(L) = m\).
  Let basis \(\{v_j\}_{j=1}^n\) and \(\{\ell_i\}_{i=1}^m\) of \(V\) and \(L\),
  respectively. Then we find the corresponding dual basis \(\{v_j^*\}_{j=1}^n\)
  and \(\{\ell_i^*\}_{i=1}^m\). Notice that \(\phi :v_j^* \otimes \ell_i \mapsto
  (v \mapsto v_j^*(v) \ell_i)\), hence, given \(g \in \im(\phi) \subseteq
  \Hom(V, L)\) we can write its matrix representation \(k^n \to k^m\) with
  factors \(a_{ij}\) defined by 
  \[
    g(v_k) = \sum_{i=1}^m a_{ik} \ell_i = v_j^*(v_k)\ell_i = 
    \begin{cases}
      \ell_i, &k = j\\
      0, &\text{otherwise}
    \end{cases}
  \]
  for some \((v_j^*, \ell_i) \in \{v_j^*\}_{j=1}^m \times \{\ell_i\}_{i=1}^m\),
  so that \(a_{ik} = 0\) for all \(k \neq j\) and \(a_{ij} = 1\). Notice that
  this makes \(\im(\phi)\) a basis for \(\Hom(V, L)\), transforming a basis into
  other, which qualifies \(\phi\) as an isomorphism.
\end{proof}

Notice that if \(V\) is finite dimensional \(k\)-vector space, we can consider
the special case of the endomorphism \(\End(V) \iso V^* \otimes V\). Notice
that \(\Id_V \in \End(V)\) is such that, given \(v_j \in \{v_i\}_{i=1}^n\),
where the last is a basis for \(V\), we have \(\Id_V(v_j) = \sum_{i=1}^n
\delta_{ij} v_i = \sum_{i=1}^n v_i^*(v_j) v_i\). This implies in the mapping
\(\Id_V \mapsto \sum_{i=1}^n v_i^* \otimes v_i\) for the canonical isomorphism.

\begin{definition}[Trace]\label{def: trace}
  Given a finite \(k\)-vector space \(V\), we define the canonical linear
  functional
  \[
    \Tr: V^* \otimes V \to k,\ \alpha \otimes v \mapsto
    \alpha(v) 
  \] 
  Since \(V^* \otimes V \iso \End(V)\), then we can view the above definition in
  terms of endomorphisms, so that \(\Tr: \End(V) \to k\). For a more
  computational interpretation, let \(\dim_k V = n\) and \(\{v_i\}_{i=1}^n\) be
  a basis for \(V\). Given a linear endomorphism \(f \in \End(V)\), suppose
  that it's coefficients in the matrix representation are \(a_{ij}\), so that
  for any \(v_k \in \{v_i\}_{i=1}^n\) we have \(f(v_k) = \sum_{i=1}^n a_{ij} v_i
  = \sum_{i,j=1}^n a_{ij} v_j^*(v_k) v_i\). From the canonical map described in
  \cref{prop: tensor with dual iso hom} we find that \(f \mapsto \sum_{i,j=1}^n
  a_{ij} v_j^* \otimes v_i\). Now, from the definition of the trace it follows
  that
  \[
    f \mapsto \sum_{i,j=1}^n a_{ij} v_j^* \otimes v_i \xmapsto{\Tr}
    \sum_{i=1}^n a_{ii}.
  \] 
  As one can note, this is the sum of the diagonal of the matrix representation
  of \(f\) (which happens to be independent of the base, as can be asserted from
  the more abstract definition).
\end{definition}

\begin{corollary}
  Given finite dimensional vector spaces \(V, L, W\) we have that
  \[
    \Hom(V \otimes L, W) \iso \Hom(V, \Hom(L, W)).
  \] 
\end{corollary}

\begin{proof}
  Notice that from \cref{prop: tensor with dual iso hom} we find
  \begin{align*}
    \Hom(V \otimes L, W) \iso (V \otimes L)^* \otimes W 
                         &\iso (V^* \otimes L^*) \otimes W \\
                         &\iso V^* \otimes (L^* \otimes W) \\
                         &\iso V^* \otimes \Hom(L, W) \\
                         &\iso \Hom(V, \Hom(L, W))
  \end{align*}
\end{proof}

\begin{corollary}[Tensor product of morphisms]
  Let \(\{V_i\}_{i \in I}\) and \(\{L_i\}_{i \in I}\) be a finite collection of
  finite \(k\)-vector spaces. Then the \(k\)-linear morphism
  \[
    \bigotimes_{i \in I} \Hom(V_i, L_i) \to
    \Hom\bigg( \bigotimes_{i \in I} V_i, \bigotimes_{i \in I} L_i \bigg),\
    \otimes_{i \in I} f_i \mapsto (\otimes_{i \in I} v_i \mapsto \otimes_{i \in
    I} f_i(v_i))
  \]
  is a canonical isomorphism.
\end{corollary}

\begin{proof}
  First notice that
  \begin{align*}
    \Hom\bigg( \bigotimes_i V_i, \bigotimes_i L_i \bigg)
    \iso \bigg(\bigotimes_i V_i \bigg)^* \otimes \bigg( \bigotimes_i L_i \bigg)
    &\iso \bigg( \bigotimes_i V_i^* \bigg) 
    \otimes \bigg( \bigotimes_i L_i \bigg) \\
    &\iso \bigotimes_i V_i^* \otimes L_i \\
    &\iso \bigotimes_i \Hom(V_i, L_i)
  \end{align*}
  from propositions \ref{prop: tensor with dual iso hom}, and \ref{prop: dual
  tensor isomorphism}, and \ref{prop: commutativity tensor prod}. Moreover,
  the map is clearly injective and surjective, hence an isomorphism.
\end{proof}

\subsection{Contractions, and Raising (Lowering) of Indices}

\begin{definition}[Contraction]
  Let \(\{V_i\}_{i \in I}\) be a finite collection of finite \(k\)-vector spaces
  such that for some \(k, j \in I\) we have \(V_k = V\) and \(V_j = V^*\). We
  define the contraction of the tensor product \(\bigotimes_{i \in I} V_i\) as
  the linear mapping
  \[
    \bigotimes_{i \in I} V_i \longrightarrow \bigotimes_{\substack{i \in I \\ i
    \neq i, j}} V_i\ \text{ mapping }\
    \otimes_{i \in I} v_i \longmapsto 
    v_j^*(v_k) (\otimes_{\substack{i \in I\\ i \neq j, k}} v_i)
  \]
  where \(v_j^*(v_k) = \delta_{jk}\).
\end{definition}

\begin{definition}[Raising and Lowering]
  Let \(\{V_i\}_{i=1}^p\) be a finite collection of finite \(k\)-vector spaces
  and \(g: V_i \to V_i^*\) be an isomorphism. Then we define the lowering of the
  index \(i\) as the linear morphism 
  \[
    \Id \otimes \dots \otimes g \otimes \dots \otimes \Id:
    \bigotimes_{i=1}^p V_i \longrightarrow
    V_1 \otimes \dots \otimes V_i^* \otimes \dots \otimes V_p
  \] 
  Moreover, the raising of the index \(i\) is just defined as the inverse of the
  above linear morphism.
\end{definition}

\subsection{Tensor Multiplication Functor}

\begin{definition}[Tensor multiplication functor]
  Let \(\cat{FinVect}_k\) be the category of finite \(k\)-vector spaces together
  with linear morphisms between them. Given \(M \in \cat{FinVect}_k\), we define
  the functor of tensor multiplication on \(M\) as the mapping of objects \(L
  \xmapsto F L \otimes M\) and the mapping of morphisms \(f \xmapsto F f \otimes
  \Id_M\). Hence we have \(\Id_L \mapsto \Id_L \otimes \Id_M = \Id_{L \otimes
  M}\) and given composable morphisms \(f, g \in \Mor(\cat{FinVect}_k)\) we have
  \(f \circ g \mapsto (f \circ g) \otimes \Id_M = (f \otimes \Id_M) \circ (g
  \otimes \Id_M)\) as wanted.
\end{definition}

\begin{proposition}[Exactness]
  Let \(0 \to V \overset{f}\hookrightarrow L \overset{g}\twoheadrightarrow W \to
  0\) be a short exact sequence, where \(V, L, W \in \cat{FinVect}_k\). Let \(M
  \in \cat{FinVect}_k\), then following sequence is exact
  \[
    0 \to V \otimes M \xrightarrow{f \otimes \Id_M}
    L \otimes M \xrightarrow{g \otimes \Id_M} W \otimes W \to 0.
  \] 
\end{proposition}

\begin{proof}
  (\(f \otimes \Id_M\) is injective) Let \(v \otimes m \in V \otimes M\) be a
  nonzero factorizable tensor. Then in particular \(f(v) \neq 0\), since \(f\)
  is injective, then \(f(v) \otimes m\) is also nonzero, since \(m \neq 0\),
  hence \(\ker(f \otimes \Id_M) = 0\). (\(g \otimes \Id_M\) is surjective) Let
  \(w \otimes m \in W \otimes M\) be any factorizable tensor, then in particular
  exists \(\ell \in L\) such that \(g(\ell) = w\), from the fact that \(g\) is
  surjective. Hence we find that \(\ell \otimes m \xmapsto{g \otimes \Id_M}
  g(\ell) \otimes m = w \otimes m\), since the collection of factorizable
  tensors form a base for the tensor product, we can conclude that \(g \otimes
  \Id_M\) is surjective. (\(\im(f \otimes \Id_M) = \ker(g \otimes \Id_M)\))
  Suppose \(\ell \otimes m \in \im(f \otimes \Id_M)\), then in particular we
  have \(\ell \in \im f\) and hence \(\ell \in \ker g\) since \(\im f = \ker
  g\). This implies in \(\ell \otimes m \in \ker(g \otimes \Id_M)\) and hence
  \(\im(f \otimes \Id_M) \subseteq \ker(g \otimes \Id_M)\). Take now any tensor
  \(\ell' \otimes m' \in \ker(g \otimes m)\) if \(\ell' \otimes m' = 0\) then
  clearly  \(\ell' \otimes m' \in \im(f \otimes \Id_M)\), suppose on the
  contrary that \(\ell' \otimes m' \neq 0\), then certainly \(\ell' \in \ker g\)
  and in particular \(\ell' \in \im f\). Then it follows that \(\ell' \otimes m'
  \in \im(f \otimes \Id_M)\) and hence \(\ker(g \otimes \Id_M) \subseteq \im(f
  \otimes \Id_M)\). This finishes the proof.
\end{proof}

\section{Tensor Algebra}

\begin{definition}[Mixed tensor product]
  Let \(V\) be a finite dimensional \(k\)-vector space. We define the
  \(p\)-covariant and \(q\)-contravariant mixed tensor product on \(V\) as
  \[
    T_p^q(V) = V^{* \otimes p} \otimes V^{\otimes q}
  \] 
  Elements of such object are called tensors of type \((p, q)\) and rank \(p +
  q\) on \(V\). We define \(T_0^0(V) = k\).
\end{definition}

\begin{example}
  The following examples illustrate some specific type of mixed tensor product
  on \(V\), showing that such a construction generalizes various objects in
  linear algebra.
  \begin{enumerate}[(a)]
    \item Tensors of type \((0,0)\) are called scalar tensors of rank \(0\). 
    \item Tensors of type \((1, 0)\) are linear functionals on \(V\).
    \item Tensors of type \((1, 1)\) are elements of \(V^* \otimes V \iso
      \End_{\cat{Vect}_k}(V)\), that is, linear operators.
    \item Tensors of type \((2, 0)\) are elements of \(V^* \otimes V^* \iso
      \Hom(V^{**}, V^*) \iso \Hom(V, V^*)\) for a finite dimensional \(V\).
      Moreover \(V^* \otimes V^* \iso (V \otimes V)^* \iso \Hom(V, V; k)\) of
      multilinear maps \(V \times V \to k\), that is, the mixed tensors of type
      \((2, 0)\) are inner products.
 \end{enumerate} 
\end{example}

\begin{definition}[Mixed tensor multiplication]
  Let \(V\) be a finite \(k\)-vector space, then \(T_p^q(V) = V^{*\otimes p}
  \otimes V^{\otimes q} \iso (V^{\otimes p} \otimes V^{* \otimes q})^*\), which
  in turn is isomorphic to the space of multilinear maps \(V^p \times V^{*q} \to
  k\). Let the multilinear maps \(f: V^p \times V^{*q} \to k\) and \(g:
  V^{p'} \times v^{*q'} \to k\), then we define their tensor multiplication as
  \[
    f \otimes g: V^{p + p'} \times V^{*(q + q')} \to k
  \] 
  mapping
  \[
    (v_1, \dots, v_p, v_1', \dots, v_{p'}', \widetilde v_1^*, \dots, \widetilde
    v_p^*, \widetilde v_1^{'*}, \dots, \widetilde v_{q'}^{'*})
    \xmapsto{f \otimes g}
    f(v_1, \dots, v_p, \widetilde v_1^*, \dots, \widetilde v_p^*)
    g(v_1', \dots, v_{p'}', \widetilde v_1^{'*}, \dots, \widetilde v_{q'}^{'*})
  \] 
  where \(v_i, v_i' \in V\) and \(\widetilde v_j^*, \widetilde v_j^{'*} \in V^*\). This shows
  that clearly this tensor multiplication is in gereral non-commutative.
  However, it is 
  \begin{enumerate}[(i).]
    \item (Bilinear) For all \(a, b \in k\), then \((a f_1 + b f_2) \otimes g =
      a (f_1 \otimes g) + b (f_2 \otimes g)\) and \(f \otimes (a g_1 + b g_2) =
      a (f \otimes g_1) + b (f \otimes g_2)\).
    \item (Associative) \((f \otimes g) \otimes h = f \otimes (g \otimes h)\).
  \end{enumerate}
\end{definition}

\begin{definition}[Tensor algebra]\label{def: tensor algebra vs}
  We define the tensor algebra of the \(k\)-vector space \(V\) to be the
  infinite dimensional \(k\)-vector space
  \[
    T(V) = \bigotimes_{p, q = 1}^\infty T_p^q(V).
  \] 
\end{definition}

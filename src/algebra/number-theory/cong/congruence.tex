\subsection{Congruence \(\texorpdfstring{ax \equiv b \Mod{m}}{ax = b (mod m)}\)}

\begin{definition}
    We define the number of solutions to a given congruence, say
    \(p(x_1, \dots, x_n) \equiv k \Mod{n}\), to be the number of solutions, of
    the form \((a_j)_{j=1}^n\), that are inequivalent up to congruence modulo
    \(n\). Therefore, if \((a_j)_{j=1}^n\) and \((b_j)_{j = 1}^{n}\) are
    solutions to the congruence but \(a_j \equiv b_j \Mod{n},\ \forall j \in [1,
        n]\) then we say that these solutions are equivalent and thus only count as
    one solution to the congruence.
\end{definition}

\begin{proposition}
    For the special case where \((a, m) = 1\) we have that \(\overline{a} \in
    (\mathbb{Z}/m\mathbb{Z})^\ast\) then a solution to \(ax \equiv b \Mod{m}\) is
    \(x \equiv a^{\varphi(m) - 1} b \Mod{m}\). Also, all solutions have the form
    \(a^{\varphi(m)-1} + km,\ k \in \mathbb{Z}\).
\end{proposition}

\begin{proof}
    Notice that \(\overline{a^{\varphi(m)}} = \overline{1}\) thus if we just
    multiply the classes \(\overline{a} \overline{a^{\varphi(m) - 1}} = 1\), then
    \(a (a^{\varphi(m) - 1} b) \equiv b \Mod{m}\). Now for the solution forms,
    notice that if we make \(k \in \mathbb{Z}\) then \(a (a^{\varphi(m) - 1}b +
    km) \equiv b + km \equiv b \Mod{m}\).
\end{proof}

\begin{proposition}
    For the general case where \((a,m) = d > 1\), the linear congruence \(ax
    \equiv b \Mod{m}\) has solutions if and only if \(d \mid b\) and, if so,
    there are \(d\) solutions to such system.  Let \(x_0\) be a solution, then
    all the other solutions to the congruence have the form \(x_0 + n (m/d)\)
    where \(0 \leqslant n \leqslant d-1\) is an integer.
\end{proposition}

\begin{proof}
    (\(\Leftarrow\)) Let \(x_0\) be a solution, then \(ax_0 \equiv b \Mod{m} \iff
    \exists y_0 \in \mathbb{Z} : my_0 = ax_0 - b\) therefore \(b = ax_0 -
    my_0\) which from the fact that \((a, m) = (d)\) (so that the ideals
    generated by \(a, m\) and  \(d\) are the same) implies that \(b \mid (a,
    m) = d\).

    (\(\Rightarrow\)) Suppose that \(b \mid d\) then from \((a, m) = (d)\) we
    have \(\exists x, y \in \mathbb{Z} : d = ax + my\), which implies in \(my = d
    - ax\), let \(c := b/d\) then multiplying the equation by \(c\) yields
    \(m(yc) = b - a(xc)\) and thus \(a(xc) \equiv b \Mod{m}\) where \(xc\) is a
    solution.

    Suppose \(x_0, x_1\) are solutions for  \(ax \equiv b \Mod{m}\), also, write
    \(a := a'd\) and  \(m := m'd\) where \(d \nmid a\) and \(d \nmid m'\), then
    we have \(ax_0 - ax_1 \equiv 0 \Mod{m}\) thus \(m'd \mid a'd(x_0 - x_1)
    \Rightarrow m' \mid a'(x_0 - x_1)\), but since \((a', m') = 1\) from
    construction, then necessarily \(m' \mid x_0 - x_1\) and therefore \(\exists
    q \in \mathbb{Z} : m'q = x_0 - x_1 \Rightarrow x_0 = m'q + x_1\) and also
    \(a(m'q) \equiv 0 \Mod{m}\) and  \(ax_1 \equiv b \Mod{m}\) thus \(a(m'q +
    x_1) \equiv b \Mod{m}\) as wanted. We show now that the solutions proposed
    are inequivalent: let \(q_1, q_2 \in [1, d-1]\) integers, then we have that
    \((m'q_1 + x_1) - (m'q_2 + x_1) = m' (q_1 - q_2)\) thus we have that \(m =
    m'd \nmid m'(q_1 - q_2)\) since  \(d \nmid q_1 - q_2\) and \(d \nmid m'\)
    therefore, the solutions \(x = m'q_1 + x_1\) and \(x' = m'q_2 + x_1\) are
    inequivalent modulo \(m\).
\end{proof}

\begin{proposition}
    Let \(p\) be a prime number, then  \(ax \equiv b \Mod{p}\) has a unique
    solution of the form \(x \equiv a^{p-2} b \Mod{m}\).
\end{proposition}

\subsection{Chinese Remainder Theorem}

\begin{lemma}\label{ChiRemLem1}
    If \(\{a_1, \dots, a_\ell\} \subseteq \mathbb{Z}\) are such that \((a_i, m) =
    1\) for all \(1 \leqslant i \leqslant \ell\) then \((\prod_i a_i, m) = 1\).
\end{lemma}

\begin{proof}
    Since \((a_i, m) = 1\) then  \(\overline{a_i} \in
    (\mathbb{Z}/m\mathbb{Z})^\ast\). Since \(((\mathbb{Z}/m\mathbb{Z})^\ast,
    \cdot)\) forms a commutative group then \(\prod_{i} \overline{a_i} =
    \overline{\prod_i a_i} \in (\mathbb{Z}/m\mathbb{Z})^\ast\), which proves that
    \((\prod_i a_i, m) = 1\).
\end{proof}

\begin{lemma}\label{ChiRemLem2}
    Given the set \(\{a_1, \dots, a_\ell\} \subseteq \mathbb{Z}\) and \(n \in
    \mathbb{Z}\), let \(a_i \mid n,\ \forall i \in [1, \ell]\) and \((a_i, a_j) =
    1\) for all pair of different indices. Then  \(\prod_i a_i \mid n\).
\end{lemma}

\begin{proof}
    We prove via induction. For the case \(\ell = 1\) is trivial; now suppose
    that for some \(\ell-1 \in \mathbb{Z}_{>1}\) we have
    \(\prod_{i=1}^{\ell-1}a_i \mid n\); then, for the case \(\ell\) we have that
    \((a_\ell, \prod_{i=1}^{\ell-1} a_i)=1\), thus there are \(u, v \in
    \mathbb{Z} : u a_\ell + v\prod_{i=1}^{\ell-1}a_i = 1\) if we multiply by
    \(n\) we find \((un) a_\ell + (vn) \prod_{i=1}^{\ell-1} a_i = n\), what we
    know from the induction hypothesis is that \(\prod_{i=1}^{\ell-1}a_i \mid n\)
    but also \(a_\ell \mid a_\ell\) from the reflexive property of division thus
    \(\prod_{i=1}^\ell a_i \mid (un) a_\ell + (vn) \prod_{i=1}^{\ell-1} a_i\)
    therefore we conclude finally that \(\prod_{i=1}^\ell a_i \mid n\) as wanted.
\end{proof}

\begin{theorem}[Chinese remainder]
    Let \(m = \prod_{j=1}^\ell m_j\) and \((m_j, m_i) = 1\) for all pair of
    different indices. Let now the set \(\{b_1, \dots, b_\ell\} \subseteq
    \mathbb{Z}\) and lets consider the system of congruences
    \[
        x \equiv b_1 \Mod{m_1},\ \dots,\ x \equiv b_\ell \Mod{m_\ell}
    \]
    This system of congruences \emph{always has solutions} and any two solutions
    differ by a multiple of \(m\).
\end{theorem}

\begin{proof}
    For all \(1 \leqslant j \leqslant \ell\) let \(n_j := m/m_j =
    \prod_{\substack{k = 1 \\ k \neq j}}^{\ell} m_k\). From the fact that \((m_k,
    m_j) = 1\) for all \(k \neq j\) then from lemma \ref{ChiRemLem1} we have that
    \((\prod_{\substack{k=1 \\ k \neq j}}^\ell m_k, m_j) = (n_j, m_j) = 1\).

    Since \((n_j, m_j) = 1\) there are  \(y_j, z_j \in \mathbb{Z}\) such that
    \(n_jz_j + m_jy_j = 1 \implies m_j \mid 1 - n_j z_j\) and thus \(n_jz_j
    \equiv 1 \Mod{m_j}\). Also is worth noting that \(\forall k \neq j\) we have
    \(n_jz_j \equiv 0 \Mod{m_k}\) since  \(m_k \mid n_j)\).

    Given the set \(\{b_1, \dots, b_\ell\}\), let \(x_0 := \sum_{j=1}^\ell b_j
    (n_j z_j)\), then, for all choices of \(j \in [1, \ell]\) we have \(x_0 =
    \sum_{j=1}^\ell b_j(n_jz_j) \equiv b_j(n_jz_j) \equiv b_j \Mod{m_j}\), from
    the congruences of the last paragraph. Thus, since \(x_0 \equiv b_j
    \Mod{m_j}\) for all indices  \(j\) then we say that  \(x_0\) is a solution to
    the system.

    Now we prove the equivalences of the solutions up to congruence modulo \(m\).
    Let \(x_1\) be a solution, thus, for all  \(j \in [1, \ell]\) we have \(x_1 -
    x_0 \equiv 0 \Mod{m_j} \implies m_j \mid x_1 - x_0\). We now invoke lemma
    \ref{ChiRemLem2} and conclude that \(\prod_{j=1}^{\ell} m_j = m \mid x_1 -
    x_0\) and thus \(x_1 \equiv x_1 \Mod{m}\) as wanted.
\end{proof}

\todo[inline]{Add further ring-theoretic results}

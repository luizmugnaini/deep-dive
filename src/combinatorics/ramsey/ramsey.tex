\section{Ramsey and Schur theorem}

\begin{theorem}[Ramsey theorem on graphs]\label{ramseyThmGraphs}
Given \(r \in \N_{\geq 1}\) we have that the colouring function
\(\mathcal{C}: \binom{\N}{2} \to [r]\) is such that there exists an
infinite monochromatic subset \(A \subseteq \N\), that is, for all \(a, b
\in A\), we have \(\mathcal{C}(ab) = i \in [r]\).
\end{theorem}

\begin{proof}
Firstly, define the sets \(\Gamma_k(x) \coloneq \{y \in \N \colon \mathcal{C}(x
y) = k \in [r]\}\). We know that for any element \(x \in \N\) the
colouring function \(\mathcal{C}\) provides at most \(r\) of such sets. Notice
that for every vertex \(v_i \in \N\) there are infinitely many edges
connecting to other vertices of \(\N\) this way we conclude that
since \([r]\) partitions \(\N\) only into a finite number of subsets,
by the pigeonhole principle, we can conclude that there is one set, which
we'll denote by \(\Gamma_{k_i}\), that is infinite.

Let now the sequence \(\left( v_n \right)_{n \in \N}\) such that for all
\(n \in \N,\ v_n \in \Gamma_{k_{n-1}}(v_{n-1})\) for some
colour \(k_{n-1} \in [r]\). A trivial result of such construction is that
given any \(n \in N\) we have \(\mathcal{C}(v_n v_{n-1}) = k_{n-1}\). In fact
we can extend such result by simply recalling that since we are always taking
the infinite sets, surely we'll get the sequence
\[
    \left( \Gamma_{k_{n}}(v_n) \right)_{n \in \N} \text{ such that }
    \N \supseteq \Gamma_{k_{1}}(v_1) \supseteq  \Gamma_{k_{2}}(v_2)
    \supseteq \dots
\]
What this means is that given any pair \(\{v_i, v_j\}\) of elements of such
sequence, we have that \(\mathcal{C}(v_i, v_j) = k_{\min \{i, j\}}\). Define
now the sequence \(\left( k_{\min \{i, j\}} \right)_{\{i, j\} \in
\binom{\N}{2}}\), since this is a sequence with infinitely many
elements, we can say that the finite colouring \([r]\) induces a finite
partitioning of such sequence. By the pigeonhole principle we conclude
finally that there exists an infinite monochromatic subsequence. With this we
conclude the proof since this construction is obtained directly from \(\left(
v_n \right)_{n \in \N}\).
\end{proof}

We actually didn't stated the general version of Ramsey theorem, which extends
for the colouring of the set \(\binom{X}{k}\) where \(X\) is a countably infinite
poset and \(k \in \N\).

\begin{definition}[Hypergraph]
An ordered pair \(H = (V, E)\) of sets, where \(V\) are the vertices and
\(E\) the hyperedges, is called an \textbf{hypergraph} if \(E \subseteq
\binom{V}{k}\), thus an edge connect two or more vertices.
\end{definition}

\begin{theorem}[Ramsey theorem on Hypergraphs]
Let \(X\) a countably infinite poset. Then, for all \(k, r \in \N\), the
colouring \(\mathcal{C} : \binom{X}{k} \to [r]\) is such that there exists an
infinite subset \(A \subseteq X\) such that, for all \(a, b \in A\), we
have \(\mathcal{C}(a b) = i \in [r]\).
\end{theorem}

\begin{proof}
We proceed via induction on \(k\). For the case \(k = 1\) we have that
\(\binom{X}{1} = X\) and thus the colouring \(\mathcal{C}\) induces a
partition of the infinite set \(X\) into a finite number of subsets. By the
pigeonhole principle we conclude that there is one of such subsets that is
infinite and also monochromatic.

For the inductive step, let \(\theta \in X\) be the smallest element of \(X\)
(we can do such a thing since we said \(X\) was a poset), let the colouring
\(\mathcal{C} : \binom{X}{k + 1} \to [r]\) and another colouring
\[
    \mathcal{C}_0 : \binom{X \setminus \{\theta\} }{k} \longrightarrow [r]
    \ \text{ such that }
    \mathcal{C}_0(E) \coloneq \mathcal{C}(E \cup \{\theta\} ), \text{ for all }
    E \in \binom{X}{k}.
\]
By the inductive hypothesis we say that there exists an infinite subset \(X_0
\subseteq X\) such that, for all \(E_0 \in \binom{X_0}{k}\), we have
\(\mathcal{C}_0(E_0) = c_0 \in [r]\).

Now let \(X'_0 \coloneq \{x \in X_0 \colon \theta < x\}\) and define \(x_1
\coloneq \min(X_0)\). As before, set the colouring \(\mathcal{C}_1 :
\binom{X'_0}{k} \to [r]\) such that \(\mathcal{C}_1(E) \coloneq \mathcal{C}(E
\cup \{x_1\})\).  Since \(X'_0\) is an infinite set by construction, we know
from the pigeonhole principle that there exists an infinite subset \(X_1
\subseteq X'_0\) such that, for all \(E_1 \in \binom{X_1}{k}\), we have
\(\mathcal{C}_1 (E_1) = c_1 \in [r]\).

We continue constructing a sequence of elements \((\theta, x_1, x_2, \dots)\)
in such a way that we have a corresponding sequence \((c_0, c_1, c_2, \dots)\)
of colours. Note now that we can construct a new and final colouring function
\[
    \mathcal{C}' : \{\theta, x_1, x_2, \dots\} \to [r] \text{ such that }
    \mathcal{C}'(x_j) \coloneq c_j \in [r].
\]
There are countably infinite elements on the domain and only finite colours, the
partitioning is such that the pigeonhole principle is applicable and thus there
is a infinite subset \(A \subseteq \{\theta, x_1, x_2, \dots\} \subseteq X\)
such that for all \(x \in A\) we have \(\mathcal{C}'(x) = c \in [r]\).  This
concludes the proof.
\end{proof}

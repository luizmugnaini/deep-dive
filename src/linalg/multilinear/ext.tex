\section{Alternating Tensors and Exterior Powers}

\begin{definition}[Alternating map]
  \label{def: alternating map}
  Let \(V\) and \(W\) be vector spaces. A multilinear map \(f: V^n \to W\) is
  called alternating if for any \((v_1, \dots, v_n) \in V^n\) such that exists \(i
  < j\) for which \(v_i = v_j\) then
  \[
    f(v_1, \dots, v_n) = 0.
  \]
\end{definition}

\begin{proposition}\label{prop: alternating map property}
  Let \(f: V^n \to W\) be an alternating multilinear map. Then if \(\tau \in
  S_n\) is a transposition, we have
  \[
    f(v_{\tau(1)}, \dots, v_{\tau(n)}) = - f(v_1, \dots, v_n).
  \]
  In general, if \(\sigma \in S_n\) is any permutation, then
  \[
    f(v_{\sigma(1)}, \dots, v_{\sigma(n)}) = \sign(\sigma) f(v_1,
    \dots, v_n).
  \]
\end{proposition}

\begin{proof}
  Let \(\tau\) be a transposition between indices \(i < j\), then consider
  \begin{align*}
    0 = f(v_1, \dots, v_i + v_j, \dots, v_i + v_j, \dots, v_n)
    =\, & f(v_1, \dots, v_i, \dots, v_i + v_j, \dots, v_n) \\
      & + f(v_1, \dots, v_j, \dots, v_i + v_j, \dots, v_n)
  \end{align*}
  therefore we find
  \begin{align*}
    f(v_1, \dots, v_i, \dots, v_i + v_j, \dots, v_n)
    &= - f(v_1, \dots, v_j, \dots, v_i + v_j, \dots, v_n) \\
    f(v_1, \dots, v_i, \dots, v_j, \dots, v_n)
    &= - f(v_1, \dots, v_j, \dots, v_i, \dots, v_n) \\
    f(v_1, \dots, v_n)
    &= f(v_{\tau(1)}, \dots, v_{\tau(n)})
  \end{align*}
  where we used that \(f(v_1, \dots, v_i, \dots, v_i, \dots, v_n) = f(v_1,
  \dots, v_j, \dots, v_j, \dots, v_n) = 0\). This shows the first proposition.
  For the second proposition, by means of \cref{prop: sign is a group
  homomorphism} and \cref{prop: permutations to transpositions} and the above
  proposition for transpositions, we conclude the proof.
\end{proof}

\begin{remark}
  Notice that \cref{prop: alternating map property} is not sufficient to
  characterize alternating maps, it is but a necessary property.
\end{remark}

\subsection{Alternating Tensor and Exterior Powers}

\begin{definition}[Alternating tensor]
  \label{def: alternating tensor}
  Let \(V\) be a \(k\)-vector space we define a tensor \(T \in T_0^q(V)\) to be
  alternating if for all \(\sigma \in S_q\) permutation we have
  \[
    f_\sigma(T) = \sign(\sigma) T.
  \]
  We denote by \(\Lambda^q V\) the subspace of \(T_0^q(V)\) of alternating
  tensors and call it the \(q\)-exterior power of \(V\).
\end{definition}

\begin{proposition}[Alternating tensor projection]
  \label{prop: alternating projection}
  Let \(V\) be a \(k\)-vector space and \(\Char k \nmid q!\). We
  define the linear projection operator \(A: T_0^q(V) \epi T_0^q(V)\) where
  \[
    A(T) = \frac{1}{q!} \sum_{\sigma \in S_q} \sign(\sigma)
    f_\sigma(T)
  \]
  Then \(A^2 = A\) and \(\im A = \Lambda^q V\).
\end{proposition}

\begin{proof}
  First we show that \(\im A \subseteq \Lambda^q V\) (we already know that
  clearly \(\Lambda^q V \subseteq \im A\)). Notice that
  \begin{align*}
    f_\sigma(A(T))
    &= f_\sigma \left( \frac 1 {q!} \sum_{\tau \in S_q}
    \sign(\tau) f_\tau(T) \right) \\
    &= \frac 1 {q!} \sum_{\tau \in S_q} \sign(\tau)
    f_{\sigma \tau}(T) \\
    &= \sign(\sigma) \left( \frac 1 {q!} \sum_{\tau \in \mathcal
    S_q} \sign(\sigma \tau) f_{\sigma \tau}(T) \right) \\
    &= \sign(\sigma) A(T)
  \end{align*}
  where we've used \cref{prop: sign is a group homomorphism}. Hence we conclude
  that \(\im A = \Lambda^q V\). For the second proposition, notice that
  \[
    A^2 = \frac 1 {q!^2} \sum_{\sigma, \tau \in S_q}
    \sign(\sigma \tau) f_{\sigma \tau}
    = \frac 1 {q!} \sum_{\rho \in S_q} \sign(\rho)
    f_\rho = A
  \]
  since any permutation \(\rho \in S_q\) can be represented in \(q!\)
  different ways as a form of a product \(\sigma \tau\). This concludes the
  proof.
\end{proof}

\begin{definition}[Exterior multiplication]
  \label{def: exterior multiplication}
  Let \(V\) be a \(k\)-vector space and \(v_1 \otimes \dots \otimes v_q \in
  T_0^q(V)\). We define the exterior multiplication to be the map
  \[
    A(v_1 \otimes \dots \otimes v_q) \coloneqq v_1 \wedge \dots \wedge v_q
  \]
\end{definition}

\begin{proposition}[Universal property for exterior power]
  \label{prop: exterior power universal property}
  Let \(V\) be a \(k\)-vector space. For all \(k\)-vector spaces \(L\) together
  with an alternating multilinear map \(\mu: V^q \to L\), there exists a unique
  \(k\)-linear map \(\ell: \Lambda^q V \to L\) for which the diagram commutes
  \[
    \begin{tikzcd}
      V^q \ar[d, swap, "\otimes"] \ar[r, "\mu"]  &L \\
      T_0^q(V) \ar[r, swap, two heads, "A"]
      &\Lambda^q V \ar[u, dashed, swap, "\ell"]
    \end{tikzcd}
  \]
\end{proposition}

\begin{proof}
  Since \(\Lambda^q V\) is a subspace of \(T_0^q(V)\), then we can use the
  universal property \cref{thm: universal property of tensor products}.
  %(Uniqueness) Let \(\ell, f: \Lambda^q(V) \to L\) be \(k\)-linear maps such
  %that \(\ell \circ A \circ \otimes = f \circ A \circ \otimes = \mu\). Let
  %\((v_1, \dots, v_q) \in V^q\) be any element, then since \(A \circ \otimes
  %(v_1, \dots, v_q) = v_1 \wedge \dots \wedge v_q\) we conclude that \(f(v_1
  %\wedge \dots \wedge v_q) = \ell(v_1 \wedge \dots \wedge v_q)\) and hence \(f =
  %\ell\).
  %(Existence) We must show that \(\ell\) is indeed a \(k\)-linear map. Notice
  %that making \(\ell \circ A \circ \otimes = \mu\)
\end{proof}

Now we get some geometric motivation behind the algebraic structure of the
exterior power \(\Lambda^q V\). Alternating tensors \(v_1 \wedge \dots \wedge
v_q \in \Lambda^q V\) can be seen as \(q\)-dimensional oriented volume elements,
where by oriented we mean that the transposition of two edges, say \(v_j\) and
\(v_i\), implies in a change up to a minus sign of the value (the sign is what
creates the bridge between orientation and wedge product). Notice that when
there are equal edges, the volume becomes malformed and \(v_1 \wedge \dots
\wedge v_q = 0\).

\begin{proposition}
  Let \(V\) be a \(n\)-dimensional \(k\)-vector space, where
  \(\operatorname{char} k \neq 2\). Define \(\{e_i\}_{i=1}^n\) to be a base for
  \(V\). Then the exterior product \(e_{i_1} \wedge \dots \wedge e_{i_q} = 0\)
  if there exists \(i_a = i_b\) for some \(1 \leq a, b \leq q\).
\end{proposition}

\begin{proof}
  Since \(A\) is an alternating map, we can use \cref{def: alternating map} and
  conclude the proof.
\end{proof}

For the next proposition, we proceed in a similar fashion as in \cref{prop:
basis for symmetric power}.

\begin{proposition}[Exterior power \(\Lambda^q V\) basis]
  \label{prop: exterior power basis}
  Let \(\{e_i\}_{i=1}^n\) be a basis for the \(k\)-vector space \(V\). The
  factorizable tensors
  \[
    A(e_{i_1} \otimes \dots \otimes e_{i_q}) = e_{i_1} \wedge \dots \wedge
    e_{i_q}
  \]
  with \(1 \leq i_1 < \dots < i_q \leq n\) form a basis for the subspace
  \(\Lambda^q V\).
\end{proposition}

\begin{proof}
  Let \(B := \{e_{i_1} \wedge \dots \wedge e_{i_q} : 1 \leq i_1 < \dots < i_q
  \leq n\}\). Since \(\{e_{i_1} \otimes \dots \otimes e_{i_q} : 1 \leq i_1 <
  \dots i_q \leq n\}\) generates \(T_0^q(V)\) (see \cref{lem: tensor basis}),
  then clearly \(B\) does generate the subspace \(\Lambda^q V\). We now show
  that \(B\) is linearly independent.

  Denote by \(\mathcal I := \{I = (i_j)_{j=1}^q : 1 \leq i_1 < \dots < i_q \leq
  n\}\) the set of strictly increasing \(q\)-tuples. For each \(I := (i_1,
  \dots, i_q) \in \mathcal I\) we define the alternating multilinear map \(\mu_I
  : V^q \to k\) as
  \[
    \mu_I(v_1, \dots, v_q) = \sum_{\sigma \in S_q}
    \sign(\sigma) \prod_{j=1}^q e_{i_{\sigma(j)}}^*(v_{i_j})
  \]
  where \(\{e_{i_j}^*\}_{i_j \in I}\) is the dual of \(\{e_{i_j}\}_{i_j
  \in I}\). Using \cref{prop: exterior power universal property} we can
  define a unique linear map \(f_I: \Lambda^q V \to k\) such that the diagram
  commutes
  \[
    \begin{tikzcd}
      V^q \ar[r, "\mu_I"] \ar[d, swap, "A \circ \otimes"] &k \\
      \Lambda^q V \ar[ur, dashed, swap, "f_I"]
    \end{tikzcd}
  \]
  which implies in \(\mu_I(v_1, \dots, v_q) = f_I(v_1 \wedge \dots \wedge v_q)\)
  for all \((v_1, \dots, v_q) \in V^q\).

  Let \(I' = (i_1', \dots, i_q') \in \mathcal I\) such that \(I' \neq I\). From
  the strictly ordering of the indices, we conclude that there must exists some
  \(1 \leq j_0 \leq q\) such that \(i_{j_0}' \neq i_j\) for all \(1 \leq j \leq
  q\). In particular, this implies that
  \[
    f_I(e_{i_1'} \wedge \dots \wedge e_{i_q'}) = \mu_I(e_{i_1'}, \dots,
    e_{i_q'}) = \sum_{\sigma \in S_q} \sign(\sigma)
    \prod_{j=1}^q e_{i_{\sigma(j)}}^*(e_{i_j'}) = 0
  \]
  since \(e_{i_{\sigma(j)}}^*(e_{i_{j_0}}) = 0\) for all permutations \(\sigma
  \in S_q\) and \(1 \leq j \leq q\). On the other hand, we have that
  \[
    f_I(e_{i_1} \wedge \dots \wedge e_{i_q}) = \mu_I(e_{i_1}, \dots, e_{i_q})
    = \sum_{\sigma \in S_q} \sign(\sigma) \prod_{j=1}^q
    e_{i_{\sigma(j)}}^*(e_{i_j})
    = \sign(\Id) \prod_{j=1}^q e_{i_{\Id(j)}}^*(e_{i_j})
    = 1
  \]

  Let \(c_P \in k\) for all \(P \in \mathcal I\) and consider the vanishing
  linear combination
  \begin{equation}\label{eq: vanishing exterior product combination}
    \sum_{P \in \mathcal I} c_P (e_{p_1} \wedge \dots \wedge e_{p_q}) = 0.
  \end{equation}
  Then, if we look at its image under the map \(f_I\) for all \(I \in \mathcal
  I\) we conclude that
  \[
    0 = f_I\left( \sum_{P \in \mathcal I} c_P (e_{p_1} \wedge \dots \wedge
    e_{p_q}) \right) = \sum_{P \in \mathcal I} c_P f_I(e_{p_1} \wedge \dots \wedge
    e_{p_q}) = c_I
  \]
  hence we conclude that the linear combination \cref{eq: vanishing exterior
  product combination} vanishes if and only if each coefficient vanishes. This
  concludes that \(B\) is linearly independent. Hence we've proved that \(B\) is
  a basis for \(\Lambda^q V\).
\end{proof}

\begin{proposition}
  Let \(V\) be a \(n\)-dimensional \(k\)-vector space. Then we have
  \[
    \dim_k \Lambda^q V = \binom n q
  \]
\end{proposition}

\begin{proof}
  Since we want our indices to be strictly increasing, we are left with \(n\)
  elements of which we want to arrange in groups of \(q\) elements. Hence the
  number of possible arrangements is \(\binom n q\) and this concludes the
  proof.
\end{proof}

\subsection{Exterior Algebra}

\begin{definition}[Exterior algebra]
  \label{def: exterior algebra}
  Let \(V\) be a \(k\)-vector space. We define the exterior algebra on \(V\) as
  \[
    \Lambda^\bullet V = \bigoplus_{q=0}^\infty \Lambda^q V
  \]
  together with a multiplicative structure \(\wedge: \Lambda^d V \otimes
  \Lambda^q V \to \Lambda^{d+q} V\) mapping
  \[
    (v_1 \wedge \dots \wedge v_d) \otimes (w_1 \wedge \dots \wedge w_q) \xmapsto
    \wedge v_1 \wedge \dots \wedge v_d \wedge w_1 \wedge \dots \wedge w_q.
  \]
  We interpret \(\Lambda^0 V = k\).
\end{definition}

\begin{proposition}
  The multiplicative structure of the tensor algebra \(\Lambda^\bullet V\) is
  associative and skew-commutative, that is, for all \(\alpha \in \Lambda^d V\)
  and \(\beta \in \Lambda^q V\) we have \(\alpha \wedge \beta = (-1)^{d q} \beta
  \wedge \alpha\).
\end{proposition}

\begin{proof}
  First, we prove that if \(T_1 \in T_0^q(V)\) and \(T_2 \in T_0^d(V)\) then
  \begin{equation}\label{eq: asso goal}
    A(A(T_1) \otimes T_2) = A(T_1 \otimes A(T_2)) =
    A(T_1 \otimes T_2)
  \end{equation} Notice that \[
    A(T_1) \otimes T_2 = \frac 1 {q!} \sum_{\sigma \in S_q}
    \sign(\sigma) f_\sigma(T_1) \otimes T_2
  \]
  therefore we find
  \begin{equation}\label{eq: A(A(T1) T2)}
    A(A(T_1) \otimes T_2) = \frac 1 {q!} \sum_{\sigma \in S_q}
    \sign(\sigma) A(f_\sigma(T_1) \otimes T_2)
  \end{equation}
  Moreover, we can construct an injection of the symmetry groups \(S_q
  \mono S_{q + d}\) via the mapping \(\sigma \mapsto \overline \sigma\)
  where \(\overline \sigma(i) = \sigma(i)\) for all \(i \in \{1, \dots, q\}\)
  and \(\overline \sigma(i) = i\) for all \(i \in \{q + 1, \dots, q + d\}\). In
  particular, we find that \(f_\sigma(T_1) \otimes T_2 = f_{\overline \sigma}
  (T_1 \otimes T_2)\) and clearly \(\sign(\overline \sigma) = \sign(\sigma)\).
  This implies that
  \begin{equation}\label{eq: ol sigma and A commute}
    A(f_\sigma(T_1) \otimes T_2) = A(f_{\overline \sigma}(T_1 \otimes T_2))
    = f_{\overline \sigma}(A(T_1 \otimes T_2))
    = \sign(\overline \sigma) A(T_1 \otimes T_2)
    = \sign(\sigma) A(T_1 \otimes T_2)
  \end{equation}

  If we substitute \cref{eq: ol sigma and A commute} in \cref{eq: A(A(T1) T2)}
  we find
  \[
    A(A(T_1) \otimes T_2) = \frac 1 {q!} \sum_{\sigma \in S_q}
    \sign^2(\sigma) A(T_1 \otimes T_2) = A(T_1 \otimes T_2).
  \]
  In the same manner we can show that \(A(T_1 \otimes A(T_2)) = A(T_1 \otimes
  T_2)\). Hence \cref{eq: asso goal} holds.

  Let \(\alpha \in \Lambda^d V, \beta \in \Lambda^q V, \gamma \in \Lambda^p V\).
  Notice that from the construction of the product map \(\wedge\) and \cref{eq:
  asso goal} we find
  \[
    (\alpha \wedge \beta) \wedge \gamma
    = \wedge\left( \wedge(\alpha \otimes \beta) \otimes \gamma \right)
    = \wedge\left(\alpha \otimes \wedge(\beta \otimes \gamma) \right)
    = (\alpha \wedge \beta) \wedge \gamma
  \]
  which proves associativity of the tensor algebra.

  We now prove skew-commutativity. Let \(\alpha \in \Lambda^d V\) and \(\beta
  \in \Lambda^q V\), and a permutation \(\sigma \in S_{q + d}\) such
  that \(\sigma(i) = q + d - i + 1\), consisting of \(d q\) transpositions.
  Hence we find that
  \[
    \beta \wedge \alpha = f_\sigma(\alpha \wedge \beta)
    = \sign(\sigma)(\alpha \wedge \beta) = (-1)^{d q} (\alpha \wedge \beta).
  \]
\end{proof}

\begin{proposition}\label{prop: li iff wedge nonzero}
  Let \(V\) be a \(k\)-vector space and \(\{v_i\}_{i=1}^d \subseteq V\). The
  vectors \(v_1, \dots, v_d\) are linearly independent if and only if
  \[
    \Lambda^d V \ni v_1 \wedge \dots \wedge v_d \neq 0.
  \]
\end{proposition}

\begin{proof}
  Let \(\{v_i\}_{i=1}^d\) be a set of linearly dependent vectors and choose
  a set of not-all zero scalars \(\{c_i\}_{i=1}^d \subseteq k\) such that
  \(\sum_{i=1}^d c_i v_i = 0\). Choose \(1 \leq j \leq d\) such that \(c_j
  \neq 0\) and write \(v_j = -\sum_{i=1}^d \frac{c_i}{c_j} v_i\). Then we find
  that (recall \(\mathcal M_0\) from our construction of the tensor product)
  \begin{align}
    \nonumber
    v_1 \wedge \dots \wedge v_j \wedge \dots \wedge v_d
    &= v_1 \wedge \dots \wedge \left( - \sum_{i=1}^d \frac{c_i}{c_j} v_i \right)
    \wedge \dots \wedge v_d
    \\
    \nonumber
    &= \left( v_1, \dots, - \sum_{i=1}^d \frac{c_i}{c_j} v_i, \dots, v_d \right)
    + \mathcal M_0 \\
    &= -\sum_{i=1}^d \frac{c_i}{c_j} (v_1, \dots, v_i, \dots, v_d) + \mathcal
    M_0
    \\
    \label{eq: wedge repeated v_i}
    &= - \sum_{i=1}^d \frac{c_i}{c_j} (v_1 \wedge \dots \wedge v_i \wedge \dots
    v_d)
    \\
    \label{eq: wedge zero}
    &= 0
  \end{align}
  Where \cref{eq: wedge zero} comes from the fact that we have a
  repeated \(v_i\) in the wedge product \cref{eq: wedge repeated v_i}.

  Suppose \(\{v_i\}_{i=1}^d\) is a linearly independent set. Then from
  \cref{prop: li to basis} we can build \(B \supseteq \{v_i\}_{i=0}^d\) such
  that \(B\) is a basis for \(V\). From \cref{prop: exterior power basis}, we
  find that \(\mathcal B = \{v_{i_1} \wedge \dots \wedge v_{i_d} : v_{i_j} \in B
  \text{ and } i_1 < \dots < i_d\}\) is a basis for \(\Lambda^d V\). In
  particular, \(v_1 \wedge \dots \wedge v_d \in \mathcal B\), hence necessarily
  \(v_1 \wedge \dots \wedge v_d \neq 0\).
\end{proof}

Therefore one can trivially see that the kernel of the alternating projection
\(A\) is simply the collection of all decomposable tensors \(v_1 \otimes \dots \otimes v_d\)
such that the collection \(\{v_{1}, \dots, v_d\} \subseteq V\) is linerly dependent.

\begin{corollary}\label{cor:kernel-alternating-projection}
  The kernel of the alternating projection \(A: T^d_0(V) \epi \Lambda^d V\) is given
  by the collection \(\{v_{1} \otimes \dots v_d \in T^d_0(V)\}\) such that \(\{v_{1} \otimes
  \dots v_d\}\) is linearly dependent on \(V\).
\end{corollary}

\begin{theorem}\label{thm:iso-measure-volume-ext}
Let \(V\) be a finite dimensional \(k\)-vector space. The map \(\phi: \Lambda^d V^{*}
\isoto (\Lambda^dV)^{*}\) given by the mapping
\[
  f_1 \wedge \dots \wedge f_d \overset{\phi}\longmapsto \left(
    v_1 \wedge \dots \wedge v_d \mapsto
    \sum_{\sigma \in S_d} \sign(\sigma) \prod_{j=1}^d f_j(v_{\sigma(j)})
  \right)
\]
is a \(k\)-linear isomorphism.
\end{theorem}

\begin{proof}
First of all, we show that \(\phi\) is a \(k\)-linear map. For each \(f \in
\Lambda^d V^{*}\), define the map \(\psi_{f} \in (\Lambda^dV)^{*}\) for which \(\phi(f) =
\psi_f\). Notice that if \(a \in k\) and \(f, g \in \Lambda^d(V^{*})\), then \(\phi(f + a g) =
\psi_{f + a g}\). Notice that
\begin{align*}
  \psi_{f + ag}(v_1 \wedge \dots \wedge v_d)
  &= \sum_{\sigma \in S_d} \sign(\sigma) \prod_{j=1}^d (f_{j} + a g_{j})(v_{\sigma(j)}) \\
  &= \sum_{\sigma \in S_d} \sign(\sigma) \prod_{j=1}^d f_j(v_{\sigma(j)}) + a g_j(v_{\sigma(j)}) \\
  &= \sum_{\sigma \in S_d} \sign(\sigma) \prod_{j=1}^d f_j(v_{\sigma(j)})
  + a \sum_{\sigma \in S_d} \sign(\sigma) \prod_{j=1}^d g_j(v_{\sigma(j)}) \\
  &= \psi_f + a \psi_g
\end{align*}
Thus \(\phi(f + ag) = \phi(f) + a \phi(g)\) and therefore \(\phi\) is indeed linear.

Suppose \(\dim V = n\) and let \(\{e_{j}\}_{j=1}^n\) be a base for \(V\). Define
the collection of strictly increasing \(d\)-tuples \(\mathcal I = \{I =
(i_{j})_{j=1}^d : 1 \leq i_1 < \dots < i_d \leq n\}\). From \cref{prop: exterior power
basis}, we know that \(\{e_{i_{1}} \wedge \dots \wedge e_{i_{d}} \in \Lambda^d V: i_j \in I \text{
and } I \in \mathcal I\}\) is a basis for \(\Lambda^d V\). Moreover,
\(\{e_{j}^{*}\}_{j=1}^n\) is a basis for the dual space \(V^{*}\), thus the
collection \(\{e_{i_{1}}^{*} \wedge \dots \wedge e_{i_{d}}^{*} \in \Lambda^d V: i_j \in I \text{
  and } I \in \mathcal I\}\) is a basis for \(\Lambda^d V^{*}\).

Let \(I \in \mathcal I\) be any strictly increasing \(d\)-tuple, and define the
notation \(e_I^{*} \coloneq e_{i_1}^{*} \wedge \dots \wedge e_{i_d}^{*}\), so that
\begin{equation}
  \label{eq:basislambda*}
  e_I^{*} \overset{\phi}\longmapsto
  \psi_{e_I^{*}}(v_1 \wedge \dots \wedge v_d)
  = \sum_{\sigma \in S_{d}} \sign(\sigma) \prod_{j=1}^d e_{i_j}^{*}(v_{\sigma(j)}) =
  \begin{cases}
    1, &\text{if }\ v_1 \wedge \dots \wedge v_d = e_{i_1} \wedge \dots \wedge e_{i_d} \\
    0, &\text{otherwise}
  \end{cases}
\end{equation}
Notice that we mapped a basis of \(\Lambda^d V^{*}\) into a basis of \((\Lambda^d V)^{*}\),
since any map \(\psi \in (\Lambda^d V)^{*}\) can be obtained by a unique linear combination
of maps of the collection \(\{\psi_{e_{I}^{*}} \in (\Lambda^d V)^{*} : I \in \mathcal I\}\)
--- which, together with \cref{eq:basislambda*}, makes it a basis. This shows that
the morphism \(\phi\) stablishes an isomorphism \(\Lambda^dV^{*} \iso (\Lambda^d V)\).
\end{proof}

\todo[inline]{
Add cool propositions for the exterior algebra: the Hodge Star Operator; the
isomorphisms
\begin{gather*}
  \Sym^d(V \oplus W) \iso \bigoplus_{i=0}^d \Sym^i(V) \otimes \Sym^{d-i}(W) \\
  \Lambda^d(V \oplus W) \iso \bigoplus_{i=0}^d \Lambda^i(V) \otimes \Lambda^{d-i}(W)
\end{gather*}
}

\todo[inline]{After creating sections on affine and projective geometry,
introduce grassmanian varieties.}

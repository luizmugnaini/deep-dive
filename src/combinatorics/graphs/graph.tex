\section{General Definition Of a Graph}

\begin{notation}[\(k\)-subsets]
Given a set \(A\), we define a \(k\)-subset --- where \(0 \leq k \leq |A|\)
--- to be the a subset of \(A\) with exactly \(k\) elements. We denote the
collection of all \(k\)-subsets of \(A\) as \([A]^k\).
\end{notation}

\begin{notation}[Range]
For any \(n \in \N\), we denote by \([n]\) the set of natural numbers \(\{0,
1, \dots, n\}\).
\end{notation}

\begin{definition}[General Graph]\label{def: general-graph}
We define a \emph{general graph} \(G = (V, E, d)\) as a collection of disjoint
sets \(V\) and \(E\) --- called, respectively, the vertices and edges of \(G\) ---
together with a map \(d\) that defines the relations of incidence (see
\cref{def: incidence}) between vertices and edges --- that is, it defines the
formation of edges between vertices.
\end{definition}

\begin{definition}[Incidence]\label{def: incidence}
We say that a vertex \(x\) is \emph{incident} with an edge \(e\) if \(x \in
e\). Moreover if \(x \in e\), we say that \(e\) is an edge \emph{at} \(x\).
\end{definition}

In order to ease the the way on which we talk about graphs, I'll introduce some
notation that I judge will be quite appropriate to avoid confusion.

\begin{notation}[Vertices and edges of a graph]
Given a graph \(G\), its collection of vertices and edges are denoted by,
respectively, \(V(G)\) and \(E(G)\).
\end{notation}

\begin{notation}[Existence of an edge]
Let \(G\) be any graph and consider two vertices \(x, y \in V(G)\). We denote
the relation of existence of an edge joining \(x\) and \(y\) by \(E(x, y)\) ---
that is, if \(E(x, y)\) is \texttt{true}, there exists an edge \(e \in E(G)\) such
that \(x\) and \(y\) are its end-vertices, otherwise, if \(E(x, y)\) is
\texttt{false}, then there is no edge joining the two vertices.
\end{notation}

\begin{definition}[Adjacency]
Given vertices \(x, y \in V\), we say that \(x\) and \(y\) are neighbours if
\(E(x, y)\) --- in such case, \(x\) and \(y\) are said to be end-vertices of the
existent edge joining them.

We say that two edges \(e\) and \(g\) are \emph{adjacent} or \emph{neighbours}
if there exists a unique vertex \(x\) common to both of them --- \(x \in e\)
and \(x \in g\).

A set of vertices, or edges, is said to be independent if no pair of elements
is adjacent. An independent set of vertices \(V\) is called stable.
\end{definition}

\begin{definition}[Diagonal set of vertices]
Let \(G = (V, E)\) be a graph. We define the diagonal subset of \(V\) to be
the collection \(\Delta_V = \{(x, x) : x \in V\}\). So that we have
\(V^2 \setminus \Delta_V = \{(x, y) : x \neq y\}\) --- the collection of
ordered pairs of distinct vertices.
\end{definition}

The following definition will be useful when we are talking about graphs in
which there is no concept of direction, that is, an edge joining \(x\) and \(y\)
is exactly the same as the edge joining \(y\) and \(x\) --- that is, edges have no
intrisic orientation. Graph theorists like to think about graphs as being
entities that satisfy such invariance on the order of the vertices --- but, when
it's possible, we shall take the most general approach.

\begin{definition}[Edge invariance]
We define the set \(\langle V \rangle^2\) to be the quotient
\[
  \langle V \rangle^2 = (V^2 \setminus \Delta_V)/_{(x, y) \sim (y, x)}.
\]
\end{definition}

\begin{notation}[Collection of edges]
Given sets \(X\) and \(Y\), we denote the collection of all edges of the form
\(xy\), where \(x \in X\) and \(y \in Y\), by \(E(X, Y)\). The collection of
all edges that a given vertex \(x\) is incident with is denoted \(E(x)\).
\end{notation}

\begin{definition}[Order and size]\label{def:graph-order-size}
Let \(G\) be a graph. We define the \emph{order} of \(G\) as \(|G| =|V(G)|\),
moreover, the \emph{size} of \(G\) is defined as \(\size{G} = |E(G)|\).
\end{definition}

\begin{definition}[Trivial graph]
A graph \(G\) is said to be trivial if \(|G| \leq 1\). In the case of the
\(0\) order, we denote the graph as \(\emptygraph\).
\end{definition}

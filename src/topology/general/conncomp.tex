\section{Connected Spaces}

\begin{definition}[Connected space]
\label{def:connected-space}
Let \(X\) be a topological space. We say that \(X\) is \emph{connected} if and
only if one of the following conditions hold:
\begin{enumerate}[(a)]\setlength\itemsep{0em}
\item The space \(X\) cannot be expressed as the union of two disjoint non-empty
open sets.
\item Every morphism \(f: X \to \{0, 1\}\) is constant --- where \(\{0, 1\}\) is
a space endowed with the discrete topology.
\end{enumerate}
\end{definition}

\begin{corollary}
The two conditions in \cref{def:connected-space} are equivalent.
\end{corollary}

\begin{proof}
Assume \(X\) satisfies the condition (a), and let \(x \in X\) be any
point. Suppose \(f(x) = 0\) (or \(f(x) = 1\), we do not loose generality by
choosing a point of the domain), and consider \(f^{-1}(0) \subseteq X\), which
must be open from the continuity of \(f\), thus \(f^{-1}(0)\) is a neighbourhood
of \(x\) in \(X\). If we suppose, for the sake of contradiction, that there
exists \(y \in X\) such that \(f(y) = 1\), then \(f^{-1}(1)\) is also open and
is a neighbourhood of \(y\) --- notice however that \(f^{-1}(0) \cap f^{-1}(1) =
\emptyset\), thus we arrive at a contradiction, there must be no point \(y\)
with image different than zero.

We prove the counter positive: not (b) implies not (a). Suppose \(f: X
\to \{0, 1\}\) is not constant, so that there exists two points \(x, y \in X\)
such that \(f(x) = 0\) and \(f(y) = 1\) (and therefore \(f\) is surjective),
notice however that, since \(f\) is continous,
\end{proof}

\begin{definition}[Connected components]
\label{def:connected-components}
Let \(X\) be a topological space. Define an equivalence relation \(\sim\) by,
given \(x, y \in X\), we have \(x \sim y\) if and only if there exists a
connected subspace of \(X\) containing both \(x\) and \(y\). The collection of
equivalence classes \(X/{\sim}\) is called \emph{connected components} of \(X\).
\end{definition}

\begin{definition}[Totally disconnected]
\label{def:totally-disconnected}
A space is said to be totally disconnected if the only connected subsets are
singletons.
\end{definition}

\begin{example}
\label{exp:Q-is-totally-disconnected}
The set of rational numbers \(\Q\) is totally disconnected.
\end{example}

\begin{notation}[Standard interval]
From now on, when talking about \emph{paths} and \emph{homotopies}, we shall
reserve the symbol \(I\) to denote the \emph{standard topological interval},
which is defined by
\[
  I \coloneq [0, 1] \subseteq \R.
\]
\end{notation}

\begin{definition}[Paths \& loops]
\label{def:path-loop}
A \emph{path} in a topological space \(X\) is any continuous map \(\gamma: I \to
X\). A \emph{loop} in \(X\) is a continuous map \(\ell: I \to X\) such that
\(\ell(0) = \ell(1)\).
\end{definition}

\begin{definition}[Path connected space]
\label{def:path-connected}
A topological space \(X\) is said to be \emph{path connected} if and only if
for all \(x, y \in X\) there exists a \emph{path} connecting \(x\) and \(y\).
\end{definition}

\begin{proposition}[Path connected equivalence relation]
\label{prop:path-connected-equiv-relation}
There exists an equivalence relation \(\sim\) on the topological \(X\) defined
by: given \(x, y \in X\), we have \(x \sim y\) if and only if there exists a
path in \(X\) connecting \(x\) and \(y\).
\end{proposition}

\begin{proof}
The constant path \(x: I \to X\) given by \(t \mapsto x\) is a path on
\(X\), thus \(x \sim x\). Let \(x \sim y\) and \(\gamma\) be a path from
\(\gamma(0) = x\) to \(\gamma(1) = y\), then we can define a map \(\lambda: [0,
1] \to X\) given by \(\lambda(t) \coloneq \gamma(1 - t)\), so that \(\lambda\)
is both continuous, and \(\lambda(0) = y\) while \(\lambda(1) = x\) --- thus
\(\lambda\) is a path between \(y\) and \(x\), therefore \(y \sim x\). Suppose
now \(y \sim z\) and let \(\eta\) be a path connecting \(y\) to \(z\). We
define a map \(\phi: I \to X\) given by
\[
  \phi(t) \coloneq
  \begin{cases}
    \lambda(2t), &\text{ for } t \in [0, 1/2] \\
    \eta(2t - 1), &\text{ for } t \in [1/2, 1]
  \end{cases}
\]
which is surely continuous and connects both \(x\) and \(z\) --- thus \(x \sim
z\).
\end{proof}

\begin{definition}[Path connected components]
\label{def:path-connected-components}
Let \(X\) be a path connected topological space, and \(\sim\) be the equivalence
relation described in \cref{prop:path-connected-equiv-relation}. The collection
\(X/{\sim}\) is called the \emph{path connected components} of the space
\(X\). We denote the collection of all path components of \(X\) by \(\pi_0(X)\)
--- the collection of homotopy classes between maps \(* \to X\).
\end{definition}

\begin{definition}[\(\pi_0\) functor]
\label{def:pi0-functor}
The concept of connected components of a space induce a covariant functor
\(\pi_0: \Top \to \Set\) defined by mapping objects \(X \mapsto \pi_0 X\), and
morphisms \(f: X \to Y\) to \(\pi_0 f: \pi_0 X \to \pi_0 Y\) --- which is a well
defined map since, given a path component \(P\) of \(X\), the set \(f(X)
\subseteq Y\) is connected and therefore contained in a unique path component of
\(Y\).
\end{definition}

\begin{theorem}[Morphisms preserve connectivity]
\label{thm:morphisms-preserve-connectivity}
Let \(X\) be a (path) connected space and \(f: X \to Y\) be a topological
morphism. Then \(f(X) \subseteq Y\) is (path) connected.
\end{theorem}

\begin{proof}
Suppose that \(f(X)\) is not connected, and let \(g: f(X) \to \{0, 1\}\) be a
non-constant morphism --- in particular, this is equivalent to the condition of
\(gf: X \to \{0, 1\}\) being a non-constant morphism, thus implying in the
non-connectedness of \(X\).

On the other hand, assume now that \(X\) is path connected and let \(u, v \in
f(X)\) --- define \(x, y \in X\) so that \(f(x) = u\) and \(f(y) = v\). Let
\(\gamma: I \to X\) be a path connecting \(x\) and \(y\) --- then \(f\gamma: I
\to f(X)\) is a path connecting \(u\) and \(v\), which proves the proposition.
\end{proof}

\begin{corollary}
\label{cor:connectedness-top-property}
Connectedness and path connectedness are both topological properties.
\end{corollary}

\begin{corollary}
\label{cor:quotient-path-connected-is-path-connected}
The quotient of a (path) connected topological space is (path) connected.
\end{corollary}

\begin{proposition}
\label{prop:connectivity-quotients}
Let \(f: X \epi Y\) be a morphism of topological spaces \(X\) and \(Y\). If
\(Y\) is connected and, for all \(y \in Y\), the fiber \(f^{-1}(y)\) is
connected, then \(X\) is connected.
\end{proposition}

\begin{proof}
Let \(g: X \to \{0, 1\}\) be a morphism. From the connectedness condition on the
fibers of \(f\), it follows that \(g\) must be constant throughout the fibers of
\(f\) --- this implies in the existence of a morphism \(g^{*}: Y \to \{0, 1\}\)
such that \(g = g^{*} f\). Since \(Y\) is connected, \(g^{*}\) must be constant,
therefore the composition \(g^{*} f\) is constant and so is \(g\) --- that is,
\(X\) is connected.
\end{proof}

\begin{proposition}
\label{prop:union-path-connected}
Let \(J\) be an indexing set associated to a collection \(\{X_{j}\}_{j \in J}\)
of (path) connected topological spaces. Define the space \(X \coloneq \bigcup_{j
\in J} X_j\). If the intersection \(\bigcap_{j \in J} X_j\) is non-empty, then
\(X\) is (path) connected.
\end{proposition}

\begin{proof}
Suppose the intersection is non empty and let \(x \in \bigcup_{j \in J}
X_j\) be any point. We split the proposition into the two given cases.

Suppose \(X_j\) is connected for all \(j \in J\). Let \(f: X \to \{0, 1\}\) be a
morphism --- we want to show that it has to be constant. Since \(X_j\) is
connected, it follows tha the restriction \(f|_{X_j}\) has to be constant, and,
since \(x \in X_j\), then \(f(x) = f(y)\) for all \(y \in X_j\). The fact that
this must be true for all \(X_j\) shows that \(f\) must be constant throughout
the whole set \(X\).

Suppose \(X_j\) is path connected for all \(j \in J\) --- that is, from
hypothesis, \(X_j\) is a path connected component of \(X\). We now prove that
the intersection is connected to each of the path connected components. For each
\(j \in J\), choose any point \(y \in X_j\). Since \(x \in X_j\), then \(x \sim
y\) and therefore every point of \(X\) is connected by a path.
\end{proof}

\begin{theorem}
\label{thm:connected-interval-real-space}
In the real space \(\R\), the only connected sets are intervals.
\end{theorem}

\begin{proof}
If \(A\) is not an interval, there must exist a pair \(x, y \in A\) for which
there is \(z \notin A\) with \(x < z < y\). This way, we can write \(A\) as the
union of two disjoint non-empty sets: \(A = [A \cap (-\infty, z)] \cup [A \cap
(z, \infty)]\) --- that is, \(A\) is disconnected.

We now show that intervals are connected. For the sake of contradiction, assume
there exists an interval \(I\) such that there are disjoint and non-empty sets
\(A\) and \(B\) for which \(I = A \cup B\) --- that is, we assume \(I\) is
disconnected. Let \(x, y \in I\) be any two elements with \(x < y\) and \(x \in
A\), while \(y \in B\). By the archimedian principle, valid in \(\R\), since the
set \(C \coloneq [x, y) \cap A\) is both non-empty and bounded above, thus \(C\)
has a supremum, let \(s \coloneq \sup C\) --- where \(x < s \leq y\). Notice
that \(s \notin A\), since if so then \(s\) would not be the supremum of \(C\),
therefore \(s \in B\) --- which also cannot be the case, since \(s\) would not,
again, be the supremum of \(C\). We conclude that the sets \(A\) and \(B\)
cannot be constructed at all, hence there is no disconnected interval in \(\R\).
\end{proof}

\section{Compact Spaces}

\begin{definition}[Compact space]
\label{def:compact-space}
A topological space \(X\) is said to be \emph{compact} if for \emph{every} open
cover there exists a \emph{finite subcover}.
\end{definition}

\begin{proposition}[Image of compact space]
\label{prop:image-of-compact-is-compact}
If \(f: X \to Y\) is a topological morphism and \(X\) is compact, then \(f(X)\)
is compact in \(Y\).
\end{proposition}

\begin{proof}
Let \(\mathcal{C}\) be an open cover of \(f(X)\). Consider the preimage
collection \(\mathcal{U} \coloneq \{f^{-1}(V)\}_{V \in \mathcal{C}}\), which by
construction covers \(X\). Since \(X\) is compact, let \(\mathcal{U}'\) be the
finite subcover given by \(\mathcal{U}\). If we now consider the image
collection \(\mathcal{C}' \coloneq \{f(U)\}_{U \in \mathcal{C}'}\), we find that
\(\mathcal{C}'\) covers \(f(X)\) and is contained in \(\mathcal{C}\), therefore
a finite subcover.
\end{proof}

\begin{corollary}
\label{cor:compactness-topological-invariant}
Compactness is an \emph{invariant} property of topological spaces.
\end{corollary}

\begin{corollary}
\label{cor:quotient-of-compact-space-is-compact}
The quotient of a compact space is compact.
\end{corollary}

\begin{proof}
Since any quotient space is the image of a continuous projection, the
proposition follows from \cref{prop:image-of-compact-is-compact}.
\todo[inline]{prove}
\end{proof}

\begin{proposition}[Closed subset is compact]
\label{prop:closed-subset-compact}
In a compact space any closed subset is compact.
\end{proposition}

\begin{proof}
Let \(X\) be a space and \(A \subseteq X\) any closed subset. If \(\mathcal{U}\)
is an open cover of \(A\), then \(\mathcal{U} \cup \{A^c\}\) is a cover of
\(X\). Moreover, since \(X\) is compact, there exists a finite subcover
\(\mathcal{U}' \subseteq \mathcal{U} \cup \{A^c\}\). In particular, since
\(\mathcal{U}'\) covers \(X\), it also covers \(A\).
\end{proof}

\begin{theorem}[Bolzano-Weierstra{\ss}]
\label{thm:bolzano-weierstrass}
Let \(X\) be a compact space. Any \emph{infinite subset} \(S \subseteq X\) has a
limit point.
\end{theorem}

\begin{proof}
Suppose, for the sake of contradiction, that there exists \(S \subseteq X\),
infinite, with no limit points --- from this hypothesis, any point \(x \in X\)
is not a limit point of \(S\) and is either in or out of \(S\). In the former
case \(x \in S\) there exists a neighbourhood of \(x\), say \(U_x\), for which
\(U_x \cap S = \{x\}\). In the latter case \(x \notin S\), there must exist a
neighbourhood \(U_x\) of \(x\) such that \(S\) and \(U_x\) are disjoint. From
the law of excluded middle, the collection
\(\mathcal{U} \coloneq \{U_x\}_{x \in X}\) is an open cover of \(X\) --- on the
other hand, there exists no finite subcover of \(\mathcal{U}\) since each one
must only intersect \(S\) at a unique point, but \(S\) is infinite, hence the
contradiction. The infinite set \(S\) must therefore contain at least one limit
point.
\end{proof}

\begin{remark}
\label{rem:bolzano-weierstrass}
One should beware that the converse of \cref{thm:bolzano-weierstrass} does not
hold at all. A simple counterexample goes as follows: endow \(\R\) with the
topology given by \(\{\emptyset, \R\}\) and the open intervals
\(\{(x, \infty) \colon x \in \R\}\) --- in such a space \emph{any} set has a
limit point, although the space itself is not compact.
\end{remark}

\subsection{Checking if a Space is Compact}

\begin{definition}[Finite intersection property]
\label{def:finite-intersection-property}
A collection of sets \(\mathcal{A}\) is said to satisfy the
\emph{finite intersection property} if and only if for every finite collection
\(\{A_1, \dots, A_n\} \subseteq \mathcal{A}\), we have
\(\bigcap_{j=1}^n A_j \neq \emptyset\), that is, the intersection is non-empty
\end{definition}

\begin{proposition}
\label{prop:compact-iff-FIP}
A space is compact if and only if every collection of closed subsets satisfying
the finite intersection property has non-empty intersection.
\end{proposition}

\begin{proof}
Let \(X\) be a compact space and \(\mathcal{C}\) be any collection of closed
subsets satisfying the finite intersection property. Suppose, for the sake of
contradiction, that \(\mathcal{C}\) has an empty intersection --- thus the
complement of the intersection covers \(X\). Hence there must exist a finite
collection \(\{A_1, \dots, A_n\} \subseteq \mathcal{C}\) whose corresponding
complement covers \(X\) and therefore
\(\bigcup_{j=1}^n A_j^c = \big( \bigcap_{j=1}^n A_j \big)^c = X\) --- therefore
\(\bigcap_{j=1}^n A_j = \emptyset\), which contradicts the hypothesis that
\(\mathcal{C}\) satisfies the finite intersection property.

Suppose that the latter condition is true. Suppose there exists an open cover
\(\mathcal{U}\) of \(X\) that has no finite subcover. In particular, the
collection \(\mathcal{U}' \coloneq \{U^c : U \in \mathcal{U}\}\) is composed of
closed sets and satisfies the finite intersection property --- thus \(\bigcup_{U
\in \mathcal{U}} U^c = \big( \bigcap_{U \in \mathcal{U}} U \big)^c = X\), which
immediately implies that the intersection of \(\mathcal{U}\) is empty, yielding
a contradiction. Hence \(\mathcal{U}\) must have a finite subcover.
\end{proof}

\begin{definition}[Proper map]
\label{def:proper-Top}
A continuous map \(f: X \to Y\) is said to be \emph{proper} if for all compact
sets \(C \subseteq Y\) the preimage \(f^{-1}(C) \subseteq X\) is compact.
\end{definition}

\begin{theorem}
\label{thm:hausdorff-compact-set-disjoint-neighbourhoods}
Let \(X\) be a Hausdorff space and \(x \in X\) any point. For every compact set
\(K \subseteq X \setminus \{x\}\) there exists two disjoint open sets \(U\) and
\(V\) such that \(x \in U\) and \(K \subseteq V\).
\end{theorem}

\begin{proof}
Since \(X\) is Hausdorff and \(x \notin K\), for any \(k \in K\) there exists
disjoint neighbourhoods \(U_x\) and \(V_k\) of \(x\) and \(k\),
respectively. Let \(\{V_{k}\}_{k \in K}\) be a collection of such neighbourhoods
--- which is also an open cover of \(K\). Since \(K\) is compact, there exists a
finite collection of points such that \(\{V_{k_1}, \dots, V_{k_n}\}\) is a
finite subcover. Defining \(V \coloneq \bigcup_{j=1}^n V_{k_j}\) and
\(U \coloneq \bigcup_{j=1}^n U_{k_j}\), we find that \(K \subseteq V\) and
\(x \in U\).
\end{proof}

\begin{corollary}[Hausdorff closed subsets]
\label{cor:hausdorff-closed-subset-is-compact}
Any \emph{compact} subset of a Hausdorff space is \emph{closed}.
\end{corollary}

\begin{proof}
If \(X\) is Hausdorff and \(C \subseteq X\) is a compact set, let
\(x \in X \setminus C\) be any point. From
\cref{thm:hausdorff-compact-set-disjoint-neighbourhoods} we know the existence
of a neighbourhood \(U\) of \(x\) that is disjoint from \(K\) --- thus
\(U \subseteq X \setminus C\) and hence \(C\) is closed.
\end{proof}

\begin{corollary}[Maps from compact to Hausdorff spaces]
\label{cor:map-compact-to-hausdorff-is-closed}
Let \(X\) be compact and \(Y\) Hausdorff. Every continuous map \(f: X \to Y\) is
\emph{closed} and \emph{proper}. Moreover, the following are consequent
properties:
\begin{enumerate}[(a)]\setlength\itemsep{0em}
\item If \(f\) is injective, then it is an \emph{embedding}.
\item If \(f\) is surjective, then it is a \emph{quotient map}.
\item If \(f\) is bijective, then it is an \emph{isomorphism}.
\end{enumerate}
\end{corollary}

\begin{proof}
We first prove that \(f\) is closed. Let \(C \subseteq X\) be any closed set of
\(X\), which is therefore compact. From \cref{prop:image-of-compact-is-compact}
we find that \(f(C)\) is compact and from
\cref{cor:hausdorff-closed-subset-is-compact} we have that \(f(C)\) is closed.

Now we prove that \(f\) is proper. Let \(K \subseteq Y\) be any compact set and
consider the preimage \(f^{-1}(K)\). Since \(Y\) is Hausdorff, as pointed
before, \(K\) is closed. Now since \(f\) is continuous, the preimage of closed
sets is closed --- hence \(f^{-1}(K)\) is closed. From the fact that \(X\) is
compact, we conclude that \(f^{-1}(K)\) is compact. For the last three
consequences, they come from \cref{prop:map-open-or-closed-properties}.
\end{proof}

\begin{lemma}[Tube lemma]
\label{lem:tube-lemma}
Let \(X\) be any space and \(Y\) be compact. For every \(x \in X\) and open set
\(U \subseteq X \times Y\) containing \(\{x\} \times Y\), there exists a
neighbourhood \(V \subseteq X\) of \(x\) such that \(V \times Y \subseteq U\).
\end{lemma}

\begin{proof}
Since the product of open sets form a basis for the product topology, for every
\(y \in Y\) there exists a neighbourhood \(V \times W \subseteq U\) of
\((x, y)\). Since \(\{x\} \times Y \iso Y\) and \(Y\) is compact, then
\(\{x\} \times Y\) is compact and therefore must exist a finite collection
\(\{V_j \times W_j\}_{j=1}^n\) of open sets of \(X \times Y\) covering
\(\{x\} \times Y\). Then if \(V \coloneq \bigcap_{j=1}^n V_j\) we find that
\(V \times Y \subseteq U\).
\end{proof}

\begin{theorem}[Closed projection]
\label{prop:compact-iff-projection-closed}
Given topological spaces \(X\) and \(Y\), the space \(X\) is compact if and only
if the canonical projection \(\pi: X \times Y \epi Y\) is closed.
\end{theorem}

\begin{proof}
Suppose \(X\) is compact, then if \(C \subseteq X \times Y\) is any closed set,
let \(y \in Y \setminus \pi(C)\) be any point --- we shall show that there
exists a neighbourhood of \(y\) outside of \(\pi(C)\). Consider the open set
\(U \coloneq X \times (Y \setminus \pi(C))\), which certainly contains
\(X \times \{y\}\). From \cref{lem:tube-lemma}, we find a neighbourhood
\(V \subseteq Y\) of \(y\) such that \(X \times V \subseteq U\), that is,
\(V \subseteq Y \setminus \pi(C)\), which settles that \(\pi(C) \subseteq Y\) is
closed.

Suppose now that \(X\) is a space such that \(\pi: X \times Y \epi Y\) is closed
for any space \(Y\). Let \(\mathcal{C}\) be any collection of closed subsets of
\(X\) satisfying the finite intersection property --- we'll show that
\(\bigcap_{C \in \mathcal{C}} C\) is non-empty. Define \(Y\) to be the space to
consisting of the underlying set \(X \cup \{*\}\) for some point \(*\) and the
topology given by \(2^X\) and \(\{C \cup \{*\} \colon C \in \mathcal{C}\}\). Let
\(K \subseteq X \times Y\) be the closure of the diagonal of \(X\) --- that is,
\(K \coloneq \overline{\Delta_X}\). From hypothesis, \(\pi(K) \subseteq Y\) is
closed and from construction \(X \subseteq \pi(K)\).

We now show that \(* \in \pi(K)\). Suppose on the contrary that
\(* \notin \pi(K)\). Since \(\pi(K)\) is closed, we can find a neighbourhood
\(V \subseteq Y \setminus \pi(K)\) of \(*\) --- and therefore
\(V \cap X = \emptyset\). From the construction of the topology of \(Y\), one
could only hope to write \(V\) as the intersection of finitely many sets of the
form \(C \cup \{*\}\) for \(C \in \mathcal{C}\) --- on the other hand, since
\(\mathcal{C}\) satisfies the finite intersection property, for any finite
collection of sets of \(\mathcal{C}\) one can find \(x \in X\) such that
\(x \in C_1 \cap \dots \cap C_n\), therefore
\((C_1 \cup \{*\}) \cap \dots \cap (C_n \cup \{*\})\) still contains a point of
\(X\). This shows that it is impossible to build \(V\) out of such sets ---
hence we obtain a contradiction and thus \(* \in \pi(K)\) and there exists
\((x_0, *) \in K\) for some \(x_0 \in X\).

For every \(C \in \mathcal{C}\), any neighbourhood \(U \times (C \cup \{*\})\)
of \((x_0, *)\) has a non-empty intersection with the diagonal \(\Delta_X\) ---
thus \(C \cap U\) is non-empty. Moreover, \(x_0 \in C\) for all
\(C \in \mathcal{C}\), otherwise, since \(C\) is closed, one can find a
neighbourhood \(U \subseteq X \setminus C\) of \(x_0\), which should not be
possible. Therefore \(x_0 \in \bigcap_{C \in \mathcal{C}} C\) and by
\cref{prop:compact-iff-FIP} we conclude that \(X\) is compact.
\end{proof}

\subsection{Tychonoff Theorem}

\begin{lemma}
\label{lem:tychonoff-theorem-pre-lemma}
Let \(\{X_{j}\}_{j \in J}\) be a collection of topological spaces. For any point
\(x \in \prod_{j \in J} X_j\) and subset \(A \subseteq \prod_{j \in J} X\) of
the product space, we have \(x \in \overline{A}\) if, for every finite \(F
\subseteq J\), we have \(\pi_F(x) \in \overline{\pi_F(A)}\) --- where \(\pi_F:
\prod_{j \in J} X_j \epi \prod_{j \in F} X_j\) is the canonical projection map.
\end{lemma}

\begin{proof}
Suppose, on the contrary, that \(x \notin \overline{A}\) --- then there exists a
neighbourhood \(U \subseteq \prod_{j \in J} X_j\) of \(x\) which is disjoint
from \(A\). From the definition of the product topology, the collection of
preimages \(\pi_j(U_j)\) for open sets \(U_j \subseteq X_j\) form a subbasis of
the product space. In particular, from the basis properties, this allows for the
existence of an open set \(V \coloneq U_{j_1} \times \dots \times U_{j_n}\) such
that, denoting \(F \coloneq \{j_1, \dots, j_n\} \subseteq J\), the basis element
\(\pi^{-1}_F(V)\) of the product space is a neighbourhood of \(x\) and
\(\pi_F^{-1}(V) \subseteq U\). Therefore \(\pi_F^{-1}\) and \(A\) are disjoint,
which is equivalent to \(A \subseteq (\pi_F^{-1}(V))^c = \pi_F^{-1}(V^c)\) ---
that is, \(\pi_F(A) \subseteq V^c\). Since \(V\) is a neighbourhood of
\(\pi_F(x)\) which is disjoint from \(\pi_F(A)\) we conclude, by \cref{prop:
  closure equivalent prop}, that \(\pi_F(x) \notin \overline{\pi_F(A)}\) ---
which proves the lemma.
\end{proof}

\begin{theorem}[Tychonoff]
\label{thm:tychonoff-theorem}
The cartesian product of a \emph{set} of compact topological spaces, endowed
with the product topology, is compact.
\end{theorem}

\begin{proof}
Denote by \(\{X_{\alpha}\}_{\alpha < \kappa}\) a collection of compact spaces
indexed by an ordinal \(\kappa\). We now show via induction on \(\kappa\) that,
for any space \(Y\), the canonical projection
\(Y \times \prod_{\alpha < \kappa} X_{\alpha} \epi Y\) is closed.

For the ease of notation we define
\(X^{\gamma} \coloneq Y \times \prod_{\alpha < \gamma} X_{\alpha}\) for all
\(\gamma \leq \kappa\) --- moreover for \(\lambda \leq \gamma\) we denote by
\(\pi_{\lambda}^{\gamma}: X^{\gamma} \epi X^{\lambda}\) the canonical projection
between such spaces. We also define that if \(C \subseteq X^{\kappa}\) is a
\emph{closed} set, then \(C_{\lambda} \coloneq \overline{\pi_{\lambda}^{\kappa}(C)}\)
--- thus our goal is equivalent of showing that \(\pi_0^{\kappa}(C) = C_0\).

Assume as inductive hypothesis that for all \(x_0 \in C_0\) there exists
\(x_{\lambda} \in C_{\lambda}\) for every \(\lambda < \kappa\) such that, if
\(\lambda < \gamma < \kappa\), then
\[
\pi_{\lambda}^{\gamma}(x_{\gamma}) = x_{\lambda},
\]
and in particular \(\pi_0^{\lambda}(x_{\lambda}) = x_0\). Again, equivalent to
our goal is to show that \(\pi_0^{\kappa}(x_{\kappa}) = x_0\).

If \(\kappa = \lambda + 1\) is a successor ordinal, then the projection
\(\pi_{\lambda}^{\kappa}: X^{\lambda} \times X_{\lambda} \epi X^{\lambda}\) is
closed from the fact that \(X_{\lambda}\) is compact, by
\cref{prop:compact-iff-projection-closed}. In particular,
\(\pi_{\lambda}^{\kappa}(C) \subseteq X^{\lambda}\) is closed and, by the
inductive hypothesis, \(\pi_{\lambda}^{\kappa}(C) = C_{\lambda}\) --- thus there
exists \(x_{\kappa} \in K\) for which
\(\pi_{\lambda}^{\kappa}(x_{\kappa}) = x_{\lambda}\), hence
\[
\pi_0^{\kappa}(x_{\kappa})
= \pi_0^{\lambda} \pi_{\lambda}^{\kappa}(x_k)
= \pi_0^{\lambda}(x_{\lambda})
= x_0,
\]
which was out goal.

Now, if \(\kappa\) is a limit ordinal, then
\(X^{\kappa} = \projlim_{\lambda < \kappa} X^{\lambda}\) together with
transitions maps \(\pi_{\lambda}^{\gamma}\). The limit of the tuple
\((x_{\lambda})_{\lambda < \kappa}\) defines a point
\(x_{\kappa} \in X^\kappa\), we wish to show that \(x_{\kappa} \in C\). For
every finite set of ordinals \(F\) below \(\kappa\) there exists
\(\lambda < \kappa\) above all ordinals of \(F\), therefore
\[
\pi_F(x_{\kappa})
= \pi_F^{\lambda}\pi_{\lambda}^{\kappa}(x_{\kappa})
= \pi_F^{\lambda}(x_{\lambda}).
\]
Moreover \(\pi_F^{\lambda}(x_{\lambda}) \in \pi_F^{\lambda}(C_{\lambda})\),
where from definition \(C_{\lambda} =
\overline{\pi_{\lambda}^{\kappa}(C)}\). Since \(\pi_F^{\lambda}\) is a
continuous map,
\[
\pi_F^{\lambda}(C_{\lambda})
= \pi_F^{\lambda}(\overline{\pi_{\lambda}^{\kappa}(C)})
\subseteq \overline{\pi_F^{\lambda} \pi_{\lambda}^{\kappa}(C)}
= \overline{\pi_F(C)}.
\]
This shows that \(\pi_F(x_{\kappa}) \in \overline{\pi_F(C)}\), which by
\cref{lem:tychonoff-theorem-pre-lemma} shows that \(\pi(C)\) is closed.
\end{proof}

\subsection{Other Kinds of Compactness}

\begin{definition}[Sequentially compact space]
\label{def:sequentially-compact}
A topological space \(X\) is said to be \emph{sequentially compact} if every
sequence of points \((x_j)_j\) in \(X\), there exists a subsequence \((x_j')_j
\subseteq (x_j)_j\) that converges in \(X\).
\end{definition}

\begin{proposition}[Metric space compactness]
\label{prop:metric-space-compactness-equivalences}
Compactness, limit point compactness, and sequential compactness are
\emph{equivalent} in \emph{metric spaces}.
\end{proposition}

\todo[inline]{Prove metric compactness and expand section}

\begin{proposition}
\label{prop:hausdorff-compact-implies-closed}
Let \(X\) be a Hausdorff space. If \(A \subseteq X\) is compact then \(A\) is
closed in \(X\).
\end{proposition}

\begin{proof}
Let \(x_0 \in A'\) be any limit point of \(A\). We suppose, for the sake of
contradiction, that \(x_0 \notin A\). Since \(X\) is Hausdorff, for every \(x
\in A\) we let \(U_x\) be a neighbourhood of \(x\) such that there exists a
neighbourhood of \(x_0\) that is disjoint from \(U_x\). Define now the
collection \(\mathcal{U} \coloneq \{U_x : x \in A\}\) --- thus clearly
\(\mathcal{U}\) covers \(A\) since it contains every point of \(A\), this being
possible because we assumed that \(x_0\) does not belong to \(A\). Since \(A\)
is compact, we let \(\mathcal{C} \coloneq \{U_{x_1}, \dots, U_{x_n}\} \subseteq
\mathcal{U}\) be a finite subcover. Define \(\mathcal{O} \coloneq \{O_1, \dots,
O_n\}\) of neighbourhoods of \(x_0\) such that \(U_{x_j} \cap O_j = \emptyset\)
for all \(1 \leq j \leq n\) --- thus \(U_{x_j} \cap \bigcap_{j=1}^n O_j =
\emptyset\), and therefore \(A \cap \bigcap_{j=1}^n O_j = \emptyset\) which by
\cref{def:limit-point-derived-set} implies that \(x_0\) cannot be a limit point
of \(A\), thus we arrived at a contradiction. Therefore, for \(x_0\) to be a
limit point of \(A\), it must be the case that \(x_0 \in A\), that is, \(A\) is
closed.
\end{proof}

\begin{corollary}
\label{cor:metric-space-compact-implies-bounded-closed}
In any metric space \(X\), if \(A \subseteq X\) is \emph{compact} then \(A\) is
both \emph{bounded} and \emph{closed} in \(X\).
\end{corollary}

\begin{proof}
Since every metric space is Hausdorff, \(A\) is closed by
\cref{prop:hausdorff-compact-implies-closed}. Moreover, if \(x \in A\) is any
point, we find that the collection of open balls \(\mathcal{B}_x \coloneq
\{B_x(n)\}_{n \in \N}\) forms an open cover of \(A\) and, since \(A\) is
compact, there exists a finite subcover of \(\mathcal{B}_x\). Therefore there
exists a maximal \(m \in \N\) for which \(A \subseteq B_x(m)\) --- thus \(A\) is
bounded.
\end{proof}

\begin{remark}
\label{rem:closed-and-bounded-not-compact}
Notice that a bounded and closed set in a metric space does \emph{not} need to
be compact, for instance, consider the space \(\ell^2(\N)\) (see
\cref{exp:p-norms}) and the subset \(A \coloneq \{f_n\}_{n \in \N}\) composed of
sequences \(f_n \in \ell^2(\N)\) such that \(f_n(j) \coloneq \delta_{n j}\) ---
that is, sequences where the only nonzero term is the \(n\)-th one, which equals
\(1\). Clearly \(A\) is bounded since \(\| f_{n} \|_2 = 1\), moreover, \(A\) is
closed because it has no limit points. However, since no subset of \(A\)
contains a limit point of \(A\), we find that \(A\) is not limit point compact
--- thus not compact (see \cref{prop:metric-space-compactness-equivalences})
\end{remark}

\begin{proposition}
\label{prop:convergent-subsequence-implies-compact}
Let \(X\) be a metric space and \(A \subseteq X\) be a subset. Then, \(A\) is
compact if and only if every sequence of points in \(A\) has a convergent
subsequence in \(A\).
\end{proposition}



%%% Local Variables:
%%% mode: latex
%%% TeX-master: "../../../deep-dive"
%%% End:

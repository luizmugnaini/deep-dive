\documentclass[12pt, reqno]{memoir}

\usepackage[english]{babel}

\usepackage{layout}
\usepackage{afterpage}
\usepackage[
  asymmetric,
  textheight     = 673pt,
  marginparsep   = 7pt,
  footskip       = 27pt,
  hoffset        = 0pt,
  paperwidth     = 597pt,
  textwidth      = 452pt,
  marginparwidth = 101pt,
  voffset        = 0pt,
  paperheight    = 845pt,
]{geometry}

\newcommand{\changegeometry}{\newgeometry{includehead,headheight=89pt}%
  \afterpage{\aftergroup\restoregeometry}%
}

% Math stuff: do not mess with the ordering!
\usepackage{mathtools}
\usepackage{amsthm}
\usepackage{amssymb}
\usepackage{stmaryrd}
\usepackage{tikz-cd}

% Font
\usepackage[no-math]{newpxtext}
\usepackage{newpxmath}
% Set arrow tip to that of newpxmath
\tikzset{>=Straight Barb, commutative diagrams/arrow style=tikz}

% Utilities
\usepackage{enumerate}
\usepackage{todonotes}
\usepackage{graphicx}

% Color
\usepackage{xcolor}
\definecolor{brightmaroon}{rgb}{0.76, 0.13, 0.28}

% References
\usepackage{hyperref}
\hypersetup{
  colorlinks = true,
  allcolors  = brightmaroon,
}
\usepackage[capitalize,nameinlink]{cleveref}
\usepackage[
  backend = biber,
  style   = alphabetic,
]{biblatex}
\addbibresource{src/bibliography.bib}

% Table of contents: show subsections
\setcounter{tocdepth}{2}

\linespread{1.05}
\vfuzz=14pt % No more vbox errors all over the place

%%%%%%%%%%%%%%%%%%%%%%%%%%%%%%%%%%%%%%%%%%%%%%%%%%%%%%%%%%%%%%%%%%%%%%%%%%%%%%%
% ** Environments **

\theoremstyle{definition}
\newtheorem{theorem}{Theorem}[section]
\newtheorem{proposition}[theorem]{Proposition}
\newtheorem{lemma}[theorem]{Lemma}
\newtheorem{corollary}[theorem]{Corollary}
\newtheorem{axiom}[theorem]{Axiom}
\newtheorem{definition}[theorem]{Definition}
\newtheorem{remark}[theorem]{Remark}
\newtheorem{example}[theorem]{Example}
\newtheorem{notation}[theorem]{Notation}

%%%%%%%%%%%%%%%%%%%%%%%%%%%%%%%%%%%%%%%%%%%%%%%%%%%%%%%%%%%%%%%%%%%%%%%%%%%%%%%
% ** symbols **

\renewcommand{\qedsymbol}{\(\natural\)}
\renewcommand{\leq}{\leqslant}
\renewcommand{\geq}{\geqslant}
\renewcommand{\setminus}{\smallsetminus}
\renewcommand{\preceq}{\preccurlyeq}

% ':' for maps and '\colon' for relations on collections
\DeclareMathSymbol{:}{\mathpunct}{operators}{"3A}
\let\colon\relax
\DeclareMathSymbol{\colon}{\mathrel}{operators}{"3A}

\newcommand{\disj}{\squarecup}

% Disjoint unions over sets
\renewcommand{\disj}{\amalg}     % Disjoint union
\newcommand{\bigdisj}{\coprod} % Indexed disjoint union

\DeclareMathOperator{\Log}{Log}
\newcommand{\img}{\text{i}}

% Constant map
\DeclareMathOperator{\const}{cons}

% Multiplication map
\DeclareMathOperator{\mul}{mul}

%%%%%%%%%%%%%%%%%%%%%%%%%%%%%%%%%%%%%%%%%%%%%%%%%%%%%%%%%%%%%%%%%%%%%%%%%%%%%%%
% ** arrows **

% Alias for Rightarrow
\newcommand{\To}{\Rightarrow}

% Monomorphism arrow
\newcommand{\mono}{\rightarrowtail}

% Epimorphism arrow
\newcommand{\epi}{\twoheadrightarrow}

% Unique morphism
\newcommand{\unique}{\to}%{\dashrightarrow}
\newcommand{\xdashrightarrow}[2][]{\ext@arrow 0359\rightarrowfill@@{#1}{#2}}

% Isomorphism symbol
\newcommand{\iso}{\simeq}

\newcommand{\arrowiso}{\iso}
% Isomorphism arrow
\newcommand{\isoto}{\xrightarrow{\raisebox{-.6ex}[0ex][0ex]{\(\arrowiso\)}}}

% How isomorphisms are depicted in diagrams: either \sim or \simeq
\newcommand{\dis}{\iso}

\newcommand{\isounique}{%
  \xdashrightarrow{\raisebox{-.6ex}[0ex][0ex]{\(\arrowiso\)}}
}%

% Natural transformation arrow
\newcommand{\nat}{\Rightarrow}

% Natural isomorphism
\newcommand{\isonat}{\xRightarrow{\raisebox{-.8ex}[0ex][0ex]{\(\arrowiso\)}}}

% Embedding arrow
\newcommand{\emb}{\hookrightarrow}

% Parallel arrows
\newcommand{\para}{\rightrightarrows}

% Adjoint arrows
\newcommand{\adj}{\rightleftarrows}

% Implication
\renewcommand{\implies}{\Rightarrow}
\renewcommand{\impliedby}{\Leftarrow}

% Limits
\DeclareMathOperator{\Lim}{lim}     % Limit
\DeclareMathOperator{\Colim}{colim} % Colimit
\DeclareMathOperator{\Eq}{eq}       % Equalizer
\DeclareMathOperator{\Coeq}{coeq}   % Coequalizer

%%%%%%%%%%%%%%%%%%%%%%%%%%%%%%%%%%%%%%%%%%%%%%%%%%%%%%%%%%%%%%%%%%%%%%%%%%%%%%%
% ** Common collections **

\newcommand{\Z}{\mathbf{Z}}
\newcommand{\N}{\mathbf{N}}
\newcommand{\Q}{\mathbf{Q}}
\newcommand{\CC}{\mathbf{C}}
\newcommand{\R}{\mathbf{R}}

\renewcommand{\emptyset}{\varnothing}

\newcommand{\Uhs}{\mathbf{H}}  % Upper half space
\newcommand{\Proj}{\mathbf{P}} % Projective space

%%%%%%%%%%%%%%%%%%%%%%%%%%%%%%%%%%%%%%%%%%%%%%%%%%%%%%%%%%%%%%%%%%%%%%%%%%%%%%%
% ** Categories **

% Font for categories
\newcommand{\cat}{\mathcal}
\newcommand{\catfont}{\texttt}

% Opposite category
\newcommand{\op}{\mathrm{op}}

% Common categories
\newcommand{\Set}{{\catfont{Set}}}          % Sets
\newcommand{\FinSet}{{\catfont{FinSet}}}    % Finite sets
\newcommand{\pSet}{{\catfont{pSet}}}        % Pointed sets

\newcommand{\Vect}{{\catfont{Vect}}}        % Vector spaces
\newcommand{\FinVect}{{\catfont{FinVect}}}  % Finite vector spaces

\newcommand{\TVect}{{\catfont{TVect}}}      % Topological vector spaces
\newcommand{\Ban}{{\catfont{Ban}}}          % Banach spaces
\newcommand{\Man}{{\catfont{Man}}}          % Manifolds

\newcommand{\Grp}{{\catfont{Grp}}}          % Groups
\newcommand{\Ab}{{\catfont{Ab}}}            % Abelian groups
\newcommand{\GSet}[1]{{{#1}\text{-}\Set}}            % Abelian groups

\newcommand{\Graph}{{\catfont{Graph}}}      % Graphs
\newcommand{\SimpGraph}{{\catfont{sGraph}}} % Simple graphs
\newcommand{\ProfCol}{{\catfont{Prof}(\Col)}}   % C-profile category


\newcommand{\Rng}{{\catfont{Ring}}}             % Rings
\newcommand{\cRng}{{\catfont{CRing}}}           % Commutative rings
\newcommand{\rMod}[1]{{\texttt{Mod}_{#1}}}      % Right modules
\newcommand{\lMod}[1]{{{}_{#1}\catfont{Mod}}}   % Left modules
\newcommand{\Mod}[1]{{#1\text{-}\catfont{Mod}}} % Modules over comm. ring
\newcommand{\Alg}[1]{{#1\text{-}\catfont{Alg}}} % Algebras
\newcommand{\cAlg}[1]{{#1\text{-}\catfont{CAlg}}} % Commutative algebras

\newcommand{\Cat}{{\catfont{Cat}}}          % Small categories
\newcommand{\CAT}{{\catfont{CAT}}}          % Big categories
\newcommand{\UCat}{{\mathcal{U}\text{-}\catfont{Cat}}} % U-Categories

\newcommand{\Psh}[1]{{\catfont{Psh}({#1})}}   % Category of presheaves
\newcommand{\comma}{\downarrow} % Comma category separator
\DeclareMathOperator{\El}{El}              % Category of elements

% Operators
\DeclareMathOperator{\Hom}{Mor}   % Morphisms
\DeclareMathOperator{\Fct}{Fct}   % Functors
\DeclareMathOperator{\Obj}{Obj}   % Objects
\DeclareMathOperator{\Mor}{Mor}   % Morphisms, again
\DeclareMathOperator{\End}{End}   % Endomorphisms
\DeclareMathOperator{\Aut}{Aut}   % Automorphisms
\DeclareMathOperator{\Id}{id}     % Identity
\DeclareMathOperator{\im}{im}     % Image
\DeclareMathOperator{\dom}{dom}   % Domain
\DeclareMathOperator{\codom}{cod} % Codomain
\DeclareMathOperator{\supp}{supp} % Support

% Yoneda embedding
\newcommand{\yo}{\text{\usefont{U}{min}{m}{n}\symbol{'210}}}
\DeclareFontFamily{U}{min}{}
\DeclareFontShape{U}{min}{m}{n}{<-> udmj30}{}

%%%%%%%%%%%%%%%%%%%%%%%%%%%%%%%%%%%%%%%%%%%%%%%%%%%%%%%%%%%%%%%%%%%%%%%%%%%%%%%
% ** algebra **
\DeclareMathOperator{\rank}{rank}
\DeclareMathOperator{\coker}{coker}
\DeclareMathOperator{\codim}{codim}
\DeclareMathOperator{\Tr}{tr}   % Trace
\DeclareMathOperator{\Sym}{Sym} % Symmetric space
\DeclareMathOperator{\Alt}{Alt} % Alternating space
\DeclareMathOperator{\Char}{char}
\DeclareMathOperator{\Span}{span}
\DeclareMathOperator{\Inn}{Inn}     % Inner automorphisms
\DeclareMathOperator{\Spec}{Spec}   % Prime spectrum
\DeclareMathOperator{\Specm}{Spec_m} % Maximal spectrum
\newcommand{\lie}[1]{\mathfrak{#1}} % Font for Lie structures
\DeclareMathOperator{\Rees}{Rees}   % Rees algebra
\DeclareMathOperator{\Frac}{Frac}   % Field of fractions

\DeclareMathOperator{\eval}{eval}
\DeclareMathOperator{\sign}{sign}

% Matrices
\DeclareMathOperator{\Mat}{Mat}
\DeclareMathOperator{\GL}{GL}
\DeclareMathOperator{\SL}{SL}
\DeclareMathOperator{\PSL}{PSL}
\DeclareMathOperator{\SO}{SO}
\DeclareMathOperator{\SU}{SU}
\DeclareMathOperator{\Unit}{U}
\DeclareMathOperator{\Orth}{O}

% Symbol for the group of units --- for instance, the group of units of a ring
% \(R\) will be denoted by \(R^{\unit}\).
\newcommand{\unit}{\times}

% Orbit and stabilizer
\DeclareMathOperator{\Orb}{Orb}
\DeclareMathOperator{\Stab}{Stab}

% Ring ideals font
\newcommand{\ideal}[1]{\mathfrak{#1}}

%%%%%%%%%%%%%%%%%%%%%%%%%%%%%%%%%%%%%%%%%%%%%%%%%%%%%%%%%%%%%%%%%%%%%%%%%%%%%%%
% ** Topology **

\let\Top\relax
\newcommand{\Top}{{\catfont{Top}}}                       % Topological spaces

\newcommand{\wHTop}{{\catfont{wH}\text{-}\catfont{Top}}} % Weak Hausdorff
\newcommand{\kTop}{{k\text{-}\catfont{Top}}}             % k-spaces
\newcommand{\cgTop}{{\catfont{cgTop}}} % Compactly generated

\newcommand{\pTop}{{\catfont{Top}^{*}}}   % Pointed top spaces
\newcommand{\bpTop}{{\catfont{Top}^{*/}}} % Base point top spaces

\newcommand{\Ho}[1]{{\catfont{Ho}(#1)}}            % Homotopy category
\newcommand{\HoTop}{{\catfont{Ho}(\catfont{Top})}} % Classical Homotopy cat

\newcommand{\Splx}{{\mathbf{\Delta}}}           % Simplex category
\newcommand{\sSet}{{\catfont{sSet}}}            % Simplicial sets
\newcommand{\Simp}[1]{{\catfont{Simp}(#1)}}     % Simplicial category
\newcommand{\CoSimp}[1]{{\catfont{CoSimp}(#1)}} % Cosimplicial category

\DeclareMathOperator{\Sing}{Sing} % Singular complex functor

% attaching spaces
\newcommand{\att}{\amalg}     % Disjoint union
\newcommand{\bigatt}{\coprod} % Indexed disjoint union

% Homotopy
\newcommand{\simht}{\sim_{\text{h}}}              % Homotopy between maps
\newcommand{\simhtrel}[1]{\sim_{\text{rel }{#1}}} % Relative htpy
\newcommand{\isoht}{\iso_{\text{h}}}              % Homotopy equivalence
\newcommand{\htpy}{\Rightarrow}                   % Htpy arrow
\newcommand{\htpyrel}[1]{\Rightarrow_{\text{rel }{#1}}} % Relative htpy arrow


% Functors on topological spaces
\DeclareMathOperator{\Cone}{Cone}       % Cone
\DeclareMathOperator{\Cyl}{Cyl}         % Cylinder
\DeclareMathOperator{\Susp}{S}          % Suspension
\DeclareMathOperator{\rSusp}{\Sigma}    % Reduced suspension
\DeclareMathOperator{\Path}{Path}       % Path object
\DeclareMathOperator{\Eval}{eval}       % Evaluation map
\DeclareMathOperator{\curry}{curry}     % Currying a map
\DeclareMathOperator{\uncurry}{uncurry} % Currying a map
\DeclareMathOperator{\co}{co}           % Compact open co(K, U)

% Set operators
\DeclareMathOperator{\Cl}{Cl}       % Closure
\DeclareMathOperator{\Bd}{\partial} % Boundary
\DeclareMathOperator{\Int}{Int}     % Interior
\DeclareMathOperator{\Ext}{Ext}     % Exterior

%%%%%%%%%%%%%%%%%%%%%%%%%%%%%%%%%%%%%%%%%%%%%%%%%%%%%%%%%%%%%%%%%%%%%%%%%%%%%%%
% ** Differentiable structures **

% Norm
\DeclarePairedDelimiter{\norm}{\lVert}{\rVert}

% Differential operators
\newcommand{\diff}{\mathrm{d}}
\newcommand{\Diff}{\mathrm{D}}
\DeclareMathOperator{\grad}{grad} % Gradient
\DeclareMathOperator{\Hess}{Hess} % Hessian
\DeclareMathOperator{\Jac}{Jac}   % Jacobian
\DeclareMathOperator{\Curl}{Curl} % Curl

% Set operators
\DeclareMathOperator{\Vol}{vol}   % Volume
\DeclareMathOperator{\Mesh}{mesh} % Mesh

%%%%%%%%%%%%%%%%%%%%%%%%%%%%%%%%%%%%%%%%%%%%%%%%%%%%%%%%%%%%%%%%%%%%%%%%%%%%%%%
% ** Graphs **

% Colouring
\newcommand{\Col}{\mathfrak{C}}
\newcommand{\prof}[1]{\underline{#1}}

\DeclareMathOperator{\Edge}{Edge}
\DeclareMathOperator{\Vertex}{Vert}
\DeclareMathOperator{\Circ}{circ}
\DeclareMathOperator{\diam}{diam}
\newcommand{\emptygraph}{\varnothing}
\DeclarePairedDelimiterX{\size}[1]{\lVert}{\rVert}{#1}

%%%%%%%%%%%%%%%%%%%%%%%%%%%%%%%%%%%%%%%%%%%%%%%%%%%%%%%%%%%%%%%%%%%%%%%%%%%%%%%
% ** MACROS END HERE **
%%%%%%%%%%%%%%%%%%%%%%%%%%%%%%%%%%%%%%%%%%%%%%%%%%%%%%%%%%%%%%%%%%%%%%%%%%%%%%%

\author{Luiz G. Mugnaini A.}
\date{Last modification: \today}
\title{Deep Dive}

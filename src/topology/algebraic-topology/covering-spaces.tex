\section{The van Kampen Theorem}

\subsection{For the Fundamental Groupoid}

\begin{theorem}[van Kampen theorem for the fundamental groupoid]
\label{thm:van-Kampen-groupoid}
Let \(X\) be a space, and \(\mathcal{O}\) be a connected open
cover\footnote{That is, composed of connected open subsets of \(X\).} of \(X\)
that is closed under finite intersections---which is a subcategory of \(\Top\)
whose morphisms are inclusions. Considering the functor
\(\Pi_1: \Top \to \Grpd\), the fundamental groupoid \(\Pi_1 X\) is the colimit
of the functor \(\Pi_1|_{\mathcal{O}}\), that is:
\[
\Pi_1 X \iso \Colim_{U \in \mathcal{O}} \Pi_1 U
\]
in the category \(\Grpd\).
\end{theorem}

\begin{proof}
We'll show that \(\Pi_1|\) Let \(\mathcal{G}\) be a groupoid and consider the
constant functor \(C: \mathcal{O} \to \Grpd\) mapping \(U \mapsto \mathcal{G}\)
and \(\iota \mapsto \Id_{\mathcal{G}}\) for any object \(U \in \mathcal{O}\) and
morphism \(\iota\) of \(\mathcal{O}\). Let \(\eta: \Pi_1|_{\mathcal{O}} \nat C\)
be a natural transformation. The pair \((\mathcal{G}, \eta)\) forms a
\emph{cocone} over the functor \(\Pi_1|_{\mathcal{O}}\): indeed, given an object
\(U \in \mathcal{O}\) there exists a morphism of groupoids
\(\eta_U: \Pi_1 U \to C U = \mathcal{G}\) and from naturality, given any
inclusion \(\iota: U \emb V\) in \(\mathcal{O}\) one has that
\[
\begin{tikzcd}
\Pi_1 U \ar[d, "\Pi_1 \iota"'] \ar[r, "\eta_U"]
&C U = \mathcal{G} \ar[d, "C \iota = \Id_{\mathcal{G}}"] \\
\Pi_1 V \ar[r, "\eta_V"'] &C V = \mathcal{G}
\end{tikzcd}
\]
commutes in \(\Grpd\)---showing that
\(\eta_U = \Id_{\mathcal{G}} \eta_U = \eta_V \circ \Pi_1 \iota\), hence
compatibility \(\Pi_1\) is satisfied, making \((\mathcal{G}, \eta)\) a cocone.

Consider the collection \((i_U: \Pi_1 U \to \Pi_1 X)_{U \in \mathcal{O}}\) of
canonical inclusions of groupoids. To show the universal property of \(\Pi_1 X\)
we must construct a \emph{unique morphism} of groupoids (that is, a functor
between groupoids) \(\chi: \Pi_1 X \to \mathcal{G}\) such that
\begin{equation}\label{eq:van-kampen-grpd}
\begin{tikzcd}
\mathcal{G} &\Pi_1 U \ar[l, "\eta_U"'] \ar[dl, bend left, "i_U"] \\
\Pi_1 X \ar[u, dashed, "\chi"] &
\end{tikzcd}
\end{equation}
commutes in \(\Grpd\) for all \(U \in \mathcal{O}\).

To that end, for every \(x \in X\), if \(x \in U\) let
\(\chi x \coloneq \eta_U x\). For the morphisms of \(\Pi_1 X\), consider any
path \(f \in \Path_X(x, y)\) on \(X\). If \(\im f\) lies interely in some object
\(U \in \mathcal{O}\), simply define \(\chi [f] \coloneq \eta_U [f]\)---since
\(\mathcal{O}\) is closed under finite intersections, if
\(\im f \subseteq U \cap V\) for some \(U, V \in \mathcal{O}\), then the image
\(\eta_{\bullet} [f]\) is independent of the choice of \(U\) or \(V\). Consider
now the case where \(f\) has an image not entirely contained in a single element
of \(\mathcal{O}\), but multiple ones, say
\(\im f \subseteq \bigcup_{j=1}^n U_j\) for some finite collection of sets
\(U_j \in \mathcal{O}\)---since \(\mathcal{O}\) is closed under finite
intersections, this, we shall define a corresponding collection of paths
\((f_j: I \to U_j)_{j=1}^n\) such that
\[
f = f_n f_{n-1} \cdots f_2 f_1.
\]
With this collection in hands we may define
\(\chi[f] \coloneq \eta_{U_n}[f_n] \cdots \eta_{U_1}[f_1]\).

We must ensure that this is well defined: let \(g \in \Path_X(x, y)\) be another
path and suppose there exists a homotopy \(\varepsilon: f \nat g\). Take a
decomposition \((g_j: I \to V_j)_{j=1}^m\) for sets \(V_j \in
\mathcal{O}\). Consider a partition \((J_j \times S_j)_{j=1}^{\ell}\) of the
square \(I \times I\) such that
\(\im \varepsilon|_{J_j \times S_j} \subseteq U\) for some
\(U \in \mathcal{O}\), and such that \((J_j)_{j=1}^{\ell}\) is a refinement for
the decompositions of \(f\) and \(g\)---that is, \(f|_{J_j}\) and \(g|_{J_j}\)
are paths entirely contained in some set of \(\mathcal{O}\). In this way we see
that \(\varepsilon\) induces a collection of homotopies
\((\varepsilon_j: f_j \nat g_j)_{j=1}^{\ell}\) proving that \([f_j] = [g_j]\) in
some \(\Pi_1 U\), therefore \(\chi [f] = \chi [g]\). Hence \(\chi\) is a
uniquely defined functor satisfying \cref{eq:van-kampen-grpd}.
\end{proof}

\subsection{For the Fundamental Group}

\begin{theorem}[van Kampen theorem for the fundamental group]
\label{thm:van-kampen-group}
Let \((X, x) \in \bpTop\) be a path-connected space, and \(\mathcal{O}\) be a
path-connected open cover of \(X\) closed under finite intersections---and such
that \(x \in U\) for every \(U \in \mathcal{O}\), therefore \(\mathcal{O}\) is a
subcategory of \(\bpTop\) whose morphisms are inclusions. Then the fundamental
groupoid of \(X\) is the colimit of the functor \(\pi_1|_{\mathcal{O}}:
\mathcal{O} \to \Grp\), that is:
\[
\pi_1(X, x) \iso \Colim_{U \in \mathcal{O}} \pi_1(U, x).
\]
\end{theorem}

We first prove a particular case of the classical van Kampen theorem and after
generalize.

\begin{lemma}
\label{lem:van-kampen-finite-cover}
The van Kampen theorem for the fundamental group holds when \(\mathcal{O}\) is
finite.
\end{lemma}

\begin{proof}
Let \(G\) be a group and \(C: \mathcal{O} \to \Grp\) be the constant functor on
\(G\) and consider the cocone \((G, \eta: \pi_1|_{\mathcal{O}} \nat C)\) over
the functor \(\pi_1|_{\mathcal{O}}\). We'll construct a morphism of groups
\(\chi: \pi_1(X, x) \to G\). Recall that the inclusion functor
\(J: \pi_1(X, x) \to \Pi_1 X\) is an equivalence of categories, since
\(\pi_1(X, x)\) is a skeleton of \(\Pi_1 X\) by
\cref{prop:pi1-equivalent-to-Pi1-for-connected-space}. Define a quasi-inverse of
\(J\) as follows: consider a collection \((\gamma_y)_{y \in X}\) of paths
\(\gamma_y \in \Path_X(x, y)\) where \(\im \gamma_y \subseteq U\) when
\(y \in U\) and \(\gamma_x \coloneq \const_x\)---this is possible because
\(\mathcal{O}\) is closed under finite intersections---then define
\(F: \Pi_1 X \to \pi_1(X, x)\) by mapping \(f: a \to b\) to
\(F f \coloneq \gamma_b f \gamma_a^{-1}: x \to x\).

Notice that the quasi-inverse functors \(J\) and \(F\) induce, for each \(U \in
\mathcal{O}\), a corresponding pair of quasi-inverse functors
\[
F_U \colon \Pi_1 U \rightleftarrows \pi_1(U, x) \colon J_U.
\]
Then we can construct a cocone \((G, \delta: \Pi_1|_{\mathcal{O}} \nat C)\) over
the functor \(\Pi_1|_{\mathcal{O}}\), where
\(\delta_U \coloneq \eta_U F_U: \Pi_1 U \to G\). By means of
\cref{thm:van-Kampen-groupoid} there exists a \emph{unique} morphism of
groupoids \(\xi: \Pi_1 X \to G\) (where \(G\) is interpreted as a groupoid with a
single object) such that
\[
\begin{tikzcd}
\Pi_1 U \ar[r, "F_U"] \ar[rrd, bend right, hook, "i_U"']
&\pi_1(U, x) \ar[r, "\eta_U"] &G \\
&&\Pi_1 X \ar[u, dashed, "\xi"']
\end{tikzcd}
\]
commutes in \(\Grpd\) for every \(U \in \mathcal{O}\). Define
\(\chi \coloneq \xi J: \pi_1(X, x) \to G\), and notice that since
\(\eta_U F_U = \xi i_U\) we can precompose with \(J_U: \pi_1(U, x) \to \Pi_1 U\)
and use that \(F_U J_U = \Id_{\pi_1(U, x)}\) to obtain that
\(\eta_U = \xi i_U J_U\). Notice however that given any \([g] \in \pi_1(U, x)\)
one has \(i_U J_U [g] = [g] \in \Pi_1 X\) while \(J j_U [g] = [g] \in \Pi_1 X\)
again---for the canonical inclusion
\(j_U: \pi_1(U, x) \emb \pi_1(X, x)\)---therefore \(i_U J_U = J j_U\). This
proves that \(\eta_U = \xi J j_U = \chi j_U\) for every \(U \in \mathcal{O}\),
that is
\[
\begin{tikzcd}
\pi_1(U, x) \ar[r, "\eta_U"] \ar[rdd, bend right, "j_U", hook] &G \\
&\Pi_1 X \ar[u, dashed, "\xi"] \\
&\pi_1(X, x) \ar[u, "J"] \ar[uu, dashed, bend right=50, "\chi"']
\end{tikzcd}
\]
commutes in \(\Grp\), which proves the universal property for the colimit
\(\pi_1(X, x)\).
\end{proof}

\subsubsection{Proof of the Classical van Kampen Theorem}

Let \(\mathcal{O}\) be a path-connected open cover of \(X\) closed under
intersections and composed of neighbourhoods of the chosen base-point \(x\). Let
\(\mathfrak{F} \subseteq 2^{\mathcal{O}}\) be the category whose objects are the
\emph{finite} subsets of \(\mathcal{O}\) of the cover that is \emph{closed under
  finite intersections}, and morphisms are \emph{inclusions}. Given any such
subset \(\mathcal{C} \in \mathfrak F\), we know from
\cref{lem:van-kampen-finite-cover} that the space
\(U_{\mathcal{C}} \coloneq \bigcup_{U \in \mathcal{C}} U\) satisfies
\begin{equation}\label{eq:van-kampen-grp-1}
\pi_1(U_{\mathcal{C}}, x) \iso \Colim_{U \in \mathcal{C}} \pi_1(U, x).
\end{equation}

\begin{itemize}\setlength\itemsep{0em}
\item Let's prove that the colimit of the functor
  \(\pi_1|_{\mathfrak{F}}: \mathfrak{F} \to \Grp\)---which maps each
  \(\mathcal{C} \in \mathfrak{F}\) to the group
  \(\pi_1(U_{\mathcal{C}}, x)\)---is the fundamental group \(\pi_1(X,
  x)\). Given any group \(G\) and a its corresponding constant functor
  \(C_G: \mathfrak{F} \to \Grp\) with \(C_G \mathcal{C} \coloneq G\), let
  \(\eta: \pi_1|_{\mathfrak{F}} \nat C_G\) be a natural transformation. The pair
  \((G, \eta)\) is then a cocone over the functor \(\pi_1|_{\mathfrak F}\).

  We'll construct a unique morphism of groups \(\chi: \pi_1(X, x) \to G\)
  satisfying the coherence of the cocones using the same technique from
  \cref{thm:van-Kampen-groupoid}. If \(f: x \to x\) is a loop contained entirely
  in a set \(U_{\mathcal{C}} \subseteq X\) for some
  \(\mathcal{C} \in \mathfrak{F} F\), we simply map
  \(\chi [f] \coloneq \eta_{\mathcal{C}} [f]\). If on the other hand \(f\) is
  not entirely contained in a single set, say that \(f\) is contained in the
  union \(\bigcup_{j=1}^n U_j\) for sets \(U_j \in \mathcal{C}\) and define a
  collection of decompositions of \(f\), namely \((f_j: I \to U_j)_{j=1}^n\),
  for which \(f\) is the result of the concatenation of paths. From the same
  argument as before, merely map
  \(\chi [f] \coloneq \eta_{U_n} [f_n] \cdots \eta_{U_1} [f_1]\), which is well
  defined and unique\footnote{Simply refer to the proof of
    \cref{thm:van-Kampen-groupoid}, now it should be clear that the proof
    follows exacly the same steps}. Therefore one has
  \begin{equation}\label{eq:van-kampen-grp-2}
  \Colim_{\mathcal{C} \in \mathfrak{F}} \pi_1(U_{\mathcal{C}}, x) \iso \pi_1(X, x).
  \end{equation}

\item For the final part of the proof, we shall prove that the colimits of the
  functors \(\pi_1|_{\mathcal{O}}\) and \(\pi_1|_{\mathfrak{F}}\) agree so that
  the van Kampen theorem is true. Recalling \cref{eq:van-kampen-grp-1}, one has
  \begin{align*}
  \Colim_{\mathcal{C} \in \mathfrak{F}} \pi_1(U_{\mathcal{C}}, x)
  &\iso
  \Colim_{\mathcal{C} \in \mathfrak{F}}(\Colim_{U \in \mathcal{C}} \pi_1(U, x)) \\
  &\iso
  \Colim_{(\mathcal{O}, \mathfrak{F})} \pi_1(U, x),
  \end{align*}
  where \((\mathcal{O}, \mathfrak{F})\) is the category whose objects are pairs
  \((U, \mathcal{C}) \in \mathcal{C} \times \mathfrak{F}\), and morphisms are
  paired inclusions---also
  \(\pi_1(-, x)|_{(\mathcal{O}, \mathfrak{F})}: (\mathcal{O}, \mathfrak{F}) \to
  \Grp\) is defined to map \((U, \mathcal{C})\) to \(\pi_1(U, x)\). Notice that
  the functors \(\pi_1(-, x)|_{\mathcal{O}}\) and
  \(\pi_1(-, x)|_{(\mathcal{O}, \mathfrak{F})}\) factor as
  \[
  \begin{tikzcd}
  \mathcal{O} \ar[rr, "\pi_1{(-, x)}|_{\mathcal{O}}"] \ar[rd, "\iota"]
  & &\Grp \\
  &(\mathcal{O}, \mathfrak{F})
  \ar[ru, "\pi_1{(-, x)}|_{(\mathcal{O}, \mathfrak F)}"']
  \ar[lu, bend left=50, "p"] &
  \end{tikzcd}
  \]
  Where \(\iota U \coloneq (U, \{U\})\) and \(p (U, \mathcal{C}) \coloneq
  U\). Therefore one has an isomorphism
  \begin{equation}\label{eq:van-kampen-grp-3}
  \Colim_{U \in \mathcal{O}} \pi_1(U, x)
  \iso
  \Colim_{(U, \mathcal{C}) \in (\mathcal{O}, \mathfrak F)} \pi_1(U, x).
  \end{equation}
\end{itemize}
Therefore, by \cref{eq:van-kampen-grp-2,eq:van-kampen-grp-3} we have
\[
\Colim_{U \in \mathcal{O}} \pi_1(U, x) \iso \pi_1(X, x)
\]
as wanted.

\section{Covering Spaces}

\subsection{Initial Constructions}

\begin{definition}[Covering space]
\label{def:covering-space}
A surjective continuous map \(p: E \to B\) is said to be a \emph{covering space
  over \(B\)} if for every point \(x \in B\), there exists a neigbourhood
\(U \subseteq B\) of \(x\) that is \emph{evenly covered} by \(p\), that is:
there exists a bundle isomorphism over \(B\) between the pullback\footnote{It
  should be noted that the pullback of \(p\) over \(U\) is nothing more than the
  induced bundle \(\iota^{*} p\) given by the inclusion \(\iota: U \emb B\).} of
\(p\) over \(U\) and a product bundle \(\pi: U \times E_x \to U\) with
\emph{discrete\footnote{Here, discrete means a space together with the discrete
    topology. We shall make use of the functor \(\Disc: \Set \to \Top\) mapping
    a bare set \(S\) to the topological space \(\Disc S\) with underlying set
    \(S\) and endowed with the discrete topology.} fibre}
\(E_x = p^{-1} x\). Diagrammatically, one has that the diagram
\[
\begin{tikzcd}
U \times E_x \ar[r, "\dis"] \ar[rd, "\pi"']
&p^{-1} U \ar[r, hook] \ar[d]
\ar[rd, "\lrcorner", phantom, very near start]
&E \ar[d, "p"] \\
&U \ar[r, hook] &B
\end{tikzcd}
\]
commutes in \(\Top\), where the topological isomorphism
\(U \times E_x \isoto p^{-1} U\) is explicitly given by \((u, e) \mapsto
e\). Said concisely, a covering space is a locally trivial bundle with discrete
fibre.

Yet another equivalent way of defining a covering space goes as follows. \(B\) is
evenly covered by \(p\) if there exists an open cover \(\mathcal{U}\) of \(B\)
such that any \(U \in \mathcal{U}\) has a preimage \(p^{-1} U = \bigdisj_{\alpha}
V_{\alpha}\) where \(V_{\alpha}\) is an open set of \(E\), and the restriction
\(p|_{V_{\alpha}}: V_{\alpha} \isoto U\) is a topological isomorphism.
\end{definition}

\begin{corollary}
\label{cor:covering-space-is-locally-top-iso}
A covering space is a local topological isomorphism.
\end{corollary}

\begin{proof}
Given a covering space \(p: E \to B\) and \(x \in E\), we can consider the
neighbourhood \(U \subseteq B\) of \(p x\) that is evenly covered by \(p\). Then
from our last definition we know that \(U\) is isomorphic to a disjoint union of
open sets of \(E\). Let \(V \subseteq p^{-1} U\) be one such open set which is
also a neighbourhood of \(x\). Then the restriction \(p|_V: V \to p(V) = U\) is
an isomorphism, showing that \(p\) is indeed a local isomorphism.
\end{proof}

\begin{notation}[Sheets]
\label{not:coveing-space-sheet}
Let \(p: E \to B\) be a covering space and \(x \in B\) be any point. Consider a
neighbourhood \(U \subseteq B\) of \(x\) such that \(p\) admits a trivialisation
\(\phi: p^{-1} U \isoto U \times E_x\). Since \(E_x\) is discrete, we have an
isomorphism \(U \times E_x \iso \bigdisj_{e \in E_x} U \times e\). Moreover,
clearly \(U \iso U \times e\) for any \(e \in E_x\). We shall define a
\emph{sheet} over \(U\) to be an open set
\begin{equation}\label{eq:covering-space-sheet}
U_e \coloneq \phi^{-1}(U \times \{e\}) \subseteq E.
\end{equation}
The motivation comes from the fact that \(U_e \iso U \times e \iso U\).
\end{notation}

\begin{proposition}[Covering projections are open]
\label{prop:covering-projection-is-open}
Given a covering space \(p: E \to B\), the projection \(p\) is open.
\end{proposition}

\begin{proof}
Consider the open cover \((U_x)_{x \in B}\) of \(B\)---where \(U_x\) is a
neighbourhood of \(x\) such that there exists a topological isomorphism
\(p^{-1} U_x \iso U_x \times \Disc E_x\), for each \(x \in B\). Under the
product topology, projections are open maps (see
\cref{lem:projections-open-under-product-top}), therefore \(p|_{p^{-1} U_x}\) is
an open map for each \(x \in B\). Given any open set \(V \subseteq E\), notice
that \(V = \bigcup_{x \in B} (W \cap p^{-1} U_x)\), therefore
\[
p V = p\Big(\bigcup_{x \in B} (W \cap p^{-1} U_x)\Big)
= \bigcup_{x \in B} p(W \cap p^{-1} U_x)
\]
is the union of open sets, thus open---which shows that \(p\) is an open map.
\end{proof}

\begin{lemma}[Fibre-wise diagonal of covering space is open and closed]
\label{lem:fibre-wise-diag-open-and-closed}
Let \(p: E \to B\) be a covering space, and \(E \times_B E\) be the pullback of
\(p\) with itself. Then the diagonal of \(E\) with respect to the fibre product
over \(B\), namely
\[
\Delta_B E = \{(e, e) \in E \times_B E \colon e \in E\},
\]
is an \emph{open and closed set} in \(E \times_B E\).
\end{lemma}

\begin{proof}
First we prove that \(\Delta_B E\) is open. Let \(e \in E\) be any point and
\(U_{p e} \subseteq B\) be a neighbourhood of \(p e\) together with an
isomorphism
\[
p^{-1} U_{p e} \iso U_{p e} \times E_{p e} = U_{p e} \times \{e\}.
\]
Since \(p^{-1} U_{p e}\) is open, it follows that
\(U_{p e} \times \{e\} \emb E\) is open. Then the set
\((E \times_B E) \cap (U_{p e} \times U_{p e})\) is an open neighbourhood of
\((e, e)\) in \(E \times_B E\). This shows that \(\Delta_B E\) is open.

To show that \(\Delta_B E\) is closed, let \((x, y) \in E \times_B E\) with
\(x \neq y\). Let \(U \subseteq B\) be a neighbourhood of \(p x = p y\) that is
evenly covered, and consider the induced sheets \(U_x, U_y \subseteq E\) for
\(x\) and \(y\), respectively. From the assumption that \(x\) and \(y\) are
distinct points, we have an empty intersection
\(U_x \cap U_y = \emptyset\)---which shows that \(U_x \times U_y\) is disjoint
from \(\Delta_B E\). Therefore \((E \times_B E) \cap (U_x \times U_y)\) is a
neighbourhood for \((x, y) \in E \times_B E\) outside of \(\Delta_B E\), showing
that \(\Delta_B E\) is closed.
\end{proof}

\section{Lifting Properties}

\begin{definition}[Lift]
\label{def:lift-top}
Let \(f: Y \to X\) and \(g: Z \to X\) be morphisms. We define a \emph{lift of
  \(g\) along \(f\)} to be a morphism \(\lift{g}: Z \to Y\) such that the
following diagram commutes:
\[
\begin{tikzcd}
 &Y \ar[d, "f"] \\
Z \ar[r, "g"'] \ar[ru, "\lift g", bend left] & X
\end{tikzcd}
\]
\end{definition}

\begin{proposition}[Homotopy lift via covering space]
\label{prop:homotopy-lift-via-covering-space}
Let \(p: E \to B\) be a covering space and consider a homotopy \(\eta: X \times
I \to B\). Let \(\lift \eta_0: X \to E\) be a lift of \(\eta(-, 0)\) along
\(p\). Then there exists a \emph{unique lift} \(\lift \eta: X \times I \to E\)
of \(\eta\) along \(p\) which \emph{extends} \(\lift \eta_0\). That is, the
following diagram commutes:
\[
\begin{tikzcd}
X \times 0 \ar[r, "\lift \eta_0"] \ar[d]
&E \ar[d, "p"] \\
X \times I \ar[ru, dashed, "\lift \eta"] \ar[r, "\eta"']
&X
\end{tikzcd}
\]
\end{proposition}

\begin{proof}
We shall prove this proposition in three acts, first we shall construct a lift
with respect to a neighbourhood of any given point \(x \in X\), later the
uniqueness of such lift, and finally how all this pieces together to form the
required lift.

\begin{enumerate}[(i)]\setlength\itemsep{0em}
\item Let \(x \in X\) be any point and take, for each \((x, t) \in x \times I\),
  an evenly covered neighbourhood \(U_t \subseteq B\) of \(\eta(x, t)\). Now we
  can choose a neighbourhood \(N_t \times (a_t, b_t)\) of \((x, t)\) such that
  \(\eta(N_t \times (a_t, b_t)) \subseteq U_t\) is again evenly covered by
  \(p\).

  Since \(x \times I\) is compact, let \(0 = t_0 < t_1 < \dots < t_n = 1\) be a
  partition of \(I\) such that \(\{N_{t_j} \times [t_j, t_{j+1}]\}_{j=0}^{n-1}\)
  covers \(x \times I\)---and also satisfies the requirement that
  \(\eta(N_{t_j} \times [t_j, t_{j+1}])\) is evenly covered by \(p\) as in the
  previous construction. Define the neighbourhood
  \(N \coloneq \bigcap_{j=0}^n N_{t_j}\) of \(x\), so that
  \(\eta(N \times [t_j, t_{j+1}])\) is still evenly covered by \(p\) for any \(j\).

  We proceed by induction assuming that, for some \(j\), the lift \(\lift \eta\)
  has been constructed on \(N \times [0, t_j]\) and which extends the given
  starting lift \(\lift \eta_0\). Let \(U_j \subseteq B\) be an evenly covered
  set containing \(\eta(N \times [t_j, t_{j+1}])\), and take
  \(V_j \subseteq E\) to be an open set projecting isomorphically onto \(U_j\)
  via \(p\) and with \(\lift\eta(x, t) \in V_j\).  Define
  \[
  N' \times t_j \coloneq (N \times t_j) \cap (\lift\eta|_{N \times t_j})^{-1} V_j,
  \]
  which is again a neighbourhood of \((x, t_j)\). From this construction we find
  that \(\lift\eta(N' \times t_j) \subseteq V_j\). Now we define \(\lift\eta\)
  on \(N' \times [t_j, t_{j+1}]\) to be the composition of \(\eta\) with
  \(p^{-1}: U_j \to V_j\), that is:
  \[
  \begin{tikzcd}
  &V_j \\
  N' \times [t_j, t_{j+1}]
  \ar[ru, bend left, "\lift\eta"]
  \ar[r, "\eta"']
  & U_j \ar[u, "p^{-1}"', "\dis"]
  \end{tikzcd}
  \]
  This procedure is to be done a finite number of iterations, so that in the end
  we obtain a lift \(\lift\eta: M \times I \to E\) for some neighbourhood
  \(M \subseteq X\) of \(x\).

\item To prove the uniqueness of the lifting \(\lift\eta\) we'll work simply on
  the case of a single point \(x\), so that uniqueness for the total set
  \(X \times I\) will follow from the uniqueness of the lift on segments
  \(x \times I\) for each \(x \in X\). Let \(\lift\eta\) and \(\lift\eta'\) be
  two lifts of \(\eta: x \times I \to B\) with \(\lift\eta(0) =
  \lift\eta'(0)\). From the same argument as before, take a partition
  \(0 = t_0 < t_1 < \dots < t_n = 1\) of \(I\) such that
  \(\eta([t_j, t_{j+1}])\) is contained in an evenly covered neighbourhood
  \(U_j \subseteq B\) by \(p\). Using induction, we may assume that
  \(\lift\eta|_{[0, t_j]} = \lift\eta'|_{[0, t_j]}\) for some \(j\). Using the
  fact that \([t_j, t_{j+1}]\) is a connected set, the image
  \(\lift\eta([t_j, t_{j+1}]) \subseteq E\) will also be connected. Therefore
  there exists a set \(V_j \subseteq E\) containing
  \(\lift\eta([t_j, t_{j+1}])\), and with \(V_j \iso U_j\) via \(p\). Since
  \(\lift\eta t_j = \lift\eta' t_j\) then by connectedness
  \(\lift\eta'([t_j, t_{j+1}]) \subseteq V_j\). The lift condition implies in
  \(p \lift\eta = p \lift\eta'\), however we also know that \(p\) is injective
  in \(V_j\)---therefore it follows that
  \(\lift\eta|_{[t_j, t_{j+1}]} = \lift\eta'|_{[t_j, t_{j+1}]}\). This procedure
  is continued for finitely many steps and shows that
  \(\lift\eta = \lift\eta'\).

\item Notice that, by choosing a point \(x \in X\), we obtain a lift piece of
  the form \(M \times I \to E\) where \(M \subseteq X\) is a neighbourhood of
  \(x\). Notice however that in the overlap of such neighbourhoods, when one
  considers the whole range of points contained in \(X\), we must have an
  agreement of the lift pieces. Therefore, such neighbourhoods may be glued to
  create a well defined lift \(\lift\eta: X \times I \to E\). Moreover, using
  the proof of the last item we know that each segment
  \(\lift\eta|_{x \times I}\) is unique for any \(x \in X\), thus \(\lift\eta\)
  itself is unique.
\end{enumerate}
\end{proof}

\begin{corollary}
\label{cor:lift-via-covering-space}
Let \(p: E \to B\) be a covering space. The following are lifting properties
associated to \(p\):
\begin{enumerate}[(a)]\setlength\itemsep{0em}
\item For each path \(\gamma: I \to B\) starting at \(x_0 \in B\) and each \(e_0
  \in p^{-1} x_0\) there exists a \emph{unique lift} \(\lift \gamma: I \to E\)
  which starts at \(e_0\).

\item For each homotopy \(\eta: I \times I \to B\) of paths starting at \(x_0\)
  and each \(e_0 \in p^{-1} x_0\) there is a unique lifted homotopy
  \(\lift \eta: I \times I \to E\) of paths starting at \(e_0\).

\item The lift of a constant path is constant.

\item Every homotopy between paths lifts to a homotopy of paths\footnote{Since
    the end-points are fixed---which was pointed out in item (c).}.
\end{enumerate}
\end{corollary}

\begin{proposition}
\label{proposition:covering-induces-monomorphism-fundamental-grps}
Given a covering \(p: E \epi B\), the induced morphism of groups
\[
p_{*}: \pi_1(E, e) \mono \pi_1(B, p e)
\]
is a monomorphism for any \(e \in E\).
\end{proposition}

\begin{proof}
Let \([\lift f]\) be a kernel element of \(p_{*}\), so that
\(p_{*}[\lift f] = [f] = [\const_{p e}]\). Thus there exists a homotopy
\(\eta: f \nat \const_{p e}\) and via
\cref{prop:homotopy-lift-via-covering-space} we obtain a lifted homotopy of
loops \(\lift\eta: \lift f \nat \const_e\), showing that
\([\lift f] = [\const_e]\). Therefore \(p\) is indeed a monomorphism of groups.
\end{proof}

\begin{proposition}
\label{prop:covering-space-n-of-sheets-is-index}
Let \(E\) and \(B\) be path-connected spaces. The number of sheets of a pointed
covering space \(p: (E, e) \to (B, b)\) is equal to the index of the subgroup
\(p_{*}\pi_1(E, e)\) in \(\pi_1(B, b)\).
\end{proposition}

\begin{proof}
Define the notation \(G \coloneq p_{*} \pi_1(E, e)\). Let
\(\gamma \in \Loop(B, b)\) be a loop and consider its lifted loop
\(\lift \gamma \in \Loop(E, e)\).
\todo[inline]{Continue}
\end{proof}

\begin{theorem}[Lifting out of connected space]
\label{thm:lifting-out-of-connected-space}
Let \(p: E \to B\) be a covering space, and \(f: Y \to X\) be a continuous map,
where \(Y\) is a connected space. Consider two \emph{lifts of \(f\) along
  \(p\)}: continuous maps \(\lift f_1, \lift f_2: Y \para E\) such that
the triangle
\[
\begin{tikzcd}
&E \ar[dd, "p"] \\
Y \ar[ru, shift left, "\lift f_1"]
\ar[ru, shift right, "\lift f_2"']
\ar[dr, "f"']
&
\\
&X
\end{tikzcd}
\]
commutes in \(\Top\). If there exists \(y \in Y\) such that
\(\lift f_1 y = \lift f_2 y\), then the lifts agree everywhere
\(\lift f_1 = \lift f_2\).
\end{theorem}

\begin{proof}
Consider the pullback \(E \times_B E\) of \(p\) with itself and consider the
uniquely defined morphism \((\lift f_1, \lift f_2): Y \to E \times_B E\)
making the diagram
\[
\begin{tikzcd}
Y \ar[rrd, bend left, "\lift{f}_1"]
\ar[ddr, "\lift{f}_2"', bend right]
\ar[rd, dashed]
& &
\\
&E \times_B E \ar[r] \ar[d]
\ar[rd, phantom, "\lrcorner", very near start]
&E \ar[d, "p"]
\\
&E \ar[r, "p"']
&B
\end{tikzcd}
\]
commute. Define \(\Delta_B E \subseteq E \times_B E\) for the diagonal of \(E\)
with respect to the fibre product. Using
\cref{lem:fibre-wise-diag-open-and-closed} we know that \(\Delta_B E\) is both
open and closed, thus
\(V \coloneq (\lift f_1, \lift f_2)^{-1}(\Delta_B E) \subseteq Y\) is both
open and closed in \(Y\), which is a non-empty set since by hypothesis
\(\lift f_1\) and \(\lift f_2\) agree at least in one point of
\(Y\). Since \(V\) is closed, then \(Y \setminus V\) is open in \(Y\) and
certainly disjoint from \(V\). Since their union is the whole space \(Y\), by
the hypothesis that \(Y\) is connected, it must be the case that \(V =
Y\). Therefore \(\lift f_1 = \lift f_2\) as wanted.
\end{proof}

\begin{proposition}
% \label{prop:}
Let \(q: E \to B \times I\) be a locally trivial covering space with
fibre \(F\). Then \(B\) admits an open cover \(\mathcal{U}\) for which \(q\) is
trivial over \(U \times I\) for each \(U \in \mathcal{U}\).
\end{proposition}

\todo[inline]{Prove when needed}

% \begin{proof}
% For each element of \(B \times I\) take an evenly covered neighbourhood
% \(U \times [a, b]\), and construct \(\mathcal U\) to be the set whose elements
% are these chosen neighbourhoods. Fix any point \(x_0 \in B\) and run \(t \in I\)
% \end{proof}

\section{Coverings and \texorpdfstring{\(G\)}{G}-Actions}

\begin{definition}[Properly discontinuous action]
\label{def:properly-discontinuous-action}
Given a discrete topological group \(G\), a left action \(G \times E \to E\) is
said to be \emph{properly discontinuous} if for each pair
\((g, x) \in G \times E\), where \(g \neq e\), there exists a neighbourhood
\(U \subseteq E\) of \(x\) such that
\[
U \cap g U = \emptyset.
\]
In particular, every properly discontinuous action is free.
\end{definition}

\begin{definition}[\(G\)-principal covering space]
\label{def:G-principal-covering}
Let \(G\) be a discrete topological group. A \emph{left \(G\)-principal covering
  space} is a covering \(p: E \to B\) together with a properly discontinuous
left action \(G \laction E\) for which \(p(g x) = p x\) for every pair
\((g, x) \in G \times E\), and such that the induced action on the fibres is
transitive.
\end{definition}

\begin{proposition}
\label{prop:properly-discontinuous-induces-covering}
If \(G \laction X\) is a properly discontinuous action of a discrete group \(G\)
on a space \(X\), the canonical projection \(q: X \epi X/G\) is a
\(G\)-principal covering space.
\end{proposition}

\begin{proof}
First of all, it is clear that \(q(g x) = q x \in X/G\) for any pair
\((g, x) \in G \times X\). Let \([x] \in X/G\) be any point, and
\(V \subseteq X\) be a neighbourhood of \(x\) such that
\(V \cap g V \neq \emptyset\) implies \(g = e\), and define \(U \coloneq q
V\). Notice that since \(G\) acts by topological isomorphisms, one has
\(q^{-1} U = \bigcup_{g \in G} g V\)---now, since each \(g V\) is open, it
follows that \(q^{-1} U \subseteq X\) is open. Since \(X/G\) has the quotient
topology, then \(U \subseteq X/G\) is open.

Consider \(q^{-1} U = \bigdisj_{g \in G} V_g\), where \(V_g \coloneq g
V\). Given any \(g \in G\), suppose there exists \(h \in G\) such that
\(V_g \cap V_h \neq \emptyset\)---then for any \(x' \in V_g \cap V_h\), one has
\(h^{-1} x' \in V_g \cap V\), and from construction this implies in
\(h^{-1} g = e\), thus \(h = g\).

It remains for us to show a trivialisation for \(q\)---we shall prove that
\(q|_{V_g}: V_g \to U\) is an isomorphism. Since quotient maps are open, it
suffices to show that \(q|_{V_g}\) is bijective. Let \([x'] \in U\) be any
point, and take \(x'' \in V\) such that \(q x'' = [x']\), then in particular
\(g x'' \in g V\) and \(q(g x'') = [g x''] = [x']\)---thus \(q|_{V_g}\) is
surjective. For injectivity, let \(x', x'' \in V\) be a pair of points such that
\(q(g x') = q(g x'')\), then \([x'] = [x'']\) in \(X/G\), which implies in the
existence of a point \(h \in G\) such that \(x' = h x''\), then
\(x'' \in V \cap V_h\)---therefore \(h = e\) and hence \(x' = x''\), thus in
particular \(g x' = g x''\) as wanted.
\end{proof}

\begin{definition}[Deck transformations]
\label{def:deck-transformations}
Given a covering space \(p: E \to B\), we define a group of automorphisms
\(\Aut(p)\) of the cover \(p\) to be composed of topological isomorphisms
\(\alpha: E \isoto E\) such that \(p \alpha = p\)---such maps are called
\emph{deck transformations} of the covering \(p\).
\end{definition}

\begin{example}[Translations]
\label{exp:left-translation-is-deck-transformation}
Given a left \(G\)-principal covering \(p: E \to B\), the \emph{left
  translation} of \(E\) by \(g \in E\), the isomorphism \(\ell_g: E \isoto E\)
mapping \(e \mapsto g e\), is a deck transformation of \(p\). The collection of
such maps \((\ell_g)_{g \in G}\) define a morphism of groups
\[
\ell: G \longrightarrow \Aut(p).
\]

Assume that \(E\) is a connected space. Let \(x \in B\) be any point and
consider any deck transformation \(\alpha \in \Aut(p)\). From the fact that
\(\alpha\) is bijective and \(p \alpha = p\), it acts as a permutation on the
fibre \(p^{-1} x\). Since \(p\) is a \(G\)-principal covering, then \(G\) acts
transitively on the fibres: hence given any two points \(e, e' \in p^{-1} x\),
since \(\alpha e \in p^{-1}(p e)\), it follows that there exists \(g \in G\)
such that \(\alpha e' = g(\alpha e)\). Therefore \(\alpha\), using the
connectedness of \(E\), is uniquely defined by its image under a single
point. This shows that, under these assumptions, \(\ell\) is an isomorphism
\(G \iso \Aut(p)\).
\end{example}

\begin{proposition}
\label{prop:automorphism-grp-covering-properties}
Let \(p: E \to B\) be a covering space. The following are properties concerning
the automorphism group of \(p\):
\begin{enumerate}[(a)]\setlength\itemsep{0em}
\item If \(E\) is a connected space, then \(\Aut(p)\) has a properly
  discontinuous action on \(E\).

\item If \(B\) is locally path connected and \(H \subseteq \Aut(p)\) is a
  subgroup, then the induced map \(q: E/H \to B\) is a covering.
\end{enumerate}
\end{proposition}

\begin{proof}
\begin{enumerate}[(a)]\setlength\itemsep{0em}
\item Consider any point \(e \in E\) and deck transformation \(g \in
  \Aut(p)\). Since \(p\) is a covering, choose \(U \subseteq B\) to be an evenly
  covered neighbourhood of \(p e\) and let \(U_e\) be a sheet over \(U\) with
  \(e \in U_e\). Suppose there exists a point \(e' \in U_e \cap g U_e\) and
  notice that, since \(p g = p\), we have \(p e' = p(g^{-1} e')\).

\todo[inline]{Continue proof of the properties of the automorphism group of covering spaces.}
\end{enumerate}
\end{proof}

\begin{proposition}
\label{prop:prop-disc-and-1-connected-simply-connected-iso-grps}
Let \(E\) be a simply connected space. If \(G \laction E\) is a properly
discontinuous action of a discrete topological group \(G\), then the action
induces an isomorphism of groups
\[
\pi_1(E/G) \iso G.
\]
\end{proposition}

\begin{proof}
Let \([x] \in E/G\) be any base point, and choose \(x \in q^{-1}[x]\)---where
\(q: E \epi E/G\) is the canonical projection. Define a map
\(\psi_x: \pi_1(E/G, [x]) \to G\) such that \(\psi_x[\alpha] = g\) if and only
if \(\lift \alpha_x(1) = g x\)---where \(\lift \alpha_x: I \to E\) is the lift of
\(\alpha\) over \(q\). We now show that \(\psi_x\) is the required isomorphism
of groups:
\begin{itemize}\setlength\itemsep{0em}
\item (Well defined) Suppose that \([\alpha] = [\beta]\) and that
  \(\psi_x[\alpha] = g\) while \(\psi_x[\beta] = h\). This means that
  \(\lift \alpha_x(1) = g x\) and \(\lift \beta_x(1) = h x\), but since
  \(\lift \alpha_x = \lift \beta_x\), then \(g x = h x\) and hence
  \(h^{-1} g x = x\). Since \(G\) has a properly discontinuous action, there
  exists a neighbourhood \(U \subseteq E\) of \(x\) such that
  \(U \cap (h^{-1} g) U\) is \emph{non-empty}, thus it is necessarily the case
  that \(h^{-1} g = e\), thus \(h = g\). This proves that
  \(\psi_x[\alpha] = \psi_x[\beta]\).

\item (Injective) Suppose \(\psi_x [\alpha] = \psi_x [\beta]\), then
  \(\lift \alpha_x(1) = g x = \lift \beta_x(1)\) for some \(g \in G\). This
  shows that \(\lift \alpha_x \simhtrel{\Bd I} \lift \beta_x\), hence
  \(\alpha \simhtrel{\Bd I} \beta\) thus \([\alpha] = [\beta]\).

\item (Surjective) From the fact that \(\pi_1(E/G, [x])\) acts transitively on
  the fibres of \(q\), it folows that \(\psi_x\) is surjective.

\item (Group morphism) Let \(\psi_x[\alpha] = g\) and \(\psi_x [\beta] = h\) and
  define \(k \coloneq \psi_x[\beta \cdot \alpha]\), so that
  \[
  k x = (\lift{\beta \cdot \alpha})_x(1)
  = (\lift \beta_{\lift \alpha_x(1)} \cdot \lift \alpha_x)(1)
  = \lift \beta_{g x}(1)
  = g \lift \beta_x(1)
  = g h x,
  \]
  therefore \(k = g h\).
\end{itemize}
\end{proof}

\section{Fibre Transport}

\begin{definition}[Homotopy lifting property]
\label{def:homotopy-lifting-property}
A continuous map \(p: E \to B\) is said to heve the \emph{homotopy lifting
  property} (HLP) for a given space \(X\) if: for any homotopy \(\eta: X \times
I \to B\) and continuous map \(a: X \to E\) such that \(p a x = \eta(x, 0)\)
\emph{exists} a homotopy \(\delta: X \times I \to E\) such that the diagram
\[
\begin{tikzcd}
X \ar[r, "a"] \ar[d, "i_0"']
&E \ar[d, "p"] \\
X \times I \ar[ru, "\delta" description]
\ar[r, "\eta"']
&B
\end{tikzcd}
\]
commutes in \(\Top\)---the map \(i_0: X \emb X \times I\) is defined as
\(x \mapsto (x, 0)\)
\end{definition}

\todo[inline]{Check proof of these last fibration stuff---tom Dieck's book}

\begin{definition}[Fibration]
\label{def:fibration}
A continuous map \(p: E \to B\) is said to be a \emph{fibration} if it satisfies
the homotopy lifting property for every space.
\end{definition}

\begin{theorem}
\label{thm:coverings-are-fibrations}
A covering space is a fibration.
\end{theorem}

% \begin{proof}
% Let \(p: E \to B\) be a covering space and \(\eta: X \times I \to B\) be a
% homotopy and \(a: X \to E\) be a morphism with \(p a = \eta i_0\).
% \end{proof}

\begin{proposition}[Unique path lifting]
\label{prop:path-lifting}
Let \(p: E \to B\) be a covering space, and \(\gamma: I \to B\) be a path with
initial point \(\gamma 0 \coloneq p e \in B\). Then there exists a \emph{unique
  lifting} \(I \to E\) of \(\gamma\) that begins in \(e\). Furthermore, any two
paths \(u, v: I \para E\) with \(u 0 = v 0\) are \emph{homotopic} if and only if
their images are homotopic in \(B\).
\end{proposition}

\begin{lemma}
\label{lem:general-lifting-lemma}
Let \(p: E \to B\) be a covering space and define \(p e_0 \coloneq b_0\). Let
\(Y\) be a path connected and locally path connected space, and \(f: Y \to B\)
be a continuous map with \(f y_0 = b_0\). There exists a lift \(Y \to E\) of
\(f\) over \(p\) such that \(y_0 \mapsto e_0\) if and only if
\[
f_{*} \pi_1(Y, y_0) \subseteq p_{*} \pi_1(E, e_0).
\]
If such a lift exists, then it's unique.
\todo[inline]{See Munkres, page 478}
\end{lemma}

%%% Local Variables:
%%% mode: latex
%%% TeX-master: "../../../deep-dive"
%%% End:

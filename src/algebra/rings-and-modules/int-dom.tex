\section{Torsion}

\begin{definition}[Torsion]
    \label{def:torsion-module}
    Let \(R\) be an integral domain, and \(M\) be an \(R\)-module. We define the
    \emph{torsion} of \(M\) to be the submodule
    \[
        \torsion M \coloneq
        \{m \in M \colon r m = 0 \text{ for some } r \in R \setminus 0\}.
    \]
    If \(\torsion M = 0\), we say that \(M\) is of \emph{free-torsion}. On the
    contrary, if \(\torsion M = M\), then \(M\) is said to be of \emph{torsion}.
\end{definition}

\begin{definition}[Rank]
    \label{def:rank-of-general-module}
    Given an integral domain \(R\) and an \(R\)-module \(M\), we define the
    \emph{rank} of \(M\) to be the cardinality of the maximal \(R\)-linearly
    independent subset of \(M\).
\end{definition}

\begin{example}
    \label{exp:torsion-module-has-rank-zero}
    If \(R\) is an integral domain, and \(M\) is a torsion \(R\)-module, then
    \(\rank_R M = 0\). Indeed, for any singleton \(\{m\} \subseteq M\) there exists
    a non-zero \(r \in R\) such that \(r m = 0\), which shows that every singleton
    is \(R\)-linearly dependent.
\end{example}

\section{Noetherian Rings \& Modules}

\subsection{Chain Conditions}

Recalling the definition of Noetherian rings (see \cref{def:noetherian-ring}),
we can extend this notion to the environment of modules:

\begin{definition}
    \label{def:noetherian-module}
    Let \(R\) be a commutative ring. We say that an \(R\)-module \(M\) is
    \emph{Noetherian} if every submodule of \(M\) is finitely generated.
\end{definition}

\begin{proposition}
    \label{prop:equiv-conditions-noetherian}
    Let \(R\) be a commutative ring, and \(M\) be an \(R\)-module. The following are
    equivalent propositions:
    \begin{enumerate}[(a)]\setlength\itemsep{0em}
        \item \(M\) is a \emph{Noetherian} module.

        \item Every \emph{ascending chain} of submodules of \(M\) \emph{stabilizes}. In
              other words, if \((N_j)_{j \in \N}\) is a collection of submodules of \(M\)
              such that \(N_j \subseteq N_{j+1}\), then there exists an index \(j_0 \in \N\)
              such that \(N_j = N_{j+1}\) for all \(j \geq j_0\).

        \item Every non-empty collection of submodules of \(M\) has a \emph{maximal}
              element with respect to inclusion.
    \end{enumerate}
\end{proposition}

\begin{proof}
    \begin{itemize}\setlength\itemsep{0em}
        \item (a) \(\implies\) (b). Let \(M\) be Noetherian and define the module
              \(N \coloneq \bigcup_{j \in \N} N_j\), which is a submodule of \(M\). Since
              submodules of Noetherian rings are finitely generated, it follows that \(N\)
              is finitely generated. Let \(N = \langle n_1, \dots, n_k \rangle\) be its
              generating set. For all \(1 \leq i \leq k\), there must exist \(j_i \in \N\)
              such that \(n_i \in N_j\) for all \(j \geq j_i\). Taking the maximum
              \(j_0 \coloneq \max(j_1, \dots, j_k)\), one finds that \(n_i \in N_j\) for
              each \(1 \leq i \leq k\) and every \(j \geq j_0\). Therefore
              \(N \subseteq N_j\) for all \(j \geq j_0\), which implies that \(N_j = N\) for
              each of those indexes --- therefore the chain stabilizes.

        \item (b) \(\implies\) (c). We prove the contrapositive. Suppose there exists a
              non-empty collection \(\mathcal{N}\) of submodules of \(M\) admitting no
              maximal element. Let \(N_0 \in \mathcal{N}\) be any element. Inductively, for
              all \(j \geq 1\), define \(N_j \in \mathcal{N}\) such that \(N_{j-1}
              \subsetneq N_j\), that is, \(N_{j-1}\) is a \emph{proper} subset of \(N_j\)
              --- this is possible since \(N_{j-1}\) isn't maximal. The collection
              \((N_j)_{j \in \N}\) forms an ascending chain of submodules, but by
              construction does not stabilize.

        \item (c) \(\implies\) (a). Let \(N \subseteq M\) be any submodule. Since
              \((0) \subseteq N\) is a finitely generated submodule of \(N\), one can define
              a non-empty collection \(\mathcal{N}\) of finitely generated submodules of
              \(N\). From (c) one has that \(\mathcal{N}\) admits a maximum element, say
              \(W \coloneq \langle n_1, \dots, n_k \rangle\). Let \(n \in N\) be any element
              and consider the finitely generated submodule \(\langle n_1, \dots, n_k, n
              \rangle \in \mathcal{N}\). Since \(W\) is maximal, we have \(\langle n_1,
              \dots, n_k, n \rangle \subseteq W\) --- therefore \(n \in W\) and \(N
              \subseteq W\). Therefore \(N = W\) is finitely generated, which proves that
              \(M\) is Noetherian.
    \end{itemize}
\end{proof}

\begin{corollary}
    \label{cor:pid-is-noetherian}
    Every principal ideal domain \(R\) is a Noetherian module over itself, and thus
    every non-empty collection of ideals of \(R\) admits a maximal element.
\end{corollary}

\begin{lemma}[Quotient of Noetherian rings is Noetherian]
    \label{lem:quotient-is-noetherian}
    Let \(R\) be a Noetherian ring, and \(\ideal{a} \subseteq R\) be an ideal. Then
    the quotient ring \(R/\ideal{a}\) is Noetherian.
\end{lemma}

\begin{proof}
    From \cref{prop:image-of-noetherian-is-noetherian} we find that the canonical
    projection \(R \epi R/\ideal{a}\) implies that \(R/\ideal{a}\) is Noetherian.
\end{proof}

\begin{theorem}[Generalized Hilbert's basis theorem]
    \label{thm:general-hilbert-basis}
    Let \(R\) be a ring. Then \(R\) is Noetherian if and only if the polynomial ring
    \(R[x_1, \dots, x_n]\).
\end{theorem}

\todo[inline]{Prove generalized Hilbert's basis theorem}

\begin{corollary}
    \label{cor:noetherian-quotient-poly-ring}
    Let \(R\) be a Noetherian ring, and \(\ideal{a} \subseteq R[x_1, \dots, x_n]\)
    be an ideal of the polynomial ring. Then the quotient ring
    \(R[x_1, \dots, x_n]/\ideal{a}\) is Noetherian.
\end{corollary}

\begin{proof}
    Since \(R[x_1, \dots, x_n]\) is Noetherian by \cref{thm:general-hilbert-basis},
    applying \cref{lem:quotient-is-noetherian} we find that \(R[x_1, \dots,
            x_n]/\ideal a\) is Noetherian.
\end{proof}

\begin{corollary}
    \label{cor:finite-type-alg-noetherian}
    Every \emph{finite-type} algebra over a Noetherian ring is \emph{Noetherian}.
\end{corollary}

\subsection{Existence of Maximal Ideals}

\begin{proposition}
    \label{prop:commutative-rings-have-maximal-ideals}
    Let \(R\) be a commutative ring. If \(\ideal a\) is any \emph{proper} ideal of
    \(R\), then there exists a \emph{maximal ideal} \(\ideal m\) of \(R\) containing
    \(\ideal a\).
\end{proposition}

\begin{proof}
    Consider the collection \(I\) of proper ideals of \(R\) containing
    \(\ideal a\)---which is ordered by inclusion. Using this ordering, let
    \((\ideal a_j)_{j \in J}\) be the chain of all proper ideals with
    \(\ideal a_j \in I\) and \(\ideal a_j \subseteq \ideal a_{j+1}\). Define the set
    \(\ideal m \coloneq \bigcup_{j \in J} \ideal a_j\), which is again an ideal of
    \(R\). Since every \(\ideal a_j\) contains \(\ideal a\) and does \emph{not}
    contain \(1\), it follows that \(\ideal m\) contains \(\ideal a\) and also
    doesn't contain \(1\)---hence \(\ideal m\) is a proper ideal. Therefore
    \(\ideal m\) is a \emph{maximal} ideal of \(R\) containing \(\ideal a\), which
    proves the statement.
\end{proof}

\section{Localization}

\begin{definition}[Multiplicative subset]
    \label{def:multiplicative-subset}
    Given a commutative ring \(R\), a subset \(S \subseteq R\) is said to be
    \emph{multiplicative} if \(1_R \in S\) and \(S\) is closed under
    multiplication---that is, given \(s, s' \in S\), we have \(s s' \in S\).
\end{definition}

\begin{definition}[Localization]
    \label{def:localization-ring}
    Let \(R\) be a commutative ring, and \(S \subseteq R\) be a \emph{multiplicative
        subset}. A \emph{localization} of \(R\) over the set \(S\) is a morphism of
    \emph{commutative rings} \(L_S: R \to R[S^{-1}]\) such that:
    \begin{enumerate}[(a)]\setlength\itemsep{0em}
        \item For all \(s \in S\), the image \(L_S(s) \in R[S^{-1}]\) is a \emph{unit}.
        \item For every morphism of rings \(f: R \to K\) satisfying property (a), there
              exists a \emph{unique} ring morphism \(\phi: R[S^{-1}] \to K\) such that the
              diagram
              \[
                  \begin{tikzcd}
                      R \ar[r, "L_S"] \ar[rd, bend right, "f"']
                      &{R[S^{-1}]} \ar[d, "\phi", dashed] \\
                      &K
                  \end{tikzcd}
              \]
    \end{enumerate}
    Equivalently, consider the minimal equivalence relation \(\sim_{\text{frac}}\)
    on the set \(R \times S\) defined by \((r, s) \sim_{\text{frac}} (r', s')\) if
    and only if \(r s' - s r' = 0\). Then the localization of \(R\) under the
    multiplicative subset \(S\) is simply
    \[
        R[S^{-1}] = (R \times S)/{\sim_{\text{frac}}}.
    \]
\end{definition}

\begin{example}
    \label{exp:Z[1/p]/Z-finite-submodules}
    Let \(p \in \Z\) be a prime number, and consider the ring of fractions
    \(\Z[p^{-1}]\). Then every proper submodule of the \(\Z\)-module given by the
    quotient \(\Z[p^{-1}]/\Z\) is \emph{finite}.
    \todo[inline]{How to solve this?}
\end{example}

\begin{definition}
    \label{def:field-of-fractions}
    Let \(R\) be an integral domain. We define the \emph{field of fractions} of
    \(R\) to be the localization
    \[
        \Frac(R) \coloneq R[(R \setminus \{0\})^{-1}].
    \]
\end{definition}

It is immediate that \(\Frac(R)\) is a field, since a field is simply an
integral domain whose non-zero elements are units---and this exactly what we did
with the above localization.

\section{Principal Ideal Domains}

\subsection{Modules Over PID's}

\begin{theorem}
    \label{thm:module-over-PID}
    Let \(R\) be a principal ideal domain, and \(M\) be a free \(R\)-module with
    \(\rank_R M = m\) finite, and \(N \subseteq M\) a submodule. Then:
    \begin{enumerate}[(a)]\setlength\itemsep{0em}
        \item The submodule \(N\) is free with \(\rank_R N = n\) satisfying \(n \leq m\).
        \item There exists a basis \((y_1, \dots, y_m)\) of \(M\), and a collection
              \((a_1, \dots, a_m)\) of \(R\) such that \((a_1 y_1, \dots, a_n y_n)\) is a
              basis for \(N\) and \(a_1 \mid a_2 \mid \dots \mid a_n\).
    \end{enumerate}
\end{theorem}

\begin{proof}
    Fix a basis \((x_1, \dots, x_m)\) for \(M\) and define a collection of
    \(R\)-linear morphisms \((\pi_j: M \epi R)_{j=1}^n\) with
    \(\pi_j x_i = \delta_{i j}\).
    \begin{enumerate}[(a)]\setlength\itemsep{0em}
        \item Since \(R\) is a domain, \(M\) is torsion-free and so is \(N\). Since
              \(M\) has a finite rank, it must also be the case that \(N\) has a finite rank
              \(n\) satisfying \(n \leq m\)---which follows from
              \cref{prop:LI-leq-maximal-LI}. We proceed via induction on \(n\): for the base
              case \(n = 0\) we have \(N = 0\) and therefore \(N\) is free. We assume for
              the induction hypothesis that the proposition is true for some
              \(0 < n-1 < m\). Now we consider the case \(0 < n \leq m\): since \(n\) is
              non-zero, then \(N\) is a non-zero module and therefore there exists a
              non-zero \(x = \sum_{j=1}^n b_j x_j \neq 0\) element of \(N\). If
              \(1 \leq j_0 \leq n\) is an index such that \(b_{j_0} \neq 0\), then in
              particular \(\pi_{j_0}|_N\) is a non-zero map of the form \(N \to R\). Since
              \(R\) is a PID and \(\pi_{j_0} N \subseteq R\) is a non-zero submodule, then
              there must exist \(b_0 \in R \setminus 0\) for which \(\pi_{j_0} N = R
              b_0\). Since \(R\) is a domain, then \(\{b_0\}\) is a basis for the free
              \(R\)-module \(\pi_{j_0} N\). Then the short exact sequence
              \[
                  \begin{tikzcd}
                      0 \ar[r]
                      &\ker(\pi_{j_0}|_N) \ar[r, hook]
                      &N \ar[r, two heads, "\pi_{j_0}|_N"]
                      &R b_0 \ar[r]
                      &0
                  \end{tikzcd}
              \]
              ends with a free module and thus splits, hence
              \[
                  N \iso \ker(\pi_{j_0}|_N) \oplus R b_0.
              \]
              Since \(\rank\) is additive over direct sums, then
              \(n = \rank_R(\ker(\pi_{j_0}|_N)) + 1\) implies that
              \(\ker(\pi_{j_0}|_N) \subseteq M\) is a submodule of rank \(n-1\). From the
              inductive hypothesis it follows that \(\ker(\pi_{j_0}|_N)\) is a free
              module. Therefore \(N\) is the direct sum of two free modules, hence \(N\) is
              itself free.

        \item Notice that the case \(M = 0\) is trivial, thus we shall consider only
              \(m \geq 1\) and do induction on \(m\)---moreover, we'll assume that \(N\) is
              anon-zero submodule of \(M\), since the zero case is trivially satisfied. For
              the base case \(m = 1\), there exists an \(R\)-module isomorphism
              \(\phi: M \isoto R\). Then from the fact that \(R\) is a PID it follows that
              there exists \(a \in R\) such that \(\phi N = R a = a R\), and if
              \(u \coloneq \phi^{-1}(1)\) then \(M = R u\)---so that \(\{u\}\) is a basis
              for \(M\). Moreover,
              \[
                  N = a (R u) = (a R) u = (R a) u = R (a u),
              \]
              therefore \(\{a u\}\) forms a basis for \(N\).

              Now let \(m > 1\) and assume as a hypothesis for induction that the
              proposition is true for all modules of rank less than \(m\). For each
              \(\phi \in \Hom_{\Mod R}(M, R)\), let \(a_{\phi} \in R\) denote a ring element
              such that \(\phi N = R a_{\phi}\), and define a collection of principal ideals
              \[
                  \Sigma \coloneq (R a_{\phi})_{\phi \in \Hom(M, R)}.
              \]
              From our previous considerations in item (a), we know that there must exist a
              non-zero \(\pi_{j_0} \in \Hom(M, R)\)---thus \(\Sigma\) is non-empty and its
              elements are not all zero. Since \(R\) is Noetherian, it follows that
              \(\Sigma\) admits a maximal element \(R a_{\phi_1} \in \Sigma\). Define
              \(a_1 \coloneq a_{\phi_1}\) and let \(y \in N\) be such that
              \(\phi_1 y = a_1\), which exists because \(\phi_1 N = R a_1\). We'll show that
              \(a_1 \mid \varphi y\) for any \(\varphi \in \Hom(M, R)\). Let
              \(\varphi: M \to R\) be any such map, and let
              \(\ideal p \coloneq \langle a_1, \varphi y \rangle\) be an ideal of \(R\), so
              that there exists \(d \in R\) such that \(\ideal p = R d\) since \(\ideal p\)
              is principal. Therefore there exists a pair \(r_1, r_2 \in R\) such that
              \(d = r_1 a_1 + r_2 \varphi y\). Define a morphism \(\psi: M \to R\) given by
              \(\psi \coloneq r_1 \phi_1 + r_2 \varphi\) so that \(\psi y = d\) and hence
              \(d \in \psi N\) from the fact that \(y \in N\). Since \(a_1 \in R d\), then
              \(R a_1 \subseteq R d\) and from maximality we obtain the equality
              \(R a_1 = R d\), which proves that \(a_1 \mid d\)---while
              \(d \mid \varphi y\), thus \(a_1 \mid \varphi y\) as we wanted.

              In particular, the last paragraph shows that \(a_1 \mid \pi_j y\) for each
              projection \(1 \leq j \leq m\), thus we may find \(b_j \in R\) such that
              \(\pi_j y = a_1 b_j\). We shall define
              \[
                  y_1 \coloneq \sum_{j=1}^m b_j x_j.
              \]
              Notice then that \(y\) can be rewritten as
              \begin{equation}\label{eq:y=a1y1-rewrite}
                  y = \sum_{j=1}^m \pi_j(y) x_j
                  = \sum_{j=1}^m (a_1 b_j) x_j
                  = a_1 \sum_{j=1}^m b_j x_j
                  = a_1 y_1.
              \end{equation}
              Furthermore, one knows that \(a_1 = \psi y = \psi (a_1 y_1) = a_1 \psi y_1\),
              therefore since \(R\) is a domain it follows that \(\psi y_1 = 1\).

              We'll now show that \(M = R y_1 \oplus \ker \psi\). Let \(x \in M\) be any
              element and write \(x = \psi(x) y_1 + (x - \psi(x) y_1)\). Notice that
              \begin{equation}\label{eq:x-psixy1-in-kernel}
                  \psi(x - \psi(x) y_1) = \psi x - \psi(x) \psi(y_1) = 0
              \end{equation}
              so \(x - \psi(x) y_1 \in \ker \psi\). Therefore \(M = R y_1 + \ker \psi\). Now
              if \(r y_1 \in \ker \psi\) then from the fact that \(\psi y_1 = 1\) we have
              \(r = 0\), hence \(r y_1 = 0\) and thus \(R y_1 \cap \ker \psi = 0\). We thus
              conclude that \(M = R y_1 \oplus \ker \psi\).

              Further, we'll prove that \(N = R (a_1 y_1) \oplus (N \cap \ker \psi)\). Let
              \(x \in N\) be any element and as before rewrite it as \(x = \psi(x) y_1 + (x
              - \psi(x) y_1)\). Using the fact that \(\psi N = R a_1\), there exists \(b \in
              R\) such that \(\psi x = b a_1\)---then one has
              \[
                  x = (b a_1) y_1 + (x - (b a_1) y_1)
              \]
              where \(b a_1 y_1 \in R (a_1 y_1)\) and \(x - (b a_1) y_1 \in N \cap \ker \psi\)
              from \cref{eq:x-psixy1-in-kernel}. Therefore \(N = R (a_1 y_1) + (N \cap \ker
              \psi)\), and since \(R(a_1 y_1) \subseteq R y_1\) while \(N \cap \ker \psi
              \subseteq \ker \psi\) then \(R(a_1 y_1) \cap (N \cap \ker \psi) = 0\). Thus
              \[
                  N = R(a_1 y_1) \oplus (N \cap \ker \psi).
              \]

              Since \(\ker \psi \subseteq M\) then it is free and the rank of the direct sum
              yields \(\rank_R \ker \psi = m - 1\). Thus from the inductive hypothesis we
              conclude that there exists a basis \((y_2, \dots, y_m)\) of \(\ker \psi\) and
              a collection of ring elements \((a_2, \dots, a_n)\) such that
              \((a_2 y_2, \dots, a_n y_n)\) forms a basis for \(N \cap \ker \psi\), and
              \(a_2 \mid a_3 \mid \dots \mid a_m\). Since \(M = R y_1 \oplus \ker \psi\)
              then \((y_1, y_2, \dots, y_m)\) is a basis for \(M\), and since \(N = R (a_1
              y_1) \oplus (N \cap \ker \psi)\) we have that \((a_1 y_1, a_2 y_2, \dots, a_n
              y_n)\) is a basis for \(N\).

              To conclude the proof, we must show that \(a_1 \mid a_2\). Since
              \((y_1, \dots, y_m)\) is a basis of \(M\), let \(f: M \to R\) be an
              \(R\)-linear morphism such that \(f y_1 = f y_2 = 1\) while \(f y_j = 0\) for
              each \(3 \leq j \leq m\). From \cref{eq:y=a1y1-rewrite} we have
              \(a_1 = a_1 f y_1 = f(a_1 y_1) = f y\), therefore since \(y \in N\) then \(a_1
              \in f N\). This implies in \(R a_1 \subseteq f N\) but using maximality this
              yields the equality \(f N = R a_1\). We also know that \(a_2 = f(a_2 y_2)\) in
              a similar fashion, and since \(a_2 y_2 \in N\) then \(a_2 \in f N\). This
              concludes the proof that \(a_1 \mid a_2\), which was the last step for the end
              of the proof at large.
    \end{enumerate}
\end{proof}

\subsection{Decomposition Theorems}

\begin{theorem}[Fundamental theorem of invariant factors]
    \label{thm:fundamental-theorem-of-invariant-factors}
    Let \(R\) be a principal ideal domain and \(M\) a finitely generated
    \(R\)-module. Then the following holds:
    \begin{enumerate}[(a)]\setlength\itemsep{0em}
        \item There exists a number \(r \in \N\) and a collection of non-zero and
              non-invertible elements \((a_1, \dots, a_m)\) of \(R\) such that
              \[
                  M \iso R^r \oplus \Big( \bigoplus_{j=1}^m R/R a_j \Big),
              \]
              and \(a_1 \mid a_2 \mid \dots \mid a_m\). That is, \(M\) is isomorphic to the
              direct sum of a finite collection of cyclic \(R\)-modules. The elements
              \(a_j\) are said to be invariant factors of \(M\).

        \item The module \(M\) is torsion-free if and only if \(M\) is free.

        \item There exists an isomorphism
              \[
                  \torsion M \iso \bigoplus_{j=1}^m R/Ra_j.
              \]
              In particular, \(M\) is a torsion module if and only if \(r = 0\), and in such
              a case we have \(\Ann M = R a_m\)\footnote{The annihilator of a module \(M\)
                  is defined to be the collection
                  \[
                      \Ann M \coloneq \{r \in R \colon r m = 0 \text{ for all } m \in M\}.
                  \]}.
    \end{enumerate}
\end{theorem}

\begin{proof}
    Let \(M = \langle x_1, \dots, x_n \rangle\) and let
    \(e_j \coloneq (\delta_{ij})_{i=1}^n \in R^n\) be a base element of \(R^n\). Define a morphism
    \(\phi: R^n \to M\) as the map that sends \(e_j \mapsto x_j\) for each
    \(j\)---which is unique by the free module universal property. Since
    \(\{x_1, \dots, x_n\}\) generates \(M\), then \(\im \phi = M\) and \(\phi\) is thus
    surjective. Via the first isomorphism theorem we obtain
    \(R^n/\ker \phi \iso M\). Since \(\ker \phi\) is a submodule of \(R^n\) we can use
    \cref{thm:module-over-PID} to obtain that \(\ker \phi\) is free and has
    \(\rank(\ker \phi) \coloneq m \leq n\). Furthermore, there exists a basis
    \((t_1, \dots, t_n)\) of \(R^n\) and a list of ring elements
    \((a_1, \dots, a_m)\) such that \((a_1 t_1, \dots, a_m t_m)\) is a basis of
    \(\ker \phi\) satisfying \(a_1 \mid a_2 \mid \dots \mid a_m\). From the decomposition of
    free modules we obtain
    \[
        M \iso R^n/\ker \phi =
        \frac{R t_1 \oplus \dots \oplus R t_n}{R a_1 t_1 \oplus \dots \oplus R a_m t_m}
    \]
    Define a morphism of \(R\)-modules
    \(\psi: \bigoplus_{j=1}^n R t_j \to \big( \bigoplus_{j=1}^m R/Ra_j \big) \oplus R^{n-m}\)
    by mapping
    \[
        \psi(r_1 t_1, \dots, r_n t_n) \coloneq
        (r_1 + R a_1, \dots, r_m + R a_m, r_{m+1}, \dots, r_n).
    \]
    From the definition, this map is certainly surjective. Moreover, we can see that
    \[
        \ker \psi
        = \Big( \bigoplus_{j=1}^m R a_j \Big) \oplus \{0\}^{\oplus (n-m)}
        \iso \bigoplus_{j=1}^m R a_j,
    \]
    therefore by the first isomorphism theorem we find
    \[
        \frac{R t_1 \oplus \dots \oplus R t_n}{R a_1 t_1 \oplus \dots \oplus R a_m t_m}
        \iso \Big( \bigoplus_{j=1}^m R/R a_j \Big) \oplus R^{n-m}.
    \]
    Notice that if \(a \in R\) is an invertible element, then \(R a = R\) and hence
    \(R/Ra = 0\). Therefore the proposition of item (a) is stablished since we can
    cut off of the direct sum every summand \(R/Ra_j\) where \(a_j\) is invertible.
    Let \(b \in R\) be any element, then given any \(r + R b \in R/Rb\) we find that
    \[
        b(r + R b) = b r + R b = r b + R b = R b
    \]
    therefore \(R/Rb\) is a torsion \(R\)-module. This shows that \(M\) is
    torsion-free if and only if \(M \iso R^r\), which proves item (b). As we just
    noted, one has \(\torsion M \iso \bigoplus_{j=1}^m R/R a_j\), and \(M\) is a
    torsion module we have \(\Ann\)
    \todo[inline]{Notice that this isn't finished: it is part of the lost
        work---this chapter was the most affected by the losses.}
\end{proof}

\subsection{Chinese Remainder Theorem}

\begin{theorem}[Chinese remainder]
    \label{thm:chinese-remainder-theorem}
    Let \(R\) be a commutative ring, and \(\ideal{a}_1, \dots, \ideal{a}_k\)
    be ideals of \(R\) such that \(\ideal{a}_i + \ideal{a}_j = R\) for all
    \(i \neq j\). Then:
    \begin{itemize}\setlength\itemsep{0em}
        \item We have the equality \(\ideal{a}_1 \cap \dots \cap \ideal{a}_k =
              \ideal{a}_1 \cdot \ldots \cdot \ideal{a}_k\).
        \item The natural projection
              \(R \epi R/\ideal{a}_1 \times \dots \times R/\ideal{a}_k\) is
              \emph{surjective}, and induces a natural \emph{isomorphism} of rings
              \[
                  \frac{R}{\ideal{a}_1 \dots \ideal{a}_k}
                  \iso (R/\ideal{a}_1) \times \dots \times (R/\ideal{a}_{k})
              \]
    \end{itemize}
\end{theorem}

\begin{proof}
    In particular, we have \(\ideal{a}_j + \ideal{a}_k = R\) for all \(1 \leq j \leq
    k - 1\). Therefore, for all such indices there exists \(a_j \in \ideal{a}_k\)
    such that \(1 - a_j \in \ideal a_j\), hence
    \[
        (1 - a_1) \dots (1 - a_{k-1}) \in
        \ideal{a}_1 \cdot \ldots \cdot \ideal a_{k-1},
    \]
    and since \(a_j \in \ideal{a}_k\), then
    \(1 - \prod_{j=1}^{k-1} (1 - a_j) \in \ideal{a}_k\). This shows that
    \begin{equation}\label{eq:coprime-ideals-CRT}
        (\ideal{a}_1 \cdot \ldots \cdot \ideal{a}_{k-1}) + \ideal{a}_k = R.
    \end{equation}

    Notice that
    \(\ideal{a}_1 \cdot \ldots \cdot \ideal a_k \subseteq \ideal a_1 \cap \dots \cap
    \ideal a_k\), thus it remains to prove the other side of the inclusion. From our
    last paragraph, we know that
    \(\ideal a_1 \cap \dots \cap \ideal a_k \subseteq \ideal{a}_1 \cdot \ldots \cdot
    \ideal a_k \) for all \(k \geq 3\). Since \(k = 1\) is trivial, we just need to
    prove the case for \(k = 2\). Let \(\ideal{b}, \ideal{c} \subseteq R\) be ideals
    such that \(\ideal b + \ideal c = R\)---then there exists \(b_0 \in \ideal b\)
    and \(c_0 \in \ideal c\) such that \(b_0 + c_0 = 1\). If \(x \in \ideal{b} \cap
    \ideal c\), then \(x = b_0 x + c_0 x\), implying in \(x \in \ideal b \cdot
    \ideal c\). Thus \(\ideal b \cap \ideal c \subseteq \ideal b \cdot \ideal c\).

    We now prove the second assertion via induction on \(k\). For the base case
    \(k = 1\), the statement follows trivially from the first isomorphism
    theorem. Assume as the hypothesis of induction that the statement is true
    \(k - 1 > 1\)---that is, we have an isomorphism
    \(R/(\ideal a_1 \cdot \ldots \cdot \ideal a_{k-1}) \iso (R/\ideal a_1) \times
    \dots \times (R/\ideal a_{k-1})\). Consider the natural map
    \begin{equation}\label{eq:natural-projection-CRT}
        \pi: R \longrightarrow \frac{R}{\ideal a_{1} \cdot \ldots \cdot \ideal a_{k-1}}
        \times (R/\ideal a_k).
    \end{equation}
    From \cref{eq:coprime-ideals-CRT} the statement is reduced for the case of two
    ideals \(\ideal b \coloneq \ideal a_1 \cdot \ldots \cdot a_{k-1}\) and
    \(\ideal c \coloneq \ideal a_k\) such that \(\ideal b + \ideal c = R\). Let
    \(b, c \in R\) be any two elements---we shall show that there exists \(r \in R\)
    such that \(r - b \in \ideal b\) and \(r - c \in \ideal c\). Since \(\ideal b\)
    and \(\ideal c\) are relatively prime, let as before \(b_0 + c_0 = 1\) and
    define \(r \coloneq b_0 c + c_0 b\). Then, one has
    \begin{align*}
        r & = b_0 c + (1 - b_0) b = b + b_0 (c - b) \equiv b \pmod{\ideal b} \\
        r & = (1 - c_0) c + c_0 b = c + c_0 (b - c) \equiv c \pmod{\ideal c}
    \end{align*}
    since \(b_0 \in \ideal b\) and \(c_0 \in \ideal c\). Therefore
    \(r - c \in \ideal b\) and \(r - b \in \ideal c\) a wanted. This shows that the
    natural morphism of rings \(\pi\) (see \cref{eq:natural-projection-CRT}) is
    surjective. From our first considerations, we know that
    \(\ker \pi = \ideal b \cap \ideal c = \ideal b \cdot \ideal c\), therefore the
    first isomorphism theorem establishes that
    \(R/(\ideal b \cdot \ideal c) \iso (R/\ideal b) \times (R/\ideal c)\).
\end{proof}

\begin{corollary}[Chinese remainder for PIDs]
    \label{cor:chinese-remainder-for-PID}
    Let \(R\) be a principal ideal domain, and \(a_1, \dots, a_k \in R\) be elements
    such that \(\gcd(a_i, a_j) = 1\) for all pairs \(i \neq j\). Then the natural
    map \(r + (a_1 \cdots a_k) \mapsto (r + (a_1), \dots, r + (a_k))\) establishes an
    \emph{isomorphism} of rings
    \[
        R/(a_1 \cdots a_k) \iso \frac{R}{(a_{1})} \times \dots \times \frac{R}{(a_k)}.
    \]
\end{corollary}

\section{Unique Factorisation Domains}


%%% Local Variables:
%%% mode: latex
%%% TeX-master: "../../deep-dive"
%%% End:

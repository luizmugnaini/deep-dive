\section{Prelude}

\subsection{Behaviour of Maps}

Before we start our journey through differential calculus on several variables,
I would like to point out some really important definitions for classifying the
behavior of maps --- of which we'll use extensively.

\begin{definition}[Ultimately]
We say that a property \(P\) is ultimately satisfied by a function \(f\) over
a filter base \(\mathcal B\) if there exists a \(B \in \mathcal B\) such that
\(P(f|_B)\).
\end{definition}

\begin{definition}[Little-oh]\label{def: little oh}
A function \(f\) is said to be little-oh (or infinitesimal) of another
function \(g\), which we write as \(f =_{\mathcal B} o(g)\), if there exists
a function \(\alpha\) such that \(f(x) = \alpha(x) g(x)\) holds ultimately
over \(\mathcal B\), and \(\alpha\) is infinitesimal over \(\mathcal B\).
\end{definition}

\begin{definition}[Big-oh]\label{def: big oh}
Given functions \(f\) and \(g\), we say that \(f\) is big-oh of \(g\), and
write \(f =_{\mathcal B} O(g)\), if there exists a function \(\beta\) such
that ultimately over \(\mathcal B\) we have \(f(x) = \beta(x) g(x)\), and
\(\beta\) is ultimately bounded over \(\mathcal B\).
\end{definition}

\begin{definition}[Order over base]\label{def: asymp-order}
We say that functions \(f\) and \(g\) have the same order over \(\mathcal
B\), and write \(f \asymp_{\mathcal B} g\), if \(f =_{\mathcal B} O(g)\) and
\(g =_{\mathcal B} O(f)\), or equivalently, if exists \(a, b > 0\) such that
for some \(B \in \mathcal B\) we have \(a |g(x)| \leq |f(x)| \leq b|g(x)|\).
\end{definition}

\begin{definition}
Given functions \(f, g\), we say that \(f\) behaves asymptotically like \(g\)
over \(\mathcal B\), and write \(f \sim_{\mathcal B} g\), if there exists a
function \(\gamma\) such that \(\lim_{\mathcal B} \gamma(x) = 1\) and \(f(x) =
\gamma(x) g(x)\) ultimately over \(\mathcal B\).
\end{definition}

\subsection{Fundamental Inequalities}

Now we take a look at some fundamental inequalities that are used in some of the
proofs of \cref{sec:differentiable-maps}.

\begin{lemma}\label{lem: CI}
For \(x > 0\) we have
\begin{gather}
\label{eq: CI-1}
x^\alpha - \alpha x + \alpha - 1 \leq 0,\ \text{ for } 0 < \alpha < 1, \\
\label{eq: CI-2}
x^\alpha - \alpha x + \alpha - 1 \geq 0,\ \text{ for } \alpha < 0 \text{ or
} 1 < \alpha.
\end{gather}
\end{lemma}

\begin{proof}
Let \(f(x) = x^\alpha - \alpha x + \alpha - 1\), then \(f'(x) =
\alpha(x^{\alpha - 1} - 1)\). Notice that \(f'(1) = 0\) and that for \(\alpha
\in (0, 1)\) we have that for some \(\delta > 0\), \(f'(1 - \delta) > 0\) and
\(f'(1 + \delta) < 0\), which shows that \(x = 1\) is a strict maximum. In the
case where \(\alpha < 0\) or \(\alpha > 1\), \(f'(1 - \delta) < 0\) and \(f'(1
+ \delta) > 0\), showing that \(x = 1\) is a strict minimum. The strictness
comes from the fact that \(f\) is monotone in the intervals \(x \in (0, 1)\)
and \(x > 1\). Since \(f(1) = 0\), then for \(\alpha \in (0, 1)\) the function
is non-positive, and for \(\alpha < 0\) or \(\alpha > 1\) the function is
non-negative.
\end{proof}

\begin{proposition}[Young's inequalities]\label{prop: young-ineq}
Let \(a, b >0\) and \(p, q \not\in \{0, 1\}\) such that \(p^{-1} + q^{-1} =
1\). Then
\begin{gather}
\label{eq: young-1}
a^{p^{-1}} b^{q^{-1}} \leq \frac a p  + \frac b q,\ \text{ if } p > 1,
\\
\label{eq: young-2}
a^{p^{-1}} b^{q^{-1}} \geq \frac a p  + \frac b q,\ \text{ if } p < 1.
\end{gather}
The equality of such relations hold only when \(a = b\).
\end{proposition}

\begin{proof}
Let \(\alpha = p^{-1}\) and set \(x = \frac a b\). From \cref{eq: CI-1} we
have
\begin{align*}
0 &\geq \left(\frac a b\right)^{\frac 1 p} - \frac 1 p \frac a b + \frac 1 p
- 1 = \left(\frac a b \right)^{\frac 1 p} - \frac 1 p \frac a b - \frac 1 q
\\
\frac 1 p \frac a b + \frac 1 q
    &\geq \left( \frac a b \right)^{\frac 1 p}
\\
\frac a p + \frac b q
    &\geq a^{\frac 1 p} b^{1 - \frac 1 p} = a^{p^{-1}} b^{q^{-1}}.
\end{align*}
Now, from \cref{eq: CI-1} we have equivalently that
\begin{align*}
\frac a p + \frac b q \geq a^{p^{-1}} b^{q^{-1}}.
\end{align*}
\end{proof}

\begin{proposition}[H\H{o}lder's inequalities]\label{prop: holder-ineq}
Let \(x_j, y_j \geq 0\) for \(1 \leq j \leq n\) and \(p^{-1} + q^{-1} = 1\).
Then
\begin{gather}
\label{eq: holder-1}
\sum_{1 \leq j \leq n} x_j y_j \leq \left( \sum_{1 \leq j \leq n} x_j^p
\right)^{\frac 1 p} \left( \sum_{1 \leq j \leq n} y_j^q \right)^{\frac 1 q}\
\text{ for } p > 1, \\
\label{eq: holder-2}
\sum_{1 \leq j \leq n} x_j y_j \geq \left( \sum_{1 \leq j \leq n} x_j^p
\right)^{\frac 1 p} \left( \sum_{1 \leq j \leq n} y_j^q \right)^{\frac 1 q}\
\text{ for } p < 1 \text{ and } p \neq 0.
\end{gather}
If \(p < 0\), then we need the strictness \(x_j > 0\) for all \(1 \leq j \leq
n\). Equality is obtained for the case where \((x_j^p)_{j=1}^n\) and
\((y_j^q)_{j=1}^n\) are linearly dependent.
\end{proposition}

\begin{proof}
Define \(x = \sum_{j=1}^n x_j > 0\) and \(y = \sum_{j=1}^n y_j > 0\). We can
use \cref{eq: young-1} with \(a = \frac {x_j^p} x\) and \(b = \frac {y_j^q}
y\), for which we find that
\[
\frac{x_j}{x^{p^{-1}}} \frac{y_j}{y^{q^{-1}}} \leq \frac 1 p \frac{x_j^p} x
+ \frac 1 q \frac{y_j^q} y
\]
hence, summing such inequality over \(1 \leq j \leq n\) we find
\[
\frac{\sum_{j=1}^n x_j y_j}{x^{p^{-1}} y^{q^{-1}}} \leq \frac 1 p
\frac{\sum_{j=1}^n x_j^p} x + \frac 1 q \frac{\sum_{j=1}^n y_j^p} y = \frac
1 p + \frac 1 q = 1
\]
and finally \cref{eq: holder-1} is shown
\[
\sum_{j=1}^n x_j y_j \leq x^{p^{-1}} y^{q^{-1}}
= \left(\sum_{j=1}^n x_j^p\right)^{\frac 1 p}
\left(\sum_{j=1}^n y_j^q\right)^{\frac 1 q}.
\]
The same equivalent proof can be made with \cref{eq: young-2} for \cref{eq:
holder-2}. Since the equality of the Young's inequalities occurs only for \(a
= b\), we find that the linear dependence \(x_j^p = \frac x y y_j^q\) implies
in the equality of H\H{o}lder's inequalities.
\end{proof}

\begin{proposition}[Minkowski's inequalities]\label{prop: minkowski-ineq}
Let \(x_j, y_j \geq 0\) for \(1 \leq j \leq n\). Then
\begin{gather}
\label{eq: minkowski-1}
\left(\sum_{1 \leq j \leq n} (x_j + y_j)^p \right)^{\frac 1 p}
\leq \left(\sum_{1 \leq j \leq n} x_j^p \right)^{\frac 1 p}
+ \left(\sum_{1 \leq j \leq n} y_j^p\right)^{\frac 1 p}
\text{ for } p > 1,
\\ \label{eq: minkowski-2}
\left(\sum_{1 \leq j \leq n} (x_j + y_j)^p \right)^{\frac 1 p}
\geq \left(\sum_{1 \leq j \leq n} x_j^p \right)^{\frac 1 p}
+ \left(\sum_{1 \leq j \leq n} y_j^p\right)^{\frac 1 p}
\text{ for } p < 1 \text{, and } p \neq 0
\end{gather}
The equality occurs when \((x_j)_{1 \leq j \leq n}\) and \((y_j)_{1 \leq j
\leq n}\) are linearly dependent.
\end{proposition}

\begin{proof}
Notice that
\[
\sum_{j=1}^n (x_j + y_j)^p = \sum_{j=1}^n (x_j + y_j)(x_j + y_j)^{p-1}
= \sum_{j=1}^n x_j (x_j + y_j)^{p-1} \sum_{j=1}^n y_j (x_j + y_j)^{p-1}
\]
If \(p > 0\), then applying \cref{eq: holder-1} we find (noting that \(q =
\frac p {p - 1}\))
\[
\sum_{j=1}^n (x_j + y_j)^p \leq \left[
    \left(\sum_{j=1}^n x_j^p\right)^{\frac 1 p}
    + \left(\sum_{j=1}^n y_j^p\right)^{\frac 1 p}
\right] \left(\sum_{j=1}^n (x_j + y_j)^p\right)^{\frac 1 q}.
\]
For \(p < 1\) with \(p \neq 0\), from \cref{eq: holder-2} we get
\[
\sum_{j=1}^n (x_j + y_j)^p \geq \left[
    \left(\sum_{j=1}^n x_j^p\right)^{\frac 1 p}
    + \left(\sum_{j=1}^n y_j^p\right)^{\frac 1 p}
\right] \left(\sum_{j=1}^n (x_j + y_j)^p\right)^{\frac 1 q}.
\]
Now, dividing both inequalities by the term \(\left(\sum_{j=1} (x_j + y_j)^p
\right)^{\frac 1 q}\) we find respectively \cref{eq: minkowski-1} and
\cref{eq: minkowski-2}. The equality occurs the same as with H\H{o}lder
inequalities.
\end{proof}

\subsection{Fixed Points and Banach Spaces}

\begin{definition}[Fixed point]\label{def:fixed-point}
Let \(f: X \to X\) be any map. A fixed point of \(f\) is an element \(x \in X\)
such that \(f(x) = x\).
\end{definition}

\begin{theorem}[Fixed point theorem]\label{thm: fixed point}
Let \(I \subseteq \R\) be a closed set and \(f: I \to \R\) a
function such that \(f(I) \subseteq \R\) and for some fixed \(\theta
\in [0, 1)\) we have, for all \(x, y \in \R\):
\[
|f(x) - f(y)| \leq \theta |x - y|.
\]
Then there exists a unique fixed point \(c \in I\), that is \(f(c) = c\).
\end{theorem}

\begin{proof}
Let \(x_0 \in I\) and define the sequence \(x_n := f(x_{n-1})\) for all \(n
\geq 1\). We first show that \((x_n)_{n \in \N}\) is Cauchy. Let \(n >
m \geq 1\), then
\begin{equation}\label{eq: fixed point 1}
|x_n - x_m| = \left| \sum_{k=m}^n x_{k+1} - x_k \right|
\leq \sum_{k=m}^n |x_{k+1} - x_k| = \sum_{k=m}^n |f(x_k) - f(x_{k-1})|
\end{equation}
Since \(|f(x_k) - f(x_{k-1})| \leq \theta |x_k - x_{k-1}|\) and \(f(x_k) =
x_{k+1}\), we can make  \(\prod_{i=1}^k |f(x_i) - f(x_{i-1})| \leq \theta^k
\prod_{i=1}^n |x_i - x_{i-1}|\) and divide both the inequality by
\(\prod_{i=1}^k |f(x_i) - f(x_{i-1})| = \prod_{i=1}^k |x_i - x_{i-1}|\) in
order to obtain
\begin{equation}\label{eq: fixed point 2}
|f(x_k) - f(x_{k-1})| \leq \theta^k |x_1 - x_0|.
\end{equation}
Now we can substitute \cref{eq: fixed point 2} in \cref{eq: fixed point 1},
then
\[
|x_n - x_m| \leq \sum_{k=m}^{n-1} \theta^k |x_1 - x_0|
\]
Moreover, since \(\theta \in [0, 1)\) we have from the geometric series that
\(\sum_{k=0}^\infty \theta^k = \frac{1}{1-\theta}\), so we can conclude that
\[
|x_n - x_m| \leq \frac{\theta^m}{1 - \theta} |x_1 - x_0|.
\]
and thus \((x_n)_{n \in \N}\) is indeed a Cauchy sequence.

Let \(x_n \to c \in I\), since \(I\) is closed and thus \(\overline I = I\).
Since \(|f(x) - f(y)| \leq \theta |x - y|\) the function is Lipschitz
continuous, hence
\[
\lim_{n \to \infty} f(x_n) = f(c) = \lim_{n \to \infty} x_{n+1} = c
\]
and therefore \(c\) is a fixed point of \(f\).

For the uniqueness of the fixed point, let \(c_1, c_2\) be fixed points of
\(f\), then \(|f(c_1) - f(c_2)| = |c_1 - c_2| \leq \theta |c_1 - c_2|\) and
thus \((1 - \theta)|c_1 - c_2| \leq 0\), but \(\theta \in [0, 1)\), hence
\(c_1 = c_2\).
\end{proof}

\begin{corollary}
Let \(I \subseteq \R\) be closed and \(f : I \to I\) be a
differentiable function such that exists \(\theta \in [0, 1)\) for which
\(|f'(x)| \leq \theta\), for all \(x \in I\). Then there exists a unique fixed
point of \(f\).
\end{corollary}

\begin{proof}
Choose any distict points \(x, y \in I\), from the mean value theorem, there
exists \(x_0 \in (x, y)\) such that \(f(x) - f(y) = f'(x_0)(x - y)\), then
\(|f(x) - f(y)| \leq \theta |x - y|\), which satisfies the condition of
\cref{thm: fixed point}, hence the proposition holds.
\end{proof}

\begin{definition}
Let \((V, \norm \cdot)\) be a normed vector space. We say that a sequence
\((x_n)_{n \in \N} \subseteq V\) is Cauchy with respect to the norm \(\norm \cdot\) if for all
\(\varepsilon > 0\) there exists an index \(N \in \N\) such that, for all \(n, m \geq N\), we
have \(\norm{x_n - x_m} < \varepsilon\).
\end{definition}


\begin{definition}[Banach space]\label{def: Banach space}
A normed vector space \((V, \norm \cdot)\) is a Banach space if every Cauchy
sequence converges with respect to \(\norm \cdot\).
\end{definition}

\begin{definition}
Let \(B\) be a Banach space. A subset \(A \subseteq B\) is said to be closed
if the limit of every convergent sequence in \(A\) belongs to \(A\).
\end{definition}

\begin{definition}[Contraction]
\label{def:contraction}
Let \(B\) be a Banach space and \(0 < \theta < 1\), then a map \(f: B \to B\)
is said to be a \(\theta\)-contraction if for all \(v, w \in B\) we have
\[
\norm{f(v) - f(w)} \leq \theta \norm{v - w}.
\]
\end{definition}

\begin{theorem}[Banach fixed point]\label{thm: Banach fixed point}
Let \(B\) be a Banach space and \(A \subseteq B\) be a closed subset. Let \(f:
A \to A\) be a \(\theta\)-contraction. Then \(f\) has a unique fixed point.
\end{theorem}

\begin{proof}
The proof of the Banach fixed point is merely the same analogous proof as the
one developed in \cref{thm: fixed point}.
\end{proof}

\begin{proposition}[Fixed point stability]
\label{prop:fixed-point-stability}
Let \(A \subseteq B\) be a closed subspace of the Banach space \(B\). Let \(\Omega \subseteq B\) e
an open subspace of \(B\). Consider the collection \(\{f_{x} \in B(A, A) : x \in
\Omega\}\)   of \(\theta\)-contractions such that the map \(x \mapsto f_x(y)\) is continuous ---
that is, \(\lim_{x\to x_0} f_x(y) = f_{x_0}(y)\).
Then the solution map \(s: \Omega \to A\) defined as
\[
s(x) = y \text{ if and only if } f_x(y) = y
\]
is continuous at \(x_0\) --- that is, \(\lim_{x \to x_0} s(x) = s(x_0)\).
\end{proposition}

\begin{proof}
We know from \cref{thm: Banach fixed point} that --- given any \(x \in \Omega\) ---
the fixed point (unique) solution can be obtained as the limit of
a sequece recursively defined as \(y_j = f_x(y_{j-1})\) and \(y_0 \in A\) being
any element. This way, consider such sequence \((y_j)_{j=1}^{\infty}\) but define
\(y_0 = s(x_0)\). Notice that since \(\sum_{j=1}^n y_j - y_{j-1} = y_n - y_0\),
then we can write \(y_n\) in the following form
\[
y_n = \sum_{j=1}^n (y_j - y_{j-1}) + y_0
= \sum_{j=2}^n (f_x(y_{j-1}) - f_x(y_{j-2}))
+ y_0 = \sum_{j=1}^n \left(f_x^{j-1}(y_1) - f_x^{j-1}(y_0)\right) + y_{0}.
\]
Now observe that
\[
\sum_{j=1}^n \norm{f_x^{j-1}(y_1) - f_x^{j-1}(y_0)}
\leq \sum_{j=1}^n \theta^{j-1} \norm{y_1 -
y_0} = \frac{\norm{y_1 - y_0}}{1 - q}.
\]
That is, \(\norm{y_n - y_0} \leq \frac{\norm{y_1 - y_0}}{1 - q}\), hence --- since
\((y_j)_{j=1}^{\infty}\) converges to the fixed point of \(f_x\), we have
\begin{align*}
\norm{s(x) - s(x_0)} = \norm{f(s(x)) - f(s(x_0))}
\leq \frac{\norm{y_1 - y_0}}{1-q}
= \frac{\norm{f_x(s(x_0)) - f_{x_0}(s(x_0))}}{1 - q}.
\end{align*}
On the other hand we also know that \(\lim_{x \to x_0}f_x(y) = f_{x_0}(y)\) thus
\[
\lim_{x \to x_0} \norm{f_x(s(x_0)) - f_{x_0}(s(x_0))} = 0.
\]
This shows that \(\lim_{x \to x_0} \norm{s(x) - s(x_0)} = 0\) and therefore
\[
\lim_{x \to x_0} s(x) = s(x_0).
\]
\end{proof}

\begin{definition}
\label{def:banach-space-hom-set}
Let \(V\) and \(W\) be Banach spaces. We define the set \(B(V, W)\) as the
collection of all linear maps \(f: V \to W\).
\end{definition}

\section{Continuity}

\begin{remark}
This part will be mainly concerned with the euclidean metric space given by
\(\R^n\) and the metric
\[
d(x, y) = \sqrt{\sum_{j=1}^n (x_j - y_j)^2}
\]
where \(x = (x_j)_{j=1}^n, y = (y_j)_{j=1}^n \in \R\).
\end{remark}

\subsection{Compact sets in \texorpdfstring{\(\R^n\)}{Rn}}

\begin{definition}
A set \(K \subseteq \R^n\) is compact if for every open cover \(\mathcal C\)
of \(K\) there exists a finite subcover \(\mathcal U \subseteq \mathcal C\).
\end{definition}

\begin{definition}[General closed interval]
We define a closed interval \(I\) in \(\R^n\) to be the set
\[
I = \{x \in R^n : x_j \in [a_j, b_j], 1 \leq j \leq n\}.
\]
where \(a, b \in \R^n\) are the boundaries of the interval \(I\).
\end{definition}

\begin{proposition}\label{prop: closed-interval compact}
A closed interval in \(\R^n\) is compact.
\end{proposition}

\begin{proof}
Suppose for the sake of contradiction that \(\mathcal U\) is a cover of \(I\),
closed interval in \(\R^n\), such that \(\mathcal U\) doesn't admit a finite
subcover. Consider the set of bisections of \(I\) in which for each component
of the vectors \(x \in I\), that is, we create the sets \(I_j^1 = \{x \in I:
x_j \in [a_j, (a_j + b_j)/2]\}\) and \(I_j^2 = \{x \in I: x_j \in [(a_j +
b_j)/2, b_j]\}\) for each \(1 \leq j \leq n\), generating \(2^n\) subsets of
\(I\). Notice that since these sets are contained in \(I\), at least one of
those should not admit a finite subcover from \(\mathcal U\), otherwise \(I\)
would be compact. Hence define such set as \(I_1\). Now recursively do the
same bisection process for \(I_1\). We end up with a chain of nested intervals
\[
I \supset I_1 \supset I_2 \supset \dots
\]
each of which does not admit a finite subcover from \(\mathcal U\). Consider
the interval \(I_m = \{x \in \R^n: x_j \in [a^m_j, b_j^m], 1 \leq j \leq n\}\)
from the nested chain. Notice that for each \(1 \leq j \leq n\) we have that
the coordinate closed intervals form again a chain of nested intervals
\[
[a_j, b_j] \supseteq [a_j^1, b_j^1] \supseteq [a_j^2, b_j^2] \supseteq \dots
\]
hence \(\lim_{t \to \infty} d(a_j^t, b_j^t) = 0\). Since they form a nested
chain, their intersection is nonempty and therefore there exists a point
\(\zeta_j \in  [a_j^m, b_j^m]\) common to all such intervals. In doing so for
\(1 \leq j \leq n\) we find a point \(\zeta = (\zeta_j)_{j=1}^n \in \R^n\)
such that \(\zeta \in I_i\) for all \(i \geq 1\) and \(\zeta \in I\). From the
last assertion one sees that there must exist \(U \in \mathcal U\) with
\(\zeta \in U\), hence exists \(\varepsilon > 0\) such that
\(B_\zeta(\varepsilon) \subseteq U\). Now, since \(\lim_{t \to \infty}
d(a_j^t, b_j^t) = 0\), it must be true that there exists \(M > 0\) such that
for all \(m > M\) we have \(I_m \subseteq B_\zeta(\varepsilon) \subseteq U\),
which clearly covers finitely \(I_m\). This contradicts the assumption that
all \(I_m\) couldn't be finitely covered by a subcover of \(\mathcal U\). This
shows that we cannot pick a subset of \(I\) with such property, implying that
\(I\) itself should be compact.
\end{proof}

\begin{proposition}\label{prop: compact-close}
Let \(K\) be a compact set of \(\R^n\), then
\begin{enumerate}[(a)]
\item \(K\) is closed in \(\R^n\).
\item Any closed subset of \(K\) is compact.
\end{enumerate}
\end{proposition}

\begin{proof}
\begin{enumerate}[(a)]
\item Let \(y \in \R^n\) be any limit point of \(K\). Suppose that \(y
    \not\in K\). For each point in \(K\), say \(x\), denote \(U_x\) a
    neighbourhood. Consider the collection \(\mathcal U = \{U_x: x \in K\}\),
    which covers the set \(K\). Since \(K\) is said to be compact, there
    exists a finite subcover \(\mathcal U' = \{U_{x_1}, \dots, U_{x_m}\}
    \subseteq \mathcal U\). From the hypothesis \(y\) does not belong to
    \(K\), we can find a neighbourhood \(V_j\) of \(y\) for \(1 \leq j \leq
    m\) such that \(U_{x_j} \cap V_j = \emptyset\). Consider now the
    neighbourhood \(V = \bigcap_{1 \leq j \leq m} V_j\) of \(y\). Since \(K =
    \bigcup \mathcal U'\), we have \(K \cap V = \emptyset\), therefore \(y\)
    cannot be a limit point of \(K\), which is a contradiction. This implies
    that \(y \in K\), if not, problematic neighbourhoods \(V_j\) can be
    choosen.
\item Let \(C \subseteq \R^n\) be a closed set and \(C \subseteq K\). Let
    \(\mathcal G\) be an open cover of \(C\) in \(\R^n\). Notice that
    \(\mathcal U = \mathcal G \cup (\R^n \setminus C)\) is an open cover of
    \(\R^n\), which clearly covers \(K\). Therefore there exists a finite
    subcover of \(K\), \(\mathcal U' \subseteq \mathcal U\), but since \(C
    \subseteq K\), then \(\mathcal U'\) also covers \(C\). Since \((\R^n
    \setminus C) \cap C = \emptyset\), then \(\mathcal U' \setminus \{\R^n
    \setminus C\}\) is a finite subcover of \(C\) from \(\mathcal G\),
    therefore \(C\) is a compact set.
\end{enumerate}
\end{proof}

\begin{definition}\label{def:Rn-diameter}
The diameter of a set \(A \subseteq \R^n\) is defined to be
\[
d(A) = \sup_{x, y \in A} d(x, y).
\]
\end{definition}

\begin{definition}\label{def: bounded}
A set \(A \subseteq \R^n\) is said to be bounded if \(d(A)\) is finite.
\end{definition}

\begin{proposition}\label{prop: compact then bounded}
If \(K \subseteq \R^n\) is compact, then \(K\) is also bounded in \(\R^n\).
\end{proposition}

\begin{proof}
Let \(\mathcal B\) be the collection of all open balls around a given point
\(x \in \R^n\), the set \(\mathcal B\) covers \(\R^n\) and therefore also
covers \(K\). Notice that since \(K\) is compact, there exists a finite number
of open balls \(\mathcal B' \subseteq \mathcal B\) that cover \(K\), hence the
distance between any elements of \(K\) must be finite.
\end{proof}

\begin{theorem}[Heine-Borel]\label{heine-borel}
Let \(K \subseteq \R^n\) be any set. The following statements are equivalent:
\begin{enumerate}[(a)]
\item \(K\) is closed and bounded.
\item \(K\) is compact.
\end{enumerate}
\end{theorem}

\begin{proof}
Notice that the implication (b) \(\Rightarrow\) (a) is already proven by the
last two propositions (\cref{prop: compact-close} and \cref{prop: compact then
bounded}). Suppose now that \(K\) is a closed and bounded set. Since \(K\) is
bounded, there exists a closed interval \(I \supset K\). Since \(I\) is
compact (\cref{prop: closed-interval compact}) and \(K\) is closed we find
that \(K\) itself is compact (\cref{prop: compact-close}).
\end{proof}

\subsection{Limits}

\begin{remark}
In this subsection we shall denote a general set as \(X\).
\end{remark}

\begin{definition}[Limit]\label{def: limit-several}
Let \(f: X \to \R^n\) be a map. We say that \(x \in \R^n\) is the limit of
\(f\) over a filter base \(\mathcal B \subseteq  2^X\) if for every
neighbourhood \(V\) of \(x\) there exists \(B \in \mathcal B\) for which
\(f(B) \subseteq V\).
\end{definition}

\begin{definition}[Bounded]\label{def: bounded-several}
A map \(f: X \to \R^n\) is said to be bounded if \(f(X) \subseteq \R^n\) is
bounded.
\end{definition}

\begin{definition}[Ultimately bounded]\label{def: ulti-bounded-several}
Given a filter base \(\mathcal B \subseteq 2^X\), a map \(f: X \to \R^n\) is
ultimately bounded over the base \(\mathcal B\) if there exists \(B \in
\mathcal B\) for which \(f\) is bounded.
\end{definition}

\begin{proposition}[Unique limit]
A map can have at most one limit over a filter base.
\end{proposition}

\begin{proof}
Let \(f: X \to \R^n\) be a map and suppose that \(\lim_\mathcal{B} f(x) = a\)
and \(\lim_\mathcal{B} f(x) = b\), where \(a \neq b\). Since they are distinct
points, there must exist neighbourhoods \(V_a\) and \(V_b\) for which \(V_a
\cap V_b = \emptyset\). Now, remember from the definition that there must
exist \(B_a, B_b \in \mathcal B\) such that \(f(B_a) \subseteq V_a\) and
\(f(B_b) \subseteq V_b\). From the downward direction property of filter
bases, there exists \(B \subseteq B_a \cap B_b\) in \(\mathcal B\). Since
\(\emptyset \not\in \mathcal B\), then \(B \neq \emptyset\), hence \(f(B)
\subseteq V_a \cap V_b\) is nonempty, contradicting the assumption that there
could be chosen non-intersecting neighbourhoods of \(a\) and \(b\), which
implies that \(a = b\) in \(\R^n\).
\end{proof}

\begin{proposition}
If a map has a limit over a given filter base, then the map is ultimately
bounded over that filter base.
\end{proposition}

\begin{proof}
Let \(f: X \to \R^n\) be a map and \(\mathcal B \subseteq 2^X\) be a filter
base. Assume that \(\lim_\mathcal{B} f(x) = \ell \in \R^n\). For the sake of
contradiction, suppose that \(f\) is not ultimately bounded over \(\mathcal
B\). Let \(B_\ell(\varepsilon)\) be an open ball centered at \(\ell\) with
radius \(\varepsilon > 0\). From the definition of a limit over a filter,
there exists \(B \in \mathcal B\) for which \(f(B) \subseteq
B_\ell(\varepsilon)\), but \(\sup_{x, y \in B_\ell(\varepsilon)} d(x, y) = 2
\varepsilon\), which contradicts the fact that \(d(f(B))\) is not bounded in
\(\R\). Therefore, \(f\) needs to be ultimately bounded over \(\mathcal B\).
\end{proof}

\begin{corollary}\label{cor: flattening-limit}
Let \(f: X \to \R^n\) and \(\mathcal B\) be a filter base over \(X\). The map
has a limit \(y\) over \(\mathcal B\) if and only if each of the projection
functions \(\pi_j  f\) have limit \(y_j\). That is,
\[
\lim_\mathcal{B} f(x) = y \Leftrightarrow \lim_\mathcal{B} \pi_j  f(x)
= y_j,\ \text{ for } 1 \leq j \leq n
\]
\end{corollary}

\begin{definition}[Cauchy sequence]
A sequence \((x_j)_{j \in \N}\) of points in \(\R^n\) is a Cauchy sequence
if for every \(\varepsilon > 0\) there exists an index \(N \in \N\) for which
\(d(x_i, x_j) < \varepsilon\) for all \(i, j > N\).
\end{definition}

\begin{proposition}
A sequence \((x_j)_{j \in \N}\) of points in \(\R^n\) is Cauchy if and only if
\((x_j^i)_{j \in \N}\) is Cauchy for all \(1 \leq i \leq n\), where \(x_j =
(x_j^i)_{i=1}^n\).
\end{proposition}

\begin{proof}
Notice that since
\[
d(x_j^i, x_k^i) \leq d(x_j, x_k) \leq \sqrt{n} \max_{1 \leq i \leq n}
d(x_j^i, x_k^i)
\]
then, if \((x_j)_{j \in \N}\) is Cauchy, we have that the inequality
\(d(x_j^i, x_k^k) \leq d(x_j, x_k)\) implies that \((x_j^i)_{j \in \N}\) is
Cauchy for each \(1 \leq i \leq n\). Now, if in turn we have that for each
component the sequence \((x_j^i)_{j \in \N}\) is Cauchy, then since \(d(x_j,
x_k) \leq \sqrt{n} \max_{1 \leq i \leq n} d(x_j^i, x_k^i)\) we find that
\((x_j)_{j \in \N}\) is Cauchy.
\end{proof}

\begin{proposition}
A sequence in \(\R^n\) is convergent if and only if the sequence is Cauchy.
\end{proposition}

\begin{proof}
Suppose that \((x_j)_{j \in \N}\) is a convergent sequence in \(\R^n\) with
\(x_j \to x\). Let \(\varepsilon > 0\), choose any neighbourhood \(U\) of
\(x\) such that \(d(U) = \varepsilon\). Since the sequence converges, there
exists \(N \in \N\) for which \(x_j \in U\) for all \(j \geq N\), that is, for
all \(j, k \geq N\) we have \(d(x_j, x_k) < \varepsilon\). Hence we conclude
that \((x_j)_{j \in \N}\) is Cauchy. For the opposite case, let \((x_j)_{j \in
\N}\) be a Cauchy sequence. Then clearly there exists an element \(x\) for
which every neighbourhood contains infinitely many points of \((x_j)_{j \in
\N}\).
\end{proof}

\begin{definition}[Oscillation]
The oscillation of \(f: X \to \R^n\) on \(E \subseteq X\) is given by
\[
\omega(f, E) = d(f(E)).
\]
\end{definition}

\begin{theorem}[Cauchy criterion for several variables]
\label{thm: cauchy-criterion several}
Let \(f: X \to \R^n\) be a map and \(\mathcal B\) be a filter base over \(X\).
The map \(f\) has a limit over \(\mathcal B\) if and only if for all
\(\varepsilon > 0\) there exist \(B \in \mathcal B\) such that \(\omega(f, B)
< \varepsilon\).
\end{theorem}

\begin{proof}
Apply the Cauchy criterion for single variable maps on each of \(\pi_j  f\) for
\(1 \leq j \leq n\), now, using \cref{cor: flattening-limit}, we see that theorem
is true for \(f\).
\end{proof}

\begin{theorem}
Let \(g: Y \to R^n\) and \(f: X \to Y\) be mappings. Let filter basis
\(\mathcal B_Y\) on \(Y\) --- such that \(g\) has a limit over \(\mathcal
B_Y\) --- and \(\mathcal B_X\) on \(X\) such that for all \(B_Y \in \mathcal
B_Y\) there exists \(B_X \in \mathcal B_X\) for which \(f(B_X) \subseteq
B_Y\). Then the composition \(g  f: X \to \R^n\) has a limit over
\(\mathcal B_X\) and we have the relation
\[
\lim_{\mathcal B_X} g  f (x) = \lim_{\mathcal B_Y} g(y).
\]
\end{theorem}

\begin{proof}
Apply the property of the limit of the composition of single variable maps to
each of the \(\pi_j  g\) and \(\pi_j  f\). From \cref{cor: flattening-limit} we
see that the theorem is true for \(g\) and \(f\).
\end{proof}

\begin{definition}[Limit at infinity]
Let \(f: E \to \R^n\), where \(E \subseteq \R^m\). The filter base that yields
the limit \(x \to \infty\) is given by \(\mathcal B_\infty = \{\R^m \setminus
B_a(r) : r \in \R\}\) for any fixed point \(a \in \R^m\).
\end{definition}

\begin{definition}[Limit to infinity]
Let \(f: E \to \R^n\), where \(E \subseteq \R^m\), and a filter base
\(\mathcal B\) on \(E\). We say that \(f(x) \to_\mathcal{B} \infty\) if ---
given any fixed point \(y \in \R^n\) --- any open ball \(B_y(r) \subseteq
\R^n\) is such that there exists \(B \in \mathcal B\) for which \(f(B)
\subseteq \R^n \setminus B_y(r)\).
\end{definition}

\subsection{Continuity}

\begin{remark}
Throughout this subsection we shall assume that \(E\) is a subset of \(\R^m\).
\end{remark}

\begin{definition}[Continuous]\label{def: continuous-several}
A map \(f: E \to \R^n\) is said to be continuous at a point \(x \in E\) if for
every neighbourhood \(V\) of \(f(x)\) there exists a neighbourhood \(U
\subseteq E\) of \(x\) such that \(f(U) \subseteq V\).
\end{definition}

\begin{corollary}
A map \(f: E \to \R^n\) is continuous at a point \(x\) if and only if
\(\pi_j  f\) is continuous at \(x\) for each \(1 \leq j \leq n\).
\end{corollary}

\begin{definition}[Path]
We define a path in \(\R^n\) to be a continuous map between an interval \(I
\subseteq \R\) and \(\R^n\).
\end{definition}

\begin{definition}[Support]
We define the support of a path \(\gamma: I \to \R^n\) to be the image
\(\gamma(I)\).
\end{definition}

\begin{definition}[Oscillation at a point]
Let \(f: E \to \R^n\) be a map and \(x \in E\). We define the oscillation of
\(f\) at the point \(x\) as the limit
\[
\omega(f, x) = \lim_{r \to +0} \omega(f, B_a(r) \cap E).
\]
\end{definition}

\begin{proposition}[Local properties]
Let \(f: E \to \R^n\) be a map.
\begin{enumerate}[(a)]
\item \(f\) is continuous at \(x \in E\) if and only if \(\omega(f, x) =
    0\).
\item If \(f\) is continuous at a point \(x \in E\), then \(f\) is bounded
    in some neighbourhood \(U_x \cap E\) of \(x\).
\item Let set \(X \subseteq \R^m\) and \(Y \in \R^n\). Let \(g: Y \to \R^k\)
    be a continuous map at \(y \in Y\). Let \(f: X \to Y\) be continuous at
    \(x \in X\) and \(f(x) = y\). Then the map \(g  f: X \to R^k\) is
    continuous at \(x\).
\end{enumerate}
If the map is real valued, we also have more properties. Let \(f, g: E \to
\R\), then
\begin{enumerate}[(a)]
\item If \(f\) is continuous at a point \(\bar x \in E\), there exists a
    neighbourhood \(U \cap E\) of \(\bar x\) such that \(f(x)f(\bar x) > 0\)
    for all \(x \in U \cap E\).
\item If \(f\) and \(g\) are continuous at a point \(x \in E\), then for any
    \(\alpha, \beta \in \R\) we have that the linear combination \(\alpha f +
    \beta g: E \to \R\), their product \(f \cdot g: E \to \R\) and --- if \(g(x)
    \neq 0\) --- the quotient \(\frac f g: E \to \R\) are all continuous at
    the point \(x\).
\end{enumerate}
\end{proposition}

\begin{proof}
\todo[inline]{Write proofs: local properties}
\end{proof}

\begin{definition}[Uniformly continuous]\label{def:uniformly-continuous}
Let \(f: X \to Y\) be a map between metric spaces. We say that \(f\) is
uniformly continuous on \(X\) if for every \(\varepsilon > 0\) there exists
\(\delta > 0\) such that, for every subset \(E \subseteq X\) of diameter \(d(E)
< \delta\), we have an oscillation \(\omega(f, E) < \varepsilon\).
\end{definition}

\begin{theorem}[Heine-Cantor theorem]
\label{thm:heine-cantor}
Let \(f: X \to Y\) be a continous map between matric spaces \(X\) and \(Y\). If
\(X\) is compact, then \(f\) is uniformly continuous.
\end{theorem}

\begin{proof}
Let any \(\varepsilon > 0\). Since \(f\) is continuous, there exists, for any
\(x \in X\) a \(\delta_x > 0\) for which, if \(d_X(x, y) < \delta_x\), then
\(d_Y(f(x), f(y)) < \varepsilon/2\). Lets consider the open cover \(\mathcal U\)
of \(X\) consisting of the neighbourhoods \(U_x \coloneq \{y \in X : d_{X}(x, y)
< \frac {\delta_{x}} 2\}\) for each \(x \in X\). From definition, if \(X\) is
compact, then there exists a finite subcover \(\{U_{x_j}\}_{j=1}^n \subseteq
\mathcal U\) of \(X\). Define the minimum radius of the given neighbourhoods as
\(\delta \coloneq \min_{1 \leq j \leq n} \delta_{x_j}/2\).

Let \(x, y \in X\) be any points such that \(d_X(x, y) < \delta\). From the
finite subcover, we have that there exists \(1 \leq j_0 \leq n\) such that \(x
\in U_{x_{j_0}}\), which implies that \(d_X(x, x_{j_0}) < \delta_{x_{j_0}}/2\),
thus \(d_Y(f(x), f(x_{j_0})) < \varepsilon/2\), from construction. Notice however
that
\[
  d_X(y, x_{j_0}) \leq d_X(y, x) + d_X(x, x_{j_0})
  < \delta + \delta_{x_{j_0}}
  \leq \delta_{x_{j_0}},
\]
therefore it follows that \(d_Y(f(x_{j_0}), f(y)) < \varepsilon/2\). Using again
the triangle inequality we observe that
\[
  d_Y(f(x), f(y)) \leq d_Y(f(x), f(x_{j_0})) + d_Y(f(x_{j_0}), f(y))
  < \varepsilon.
\]
Therefore, \(f\) is indeed uniformly continuous with constant \(\delta\).
\end{proof}

\begin{definition}[Pathwise connected]\label{def: path-connected}
A set \(X \subseteq \R^n\) is pathwise connected if for all \(x, y \in E\)
there exists a path \(\gamma: I \to E\) with endpoints at \(x\) and \(y\) and
support in \(E\).
\end{definition}

\begin{definition}[Domain]\label{def: domain}
A domain in \(\R^n\) is an open pathwise connected subset of \(\R^n\).
\end{definition}

\begin{proposition}[Global properties]
\label{prop:global-properties-continuous-on-compact}
The following are global properties on continuous maps of several
variables. Let \(K \subseteq \R^m\) be a compact set and \(E \subseteq \R^m\)
be pathwise connected.
\begin{enumerate}[(a)]
\item A continuous map \(f: K \to \R^n\) is uniformly continuous.
\item A continuous map \(f: K \to \R^n\) is bounded on \(K\).
\item A continuous map \(f: K \to \R\) assumes its maximal and minimal
    values at least once in \(K\).
\item Let \(f: E \to \R\) be a continuous map and assume \(f(a) = A\) and
    \(f(b) = B\) at \(a, b \in E\). For any \(C \in [A, B] \subseteq \R\)
    there exists \(c \in E\) such that \(f(c) = C\).
\end{enumerate}
\end{proposition}

\begin{proof}
\todo[inline]{Write proofs: global properties}
(d) From the connectedness of \(E\), let \(\gamma: [x, y] \to E\) be a
continuous path such that \(\gamma(x) = a\) and \(\gamma(y) = b\). Consider
the composition of continuous maps \(f  \gamma: I \to \R\). Since \(f
\gamma(a) = A\) and \(f  \gamma(b) = B\), for any given \(C \in [A,
B]\), there exists \(z \in [x, y]\) such that \(f  \gamma(z) = C\), hence
there exists \(c = \gamma(z) \in E\) for which \(f(c) = C\).
\end{proof}

\begin{proposition}\label{prop: linear-continuous}
Every linear map of the form \(L: \R^m \to \R^n\) is continuous. Moreover, it
is uniformly continuous.
\end{proposition}

\begin{proof}
Let \(f: E \to \R^n\) be any map, then for all \(1 \leq j \leq n\) we have
\[
\norm{\pi_j  f(x)}_\R \leq \norm{f(x)}_{\R^n} \leq \sum_{j=1}^n
\norm{\pi_j  f(x)}_\R
\]
In particular, for the linear map \(L\) we have that --- given any \(x \in
\R^m\)
\[
\norm{L(x)}_{\R^n} = \norm{\sum_{j=1}^m x_j L(e_j)}_{\R^n}
\leq \sum_{j=1}^m \norm{x_j}_\R \norm{L(e_j)}_{\R^n}
\leq \norm{x}_{\R^m} \left( \sum_{j=1}^m \norm{L(e_j)}_{\R^n} \right).
\]
Hence \(L = O(\Id)\) --- where \(\Id: \R^m \to \R^m\) mapping \(x \mapsto x\). It
follows from this that as \(x \to x_0\), we have \(L(x - x_0) = L(x) - L(x_0) \to
0\). This shows that \(L\) is continuous at any point of \(\R^m\). Notice that
given any \(\varepsilon > 0\) if \(\norm{x - y}_{\R^m} < \frac \varepsilon \ell\) --- where \(\ell :=
\sum_{j=1}^m L(e_j)\) --- then
\[
\norm{L(x) - L(y)}_{\R^n} = \norm{\sum_{j=1}^m (x_j - y_j) L(e_j) }_{\R^n}
\leq \sum_{j=1}^m \norm{L(e_j)}_{\R^n} \norm{x_j - y_j}_\R
\leq \ell \norm{x - y}_{\R^m}
< \ell \frac \varepsilon \ell = \varepsilon
\]
where we conclude that \(L\) is uniformly continuous.
\end{proof}

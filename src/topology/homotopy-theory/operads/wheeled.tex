\section{Colouring the Portrait}

\begin{definition}[Colours \& Profiles]
\label{def:colors-profiles}
Fix, for the remaining of the chapter, a \emph{non-empty set} \(\Col\)---whose
elements will be called \emph{colours}. A \emph{\(\Col\)-profile} is a
\emph{finite\footnote{Empty sequences are also admitted.} sequence} of colours of
\(\Col\).
\end{definition}

\begin{notation}
\label{not:profiles-and-operations}
We denote a \(\Col\)-profile by \(\prof c = (c_1, \dots, c_n)\) or also
\(c_{[1, n]} = \prof c\) whenthe indexing matters. Some of the operations on
profiles are the following:
\begin{itemize}\setlength\itemsep{0em}
\item The lenght of the profile is denoted by \(|\prof c| = n\).
\item Given \(1 \leq j \leq n\), we define the notion of colour removal:
  \[
  \prof c \setminus c_j \coloneq (c_1, \dots, c_{i-1}, c_{i+1}, \dots, c_n).
  \]
\item Given another \(\Col\)-profile \(\prof d\), with \(|\prof d| = m\), we
  define the \emph{concatenation} of \(\prof d\) and \(\prof c\) to be the
  \(\Col\)-profile
  \[
  (\prof d, \prof c) \coloneq (d_1, \dots, d_m, c_1, \dots, c_n).
  \]
\item Given a permutation \(\sigma \in \Sym_n\), we define the action of
  \(\sigma\) on the profile \(\prof c\) to be the \(\Col\)-profile \(\sigma
  \prof c\) given by
  \[
  \sigma \prof c \coloneq (c_{\sigma(1)}, \dots, c_{\sigma(n)}).
  \]
\end{itemize}
\end{notation}

\begin{definition}[\(\Col\)-profile category]
\label{def:profile-category}
We define a category \(\ProfCol\) whose objects are \(\Col\)-profiles, and a
morphism \(\sigma: \prof c \to \prof d\) is a \emph{permutation} such that
\(\sigma \prof c = \prof d\).
\end{definition}

%%% Local Variables:
%%% mode: latex
%%% TeX-master: "../../../../deep-dive"
%%% End:

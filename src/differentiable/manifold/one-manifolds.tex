\section{Classification of 1-Manifolds}

\begin{definition}[Parametrization by arc-length]
    \label{def:parametrization-by-arc-length}
    Let \(I \subseteq \R\) be an interval, and \(M\) be a smooth manifold. We say that a
    \(C^{\infty}\)-morphism \(f: I \to M\) is a \emph{parametrization by arc-length} if the
    restriction \(f: I \to f I\) is a \(C^{\infty}\)-isomorphism, and if the velocity
    vector \(f_{*\, s} 1 \in T_{f s} M\) is \emph{unitary} for each \(s \in I\).
\end{definition}

\begin{lemma}
    \label{lem:image-components-arc-len-parametrization}
    Consider a pair of arc-length parametrizations
    \(I \xrightarrow f M \xleftarrow g J\). Then \(f I \cap g J\) has at most two
    connected components. One has the following properties concerning the number of
    connected components:
    \begin{enumerate}[(a)]\setlength\itemsep{0em}
        \item If \(f I \cap g J\) has only \emph{one} connected component, then one can
              \emph{extend} \(f\) to an arc-length parametrization of \(f I \cup g J\).
        \item If \(f I \cap g J\) has \emph{two} components, then \(M \iso S^1\).
    \end{enumerate}
\end{lemma}

\begin{proof}
    Consider the \(C^{\infty}\)-isomorphism
    \[
        g^{-1} f: f^{-1}(f I \cap g J) \isoto g^{-1}(f I \cap g J),
    \]
    sending open sets of \(I\) to open sets of \(J\), and with derivative \(\pm 1\)
    everywhere---from the definition of the parametrization. Let \(\Gamma\) be the
    pullback of the pair \((f, g)\)---that is, composed of pairs
    \((s, t) \in I \times J\) such that \(f s = g t\). Therefore \(\Gamma\) is a
    closed subset of \(I \times J\), with the product topology, and consists of line
    segments with slope \(\pm 1\) by the behaviour of \(g^{-1} f\). Since
    \(g^{-1} f\) is an isomorphism, it must be the case that the line segments don't
    end abruptly, but extend from edge to edge of \(I \times J\). From the fact that
    \(g^{-1} f\) is injective and constant derivative, it must be the case that
    \(\Gamma\) is composed of at most two line segments---furthermore, if \(\Gamma\)
    has two components, then they must have the same slope and the start and end
    edges has to be distinct. We analyse the number of components of \(\Gamma\):
    \begin{itemize}\setlength\itemsep{0em}
        \item If \(\Gamma\) has a single component, then one can extend \(g^{-1} f\) to
              a line \(\ell: \R \to \R\). Therefore the map
              \(g \ell: I \cup \ell^{-1} J \to f I \cup g J\) forms an extension of \(f\).

        \item If \(\Gamma\) has two components, let those be the line segments
              connecting \((a, \alpha) \to (b, \beta)\) and
              \((c, \gamma) \to (d, \delta)\)---whose points lie in \(I \times J\), with
              ends in the boundary of the square, that is, \(I = (a, d)\) and
              \(J = (\gamma, \delta)\) for instance. By merely a translation of \(J\), we
              may assume that \(\gamma = c\) and \(\delta = d\), for which we obtain the
              relations
              \[
                  a < b \leq c < d \leq \alpha < \beta.
              \]
              We may define a continuous maps \(\theta: [a, \beta] \to \R\) given by
              \(t \mapsto \frac{2 \pi t}{\alpha - a}\), and \(h: S^1 \to M\) mapping
              \[
                  h(\cos(\theta t), \sin(\theta t)) \coloneq
                  \begin{cases}
                      f t, & \text{if } a < t < d     \\
                      g t, & \text{if } c < t < \beta
                  \end{cases}
              \]
              which is well defined since \(f\) and \(g\) agree on \([c, d]\) due to the
              translation of \(J\). Since \(h S^1\) is a compact open set of \(M\), it
              follows that it must be the case that \(h S^1 = M\). Therefore, since the
              restrictions of \(f\) and \(g\) are \(C^{\infty}\)-isomorphisms, it follows
              that \(h\) is a \(C^{\infty}\)-isomorphism.
    \end{itemize}
\end{proof}

\begin{theorem}[Classification of \(1\)-manifolds]
    \label{thm:classification-of-1-manifolds}
    Any smooth connected \(1\)-manifold \(M\) is \(C^{\infty}\)-isomorphic to either
    the \emph{circle} \(S^1\) or to some \emph{interval} of real numbers:
    \([0, 1]\), \([0, 1)\), \((0, 1]\) or \((0, 1)\). In fact, the following is a
    complete classification list:
    \begin{enumerate}[(1)]\setlength\itemsep{0em}
        \item If \(M\) is compact without boundary, then \(M\) is
              \(C^{\infty}\)-isomorphic to a circle.

        \item If \(M\) is compact with boundary, then \(M\) is \(C^{\infty}\)-isomorphic
              to a closed interval.

        \item If \(M\) is non-compact without boundary, then \(M\) is
              \(C^{\infty}\)-isomorphic to an open interval.

        \item If \(M\) is non-compact with boundary, then \(M\) is
              \(C^{\infty}\)-isomorphic to a half-open interval.
    \end{enumerate}
\end{theorem}

\begin{proof}
    Given a parametrization by arc-length \(f': J \to M\), via
    \cref{lem:image-components-arc-len-parametrization} one can extend \(f'\) to a
    \emph{maximal} arc-length parametrization \(f: I \to M\)---so that \(I\) is the
    maximal interval \(f\) can be extended to.

    Assuming \(M \not\iso S^{1}\), suppose that there exists a limit point \(x\) of
    \(f I\) with \(x \in M \setminus f I\), so that \(f\) is not surjective. Let
    \(U\) be a neighbourhood of \(x\) and \(g: I' \to U\) be an arc-length
    parametrization of \(U\). From
    \cref{lem:image-components-arc-len-parametrization} we can use \(g\) to extend
    \(f\) to a parametrization \(f I \cup g J\), contradicting the hypothesis of
    maximality of \(f\).
\end{proof}

\begin{corollary}
    \label{cor:1-manifold-boundary-even-number-of-points}
    The boundary of a compact \(1\)-manifold has an \emph{even} number of points.
\end{corollary}

\begin{proof}
    Indeed, if \(M\) is a compact \(1\)-manifold with boundary, then it's
    isomorphic to the disjoint union of a collection of closed intervals, each of
    which has two boundary points, thus \(M\) has an even number of boundary points.
\end{proof}

%%% Local Variables:
%%% mode: latex
%%% TeX-master: "../../../deep-dive"
%%% End:

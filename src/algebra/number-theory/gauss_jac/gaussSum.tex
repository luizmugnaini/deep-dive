\section{Gau{\ss} Sums}

\begin{definition}
    Let \(\zeta := e^{\frac{2\pi i}{p}}\), the \(p\)th root of unity, and
    \(\chi\) be a character on \(\F_p\). For any \(a \in \F_p\) we define
    \[
        g_a(\chi) = \sum_{t \in \F_p} \chi(t) \zeta^{at}
    \]
    to be the Gau{\ss} sum of \(a\) on \(\F_p\) belonging to the character
    \(\chi\).
\end{definition}

\begin{proposition}
    Some properties of the Gau{\ss} sum:
    \begin{enumerate}[I.]
        \item If \(a \in \F_p^\ast\) and \(\chi \neq \varepsilon\) then
              \(g_a(\chi) = \chi(a^{-1})g_1(\chi)\).
        \item If \(a \in \F_p^\ast\) and \(\chi = \varepsilon\) then
              \(g_a(\varepsilon) = 0\).
        \item If \(a = 0\) and  \(\chi \neq  \varepsilon\) then  we have
              \(g_0(\chi) = 0\).
        \item  If \(a = 0\) and \(\chi = \varepsilon\) then \(g_0(\varepsilon) =
              p\).
    \end{enumerate}
\end{proposition}

\begin{proof}
    We prove each case:
    \begin{enumerate}
        \item For \(a \neq  0\) we have that
              \[
                  \chi(a)g_a(\chi) = \chi(a) \sum_{t \in \F_p} \chi(t) \zeta^{at}
                  = \sum_{t \in \F_p} \chi(at) \zeta^{at}
                  = \sum_{x = at \in \F_p} \chi(x) \zeta^x
                  = g_1(\chi)
              \]
              Therefore \(g_a(\chi) = \chi(a)^{-1} g_1(\chi)\).
        \item For \(a \neq  0\) and \(\chi = \varepsilon\) we have
              \[
                  g_a(\varepsilon) = \sum_{t \in \F_p} \varepsilon(t)\zeta^{at} =
                  \sum_{t \in \F_p}\zeta^{at} = \frac{\zeta^{ap} - 1}{\zeta^a - 1} = 0
              \]
              since \(a \not\equiv 0 \pmod{p} \Rightarrow \zeta^{a} \neq 1\).
        \item For \(a = 0\) we have  \(g_0(\chi) = \sum_{t \in \F_p} \chi(t)
              \zeta^{0t} = \sum_{t \in \F_p} \chi(t) = 0\).
        \item Now if \(a = 0\) and \(\chi = \varepsilon\) we have \(g_0(\chi) =
              \sum_{t \in \F_p} \varepsilon(t) \zeta^{0t} = p\).
    \end{enumerate}
\end{proof}

\begin{remark}
    We'll use \(g(\chi) := g_1(\chi)\) as a notation.
\end{remark}

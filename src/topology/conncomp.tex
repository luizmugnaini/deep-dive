\section{Connected Spaces}

\begin{definition}[Connected space]
\label{def:connected-space}
Let \(X\) be a topological space. We say that \(X\) is \emph{connected} if and
only if one of the following conditions hold:
\begin{enumerate}[(a)]\setlength\itemsep{0em}
\item The space \(X\) cannot be expressed as the union of two disjoint non-empty
open sets.
\item Every morphism \(f: X \to \{0, 1\}\) is constant --- where \(\{0, 1\}\) is
a space endowed with the discrete topology.
\end{enumerate}
\end{definition}

\begin{corollary}
The two conditions in \cref{def:connected-space} are equivalent.
\end{corollary}

\begin{proof}
Assume \(X\) satisfies the condition (a), and let \(x \in X\) be any
point. Suppose \(f(x) = 0\) (or \(f(x) = 1\), we do not loose generality by
choosing a point of the domain), and consider \(f^{-1}(0) \subseteq X\), which
must be open from the continuity of \(f\), thus \(f^{-1}(0)\) is a neighbourhood
of \(x\) in \(X\). If we suppose, for the sake of contradiction, that there
exists \(y \in X\) such that \(f(y) = 1\), then \(f^{-1}(1)\) is also open and
is a neighbourhood of \(y\) --- notice however that \(f^{-1}(0) \cap f^{-1}(1) =
\emptyset\), thus we arrive at a contradiction, there must be no point \(y\)
with image different than zero.

We prove the counter positive: not (b) implies not (a). Suppose \(f: X
\to \{0, 1\}\) is not constant, so that there exists two points \(x, y \in X\)
such that \(f(x) = 0\) and \(f(y) = 1\) (and therefore \(f\) is surjective),
notice however that, since \(f\) is continous,
\end{proof}

\begin{definition}[Connected components]
\label{def:connected-components}
Let \(X\) be a topological space. Define an equivalence relation \(\sim\) by,
given \(x, y \in X\), we have \(x \sim y\) if and only if there exists a
connected subspace of \(X\) containing both \(x\) and \(y\). The collection of
equivalence classes \(X/{\sim}\) is called \emph{connected components} of \(X\).
\end{definition}

\begin{definition}[Totally disconnected]
\label{def:totally-disconnected}
A space is said to be totally disconnected if the only connected subsets are
singletons.
\end{definition}

\begin{example}
\label{exp:Q-is-totally-disconnected}
The set of rational numbers \(\Q\) is totally disconnected.
\end{example}

\begin{notation}[Standard interval]
From now on, when talking about \emph{paths} and \emph{homotopies}, we shall
reserve the symbol \(I\) to denote the \emph{standard topological interval},
which is defined by
\[
  I \coloneq [0, 1] \emb \R.
\]
\end{notation}

\begin{definition}[Paths \& loops]
\label{def:path-loop}
A \emph{path} in a topological space \(X\) is any continuous map \(\gamma: I \to
X\). A \emph{loop} in \(X\) is a continuous map \(\ell: I \to X\) such that
\(\ell(0) = \ell(1)\).
\end{definition}

\begin{definition}[Path connected space]
\label{def:path-connected}
A topological space \(X\) is said to be \emph{path connected} if and only if
for all \(x, y \in X\) there exists a \emph{path} connecting \(x\) and \(y\).
\end{definition}

\begin{proposition}[Path connected equivalence relation]
\label{prop:path-connected-equiv-relation}
There exists an equivalence relation \(\sim\) on the topological \(X\) defined
by: given \(x, y \in X\), we have \(x \sim y\) if and only if there exists a
path in \(X\) connecting \(x\) and \(y\).
\end{proposition}

\begin{proof}
The constant path \(x: I \to X\) given by \(t \mapsto x\) is a path on \(X\),
thus \(x \sim x\). Let \(x \sim y\) and \(\gamma\) be a path from
\(\gamma(0) = x\) to \(\gamma(1) = y\), then we can define a map
\(\lambda: [0, 1] \to X\) given by \(\lambda(t) \coloneq \gamma(1 - t)\), so
that \(\lambda\) is both continuous, and \(\lambda(0) = y\) while
\(\lambda(1) = x\) --- thus \(\lambda\) is a path between \(y\) and \(x\),
therefore \(y \sim x\). Suppose now \(y \sim z\) and let \(\eta\) be a path
connecting \(y\) to \(z\). We define a map \(\phi: I \to X\) given by
\[
  \phi(t) \coloneq
  \begin{cases}
    \lambda(2t), &\text{ for } t \in [0, 1/2] \\
    \eta(2t - 1), &\text{ for } t \in [1/2, 1]
  \end{cases}
\]
which is surely continuous and connects both \(x\) and \(z\) --- thus \(x \sim
z\).
\end{proof}

\begin{definition}[Path connected components]
\label{def:path-connected-components}
Let \(X\) be a path connected topological space, and \(\sim\) be the equivalence
relation described in \cref{prop:path-connected-equiv-relation}. The collection
\(X/{\sim}\) is called the \emph{path connected components} of the space
\(X\). We denote the collection of all path components of \(X\) by \(\pi_0(X)\)
--- the collection of homotopy classes between maps \(* \to X\).
\end{definition}

\begin{definition}[\(\pi_0\) functor]
\label{def:pi0-functor}
The concept of connected components of a space induce a covariant functor
\(\pi_0: \Top \to \Set\) defined by mapping objects \(X \mapsto \pi_0 X\), and
morphisms \(f: X \to Y\) to \(\pi_0 f: \pi_0 X \to \pi_0 Y\) --- which is a well
defined map since, given a path component \(P\) of \(X\), the set \(f(X)
\subseteq Y\) is connected and therefore contained in a unique path component of
\(Y\).
\end{definition}

\begin{theorem}[Morphisms preserve connectivity]
\label{thm:morphisms-preserve-connectivity}
Let \(X\) be a (path) connected space and \(f: X \to Y\) be a topological
morphism. Then \(f(X) \subseteq Y\) is (path) connected.
\end{theorem}

\begin{proof}
Suppose that \(f(X)\) is not connected, and let \(g: f(X) \to \{0, 1\}\) be a
non-constant morphism --- in particular, this is equivalent to the condition of
\(gf: X \to \{0, 1\}\) being a non-constant morphism, thus implying in the
non-connectedness of \(X\).

On the other hand, assume now that \(X\) is path connected and let
\(u, v \in f(X)\) --- define \(x, y \in X\) so that \(f(x) = u\) and
\(f(y) = v\). Let \(\gamma: I \to X\) be a path connecting \(x\) and \(y\) ---
then \(f\gamma: I \to f(X)\) is a path connecting \(u\) and \(v\), which proves
the proposition.
\end{proof}

\begin{corollary}
\label{cor:connectedness-top-property}
Connectedness and path connectedness are both topological properties.
\end{corollary}

\begin{corollary}
\label{cor:quotient-path-connected-is-path-connected}
The quotient of a (path) connected topological space is (path) connected.
\end{corollary}

\begin{proposition}
\label{prop:connectivity-quotients}
Let \(f: X \epi Y\) be a morphism of topological spaces \(X\) and \(Y\). If
\(Y\) is connected and, for all \(y \in Y\), the fiber \(f^{-1}(y)\) is
connected, then \(X\) is connected.
\end{proposition}

\begin{proof}
Let \(g: X \to \{0, 1\}\) be a morphism. From the connectedness condition on the
fibers of \(f\), it follows that \(g\) must be constant throughout the fibers of
\(f\) --- this implies in the existence of a morphism \(g^{*}: Y \to \{0, 1\}\)
such that \(g = g^{*} f\). Since \(Y\) is connected, \(g^{*}\) must be constant,
therefore the composition \(g^{*} f\) is constant and so is \(g\) --- that is,
\(X\) is connected.
\end{proof}

\begin{proposition}
\label{prop:union-path-connected}
Let \(J\) be a set, and consider a collection \((X_j)_{j \in J}\) of (path)
connected topological spaces. Define the space
\(X \coloneq \bigcup_{j \in J} X_j\). If the intersection
\(\bigcap_{j \in J} X_j\) is non-empty, then \(X\) is (path) connected.
\end{proposition}

\begin{proof}
Suppose the intersection is non empty and let \(x \in \bigcup_{j \in J}
X_j\) be any point. We split the proposition into the two given cases.

Suppose \(X_j\) is connected for all \(j \in J\). Let \(f: X \to \{0, 1\}\) be a
morphism --- we want to show that it has to be constant. Since \(X_j\) is
connected, it follows that the restriction \(f|_{X_j}\) has to be constant, and,
since \(x \in X_j\), then \(f(x) = f(y)\) for all \(y \in X_j\). The fact that
this must be true for all \(X_j\) shows that \(f\) must be constant throughout
the whole set \(X\).

Suppose \(X_j\) is path connected for all \(j \in J\) --- that is, from
hypothesis, \(X_j\) is a path connected component of \(X\). We now prove that
the intersection is connected to each of the path connected components. For each
\(j \in J\), choose any point \(y \in X_j\). Since \(x \in X_j\), then
\(x \sim y\) and therefore every point of \(X\) is connected by a path.
\end{proof}

\begin{theorem}
\label{thm:connected-interval-real-space}
In the real space \(\R\), the only connected sets are intervals and singletons.
\end{theorem}

\begin{proof}
If \(A\) is not an interval, there must exist a pair \(x, y \in A\) for which
there is \(z \notin A\) with \(x < z < y\). This way, we can write \(A\) as the
union of two disjoint non-empty sets:
\(A = [A \cap (-\infty, z)] \cup [A \cap (z, \infty)]\) --- that is, \(A\) is
disconnected.

We now show that intervals are connected. For the sake of contradiction, assume
there exists an interval \(I\) such that there are disjoint and non-empty sets
\(A\) and \(B\) for which \(I = A \cup B\) --- that is, we assume \(I\) is
disconnected. Let \(x, y \in I\) be any two elements with \(x < y\) and
\(x \in A\), while \(y \in B\). By the archimedian principle, valid in \(\R\),
since the set \(C \coloneq [x, y) \cap A\) is both non-empty and bounded above,
thus \(C\) has a supremum, let \(s \coloneq \sup C\) --- where \(x < s \leq
y\). Notice that \(s \notin A\), since if so then \(s\) would not be the
supremum of \(C\), therefore \(s \in B\) --- which also cannot be the case,
since \(s\) would not, again, be the supremum of \(C\). We conclude that the
sets \(A\) and \(B\) cannot be constructed at all, hence there is no
disconnected interval in \(\R\).
\end{proof}

\begin{proposition}
\label{prop:path-connected-is-connected}
Any path connected space is connected.
\end{proposition}

\begin{proof}
Let \(X\) be a path connected space and \(f: X \to \{0, 1\}\) be any continuous
map. Let \(x, y \in X\) be any two points. Since \(X\) is path connected, there
exists a path \(\gamma: I \to X\) between \(x\) and \(y\). From the hypothesis
that \(f\) is continuous, \(f\) must be continous on \(\gamma(I)\), therefore
constant. Notice that, since this is true for all points of \(X\) we find that
\(f(x) = f(y)\) for all \(x, y \in X\).
\end{proof}

\begin{proposition}
\label{prop:path-conn-htpy-invariant}
Connectedness and path connectedness are \emph{homotopy invariants}.
\end{proposition}

\begin{proof}
Let \(f: X \isoto Y\) be a homotopy equivalence, and \(g: Y \isoto X\) be its
homotopy inverse --- furthermore, consider a homotopy \(h: fg \to \Id_Y\).

Suppose \(X\) is connected, we shall prove that \(Y\) is connected. Let
\(k: Y \to \{0, 1\}\) be any continuous map and \(y, y' \in Y\) be any two
points. Since \(X\) is connected, in particular the map \(k f: X \to \{0, 1\}\)
is constant --- therefore, \(k f g(y) = k f g(y')\). Since \(h\) is a homotopy,
then the induced maps \(h(y, -), h(y', -): I \para Y\) are such that
\(h(y, 0) = f g(y)\) and \(h(y, 1) = y\), while \(h(y', 0) = f g(y')\) and
\(h(y', 1) = y'\). We conclude that \(h(y, -)\) is a path from \(fg(y)\) to
\(y\) and \(h(y', -)\) is a path from \(fg(y')\) to \(y'\). Since \(k\) is
continuous, it follows that \(k f g(y) = k(y)\) and \(kfg(y') = k(y')\). Finally
we obtain \(k(y) = k(y')\), showing that \(k\) is constant.

Suppose \(X\) is path connected. Then \(f(X)\) is path connected and therefore
\(k\) must be constant in every point of \(f(X)\). Now, if
\(y \in Y \setminus f(X)\), we consider the induced map \(h(y, -): I \para Y\)
--- which is a path from \(fg(y) \in f(X)\) to \(y\). Therefore, every point of
\(Y \setminus f(X)\) can be connected by a path to a point of \(f(X)\), proving
that \(Y\) itself is path connected.
\end{proof}

\begin{proposition}[Products preserve connectedness]
\label{prop:products-preserve-connectedness}
Let \((X_{j})_{j \in J}\) be a collection of (path) connected topological
spaces. Then \(\prod_{j \in J} X_j\) is (path) connected.
\end{proposition}

\begin{proof}
Lets define the notation \(X \coloneq \prod_{j \in J} X_j\).

Suppose \((X_j)_{j \in J}\) is composed of connected spaces. Let
\(k: X \to \{0, 1\}\) be any continuous map. For every \(j_0 \in J\), let
\(p \in \prod_{j \in J \setminus j_0} X_j\) be any point --- which, from
construction, excludes the \(j_0\)-th coordinate from the original product space
\(X\). Consider the continuous map \(\iota_{j_0}: X_{j_0} \to X\) for which
\(\pi_j \iota_{j_0}(x) \coloneq \pi_j(p)\) for \(j \neq j_0\) and
\(\pi_{j_0}\iota_{j_0}(x) \coloneq x\) --- that is, \(\iota_{j_0}\) embedds
\(X_{j_0}\) in \(X\) where every coordinate, but \(j_0\), is fixed using the
pre-chosen point \(p\). Since \(k\) is continuous and \(X_j\) is connected, the
continuous map \(k \iota_{j_0}: X_j \to \{0, 1\}\) has to be constant. Now, if
we take any pair of points \(x, y \in X\), one sees that \(k(x) = k(y)\) from
the fact that \(k \iota_j\) has to be constant for all \(j \in J\) --- for the
suitable choice of inicial fixed point.

Suppose \((X_j)_{j \in J}\) is a collection of path connected spaces. Let
\(x, x' \in X\) be any two elements. For each \(j \in J\), let
\(\gamma_j: I \to X\) be a path connecting \(\gamma_j(x)\) to
\(\gamma_j(x')\). By the universal property of the product topology, there
exists a unique continuous map \(\gamma: I \to X\) such that
\(\pi_j \gamma = \gamma_j\) for every \(j \in J\). Since the product of the
paths, \(\gamma\), is a path from \(x\) to \(x'\), we conclude that \(X\) is
path connected.
\end{proof}

\begin{theorem}
\label{thm:connected-iff-cov-preserves-coprod}
A space \(X\) is \emph{connected} if and only if the covariant functor
\(\Hom_{\Top}(X, -)\) \emph{preserves coproducts}.
\end{theorem}

\todo[inline]{Prove connectedness iff preserves coproducts.}

\section{Compact Spaces}

\begin{definition}[Compact space]
\label{def:compact-space}
A topological space \(X\) is said to be \emph{compact} if for \emph{every} open
cover there exists a \emph{finite subcover}.
\end{definition}

\begin{proposition}[Image of compact space]
\label{prop:image-of-compact-is-compact}
If \(f: X \to Y\) is a topological morphism and \(X\) is compact, then \(f(X)\)
is compact in \(Y\).
\end{proposition}

\begin{proof}
Let \(\mathcal{C}\) be an open cover of \(f(X)\). Consider the preimage
collection \(\mathcal{U} \coloneq \{f^{-1}(V)\}_{V \in \mathcal{C}}\), which by
construction covers \(X\). Since \(X\) is compact, let \(\mathcal{U}'\) be the
finite subcover given by \(\mathcal{U}\). If we now consider the image
collection \(\mathcal{C}' \coloneq \{f(U)\}_{U \in \mathcal{C}'}\), we find that
\(\mathcal{C}'\) covers \(f(X)\) and is contained in \(\mathcal{C}\), therefore
a finite subcover.
\end{proof}

\begin{corollary}
\label{cor:compactness-topological-invariant}
Compactness is an \emph{invariant} property of topological spaces.
\end{corollary}

\begin{corollary}
\label{cor:quotient-of-compact-space-is-compact}
The quotient of a compact space is compact.
\end{corollary}

\begin{proof}
Since any quotient space is the image of a continuous projection, the
proposition follows from \cref{prop:image-of-compact-is-compact}.
\end{proof}

\begin{proposition}[Closed subset is compact]
\label{prop:closed-subset-compact}
In a compact space any closed subset is compact.
\end{proposition}

\begin{proof}
Let \(X\) be a space and \(A \subseteq X\) any closed subset. If \(\mathcal{U}\)
is an open cover of \(A\), then \(\mathcal{U} \cup \{X \setminus A\}\) is a
cover of \(X\). Moreover, since \(X\) is compact, there exists a finite subcover
\(\mathcal{U}' \subseteq \mathcal{U} \cup \{X \setminus A\}\). In particular,
since \(\mathcal{U}'\) covers \(X\), it also covers \(A\).
\end{proof}

\subsection{Checking if a Space is Compact}

\begin{definition}[Finite intersection property]
\label{def:finite-intersection-property}
A collection of sets \(\mathcal{A}\) is said to satisfy the
\emph{finite intersection property} if and only if for every finite collection
\(\{A_1, \dots, A_n\} \subseteq \mathcal{A}\), we have
\(\bigcap_{j=1}^n A_j \neq \emptyset\), that is, the intersection is non-empty
\end{definition}

\begin{proposition}
\label{prop:compact-iff-FIP}
A space is compact if and only if every collection of closed subsets satisfying
the finite intersection property has non-empty intersection.
\end{proposition}

\begin{proof}
Let \(X\) be a compact space and \(\mathcal{C}\) be any collection of closed
subsets satisfying the finite intersection property. Suppose, for the sake of
contradiction, that \(\mathcal{C}\) has an empty intersection --- thus the
complement of the intersection covers \(X\). Hence there must exist a finite
collection \(\{A_1, \dots, A_n\} \subseteq \mathcal{C}\) whose corresponding
complement covers \(X\) and therefore
\(\bigcup_{j=1}^n (X \setminus A_j) = X \setminus \big( \bigcap_{j=1}^n A_j \big)
= X\) --- then \(\bigcap_{j=1}^n A_j = \emptyset\), which contradicts the
hypothesis that \(\mathcal{C}\) satisfies the finite intersection property.

Suppose that the latter condition is true. Suppose there exists an open cover
\(\mathcal{U}\) of \(X\) that has no finite subcover. In particular, the
collection \(\mathcal{U}' \coloneq \{X \setminus U : U \in \mathcal{U}\}\) is
composed of closed sets and satisfies the finite intersection property --- thus
\(\bigcup_{U \in \mathcal{U}} (X \setminus U) = X \setminus \big( \bigcap_{U \in
  \mathcal{U}} U \big) = X\), which immediately implies that the intersection of
\(\mathcal{U}\) is empty, yielding a contradiction. Hence \(\mathcal{U}\) must
have a finite subcover.
\end{proof}

\begin{lemma}[Non-empty countable intersection]
\label{lem:cpct-countable-intersection-nested-non-empty}
Let \(X\) be a compact space and \(\{C_j\}_{j \in \N}\) be a countable
collection of non-empty subsets of \(X\) for which \(C_{j+1} \subseteq C_j\) for
every \(j \in \N\) --- that is, the sets are nested. Then the countable
intersection \(\bigcap_{j \in \N} C_j\) is non-empty.
\end{lemma}

\begin{proof}
\todo[inline]{Prove, this is an exercise}
\end{proof}

\begin{definition}[Proper map]
\label{def:proper-Top}
A continuous map \(f: X \to Y\) is said to be \emph{proper} if for all compact
sets \(C \subseteq Y\) the preimage \(f^{-1}(C) \subseteq X\) is compact.
\end{definition}

\begin{theorem}
\label{thm:hausdorff-compact-set-disjoint-neighbourhoods}
Let \(X\) be a Hausdorff space and \(x \in X\) any point. For every compact set
\(K \subseteq X \setminus \{x\}\) there exists two disjoint open sets \(U\) and
\(V\) such that \(x \in U\) and \(K \subseteq V\).
\end{theorem}

\begin{proof}
Since \(X\) is Hausdorff and \(x \notin K\), for any \(k \in K\) there exists
disjoint neighbourhoods \(U_x\) and \(V_k\) of \(x\) and \(k\),
respectively. Let \(\{V_{k}\}_{k \in K}\) be a collection of such neighbourhoods
--- which is also an open cover of \(K\). Since \(K\) is compact, there exists a
finite collection of points such that \(\{V_{k_1}, \dots, V_{k_n}\}\) is a
finite subcover. Defining \(V \coloneq \bigcup_{j=1}^n V_{k_j}\) and
\(U \coloneq \bigcup_{j=1}^n U_{k_j}\), we find that \(K \subseteq V\) and
\(x \in U\).
\end{proof}

\begin{corollary}[Hausdorff compact subsets]
\label{cor:hausdorff-compact-subset-is-closed}
Any \emph{compact} subset of a Hausdorff space is \emph{closed}.
\end{corollary}

\begin{proof}
If \(X\) is Hausdorff and \(C \subseteq X\) is a compact set, let
\(x \in X \setminus C\) be any point. From
\cref{thm:hausdorff-compact-set-disjoint-neighbourhoods} we know the existence
of a neighbourhood \(U\) of \(x\) that is disjoint from \(K\) --- thus
\(U \subseteq X \setminus C\) and hence \(C\) is closed.
\end{proof}

\begin{corollary}[Maps from compact to Hausdorff spaces]
\label{cor:map-compact-to-hausdorff-is-closed}
Let \(X\) be compact and \(Y\) Hausdorff. Every continuous map \(f: X \to Y\) is
\emph{closed} and \emph{proper}. Moreover, the following are consequential
properties:
\begin{enumerate}[(a)]\setlength\itemsep{0em}
\item If \(f\) is injective, then it is an \emph{embedding}.
\item If \(f\) is surjective, then it is a \emph{quotient map}.
\item If \(f\) is bijective, then it is an \emph{isomorphism}.
\end{enumerate}
\end{corollary}

\begin{proof}
We first prove that \(f\) is closed. Let \(C \subseteq X\) be any closed set of
\(X\), which is therefore compact. From \cref{prop:image-of-compact-is-compact}
we find that \(f(C)\) is compact and from
\cref{cor:hausdorff-compact-subset-is-closed} we have that \(f(C)\) is closed.

Now we prove that \(f\) is proper. Let \(K \subseteq Y\) be any compact set and
consider the preimage \(f^{-1}(K)\). Since \(Y\) is Hausdorff, as pointed
before, \(K\) is closed. Now since \(f\) is continuous, the preimage of closed
sets is closed --- hence \(f^{-1}(K)\) is closed. From the fact that \(X\) is
compact, we conclude that \(f^{-1}(K)\) is compact. For the last three
consequences, they come from \cref{prop:map-open-or-closed-properties}.
\end{proof}

\begin{lemma}[Tube lemma]
\label{lem:tube-lemma}
Let \(X\) be any space and \(Y\) be compact. For every \(x \in X\) and open set
\(U \subseteq X \times Y\) containing \(\{x\} \times Y\), there exists a
neighbourhood \(V \subseteq X\) of \(x\) such that \(V \times Y \subseteq U\).
\end{lemma}

\begin{proof}
Since the product of open sets form a basis for the product topology, for every
\(y \in Y\) there exists a neighbourhood \(V \times W \subseteq U\) of
\((x, y)\). Since \(\{x\} \times Y \iso Y\) and \(Y\) is compact, then
\(\{x\} \times Y\) is compact and therefore must exist a finite collection
\(\{V_j \times W_j\}_{j=1}^n\) of open sets of \(X \times Y\) covering
\(\{x\} \times Y\). Then if \(V \coloneq \bigcap_{j=1}^n V_j\) we find that
\(V \times Y \subseteq U\).
\end{proof}

\begin{theorem}[Closed projection]
\label{prop:compact-iff-projection-closed}
Given topological spaces \(X\) and \(Y\), the space \(X\) is compact if and only
if the canonical projection \(\pi: X \times Y \epi Y\) is closed.
\end{theorem}

\begin{proof}
Suppose \(X\) is compact, then if \(C \subseteq X \times Y\) is any closed set,
let \(y \in Y \setminus \pi(C)\) be any point --- we shall show that there
exists a neighbourhood of \(y\) outside of \(\pi(C)\). Consider the open set
\(U \coloneq X \times (Y \setminus \pi(C))\), which certainly contains
\(X \times \{y\}\). From \cref{lem:tube-lemma}, we find a neighbourhood
\(V \subseteq Y\) of \(y\) such that \(X \times V \subseteq U\), that is,
\(V \subseteq Y \setminus \pi(C)\), which settles that \(\pi(C) \subseteq Y\) is
closed.

Suppose now that \(X\) is a space such that \(\pi: X \times Y \epi Y\) is closed
for any space \(Y\). Let \(\mathcal{C}\) be any collection of closed subsets of
\(X\) satisfying the finite intersection property --- we'll show that
\(\bigcap_{C \in \mathcal{C}} C\) is non-empty. Define \(Y\) to be the space to
consisting of the underlying set \(X \cup \{*\}\) for some point \(*\) and the
topology given by \(2^X\) and \(\{C \cup \{*\} \colon C \in \mathcal{C}\}\). Let
\(K \subseteq X \times Y\) be the closure of the diagonal of \(X\) --- that is,
\(K \coloneq \Cl \Delta_X\). From hypothesis, \(\pi(K) \subseteq Y\) is
closed and from construction \(X \subseteq \pi(K)\).

We now show that \(* \in \pi(K)\). Suppose on the contrary that
\(* \notin \pi(K)\). Since \(\pi(K)\) is closed, we can find a neighbourhood
\(V \subseteq Y \setminus \pi(K)\) of \(*\) --- and therefore
\(V \cap X = \emptyset\). From the construction of the topology of \(Y\), one
could only hope to write \(V\) as the intersection of finitely many sets of the
form \(C \cup \{*\}\) for \(C \in \mathcal{C}\) --- on the other hand, since
\(\mathcal{C}\) satisfies the finite intersection property, for any finite
collection of sets of \(\mathcal{C}\) one can find \(x \in X\) such that
\(x \in C_1 \cap \dots \cap C_n\), therefore
\((C_1 \cup \{*\}) \cap \dots \cap (C_n \cup \{*\})\) still contains a point of
\(X\). This shows that it is impossible to build \(V\) out of such sets ---
hence we obtain a contradiction and thus \(* \in \pi(K)\) and there exists
\((x_0, *) \in K\) for some \(x_0 \in X\).

For every \(C \in \mathcal{C}\), any neighbourhood \(U \times (C \cup \{*\})\)
of \((x_0, *)\) has a non-empty intersection with the diagonal \(\Delta_X\) ---
thus \(C \cap U\) is non-empty. Moreover, \(x_0 \in C\) for all
\(C \in \mathcal{C}\), otherwise, since \(C\) is closed, one can find a
neighbourhood \(U \subseteq X \setminus C\) of \(x_0\), which should not be
possible. Therefore \(x_0 \in \bigcap_{C \in \mathcal{C}} C\) and by
\cref{prop:compact-iff-FIP} we conclude that \(X\) is compact.
\end{proof}

\subsection{Tychonoff Theorem}

\begin{lemma}
\label{lem:tychonoff-theorem-pre-lemma}
Let \(J\) be a set, and \((X_{j})_{j \in J}\) be a collection of topological
spaces. For any point \(x \in \prod_{j \in J} X_j\) and subset
\(A \subseteq \prod_{j \in J} X\) of the product space, we have \(x \in \Cl A\)
if, for every finite \(F \subseteq J\), we have \(\pi_F(x) \in \Cl(\pi_F(A))\)
--- where \(\pi_F: \prod_{j \in J} X_j \epi \prod_{j \in F} X_j\) is the
canonical projection map.
\end{lemma}

\begin{proof}
Suppose, on the contrary, that \(x \notin \Cl A\) --- then there exists a
neighbourhood \(U \subseteq \prod_{j \in J} X_j\) of \(x\) which is disjoint
from \(A\). From the definition of the product topology, the collection of
preimages \(\pi_j(U_j)\) for open sets \(U_j \subseteq X_j\) form a sub-basis of
the product space. In particular, from the basis properties, this allows for the
existence of an open set \(V \coloneq U_{j_1} \times \dots \times U_{j_n}\) such
that, denoting \(F \coloneq \{j_1, \dots, j_n\} \subseteq J\), the basis element
\(\pi^{-1}_F(V)\) of the product space is a neighbourhood of \(x\) and
\(\pi_F^{-1}(V) \subseteq U\). Therefore \(\pi_F^{-1}\) and \(A\) are disjoint,
which is equivalent to
\[A \subseteq \bigg( \prod_{j \in J} X_j \bigg) \setminus \pi_F^{-1}(V) =
\pi_F^{-1}\bigg( \prod_{j \in F} X_j \setminus V \bigg)
\]
that is, \(\pi_F(A) \subseteq \big( \prod_{j \in F} X_j \big) \setminus
V\). Since \(V\) is a neighbourhood of \(\pi_F(x)\) which is disjoint from
\(\pi_F(A)\) we conclude, by \cref{prop: closure equivalent prop}, that
\(\pi_F(x) \notin \Cl(\pi_F(A))\) --- which proves the lemma.
\end{proof}

\begin{theorem}[Tychonoff]
\label{thm:tychonoff-theorem}
The Cartesian product of a \emph{set} of compact topological spaces, endowed
with the product topology, is compact.
\end{theorem}

\begin{proof}
Denote by \(\{X_{\alpha}\}_{\alpha < \kappa}\) a collection of compact spaces
indexed by an ordinal \(\kappa\). We now show via induction on \(\kappa\) that,
for any space \(Y\), the canonical projection
\(Y \times \prod_{\alpha < \kappa} X_{\alpha} \epi Y\) is closed.

For the ease of notation we define
\(X^{\gamma} \coloneq Y \times \prod_{\alpha < \gamma} X_{\alpha}\) for all
\(\gamma \leq \kappa\) --- moreover for \(\lambda \leq \gamma\) we denote by
\(\pi_{\lambda}^{\gamma}: X^{\gamma} \epi X^{\lambda}\) the canonical projection
between such spaces. We also define that if \(C \subseteq X^{\kappa}\) is a
\emph{closed} set, then \(C_{\lambda} \coloneq \Cl(\pi_{\lambda}^{\kappa}(C))\)
--- thus our goal is equivalent of showing that \(\pi_0^{\kappa}(C) = C_0\).

Assume as inductive hypothesis that for all \(x_0 \in C_0\) there exists
\(x_{\lambda} \in C_{\lambda}\) for every \(\lambda < \kappa\) such that, if
\(\lambda < \gamma < \kappa\), then
\[
\pi_{\lambda}^{\gamma}(x_{\gamma}) = x_{\lambda},
\]
and in particular \(\pi_0^{\lambda}(x_{\lambda}) = x_0\). Again, equivalent to
our goal is to show that \(\pi_0^{\kappa}(x_{\kappa}) = x_0\).

If \(\kappa = \lambda + 1\) is a successor ordinal, then the projection
\(\pi_{\lambda}^{\kappa}: X^{\lambda} \times X_{\lambda} \epi X^{\lambda}\) is
closed from the fact that \(X_{\lambda}\) is compact, by
\cref{prop:compact-iff-projection-closed}. In particular,
\(\pi_{\lambda}^{\kappa}(C) \subseteq X^{\lambda}\) is closed and, by the
inductive hypothesis, \(\pi_{\lambda}^{\kappa}(C) = C_{\lambda}\) --- thus there
exists \(x_{\kappa} \in K\) for which
\(\pi_{\lambda}^{\kappa}(x_{\kappa}) = x_{\lambda}\), hence
\[
\pi_0^{\kappa}(x_{\kappa})
= \pi_0^{\lambda} \pi_{\lambda}^{\kappa}(x_k)
= \pi_0^{\lambda}(x_{\lambda})
= x_0,
\]
which was out goal.

Now, if \(\kappa\) is a limit ordinal, then
\(X^{\kappa} = \Lim_{\lambda < \kappa} X^{\lambda}\) together with
transitions maps \(\pi_{\lambda}^{\gamma}\). The limit of the tuple
\((x_{\lambda})_{\lambda < \kappa}\) defines a point
\(x_{\kappa} \in X^\kappa\), we wish to show that \(x_{\kappa} \in C\). For
every finite set of ordinals \(F\) below \(\kappa\) there exists
\(\lambda < \kappa\) above all ordinals of \(F\), therefore
\[
\pi_F(x_{\kappa})
= \pi_F^{\lambda}\pi_{\lambda}^{\kappa}(x_{\kappa})
= \pi_F^{\lambda}(x_{\lambda}).
\]
Moreover \(\pi_F^{\lambda}(x_{\lambda}) \in \pi_F^{\lambda}(C_{\lambda})\),
where from definition \(C_{\lambda} =
\Cl(\pi_{\lambda}^{\kappa}(C))\). Since \(\pi_F^{\lambda}\) is a
continuous map,
\[
\pi_F^{\lambda}(C_{\lambda})
= \pi_F^{\lambda}(\Cl(\pi_{\lambda}^{\kappa}(C)))
\subseteq \Cl(\pi_F^{\lambda} \pi_{\lambda}^{\kappa}(C))
= \Cl(\pi_F(C)).
\]
This shows that \(\pi_F(x_{\kappa}) \in \Cl(\pi_F(C))\), which by
\cref{lem:tychonoff-theorem-pre-lemma} shows that \(\pi(C)\) is closed.
\end{proof}

\subsubsection{Applications to Real \& Metric spaces}

\begin{lemma}[Closed intervals are compact]
\label{lem:closed-interval-is-compact}
Every closed and bounded interval in \(\R\) is compact.
\end{lemma}

\begin{proof}
Consider a closed interval \([a, b]\) in \(\R\) and let \(\mathcal{U}\) be a
cover for \([a, b]\). Define \(S\) to be the set of all points \(x \in [a, b]\)
for which the interval \([a, x]\) is covered by finitely many sets of
\(\mathcal{U}\) --- since there must exist a set \(U \in \mathcal{U}\) for
which \(a \in U\), then \(a \in S\). Since \(S\) is non-empty, by the
least upper bound property one can define a point \(x_0 \coloneq \sup S\).

Let \(U_0 \in \mathcal{U}\) be a set containing \(x_0\) and let
\(\varepsilon > 0\) be such that \((x_0 - \varepsilon, x_0] \subseteq
U_0\). Since \(x_0\) is the supremum of \(S\), there must also exist \(x \in S\)
such that \(x \in (x_0 - \varepsilon, x_0]\) --- that is, the interval
\([a, x_0]\) can be covered by finitely many sets of \(\mathcal{U}\), say
\([a, x_0] \subseteq U_0 \cup U_1 \cup \dots \cup U_n\), therefore
\(x_0 \in S\). If, for the sake of contradiction, \(x_0 < b\), then by the fact
that \(U_0\) is an open set, there must exist \(x \in U_0\) with \(x > x_0\) and
yet \(x \in [a, b]\) --- which implies that
\([a, x] \subseteq \bigcup_{j=0}^n U_j\) and therefore \(x_0\)
isn't the supremum of \(S\), leading to a contradiction. Thus \(x_0 = b\) and
hence \([a, b] \subseteq \bigcup_{j=0}^n U_j\) --- which proves the
proposition.
\end{proof}

\begin{corollary}[Heine-Borel]
\label{cor:heine-borel}
A subset of \(\R^n\) is compact if and only if it is both closed and bounded.
\end{corollary}

\begin{proof}
Let \(K \subseteq \R^n\) be a compact set. If we let \(\mathcal{U}\) be the
cover of \(\R^n\) by open balls centred at the origin --- with any real radius
--- in particular \(\mathcal{U}\) will be a cover for \(K\) and since \(K\) is
compact, there exists a finite subcover \(\mathcal{U}\) covering \(K\). From
this we conclude that there are only finitely many balls of real radius that
cover \(K\) --- hence \(K\) is necessarily bounded. From the Hausdorffness of
\(\R^n\), we obtain from \cref{cor:hausdorff-compact-subset-is-closed} that
\(K\) is closed.

On the other hand, let \(C \subseteq \R^n\) be a closed and bounded set. Since
\(C\) is a bounded set, its projection into each coordinate must also be
bounded, therefore one can obtain a collection of intervals
\(([a_j, b_j])_{j=1}^n\) in \(\R\) for which
\[
C \subseteq \prod_{j=1}^n [a_j, b_j].
\]
From \cref{lem:closed-interval-is-compact} we know that each \([a_j, b_j]\) is
compact --- thus by \cref{thm:tychonoff-theorem} we conclude that
\(\prod_{j=1}^n [a_j, b_j]\) is compact, but since \(C\) is a closed subset of a
compact set, by \cref{prop:closed-subset-compact} we find that \(C\) is compact.
\end{proof}

\begin{corollary}[Extreme values on compact sets]
\label{cor:extreme-values-on-compact-sets}
Let \(X\) be a compact space and \(f: X \to \R\) be a continuous map. Then \(f\)
is \emph{bounded} and attains its \emph{maximum} and \emph{minimum} values on
\(X\).
\end{corollary}

\begin{proof}
Since \(f(X) \subseteq \R\) is compact, then by \cref{cor:heine-borel} we obtain
that \(f(X)\) is both closed and bounded --- thus in particular \(f(X)\)
contains its supremum and infimum.
\end{proof}

We can even extend \cref{cor:heine-borel} to a more general context,
encompassing all metric spaces --- this is what the following proposition does.

\begin{proposition}
\label{prop:metric-space-compact-implies-bounded-closed}
In any metric space \(X\), if \(A \subseteq X\) is \emph{compact} then \(A\) is
both \emph{bounded} and \emph{closed} in \(X\).
\end{proposition}

\begin{proof}
Since every metric space is Hausdorff, \(A\) is closed by
\cref{cor:hausdorff-compact-subset-is-closed}. Moreover, if \(x \in A\) is any
point, we find that the collection of open balls
\(\mathcal{B}_x \coloneq \{B_x(n)\}_{n \in \N}\) forms an open cover of \(A\)
and, since \(A\) is compact, there exists a finite subcover of
\(\mathcal{B}_x\). Therefore there exists a maximal \(m \in \N\) for which
\(A \subseteq B_x(m)\) --- thus \(A\) is bounded.
\end{proof}

\begin{remark}
\label{rem:closed-and-bounded-not-compact}
Notice that a bounded and closed set in a metric space does \emph{not} need to
be compact, for instance, consider the space \(\ell^2(\N)\) (see
\cref{exp:p-norms}) and the subset \(A \coloneq \{f_n\}_{n \in \N}\) composed of
sequences \(f_n \in \ell^2(\N)\) such that \(f_n(j) \coloneq \delta_{n j}\) ---
that is, sequences where the only non-zero term is the \(n\)-th one, which equals
\(1\). Clearly \(A\) is bounded since \(\| f_{n} \|_2 = 1\), moreover, \(A\) is
closed because it has no limit points. However, since no subset of \(A\)
contains a limit point of \(A\), we find that \(A\) is not limit point compact
--- thus not compact (see \cref{thm:metric-2d-ctbl-hausdorff-equiv-compactness}).
\end{remark}

\subsubsection{Tychonoff \& The Axiom of Choice}

\begin{theorem}[Tychonoff \& Choice]
\label{thm:tychonoff-equivalent-choice}
Tychonoff's theorem is \emph{equivalent} to the axiom of choice.
\end{theorem}

\todo[inline]{Prove}

\subsection{Examples of Explicit Isomorphisms}

\begin{example}[Torus \(T^2\)]
\label{exp:T2-isomorphic-S1*S1}
The \(2\)-torus \(T^2\) is defined as the quotient
\(T^2 \coloneq (I \times I)/{\sim}\) where \(\sim\) is the smallest equivalence
relation on \(I \times I\) for which \((0, t) \sim (1, t)\) and
\((s, 0) \sim (s, 1)\) for all \(t, s \in I\). Let's prove that there exists a
topological isomorphism
\[
T^2 \iso S^1 \times S^1.
\]

First, lets consider the interval \(I\) and the equivalence relation
\(0 \sim_1 1\) so that \(I/{\sim_1}\) is just the interval with its ends glued to
a common point. Consider the morphism \(f: I \epi S^1\) given by
\(t \mapsto (\cos(2 \pi t), \sin(2 \pi t))\) and notice that \(f(1) = f(0) = 1\)
--- that is, \(f\) is constant on the fibers of the canonical projection
\(\pi: I \epi I/{\sim_1}\) --- moreover, \(f\) is certainly surjective since
cosine and sine are both maps of period \(2 \pi\). By
\cref{thm:universal-property-quotient-topology} we find a unique morphism
\(g: I/{\sim_1} \to S^1\) such that the following diagram commutes
\[
\begin{tikzcd}
I \ar[d, two heads] \ar[rd, bend left, "f"] & \\
I/{\sim_1} \ar[r, dashed, "\dis", "g"']     &S^1
\end{tikzcd}
\]
Notice that besides \(g\) being surjective, given \([t], [s] \in I/{\sim_1}\)
such that \(g([t]) = g([s])\) then \(t - s \in \Z\) since cosine and sine have a
period of \(2 \pi\) --- but since \(s, t \in I\), this can only be the case if
\(t = s\), thus \(g\) is injective. We therefore conclude that \(g\) is a
bijective map. Since \(I\) is compact, by
\cref{cor:quotient-of-compact-space-is-compact} we know that \(I/{\sim_1}\) is
compact. Moreover, the inclusion map \(S^1 \emb \R^2\) is continuous and induces
the subspace topology on \(S^1\) --- since \(\R^2\) is Hausdorff, then \(S^1\)
is Hausdorff. Therefore \(g\) is a bijective map from a compact space
\(I/{\sim_1}\) to a Hausdorff space \(S^1\), which by
\cref{cor:map-compact-to-hausdorff-is-closed} is an isomorphism.

Therefore, we find a unique isomorphism
\(g \times g: T^2 \isoto S^1 \times S^1\) such that the following diagram
commutes
\[
\begin{tikzcd}
I \times I \ar[d, two heads] \ar[rd, bend left, "f \times f"] & \\
T^2 \ar[r, dashed, "\dis", "g \times g"']                     &S^1 \times S^1
\end{tikzcd}
\]
which proves the isomorphism.

Let \(\R^2 \times \Z^2 \to \R^2\) be a group action of \(\Z^2\) on \(\R^2\)
given by
\[
((x, y), (n, m)) \mapsto (x + n, y + m),
\]
then the quotient of \(\R^2\) by such group action, which we'll denote by
\(\R^2/\Z^2\) is such that there exists a topological isomorphism
\[
T^2 \iso \R^2/\Z^2.
\]
Indeed, one sees right away that \(\R^2/\Z \iso (I \times I)/{\sim}\), where we
map \((x, y) + \Z^2 \mapsto (x - \lfloor x \rfloor, y - \lfloor y \rfloor)\) is
the explicit isomorphism. For free we then obtain an isomorphism
\[
S^1 \times S^1 \iso \R^2/\Z^2.
\]
\end{example}

% \begin{example}
% \label{exp:bicone}
% Consider the space \((I \times I)/{\sim}\) where \(\sim\) is the smallest
% equivalence relation on \(I \times I\) such that \((0, t) \sim (1 - t, 1)\) and
% \((s, 0) \sim (1, 1 - s)\) for every \(t, s \in I\). We'll prove that there
% exists an isomorphism
% \[
% (I \times I)/{\sim} \iso S^2.
% \]

% First notice that there exists an isomorphism
% \[
% (I \times I)/{\sim} \iso (S^1 \times [-1, 1])/{\sim_1},
% \]
% where \(\sim_1\) is the smallest equivalence relation on \(S^1 \times I\) for
% which \((p, 1) \sim_1 (q, 1)\) and \((p, -1) \sim (q, -1)\), for all
% \(p, q \in S^1\).

% Since translations in \(\R^2\) won't affect the topological properties of
% \(S^2\), suppose \(S^2\) is centered at \((1/2, 1/2, )\) --- so that \(I \times
% I\) Let \(f: I \times I \to S^2\) be the mapping
% \(u \mapsto \frac{u}{\| u \|_{\R^{2}}}\), which is certainly
% continuous. Moreover, for every \(t, s \in I\), one has
% \(f(0, t) = (0, 1) = f(t - 1, 1)\)
% \end{example}

\begin{example}[Projective space \(\R \Proj^2\)]
\label{exp:RP2-iso-S2-D2}
Let \(\sim_1\) and \(\sim_2\) be the smallest equivalence relations on,
respectively, \(S^2\) and \(D^2\) such that \(p \sim_1 -p\) for all
\(p \in S^2\), and \(v \sim_2 u\) if \(v = \lambda u\) for some
\(\lambda \in \R\), for all \(u, v \in D^2\). We'll show that there exists
topological isomorphisms
\[
\R \Proj^2 \iso S^2/{\sim_1} \iso D^2/{\sim_2}.
\]

For the first isomorphism, consider the continuous mapping
\(f_1: \R \Proj^2 \to S^2/{\sim_1}\) given by
\(\R \ell \mapsto \big[ \frac{\ell}{\| \ell \|}\big]\) --- which collapses the
line \(\R \ell\) to the class of the unitary vector \(\frac{\ell}{\| \ell \|}\)
generating the line, which ensures that \(f_1\) is surjective. Notice that
\(f_1\) is necessarily injective since, if \(f_1(\R \ell) = f_1(\R \ell')\),
then the lines \(\R \ell\) and \(\R \ell'\) intersect both at \(0\) and at the
common point of the sphere --- this implies in \(\R \ell = \R \ell'\). Notice
that the inclusion \(i_1: S^2/{\sim_1} \emb \R \Proj^2\) is continuous and is
the inverse of \(f_1\), sending each unitary vector \(v \in S^2/{\sim_1}\) of
the sphere quotient to the respective line generated by \(v\), namely
\(\R v \in \R \Proj^2\).

For the second isomorphism we make an analogous construction, take the map
\(f_2: D^2/{\sim_2} \to S^2/{\sim_1}\) given by the identification
\([v]_{\sim_2} \mapsto \big[ \frac{v}{\| v \|} \big]_{\sim_1}\), which is both
continuous and bijective. Moreover, the inclusion
\(i_2: S^2/{\sim_1} \emb D^2/{\sim_2}\) is a continuous map which is inverse to
\(f_2\), thus \(f_2\) establishes the required isomorphism.
\end{example}

\section{Sequential \& Limit Point Compactness}

\begin{definition}[Limit point compactness]
\label{def:limit-point-compact}
A space \(X\) is said to be \emph{limit point compact} if every infinite subset
of \(X\) has a limit point in \(X\).
\end{definition}

The following important theorem establishes that every compact space is limit
point compact.

\begin{theorem}[Bolzano-Weierstra{\ss}]
\label{thm:bolzano-weierstrass}
Let \(X\) be a compact space. Any \emph{infinite subset} \(S \subseteq X\) has a
limit point.
\end{theorem}

\begin{proof}
Suppose, for the sake of contradiction, that there exists \(S \subseteq X\),
infinite, with no limit points --- from this hypothesis, any point \(x \in X\)
is not a limit point of \(S\) and is either in or out of \(S\). In the former
case \(x \in S\) there exists a neighbourhood of \(x\), say \(U_x\), for which
\(U_x \cap S = \{x\}\). In the latter case \(x \notin S\), there must exist a
neighbourhood \(U_x\) of \(x\) such that \(S\) and \(U_x\) are disjoint. From
the law of excluded middle, the collection
\(\mathcal{U} \coloneq \{U_x\}_{x \in X}\) is an open cover of \(X\) --- on the
other hand, there exists no finite subcover of \(\mathcal{U}\) since each one
must only intersect \(S\) at a unique point, but \(S\) is infinite, hence the
contradiction. The infinite set \(S\) must therefore contain at least one limit
point.
\end{proof}

\begin{remark}
\label{rem:bolzano-weierstrass}
One should beware that the converse of \cref{thm:bolzano-weierstrass} does not
hold at all. A simple counterexample goes as follows: endow \(\R\) with the
topology given by \(\{\emptyset, \R\}\) and the open intervals
\(\{(x, \infty) \colon x \in \R\}\) --- in such a space \emph{any} set has a
limit point, although the space itself is not compact.
\end{remark}

\begin{definition}[Sequential compactness]
\label{def:sequentially-compact}
A space \(X\) is said to be \emph{sequentially compact} if every
sequence of points \((x_j)_j\) in \(X\), there exists a subsequence \((x_j')_j
\subseteq (x_j)_j\) that converges in \(X\).
\end{definition}

\begin{lemma}[In first countable Hausdorff spaces]
\label{lem:fst-countable-hausdorff-limit-pt-implies-seq-comp}
Let \(X\) be a first countable Hausdorff space. If \(X\) is limit point compact,
then \(X\) is sequentially compact.
\end{lemma}

\begin{proof}
Suppose \(X\) is indeed limit point compact and let \((x_j)_{j \in \N}\) be any
sequence of points in \(X\) and let \(S \coloneq \{x_j\}_{j \in \N}\) be the
set of values that the sequence attains. If \(S\) is finite, then the sequence
contains a constant subsequence --- which is therefore convergent in \(X\).

On the other hand, if \(S\) is infinite, then by the limit point compactness
property of \(X\) we find that there exists a limit point \(x \in X\) of
\(S\). If it is the case that \(x_j = x\) for infinitely many \(j \in \N\), then
the collection of such points form a constant sequence --- which converges to
\(x\). If the former is not the case, then at most a finite amount of points
\(x_j\) are equal to \(x\) --- therefore, one can discard these points from the
sequence and only consider the subsequence
\((x_j')_j \subseteq (x_j)_{j \in \N}\) such that \(x_j' \neq x\) for all
indices \(j\). Since \(X\) is assumed to be first countable, there must exist a
neighbourhood basis \((B_j)_{j \in \N}\) at \(x\).

We'll now construct a sequence \((x_{j_i})_{i \in \N} \subseteq (x_j')_j\) such
that \(x_{j_i} \in B_i\) and thus \(x_{j_i} \to x\). Since \(x\) is a limit
point, one can choose \(x_{j_0} \in B_0\). For the hypothesis of induction,
suppose we have chosen \(j_0 < j_1 < \dots < j_n\) indices so that \(x_{j_i} \in
B_i\). By \cref{prop:nbhd-limit-pt} we find that, since \(B_{n + 1}\) is a
neighbourhood of \(x\), then \(B_{x + 1}\) has infinitely many points of \(S\)
--- therefore, one can certainly choose \(j_{n+1} > j_n\) for which
\(x_{j_{n+1}} \in B_{n+1}\). Thus the sequence \((x_{j_i})_{i \in \N}\) was
successfully constructed so that it converges to \(x\).
\end{proof}

\begin{lemma}
\label{lem:metric-sequencially-comp-is-2nd-ctbl}
Every sequentially compact metric space is second countable.
\end{lemma}

\begin{proof}
Evoking \cref{prop: metric space properties} it's sufficient to show that a
sequentially compact metric space \(M\) is separable. We first show that, for
every \(\varepsilon > 0\), the open cover composed of \(\varepsilon\)-balls has
a finite subcover. For the sake of contradiction, assume there exists some
\(\varepsilon_0 > 0\) such that the statement is false.

We now construct a sequence \((x_j)_{j \in \N}\): choose any \(x_0 \in M\), now
since \(B_{\varepsilon_0}(x_0)\) does not cover \(M\), one can choose
\(x_1 \in M \setminus B_{\varepsilon_0}(x_0)\) --- since no finite collection of
\(\varepsilon_0\)-balls covers \(M\), we may proceed indefinitely for each
\(j \in \N\), always choosing
\(x_j \in M \setminus \bigcup_{i < j} B_{\varepsilon_0}(x_i)\).

Since \(M\) is sequentially compact, there must exist a convergent subsequence
\((x_{j_n})_{n \in \N}\) of \((x_j)_{j \in \N}\) --- with \(x_{j_n} \to x\) for
some \(x \in M\). Since convergent sequences are Cauchy in metric spaces,
for large enough \(j \in \N\), we have \(d(x_{j+1}, x_j) < \varepsilon_0\) ---
which implies in \(x_{j+1} \in B_{\varepsilon_0}(x_j)\), a contradiction by the
construction of the sequence. Therefore there must indeed exist a finite
collection of \(\varepsilon_0\)-balls that cover \(M\).

For the separability of \(M\), for each \(n \in \N\), define \(F_n\) to be a
finite set of points of \(M\) for which the \(1/n\)-balls centred at each point
of \(F_n\) covers \(M\). Since each \(F_n\) is finite, the union
\(F \coloneq \bigcup_{n \in \N} F_n\) is countable. Now, if \(U \subseteq X\) is
a non-empty open subset of \(X\) and \(p \in U\) is any point and let
\(B_{\varepsilon}(p) \subseteq U\) be any neighbourhood of \(p\). Since \(\R\)
is archimedian, there must exist \(n \in \N\) for which \(1/n < \varepsilon\)
--- hence there exists \(q \in F_n\) for which
\(p \in B_{1/n}(q)\) and \(q \in B_{\varepsilon}(p)\). We conclude that \(F\) is
dense in \(M\) --- thus \(M\) is separable.
\end{proof}

\begin{proposition}[Metric \& second countable spaces]
\label{prop:metric-2nd-ctbl-seq-comp-implies-comp}
For second countable spaces and metric spaces, sequential compactness implies
compactness.
\end{proposition}

\begin{proof}
Let \(X\) be a second countable and sequentially compact space. Let
\(\mathcal{U}\) be any open cover of \(X\). Since every second countable space
is Lindel\"{o}f (see \cref{prop: second countable properties}), there exists a
countable subcover
\(\mathcal{U}' \coloneq \{U_j\}_{j \in \N} \subseteq \mathcal{U}\). For the sake
of contradiction, suppose that there is no finite subcover of \(\mathcal{U}'\)
--- that is, for every \(j \in \N\), there exists \(x_j \in X\) such that
\(x_j \notin \bigcup_{j=0}^n U_j\). If \((x_j)_{j \in \N}\) is a sequence formed
by points with such a property, since \(X\) is sequentially compact, there must
exist a convergent subsequence \((x_{j_n})_{n \in \N}\) with \(x_{j_n} \to x\)
for some point \(x \in X\). Let \(m \in \N\) be such that \(x \in U_m\) --- from
the definition of limit point, there must exist only finitely many \(x_{j_n}\)
not contained in \(U_m\). However, that cannot be the case since, for all
\(j_n \geq m\), we must have \(x_{j_n} \notin \bigcup_{n=0}^m U_{n}\) thus in
particular \(x_{j_n} \notin U_m\) --- this implies that \((x_{j_n})_{n \in \N}\)
does not converge to \(x\), which is a contradiction.

By \cref{lem:metric-sequencially-comp-is-2nd-ctbl} if \(M\) is sequentially
compact then it's second countable --- therefore, by what was shown above we
conclude that \(M\) is compact.
\end{proof}

\begin{theorem}
\label{thm:metric-2d-ctbl-hausdorff-equiv-compactness}
For metric spaces and second countable Hausdorff spaces, limit point
compactness, sequential compactness and compactness are \emph{equivalent}
properties.
\end{theorem}

\begin{proof}
This is the result of \cref{thm:bolzano-weierstrass},
\cref{lem:fst-countable-hausdorff-limit-pt-implies-seq-comp} and
\cref{prop:metric-2nd-ctbl-seq-comp-implies-comp}.
\end{proof}

\begin{corollary}[Metric space completeness]
\label{cor:compact-metric-space-is-complete}
Every compact metric space is complete.
\end{corollary}

\begin{proof}
Let \(M\) be compact and \((x_j)_{j \in \N}\) be any Cauchy sequence. Since
\(M\) is compact, then it's also sequentially compact, which implies in the
existence of a convergent subsequence \((x_j')_j \subseteq (x_j)_{j \in \N}\)
for which \(x_j' \to x\) for some \(x \in M\). We'll show that, in fact,
\(x_j \to x\). Let \(B_{\varepsilon}(x)\) be any open ball centred at
\(x\). Since the sequence is Cauchy, there exists \(N \in \N\) such that
\(d(x_m, x_n) < \varepsilon/2\) for every \(m, n > N\). On the other hand, there
exists \(M \in \N\) such that \(d(x_{\ell}', x) < \varepsilon/2\) for every
\(\ell > M\) --- therefore, for all \(n, \ell > \max(N, M)\) we have
\(d(x_n, x) < d(x_n, x_{\ell}') + d(x_{\ell}', x) < \varepsilon\), thus
\(x_n \in B_{\varepsilon}(x)\) for all \(n > \max(N, M)\). We conclude that all
but finitely many points of \((x_j)_{j \in \N}\) are contained in
\(B_{\varepsilon}(x)\) for an arbitrary \(\varepsilon > 0\) --- therefore
\(x_j \to x\).
\end{proof}

\section{Local Compactness}

\begin{definition}[Locally compact spaces]
\label{def:locally-compact}
A space \(X\) is said to be \emph{locally compact} if and only if, for every
point \(x \in X\), there exists a compact set \(K \subseteq X\) and a
neighbourhood \(U \subseteq X\) of \(x\) for which \(U \subseteq K\).
\end{definition}

\begin{definition}[Relatively compact]
\label{def:relatively-compact}
Given a space \(X\), a subset \(A \subseteq X\) is said to be \emph{relatively
  compact in \(X\)} if its closure \(\Cl A\) is compact.
\end{definition}

\begin{proposition}[Local \& relative compactness in Hausdorff spaces]
\label{prop:hausdorff-locally-and-relatively-compact}
Let \(X\) be a \emph{Hausdorff} space. The following are equivalent properties:
\begin{enumerate}[(a)]\setlength\itemsep{0em}
\item \(X\) is locally compact.
\item Every point of \(X\) has a relatively compact neighbourhood in \(X\).
\item \(X\) has a basis of relatively compact open sets.
\end{enumerate}
\end{proposition}

\begin{proof}
From the definition of local and relative compactness, it is clear that (c)
implies (b) and that (b) implies (a). We therefore only prove that (a) implies
(c). Let \(X\) be a locally compact Hausdorff space. Take \(x_0 \in X\) to be
any point and let \(K \subseteq X\) be a compact set containing a neighbourhood
\(U \subseteq X\) of \(x_0\). If we let \(\mathcal{B}\) be the collection of all
neighbourhoods of \(x_0\) which are \emph{contained} in \(U\), then
\(\mathcal{B}\) forms a neighbourhood basis at \(x_0\). Moreover, since \(X\) is
Hausdorff, \(K\) is closed and every \(V \in \mathcal{B}\) is also contained in
\(K\) --- therefore \(\Cl V \subseteq K\) and since a closed subset of a compact
set is compact then \(\Cl V\) is compact. We conclude that \(V\) is relatively
compact in \(X\) and thus \(\mathcal{B}\) is a basis composed of relatively
compact open sets.
\end{proof}

% \begin{proposition}[Product of quotient morphisms]
% \label{prop:product-of-quotient-maps-locally-cpct-hausdorff}
% Let \(\pi: X \epi Y\) and \(\pi': X' \epi Y'\) be \emph{quotient morphisms}, and
% both \(Y\) and \(X'\) be \emph{locally compact Hausdorff} spaces. The induced
% map
% \[
% \pi \times \pi': X \times X' \epi Y \times Y'
% \]
% is a quotient morphism.
% \end{proposition}

\begin{lemma}
\label{lem:loc-cpct-haus-rel-cpct-nbhd}
Let \(X\) be a locally compact Hausdorff space. For every point \(x \in X\) and
neighbourhood \(U \subseteq X\) of \(x\), there exists a relatively compact
neighbourhood \(V\) of \(x\) for which \(\Cl V \subseteq U\).
\end{lemma}

\begin{proof}
Since \(X\) is locally compact Hausdorff space, by
\cref{prop:hausdorff-locally-and-relatively-compact} there exists a relatively
compact neighbourhood \(W\) of \(x\). Since \(\Cl(W) \setminus U\) is closed in
\(\Cl W\), it follows that it's compact. Evoking
\cref{thm:hausdorff-compact-set-disjoint-neighbourhoods} one can find disjoint
open sets \(T\) and \(Q\) such that \(x \in T\) and
\(\Cl(W) \setminus U \subseteq Q\). Define a set \(V \coloneq T \cap W\), then
\(\Cl V \subseteq \Cl W\) implies in \(\Cl V\) compact --- thus \(V\) is a
relatively compact neighbourhood of \(x\), we just need to show that
\(\Cl V \subseteq U\).

Since \(T\) and \(Q\) are disjoint, then \(T \subseteq X \setminus Q\) and
therefore \(V \subseteq X \setminus Q\) --- since \(X \setminus Q\) is closed,
this implies in \(\Cl V \subseteq X \setminus Q\). Therefore, since
\(\Cl V \subseteq \Cl W\), we find that \(\Cl V \subseteq \Cl(W) \setminus
Q\). From construction, we have \(\Cl(W) \setminus U \subseteq Q\), which
implies in \(\Cl(W) \setminus Q \subseteq U\) --- hence in particular
\(\Cl V \subseteq U\).
\end{proof}

\begin{proposition}
\label{prop:subset-loc-cpct-hausdorff}
Any open or closed subset of a locally compact Hausdorff space is itself locally
compact Hausdorff space.
\end{proposition}

\begin{proof}
Let \(X\) be a locally compact Hausdorff space. If \(U \subseteq X\) is an open
set, then by \cref{prop:hausdorff-locally-and-relatively-compact}, since \(U\)
is also Hausdorff, any point \(x \in U\) has a relatively compact neighbourhood
of \(x\) contained in \(U\) --- thus \(U\) is locally compact Hausdorff space.

Let \(C \subseteq X\) be a closed set and let \(x \in C\) be any point. Since
\(C\) is also Hausdorff, we argue analogously that \(x\) has a relatively
compact neighbourhood \(K\) of \(x\) contained in \(C\). Since \(\Cl K\) is
compact, the closed subset \(\Cl(K \cap C) \subseteq \Cl K\) is
compact. Moreover, since \(C\) is closed, \(\Cl(K \cap C) \subseteq \Cl C = C\),
therefore\(K \cap C\) is a relatively compact neighbourhood of \(x\) in \(C\)
--- hence \(C\) is a locally compact Hausdorff space.
\end{proof}

\begin{lemma}[Product of locally compact spaces]
\label{lem:product-loc-cpct}
Any finite product of locally compact spaces is locally compact.
\end{lemma}

\begin{proof}
Let \(\{X_1, \dots, X_n\}\) be a finite collection of locally compact spaces and
consider the product space \(X \coloneq \prod_{j=1}^n X_j\). Let \(x \in X\) be
any point and consider, for every index \(1 \leq j \leq n\), the projection
\(\pi_j(x) \in X_j\). Since \(X_j\) is locally compact, there exists a compact
set \(K_j \subseteq X_j\) containing a neighbourhood \(U_j \subseteq X_j\) of
\(\pi_j(x)\). From this we consider the product sets
\(K \coloneq \prod_{j=1}^n K_j\) and \(U \coloneq \prod_{j=1}^n\). By
\cref{thm:tychonoff-theorem} we know that \(K\) is compact and by the product
topology the set \(U\) is open and also a neighbourhood of \(x\). Since
\(U_j \subseteq K_j\) for all \(1 \leq j \leq n\), it follows that
\(U \subseteq K\) --- which proves that \(X\) is locally compact.
\end{proof}

\begin{theorem}[Baire category theorem]
\label{thm:baire-category}
In every locally compact Hausdorff space or complete metric space, each
\emph{countable} collection of \emph{dense} open subsets has a \emph{dense
  intersection}.
\end{theorem}

\begin{proof}
Let \(X\) be either a locally compact Hausdorff space or a complete metric space
and \(\{D_n\}_{n \in \N}\) be a countable collection of dense open subsets of
\(X\). Evoking \cref{prop:dense-non-empty-intersects}, it suffices to prove that
every non-empty subset \(U \subseteq X\) contains a point of
\(D \coloneq \bigcap_{n \in \N} D_n\). We split the cases in two:
\begin{enumerate}\setlength\itemsep{0em}
\item If \(X\) is a locally compact Hausdorff space, we construct a nested
  sequence of compact sets inductively as follows. From hypothesis, \(D_0\) is
  dense, thus in particular \(U \cap D_0\) contains a point of \(X\), therefore
  evoking \cref{prop:hausdorff-locally-and-relatively-compact} we can find a
  non-empty relatively compact open set \(C_0 \subseteq X\) for which
  \(\Cl(C_0) \subseteq U \cap D_0\). We proceed analogously, finding a
  relatively compact set \(C_1 \subseteq X\) such that
  \(\Cl(C_1) \subseteq C_0 \cap D_1 \subseteq U \cap D_0 \cap D_1\). By
  induction we find a sequence of compact open sets \((\Cl(C_n))_{n \in \N}\)
  such that \(\Cl(C_{n+1}) \subseteq \Cl(C_{n})\) and
  \(\Cl(C_n) \subseteq U \cap \bigcap_{j=1}^n D_j\) for every \(n \in \N\). By
  \cref{lem:cpct-countable-intersection-nested-non-empty}, there exists a point
  \(x \in \bigcap_{n \in \N} \Cl(C_n) \subseteq U\) and from construction
  \(x \in \bigcap_{n \in \N} D_n\).

\item If \(X\) is a complete metric space, we use the inductive argument made
  above to construct a Cauchy sequence. For every \(n \in \N\), we have a
  non-empty open set \(C_{n-1} \cap D_n\) --- hence we may choose a point
  \(x_n\) and a neighbourhood
  \(B_{\varepsilon_n}(x_n) \subseteq C_{n-1} \cap D_n\) for some
  \(\varepsilon_n > 0\). By choosing for each \(n \in \N\) a radius
  \(r_n < \min(\varepsilon_n, 1/n)\), we construct a sequence of closed balls
  \((\Cl(B_{r_n}(x_n)))_{n \in \N}\) for which
  \(\Cl(B_{r_n}(x_n)) \subseteq U \cap \bigcap_{j=1}^n D_j\). In the limit
  \(n \to \infty\) the radius \(r_n\) of the closed balls goes to zero and
  \((x_n)_{n \in \N}\) form a Cauchy sequence. Since \(X\) is complete, such
  sequence converges to a point of \(U \cap \bigcap_{n \in \N} D_n\).
\end{enumerate}
\end{proof}

%%% Local Variables:
%%% TeX-master: "../../deep-dive"
%%% End:
\section{Combinatorial Complex}

\subsection{First Characterisations}

\begin{definition}[Combinatorial complex]
\label{def:combinatorial-complex}
A \emph{combinatorial complex} (CC) consists of a triple
\(\mathcal{X} = (V, C, \rank)\) where we have a set of \emph{vertices} \(V\), a set of
\emph{cells} \(C \subseteq 2^V \setminus \emptyset\) (endowed with the inclusion preorder) and a map
\(\rank: C \to \Z_{\geq 0}\) such that:
\begin{enumerate}[(a)]\setlength\itemsep{0em}
\item For all \(v \in V\) we have \(\{v\} \in C\).
\item The map \(\rank\) is order preserving with respect to inclusions: if \(x,
  y \subseteq C\) are such that \(x \subseteq y\) then \(\rank x \leq \rank y\).
\end{enumerate}
The dimension of \(\mathcal{X}\) is defined to be
\[
\dim \mathcal{X} \coloneq \max_{x \in C} \rank x.
\]
A cell \(x \in C\) is said to be of rank \(k \in \Z_{\geq 0}\) if \(\rank x = k\), which we
summarise by saying that \(x\) is a \(k\)-cell. The \(k\)-skeleton of \(\mathcal{X}\) is
defined as
\[
\Sk_k \mathcal{X} \coloneq \{x \in C \colon \rank x \leq k\}.
\]
The collection of all cells of rank \(k\) will be denoted by
\(\mathcal{X}^k \coloneq \rank^{-1} k\).
\end{definition}

\begin{notation}
\label{not:notation-combinatorial-complex}
Let \(\mathcal X = (V, C, \rank)\) be a combinatorial complex. For the sake of
clearness we shall also adopt the following notation: vertices of
\(\mathcal{X}\) are denoted by \(\Vertex \mathcal{X} \coloneq V\), while cells of
\(\mathcal{X}\) are \(\Cell \mathcal{X} \coloneq C\), and when needed we'll also write
\(\rank_{\mathcal{X}} = \rank\).
\end{notation}

\begin{example}
\label{exp:graph-based-cc}
The particular class of combinatorial complexes whose \(1\)-cells (edges) are
composed of exactly \(2\) vertices are called \emph{graph-based combinatorial
  complexes}.
\end{example}

\begin{definition}[Morphism of combinatorial complexes]
\label{def:cc-morphism}
Let \(\mathcal{X}\) and \(\mathcal{Y}\) be combinatorial complexes. We define a \emph{CC-morphism}
\(\phi: \mathcal{X} \to \mathcal{Y}\) to be a map \(\phi: \Cell \mathcal{X} \to \Cell \mathcal{Y}\) satisfying:
\begin{enumerate}[(a)]\setlength\itemsep{0em}
\item Given a pair \(x, x' \in \Cell \mathcal{X}\) such that \(x \subseteq x'\), then \(\phi x \subseteq \phi
  x'\)---that is, \(\phi\) is order preserving.
\item For any \(x \in \Cell \mathcal{X}\) we have
  \(\rank_{\mathcal{X}} x \geq \rank_{\mathcal{Y}}(\phi x)\)---that is, for any
  \(k \in \Z_{\geq 0}\) we have \(\phi(\Sk_k \mathcal{X}) \subseteq \Sk_k \mathcal{Y}\).
\end{enumerate}
\end{definition}

\begin{definition}
\label{def:cc-embedding}
A CC-morphism \(\phi: \mathcal{X} \to \mathcal{Y}\) is said to be a \emph{CC-embedding} if
\(\phi\) is an injective set function on the cells, and
\(\rank_{\mathcal{X}} x = \rank_{\mathcal{Y}}(\phi x)\) for every cell \(x \in \Cell \mathcal{X}\).
\end{definition}

\begin{definition}[Sub-combinatorial complex]
\label{def:sub-combinatorial-complex}
Let \(\mathcal{X}\) be a combinatorial complex. We say that \(\mathcal{Y}\) is a
\emph{sub-combinatorial complex} (or sub-CC) if there exists a CC-embedding
\(\mathcal{Y} \emb \mathcal{X}\). We shall denote by \(\Sub \mathcal{X}\) the collection of all
sub-combinatorial complexes of \(\mathcal{X}\).
\end{definition}

\begin{example}
\label{exp:sub-cc}
For any subset \(V \subseteq \Vertex \mathcal{X}\), we can induce a sub-CC given by
\[
\mathcal{X}_V \coloneq (V, C, \rank_{\mathcal{X}}|_C)
\]
where \(C \coloneq \{x \in \Cell \mathcal{X} \colon x \subseteq V\}\). In particular, for
\(k \in \Z_{\geq 0}\), the skeleton \(\Sk_k \mathcal{X}\) is a sub-CC of \(\mathcal{X}\). A cell \(x \in
\Cell \mathcal{X}\) also induces a sub-CC whose vertices are the nodes contained in \(x\)
and whose cells are \(\{y \in \Cell \mathcal{X} \colon y \subseteq x\}\)---we shall abuse the notation
and say that \(x\) is a sub-CC of \(\mathcal{X}\).
\end{example}

\subsection{Neighbourhood Functions}

\begin{definition}[Neighbourhood functions \& matrices]
\label{def:neighbourhood-function}
A neighbourhood function on a combinatorial complex \(\mathcal{X}\) is a map
\(N: \Sub \mathcal{X} \to 2^{\Vertex \mathcal{X}} \setminus \emptyset\), assigning to each sub-CC of \(\mathcal{X}\) a non-empty
subset of nodes of \(\mathcal{X}\).

Let \(C = \{c_1, \dots, c_n\}\) and \(C' = \{c'_1, \dots, c'_m\}\) be two finite
subsets of the cells of \(\mathcal{X}\) for which \(N c_j \subseteq C'\) for any \(1 \leq j \leq n\). We
define the \emph{neighbourhood matrix of \(N\) with respect to \(C\) and \(C'\)}
to be the \(m \times n\) binary matrix \(\Mat N = [a_{i j}]\) given by
\[
a_{i j} =
\begin{cases}
  1, &\text{if } c'_i \in N c_j \\
  0, &\text{otherwise}
\end{cases}
\]
\end{definition}

\subsubsection{Important Examples of Neighbourhood Functions}

Now we shall define some of the most important neighbourhood functions for
applications in topological neural networks.

\begin{definition}[Down \& up incidence neighbourhood function]
\label{def:down/up-incidence-neighbourhood}
Let \(\mathcal{X}\) be a combinatorial complex. Given a pair of distinct cells
\(x, y \in \Cell \mathcal{X}\), we say that \(x\) and \(y\) are \emph{incident} if either
\(x \subsetneq y\) or \(y \subsetneq x\). We define the following:
\begin{enumerate}[(a)]\setlength\itemsep{0em}
\item The \emph{down-incidence neighbourhood function}
  \(N_{\searrow}: \Sub \mathcal{X} \to 2^{\Vertex \mathcal{X}} \setminus \emptyset\) is the map
  \[
  N_{\nearrow} x \coloneq \{y \in \Cell \mathcal{X} \colon y \subsetneq x\}.
  \]

\item The \emph{up-incidence neighbourhood function}
  \(N_{\nearrow}: \Sub \mathcal{X} \to 2^{\Vertex \mathcal{X}} \setminus \emptyset\) is the map
  \[
  N_{\nearrow} x \coloneq \{y \in \Cell \mathcal{X} \colon x \subsetneq y\}.
  \]

\item Let \(k \in \Z_{\geq 0}\) be any non-negative integer. We define the
  \emph{\(k\)-down incidence neighbourhood function}
  \(N_{\searrow, k}: \Sub \mathcal{X} \to 2^{\Vertex \mathcal{X}} \setminus \emptyset\) to be the map
  \[
  N_{\searrow, k} x \coloneq
  \{y \in \Cell \mathcal{X} \colon y \subsetneq x \text{ such that } \rank y = \rank(x) - k\}.
  \]

\item Let \(k \in \Z_{\geq 0}\) be any non-negative integer. We define the
  \emph{\(k\)-up incidence neighbourhood function}
  \(N_{\searrow, k}: \Sub \mathcal{X} \to 2^{\Vertex \mathcal{X}} \setminus \emptyset\) to be the map
  \[
  N_{\nearrow, k} x \coloneq
  \{y \in \Cell \mathcal{X} \colon x \subsetneq y \text{ such that } \rank y = \rank(x) + k\}.
  \]
\end{enumerate}
\end{definition}

\begin{corollary}
\label{cor:relation-down/up-to-k-down/up}
If \(\mathcal{X}\) is a CC, then \(N_{\searrow} x = \bigcup_{k \geq 0} N_{\searrow, k} x\) and \(N_{\nearrow} x = \bigcup_{k \geq
  0} N_{\nearrow, k} x\) for every \(x \in \Cell \mathcal{X}\).
\end{corollary}

\begin{definition}
\label{def:cc-cell-face-and-coface}
Given a CC \(\mathcal{X}\) we define, for each cell \(x \in \Cell \mathcal{X}\) the following:
\begin{enumerate}[(a)]\setlength\itemsep{0em}
\item The \emph{faces} of \(x\) is defined to be \(N_{\searrow, 1} x\).

\item The \emph{cofaces} of \(x\) is defined to be \(N_{\nearrow, 1} x\).
\end{enumerate}
\end{definition}

\begin{definition}[Incidence matrix]
\label{def:cc-incidence-matrix}
Let \(\mathcal{X}\) be a CC with finitely many vertices and take non-negative integers
\(0 \leq r < k \leq \dim \mathcal{X}\). We define the \emph{\((r, k)\)-incidence matrix
  \(B_{r, k} = [b_{i j}]\)} between \(\mathcal{X}^r = \{x_1, \dots, x_m\}\) and
\(\mathcal{X}^k = \{x_1', \dots, x_n'\}\) to be the binary \(m \times n\) matrix such that
\[
b_{i j} =
\begin{cases}
  1, &\text{if } x_i \text{ is incident to } x_j' \\
  0, &\text{otherwise}
\end{cases}
\]
\end{definition}

\begin{definition}[(Co)adjacency neighbourhood function]
\label{def:co-adjacency-neighbourhood-function}
Let \(\mathcal{X}\) be a CC. We shall define the following neighbourhood functions on \(\mathcal{X}\):
\begin{enumerate}[(a)]\setlength\itemsep{0em}
\item The \emph{adjacency neighbourhood function}
  \(N_{\text{ad}}: \Cell \mathcal{X} \to 2^{\Vertex \mathcal X} \setminus \emptyset\) to be the map sending
  a cell \(x \in \Cell \mathcal{X}\) to the collection of all
  \(y \in \Cell \mathcal{X}\) having \(\rank y = \rank x\) for which there exists
  \(z \in \Cell \mathcal{X}\) such that \(\rank z > \rank x\) and \(x, y \subsetneq z\).

\item The \emph{coadjacency neighbourhood function}
  \(N_{\text{coad}}: \Cell \mathcal{X} \to 2^{\Vertex \mathcal X} \setminus \emptyset\) to be the map sending
  a cell \(x \in \Cell \mathcal{X}\) to the collection of all
  \(y \in \Cell \mathcal{X}\) having \(\rank y = \rank x\) for which there exists
  \(z \in \Cell \mathcal{X}\) such that \(\rank z < \rank x\), and having both \(z \subsetneq y\) and
  \(z \subsetneq x\).
\end{enumerate}
A cell \(z \in \Cell \mathcal{X}\) that satisfies either of the conditions imposed by the
sets \(N_{\text{ad}} x\) or \(N_{\text{coad}} x\) is said to be a \emph{bridge cell}.
\end{definition}



%%% Local Variables:
%%% mode: latex
%%% TeX-master: "../../deep-dive"
%%% End:

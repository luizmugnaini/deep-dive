\section{Symmetric Tensors and Symmetric Algebra}

\begin{definition}[Symmetric tensor]
  Let \(V\) be a \(k\)-vector space and consider the mixed tensor product
  \(T_0^q(V)\). For every permutation \(\sigma \in \mathcal S_q\) (where
  \(\mathcal S_q\) denotes the symmetry group of \(q\) elements), define the
  linear transformation
  \[
    f_\sigma: T_0^q(V) \to T_0^q(V),\ \otimes_{i=1}^q v_i \xmapsto{f_\sigma}
    \otimes_{i=1}^q v_{f_\sigma(i)}.
  \] 
  We say that a tensor \(T \in T_0^q(V)\) is symmetric if
  \[
     f_\sigma(T) = T.
  \] 
\end{definition}

\begin{definition}[Symmetric tensor space]
  We denote the subspace of \(T_0^q(V)\) consisting of symmetric tensors as the
  symmetric space \(\Sym^q(V)\), also called symmetric power.
\end{definition}

\begin{definition}[Symmetrization map]
  Let \(V\) be a \(k\)-vector space. We define the projection operator \(S:
  T_0^q(V) \to T_0^q(V)\), which maps factorizable tensors to symmetric tensors.
\end{definition}

\begin{proposition}[Symmetric space universal property]
  Let \(V\) be a \(k\)-vector space. Given any \(k\)-vector space \(L\) and a
  multilinear symmetric map \(\mu: V^q \to L\), there exists a unique linear map
  \(\ell: \Sym^q(V) \to L\) such that \(\ell \circ S \circ \otimes = \mu\). That
  is, the following diagram commmutes
  \[
    \begin{tikzcd}
      V^q \ar[rd, "\mu"] \ar[d, "\otimes"] \\
      T_0^q(V) \ar[d, two heads, "S"]
          &L \\
      \Sym^q(V) \ar[ru, dashed, swap, "\ell"]
    \end{tikzcd}
  \] 
\end{proposition}

\begin{proof}
  Since \(\Sym^q(V)\) is a subspace of \(T_0^q(V)\), then we can use the
  universal property \cref{thm: universal property of tensor products}.
\end{proof}

\begin{proposition}
  For \(\Char k \nmid q!\), we have that
  \[
    S(T) = \frac{1}{q!} \sum_{\sigma \in \mathcal S_q} f_\sigma(T)
  \] 
  and hence \(S^2 = S\), and \(S(T_0^q(V)) = \Sym^q(V)\).
\end{proposition}

\begin{proof}
  Let \(\{e_i\}_{i=1}^n\) be a basis for \(V\). First, notice that if \(T =
  T^{i_1, \dots, i_q} \in \Sym^q(V)\),
  then
  \[
    S(T) = \frac{1}{q!} \sum_{\sigma \in \mathcal S_q} f_\sigma(T^{i_1, \dots,
    i_q}) = \frac{1}{q!} (q! T^{i_1, \dots, i_q}) = T
  \] 
  where we assumed that \(\Char k \nmid q!\) in order to obtain a nonzero tensor
  after the summation. Hence \(S|_{\Sym^q(T)} = \Id_{\Sym^q(V)}\) therefore we
  find that \(\Sym^q(V) \subseteq \im S\). Moreover, let \(T = T^{i_1, \dots,
  i_q} \in T_0^q(V)\) be any factorizable tensor, then its image under \(S\) is
  clearly a symmetric tensor, that is \(S(T) \in \Sym^q(V)\), hence \(\im S
  \subseteq \Sym^q(V)\). This implies that \(\im S = \Sym^q(V)\) for \(\Char k
  \nmid q!\). Moreover, \(S(S(T)) = S(T)\) from the first argument, since
  \(S(T) \in \Sym^q(V)\).
\end{proof}

\begin{remark}
  From now on we are going to assume that \(\Char k \nmid q!\)
\end{remark}

\begin{notation}
  Notice that since \(\im S = \Sym^q(V)\) then we can write any symmetrized
  tensor in any permutation that we choose, hence, we sometimes use the dot
  notation \(\Sym^q(V) \ni v_1 \otimes \dots \otimes v_q = v_1 \cdot \ldots
  \cdot v_q\). Moreover, if \(\{e_i\}_{i=1}^n\) is a basis for \(V\), one can
  even adopt an exponential notation \(e_1 \cdot \ldots \cdot e_q = e_1^{a_1}
  \cdot \ldots \cdot e_q^{a_n}\), where \(a_i\) denotes the number of times the
  component \(e_i\) appears in the factorizable tensor, and \(a_1 + \dots + a_q
  = q\).
\end{notation}

\begin{proposition}[Basis for \(\Sym^q(V)\)]
  \label{prop: basis for symmetric power}
  Let \(\{e_i\}_{i=1}^n\) be a basis for the \(k\)-vector space \(V\). Then, the
  collection of tensors \(\{e_1^{a_1} \cdot \ldots \cdot e_n^{a_n} : a_1 + \dots
  + a_n = q\} \subseteq \Sym^q(V)\) form a basis for the symmetric space
  \(\Sym^q(V)\). This implies that \(\Sym^q(V)\) is the subspace of \(k[e_1,
  \dots, e_n]\) of homogeneous polynomials of total degree \(q\).
\end{proposition}

\begin{proof}
  We know from \cref{lem: tensor basis} that \(\{\otimes_{j=1}^q e_{i_j}: 1 \leq
  i_j \leq n\}\) forms a basis for \(T_0^q(V)\). Since \(S\) is multilinear,
  then \(S(\{\otimes_{j=1}^q e_{i_j}\}) = \Sym^q(V)\), that is \(\{e_{i_1}^{a_1}
  \cdot \ldots \cdot e_n^{a_n}\} = \{S(\otimes_{j=1}^q e_{i_j})\}\) generates
  the space \(\Sym^q(V)\). Hence, to show that the collection of symmetric
  tensors of the basis elements of \(V\) forms a basis for the symmetric space,
  we just need to show their linear independence.

  To do that, fix a sequence of indices \(I = (i_1, \dots, i_q)\) such that  \(1
  \leq i_1 \leq \dots \leq i_q \leq n\). For all \(v_j = \sum_{i=1}^n a_{ij} e_i
  \in V\) where \(1 \leq j \leq q\), let \(\mu_I: V^q \to k\) be the mapping
  \[
    \mu_I(v_1, \dots, v_n) = \sum_{\sigma \in \mathcal S_q} \prod_{j=1}^q
    e_{i_{\sigma(j)}}^*(v_j).
  \] 
  Then, from the universal property of tensor products we find a unique
  functional \(f_I: \Sym^q(V) \to k\) such that the diagram commutes:
  \[
    \begin{tikzcd}
      V^q \ar[r, "\mu_I"] \ar[d, swap, "S \circ \otimes"] &k \\
      \Sym^q(V) \ar[ru, swap, dashed, "f_I"]
    \end{tikzcd}
  \] 
  hence for all \((v_1, \dots, v_q) \in T_0^q(V)\) we have \(\mu(v_1, \dots,
  v_q) = f(v_1 \cdot \ldots \cdot v_n)\). Notice that if we have a monotonically
  increasing sequence of indices \(I' = (i_1', \dots, i_q') \neq I\), there must
  be an index \(i_{j_0}' \in I'\) such that \(i_{j_0}' \neq i_j\) for all \(i_j
  \in I\). This way we find that for all \(\sigma \in \mathcal S_q\) the product
  \(\prod_{j=1}^q e_{i_{\sigma(j)}}^*(e_{i_j'}) = 0\) because
  \(e_{i_{\sigma(j)}}^*(e_{i_{j_0}'}) = 0\). In particular, this implies in
  \[
    \mu_I(e_{i_1'}, \dots, e_{i_q'}) = f_I(e_{i_1'} \cdot \ldots \cdot e_{i_q'})
    = 0.
  \] 

  Lets compute the value of \(f_I(e_{i_1} \cdot \ldots \cdot e_{i_q})\). To do
  this, consider again the fixed monotone increasing sequence of indices that we
  started with, that is, \(I\). Let \(Q \leq q\) denote the number of distinct
  values of \(i_j \in I\) for \(1 \leq j \leq q\). Define now, for all \(1 \leq
  r \leq Q\) the index sets \(J_r := \{j : e_{i_j} \text{ have equal
  values}\}\). Define \(n_r := |J_r|\), this construction yields a total of
  \(\prod_{r=1}^Q (n_r!)\) permutations. Notice that all permutations of
  \(J_r\)'s are such that \(\sigma_{J_r}(I) = I\) (they change the position of
  equal elements, causing no alteration of the sequence), we'll denote such
  permutations by \(\sigma_J \in \mathcal S_q\). Moreover, these are
  all the possible permutations that leave \(I\) unaltered. Hence
  \[
    f_I(e_{i_1} \cdot \ldots \cdot e_{i_q}) = \mu_I(e_{i_1}, \dots, e_{i_q})
    = \sum_{\sigma \in \mathcal S_q} \prod_{j=1}^q e_{i_{\sigma(j)}}^*(e_j)
    = \sum_{\sigma_J \in \mathcal S_q} \prod_{j=1}^q e_{i_{\sigma_J(j)}}^*(e_j)
    = \prod_{r=1}^Q (n_r!) \in k.
  \]
  Therefore, if \(\Char k \nmid q!\), we find \(f_I(e_{i_1} \cdot \ldots \cdot
  e_{i_q}) \neq 0 \in k\).

  If there exists a linear relation
  \[
    \sum_{i_1', \dots, i_q'} c_{i_1', \dots, i_q'} (e_{i_1'} \cdot \ldots \cdot
    e_{i_q'}) = 0
  \] 
  where the indices are arranged in monotonic increasing order, then we conclude
  that
  \begin{align*}
    0 = f_I \left( \sum_{i_1', \dots, i_q'} c_{i_1', \dots, i_q'} (e_{i_1'} \cdot
    \ldots \cdot e_{i_q}) \right) 
    &= c_{i_1', \dots, i_q'} \sum_{\sigma \in \mathcal S_q} \prod_{j=1}^q
    e_{i_{\sigma(j)}}^*(e_{i_j'}) \\
    &= c_{i_1, \dots, i_q} \sum_{\sigma \in \mathcal S_q} \prod_{j=1}^q
    e_{i_{\sigma(j)}}^*(e_j) \\
    &=  c_{i_1, \dots, i_q} f_I(e_{i_1} \cdot \ldots \cdot e_{i_q}) 
  \end{align*}
  hence \(c_{i_1, \dots, i_q} = 0\) since, as proved above, \(f_I(e_{i_1} \cdot
  \ldots \cdot e_{i_q}) \neq 0\). Now, passing \(I\) through all the monotone
  increasing sequences of indices, we prove that all of the coefficients
  \(c_{i_1', \dots, i_c'}\) vanish, as wanted. This proves that \(\{e_1^{a_1}
  \cdot \ldots \cdot e_n^{a_n} : a_1 + \dots + a_n = q\}\) is indeed a linearly
  independent collection.
\end{proof}

\begin{corollary}
  If \(V\) is a finite \(n\)-dimensional \(k\)-vector space, then
  \[
    \dim_k(\Sym^q(V)) = \binom{n + q - 1}{q}
  \] 
\end{corollary}

\begin{proof}
  Notice that from \cref{prop: basis for symmetric power} we have that
  \(\Sym^q(V)\) is the subspace of \(k[e_1, \dots, e_n]\) consisting of
  homogeneous polynomials of degree \(q\). In order to find the dimension of
  such space, it is sufficient to find the cardinality \(|\{a_1 + \dots + a_n =
  q : a_i \geq 0\}|\). A combinatorial way to do so is to define \(q\)
  symbols \(|\) (which represent the unity \(1\)), which can be divided into
  \(n\) different groups. In order to do so, we introduce \(n - 1\) symbols (we
  are going to use \(+\)), which are arranged in order to separate the bars into
  the \(n\) requested groups (example: for \(n = 3\) and \(q = 5\), one possible
  arrangement is \(| + || + ||\)). We have a total of \(n + q - 1\) symbols, and
  we want \(q\) of those (the bars) in order to create the groupings, that
  is, we have a total of \(\binom{n + q - 1}{q}\) ways of doing so. This proves
  the corollary.
\end{proof}

\begin{definition}
  We define the symmetric algebra over a finite \(n\)-dimensional \(k\)-linear
  space \(V\) to be the object \(\Sym V = \bigoplus_{q=1}^\infty \Sym^q V\).
  This is the space of all polynomials in \(e_1, \dots, e_n\) variables (where
  \(e_1, \dots, e_n\) form a basis for \(V\)).

  We define multiplication in \(\Sym V\) as the map
  \begin{equation}\label{def: multiplication symmetric algebra}
    \Sym^q V \times \Sym^d V \to \Sym^{q + d} V,\ \text{ mapping }\
    (T_1, T_2) \mapsto S(T_1 \otimes T_2) =: T_1 T_2
  \end{equation}
\end{definition}

\begin{proposition}
  The multiplicative structure defined by \cref{def: multiplication symmetric
  algebra} makes \(\Sym V\) a commutative associative \(k\)-algebra. Under the
  isomorphism of tensors and homogeneous polynomials as described in
  \cref{prop: basis for symmetric power}, this multiplicative structure
  preserves that of the multiplication of polynomials, making such isomorphism
  an isomorphism of algebras.
  \[
    \{\text{polynomials in } e_1, \dots, e_n\} \iso \Sym V
  \] 
  where \(e_1, \dots, e_n\) forms a basis for \(V\).
\end{proposition}

\begin{proof}
  Let \(T_1 \in T_0^q(V)\) and \(T_2 \in T_0^d(V)\), then we find that
  \[
    S(T_1) \otimes T_2 = \left( \frac{1}{q!} \sum_{\sigma \in \mathcal S_q}
    f_\sigma(T_1) \right) \otimes T_2
  \] 
  This way we find that
  \begin{align*}
    S(S(T_1) \otimes T_2) = S \left( \frac{1}{q!} \sum_{\sigma \in \mathcal S_q}
    f_\sigma(T_1) \otimes T_2 \right)
    &= \frac{1}{q!} \sum_{\sigma \in \mathcal S_q} S(f_\sigma(T_1) \otimes T_2)
    \\
    &= \frac{1}{q!} \sum_{\sigma \in \mathcal S_q} S(T_1 \otimes T_2) \\
    &= S(T_1 \otimes T_2)
  \end{align*}
  where we used the fact that \(S(f_\sigma(T_1) \otimes T_2) = S(T_1 \otimes
  T_2)\). Hence, we can conclude in general that
  \[
    S(S(T_1) \otimes T_2) = S(T_1 \otimes S(T_2)) = S(T_1 \otimes T_2)
  \] 
  If we extend such argument for tensors \(T_1, T_2, T_3\), we find that
  \[
    (T_1 T_2) T_3 = S(S(T_1 \otimes T_2) \otimes T_3)
    = T_1 (T_2 T_3) = S(T_1 \otimes (T_1 \otimes T_3))
    = T_1 T_2 T_3 = S(T_1 \otimes T_2 \otimes T_3)
  \] 
  which proves associativity for the multiplication. From tensor commutativity
  we find that 
  \[
    T_1 T_2 = S(T_1 \otimes T_2) = S(T_2 \otimes T_1) = T_2 T_1
  \]

  Finally, notice that from this multiplicative structure we have that the
  multiplication of polynomials is given by \((e_1^{a_1} \cdots
  e_n^{a_n}) (e_n^{b_1} \cdots e_n^{b_n}) = e_1^{a_1 + b_1} \cdots e_n^{a_n +
  b_n}\), which shows the algebra isomorphism.
\end{proof}

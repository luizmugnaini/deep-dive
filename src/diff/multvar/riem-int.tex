\section{Riemann Integration}

\subsection{Primary Definitions}

The main setting we are going to be working in this section, which will
encompass the study of multiple Riemann integrals, is the standard euclidean
space \(\R^n\) and the \(n\)-dimensional closed intervals \(I = [a, b] = \{x \in
\R^n : a_j \leq x_j \leq b_j \text{ for } 1 \leq j \leq n\}\). If it seems fit,
we can denote that a point \(x \in \R^n\) lies in the interval generated by
given points \(a, b \in \R^n\) by simply saying that \(a \leq x \leq
b\). Another terminology we are going to adopt is that, the interval \(I\) is
non-degenerate if \(a_j < b_j\) for all \(1 \leq j \leq n\).

\begin{notation}
In \emph{this} section we denote by \(\mathcal I^n\) the collection of all
\emph{closed} intervals of \(\R^n\).
\end{notation}

\begin{definition}[Interval measure]
\label{def:interval-measure}
We define the map \(\Vol: \mathcal I^n \to \R\) as the \emph{measure} (or
volume) of the \(n\)-dimensional closed intervals of \(\R^n\), it's defined as
the product of the interval sides, that is
\[
  \Vol I := \prod_{j=1}^n b_j - a_j,
  \ \text{ for }\ I = [a, b] \in \mathcal I^n.
\]
\end{definition}

\begin{corollary}[Measure of intervals in \(\R^n\)]
\label{cor:interval-measure-properties}
Let \(I \coloneq [a, b] \subseteq \R^n\) be an \(n\)-dimensional closed
interval, then the following properties are satiesfied concerning the measure
\(\Vol\):
\begin{enumerate}[(a)]\setlength\itemsep{0em}
\item (Homogeneity) Let \(\gamma \geq 0\) be a scalar and define the
  multiplication of the interval by \(\gamma\) as \(\gamma I \coloneq [\gamma
  a, \gamma b]\). Then we have that
  \[
    \Vol(\lambda I) = \lambda^n \Vol I.
  \]
\item (Additivity) Given a finite collection of closed intervals \(\{I_{j}
  \subseteq \R^n\}_{j = 1}^p\), we have that
  \[
    \Vol \bigcup_{j=1}^p I_j = \sum_{j=1}^p \Vol I_j.
  \]
\item (Cover inequality) Given a finite closed cover \(\{I_{j}\}^{p}\), by
  \(n\)-dimensional closed intervals, of \(I\) --- that is \(I \subseteq
  \bigcup_{j=1}^p I_j\) --- then
  \[
    \Vol I \leq \sum_{j=1}^p \Vol I_j.
  \]
\end{enumerate}
\end{corollary}

\begin{definition}[Partition]
\label{def:interval-partition}
Let \(I \subseteq \R^n\) be a closed interval. A \emph{partition} on \(I\) is a
\emph{finite} collection of closed intervals \(\{I_{j}\}_{j=1}^p\) such that \(I
= \bigcup_{j=1}^p I_j\). The intervals pertaining to the partition are said to
be \emph{finer} than \(I\).
\end{definition}

\begin{definition}[Partition mesh]
\label{def:partition-mesh}
Given a partition \(P \in 2^{\mathcal I^n}\), we define the \emph{mesh} of \(P\)
as the maximum diameter (recall \cref{def:Rn-diameter}) of the elements of the
partition. That is, \(\Mesh: 2^{\mathcal I^n} \to \R\) is a map defined by
\[
  \Mesh(P) \coloneq \max_{I \in P} d(I).
\]
\end{definition}

\begin{definition}[Distinguished points]
\label{def:distinguished-points}
Given a partition \(P = \{I_{j}\}_{j=1}^p \in 2^{\mathcal I^n}\), we define a
collection of \emph{distinguished points} of the partition as a collection of
points \(\xi \coloneq \{\xi_j \in I_{j}\}_{j=1}^p\). The partition \(P\)
together with the distinguished points \(\xi\) will be denoted as the pair \((P,
\xi)\) --- the collection of pairs \((P, \xi)\) will be denoted by \(\mathcal
P\).
\end{definition}

An important filter base \(\mathcal B \subseteq 2^{\mathcal P}\) is
defined as the collection of sets \(B_d\), where \(d > 0\) is a scalar, such
that \(B_d \coloneq \{(P, \xi) \in \mathcal P : \Mesh(P) < d\}\). We'll comonly
denote \(\mathcal B\) by \(\Mesh(P) \to 0\).

\subsection{Riemann Sums and Integrals}

\begin{definition}[Riemann sum]
\label{def:riemann-sum}
Let \(f: I \to \R\) be a map where \(I \in \mathcal I^n\). Consider the
partition together with distinguished points \((P, \xi) \in \mathcal P\), then,
we define the \emph{Riemann sum} \(\sigma: \R^I \times \mathcal P \to \R\) by
\[
  \sigma(f, P, \xi) \coloneq \sum_{j=1}^p f(\xi_j) \Vol(I_j),
\]
where \(P \coloneq \{I_{j}\}_{j=1}^p\). We say that \(\sigma(f, P, \xi)\) is the
Riemann sum of the map \(f\) with respect to the partition \(P\) and
distinguished points \(\xi\).
\end{definition}

\begin{definition}[Riemann integrable maps]
\label{def:riemann-integrable}
A map \(f: I \to \R\) is said to be \emph{Riemann integrable} if the limit
\[
  \lim_{\Mesh(P) \to 0} \sigma(f, P, \xi)
\]
exists in \(\R\). We'll denote the \(\R\)-vector space of Riemann integrable
maps with domain \(I \subseteq \R^n\) by \(\mathcal R(I)\).
\end{definition}

\begin{proposition}[Boundness of Riemann integrable maps]
\label{prop:riemann-integral-map-bounded}
Let \(f: I \to \R\) be a Riemann integrable map. Then, \(f\) is bounded on \(I\).
\end{proposition}

\begin{proof}
We prove the contrapositive proposition. Suppose that \(f\) is unbounded on
\(I\) and let \(P\) be any partition of the interval \(I\). In particular, since
\(P\) covers \(I\), then there exists an interval \(I_{k} \in P\) for which
\(f\) is unbounded. Let \(\xi\) be any collection of distinguished points of
\(P\) and define \(\xi'\) as the collection of distinguished points \(\xi_j'
\coloneq \xi_j\) for \(j \neq m\), and \(\xi_k' \in I_k\) to be such that
\(\xi_k' \neq \xi_k\). Then, from construction, it follows that \(\sigma(f, P,
\xi) - \sigma(f, P, \xi') = (f(\xi_k) - f(\xi_k')) \Vol(I_k)\). Since from
hypothesis \(f\) is unbounded in \(\xi_k\), for every \(M > 0\), there exists
\(\xi_k' \in I_k\) such that \(\norm{f(\xi_k) - f(\xi_k')} > M\) --- that is,
\(\norm{f(\xi_k) - f(\xi_k')}\) is obviously unbounded, which implies in the
divergence of the Riemann sums, hence \(f\) is non-Riemann integrable.
\end{proof}

\begin{definition}[Riemann integral]
\label{def:riemann-integral}
The Riemann integral of real valued maps is an \(\R\)-linear map \(\int:
\mathcal R \to \R\) defined by mapping any \(f \in \mathcal R(I)\) to
\[
  \int_I f(x)\, \diff x \coloneq \lim_{\Mesh(P) \to 0} \sigma(f, P, \xi).
\]
\end{definition}

\subsection{Sets of Lebesgue Measure Zero}

\begin{definition}[Set of Lebesgue measure zero]
\label{def:measure-zero-set}
A set \(E \subseteq \R^n\) is said be of Lebesgue \emph{measure zero} if for
every \(\varepsilon > 0\) there exists a \emph{countable open cover} \(\mathcal
U\) of \(E\) by \(n\)-dimensional \emph{open} intervals whose total volume
\(\sum_{I \in \mathcal U} \Vol \overline{I}\) does \emph{not} exceed
\(\varepsilon\).
\end{definition}

\begin{lemma}
\label{lem:measure-zero-properties}
The following are preoperties of sets of measure zero:
\begin{enumerate}[(a)]\setlength\itemsep{0em}
\item A subset of a set of measure zero is of measure zero.
\item The countable union of sets of measure zero is of measure zero.
\item A countable set is of measure zero.
\item A non-degenerate interval is \emph{not} a set of measure zero.
\end{enumerate}
\end{lemma}

\begin{proof}
\begin{enumerate}[(a)]\setlength\itemsep{0em}
\item Let \(E \subseteq \R^n\) be a set of measure zero and \(A \subseteq E\) be
  a subset. If \(\mathcal U\) is a closed cover by intervals satisfying the
  measure zero condition, then in particular \(\mathcal U\) covers \(A\)
  therefore \(A\) is of zero measure.

\item Let \(\{E_{j} \subseteq \R^{n}\}_{j \in J}\) be a countable collection of
  sets of measure zero, and let \(\{\mathcal U_{j}\}_{j \in J}\) be a collection
  where \(\mathcal U_j\) is the corresponding closed cover by intervals for
  \(E_j\). Notice that the countable union \(E \coloneq \bigcup_{j \in J} E_j\)
  can be covered by \(\mathcal U \coloneq \bigcup_{j \in J} \mathcal U_j\),
  therefore, since the union of countable collections is countable, it follows
  that \(\mathcal U\) is a countable cover for \(E\) which satisfies the wanted
  property.

\item We initially consider a single point in space. Notice that, for any given
  \(\varepsilon\), there exists a closed interval (for instance, one could
  choose an interval of equal sides containing the point, whose sides have
  length less than \(\varepsilon^{1/n}\)), whose volume is less than
  \(\varepsilon\), contaning the given point --- that is, this one interval is
  sufficient to cover the point. We conclude that a singleton is of measure
  zero. Using the last item, we find that a countable set is of measure zero.

\item Let \(I = [a, b] \subseteq \R^n\) be a non-degenerate interval. Since
  \(\R^n\) is Lindel\"{o}f, every cover of \(I\) has a finite subcover so, we can
  proceed by induction on the cardinality \(m \in \N\) of the open cover. For
  \(m = 1\), let \((\alpha, \beta) \subseteq \R^n\) be an open interval covering
  \(I\). Notice that every \(x \in I\) is such that \(a_j \leq x_j \leq b_j\),
  for all \(1 \leq j \leq n\), then, since \(x\) must lie at \((\alpha,
  \beta)\), we necessarily have \(\alpha_j < a \leq x_j \leq b < \beta_j\) so
  that \(b_j - a_j < \beta_j - \alpha_j\) and hence \(\Vol I < \Vol [\alpha,
  \beta]\). For the hypothesis of induction, suppose the proposition holds for a
  cover of cardinality \(n - 1 \in \N_{> 1}\). Let \(\{(\alpha^{i},
  \beta^i) \subseteq \R^n\}_{i = 1}^m\) be a cover of \(I\) by open
  intervals. Let \(1 \leq k \leq n\) be such that \(a \in (\alpha^k, \beta^k)\),
  that is, \(\alpha_k < a_j < \beta_{j}^{k}\).  If for some index \(1 \leq j_0
  \leq n\) we have \(b_{j_0} > \beta_{j_0}^k\), we define the point \(\beta'\) to
  be such that, if \(\beta_j^k < b_j\) then \(\beta_j' = \beta_j^k\), otherwise,
  if \(\beta_j^k \geq b_j\), we let \(a_j < \beta_j' < b_j\) --- that is, we
  constructed a point so that the closed interval \([\beta', b]\) is
  non-degenerate. From the hypothesis of induction, every cover of \([\beta',
  b]\) with cardinality \(m - 1\) has total volume strictly greater than \(\Vol
  [\beta', b]\). In particular, since \([\beta', b] \subset [a, b]\) then
  \(\{(\alpha^i, \beta^i)\}_{i=1}^m\) is a cover of \([\beta', b]\), notice that
  \([\beta', b] \cap (\alpha^k, \beta^k) = \emptyset\), thus the cover
  \(\{(\alpha^{i}, \beta^i)\}_{i = 1, i \neq k}^m\) is a cover of \([\beta', b]\)
  with cardinality \(n - 1\), hence \(\Vol [\beta', b] < \sum_{i=1, i \neq k}^m
  \Vol [\alpha^i, \beta^i]\). Then we find that
  \begin{align*}
    b_j - a_j
    &< (b_j - \beta'_j) + (\beta_j' - a_j) \\
    &\leq (b_j - \beta_j') + (\beta_j^k - a_j) \\
    &< (b_j - \beta_j') + (\beta_j^k - \alpha_j^k).
  \end{align*}
  Therefore we conclude that
  \begin{align*}
    \Vol [a, b]
    &< \Vol [\beta', b] + \Vol [\alpha^k, \beta^k] \\
    &< \sum_{i=1, i \neq k}^{m} \Vol [\alpha^i, \beta^{i}] + \Vol [\alpha^k,
    \beta^k] \\
    &= \sum_{i=1}^m \Vol [\alpha^i, \beta^i],
  \end{align*}
  which proves that the proposition is true for all \(m \in \N\).
\end{enumerate}
\end{proof}

\begin{notation}
Given a set \(X\) and a property \(P\), we say that \(P\) holds \emph{almost
everywhere} on \(X\) if the subset \(A \subseteq X\), such that \(P\) is not
true, is a set of measure zero.
\end{notation}

\begin{theorem}[Lebesgue's criterion]
\label{thm:lebesgue-criterion-integrable}
A map \(f: I \to \R\) is Riemann integrable if and only if \(f\) is
\emph{bounded} on \(I\) and \(f\) is \emph{continuous almost everywhere} on
\(I\).
\end{theorem}

\begin{proof}
(Necessary condition) Let \(f\) be Riemann integrable, then, from
\cref{prop:riemann-integral-map-bounded}, \(f\) is bounded on \(I\). For the
sake of contradiction, let \(E \subseteq I\) be the set composed of the points
of discontinuity of \(f\), we'll suppose that \(E\) doesn't have measure
zero. Notice that if \(x \in E\), then there exists \(n \in \N\) for which
\(\omega(f, x) \geq 1/n\) --- that is, \(f\) does not converge to a value in
\(x\). We can then define \(E_n \coloneq \{x \in I : \omega(f, x) \geq 1/n\}\)
for every \(n \in \N\) so that \(E_n \subseteq E\) and thus \(E = \bigcup_{n \in
\N} E_n\). Since \(E\) isn't of measure zero from assumption, it follows that
there necessarily exists at least one \(n_0 \in \N\) such that \(E_{n_0}\) isn't
of measure zero.

Let \(P\) be a partition of \(I\), we'll consider two subsets of this partition:
\(A \coloneq \{R \in P : R \cap E_{n_{0}} \neq \emptyset \text{ and } \omega(f,
R) \geq 1/(2n_0)\}\), and \(B \coloneq P \setminus A\). Since \(P\) partitions
the interval, for any \(x \in E_{n_0}\), there exists \(R \in P\) such that \(x
\in \Int R\) or \(x \in \partial R\) --- we now analyse both cases:
\begin{itemize}\setlength\itemsep{0em}
\item In the case where \(x\) is an interior point, since \(\omega(f, x) \geq
  1/n_0\), it follows that \(R \in A\).
\item Otherwise, if \(x\) is a boundary point, then there actually exists at
  least another \(R' \in P\) such that \(x \in \partial R \cap \partial
  R'\). Suppose, for the sake of contradiction, that every interval \(R' \in P\)
  containing \(x\) as a boundary point is such that \(f\) has an oscillation
  \(\omega(f, R') < 1/(2n_0)\), then, if we take any ball \(B_x(r) \cap I\),
  neighbourhood of \(x\), we find that, \(\omega(f, B_x(r)) < 1/(2n_0) +
  1/(2n_0) = 1/n_0\), that is, the limit \(\omega(f, x) < 1/n_0\), which is a
  contradiction to the assumption that \(x \in E_n\) --- thus there must exist
  \(R' \in P\) with \(x \in \partial R'\) such that \(\omega(f, R') \geq
  1/(2n_0)\) so that \(R' \in A\).
\end{itemize}
This implies that \(A\) covers the interval \(E_{n_0}\) by closed intervals and,
by assumption, \(\sum_{R \in A} \Vol R > \Vol E_{n_0}\).

We are now going to consider any two distinct choices of distinguished points
\(\xi\) and \(\xi'\) of \(P\) such that, if \(\xi_j\) and\(\xi_j'\) are elements
of a common interval of \(B\) then \(\xi_j = \xi'_j\), and if \(\xi_j\) and
\(\xi_j'\) are elements common to an interval of \(A\), we choose \(\xi_j\) and
\(\xi_j\) to be any points such that \(f(\xi_j) - f(\xi_j') > 1/(3n_0)\) ---
which is always possible from the construction of \(A\). Notice that we have
\[
  \norm{\sigma(f, P, \xi) - \sigma(f, P, \xi')}
  = \norm{\sum_{R_{j} \in A} (f(\xi_j) - f(\xi_j')) \Vol R_j}
  > \frac 1 {3 n_0} \sum_{R_j \in A} \Vol R_j
  > \frac 1 {3 n_0} \Vol E_{n_0} > 0
\]
\end{proof}
\todo[inline]{Continue proof}

%%% Local Variables:
%%% mode: latex
%%% TeX-master: "../../../deep-dive"
%%% End:

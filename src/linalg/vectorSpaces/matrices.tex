\section{Matrices}

\begin{definition}[Matrix]
   We define a \(m \times n\) matrix with entries in \(k\) as morphism \(k^n \to
   k^m\) in the category \(\Vect_k\), that is, regarding \(k^n, k^m\) as
   \(k\)-vector spaces.
\end{definition}

\subsection{Classifying Matrices}

Let \(e_j\) be defined the be a tuple whose \(j\)-th element is \(1 \in k\) and
all of the other elements of the tuple are \(0 \in k\), moreover, if \(e_j \in
k^n\) it is an \(n\)-tuple. Let the morphism \(\varphi \in \Hom_{\Vect_k}
(k^n, k^m)\) and define for all \(1 \leq j \leq n\) the image
\[
   \varphi(e_j) \coloneq (t_{1 j}, t_{2 j}, \dots, t_{mj}) = \sum_{i = 1}^m t_{ij} e_i
   \in k^m.
\]

We now prove that in fact the \(mn\) elements \(t_{ij} \in k\) determine
completely the behaviour of \(\varphi\), since for any element \((a_j)_{j=1}^n
\in k^n\) we have
\[
   \varphi((a_j)_{j=1}^n) = \varphi \Bigg( \sum_{j=1}^n a_j e_j \Bigg) =
   \sum_{j=1}^n a_j \varphi(e_j) = \sum_{j=1}^n \left( \sum_{i=1}^m a_j t_{i j}
   e_i\right) \in k^m.
\]
Moreover, it can trivially be seen that the mapping \((a_j)_{j=1}^n \mapsto
\sum_{j=1}^n \sum_{i=1}^m a_j t_{ij} e_i\)is indeed a morphism of vector
spaces since, for another \((b_j)_{j=1}^n \in k^n\), we have
\[
   \varphi((a_j + b_j)_{j=1}^n) = \varphi((a_j)_{j=1}^n) +
   \varphi((b_j)_{j=1}^n)
\]
also, given \(c \in k\) we have
\[
   \varphi((c a_j)_{j=1}^n) = c\varphi((a_j)_{j=1}^n).
\]

Since the matrix \(k^n \to k^m\) can be identified and completely determined
with elements \((t_{ij})_{i, j}\) where \(1 \leq i \leq m\) and \(1 \leq j \leq
n\), then we visually can represent it by
\[
   \varphi =
   \begin{pmatrix}
      t_{1 1} & \dots & t_{1 n} \\
      \vdots & \ddots & \vdots \\
      t_{m 1} & \dots & t_{m n}
   \end{pmatrix}
\]
Also, we can regard the fact that \(\varphi((a_i)_{i=1}^n) = (b_i)_{i=1}^m\)
visually as a system of equations
 \[
    \begin{pmatrix}
      t_{1 1} & \dots & t_{1 n} \\
      \vdots & \ddots & \vdots \\
      t_{m 1} & \dots & t_{m n}
    \end{pmatrix}
    \begin{pmatrix} a_1 \\ \vdots \\ a_n \end{pmatrix}
    =
    \begin{pmatrix} b_1 \\ \vdots \\ b_m \end{pmatrix}
\]

\begin{definition}[Square diagonal Matrix]
  \label{def: square diagonal matrix}
  We define a \(n \times n\) matrix \(A\) to be diagonal if for all \(i \neq j\)
  indices we have \(a_{ij} = 0\).
\end{definition}

\begin{definition}[Matrix for a linear map]
  \label{def: matrix for a linear map}
  Suppose \(V \iso k^n\) and \(W \iso k^m\) are \(k\)-vector spaces and \(L : V
  \to W\) is a linear morphism. Let also \(\{v_j\}_{j=1}^n\) and
  \(\{w_i\}_{i=1}^m\) be basis for the respective given finite dimensional
  vector spaces. By the isomorphism, we can represent \(L\) as a matrix \(k^n
  \to k^m\) whose components \(t_{i, j}\) with \(1 \leq i \leq m\) and \(1 \leq
  i \leq n\) are defined with respect to the given basis as
  \[
    L v_j = \sum_{i=1}^m t_{i, j} w_i
  \]
\end{definition}

\subsection{Matrix multiplication}

\begin{definition}[Multiplication of matrices]\label{def: matrix multiplication}
  Let \(A = [a_{i,j}] : k^n \to k^m\) and \(B = [b_{i, j}] : k^\ell \to k^n\) be
  matrices. Then, the product of the matrices \(A\) and \(B\) is defined as
  \(AB : k^n \to k^\ell\) with coefficients
  \[
    c_{i, j} = \sum_{k=1}^n a_{i, k} b_{k, j}
  \]
\end{definition}

\begin{proposition}[Composition of morphisms matrix]
  \label{prop: matrix of the composition}
  Let \(k\)-linear morphisms \(V \xrightarrow g W \xrightarrow f L\) of finite
  \(k\)-dimensional vector spaces, and choose basis \(\{v_j\}, \{w_k\}, \{l_i\}\)
  to be basis of \(V, W, L\) respectively, and let \(A_g\) and \(A_f\) be the
  matrix representation of the morphisms \(g\) and \(f\) with respect to the
  given basis. Then the matrix representation of the composition \(f  g: V
  \to L\) is given by \(A_{f  g} = A_f A_g\).
\end{proposition}

\begin{proof}
  Let \(A_f \coloneq [a_{i, k}]\), and \(A_g \coloneq [b_{k, j}]\), and \(A_{f  g} =
  [c_{i, j}]\). Then, since from definition we have \(g(v_j) = \sum_k b_{k, j}
  w_k\), hence
  \[
    f  g(v_j) =
    \sum_k b_{k, j} f(w_k) =
    \sum_k \bigg( b_{k, j} \bigg( \sum_i a_{i, k} l_i \bigg) \bigg)
    = \sum_i \bigg( \sum_k a_{i, k} b_{k, j} \bigg) l_i
    = \sum_i c_{i, j} l_i
  \]
  thus \(c_{i, j} = \sum_k a_k{i, k} b_{k, j}\) and thus \(A_{f  g} = A_f
  A_g\) as wanted.
\end{proof}

\begin{definition}[Conjugation]\label{def: matrix conjugation}
  Let \(M_n(k)\) be the collection of matrices \(k^n \to k^n\). We define a
  matrix conjugation as the linear morphism \(M_n(k) \to M_n(k)\) with the
  mapping \(A \mapsto B^{-1} A B\), where \(B\) is an invertible matrix.
\end{definition}

\begin{proposition}
  Every conjunjugation is an automorphism of the matrix algebra \(M_n(k)\).
\end{proposition}

\section{Sets and Universes}

In order to deal with size issues in the theory of categories, we shall work
inside what are called \emph{universes}.

\begin{definition}[Universe]
\label{def:universe}
A \emph{universe} \(\mathcal U\) is defined as a \emph{set} satisfying the
following properties
\begin{enumerate}[(U1)]\setlength\itemsep{0em}
\item \(\emptyset \in \mathcal U\).
\item If \(u \in \mathcal U\), then \(u \subseteq \mathcal U\).
\item If \(u \in \mathcal U\), then \(\{u\} \in \mathcal U\).
\item If \(u \in \mathcal U\), then \(2^u \in \mathcal U\).
\item If we have an indexing set \(I \in \mathcal U\), for which we associate
  a collection \(\{u_i \in \mathcal U\}_{i \in I}\), then \(\bigcup_{i \in I} u_i
  \in \mathcal U\).
\item \(\N \in \mathcal U\).
\end{enumerate}
As a consequence of such properties, a universe also satisfies
\begin{enumerate}[(U1)]\setlength\itemsep{0em}
\setcounter{enumi}{6}
\item If \(u \in \mathcal U\), then \(\bigcup_{x \in u} x \in \mathcal U\).
\item Given \(u, v \in \mathcal U\), we have \(u \times v \in \mathcal U\).
\item If \(v \in \mathcal U\), and we have \(u \subseteq v\), then \(u \in
  \mathcal U\).
\item If \(I \in \mathcal U\) is an idexing set with an associated collection
  \(\{u_{i} \in \mathcal U\}_{i \in I}\), then \(\prod_{i \in I} u_i \in
  \mathcal U\).
\end{enumerate}
\end{definition}

\begin{axiom}[Grothendieck's axiom to ZF set theory]
\label{def:Grothendieck-axiom-ZF-set-theory}
For any set \(X\), there exists an universe \(\mathcal U\) for which \(X \in
\mathcal U\).
\end{axiom}

For some commonly used terminology, we say that a set \(x\) is a \(\mathcal
U\)-set if \(x \in \mathcal U\). Moreover, a set \(x\) is called \(\mathcal
U\)-small if it is isomorphic to some set \(s \in \mathcal U\).

Orderings are another important topic when dealing with sets, we shall define
now some of these concepts.

\begin{definition}
\label{def:ordering-miscellaneous}
Let \(I\) be a set. We define the following:
\begin{itemize}\setlength\itemsep{0em}
\item An \emph{order} on the set \(I\) is a relation \(\leq\) satisfying the
  following properties:
  \begin{enumerate}[(a)]\setlength\itemsep{0em}
  \item The order is \emph{reflexive} --- that is, for all \(i \in I\), we have
    \(i \leq i\).
  \item The order is \emph{transitive} --- given any three elements \(i, j, k
    \in I\) such that \(i \leq j\) and \(j \leq k\), it follows that \(i \leq
    k\).
  \item The order is \emph{anti-symmetric} --- given any two elements \(i, j \in
    I\), if \(i \leq j\) and \(j \leq i\), then \(i = j\).
  \end{enumerate}
\item An order is said to be \emph{directed}, or \emph{filtrant}, if \(I\) is
  non-empty and, for every \(i, j \in I\), there exists \(k \in I\) for which
  \(i \leq k\) and \(j \leq k\).
\item An order is said to be \emph{total} if, given any \(i, j \in I\),
  necessarily have at least one of the following relations: \(i \leq j\) or \(j
  \leq i\).
\item The set \(I\) is said to be \emph{inductively ordered} if, for any totally
  ordered subset \(J \subseteq I\), \(J\) has an upper bound \(u \in I\) for
  which \(j \leq u\) for all \(j \in J\)
\item If \(I\) is ordered by the relation \(\leq\), we define a strict relation
  \(<\) as, given \(i, j \in I\), we have \(i \leq j\) if and only if \(i \leq
  j\) and \(i \neq j\).
\end{itemize}
\end{definition}

\section{Categories}

\begin{definition}[Category]\label{def: category}
A category \(\cat C\) consists of the following data
\begin{enumerate}[(C1)]
\item A collection of objects. We say that \(X\) is an object of \(\cat C\)
  by writing \(X \in \cat C\) or \(X \in \Obj(\cat C)\)\footnote{We shall adopt
  the former notation, which should not cause any confusion.}.
\item For every given pair of objects \(X, Y \in \cat C\) there exists a
  collection of morphisms \(\Hom_{\cat C}(X, Y)\) with source \(X\) and
  target \(Y\). The collection of morphisms between objects of \(\cat C\) is
  denoted \(\Mor(\cat C)\).
\item For every object \(X \in \cat C\), there exists an identity morphism
  \(\Id_X \in \Hom_{\cat C}(X, X)\).
\item For every triple of given objects \(X, Y, Z \in \cat C\), there exists
  a composition map
  \[
    \Hom_{\cat C}(X, Y) \times \Hom_{\cat C}(Y, Z) \to \Hom_{\cat C}(X, Z).
  \]
  So that for given morphisms \(f \in \Hom_{\cat C}(X, Y)\) and \(g \in
  \Hom_{\cat C}(Y, Z)\) there exists a uniquely defined map \(g  f \in
  \Hom_{\cat C}(X, Z)\) such that the following diagram commutes
  \[
    \begin{tikzcd}
      Y \ar[r, "g"]
        &Z \\
      X \ar[u, "f"] \ar[ru, dashed, swap, "g  f"]
    \end{tikzcd}
  \]
\item For every morphism \(f: X \to Y\) we have that
  \[
    \begin{tikzcd}
      X \ar[r, "f"] \ar[loop left, "\Id_X"] &Y \ar[loop right, "\Id_Y"]
    \end{tikzcd}
  \]
  so that \(\Id_Y  f = f = f  \Id_X\).
\item Given objects \(W, X, Y, Z \in \cat C\), the following diagram commutes
  \[
    \begin{tikzcd}
      W
      \ar[r, "f"]
      \ar[rr, bend right, swap, "g  f"]
      \ar[rrr, bend right = 50, swap, "h  (g  f)"]
      \ar[rrr, bend left = 50, "(h  g)  f"]
        &X
        \ar[r, "g"] \ar[rr, bend left, "h  g"]
          &Y
          \ar[r, "h"]
            &Z
    \end{tikzcd}
  \]
  that is, \(h  (g  f) = (h  g)  f\).
\end{enumerate}
\end{definition}

\begin{notation}[On arrows and diagrams]
I'll adopt throughout this whole text a series of notations regarding morphisms,
diagrams and so on, here I collect some of those: given a category \(\cat
C\) and objects \(x, y, z \in \cat C\)
\begin{itemize}\setlength\itemsep{0em}
\item An arrow \(x \mono y\) denotes a \emph{monomorphism} --- to be seen in
  \cref{def: monomorphism}.
\item An arrow \(x \epi y\) denotes an \emph{epimorphism} --- to be seen in
  \cref{def: epimorphism}.
\item An arrow \(x \isoto y\) denoted an \emph{isomorphism} --- to be seen in
  \cref{def:isomorphism}. We say that \(x\) is isomorphic to \(y\), and write
  \(x \iso y\), if there exists an isomorphism \(x \isoto y\).
\item Given arrows \(f: x \to y\) and \(g: y \to z\) in \(\cat C\), we denote
  the \emph{composition} of \(f\) with \(g\) by the \emph{juxtaposition} \(gf: x
  \to z\)\footnote{When need be, we may use the symbol \(g \circ f\) to denote
    the composition \(g f\).}.
\item Let the following be a commutative diagram on \(\cat C\) (that is, \(fg =
  h\)):
  \[
    \begin{tikzcd}
      x \ar[r, "f"] \ar[rd, swap, "h"] &y \ar[d, dashed, "g"] \\ &z
    \end{tikzcd}
  \]
  The \emph{dashed} arrow \(g: y \to z\) denotes that \(g\) is the \emph{unique}
  morphism between \(y\) and \(z\) in the category \(\cat C\) such that \(fg =
  h\).
\item The diagram in \(\cat C\)
  \[
    \begin{tikzcd}
      x \ar[r, shift right, swap, "h"] \ar[r, shift left, "g"]
      &y \ar[r, "f"]
      &z
    \end{tikzcd}
  \]
  denotes that \(f g = f h\).
\end{itemize}
\end{notation}

\begin{definition}[Small]\label{def: small cat}
A category \(\cat C\) is said to be small if it is composed of a set's worth of
morphisms.
\end{definition}

\begin{definition}[Locally small]
A category \(\cat C\) is said to be locally small if for any objects \(A,B \in
\cat C\) there exists a set's worth of morphisms \(A\) and \(B\).
\end{definition}

\begin{corollary}[Unique identity]\label{cor: unique identity}
Given a category \(\cat C\) and an object \(c \in \cat C\), the identity \(\Id_c
\in \Mor(\cat C)\) is unique.
\end{corollary}

\begin{proof}
Let \(f: c \to c\) be an identity of \(c\), then \(f = f \Id_c = \Id_c\).
\end{proof}

\begin{definition}[Isomorphism]\label{def:isomorphism}
Given a category \(\cat{C}\) and objects \(A, B \in \cat{C}\), we define a
morphism \(f \in \Hom(A, B)\) to be an \emph{isomorphism} if and only if it has
a both sided inverse, so that exists \(f^{-1} \in \Hom(B, A)\) such that
\(f^{-1}f = \Id_A\) and \(ff^{-1} = \Id_B\).
\end{definition}

\begin{proposition}\label{prop: iso unique inverse}
Given an isomorphism \(f\), its inverse is unique.
\end{proposition}

\begin{proof}
Suppose for instance that there are two such functions, \(g, h \in \Hom(B, A)\),
that act as an inverse for \(f \in \Hom(A, B)\). Note that
\[
  g = g \Id_B = g(f h) = (g f) h = \Id_A h = h
\]
Thus \(g = h\) and therefore the inverse is indeed unique.
\end{proof}

\begin{definition}[Non-empty category]
\label{def:non-empty-category}
A category is said to be \emph{non-empty} if the collection of objects is
non-empty.
\end{definition}

\begin{definition}[Discrete category]
\label{def:discrete-category}
A category is said to be \emph{discrete} if all morphisms are the identity
morphisms.
\end{definition}

\begin{definition}[Finite category]
\label{def:finite-category}
A category is said to be \emph{finite} if the collection of all morphisms is a
finite set.
\end{definition}

\begin{definition}[Connected category]
\label{def:connected-category}
A category \(\cat C\) is said to be connected if it is non-empty and, for every
pair \(x, y \in \cat C\), there exists a finite sequence of objects \((x_0,
\dots, x_n)\), \(x_j \in \mathcal C\) for all \(0 \leq j \leq n\), such that
\(x_0 = x\), \(x_n = y\) and at least one of the collections of morphisms
\(\Hom_{\cat C}(x_j, x_{j+1})\) or \(\Hom_{\cat C}(x_{j+1}, x_j)\) is non-empty
for every \(0 \leq j \leq n - 1\)
\end{definition}


\begin{definition}[Monoid]\label{def: monoid}
A monoid is a set \(M\) equipped with a binary operation \(\otimes: M \times M
\to M\) and a neutral element \(e \in M\). The binary operation is associative
and obeys the right and left unit laws, that is
\[
  x \otimes (y \otimes z) = (x \otimes y) \otimes z \qquad \text{ and } \qquad
  e \otimes x = x = x \otimes e.
\]
\end{definition}

\begin{example}
A monoid \(M\) defines a category with one object, denoted \(\cat{BM}\), such
that \(\Obj(\cat{BM}) = \{*\}\) and \(\Hom_{\cat{BM}}(*, *) =
\{*\}\). Composition of morphisms \(f, g: * \to *\) is defined as \(g f = g
\otimes f\). The identity morphism is \(\Id_* = e\). Hence we have \(e * f = f =
f * e\) for all morphisms \(f \in \Mor(\cat{BM})\).
\end{example}

\begin{definition}[Groupoids]\label{def: groupoids}
A \emph{groupoid} is the name given to a category in which all of its
morphisms are isomorphisms.
\end{definition}

\begin{definition}[Group]\label{def: group}
A group is a groupoid with one object.
\end{definition}

\begin{definition}[Automorphism]
Given a category \(\cat C\) and an object  \(A \in \cat C\), we define an
automorphism of \(A\) to be an isomorphism \(A \to A\). The set consisting of
all such automorphisms of this object is denoted \(\Aut(A)\), the
automorphisms have properties:
\begin{enumerate}[i.]
\item The composition of two automorphisms is an automorphism.
\item Composition is associative.
\item \(\Id_A \in \Aut(A)\).
\item Every automorphism \(f \in \Aut(A)\) has an inverse \(f^{-1}
    \in \Aut(A)\).
\end{enumerate}
With this, the structure \(\Aut(A)\) is a \emph{group}, for all choices of
objects \(A \in \cat{C}\).
\end{definition}

\begin{definition}[Subcategory]\label{def: subcategory}
Let \(\cat C\) be a category. We define \(\cat D \subseteq \cat C\) as a
subcategory of \(\cat C\) if: \(\Obj(\cat D)\) is a restriction of \(\Obj(\cat
C)\); for all \(A \in \Obj(\cat D)\) there exists \(\Id_A \in \Mor(\cat D)\);
for any \(f \in \Mor(\cat D)\) there exists \(\dom(f), \codom(f) \in
\Obj(\cat D)\); for any composable pair of morphisms \(f, g \in \Mor(\cat D)\)
there exists \(f g \in \Mor(\cat D)\).
\end{definition}

\begin{lemma}
Any category \(\cat C\) contains a subcategory containing all of the objects
and whose morphisms are only the isomorphisms. Such subcategory is called a
maximal groupoid.
\end{lemma}

\begin{proof}
We are going to prove that the maximal groupoid, name it \(\cat G\) is indeed
a subcategory of \(\cat C\). Notice that \(\Obj(\cat G) = \Obj(\cat C)\),
moreover every identity is an isomorphism, then \(\Id_{*} \in \Mor(\cat G)\).
Let \(f \in \Mor(\cat G)\) then in particular we have \(f \in \Mor(\cat C)\)
and hence \(\operatorname{dom} f, \operatorname{codom} f \in \Obj(\cat C) =
\Obj(\cat G)\). Consider \(f \in \Hom_{\cat G}(A, B)\) and \(g \in \Hom_{\cat
G}(B, C)\), composable isomorphisms, and notice that \(f^{-1}  g^{-1}
\in \Hom_{\cat C}(C, A)\) is such that \((f^{-1}  g^{-1})  (g
f) = \Id_A\) and \((g  f)  (f^{-1}  g^{-1}) = \Id_C\), hence we
conclude that \(g  f\) is an isomorphism and therefore \(g  f \in
\Hom_{\cat G}(A, C)\).
\end{proof}

\begin{definition}[Full subcategory]\label{def: full subcategory}
A subcategory \(\cat D\) of \(\cat C\) (see \cref{def: subcategory}) is said to
be a \emph{full subcategory} if for all \(x, y \in \cat D\) we have
\(\Hom_{\cat D}(x, y) = \Hom_{\cat C}(x, y)\). The full subcategory \(\cat D\)
is said to be \emph{saturated} if \(x \in \cat C\) is also an object of \(\cat D\)
whenever there exists an object \(u \in \cat D\) such that \(x \iso u\) in
\(\mathcal C\).
\end{definition}

\begin{example}[Set based categories]
\label{exp:set-based-categories}
The following are some important categories regarding sets:
\begin{itemize}\setlength\itemsep{0em}
\item The category \(\Set\) consists of \(\mathcal U\)-sets and set-maps between
  such objects.
\item The category consisting of \emph{finite} \(\mathcal U\)-sets and set-maps
  between them is a full subcategory of \(\Set\), we denote it by \(\FinSet\).
\item We also define the category \(\pSet\) of \emph{pointed} \(\mathcal
  U\)-sets, that is, objects are pairs \((X, x)\), where \(X\) is a \(\mathcal
  U\)-set and \(x \in X\). Morphisms \(f: (X, x) \to (Y, y)\) are defined to be
  maps \(f: X \to Y\) such that \(f(x) = y\).
\end{itemize}
\end{example}

\begin{remark}
The structure consisting of all sets and the set-maps between them does
\emph{not} shape a category, since the collection of all sets is not itself a
set.
\end{remark}

\begin{example}
\label{exp:order-category}
Let \((I, \leq)\) be an ordered set. We define a category \(\cat I\)
associated with \(I\) to consist of the collection of objects contained in
\(I\), and
\[
  \Hom_{\cat I}(i, j) =
  \begin{cases}
    \{*\}, &\text{if } i \leq j \\
    \emptyset, &\text{otherwise}
  \end{cases}
\]
\end{example}

\begin{definition}[Morphism category]
\label{def:morphism-category}
Let \(\cat C\) be a category. We'll denote by \(\Mor(\cat C)\) the category
whose objects are \emph{morphisms} in \(\cat C\) and whose morphisms between
given objects \(f: x \to y\) and \(g: z \to w\) are pairs of morphisms \((u,
v)\), with \(u: x \to z\) and \(v: y \to w\) such that the following diagram
commutes
\[
  \begin{tikzcd}
    x \ar[d, swap, "f"] \ar[r, "u"] &z \ar[d, "g"] \\
    y \ar[r, "v"] &w
  \end{tikzcd}
\]
That is, \(\Hom_{\Mor(\cat C)}(f, g) \coloneq \{(u, v) \in \Hom_{\cat C}(x, z)
\times \Hom_{\cat C}(y, w) \colon g u = v f\}\).
\end{definition}

\begin{proposition}[Slice category]\label{prop: slice cat}
Given a category \(\cat C\) and an object \(c \in \cat C\). The following
define categories:
\begin{enumerate}[(SC1)]
\item\label{prop: slice under}
  (Slice under \(c\)) A category \(c/\cat C\), called slice category of
  \(\cat C\) under \(c\), whose objects are morphisms \(f \in \Hom_{\cat
  C}(c, *)\). Given objects \(f, g \in c/\cat C\) such that \(f: c \to x\)
  and \(g: c \to y\), we define a morphism \(f \to g\) as a map \(h: x \to
  y\) such that the following diagram commutes
  \[
    \begin{tikzcd}
      &c \ar[dl, swap,"f"] \ar[dr, "g"]\\
      x \ar[rr, "h"] & &y
    \end{tikzcd}
  \]
  that is, \(g = h  f\).
\item\label{prop: slice over}
  (Slice over \(c\)) A category \(\cat C/c\), called the slice category of
  \(\cat C\) over \(c\), whose objects are morphisms \(f \in \Hom_{\cat
  C}(*, c)\). Morphisms between objects \(f, g \in \cat C/c\) such that \(f:
  x \to c\) and \(g: y \to c\) are maps \(h: x \to y\) such that the
  following diagram commutes
  \[
    \begin{tikzcd}
      x \ar[rr, "h"] \ar[dr, swap, "f"]
        & &y \ar[dl, "g"] \\
        &c
    \end{tikzcd}
  \]
  so that \(f = g  h\).
\end{enumerate}
\end{proposition}

\begin{proof}
(SC1) Given objects \(f, g \in c/\cat C\) we have from construction that
\[
  \Hom_{c/\cat C}(f, g) = \Hom_{\cat C}(\codom f, \codom g),
\]
hence we are
ensured of the existence of such morphisms between objects. Given an object
\(f: c \to x\), the morphism \(\Id_x \in \Mor(\cat C)\) is such that \(f =
\Id_x  f\), so that \(\Id_x\) is the identity morphism for \(f\). Let
\(f, g, u \in c/\cat C\) be objects such that \(f: c \to x,\ g: c \to y,\ u: c
\to z\), then there exists morphisms \(h \in \Hom_{c/\cat C}(f, u)\) and
\(\ell: \Hom_{c/\cat C}(u, g)\) so that the following diagram commutes
\[
  \begin{tikzcd}
      &c \ar[dl, swap, "f"] \ar[d, "u"] \ar[dr, "g"] & \\
    x \ar[r, "h"]
      &z \ar[r, "\ell"]
        &y
  \end{tikzcd}
\]
so that we have a uniquely defined morphism \(\ell  h \in \Hom_{\cat
C}(x, y)\), where \(\ell  h \in \Hom_{c/\cat C}(f, g)\) and
thus this defines a map \(\Hom_{c/\cat C}(f, u) \times \Hom_{c/\cat C}(u, g)
\to \Hom_{c/\cat C}(f, g)\). Since \(\Id_x, \Id_z \in \Mor(c/\cat C)\) for any
\(x, y \in \cat C\), then given \(f, u \in c/\cat C\) just as above, the
morphism \(h: x \to z\) is such that \(\Id_z  h = h = h  \Id_x\). In
addition to the objects and morphisms named above, define \(v: c \to w\) and
the corresponding morphism \(t \in \Hom_{c/\cat C}(g, v)\) so that \(t: y \to
w\) and \(v = t  g\). From the fact that \(\cat C\) is a category, we
find that \((t  \ell)  h = t  (\ell  h)\) and hence the
same is true for \(c/\cat C\).

(SC2) Let \(f, g \in \cat C/c\), then from definition we have \(\Hom_{\cat
C/c}(f, g) = \Hom_{\cat C}(\dom f, \dom g)\), which is well defined on \(\cat
C\). Given an object \(f: x \to c\), there exists \(\Id_x: x \to x\) so that
\(f = f  \Id_x\) and hence \(\Id_x\) is the identity morphism for \(f\).
Define objects \(f, g, u \in \cat C/c\) such that \(f: x \to c,\ g: y \to c,\
u: c \to z\), and morphisms \(h \in \Hom_{\cat C/c}(u, f)\) and \(\ell \in
\Hom_{\cat C/c}(g, u)\). Then, the following diagram commutes
\[
  \begin{tikzcd}
    x
      &z \ar[l, swap, "h"]
        &y \ar[l, swap, "\ell"] \\
      &c \ar[lu, "f"] \ar[u, "u"] \ar[ru, swap, "g"]
  \end{tikzcd}
\]
hence the morphism \(h  \ell \in \Hom_{\cat C}(y, x)\) defines a morphism
\(h  \ell \in \Hom_{\cat C/c}(g, f)\) so that we can construct a map well
defined map \(\Hom_{\cat C/c}(g, u) \times \Hom_{\cat C/c}(u, f) \to
\Hom_{\cat C/c}(g, f)\). Moreover, given \(f\) and \(u\) as above, we have
that there exists \(\Id_x, \Id_z \in \Mor(\cat C/c)\) so that from the
category \(\cat C\) it follows that \(\Id_x  h = h  \Id_z\). In
addition to the above, define \(v: w \to c\) and the morphism \(t \in
\Hom_{\cat C/c}(v, g)\) so that \(t: w \to y\). Then we have \(g = t  v\)
and since \(\cat C\) is a category, we find that \((h  \ell)  t = h
  (\ell  t)\).
\end{proof}

\subsection{Initial \& Final Objects}

\begin{definition}[Initial, final and zero objects]
\label{def: initial and final objects}
Given a category \(\cat C\), we define the following objects:
\begin{enumerate}[(a)]\setlength\itemsep{0em}
\item An object \(I \in \cat C\) is said to be an \emph{initial object} in the
  category \(\cat{C}\) if, for all \(A \in \cat C\), there exists exactly one
  morphism \(f \in \Hom(I, A)\) so that \(\Hom(I, A) = f\).
\item An object \(F \in \cat C\) is said to be a \emph{final object} in \(\cat
  C\) if there is exactly one morphism \(g \in \Hom(A, F)\) for all given \(A \in
  \cat C\) and thus \(\Hom(A, F) = g\).
\item An object \(O \in \cat C\) is said to be a \emph{zero} object if it is both
  initial and terminal.
\end{enumerate}
\end{definition}

\begin{proposition}\label{prop:initial-final-unique}
Let a category \(\cat{C}\), then initial and final objects are said to be
unique up to a unique isomorphism:
\begin{enumerate}[I.]
\item If \(I, I' \in \cat{C}\) are initial objects of the
    category, then \(I \iso I'\), where the isomorphism \(\varphi_I: I
    \isoto I'\) is unique.
\item If \(F, F' \in \cat{C}\) are initial objects of the
    category, then \(F \iso F'\), where the isomorphism \(\varphi_F: F
    \isoto F'\) is unique.
\end{enumerate}
\end{proposition}

\begin{proof}
Since \(I, I'\) are both initial objects of \(\cat{C}\) then exists a unique
morphism \(f \in \Hom(I, I')\) and there exists a unique \(g \in \Hom(I',
I)\) from which we can compose and conclude that \(f g = \Id_{I'}\) and \(gf =
\Id_{I}\) and thus \(f\) and \(g\) are isomorphisms and also \(f^{-1} = g\) and
\(g^{-1} = f\).  For final objects the same reasoning works and thus the proof
will be omitted.
\end{proof}

We normally say that a given construction \emph{satisfies a universal property}
if such construction is a terminal object of the category.
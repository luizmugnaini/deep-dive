\section{Universal Properties of \texorpdfstring{\(\Graph\)}{Graph}}

\subsection{Quotients}

\begin{definition}[Quotient]\label{def: quotient-graph}
  Let \(G\) be a graph and \(\sim\) be an equivalence relation on the collection
  of vertices \(V(G)\). We define the quotient graph \(G/{\sim}\) to be the graph
  whose vertices are the vertex classes \(V(G)/{\sim}\) and whose edges are
  such that \(E_{G/{\sim}}([x], [y])\) if and only if \(E_G(x, y)\).
\end{definition}

\begin{proposition}[Universal property of quotients]
  \label{prop: quotient-graph}
  The quotient graph is a quotient in the category \(\Graph\). In other words,
  let \(G\) be a graph and \(\sim\) be an equivalence relation on \(V(G)\).
  Consider any graph \(H\) together with a morphism of graphs \(\psi: G \to H\).
  Then there exists a unique morphism of graphs \(\phi: G/{\sim}\, \to H\) such
  that the following diagram commutes
  \[
    \begin{tikzcd}
      G \ar[d, two heads, swap, "\pi"] \ar[r, "\psi"] &H \\
      G/{\sim} \ar[ur, swap, bend right, dashed, "\phi"] &
    \end{tikzcd}
  \]
  where \(\pi\) is the naturally defined projection.
\end{proposition}

\begin{proof}
  First we show that \(\phi\) is indeed unique. Suppose \(\phi_1\) and
  \(\phi_2\) are both morphisms that satisfy the commutativity of the diagram.
  Given any \(x \in V(G)\) we have \(\psi(x) = \phi_1([x]) = \phi_2([x])\) and
  since \(\pi(V(G)) = V(G/{\sim})\) (surjective property) then \(\phi_1\) and
  \(\phi_2\) have equal images throughout their whole domain --- implying
  \(\phi_1 = \phi_2\). To show that \(\phi\) is a morphism of graphs, it is
  sufficient to consider any \(x, y \in G\) such that \(E_G(x, y)\): \(\psi\)
  being a morphism implies \(E_G(\psi(x), \psi(y))\), then --- since \(\phi([x])
  = \psi(x)\) and \(\phi([y]) = \psi(y)\) --- we get that \(E_G(\phi([x]),
  \phi([y]))\) is true.
\end{proof}

\subsection{Coproducts}

\begin{definition}[Coproduct]\label{def: coprod-graph}
  Let \(G = (V, E)\) and \(H = (V', E')\) be graphs, we define their coproduct
  \(G \oplus H\) to be the disjoint union of vertices and edges of the original
  graphs --- that is, \(G \oplus H = (V \amalg V', E \amalg
  E')\)\footnote{\(\amalg\) denotes the standard disjoint union in \(\Set\).}.
\end{definition}

\begin{proposition}[Universal property of coproducts]
  \label{prop: coprod-graph}
  Given graphs \(G, H \in \Graph\), the graph \(G \oplus H\) is a coproduct of
  \(G\) and \(H\) in the category of simple loopless graphs. That is, given any
  graph \(W \in \Graph\) and graph morphisms \(f: G \to W\) and \(g: H \to W\),
  there exists a unique graph morphism \(\phi: G \oplus H \to W\) such that the
  following diagram commutes
  \[
    \begin{tikzcd}
      G \ar[dr, hook, "\iota_G"] \ar[ddr, swap, bend right, "f"]
      & &H \ar[dl, swap, hook', "\iota_{H}"] \ar[ddl, bend left, "g"]
      \\
      &G \oplus H \ar[d, dashed, "\phi"]
      \\
      &W &
    \end{tikzcd}
  \]
  Where the inclusion morphisms \(\iota_G\) and \(\iota_{H}\) are naturally
  defined.
\end{proposition}

\begin{proof}
  Notice that \(\phi\) is defined by a map of vertices \(V(G) \cup V(H) \to
  V(W)\) with the restriction that \(E_{G \oplus H}(x, y)\) implies
  \(E_W(\phi(x), \phi(y))\), we'll first prove that \(f\) together with \(g\)
  completely identify \(\phi\). Let \(v \in G \oplus H\) be any vertex. If \(v
  \in V(G)\), suppose \(f(v) = h\) then necessarily \(\phi \iota_G(v) = h\). On
  the other hand, if \(v \in V(H)\) and if \(g(v) = h'\) then \(\phi
  \iota_{H}(v) = h'\). Since \(\codom \phi = V(G) \amalg V(H)\) and we have
  \(\iota_G(G) = V(G)\) and \(\iota_{H}(H) = V(H)\), this shows that \(f\) and
  \(g\) completely determine the image of \(\phi\) --- hence \(\phi\) is
  uniquely defined.

  Finally we show that \(\phi\) is indeed a morphism of graphs. Given \(x, y \in
  V(G)\), suppose \(E_G(x, y)\), then \(E_W(f(x), f(y)) = E_W(\phi(x),
  \phi(y))\) --- since \(\phi(x) = f(x)\) and \(\phi(y) = f(y)\). The case \(x,
  y \in V(H)\) is completely analogous and we'll therefore omit for the sake of
  brevity. We don't need to inspect the case where \(x \in V(G)\) and \(y \in
  V(H)\) since \(E_{G \oplus H}(x, y)\) is always false in such instance. Thus
  \(\phi\) is a morphism of graphs.
\end{proof}

\subsection{Products}

\begin{definition}[Kronecker product]
  Let \(G, H \in \Graph\). We define the Kronecker product of \(G\) and \(H\) to
  be the graph \(G \otimes H = (V, E)\) whose vertices are \(V = V(G) \times
  V(H)\) and edges defined by \(E_{G \otimes H}(v \otimes h, v' \otimes h')\) if
  and only if \(E_G(v, v')\) and \(E_H(h, h')\).
\end{definition}

\begin{proposition}[Products]
  The Kronecker product as defined above is a product in the category of simple
  loopless graphs \(\Graph\). That is, given graphs \(G, H, W \in \Graph\) and
  graph morphisms \(f: W \to G\) and \(g: W \to H\), there exists a unique
  morphism of graphs \(\phi: W \to G \otimes H\) such that the following diagram
  commutes
  \[
    \begin{tikzcd}
      &W \ar[d, dashed, "\phi"]
      \ar[ddl, bend right, swap, "f"]
      \ar[ddr, bend left, "g"]
      & \\
      &G \otimes H
      \ar[dr, two heads, swap, "\pi_H"]
      \ar[dl, two heads, "\pi_G"]
      & \\
      G & &H
    \end{tikzcd}
  \]
  Where \(\pi_G\) and \(\pi_H\) are the naturally defined projection morphisms.
\end{proposition}

\begin{proof}
  Consider the morphism \(\phi: W \to G \otimes H\) defined by the mapping \(w
  \xmapsto \phi f(w) \otimes g(w)\). Let \(w, w' \in V(W)\) be any vertices such
  that \(E_W(w, w')\), since \(f\) and \(g\) are graph morphisms, then
  \(E_G(f(w), f(w'))\) and \(E_H(g(w), g(w'))\). Since \(\phi(w) = f(w) \otimes
  g(w)\) and \(\phi(w') = f(w') \otimes g(w')\) it follows from the construction
  of the Kronecker product that \(E_{G \otimes H}(\phi(w), \phi(w'))\). This
  shows that \(\phi\) is a graph morphism.

  We now inspect its uniqueness. Let \(\phi, \phi' \in \Hom_\Graph(W, G \otimes
  H)\) be morphisms satisfying the commutativity of the diagram. Then \(f =
  \pi_G \phi = \pi_G \phi'\) and \(g = \pi_H \phi = \pi_H \phi'\), thus, given
  any \(w \in W\) we have \(\pi_G\phi(w) = \pi_G \phi'(w)\) and \(\pi_H \phi(w)
  = \pi_H\phi'(w)\), which implies in \(\phi(w) = \phi'(w)\). Therefore \(\phi\)
  is unique.
\end{proof}

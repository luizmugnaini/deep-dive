\section{Determinants and Inverse Matrices}

\begin{definition}\label{def: wedge map}
  Let \(f: V \to L\) be a \(k\)-linear map. We define the induced linear map
  \(f^{\wedge d}: \Lambda^d(V) \to \Lambda^d(W)\) by
  \[
    v_1 \wedge \dots \wedge v_d \xmapsto{f^{\wedge d}}
    f(v_1) \wedge \dots \wedge f(v_d).
  \] 
\end{definition}

\begin{proposition}\label{prop: wedge map composition}
  Let \(f: V \to W\) and \(g: W \to L\) be \(k\)-linear maps. Then the
  composition \(g f: V \to L\) satisfy
  \[
    (g f)^{\wedge d} = g^{\wedge d} f^{\wedge d}.
  \]
\end{proposition}

\begin{proof}
  Notice that for any given \(v = v_1 \wedge \dots \wedge v_d \in \Lambda(V)\)
  we have
  \[
    (g f)^{\wedge d}(v) = (g f)^{\wedge d}(v_1) \wedge \dots \wedge (g f)(v_d)
    = g(f(v_1)) \wedge \dots \wedge g(f(v_2))
    = (g^{\wedge d} f^{\wedge d})(v).
  \] 
  Therefore \((g f)^{\wedge d} = g^{\wedge d} f^{\wedge d}\).
\end{proof}

\begin{definition}[Determinant]\label{def: determinant}
  Let \(V\) be an \(n\)-dimensional \(k\)-vector space and \(f \in \End(V)\).
  Since \(\Lambda^n(V)\) is 1-dimensional, \(f^{\wedge n} \in \End(\Lambda^n
  (V))\) is a multiplication by some scalar in \(\Lambda^n(V)\).  We define the
  determinant of the \(k\)-linear endomorphism \(f\) as the element \(\det(f)
  \in \Lambda^n(V)\) such that
  \[
    f^{\wedge n}(\omega) = \det(f) \omega.
  \] 
\end{definition}

\begin{proposition}[Composition determinant]\label{prop: comp det}
  Let \(f, g \in \End(V)\) where \(V\) is a \(n\)-dimensional vector space, then
  \(\det(g f) = \det(g) \det(f)\).
\end{proposition}

\begin{proof}
  From \cref{prop: wedge map composition}, \((g f)^{\wedge n}(\omega) =
  g^{\wedge n}(f^{\wedge n}(\omega))\). From \cref{def: determinant} we find
  \(\det(g f) \omega = \det(g) \det(f) \omega\), hence \(\det(g f) = \det(g)
  \det(f)\).
\end{proof}

\begin{proposition}[Identity determinant]\label{prop: id det}
  Let \(V\) be an \(n\)-dimensional vector space. The identity map \(\Id_V \in
  \End(V)\) is such that \(\det(\Id_V) = 1\).
\end{proposition}

\begin{proof}
  Notice that \(\Id_V^{\wedge n}(\omega) = \omega\) and from \cref{def:
  determinant} we have \(\Id_V^{\wedge n}(\omega) = \det(\Id_V) \omega\). Hence
  \(\det(\Id_V) = 1\).
\end{proof}

\begin{proposition}[Dual determinant]
  Let \(V\) be a finite dimensional vector space and \(f \in \End(V)\). Then
  \(\det f = \det f^*\), where \(f^*\) is the dual map of \(f\).
\end{proposition}

\todo[inline]{Dual determinant proof}

\begin{lemma}\label{lem: not surjective 0 det}
  Let \(V\) be an \(n\)-dimensional vector space. If \(f \in \End(V)\) is not
  surjective, then \(\det f = 0\).
\end{lemma}

\begin{proof}
  Let \(\im f = W \subseteq V\). Since \(f\) is not surjective, then \(W\) is a
  subspace with \(\dim(W) < n\), hence \(\Lambda^n(W) = 0\) because any
  collection with a number of elements greater than \(\dim(W)\) is linearly
  dependent on \(W\), and from \cref{prop: li iff wedge nonzero} their wedge
  product is equal to zero. Given any nonzero \(\omega = v_1 \wedge \dots \wedge
  v_n \in \Lambda^n(V)\) we find that \(f^{\wedge n}(\omega) = f(v_1) \wedge
  \dots \wedge f(v_n) \in \Lambda^n(W) = 0\), since \(\im f = W\), and therefore
  \(f^{\wedge n}(\omega) = 0\). Moreover, since \(f^{\wedge n}(\omega) = \det(f)
  \omega\) and \(\omega \neq 0\), it follows that \(\det(f) = 0\).
\end{proof}

\begin{proposition}[Isomorphism determinant]
  \label{prop: iso det}
  Let \(f \in \End(V)\) where \(V\) is an \(n\)-dimensional vector space. Then
  \(f\) is an isomorphism if and only if \(\det f \neq 0\).
\end{proposition}

\begin{proof}
  Let \(\{v_1, \dots, v_n\}\) be a basis for \(V\). From \cref{prop: li iff
  wedge nonzero}, \(v = v_1 \wedge \dots \wedge v_n \neq 0\).  Suppose first
  that \(f\) is an isomorphism, then \(f^{\wedge n}(v) = \det(f) v\) implies
  \(\det f \neq 0\).

  Suppose now that \(\det f \neq 0\). From \cref{lem: not surjective 0 det} we
  find that \(f\) is surjective. Let \(\omega \in \Lambda^n(V)\) be such that
  \(f^{\wedge n}(\omega) = 0\), then from the fact that \(f^{\wedge n}(\omega) =
  \det(f) \omega\) and \(\det f \neq 0\), it follows that \(\omega = 0\). Hence
  \(\ker f^{\wedge n} = 0\) and therefore \(f^{\wedge n}\) is injective. This
  proves that \(f^{\wedge n}\) is an isomorphism.
\end{proof}

\begin{proposition}[Matrix determinant]
  \label{prop: matrix det}
  Let \(V\) be an \(n\)-dimensional \(k\)-vector spaces and \(f \in \End(V)\).
  Let \(A: k^n \to k^n\) be the matrix representation of \(f\) and \(a_{i, j}
  \in k\) be the entries of \(A\), where \(1 \leq i, j \leq n\). Then
  \[
    \det A = \sum_{\sigma \in \mathcal S_n} \sign(\sigma) a_{\sigma(1), 1}
    a_{\sigma(2), 2} \dots a_{\sigma(n), n}.
  \]
\end{proposition}

\begin{proof}
  Let me show you something interesting. Suppose \(f: V \to L\) (the specified
  case for \cref{prop: matrix det} is \(L = V\)). I want to show you that in
  order to define the determinant we'll need \(f\) to be an endomorphism in
  \(V\), otherwise the determinant cannot be fully well-defined. Let \(A =
  [a_{i, j}]\) with respect to the basis \(\{v_j\}_{j=1}^n\) of \(V\). From
  \cref{def: determinant} we have that
  \[
    \det(A) v_1 \wedge \dots \wedge v_n = A v_1 \wedge \dots \wedge A v_n
  \] 
  since \(A v_j = \sum_{i = 1}^n a_{i, j} v_i\) (see \cref{def: matrix for a
  linear map}) then we can substitute to the previous equation to obtain
  \begin{equation}\label{eq: det sum}
    \det(A) v_1 \wedge \dots \wedge v_n 
    = \sum_{i_1 = 1}^n a_{i_1, 1} v_{i_1} \wedge \dots \wedge \sum_{i_n = 1}^n
    a_{i_n, n} v_{i_n}
    = \sum_{1 \leq i_1, \dots, i_n \leq n} \prod_{j=1}^n a_{i_j, j} (v_{i_1}
    \wedge \dots \wedge v_{i_n}).
  \end{equation}
  However, notice that if \(i_j = i_{j'}\) for some \(1 \leq j, j' \leq n\) then
  \(v_{i_1} \wedge \dots \wedge v_{i_n} = 0\), from antisymmetry.
  Therefore we can write \cref{eq: det sum} as a sum of permutations of the set
  \(\{i_1, \dots, i_n\} = \{1, \dots, n\}\), that is
  \begin{align*}
    \det(A) v_1 \wedge \dots \wedge v_n
    &= \sum_{\sigma \in \mathcal S_n} a_{\sigma(1), 1} \dots a_{\sigma(n), n}
    (v_{\sigma(1)} \wedge \dots \wedge v_{\sigma(n)}) \\
    &= \sum_{\sigma \in \mathcal S_n} \sign(\sigma) a_{\sigma(1), 1} \dots
    a_{\sigma(n), n} (v_1 \wedge \dots \wedge v_n).
  \end{align*}
  Since \(v_1 \wedge \dots \wedge v_n \neq 0\), it follows that
  \[
    \det A = \sum_{\sigma \in \mathcal S_n} \sign(\sigma) a_{\sigma(1), 1}
    a_{\sigma(2), 2} \dots a_{\sigma(n), n}.
  \] 
\end{proof}

\begin{corollary}
  Another equivalent way of writing the determinant of \(A\) is 
  \[
    \det A = \sum_{\sigma \in \mathcal S_n} \sign(\sigma) a_{1, \sigma(1)} a_{2,
    \sigma(2)} \dots a_{n, \sigma(n)}.
  \]
\end{corollary}

\todo[inline]{Useful determinant theorems, matrix determinant, etc}

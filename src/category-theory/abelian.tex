\begin{definition}
\label{def:additive-category}
A category \(\cat C\) is said to be \emph{additive} if
\begin{enumerate}[(a)]\setlength\itemsep{0em}
\item For any pair of objects \(X, Y \in \cat C\) the morphism collection
  \(\Mor_{\cat C}(X, Y)\) is an abelian group.

\item Consider the following diagram in \(\cat C\):
  \[
  \begin{tikzcd}
  X \ar[r, "a"]
  &A \ar[r, shift left, "f"] \ar[r, shift right, "g"']
  &B \ar[r, "b"]
  &Y
  \end{tikzcd}
  \]
  Then the following relations are satisfied, which mimic the familiar
  distributive law:
  \begin{align*}
    b (f + g) &= b f + b g, \\
    (f + g) a &= f a + g a.
  \end{align*}

\item The category \(\cat C\) has a zero object.

\item The category \(\cat C\) admits finite products and finite coproducts.
\end{enumerate}

Given additive categories \(\cat C\) and \(\cat D\), we define an \emph{additive
  functor} between \(\cat C\) and \(\cat D\) to be a functor
\(F: \cat C \to \cat D\) such that for all parallel morphisms
\(f, g: A \para B\) we have
\[
F(f + g) = F f + F g,
\]
for any \(A, B \in \cat C\). In other words, the induced map
\(\Hom_{\cat C}(A, B) \to \Hom_{\cat D}(F A, F B)\) is a morphism of abelian
groups.
\end{definition}

% \begin{definition}
% \label{def:}
% Let \(k\) be a commutative ring. We say that a category \(\cat C\) is a
% \(k\)-category if for any pair of objects \(A, B \in \cat C\) the morphism
% collection \(\Hom_{\cat C}(A, B)\) has the structure of a \(k\)-module.
% \end{definition}

%%% Local Variables:
%%% mode: latex
%%% TeX-master: "../../deep-dive"
%%% End:

\section{Countability}

\begin{definition}[First countable]\label{def: first countable}
Let \(X\) be a topological space. We say that \(X\) is first countable if for
all points \(p \in X\) there exists a countable basis of \(X\) composed of
neighbourhoods of \(p\). Equivalently \(\chi(p, X) \leq \aleph_0\).
\end{definition}

\subsection{Convergence on First Countable Spaces}

\begin{proposition}[Sufficient condition for Hausdorff]
Let \(X\) be a first countable space. Then, \(X\) is Hausdorff if and only if
every sequence has at most one limit
\end{proposition}

\begin{proof}
Let \(x, y \in X\) be distinct points, and \(\mathcal B\) be a neighbourhood
basis at \(x\) and \(\mathcal A\) be a neighbourhood basis at \(y\). If \(X\)
is not Hausdorff, then for all \(n \in \N\), choose a point \(x_n \in
B_n \cap A_n\), where \(B_n \in \mathcal B\) and \(A_n \in \mathcal A\) and
consider the sequence \(\{x_n\}_{n \in \N}\). Then there exists
sub-sequences \(\{x_k\}\) and \(\{x_j\}\) for which \(x_k \to x\) and \(x_j \to
y\).
\end{proof}

\begin{definition}[Nested neighbourhood basis]
Let \(X\) be a topological space and a point \(p \in X\). The infinite sequence
of neighbourhoods of \(p\), namely \((U_i)_{i \in \N}\) is said to be
a nested neighbourhood basis at \(p\) if for all \(i \in \N, U_{i+1}
\subseteq U_i\) and for all neighbourhood \(V\) of \(p\), there exists \(i \in
\N\) for which \(U_i \subseteq V\).
\end{definition}

\begin{lemma}
Let \(X\) be a first countable topological space. Then, for all points \(p \in
X\) there exists a nested neighbourhood basis at \(p\).
\end{lemma}

\begin{proof}
Since \(X\) is first countable, let \(\mathcal V\) be a countable basis for the
topology of \(X\) at \(p\). If \(|\mathcal V| < \infty\) then define \(U_i \coloneq
V_1 \cap \dots \cap V_{|\mathcal V|}\) for all \(i \in \N\). If
\(|\mathcal V|\) is infinite, then define for all \(i \in \N\) the
set \(U_i \coloneq V_1 \cap \dots \cap V_i\). For both cases, the sequence
\((U_i)_{i \in \N}\) is a nested neighbourhood basis.
\end{proof}

\begin{definition}[Eventually in]
Let \(X\) be a topological space, and a sequence \((x_i)_{i \in \N}
\subseteq X\), and a set \(A \subseteq X\). We say that \((x_i)_{i \in
\N}\) is eventually in \(A\) if \(x_i \in A\) for all but finitely
many \(i \in \N\).
\end{definition}

\begin{lemma}[Sequence lemma]\label{lem: sequence lemma}
Let \(X\) be a first countable space, and a set \(A \subseteq X\), and a point
\(p \in X\). Then
\begin{enumerate}[(a)]
  \item \(p \in \overline A\) if and only if \(p\) is a limit of points of
    \(A\).
  \item \(A\) is closed in \(X\) if and only if \(A\) contains every limit
    point of sequences in \(A\).
  \item \(p \in \Int(A)\) if and only if all sequences that converge to
    \(p\) are eventually in \(A\).
  \item \(A\) is open in \(X\) if and only if every sequence in \(X\)
    converging to a point of \(A\) is eventually in \(A\).
\end{enumerate}
\end{lemma}

\begin{proof}
(a) (\(\Rightarrow\)) Let \(p \in \overline A\), then for all neighbourhoods
\(V \subseteq A\) of \(p\), the set \(V \cap (A \setminus \{p\})\) is
nonempty. Since \(X\) is first countable, consider \((U_i)_{i \in \N}\) a nested
neighbourhood basis at \(p\) and construct the sequence \(x : \N \to \bigcup_{i
\in \N} U_i \cap (A \setminus \{p\}) \) defined as \(i \mapsto x_i \in U_i \cap
(A \setminus \{p\})\). We'll show that \(x_i \to p\). Consider \(V \subseteq X\)
any neighbourhood of \(p\), since \((U_i)_{i \in \N}\) is a basis then there
exists an indexing set \(I_{V} \subseteq \N\) for which \(V = \bigcup_{i \in
I_V} U_i\).  Consider the index \(n \coloneq \min(I_V)\), then, from the
definition of the nested basis, we have for all \(i \geq n\) the elements \(x_i
\in V \cap A\), hence \(x_i \to p\).
(\(\Leftarrow\)) Suppose that \((x_i)_{i \in \N}\) is a sequence of points in
\(A\) such that \(x_i \to p\). From definition, for all neighbourhood \(V
\subseteq X\) of \(p\), there exists \(n \in \N\) such that \(\forall i \geq N,
x_i \in V\), moreover, since \(x_i \in A\) then \(x_i \in V \cap A\) which
implies that \(p \not\in \Ext(A)\), hence \(p \in \overline A\).

(b) (\(\Rightarrow\)) Suppose that \(A\) is closed, then \(A = \overline A\).
Consider any sequence \((x_i)_{i \in \N}\) of points in \(A\) and let
\(x_i \to p\). From item (a) we have that \(p \in A\). (\(\Leftarrow\))
Suppose the contrary, then given any \(p \in \overline A\) we have \(p \in
A\), which implies that \(A\) is closed.

(c) (\(\Rightarrow\)) Suppose \(p \in \Int(A)\). Let \((x_i)_{i \in
\N} \subseteq X\) be a sequence such that \(x_i \to p\). Consider any
neighbourhood \(V \subseteq A\) of \(p\), then from the definition of
convergence, exists \(N \in \N\) such that \(\forall i \geq N, x_i \in
V_p\), then there exists at most \(N - 1\) points of \((x_i)_{i \in
\N}\) outside \(A\), hence the sequence is eventually in \(A\).
(\(\Leftarrow\)) Suppose \((x_i)_{i \in \N}\) is not-eventually in
\(A\) and \(x_i \to p\). For the sake of contradiction, suppose that \(p \in
\Int(A)\), then given a neighbourhood \(V \subseteq A\) of \(p\), there must
exist \(N \in \N\) such that \(\forall i \geq N, x_i \in V \subseteq
A\), which is a contradiction to the fact that \((x_i)_{i \in \N}\) is
not-eventually in \(A\), hence \(p \not\in \Int(A)\).

(d) (\(\Rightarrow\)) Let \(A\) be open, then \(A = \Int(A)\). Given any \(p
\in A\) we have from item (c) that \(p\) is the limit of all of its converging
sequences are eventually in \(A\). (\(\Leftarrow\)) Suppose the contrary, and
let \(p \in A\) be any point. From item (c) we see that \(p \in \Int(A)\),
which implies that \(A \subseteq \Int(A)\) and hence \(A\) is open.
\end{proof}

\subsection{Second Countable Spaces and Covers}

\begin{definition}[Second countable]\label{def: second countable}
A topological space \(X\) is second countable if it admits a countable basis
for its topology. Equivalently \(w(X) \leq \aleph_0\).
\end{definition}

\begin{definition}[Cover]\label{def: cover}
Let \(X\) be a topological space and \(\mathcal U \subseteq 2^X\). We say that
\(\mathcal U\) is a cover of \(X\) if for all points \(x \in X\) there exists
\(U \in \mathcal U\) such that \(x \in U\).
\end{definition}

\begin{definition}[Open and closed covers]
Let \(\mathcal U\) be a cover of a space \(X\). If all elements \(U \in
\mathcal U\) are opens in \(X\), then \(\mathcal U\) is said to be open. On
the other hand, if all \(U \in \mathcal U\) are closed in \(X\), then
\(\mathcal U\) is said to be closed.
\end{definition}

\begin{definition}[Subcover]
Let \(\mathcal U\) be a cover of a space \(X\). If \(\mathcal U' \subseteq
\mathcal U\) is a cover for \(X\), then we call it a subcover of \(\mathcal
U\).
\end{definition}

Altough formally we should always write down if the cover is either open or
closed, I'll slip here and there and every time that I use the term ``cover''
and do not specify its type, it should be understood that I'm talking about open
covers --- otherwise I'll specifically write ``closed cover'' in order to avoid
any confusion.

\begin{proposition}[Basis out of covers]
\label{prop:union-cover-basis}
Let \(\mathcal U\) be an open cover of the space \(X\). For all \(U \in \mathcal
U\), define \(\mathcal{B}_U\) as the basis for the subspace \(U\). Then union
\(\bigcup_{U \in \mathcal U} \mathcal{B}_U\) is a basis for \(X\).
\end{proposition}

\begin{proof}
Define \(\mathcal B \coloneq \bigcup_{U \in \mathcal U} \mathcal B_{U}\).
Clearly \(\bigcup_{B \in \mathcal B} B = \bigcup_{U \in \mathcal U} \bigcup_{B
\in \mathcal B_U} B = X\) --- since every element of \(X\) can be found in
\(\mathcal U\) and \(\mathcal B_U \subseteq 2^U \subseteq 2^{X}\). Let \(x \in
X\) be any point and let \(U \in \mathcal U\) be a neighbourhood of
\(x\). Notice that on \(U\) the basis \(\mathcal B_U\) satisfies the local
intersecting condition (see \cref{item:basis-intersection-property}). Since \(x
\in X\) is any point, then the condition is true globally for \(\mathcal B\),
thus \(\mathcal B\) is a basis for \(X\).
\end{proof}

\begin{corollary}[Second countable out of a cover]
\label{cor:second-countable-out-of-cover}
Let \(\mathcal U\) be a countable open cover of the space \(X\). If every \(U
\in \mathcal U\) is second countable, then \(X\) is second countable.
\end{corollary}

\begin{proof}
We consider \(\mathcal B\) as in \cref{prop:union-cover-basis}, the union of the
basis \(\mathcal B_U\) for the cover elements \(U \in \mathcal U\) --- and since
\(U\) is second countable, we choose \(\mathcal B_U\) to be a countable
cover. Since \(\mathcal U\) is countable and so is every basis contained in the
basis \(\mathcal B\), we conclude that \(\mathcal B\) itself is countable and by
\cref{prop:union-cover-basis} we find that \(\mathcal B\) is a basis for \(X\).
\end{proof}

\begin{definition}[Lindelöf space]
A space \(X\) is said to be Lindelöf if every open cover of \(X\) has a
countable subcover.
\end{definition}

\begin{definition}[Separable space]
A space \(X\) is said to be separable if it contains a countable dense subset.
\end{definition}

\begin{proposition}[Properties of second countable spaces]
\label{prop: second countable properties}
The following properties hold
\begin{enumerate}[(SC1)]
  \item A second countable space is first countable.
  \item A second countable space is separable.
  \item A second countable space is Lindelöf.
\end{enumerate}
\end{proposition}

\begin{proof}
Let \(\mathcal B\) be a countable basis for \(X\).
(SC1) Let a point \(p \in X\), then the set \(\mathcal B_p \subseteq \mathcal
B\) of neighbourhoods of \(p\) is a countable basis for \(X\) at \(p\).

(SC2) Let \(f: I \to \bigcup_{B \in \mathcal B} B\) with \(n \mapsto x_n\),
where \(|I| = |B|\) is an indexing set, and \(x \in X\) be any point, and any
neighbourhood of \(V_x \subseteq X\) of \(x\). Since \(B\) is a basis, there
exists an indexing set \(I_{V_x}\) such that \(V_x = \bigcup_{i \in  I_{V_x}}
B_i\), hence \(V_x \cap A\) is nonempty. Moreover, define a sequence
\((x_i)_{i \in I_{V_x}}\) such that \(x_i \in B_i\). Notice that clearly \(x_i
\to x\), from the fact that \(V_x\) is a neighbourhood of \(x\), and \(x_i \in
A\), from \cref{lem: sequence lemma} we have that \(x \in \overline{\im(f)}\).
Hence we conclude that \(X = \overline{\im(f)}\) and thus \(\im(f)\) is a
countable dense subset of \(X\).

(SC3) Let \(\mathcal U\) be a cover for \(X\) and define \(\mathcal B' \coloneq
\{B \in \mathcal B \colon B \subseteq U (U \in \mathcal U)\}\), and \(\mathcal
U' \coloneq \{U \in \mathcal U \colon B \subseteq U (B \in \mathcal
B')\}\). We'll show that \(\mathcal U'\) is a subcover of \(\mathcal U\). Let
\(x \in X\) be any point.  since \(\mathcal U\) covers \(X\) then there exists
\(U_x \in \mathcal U\) such that \(x \in U_x\). Moreover, since \(\mathcal B\)
is a basis for the topology of \(X\), then there exists \(B_x \in \mathcal B\)
such that \(B_x \subseteq U_x\), so that \(B_x \in \mathcal B'\) and \(U_x \in
\mathcal U'\) from the construction of the collections. Hence \(\mathcal U'\)
covers \(X\).
\end{proof}

\begin{proposition}\label{prop: metric space properties}
Given a metric space \(M\), the following properties are equivalent
\begin{enumerate}[(MS1)]
\item \(M\) is second countable.
\item \(M\) is separable.
\item \(M\) is Lindelöf.
\end{enumerate}
\end{proposition}

\begin{proof}
(Separable \(\Rightarrow\) Lindelöf) Suppose \(M\) is separable and let \(A\)
be a countable dense set in \(M\). Let \(\mathcal U\) be a cover of \(M\) and
define the collection \(\mathcal U' \coloneq \{U \in \mathcal U \colon U
\supseteq B(a, r),\ (a, r) \in A \times \Q\}\), where \(B(a,r)\) is the open
ball of radius \(r\) around \(a\). We show that \(\mathcal U'\) is a subcover of
\(M\).  Let \(x \in M\) be any point. Since \(\mathcal U\) covers \(M\) then
there exists \(U_x \in \mathcal U\) such that \(x \in U_x\). Since \(M\) is a
metric space, there exists an open ball \(B(x, \ell) \subseteq U_x\) for some
\(\ell \in \R\).  Since \(A\) is dense in \(M\), there exists a point \(a \in
A\) such that \(a \in B(x, \ell/2)\). Now, from the fact that \(\Q\) is dense in
\(\R\) we conclude that there exists an \(r \in \Q\) such that \(d(x, a) < r <
\ell/2\). Notice that \(B(a, r) \subseteq B(x, r) \subseteq U_x\), hence \(U_x
\in \mathcal U'\). Therefore \(\mathcal U'\) is a subcover of \(\mathcal U\),
and since \(|\mathcal U'| = |A \times \Q|\), we find that \(\mathcal U'\) is
countable.

(Lindelöf \(\Rightarrow\) second countable) Suppose \(M\) is Lindelöf, then
given a cover \(\mathcal U \coloneq \{B(x, \frac{1}{n}) \subseteq M \colon x \in M, n
\in \N\}\) of \(M\) there exists a countable subcover \(\mathcal U'\). Let \(U
\subseteq M\) be any open set and let \(x \in U\) be any point. Since \(M\) is a
metric space, there exists \(r > 0\) such that \(B(x, r) \subseteq U\). Define
now \(n \in N\) such that \(\frac 1 n < \frac r 2\). Since \(\mathcal U'\) is a
cover of \(M\), there exists a \(B(p, \frac 1 n) \in \mathcal U'\) such that \(x
\in B(p, \frac 1 n)\). We now show that \(B(p, \frac 1 n) \subseteq B(x,
r)\). Let \(y \in B(p, \frac 1 n)\) be any point, then \(d(p, y) < \frac 1 n\),
moreover \(d(x, p) < \frac 1 n\), hence we conclude that \(d(x, y) \leq d(x, p)
+ d(p, y) < \frac 2 n < r\). This implies in particular that \(y \in B(x, r)\)
and in general that \(B(p, \frac 1 n) \subseteq B(x, r) \subseteq U\). Hence we
conclude that \(U\) can be written as a union of elements of \(\mathcal U'\),
which implies that \(\mathcal U'\) is a countable basis for \(M\).

By means of \cref{prop: second countable properties} we conclude the
equivalence chain.
\end{proof}

\begin{corollary}
\label{cor:euclidean-space-second-countable}
Euclidean spaces are countable.
\end{corollary}

\begin{proof}
Lets consider \(\R^n\), notice that \(\Q^n\) is dense in \(\R^n\) and countable,
thus \(\R^n\) is separable, which by \cref{prop: metric space properties}
implies that \(\R^n\) is second countable.
\end{proof}

\subsection{Weights and Cardinality}

\begin{proposition}\label{prop: weight m space subset union}
Let \(X\) be a topological space and \(w(X) \leq \mathbf m\). Then for every
collection \(\{U_i\}_{i \in I} \subseteq 2^X\) of open sets, there exists
\(I_0 \subseteq I\) such that \(|I_0| \leq \mathbf m\) and \(\bigcup_{i \in
I_0} U_i = \bigcup_{i \in  I} U_i\).
\end{proposition}

\begin{proof}
Since \(I_0 \subseteq I\) then clearly \(\bigcup_{i \in  I_0} U_i \subseteq
\bigcup_{i \in  I} U_i\). Let \(\mathcal B\) be a base for \(X\) such that
\(|\mathcal B| \leq \mathbf m\) and define the collection \(\mathcal B_0 \coloneq
\{U \in \mathcal B \colon U \subseteq U_i (i \in I)\}\). Define a function \(f:
\mathcal B_0 \to I\) such that for all \(U \in \mathcal B_0\) we have \(U
\subseteq  U_{f(U)} \in \{U_i\}_{i \in I}\). Define the indexing set \(I_0 \coloneq
f(\mathcal B_0) \subseteq I\). Notice that \(|I_0| \leq |\mathcal B| \leq
\mathbf m\). For any point \(x \in \bigcup_{i \in  I} U_i\) there exists \(i
\in S\) such that \(x \in U_i\), and hence exists \(U \in \mathcal B\) such
that \(x \in U \subseteq U_i\), from the fact that \(\mathcal B\) is a basis.
From the definition of \(\mathcal B_0\) it follows that \(U \in \mathcal B_0\)
and hence \(f(U) \in I_0\). Therefore, from the construction of \(f\) it
follows that \(x \in U \subseteq U_{f(U)} \subseteq \bigcup_{i \in  I_0}
U_i\). From this we conclude that \(\bigcup_{i \in  I} U_i \subseteq
\bigcup_{i \in  I_0} U_i\).
\end{proof}

\begin{proposition}
Let \(X\) be a topological space. If \(w(X) \leq \mathbf m\), then for every base
\(\mathcal B\) for the topology of \(X\) there exists a base \(\mathcal B_0\)
such that \(|\mathcal B_0| \leq \mathbf m\) and \(\mathcal B_0 \subseteq
\mathcal B\).
\end{proposition}

\begin{proof}
Suppose \(\mathbf m \geq \aleph_0\). Let \(\mathcal B \coloneq \{U_i\}_{i \in I}\)
be any base for the space \(X\). Define a base \(\mathcal B_1 \coloneq \{W_t\}_{t
\in T}\) for \(X\) such that \(|T| \leq \mathbf m\).

For all \(t \in T\) define the set \(I(t) \coloneq \{i \in I \colon U_i \subseteq
W_t\}\). From the fact that \(W_t\) is open and \(\mathcal B\) is a base for
\(X\), we have that \(W_t = \bigcup_{i \in  I(t)} U_i\).  From \cref{prop:
weight m space subset union} we find that there exists \(I_0(t) \subseteq
I(t)\) for which \(|I_0(t)| \leq \mathbf m\) and
\begin{equation}\label{eq: W_t I_0(t)}
  W_t = \bigcup_{i \in  I(t)} U_i = \bigcup_{i \in  I_0(t)} U_i
\end{equation}

Now, define the set \(\mathcal B_0 \coloneq \{U_i\}_{i \in I(t), t \in T} \subseteq
\mathcal B\). Since \(|T|, |S_0(t)| \leq \mathbf m\) and the fact that
\(\mathbf m^2 = \mathbf m\), we find that \(|\mathcal B_0| \leq \mathbf m\).
We now show that \(\mathcal B_0\) is a base for the space \(X\). Let a any
point \(x \in X\) and a neighbourhood \(U \subseteq X\) of \(x\). From the
hypothesis that \(\mathcal B_1\) is a basis, there exists \(t \in T\) such
that \(W_t \subseteq U\). On the other hand \cref{eq: W_t I_0(t)} ensures the
existence of an \(i \in I_0(t)\) such that \(U_i \subseteq W_t \subseteq U\).
Since \(U_i \in \mathcal B_0\), we find that \(\mathcal B_0\) is indeed a
basis for \(X\).
\end{proof}

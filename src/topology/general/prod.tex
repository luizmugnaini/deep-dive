\section{Product Space}

\begin{definition}[Product topology]
  \label{def: prod top}
  Let \(\{X_i\}_{i \in I}\) be a collection of sets. The product topology on the
  set \(\prod_{i \in I} X_i\) is defined to be the coarsest topology such that
  for all \(i \in I\) the projection \(\pi_i : \prod_{i \in I} X_i \epi X_i\) is
  continuous.
\end{definition}

\begin{definition}[Second definition of the product topology]
  \label{def: second def prod top}
  Let \(\{X_i\}_{i \in I}\) be a collection of topological spaces. We
  define the product topology on the set \(\prod_{i \in I} X_i\) to be the
  topology generated by the bases
  \[
    \mathcal B = \left\{ \prod_{i \in I} U_i : U_i \subseteq X_i
    \text{ is open, and only finitely many } U_i \neq X_i \right\}.
  \]
\end{definition}

\begin{corollary}\label{cor: equivalent defs prod top}
  The definitions \ref{def: prod top} and \ref{def: second def prod top}
  are equivalent.
\end{corollary}

\begin{proof}
  To prove the equivalece, we first show that if \(\prod_{i \in I} X_i\) is
  endowed with the product topology, then the collection \(\mathcal B\) is a
  basis for \(\prod_{i \in I} X_i\). (\(\Rightarrow\)) Let \(\prod_{i \in I} U_i
  \in \mathcal B\), then for all \(i_j \in I\) we have
  \[
    \pi_{i_j}^{-1}(U_j) = \prod_{i \in I} V_i, \text{ where } V_i =
    \begin{cases}
      X_i, &i \neq i_j \\
      U_j, &\text{otherwise}
    \end{cases}
  \]
  where \(\pi_{i_j}^{-1}(U_j)\) is open from the hypothesis that \(\prod_{i
  \in I} X_i\) is endowed with the product topology. Moreover, we also have
  \(\prod_{i \in I} W_i \cap \prod_{i \in I} W_i' = \prod_{i \in I} W_i \cap
  W_i'\). Notice that clearly
  \[
    \mathcal B = \left\{ \bigcap_{j \in J} \pi_j^{-1}(U_j) : U_j \subseteq X_j
    \text{ is open, and } J \subseteq I \text{ is finite}\right\}
  \]
  and from \cref{prop: top generated by collection of maps} this shows that
  \(\mathcal B\) is a basis for the product topology. (\(\Leftarrow\)) The
  converse is clear.
\end{proof}

\begin{theorem}[Product topology universal product]
  \label{thm: prod top universal prop}
  Let \(\{X_i\}_{i \in I}\) be a collection of topological spaces and \(\pi_i:
  \prod_{i \in I} X_i \epi  X_i\) be the projection map. For all given
  topological space \(Z\) and a map \(f: Z \to \prod_{i \in I} X_i\). If
  \(\prod_{i \in I} X_i\) is endowed with the product topology, then the map
  \(f\) is continuous if and only if for every \(i \in I\) the map \(\pi_i
  f: Z \to \prod_{i \in I} X_i\). That is, the following diagram commutes in
  \(\Top\)
  \[
    \begin{tikzcd}
      Z \ar[d, "f"] \ar[dr, bend left, "\pi_i  f"] \\
      \prod_{i \in I} X_i \ar[r, two heads, "\pi_i"] &X_i
    \end{tikzcd}
  \]
  Moreover, if the space \(\prod_{i \in I} X_i\) satisfies such universal
  property, then its topology is the product topology.
\end{theorem}

\begin{proof}
  Let \(\prod_{i \in I} X_i\) be endowed with the product topology.
  (\(\Rightarrow\)) Let \(f\) be a continuous map and \(U \subseteq Z\) be any
  open set. Since \(\pi_i\) is continuous for all \(i \in I\) from hypothesis
  then \(\pi_i^{-1}(U) = V \subseteq \prod_{i \in I} X_i\) is open, we conclude
  that \((\pi_i  f)^{-1}(U) = f^{-1}(\pi_i^{-1}(U)) = f^{-1}(V)\) is open,
  hence \(\pi_i  f\) is continuous. (\(\Leftarrow\)) Suppose \(\pi_i
  f\) is continuous, then for all given open set \(U \subseteq \prod_{i \in I}
  X_i\) we have \(f^{-1}(\pi_i^{-1}(U)) \subseteq Z\) open. Notice that since
  \(\pi_i^{-1}(U)\) is open in the product topology of \(\prod_{i \in I} X_i\)
  for all \(i \in I\), then \(f^{-1}(V) \subseteq Z\) is open for all open set
  \(V \subseteq \prod_{i \in I} X_i\).

  For the second part of the theorem, suppose that \((\prod_{i \in I} X_i,
  \mathcal T')\) be a space satisfying the property. In particular, consider a
  space \(Z = (\prod_{i \in I} X_i, \mathcal T')\) and a map \(f = \Id\). Then
  the following diagram commutes
  \[
    \begin{tikzcd}
      (\prod_{i \in I} X_i, \mathcal T')
      \ar[d, "\Id"]
      \ar[dr, bend left, "\pi_i  \Id = \pi_i"] \\
      (\prod_{i \in I} X_i, \mathcal T')
      \ar[r, two heads, "\pi_i"]
        &X_i
    \end{tikzcd}
  \]
  We can now assert that \(\Id\) is continuous since both domain and codomain
  have the same topology \(\mathcal T'\), hence \(\pi_i\) is continuous for all
  \(i \in I\). Since \(\pi_i\) is continuous for all \(i \in I\) for the
  topology \(\mathcal T'\) then we can use the \cref{def: prod top} to conclude
  that \(\mathcal T' \subseteq \mathcal T\), where \(\mathcal T\) is the product
  topology. For the second inclusion, consider the commutative diagram
  \[
    \begin{tikzcd}
      (\prod_{i \in I} X_i, \mathcal T')
      \ar[d, "\Id'"]
      \ar[dr, bend left, "\pi_i  \Id' = \pi_i'"] \\
      (\prod_{i \in I} X_i, \mathcal T)
      \ar[r, two heads, "\pi_i"]
        &X_i
    \end{tikzcd}
  \]
  where \(\pi_i': (\prod_{i \in I} X_i, \mathcal T') \epi X_i\). We know from
  the previous discussion that since \((\prod_{i \in I} X_i, \mathcal T')\)
  satisfies the universal property, then \(\pi_i'\) is continuous and hence
  \(\Id'\) is continuous. In particular, this implies that if  \(U \subseteq
  (\prod_{i \in I} X_i, \mathcal T)\) is open, then \(\Id'^{-1}(U) = U
  \subseteq (\prod_{i \in I} X_i, \mathcal T')\) is open. This implies in
  \(\mathcal T \subseteq \mathcal T'\). Hence we conclude that if an object
  satisfies the product topology universal property, then it is endowed with the
  product topology.
\end{proof}

\begin{lemma}
  Let \(\{X_i\}_{i \in I} \) be a collection of topological spaces. Then the
  projections
  \[
    \pi_j: \prod_{i \in I} X_i \epi X_j
  \]
  are open maps (see \cref{def: open/closed maps}) under the product
  topology. Moreover, such projections are not in general closed maps.
\end{lemma}

\begin{proof}
  Consider the basis \(\mathcal B\) from \cref{def: second def prod top}, then
  let \(U_i \subseteq X_i\) be an open set. Then notice that (see the proof of
  \cref{cor: equivalent defs prod top})
  \[
    \pi_j(\pi_i^{-1}(U_i)) =
    \begin{cases}
      U_i, &i = j \\
      X_j, &i \neq j
    \end{cases}
  \]
  which are both open sets. Moreover, we have that
  \[
    \pi_j \left(\pi_{i_1}^{-1}(U_{i_1}) \cap \dots \cap \pi_{i_k}^{-1}(U_{i_k})
    \right) =
    \begin{cases}
      X_j,        & j \neq i \\
      U_{i_\ell}, & j = i_\ell \text{ for some } 1 \leq \ell \leq k
    \end{cases}
  \]
  is open. In particular, this implies in
  \[
    \pi_j \left( \bigcup_{i_1, \dots, i_k \in I} \pi^{-1}(U_{i_1}) \cap \dots
    \cap \pi_{i_k}^{-1}(U_{i_k}) \right)
    =
    \bigcup_{i_1, \dots, i_k \in I} \pi_j \left( \pi^{-1}(U_{i_1}) \cap \dots
    \cap \pi_{i_k}^{-1}(U_{i_k}) \right)
  \]
  which from the last assertion is the union of open sets, hence open. This
  implies that \(\pi_j\) is open (from the fact that \(\mathcal B\) is a basis
  for \(\prod_{i \in I} X_i\)).

  On the other hand, we can build a counterexample to show why the projections
  are not necessarily closed. Consider the product \(\R^2\) and let the
  closed set \(C := \{(x, y) \in \R^2 : x y = 1\}\), which is the
  hyperbola on the plane. Notice that the first projection \((x, y)
  \xmapsto{\pi_1} x\) is not closed since \(\pi_1(C) = \{x \in \R : x
  \neq 0\}\).
\end{proof}

\begin{definition}[Box topology]
  We define yet another natural topology on the set \(\prod_{i \in I} X_i\): the
  box topology, which is generated by the basis
  \[
    \mathcal B_\text{box} = \left\{\prod_{i \in I} U_i : U_i \subseteq X_i
    \text{ is open}\right\}.
  \]
  For the case where \(I\) is an infinite indexing set, we find that \(\mathcal
  T_\text{box} \supsetneq \mathcal T_\text{prod}\) and from this we find that
  certainly the projections are still continuous on \(\mathcal T_\text{box}\),
  although the box topology fails to satisfy the universal product \cref{thm:
  prod top universal prop}.
\end{definition}

\section{Coproduct Space}

\begin{definition}[First definition]
  Let \(\{X_i\}_{i \in I}\) be any collection of topological spaces and consider
  the disjoint union \(\bigatt_{i \in I} X_i = \bigcup_{i \in  I} X_i \times
  \{i\}\). We define the coproduct topology on \(\bigatt_{i \in I} X_i\) via the
  following property. A set \(U \subseteq \bigatt_{i \in I} X_i\) is open if and
  only if it is of the form \(U = \bigatt_{i \in I} U_i\), where each \(U_i
  \subseteq X_i\) is open.
\end{definition}

\begin{definition}[Coproduct topology]\label{def: coproduct top}
  Let \(\{X_i\}_{i \in I}\) be a collection of topological spaces. The coproduct
  topology on the set \(\bigatt_{i \in I} X_i\) is defined to be the finest
  topology such that for all \(j \in I\) we have that the inclusions \(\iota_j:
  X_j \mono \bigatt_{i \in I} X_i\) are continuous.
\end{definition}

\begin{theorem}[Coproduct topology universal property]
  \label{thm: coprod top universal property}
  Let \(\{X_i\}_{i \in I}\) be a collection of topological spaces and let
  \(\iota_j: X_j \to \bigatt_{i \in I} X_i\) be the \(j\)th inclusion. The
  coproduct topology on \(\bigatt_{i \in I} X_i\) satisfies the following
  property. Let \(Z\) be a topological space and a collection of continuous maps
  \(\{f_i: X_i \to Z\}_{i \in I}\). Then there exists a unique map \(f:
  \bigatt_{i \in I} X_i \to Z\) such that the following diagram commutes in the
  category \(\Top\) for all \(j \in I\):
  \[
    \begin{tikzcd}
      X_j \ar[r, "\iota_j"]  \ar[rd, swap, bend right, "f_j"]
      & \bigatt_{i \in I} X_i \ar[d, dashed, "f"] \\ &Z
    \end{tikzcd}
  \]
  On the other hand, if \((\bigatt_{i \in I} X_i, \mathcal T')\) satisfies such
  property, then \(\mathcal T'\) is the coproduct topology.
\end{theorem}

\begin{proof}
  (Uniqueness) Suppose that \(f  \iota_j = f_j\), then from definition
  \(f_j(x) = f(\iota(x)) = f(x, j)\) for each \(j \in I\), which is clearly
  unique.
  (Existence) Suppose now that \(f_j\) is continuous for all \(j \in I\). Let
  \(U \subseteq Z\) be any open set. Notice that since \(f_j^{-1} = \iota_j^{-1}
   f^{-1}\) then \(f_j^{-1}(U) = \iota_j^{-1}(f^{-1}(U))\). Now, if
  \(f^{-1}(U) \subseteq \bigatt_{i \in I} X_i\) is closed then its preimage
  under \(\iota_j\) would be closed (from the continuity of \(\iota_j\)), hence
  \(f^{-1}(U)\) is open, which implies in \(f\) continuous.

  Suppose that \((\bigatt_{i \in I} X_i, \mathcal T')\) satisfies the universal
  property and denote by \(\mathcal T\) the coproduct topology. Then in
  particular we have
  \[
    \begin{tikzcd}
      X_j \ar[r, "\iota_j"] \ar[rd, bend right, swap, "\iota_j"]
        &\left( \bigatt_{i \in I} X_i, \mathcal T' \right)
        \ar[d, dashed, "g"] \\
        &(\bigatt_{i \in I} X_i, \mathcal T)
    \end{tikzcd}
    \qquad \qquad
    \begin{tikzcd}
      X_j \ar[r, "\iota_j"] \ar[rd, bend right, swap, "\iota_j"]
        &\left( \bigatt_{i \in I} X_i, \mathcal T \right)
        \ar[d, dashed, "f"] \\
        &(\bigatt_{i \in I} X_i, \mathcal T')
    \end{tikzcd}
  \]
  and hence \(f\) and \(g\) are both identities on \(\End(\bigatt_{i \in I} X_i)\). If
  \(U \subseteq (\bigatt_{i \in I} X_i, \mathcal T)\) is open, then \(g^{-1}(U) = U \subseteq (\bigatt_{i \in I}
  X_i, \mathcal T')\) is also open. On the other hand, from the second diagram,
  if \(V \subseteq (\bigatt_{i \in I} X_i, \mathcal T')\) is open, then \(f^{-1}(V) = V \subseteq (\bigatt_{i \in
  I} X_i, \mathcal T)\) is open. This concludes that \(\mathcal T' = \mathcal T\)
  and hence the coproduct topology is unique.
\end{proof}

\begin{proposition}
  Let \(\{X_i\}_{i \in I}\) be a collection of topological spaces. The following
  are properties of the coproduct space.
  \begin{enumerate}[(a)]
    \item A subset \(C \subseteq \bigatt_{i \in I} X_i\) is closed if and only
      if for all \(X_i\), we have that \(C \cap X_i\) is closed.
    \item The canonical injection \(\iota_j: X_j \to \bigatt_{i \in I} X_i\) is
      a topological embedding and an open and closed map.
    \item If \(X_i\) is Hausdorff for all \(i \in I\), then \(\bigatt_{i \in
      I} X_i\) is Hausdorff.
    \item If \(X_i\) is first countable for all \(i \in I\), then \(\bigatt_{i
      \in I} X_i\) is first countable.
    \item If \(X_i\) is second countable for all \(i \in I\) and \(I\) is a
      countable indexing set, then \(\bigatt_{i \in I} X_i\) is second
      countable.
  \end{enumerate}
\end{proposition}

\todo[inline]{Properties for the coproduct top}

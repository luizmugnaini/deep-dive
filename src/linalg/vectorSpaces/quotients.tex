\section{Quotients and Subspaces}

\begin{definition}[Subspace]\label{def: subspace}
   Let \(V\) be \(k\)-vector space. We say that \(W \subseteq  V\) is a subspace
   of \(V\) if
   \begin{enumerate}[I.]
     \item \(0 \in W\).
     \item \(v, u \in W\) implies \(v + u \in W\).
     \item \(\forall c \in k\) and \(\forall v \in W\), then \(cv \in W\).
   \end{enumerate}
\end{definition}

\begin{definition}[Quotient]\label{def: quotient}
   Let \(V\) be a \(k\)-vector space and \(W \subseteq V\) a subspace. The
   quotient of \(V\) by \(W\) is defined as \(V/W = V/{\sim}\) where \(v
   \sim u \Leftrightarrow v - u \in W\).
\end{definition}

\begin{theorem}[Universal property for quotients]
  \label{thm: universal property for quotients}
  Let \(V\) be a \(k\)-vector space and \(W \subseteq V\) be a subspace. Let
  also \(L\) be any \(k\)-vector space and a \(k\)-linear morphism \(f : V \to L
  \) such that \(\forall w \in W, f(w) = 0\), that is \(W \subseteq \ker f\).
  There exists a unique \(k\)-linear morphism \(\ell: V/W \to L\) such that the
  following diagram commutes
  \[
    \begin{tikzcd}
       &V \ar["f", r] \ar["\pi", twoheadrightarrow, d] &L \\
       &V/W \ar[swap, dashed, "\ell", ru]
    \end{tikzcd}
  \]
  Moreover, \(\ker \ell = \ker f / W\) and  \(\im(\ell) = \im(f)\).
\end{theorem}

\begin{proof}
  Since we need \(\ell  \pi = f\) then \(\ell  \pi : v \xmapsto \pi
  [v] \xmapsto \ell f(v)\). We first show that it is indeed well
  defined. For that, suppose \(v \sim u\), that is \([v] =
  [u]\), then we know that since they belong to the same class, they must have the
  same image under the mapping \(\ell\). Notice that since \(v \sim u
  \Leftrightarrow v - u \in W \subseteq \ker f\) then \(f(v - u) = f(v) - f(u) =
  0\) and therefore indeed \(f(v) = f(u)\) whenever \([v] =
  [u]\). The uniqueness of \(\ell\) comes from the fact that its image depends
  strictly on the unique image of \(f\).

  Now we show the last two propositions. For the first one, notice that \(\ker
  \ell = \{[v] \in V/W : v \in \ker f\} = \ker f / W\). For the last
  statement of the theorem we have that since \(\ell  \pi = f\) then
  \(\im(f) \subseteq \im(\ell)\) and also \(\pi\) is surjective, thus
  \(\im(\ell) \subseteq \im(f)\), which proves what is proposed at last.
\end{proof}

\begin{theorem}[First Isomorphism]\label{thm: first isomorphism}
  Let \(f : V \to W\) be a \(k\)-linear morphism, then the injective
  \(k\)-linear morphism \(\ell : V / \ker f \hookrightarrow W\) is such that
  \[
     V / \ker f \iso \im(f).
  \]
  is an isomorphism of \(k\)-vector spaces.
\end{theorem}

\begin{proof}
   This is just a special case of the universal property.
\end{proof}

\begin{theorem}[Second Isomorphism]\label{thm: second isomorphism}
   Let \(V\) be a \(k\)-vector space and \(U, W \subseteq V\) be subspaces. Then
   there is a natural isomorphism
   \[
     (W + U)/U \iso W/(W \cap U).
   \]
\end{theorem}

\begin{proof}
  Notice that this isomorphism needs to be the mapping
  \[
    (W + U)/U \ni [w + u] \longmapsto [w] \in W/(W \cap U)
  \]
  Notice that we have the mappings
  \begin{gather*}
    [w + u] + [w' + u'] = [(w + w')+(u+u')] \longmapsto [w + w'] = [w] + [w']
    \\
    c[w + u] = [cw + cu] \longmapsto [cw] = c[w]
  \end{gather*}
  which shows the conditions for the mapping to be a \(k\)-linear morphism. Now
  notice that the mapping is clearly surjective, since given any class \([w] \in
  W/(W\cap U)\) we have the element \([w + u] \in (W + U)/U\), where \(u\) is
  any element of \(U\) and moreover \([w + u] \mapsto [w]\) from construction,
  showing the surjective character of the mapping. Moreover, notice that \([w]
  \in W/(W\cap U) \Leftrightarrow w \in U\), thus the kernel of the mapping is
  equal to \(U\). Now by means of the first isomorphism theorem shows that the
  mapping is also injective.
\end{proof}

\begin{theorem}[Third Isomorphism]\label{thm: third isomorphism}
  Let \(V\) be a \(k\)-vector space and let the subspaces \(U \subseteq W
  \subseteq V\). Then there exists an embedding \(W/U \hookrightarrow V/U\) with
  a mapping \([w] \mapsto [w]\) such that \(W/U\) can be regarded as a subspace
  of \(V/U\). Moreover, we have a natural isomorphism
  \[
    V/W \iso (V/U) \big / (W/U).
  \]
\end{theorem}

\begin{proof}
  Consider the morphism \(\varphi : V/U \to V/W\) with the mapping \([v]
  \xmapsto \varphi [v]\). Since \(U \subseteq W\) then we have that \([v] = [v']
  \in V/U\) implies \([v] = [v'] \in V/W\), which shows that \(\varphi\) is well
  defined.  Moreover, given a class \([v] \in V/W\) we have the corresponding
  class \([v] \in V/U\) such that \([v] \xmapsto \varphi [v]\), thus
  \(\varphi\) is surjective.

  Moreover, notice that if \([w] \in W/U\) then from the embedding we have the
  corresponding class \([w] \in V/U\), we'll have two possibilities, if \([w] =
  [0]\), then the \(k\)-linear morphism \(\varphi\) will map it to \([0]\), on
  the other hand, if \([w] \neq [0]\) then \(w \in W\) implies that \([w]
  \xmapsto \varphi [w] = [0] \in V/W\) so that in both cases we have \([w] \in
  W/U \Rightarrow [w] \in \ker \varphi\) so that \(W/U \subseteq \ker \varphi\).
  Moreover, if \(V/U \ni [v] \in \ker \varphi\), then \([v] = [0] \in V/W\)
  which means that \(v \in W\) and therefore we have the corresponding class
  \([v] \in W/U\), which sums up to \(\ker \varphi \subseteq W/U\) and thus
  \(\ker \varphi = W/U\).  From the first isomorphism theorem we have that
  \[
    (V/U) \big / (W/U) \iso V/W
  \]
  since the image of \(\varphi\) is equal to \(V/W\), settling the proof.
\end{proof}

\section{Jacobi Sum}

\begin{definition}[Jacobi sum]
   Let \(\chi\) and \(\lambda\) be characters of \(\F_p\) and set \[J(\chi,
   \lambda) := \sum_{a + b = 1} \chi(a)\lambda(b)\] this we call the Jacobi
   sum.
\end{definition}

\begin{theorem}
   Let \(\chi, \lambda\) be non-trivial (\(\varepsilon\)) characters. Then
   \begin{enumerate}[I.]
      \item \(J(\varepsilon, \varepsilon) = p\).
      \item \(J(\varepsilon, \chi) = 0\).
      \item \(J(\chi, \chi^{-1}) = - \chi(-1)\).
      \item If  \(\chi\lambda \neq \varepsilon\) then
         \[
            J(\chi, \lambda) = \frac{g(\chi)g(\lambda)}{g(\chi\lambda)}.
         \] 
   \end{enumerate}
\end{theorem}

\begin{proof}
   We prove the properties above:
   \begin{enumerate}[I.]
      \item \(J(\varepsilon, \varepsilon) = \sum_{a + b = 1} \varepsilon(a)
         \varepsilon(b) = \sum_{a+b=1} 1 = p\) since \(a, b \in \F_p\).
      \item \(J(\varepsilon, \chi) = \sum_{a+b = 1} \chi(b) = \sum_{b \in \F_p}
         \chi(b) = 0\).
      \item \(J(\chi, \chi^{-1}) = \sum_{a+b=1} \chi(a) \chi^{-1}(b) = \sum_{a +
         b = 1} \chi(a)\chi(b^{-1}) = \sum_{\substack{a+b=1\\ b \neq 0}}
         \chi(a/b) = \sum_{a \neq 1} \chi\big(\frac{a}{1-a}\big)\) now if we
         just define \(c := \frac{a}{1-a} \Rightarrow a = \frac{c}{1 + c}\) we
         can see that \(a\) varies on \(\F_p\) but does not pass in \(-1\) and
         \(c\) does the same but does not pass in \(1\), thus \(J(\chi,
         \chi^{-1}) = \sum_{c \neq -1} \chi(c) = \sum_{c \in \F_p} \chi(c) -
         \chi(-1) = - \chi(-1)\).
      \item For this item we start from the other side of the equation, notice
         that 
         \begin{align*}
            g(\chi)g(\lambda) 
            = \sum_{x \in \F_p} \chi(x)\zeta^{x}\sum_{y \in \F_p}
            \lambda(y)\zeta^{y} 
            = \sum_{x, y \in \F_p} \chi(x)\lambda(y)\zeta^{x+y}
            = \sum_{t \in \F_p} \left( \sum_{x+y=t} \chi(x)\lambda(y) \right)
            \zeta^t
         \end{align*}
         For the case where \(t=0\) we have simply \(\sum_{x \in \F_p}
         \chi(x)\lambda(-x) = \sum_{x \in \F_p} \chi(x) \lambda(-1)\lambda(x) =
         \lambda(-1)\sum_{x \in \F_P} \chi\lambda(x) = 0\) and since
         \(\chi\lambda \neq  \varepsilon\) then this sum evaluates to \(0\).
         Now, for the case \(t \neq  0\) we can define new variables \(x' :=
         x/t\) and  \(y' := y/t\) so that  \(x + y = t \Leftrightarrow x' + y' =
         1\) and thus we can use the Jacobi sum explicitly:
         \(\sum_{x+y=t}\chi(x)\lambda(y) = \sum_{x'+y'=1} \chi(tx')\lambda(ty')
         = \chi\lambda(t) \sum_{x' + y' = 1} \chi(x)\lambda(y) = \chi\lambda(t)
         J(\chi, \lambda)\) and therefore we conclude that, substituting in the
         above equation:
          \[
             g(\chi)g(\lambda) 
             = \sum_{t \in \F_p^\ast} \chi\lambda(t) J(\chi, \lambda)\zeta^t
             = J(\chi, \lambda) g(\chi\lambda)
         \] 
         where we conclude the proof.
   \end{enumerate}
\end{proof}

\begin{proposition}
   Let \(\chi,\lambda\) be such that  \(\chi\lambda \neq  \varepsilon\), then
   \(|J(\chi,\lambda)| = \sqrt{p}\).
\end{proposition}

\begin{proposition}
   If \(p \equiv 1 \pmod{4}\) then there exists \(a, b \in \mathbb{Z}\) such
   that \(p = a^2 + b^2\). If  \(p \equiv 1 \pmod{3}\) then there are  \(a, b
   \in \mathbb{Z}\) such that \(p = a^2 - ab + b^2\).
\end{proposition}

\begin{proof}
   For \(p \equiv 1 \pmod{4}\), let \(\lambda \) be the generator of the group
   of characters of \(\F_p\), that is, the order of \(\lambda\) is \(p-1\), then
   let \(\chi := \lambda^{\frac{p-1}{4}}\) to have order \(4\), the image of
   \(\chi\) is a subset of \(\{-1, 1, i, -i\}\) (all the \(4\)th roots of
   unity). From this we conclude that \(J(\chi, \chi) = \sum_{s + t=1} \chi(st)
   \in \mathbb{Z}[i]\) and therefore \(J(\chi, \chi) = a + bi\) for some  \(a,
   b \in \mathbb{Z}\). Since \(|J(\chi,\chi)| = \sqrt{p}\) then \(p = |J(\chi,
   \chi)|^2 = a^2 + b^2\), proving the statement. 

   For the case where \(p \equiv 1 \pmod{3}\) we let  \(\lambda\) be the
   generator of the group of characters and then we let \(\chi := \lambda^{
   \frac{p-1}{3}}\) which implies in order \(3\). Then the image of  \(\chi \)
   is a subset of the \(3\)rd roots of unity \(\{1, \omega, \omega^2\}\) where
   we have \(\omega:= e^{\frac{2\pi i }{3}}\). Then \(J(\chi, \chi) = \sum_{a+b
   = 1} \chi(ab) \in \mathbb{Z}[\omega]\) and therefore exists \(x, y \in
   \mathbb{Z}\) such that \(J(\chi, \chi) =x + y\omega\). Now, from the same
   argument as before we have  \(p = |J(\chi, \chi)| = N(a + b\omega) = a^2 - ab
   + b^2\) where \(N\) is the norm function on \(\mathbb{Z}[\omega]\).
\end{proof}

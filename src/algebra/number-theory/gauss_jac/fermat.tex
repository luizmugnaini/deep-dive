\section{\texorpdfstring{\(x^n + y^n = 1\)}{xn + yn = 1} in
  \texorpdfstring{\(\F_p\) }{F_p}}

\begin{remark}
    We assume \(p \equiv 1 \pmod{n}\).
\end{remark}

We now analyse the number of solutions of the equation \(x^n + y^n = 1\) in the
field \(\F_p\). Notice that  \(N(x^n + y^n = 1) = \sum_{a + b = 1} N(x^n =
a)N(y^n = b)\). Now, if \(\chi\) is a character on \(\F_p\) and has order
\(n\) (it does exist because if \(\lambda\) is the generator of the group, we
may simply define \(\chi := \lambda^{\frac{p-1}{n}}\) since we are supposing
\(p\equiv 1 \pmod{n}\)) then we have that
\[
    N(x^n = a) = \sum_{\chi^n = \varepsilon} \chi(a) = \sum_{i = 0}^{n-1}
    \chi^i(a)
\]
and therefore we end up with
\[
    N(x^n + y^n = 1) = \sum_{a+b=1} \bigg(\sum_{i = 0}^{n-1} \chi^i(a) \sum_{j =
        0}^{n-1} \chi^j(b)\bigg)
    = \sum_{0 \leqslant i, j \leqslant n-1 } J(\chi^i, \chi^j).
\]
We now analyse what we got:
\begin{enumerate}[i.]
    \item For the case \(i = j = 0\) we simply have \(J(\varepsilon, \varepsilon)
          = p\).
    \item For \(i + j = n\) then  \(\chi^i = (\chi^j)^{-1}\) since \(\chi^i
          \chi^j = \chi^n = \varepsilon\) and therefore
          \[
              J(\chi^j, \chi^i) = J(\chi^j, (\chi^j)^{-1}) = - \chi^j(-1) \Rightarrow
              \sum_{i+j = n} J(\chi^i, \chi^j) = - \sum_{j = 1}^{n-1} \chi^j(-1).
          \]
          Notice that (warning: look at the first index, it starts with \(0\) and
          not a \(1\) as above, we'll deal with it later):
          \[
              N(x^n = -1) = \sum_{j=0}^{n-1} \chi^j(-1) =
              \begin{cases}
                  n, & \text{if } x^n = -1  \text{ has solution}, \\
                  0, & \text{otherwise}.
              \end{cases}
          \]
          Therefore we have that
          \[
              J(\chi^j, \chi^i) = -\sum_{j=1}^{n-1} \chi^j(-1)
              = - \left( \sum_{j=0}^{n-1} \chi^j(-1) - \chi^0(-1) \right)
              = 1 - \delta_n(-1)n
          \]
          where \(\delta_n(-1) := 1\) if  \(x^n = -1\) has solution and
          \(\delta_n(-1) := 0\) if the equation has no solution.
    \item If \(i = 0\) and \(j \neq 0\) or the exact opposite, then we have
          \(J(\chi^i, \chi^j) = J(\varepsilon, \chi^j) = 0\).
\end{enumerate}

Thus we conclude finally that
\[
    N(x^n + y^n = 1) = p+1-\delta_n(-1)n
    + \sum_{\substack{1 \leqslant i, j \leqslant n-1 \\ i + j \neq  n}} J(\chi^i,
    \chi^j)
\]
where the last sum has \((n-1)^2 - (n-1)  = (n-1)(n-2)\) terms. Since all of
these terms have norm \(|J(\chi^i, \chi^j)| \leqslant \sqrt{p}\) then we
conclude that
\[
    |N(x^n + y^n = 1) + \delta_n(-1) -p -1| \leqslant (n-1)(n-2) |J(\chi^i,
    \chi^j)| = (n-1)(n-2)\sqrt{p}.
\]
Where we interpret \(\delta_n(-1)\) as the number of points at infinity of the
curve.

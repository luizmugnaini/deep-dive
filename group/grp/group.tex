\section{Group definitions}

Lets recall \cref{def: group} and demystify it in the following definition.

\begin{definition}[Group]
  A group \(G\) is a groupoid \(\cat G\) with one object \(*\). The elements of
  the group are the morphisms \(\Aut_{\cat G}(*)\). From axioms of \cref{def:
  category}:
  \begin{enumerate}[(G1)]
    \item An identity element \(e = \Id_*\).
    \item An associative binary operation \(G \times G \to G\), commonly denoted
      by juxtaposition.
    \item For all \(g \in G\) there exists an inverse element \(g^{-1} \in G\)
      such that \(g g^{-1} = e = g^{-1} g\).
  \end{enumerate}
\end{definition}

\begin{proposition}
  The identity element of a group is unique.
\end{proposition}

\begin{proof}
  See \cref{cor: unique identity}.
\end{proof}

\begin{proposition}
  The inverse of an element of a group is unique.
\end{proposition}

\begin{proof}
  See \cref{prop: iso unique inverse}.
\end{proof}

\begin{proposition}[Cancellation]
  Let \(G\) be a group and elements \(a, b, c \in G\). If \(a c = b c\) then
  \(a = b\).
\end{proposition}

\begin{proof}
  Notice that \(a = a e = (a c) c^{-1} = (b c) c^{-1} = b e = b\).
\end{proof}

\begin{definition}[Commutatie group]
  A group \(G\) is said to be commutative (or abelian) if for all \(g, h \in G\)
  we have \(g h = h g\).
\end{definition}

\begin{corollary}
  Let \(G\) be a group such that for all \(g \in G\), \(g^2 = e\). Then \(G\) is
  a commutative group.
\end{corollary}

\begin{proof}
  Let \(g, h \in G\) be any elements, then \((g h) (h g) = g h^2 g = g e g = g^2
  = e = (g h) (g h)\) and from cancellation law we find that \(g h = h g\).
\end{proof}

\subsection{Orders}

\begin{definition}[Order of an element]\label{def: group elem order}
  Let \(G\) be a group and \(g \in G\) be any element. We say that \(g\) has
  finite order if there exists \(n \in \mathbb{Z}_{> 0}\) such that \(g^n = e\). 
  The order \(|g|\) of the element \(g\) is defined as the smallest such
  positive integer. If \(g\) does not have a finite order, it is common to write
   \(|g| = \infty\).
\end{definition}

\begin{lemma}\label{lem: order and multiples}
  Let \(G\) be a group and \(g \in G\) be an element with finite order. Then
  \(g^n = e\) for some \(n \in \mathbb{Z}_{> 0}\) if and only if \(|g|\) divides
  \(n\).
\end{lemma}

\begin{proof}
  Since \(|g| \leq n\), define \(m \in \mathbb{Z}_{> 0}\) such that \(n - m |g|
  \geq 0\) and \(n - (m + 1) |g| < 0\). Define \(r = n - m |g|\) to be the
  remainder, hence \(r < |g|\). Our goal is to show that \(r = 0\). Notice that
  \(g^r = g^n g^{-|g| m} = e e^{-m} = e\), which can only be the case for \(r =
  0\), since \(|g|\) is defined to be the least positive integer such that
  \(g^{|g|} = e\). This proves that \(m |g| = n\).

  For the second part, suppose \(n\) is a multiple of \(|g|\) and denote it by
  \(m|g| = n\). Then \(g^{m |g|} = e^m = e\).
\end{proof}

\begin{definition}[Order of a group]
  Let \(G\) group of finite number of elements. We define the order of \(G\) to
  be the number of its elements and denote it by \(|G|\). If \(G\) is an
  infinite group, then \(|G| = \infty\).
\end{definition}

\begin{proposition}
  Let \(G\) be a group and \(g \in G\) be an element with finite order. Then for
  all \(m \in \mathbb{Z}_{\geq 0}\) the element \(g^m\) has finite order. For
  \(m = 0\) we have \(|g^m| = 1\), for \(m > 0\) we have
  \[
    |g^m| = \frac{\operatorname{lcm}(m, |g|)}{m} =
    \frac{|g|}{\operatorname{gcd}(m, |g|)}.
  \] 
\end{proposition}

\begin{proof}
  From divisibility arguments, we have \(\operatorname{lcm}(a, b)
  \operatorname{gcd}(a, b) = ab\) for integers \(a\) and \(b\), hence the second
  equality is justified. We prove the equality \(|g^m| =
  \frac{\operatorname{lcm}(m, |g|)} m\). For the sake of notation, let \(d :=
  |g^m|\). Notice that \(g^{m d} = e\) and hence \(|g|\) divides \(m d\). Since
  \(d\) is the least positive integer with such property, it follows that \(m
  d\) is the least common multiple of \(m\) and \(|g|\). Therefore \(m |g^m| =
  \operatorname{lcm}(m, |g^m|)\), which proves the equation.
\end{proof}

\begin{proposition}
  Let \(G\) be a group and \(g, h \in G\) be such that \(g h = h g\). Then \(|g
  h|\) divides \(\operatorname{lcm}(|g|, |h|)\).
\end{proposition}

\begin{proof}
  Let \(n\) be a common multiple of \(|g|\) and \(|h|\), then \(g^n = h^n = e\)
  from \cref{lem: order and multiples}. Notice that the commutative property \(g
  h = h g\) allow us to permute the terms of \(g^n h^n = e\) in order to obtain
  \(g^n h^n = (g h)^n = e\). In particular, since \(\operatorname{lcm}(|g|,
  |h|)\) is a common multiple of \(|g|\) and \(|h|\), we find that \((g
  h)^{\operatorname{lcm}(|g|, |h|)} = e\) and again from \cref{lem: order and
  multiples} we find that \(|g h|\) divides \(\operatorname{lcm}(|g|, |h|)\)
\end{proof}

\section{Bundle}

\begin{definition}[Bundle]
\label{def:bundle}
A \emph{bundle} is defined to be a triple \((E, p, B)\), where \(E\) and \(B\)
are \emph{spaces} and \(p: E \to B\) is a \emph{morphism}. We refer to \(B\) as
the \emph{base space}, while \(E\) is the \emph{total space}, and \(p\) is the
\emph{projection} of the bundle. As usual, given any \(b \in B\), we name the
object \(p^{-1} b\) the \emph{fibre} of the bundle over \(b\).

A \emph{subbundle} of \((E, p, B)\) is a bundle \((E', p', B')\) such that
\(E'\) and \(B'\) are \emph{subspaces} of \(E\) and \(B\), respectively, and
\(p' = p|_{E'}: E' \to B'\).
\end{definition}

\begin{definition}[Product bundle]
\label{def:product-bundle}
A \emph{product bundle over \(B\) with fibre \(F\)} is a triple
\((B \times F, p, B)\) where \(p(x, y) \coloneq x\) is the first projection.
\end{definition}

\begin{definition}[Cross section]
\label{def:cross-section-bundle}
Given a bundle \((E, p, B)\), a \emph{cross section} of the bundle is a
\emph{section} \(s: B \to E\) of \(p\)---that is, \(p s = \Id_B\). As
an immediate consequence of this definition, if \((E', p', B')\) is a subbundle
of \((E, p, B)\), then \(s\) is a cross section of \((E', p', B')\) if and only
if \(s B \subseteq E'\).
\end{definition}

\begin{lemma}[Product bundle cross section]
\label{lem:cross-section-of-product-bundle}
Given a product bundle \((B \times F, p, B)\), a cross section
\(s: B \to B \times F\) will always have the form \(s = \Id_B \times f\), where
\(f: B \to F\) is a uniquely defined morphism. Therefore the collection of cross
sections of product bundles is in bijection with the collection of maps
\(B \to F\).
\end{lemma}

\begin{proof}
Let \(s\) be any cross section of \((B \times F, p, B)\), from the definition of
\(s\), there exists unique morphisms \(s': B \to B\) and \(f: F \to B\) such
that \(s = s' \times f\)---it remains to be shown that \(s'\) is the identity on
\(B\). From definition of a product bundle, we know that \(p s = s'\)---since
\(p\) is the projection of the first factor---moreover, from the definition of a
cross section, \(p s = \Id_B\), therefore \(s' = \Id_B\) as wanted.
\end{proof}

\begin{definition}[Stiefel variety]
\label{def:stiefel-variety}
We define the \emph{Stiefel variety of orthonormal \(k\)-frames\footnote{ A
    \(k\)-frame in an \(n\)-dimensional vector space is an ordered collection of
    \(k\) linearly independent vectors.  }} in \(\R^n\) to be the \emph{compact
  subspace} \(\Stie_k \R^n \subseteq (S^{n-1})^k\) for which
\((v_1, \dots, v_k) \in \Stie_k \R^n\) if and only if
\(\langle v_i, v_j \rangle = \delta_{ij}\)---where \(\langle -, - \rangle\) is
the standard euclidean inner product.
\end{definition}

\begin{definition}[Grassmann variety]
\label{def:grassmann-variety}
The real \emph{\(k\)-Grassmann variety} is defined to be the topological space
\(\Grass_k \R^n\) whose points are \(k\)-dimensional subspaces of \(\R^n\), and
endowed with the quotient topology generated by the map
\(\Stie_k \R^n \epi \Grass_k \R^n\) given by
\((v_1, \dots, v_k) \mapsto \langle v_1, \dots, v_k \rangle\). Since
\(\Stie_k \R^n\) is compact, it follows that \(\Grass_k \R^n\) is also compact.
\end{definition}

\begin{example}
\label{exp:grassmanian-projective-space}
Notice that \(\Stie_1 \R^n = S^{n-1}\) and by the construction of the
Grassmannian variety, we see that \(\Grass_1 \R^n = \R \Proj^{n-1}\).
\end{example}

\begin{definition}[Bundle morphism]
\label{def:bundle-morphism}
Given two bundles \((E, p, B)\) and \((E', p', B')\), a \emph{morphism of
  bundles} \((E, p, B) \to (E', p', B')\) is a pair \((u, f)\) of morphisms
\(u: E \to E'\) and \(f: B \to B'\) such that the diagram
\[
\begin{tikzcd}
E \ar[r, "u"] \ar[d, "p"'] &E' \ar[d, "p'"] \\
B \ar[r, "f"'] &B'
\end{tikzcd}
\]
commutes in \(\Top\)---equivalently, \(u (p^{-1} B) \subseteq p'^{-1}(f
B)\). The special case where the base spaces coincide, we define a \emph{bundle
  morphism over \(B\)} (also refered to as \(B\)-morphism)
\((E, p, B) \to (E', p', B)\) to be a morphism \(u: E \to E'\) such that the
triangle
\[
\begin{tikzcd}
E \ar[rr, "u"] \ar[dr, "p"'] & &E' \ar[ld, "p'"] \\
&B &
\end{tikzcd}
\]
commutes in \(\Top\), which can equivalently be expressed as the condition
\(u (p^{-1} B) \subseteq p'^{-1} B\).

Given any two bundle morphisms \((u, f): (E, p, B) \to (E', p', B')\) and \((v,
g): (E', p', B') \to (E'', p'', B'')\), we define the \emph{composition} of
those morphisms to be the pair
\[
(v, g) \circ (u, f) \coloneq (v u, g f):
(E, p, B) \longrightarrow (E'', p'', B''),
\]
which is again a bundle morphism, since
\[
\begin{tikzcd}
E \ar[r, "u"] \ar[d, "p"']
&E' \ar[d, "p'"] \ar[r, "v"]
&E'' \ar[d, "p''"]
\\
B \ar[r, "f"']
&B' \ar[r, "g"']
&B''
\end{tikzcd}
\]
is a commutative diagram in \(\Top\).
\end{definition}

\begin{example}[Cross section as bundle morphism]
\label{exp:cross-section-is-bundle-morphism}
Notice that a cross section is nothing more than a bundle morphism over \(B\) of
the form \(s: (B, \Id_B, B) \to (E, p, B)\).
\end{example}

\begin{definition}[Category of bundles]
\label{def:category-of-bundles}
We denote by \(\Bun\) the category composed of bundles and bundle
morphisms. Given a space \(B\), we can also define a full subcategory \(\Bun_B\)
of \(\Bun\), whose objects are bundles with base space \(B\) and bundle
morphisms over \(B\).
\end{definition}

\begin{definition}[Fibre of a bundle]
\label{def:fibre-of-a-bundle}
We say that a space \(F\) is \emph{the fibre} of a bundle \((E, p, B)\) if there
exists a topological isomorphism \(p^{-1} b \iso F\) for every \(b \in B\). A
bundle \((E, p, B)\) is said to be \emph{trivial with fibre \(F\)} if there
exists a bundle \(B\)-isomorphism \((E, p, B) \iso (B \times F, p, B)\).
\end{definition}

\subsection{Universal Properties}

\begin{proposition}[Products in \(\Bun\)]
\label{prop:bundle-products}
Given a family of bundles \((E_j, p_j, B_j)_{j \in J}\), we define the
\emph{product} of this family of bundles to be the bundle
\[
\Big( \prod_{j \in J} E_j, \prod_{j \in J} p_j, \prod_{j \in J} B_j \Big),
\]
which is defines a product in the category \(\Bun\).
\end{proposition}

\begin{proposition}[Pullbacks in \(\Bun_B\)]
\label{prop:fibre-product-over-B}
Given two bundles \(\xi = (E, p, B)\) and \(\xi' = (E', p', B)\), define
\[
E \oplus E' \coloneq \{(x, x') \in E \times E' \colon p x = p' x'\},
\]
and \(q: E \oplus E' \to B\) to be the morphism
\(q(x, x') \coloneq p x = p' x'\). Then the triple
\[
\xi \otimes \xi' \coloneq (E \oplus E', q, B),
\]
called \emph{fibre product over \(B\)} of \(\xi\) and \(\xi'\), is the
\emph{pullback} of the pair \((\xi, \xi')\) in the category \(\Bun_B\).
\end{proposition}

\section{Fibre Bundles}

\begin{definition}[Bundle projection]
\label{def:bundle-projection}
Let \(X\), \(B\), and \(F\) be Hausdorff spaces. We say that a continuous map
\(p: X \to B\) is a \emph{bundle projection} with \emph{fibre} \(F\) if for each
\(b \in B\) there exists a neighbourhood \(U \subseteq B\) of \(b\) such that
there is a \emph{topological isomorphism}
\[
\phi: U \times F \longrightarrow p^{-1} U,
\quad\text{such that}\quad
p \phi(x, y) = x
\]
for all \(x \in U\) and \(y \in F\)---the map \(\phi\) is called a
\emph{trivialization of the bundle over \(U\)}. This means that on the set
\(p^{-1} U\), the map \(p\) is a \emph{projection} of the type
\(U \times F \epi U\).
\end{definition}

\begin{definition}[Fibre bundle]
\label{def:fibre-bundle}
Let \(G\) be a topological group acting \emph{effectively} on a Hausdorff space
\(F\)---seen as a group of topological isomorphisms. Let \(X\) and \(B\) be
Hausdorff spaces. We define a \emph{fibre bundle} (or simply \emph{bundle}) over
the \emph{base space} \(B\) with \emph{total space} \(X\), \emph{fibre} \(F\),
and \emph{structure group} \(G\), to be a pair \((p, \Phi)\) where
\(p: X \to B\) is a \emph{bundle projection} and \(\Phi\) is a collection of
trivializations of \(p\) (as described in \cref{def:bundle-projection})---the
members of \(\Phi\) will be called \emph{charts} over \(U\)---such that:
\begin{itemize}\setlength\itemsep{0em}
\item For each \(b \in B\) there exists a neighbourhood \(U \subseteq B\) of
  \(b\) and a chart \(\phi \in \Phi\) of the form
  \(\phi: U \times F \to p^{-1} U\).

\item Given a chart \(\phi: U \times F \to p^{-1} U\), member of \(\Phi\), then
  any subset \(V \subseteq U\) is such that the restriction
  \(\phi|_{V \times F}\) belongs to the family \(\Phi\).

\item Given any pair of charts \(\phi, \psi \in \Phi\) over a common open set
  \(U\), there exists a continuous map \(\theta: U \to G\) such that
  \[
  \psi(u, y) = \phi(u, \theta(u)(y)).
  \]

\item The family \(\Phi\) is \emph{maximal} among the collections satisfying the
  previous properties.
\end{itemize}
The fibre bundle is said to be \emph{smooth} if each object above is a
\emph{smooth manifold} and all maps are \emph{smooth morphisms}.
\end{definition}

\section{Vector Bundle}

\begin{definition}[Vector bundle]
\label{def:vector-bundle}
A \emph{vector bundle} is a fibre bundle with a fibre \(\R^n\) and structure
group contained in \(\GL_n(\R)\). Given a vector bundle \(\xi\), we denote its
total space by \(E \xi\) and base space by \(B \xi\).
\end{definition}

\begin{definition}[Morphism of vector bundles]
\label{def:morphism-vector-bundles}
If \(\xi\) and \(\eta\) are any two vector bundles, we define a \emph{bundle
  morphism} \(\xi \to \eta\) is a pair \((u, f)\) of continuous maps \(u: E \xi
\to E \eta\) and \(f: B \xi \to B \eta\) such that the diagram
\[
\begin{tikzcd}
E \xi \ar[d, "\xi"] \ar[r, "\Psi"] &E \eta \ar[d, "\eta"] \\
B \xi \ar[r, "\psi"'] &B\eta
\end{tikzcd}
\]
commutes and the restriction \(u: p^{-1} b \to p^{-1}(f b)\) is
\emph{\(\R\)-linear} for every \(b \in B\).
\end{definition}

\begin{definition}[Vector bundle chart \& atlas]
\label{def:vector-bundle-chart-and-atlas}
Given a vector bundle \((E, p, B)\), we define an \(n\)-dimensional chart
\((U, \phi)\), for some open set \(U \subseteq B\), to be a topological
isomorphism
\[
\phi: p^{-1} U \overset{\iso}\longrightarrow U \times \R^n
\]
such that the diagram
\[
\begin{tikzcd}
p^{-1} U \ar[r, "\phi"]
\ar[d, "p"']
&U \times \R \ar[ld, bend left, "\pi_1"] \\
U &
\end{tikzcd}
\]
commutes in \(\Top\)---and \(\pi_1\) is the projection of the first factor. The
isomorphism \(\phi\) induces a collection
\((\phi_x: p^{-1} x \isoto \R^n)_{x \in U}\) of isomorphisms given by the
composition
\[
\begin{tikzcd}
p^{-1} x \ar[r, "\phi"', "\dis"]
\ar[rr, "\phi_x", bend left]
&x \times \R^n
\ar[r, "\dis", "\pi_2"']
&\R^n
\end{tikzcd}
\]
Therefore, given any \(y \in p^{-1} U\), if \(y \in p^{-1} x\), then
\(\phi y = (x, \phi_x y)\).

A family \(\Phi \coloneq (U_j, \phi_j)_{j \in J}\) of vector bundle charts on
\((E, p, B)\) with domain covering \(B\) and values in \(\R^n\) is said to form
a \emph{vector bundle atlas} for \((E, p, B)\) if for any two vector bundle
charts \((U, \phi)\) and \((V, \psi)\) of \(\Phi\) we have:
\begin{enumerate}[(a)]\setlength\itemsep{0em}
\item For every \(x \in U \cap V\), the transition map
  \[
  \psi_x \phi_x^{-1}: \R^n \overset{\iso}\longrightarrow \R^n
  \]
  is an \emph{\(\R\)-linear topological isomorphism}.

\item The map \(g: U \cap V \to \GL_n(\R)\) sending
  \(x \mapsto \psi_x \phi_x^{-1}\) is \emph{continuous}.
\end{enumerate}
If such conditions are satisfied, by the requirement of item (b), the atlas
\(\Phi\) induces a collection of continuous maps
\[
(g_{i j}: U_i \cap U_j \longrightarrow \GL_n(\R))_{(i, j) \in J \times J},
\]
called \emph{cocycle} of \(\Phi\). For any three \(i, j, k \in J\), and
\(x \in U_i \cap U_j \cap U_k\), one has that
\[
g_{ij}(x) g_{j k}(x)
= ((\phi_i)_x (\phi_j^{-1})_x) ((\phi_j)_x (\phi_k)_x)
= (\phi_i)_x (\phi_k)_{x}
= g_{i k} x.
\]
Moreover, for any \(j \in J\) we find \(g_{i i} x = \Id_{\R^n}\). The vector
bundle together with \(\Phi\), is said to be a \emph{vector bundle with
  \(n\)-dimensional fibre}.

For every \(x \in B\), we can endow the fibre \(E_x \coloneq p^{-1} x\) with the
structure of an \(\R\)-vector space for which \(\phi_x: E_x \isoto \R^n\) is an
\(\R\)-linear isomorphism
\todo[inline]{Continue, you fool!}
\end{definition}



%%% Local Variables:
%%% mode: latex
%%% TeX-master: "../../../deep-dive"
%%% End:

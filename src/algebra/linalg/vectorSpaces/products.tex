\section{Product and Direct Sum of Vector Spaces}

\subsection{Free Vector Spaces}

\begin{proposition}
    Let \(S\) be a set and define the power set \(k^S\) with addition and
    scalar multiplication, that is, given \(f, g \in k^S\) and \(a \in k\) we
    have \((f + g)(x) = f(x) + g(x)\) and \((a f)(x) = af(x)\). Then the set
    \(k^S\) is a \(k\)-vector space.
\end{proposition}

\begin{proof}
    Notice that clearly \((k^S, +)\) is an abelian group from the construction of
    the additive structure, moreover, since \(k\) is a field, it inherit the
    associativity and distributivity of scalar multiplication. The \(0\) vector
    can be regarded as the map whose image is the singleton \(\{0\}\).
\end{proof}

\begin{proposition}[Functoriality of \(k^S\)]
    Let sets \(S, S'\) and a map \(\alpha : S \to S'\), then
    \[
        \alpha^\ast : k^{S'} \to k^S \ \text{ mapping }\ (f : S' \to k) \mapsto
        (f  \alpha : S \to k)
    \]
    is a morphism of \(k\)-vector spaces. Moreover if \(\beta : S' \to S''\) then
    \((\beta  \alpha)^\ast = \alpha^\ast  \beta^\ast\).
\end{proposition}

\begin{proof}
    Let maps \(f, g : S' \to k\), then
    \[
        (f + g : S' \to k) \xmapsto{\alpha^\ast} \left( (f + g) \alpha : S
        \to k \right),
    \]
    but for every \(s \in S\) we have \((f + g)(\alpha(s)) = f(\alpha(s)) +
    g(\alpha(s))\), which shows the first property. For the second property, let
    \(a \in k\), then we get the map
    \[
        ((a \cdot f) : S' \to k) \xmapsto{\alpha^\ast} ( (a \cdot f)  \alpha
        : S \to k),
    \]
    but for each \(s \in S\) we have \((a \cdot f)(\alpha(s)) = a \cdot
    f(\alpha(s))\), proving the second property, which finishes the proof.
\end{proof}

\begin{definition}[Free vector space]\label{def: free vector space}
    Let \(S\) be a set, we define the free vector space on \(S\) to be the object
    \[
        k^{\oplus S} = \{f \in k^S \colon f(s) \neq 0 \text{ only for finitely many }
        s \in S\},
    \]
    together with an additive structure and scalar multiplication, satisfying for
    all \(f, g \in k^{\oplus S}\), \(s \in S\) and \(a \in k\):
    \[
        (f + g)(s) = f(s) + g(s)\ \text{ and }\ (a \cdot f)(s) = a \cdot f(s).
    \]
\end{definition}

\begin{proposition}
    The object \(k^{\oplus S}\) is a \(k\)-vector space.
\end{proposition}

\begin{proof}
    Let the maps \(f, g \in k^{\oplus S}\), so that the specification of
    finiteness of non-zero values is satisfied. Denote \(A, B\) the finite sets
    containing the non-zero elements of \(S\) under, respectively, \(f\) and
    \(g\). Notice now that the map \(f + g \in k^S\) is such that for all \(s \in
    A \cup B\) then \((f+g)(s) = f(s)+g(s) \neq  0\) thus, denoting \(C\) as the
    set containing all non-zero elements of \(S\) under \(f + g\), we see that \(A
    \cup B \subseteq C\). The converse is trivial since if \(s \in C\) then
    \(f(s) + g(s) \neq 0\) and thus at least one of the images is non-zero, thus
    \(s \in A \cup B\) and then \(C \subseteq  A \cup B\), which implies that \(A
    \cup B = C\) and, in particular, \(C\) is finite. Therefore \(f + g \in
    k^{\oplus S}\). Note now that for any \(c \in \R \setminus \{0\}\),
    then \(c \cdot f\) has non-zero values only in \(A \subseteq S\), in the case
    \(c = 0\) then  \(c \cdot f\) is zero all over \(S\), which makes \(c \cdot f
    \in k^{\oplus S}\). Thus indeed \(k^{\oplus S}\) is a vector space.
\end{proof}

For each \(s \in S\) we can define a map \(\mathbf{s} : S \to k\) such that
\[
    \mathbf{s}(x) \coloneq
    \begin{cases}
        1, & \text{ for } x = s    \\
        0, & \text{ for } x \neq s
    \end{cases}
\]
Also, this notion comes together with the natural monomorphism
\[
    \iota : S \hookrightarrow k^{\oplus S} \text{ mapping } s \mapsto \mathbf{s}
\]
which allow us to write any element \(v \in k^{\oplus S}\) as a linear
combination, with non-zero scalars \(a_i \in k\), of the form
\[
    v = \sum_{i=1}^n a_i \mathbf{s}_i \in k^{\oplus S},\
    \text{ mapping }\ s \mapsto
    \begin{cases}
        a_i, & \text{if } s = s_i \text{ for some } i   \\
        0,   & \text{if } s \neq s_i \text{ for all } i
    \end{cases}
\]

\begin{proposition}[Universal property of the free vector spaces]
    \label{prop: universal property free vs}
    Let \(S\) be a set, and \(V\) be a \(k\)-vector space, and a function \(f : S
    \to V\). Then there exists a unique \(k\)-linear morphism \(\ell : k^{\oplus
            S} \to V\) such that the diagram commutes:
    \[
        \begin{tikzcd}
            S \ar[r, "f"] \ar[d, hook, "\iota"] &V \\
            k^{\oplus S} \ar[ru, dashed, swap, "\ell"]
        \end{tikzcd}
    \]
\end{proposition}

\begin{proof}
    (Uniqueness) Notice that for all \(\alpha \in k^{\oplus S}\) we can write it
    as a finite sum \(\alpha = \sum_{s \in S}\alpha(s) \mathbf s\) (the
    finiteness comes from the fact that only finitely may \(\mathbf s\) will
    actually appear, since there are only finite non-zero values under
    \(\alpha\)). Let now  \(L : k^{\oplus S} \to V\) be a morphism of
    \(k\)-vector spaces such that \(f = L  \iota\) for a given \(f : S \to
    V\). Then
    \[
        L(\alpha) = \sum_{s \in S} L(\alpha(s) \mathbf{s}) = \sum_{s \in S}
        \alpha(s) L(\mathbf s) = \sum_{s \in S} \alpha(s) (L  \iota)(s)
        = \sum_{s \in S} \alpha(s) f(s),
    \]
    which implies that the morphism \(L\) is indeed unique.

    To prove the existence of \(L\), consider its definition as above. Notice
    that for any \(\alpha, \beta \in k^{\oplus S}\) we have
    \[
        L(\alpha + \beta) = \sum_{s \in S} (\alpha + \beta)(s) f(s) = \sum_{s \in
            S} \alpha(s)f(s) + \sum_{s \in S} \beta(s)f(s) = L(\alpha) + L(\beta).
    \]
    Moreover, if \(a \in k\), then \(k \alpha = k\sum_{s \in S} \alpha(s)
    \mathbf s\), thus indeed
    \[
        L(k \alpha) = \sum_{s \in S} (k \alpha)(s) f(s) = k \sum_{s \in S}
        \alpha(s) f(s) = k L(\alpha),
    \]
    proving the last condition in order to be a morphism of \(k\)-vector spaces,
\end{proof}

This way we find that actually
\[
    \Hom_{\Set}(S, V) \iso \Hom_{\Vect_k}(k^{\oplus S}, V).
\]

We now want in some way to construct a morphism that maps \(k^{\oplus S'} \to
k^{\oplus S}\), where \(S, S'\) are sets, just as we've done for the case
\(k^{S'} \to k^S\). To do so, we can first consider the maps \(\alpha: S \to
S'\), and the inclusion maps \(\iota_S : S \to k^{\oplus S}\) and \(\iota_{S'} :
S' \hookrightarrow k^{\oplus S'}\), and construct the map
\[
    \ell = \iota_{S'}  \alpha : S \to k^{\oplus S'}.
\]
Then, by means of the universal property for free vector spaces, we find that
there exists a unique morphism of \(k\)-vector spaces \(\alpha_\ast : k^{\oplus
        S} \to k^{\oplus S'} \) such that \(\ell = \alpha_\ast  \iota_S\), that is,
the following diagram commutes
\[
    \begin{tikzcd}
        &S \arrow[r, "\alpha"] \arrow[d, "\iota_S"] \arrow[dr, "\ell"]
        &S' \arrow[d, "\iota_{S'}"]
        \\
        &k^{\oplus S} \arrow[r, dashed, "\alpha_\ast"]
        &k^{\oplus S'}
    \end{tikzcd}
\]

\subsection{Product}

\begin{definition}[Pair product]
    Let \(L, V\) be \(k\)-vector spaces. Then the product
    \[
        L \times V = \{(\ell, v) \colon \ell \in L,\ v \in V\},
    \]
    equipped with pairwise sum and product by scalar is a \(k\)-vector space.
\end{definition}

\begin{definition}[General product]
    Let \(\{V_i\}_{i \in I}\) be a set of \(k\)-vector spaces, with an indexing
    set \(I\). The product of these vector spaces is defined as
    \[
        \prod_{i \in I} V_i = \{(v_i)_{i \in I} \colon v_i \in V_i\}
    \]
    equipped with addition and multiplication by scalar as
    \[
        (v_i)_{i \in I} + (u_i)_{i \in I} = (v_i + u_i)_{i \in I}\ \text{ and }\
        a(v_i)_{i \in I} = (av_i)_{i \in I}
    \]
    is a \(k\)-vector space.
\end{definition}

\begin{proposition}
    Let \(S\) be a set. Then there exists a natural isomorphism
    \[
        k^S \iso \prod_{s \in S} k,
    \]
    so that \(\prod_{s \in S} k\) generalizes the power set.
\end{proposition}

\begin{proof}
    For any \(f \in k^S\) map \((f : S \to k) \mapsto (f(s))_{s \in S}\). Notice
    that the mapping is obviously a monomorphism, since the maps \(S \to k\) are
    well defined, and is also an epimorphism because we just need to create
    a function fitting such an image.
\end{proof}

\begin{proposition}[Universal property for products]
    Let the indexed set of \(k\)-vector spaces \(\{V_i\}_{i \in I}\). For any
    given \(k\)-vector space \(L\), together with a morphism \(\varphi_j : L \to
    V\), there exists a unique morphism \(\ell : L \to \prod_{i \in I} V_i\) such
    that the diagram commutes:
    \[
        \begin{tikzcd}
            &L
            \ar[r, dashed, "\ell"]
            \ar[dr, swap, "\varphi_j"]
            &\prod_{i \in I} V_i
            \ar[d, "\pi_j", two heads] \\
            &
            &V_j
        \end{tikzcd}
    \]
\end{proposition}

\begin{proof}
    Notice that this is simply the product universal product restricted for only
    one branch of the product \(V_j \times \prod_{i \neq j} V_i\). For the
    uniqueness, notice that since we want \(\varphi_j = \pi_j  \ell\), then
    it ought to be the case that
    \[
        L \ni x \overset{\ell}\longmapsto (\varphi_j(x))_{i \in I} \in \prod_{i
            \in I} V_i
    \]
    since the composition needs to be satisfied for any index \(j \in I\).
    Therefore the morphism \(\ell\) is indeed unique, if it exists. Now we show
    its existence. Notice that since \(\varphi_j\) is a \(k\)-linear map, then
    indeed for any \(x, y \in L\) and \(a \in k\) we have \(\ell(x + ay) =
    (\varphi_i(x + ay))_{i \in I} = (\varphi_i(x))_{i \in I} + a
    (\varphi_i(y))_{i \in I} = \ell(x) + a \ell(y)\), which shows the linearity
    and the existence.
\end{proof}

\begin{definition}[Direct sum]
    Given an indexed collection of \(k\)-vector spaces \(\{V_i\}_{i \in I}\), we
    define the direct sum of them as
    \[
        \bigoplus _{i \in I} V_i \coloneq \{(v_i)_{i \in I} \colon v_i \in V_i, \text{ where
        } v_i \neq 0 \text{ finitely many times in } I\}
    \]
    together with addition and scalar multiplication being defined component-wise.
\end{definition}

\begin{proposition}
    The direct sum of \(k\)-vector spaces is again a \(k\)-vector space.
\end{proposition}

\begin{proposition}
    Let a set \(S\). Then, the exists natural isomorphism \(\bigoplus_{s \in S} k
    \iso k^{\oplus S}\).
\end{proposition}

\begin{proof}
    Notice that if we make the morphism \(\ell : \bigoplus_{s \in S} k \to
    k^{\oplus S}\) defined as the mapping
    \[
        \bigoplus_{s \in S} k \ni (a_s)_{s \in S} \overset{\ell}\longmapsto (s
        \xmapsto{f} a_s) \in k^{\oplus S}
    \]
    then we see that no information is lost since this is surely injective,
    because the morphisms of \(k^{\oplus S}\) are well defined, and given any
    morphism \(f \in k^{\oplus S}\) we know that its non-zero values are finite,
    thus we can construct its image as a tuple in \(\bigoplus_{s \in S} k\),
    which allow us to say that such mapping is also surjective. Therefore, the
    morphism \(\ell\) is an isomorphism. We now show that it is indeed
    \(k\)-linear. Let the tuples \((a_s)_{s \in S}\) and \((b_s)_{s \in S}\),
    then
    \[
        (a_s)_s + (b_s)_s = (a_s + b_s)_s \overset \ell \longmapsto (s \xmapsto f
        a_s + b_s) = (s \xmapsto g a_s) + (s \xmapsto h b_s).
    \]
    Moreover, if \(c \in k\) is any scalar, then \((c a_s)_s \longmapsto (s
    \mapsto c a_s) = c (s \mapsto a_s)\), which shows the last property for
    \(\ell\) being a \(k\)-linear morphism.
\end{proof}

\begin{proposition}[Universal property for the direct sum]
    Let the collection of \(k\)-vector spaces \(\{V_i\}_{i \in I}\) and for every
    \(j \in I\) define the inclusion \(\iota_j : V_j \to \bigoplus_{i \in I}
    V_i\), where
    \[
        v \overset{\iota_j}\longmapsto (v_i)_{i \in I}, \text{ where } v_i =
        \begin{cases}
            v, & i = j    \\
            0, & i \neq j
        \end{cases}.
    \]

    Then, for any arbitrary \(k\)-vector space \(L\), together with \(k\)-linear
    morphisms \((f_i : V_i \to L)_{i \in I}\), there exists a unique \(k\)-linear
    morphism \(\ell : \bigoplus_{i \in I} V_i \to L\) such that the diagram
    commutes for every \(j \in I\):
    \[
        \begin{tikzcd}
            &L &\bigoplus_{i \in I} V_i \ar[l, dashed, swap, "\ell"]  \\
            & &V_j \ar[u, hook, swap, "\iota_j"] \ar[ul, "f_j"]
        \end{tikzcd}
    \]
\end{proposition}

\begin{proof}
    Since we want to have \(\ell  \iota_j = f_j\), it must be the case for
    any \(v \in v_j\) that, being \(\iota_j(v) \coloneq (v_i)_i\), then
    \[
        \ell((v_i)_{i \in i}) = \ell \left( \sum_{i \in i} \iota_i(v_i) \right) =
        \sum_{i \in i} \ell(\iota_i(v_i)) = \sum_{i \in i} f_i(v_i)
    \]
    thus indeed necessarily, if such a map exists, then \(\ell\) is unique. We
    now check that \(\ell\) is indeed a \(k\)-linear morphism.
    \begin{gather*}
        \ell\left( (v_i + w_i)_i \right) = \sum_i f_i(v_i + w_i) = \sum_i f_i(v_i)
        + \sum_i  f_i(w_i) \\
        \ell((av_i)_i) = \sum_i f_i(av_i) = a \sum_i f_i(v_i)
    \end{gather*}
    Which proves that \(\ell\) is indeed a morphism of \(k\) vector spaces as
    wanted.
\end{proof}

\begin{proposition}
    Let \(\{S_i\}_{i \in S}\) be a collection of sets. There exists a canonical
    isomorphism
    \[
        \bigoplus_{i \in I} k^{\oplus S_i} \iso k^{\oplus \left( \coprod_{i \in I}
                S_i \right)}
    \]
\end{proposition}

\begin{proof}
    Just use both universal properties for coproduct and free vector spaces, this
    establishes a two way unique morphism, which proves the canonical
    isomorphism.
\end{proof}

\begin{proposition}
    Let the \(k\)-linear morphism \(T : k^n \to k^m\) and \(S:k^n \to k^\ell\).
    If we consider their matrix representation, they induce the \(k\)-linear
    morphism
    \[
        M : k^n \to k^{m+\ell}\ \text{ represented by }\
        M = \begin{pmatrix} T \\ S \end{pmatrix}
    \]
\end{proposition}

\begin{proof}
    Notice that from the universal property of products we have
    \[
        \begin{tikzcd}
            &k^n \ar["T", r] \ar["M", dr] \ar[swap, "S", d] &k^m \\
            &k^\ell &k^m \times k^\ell \iso k^{m+\ell} \ar["\pi_\ell", l]
            \ar[swap, "\pi_m", u]
        \end{tikzcd}
    \]
    Where the isomorphism \(k^m \times k^\ell \iso k^{m+\ell}\) is induced by the
    mapping  \(((a_i)_{i=1}^m, (b_i)_{i=1}^\ell) = (c_i)_{i=1}^{m+\ell}\) where
    \(c_i \coloneq a_i\) for all \(1 \leq i \leq m\) and \(c_i \coloneq b_i\) for all  \(m+1
    \leq i \leq m+\ell\). Notice then that the morphism \(M\) is induced by the
    morphism \(T, S\) so that \(T = \pi_m  M\) and \(S = \pi_\ell  M\),
    so that we need to have \(M(v) = (T(v), S(v))\) for all \(v \in k^m\), but
    notice that \(M : k^n \to k^m\times k^\ell\) is in fact isomorphic to a
    \(k\)-linear map \(k^n \to k^{m+\ell}\), so that we can encode the
    transformation of \(M\), without losing, information as follows. Suppose that
    \[
        T =
        \begin{bmatrix}
            a_{11} & \dots  & a_{1n} \\
            \vdots & \ddots & \vdots \\
            a_{m1} & \dots  & a_{mn}
        \end{bmatrix}
        \ \text{ and }\
        S =
        \begin{bmatrix}
            b_{11}     & \dots  & b_{1n}     \\
            \vdots     & \ddots & \vdots     \\
            b_{\ell 1} & \dots  & b_{\ell n}
        \end{bmatrix}
    \]
    then we can say that \(M\) can be isomorphically represented by
    \[
        M(v) = \left((u_i)_{i=1}^m, (u_i)_{i=m+1}^{m+\ell} \right) \iso
        \begin{bmatrix}
            a_{11}         & \dots  & a_{1n}         \\
            \vdots         & \ddots & \vdots         \\
            a_{m1}         & \dots  & a_{mn}         \\
            b_{(m+1) 1}    & \dots  & b_{(m+1) n}    \\
            \vdots         & \ddots & \vdots         \\
            b_{(m+\ell) 1} & \dots  & b_{(m+\ell) n}
        \end{bmatrix}
        \begin{bmatrix}
            v_1 \\ \vdots \\ v_n
        \end{bmatrix}
        =
        \begin{bmatrix} u_1 \\ \vdots \\ u_{m+\ell} \end{bmatrix}
    \]
    where \((u_i)_{i=1}^m = T(v) \) and \((u_i)_{i=m+1}^{m+\ell} = S(v)\).
\end{proof}

\begin{proposition}
    Let the \(k\)-linear morphism \(T : k^n \to k^\ell\) and \(S : k^m \to
    k^\ell\). They induce the \(k\)-linear morphism
    \[
        M : k^{n+m} \iso k^n \oplus k^m \to k^\ell \text{ represented by }
        M = \begin{pmatrix} T &S \end{pmatrix}
    \]
\end{proposition}

\begin{proof}
    From the universal property of the coproduct we have that
    \[
        \begin{tikzcd}
            &k^n \ar[hook, "\iota_n", d]  \ar[dr, "T"]& \\
            & k^{n+m} \iso k^n \oplus k^m \ar[r, "M"] &k^\ell \\
            &k^m \ar[hook, "\iota_m", u] \ar[ur, "S"] &
        \end{tikzcd}
    \]
    Where \(k^n, k^m\) are both finite, then \(k^n \oplus k^m \iso k^n \times k^m
    \iso k^{n+m}\), which proves the isomorphism written. Therefore \(M : k^n
    \oplus k^m \to k^\ell\) is isomorphic to a \(k\)-linear morphism \(k^{n+m}
    \to k^\ell\). Moreover, if \(T\) and \(S\) are represented by
    \[
        T =
        \begin{bmatrix}
            a_{11}     & \dots  & a_{1n}     \\
            \vdots     & \ddots & \vdots     \\
            a_{\ell 1} & \dots  & a_{\ell n}
        \end{bmatrix}
        \ \text{ and }\
        S =
        \begin{bmatrix}
            b_{11}     & \dots  & b_{1m}     \\
            \vdots     & \ddots & \vdots     \\
            b_{\ell 1} & \dots  & b_{\ell m}
        \end{bmatrix}
    \]
    Then from the construction \(T = M  \iota_n\) and \(S = M
    \iota_m\) we conclude that for all element \(a = (a_i)_{i=1}^n \in k^n\) and
    every \(b = (b_i)_{i=1}^m \in k^m\), will be such that
    \begin{gather*}
        k^n \ni (a_i)_{i=1}^n \xmapsto{\iota_n} (a_1, \dots, a_n, 0, \dots, 0)
        \xmapsto{M} (a'_i)_{i=1}^\ell = T(a) \in k^\ell
        \\
        k^m \ni (b_i)_{i=1}^m \xmapsto{\iota_m} (b_1, \dots, b_m, 0, \dots, 0)
        \xmapsto{M} (b'_i)_{i=1}^\ell = S(b) \in k^\ell
    \end{gather*}
    therefore, if we have the element
    \[
        (a_1, \dots, a_n, b_1, \dots, b_m) \iso \left( (a_1, \dots, a_n), (b_1,
        \dots, b_m) \right) \xmapsto M (a'_i + b'_i)_{i=1}^\ell = T(a) + S(b)
    \]
    which shows that indeed
    \[
        M =
        \begin{bmatrix}
            a_{11}     & \dots  & a_{1n}     & b_{1(n+1)}     & \dots  & b_{1(n+m)}    \\
            \vdots     & \ddots & \vdots     & \vdots         & \ddots & \vdots        \\
            a_{\ell 1} & \dots  & a_{\ell n} & b_{\ell (n+1)} & \dots  & b_{\ell(n+m)}
        \end{bmatrix}
        =
        \begin{bmatrix}
            T & S
        \end{bmatrix}
        .
    \]
\end{proof}

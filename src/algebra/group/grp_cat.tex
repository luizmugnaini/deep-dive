\section{\texorpdfstring{\(\Grp\)}{Grp} Category}

\begin{definition}[Group morphism]\label{def: grp-morphism}
    Let \((G, \cdot_G)\) and \((H, \cdot_H)\) be groups together with their binary
    operation. A group morphism --- also called homomorphism --- is a map \(\varphi:
    (G, \cdot_G) \to (H, \cdot_H)\) such that the that the following diagram
    commutes
    \[
        \begin{tikzcd}
            G \times G \ar[d, swap, "\cdot_G"] \ar[r, "\varphi \times \varphi"]
            &H \times H \ar[d, "\cdot_H"] \\
            G \ar[r, "\varphi"] &H
        \end{tikzcd}
    \]
    Where \(\varphi \times \varphi\) is uniquely defined in \(\Set\) by
    \cref{prop: product-morphism} --- mapping \((g, \ell) \xmapsto{\varphi \times
        \varphi} (\varphi(g), \varphi(\ell))\). The commutativity of such diagram can
    be viewed as the requirement that \(\varphi\) preserves the structure coming
    from the binary operations --- that is, for any \(g, \ell \in G\)
    \[
        \varphi(g \cdot_G \ell) = \varphi(g) \cdot_H \varphi(\ell).
    \]
\end{definition}

\begin{definition}[Category of groups]\label{def: grp}
    The category of groups \(\Grp\) consists of the collection of objects ---
    called groups --- and group morphisms between them.
\end{definition}

\begin{proposition}
    \(\Grp\) is a category.
\end{proposition}

\begin{proof}
    Let \((G, \cdot_G)\), \((H, \cdot_H)\) and \((K, \cdot_K)\) be any
    groups. The identity \(\Id_G: G \to G\) is a group morphism since \(\Id_G(g
    \cdot_G \ell) = g \cdot_G \ell\) for any \(g, \ell \in G\). Moreover,
    we can define a map
    \[
        f: \Hom_{\Grp}(G, H) \times \Hom_{\Grp}(H, K) \to \Hom_{\Grp}(G, K)
    \]
    with the mapping \((\psi, \varphi) \xmapsto f \psi \varphi\) --- since
    the following diagram commutes
    \[
        \begin{tikzcd}
            G \times G \ar[r, "\varphi \times \varphi"]
            \ar[d, swap, "\cdot_G"]
            \ar[rr, bend left, "(\psi \varphi) \times (\psi \varphi)"]
            &H \times H \ar[r, "\psi \times \psi"]
            \ar[d, "\cdot_H"]
            &K \times K \ar[d, "\cdot_K"]
            \\
            G \ar[r, "\varphi"]
            \ar[rr, bend right, swap, "\psi \varphi"]
            &H \ar[r, "\psi"] &K
        \end{tikzcd}
    \]
    In other words, for any \(g, \ell \in G\) we have
    \[
        \psi \varphi (g \cdot_G \ell)
        = \psi(\varphi(g \cdot_G \ell))
        = \psi(\varphi(g) \cdot_H \varphi(\ell))
        = \psi(\varphi(g)) \cdot_K \psi(\varphi(\ell))
        = \psi \varphi(g) \cdot_K \psi \varphi(\ell).
    \]
    Therefore \(\psi \varphi \in \Hom_\Grp(G, K)\). For the other part of the
    diagram, we have
    \[
        (\psi \varphi) \times (\psi \varphi) (g, \ell)
        = (\psi \varphi(g), \psi \varphi(\ell))
        = \psi \times \psi(\varphi(g), \varphi(\ell))
        = (\psi \times \psi) (\varphi \times \varphi) (g, \ell)
    \]

    Since group morphisms are maps in \(\Set\), we have that associativity is
    inherited.
\end{proof}

\begin{proposition}\label{prop: forgetful-func-grp-set}
    There exists a covariant forgetful functor \(F: \Grp \to \Set\).
\end{proposition}

\begin{proof}
    For objects, define \(F\) as \(F(G, \cdot_G) = G\) --- where we denoted \(G\)
    together with its binary operation only to express that the multiplicative
    structure is lost in the process. Let \(\varphi: (G, \cdot_G) \to (H, \cdot_H)\)
    be a group morphism, denote by \(\overline \varphi \in \Mor(\Set)\) the function
    \(\overline\varphi: G \to H\) such that \(\overline\varphi(g) = \varphi(g)\) for
    all \(g \in G\). For such morphisms we define \(F\) as \(F\varphi =
    \overline\varphi: F(G, \cdot_G) \to F(H, \cdot_H)\).

    Let \(\psi \in \Hom_\Grp(H K)\), then we have
    \(\overline{\psi \varphi} = \overline \psi \overline \varphi: H \to K\). Thus
    \[
        F(\psi \varphi) = \overline{\psi \varphi}
        = \overline \psi \overline \varphi = F\psi F\varphi.
    \]
    This shows that \(F\) is a covariant forgetful functor.
\end{proof}

\begin{proposition}
    The trivial group \(* \in \Grp\) is the initial and final object of \(\Grp\).
    That is, for any \(G \in \Grp\) the diagram
    \[
        \begin{tikzcd}
            * \ar[r, dashed, bend right, swap, "\varphi"]
            &G \ar[l, dashed, bend right, swap, "\psi"]
        \end{tikzcd}
    \]
    commutes for uniquely defined group morphisms \(\varphi\) and \(\psi\).
\end{proposition}

\begin{proof}
    Let \(G \in \Grp\) be any group. We define maps \(\varphi: * \to G\) mapping
    \(e \xmapsto\varphi e_G\), where \(e\) is the only element of \(*\) --- being
    unique possible map \(* \to G\) that preserves the group structure. Clearly,
    \(\varphi\) is a group morphism since \(\varphi(e e) = \varphi(e) = e_G =
    \varphi(e) \varphi(e)\) --- this shows that \(*\) is the initial object of
    \(\Grp\). Let \(\psi: G \to *\) be a map defined by \(g \xmapsto\psi e\) ---
    which is clearly unique. Then \(\psi\) is a morphism of groups, because
    \(\psi(g h) = e = \psi(g) \psi(h)\) --- showing that \(*\) is the final object
    of \(\Grp\).
\end{proof}

\subsection{Properties of Morphisms}

\begin{proposition}[Commuting on inverses]
    \label{prop:grp-morphism-commute-inverse}
    Let \((G, \cdot_G), (H, \cdot_H)\) be groups and consider \(\varphi \in
    \Hom_\Grp(G, H)\). Define \(\operatorname{inv}_G: G \isoto G\) and
    \(\operatorname{inv}_H: H \isoto H\) as the maps \(g \xmapsto
    {\operatorname{inv}_G} g^{-1}\) and \(h \xmapsto {\operatorname{inv}_H}
    h^{-1}\). Then the following diagram commutes
    \[
        \begin{tikzcd}
            G \ar[r, "\varphi"]
            \ar[d, swap, "\operatorname{inv}_G"]
            &H \ar[d, "\operatorname{inv}_H"] \\
            G \ar[r, "\varphi"] &H
        \end{tikzcd}
    \]
    That is, \(\varphi(g^{-1}) = \varphi(g)^{-1}\) for every \(g \in G\).
\end{proposition}

\begin{proof}
    Let \(g \in G\) be any element, then
    \[
        \varphi(g^{-1}) = \varphi(g^{-1} e_G) = \varphi(g^{-1} \cdot_G g
        \cdot_G g^{-1}) = \varphi(g^{-1}) \cdot_H \varphi(g) \cdot_H
        \varphi(g^{-1}),
    \]
    applying cancellation law on the equation above we find
    \[
        e_H = \varphi(g) \cdot_H \varphi(g^{-1}).
    \]
    Hence \(\varphi(g^{-1}) = {\varphi(g)}^{-1}\). Moreover, this implies that \(e_G
    \xmapsto \varphi e_H\) since
    \[
        \varphi(e_G) = \varphi(g \cdot_G g^{-1}) = \varphi(g) \cdot_H
        \varphi(g^{-1}) = \varphi(g) \cdot_H {\varphi(g)}^{-1} = e_H.
    \]
\end{proof}

\begin{proposition}[Generators and unique extension]
    \label{prop:generator-unique-extension}
    Let \(G\) be a group and \(S\) be a generator set for \(G\). Let \(H\) be any
    group and \(f: S \to H\) be a set-function. If there exists a morphism \(\phi: G
    \to H\) such that \(\phi|_S = f\), then \(\phi\) is unique.
\end{proposition}

\begin{proof}
    Let \(\psi: G \to H\) be another morphism satisfying the condition specified
    above --- then clearly \(\psi|_S = f = \phi|_{S}\), that is, \(\phi\) and
    \(\psi\) agree on \(S\). Since \(S\) generates \(G\), every element \(g \in G\)
    can be written as a finite product \(g = \prod_{j} s_j \in \langle S \rangle\)
    thus
    \[
        \psi(g) = \psi\Big( \prod_j s_j \Big)
        = \prod_j \psi(s_j) = \prod_j \phi(s_j)
        = \phi\Big( \prod_j s_j \Big) = \phi(g),
    \]
    which implies in \(\psi = \phi\).
\end{proof}

\begin{proposition}[Image subgroup]
    \label{prop:morphism-image-subgroup}
    Let \(\phi: G \to H\) be a morphism of groups, then \(\im \phi \subseteq H\) is
    a subgroup of \(H\).
\end{proposition}

\begin{proof}
    Let \(h \in \im \phi\) be any element and consider \(g \in \phi^{-1}(h)\), from
    \cref{prop:grp-morphism-commute-inverse} we see that \(\phi(g^{-1}) =
    \phi(g)^{-1} = h^{-1} \in \im \phi\), thus \(\im \phi\) is closed under
    inverses. Moreover, given another \(h' \in \im \phi\), there exists \(g' \in
    \phi^{-1}(h')\) and \(\phi(g g') = \phi(g) \phi(g') = h h' \in \im \phi\) ---
    hence \(\phi\) is closed under products.
\end{proof}

\begin{definition}[Kernel]
    We define the kernel of a morphism of groups \(\phi \in \Hom_\Grp(G, H)\) as
    the collection \(\ker \phi = \{g \in G \colon \phi(g) = e_H\}\).
\end{definition}

\begin{lemma}[Kernel subgroup]
    \label{lem:kernel-subgroup}
    Let \(\phi: G \to H\) be a group morphism. The kernel \(\ker \phi \subseteq G\)
    is a subgroup of \(G\).
\end{lemma}

\begin{proof}
    Let \(g \in \ker \phi\) be any element, then since \(\phi(g^{-1}) = \phi(g)^{-1}
    = e_H^{-1} = e_H\), then \(g^{-1} \in \ker \phi\). Also, if \(u \in \ker \phi\)
    is another element, then \(\phi(g u) = \phi(g) \phi(u) = e_H e_H = e_H\) and
    hence \(g u \in \ker \phi\) --- thus \(\ker \phi\) is a subgroup of \(G\).
\end{proof}

\begin{proposition}[Monomorphisms, kernels and injectivity]
    \label{prop:monic-in-grp}
    Let \(\phi: G \to H\) be a morphism of groups. Then the following properties are
    equivalent:
    \begin{enumerate}[(a)]\setlength\itemsep{0em}
        \item \(\phi\) is a monomorphism in \(\Grp\).

        \item \(\ker \phi = \{e_G\}\).

        \item \(\phi\) is injective in \(\Set\).
    \end{enumerate}
\end{proposition}

\begin{proof}
    \begin{itemize}\setlength\itemsep{0em}
        \item (a) \(\implies\) (b). Suppose \(\phi\) is a monormophism and consider the
              following commutative diagram
              \[
                  \begin{tikzcd}
                      \ker \phi \ar[r, shift left, hook, "\iota"] \ar[r, shift right, swap, "e"]
                      & G \ar[r, tail, "\phi"] & H
                  \end{tikzcd}
              \]
              From the monomorphism definition we find \(\iota = e\) and therefore \(\ker \phi
              = \{e_G\}\).

        \item (b) \(\implies\) (c). If \(\ker\phi = \{e_G\}\), then, given elements \(g,
              g' \in G\) such that \(\phi(g) = \phi(g')\), then, multiplying both sides by
              \(\phi{(g')}^{-1}\) we get
              \[
                  \phi(g) {\phi(g')}^{-1} = \phi(g) \phi(g'^{-1}) = \phi(gg'^{-1}) = e_G.
              \]
              That is, \(gg'^{-1} \in \ker\phi\), but since \(\ker\phi\) is trivial, then
              \(gg'^{-1} = e_G\) and hence \(g = g'\) --- in other words, \(\phi\) is
              injective.

        \item (c) \(\implies\) (a). Let \(\phi\) be injective --- that is, for any pair
              of \emph{set}-functions \(f, g: A \rightrightarrows G\) from a set \(A\), then
              \(\phi f = \phi g\) if and only if \(f = g\). In particular, if we impose a
              group structure in the set \(A\) and that both \(f\) and \(g\) preserve the
              group structure of \(A\), we find that the injectivity in \(\Set\) implies in
              the monomorphism \(\phi\) as a morphism in \(\Grp\).
    \end{itemize}
\end{proof}

\begin{proposition}
    \label{prop:injective-morphism-abelian}
    Let \(G\) be a group and \(H\) be an abelian group. If there exists an injective
    group morphism \(\iota: G \mono H\), then \(G\) is abelian.
\end{proposition}

\begin{proof}
    Let \(x, y \in G\) be any pair of elements, then \(\phi(x y) = \phi(x) \phi(y) =
    \phi(y) \phi(x) = \phi(y x)\), thus \(x y = y x\) --- since \(\phi\) is
    injective.
\end{proof}

The following proposition is a trivial one, but it is a really useful tool to
prove the non-existence of non-trivial morphisms between certain groups --- it
relies on arguments based on the order of elements of each group.

\begin{proposition}[Morphisms and orders]\label{prop: grp-morph-order}
    Let \(G \in \Grp\) be a group admitting an element \(g \in G\) of finite order
    \(|g| \in \N\). Let \(H\) be any group and consider the morphism \(\phi: G \to
    H\). We have that \(|\phi(g)|\) divides the order of \(|g|\).
\end{proposition}

\begin{proof}
    Notice that \({\phi(g)}^{|g|} = \phi(g^{|g|}) = \phi(e_G) = e_H\), hence \(|g|\)
    is a multiple of the order of \(\phi(g) \in H\).
\end{proof}

\begin{example}
    Consider for example the collection of morphisms \(\Hom_\Grp(C_7, C_{15})\).
    Let \(\phi\) be any such morphism. Consider any element \(g \in C_7\) and
    recall that \(|g|\) must be a divisor of \(7\) --- see \cref{prop:
        ord-cyclic-elem}. On the other hand, if \(h \in C_{15}\) then \(|h|\) divides
    \(15\). From \cref{prop: grp-morph-order} we see that if \(\phi(g) = h\), then
    \(|h|\) must divide \(|g|\), but \(\gcd(7, 15) = 1\), hence
    \(|h| = 1\) --- that is \(h = e_H\) and \(\phi\) is the trivial morphism
    \(\phi(g) = e_H\) for all \(g \in C_7\). This shows that there is no
    non-trivial morphism between \(C_7\) and \(C_{15}\).
\end{example}

\subsection{Isomorphism of Groups}

\begin{proposition}[Isomorphisms are bijections]\label{prop: grp-iso-bij}
    Let \(\phi \in \Mor(\Grp)\). Then \(\phi\) is an isomorphism if and only if it
    is a bijection.
\end{proposition}

\begin{proof}
    Consider an isomorphism \(\phi: G \to H\). Using the forgetful functor \(F:
    \Grp \to \Set\) we see that \(F\phi\) is a bijection of sets in \(\Set\) ---
    recall \cref{lem: functor preserve iso} --- thus \(\phi\) defines a bijection
    between the elements of \(G\) and \(H\).

    On the other hand, if \(\phi\) is a bijection, consider its set-function
    inverse \({(F\phi)}^{-1}: H \to G\). We now show that \({(F\phi)}^{-1}\) preserves
    the structures of groups. Since \(\phi(e_G) = e_H\), then \({(F\phi)}^{-1}(e_H)
    = e_G\). Moreover, for any \(h, h' \in H\) --- since \(\phi\) is surjective
    --- take elements \(g, g' \in G\) such that \(\phi(g) = h\) and \(\phi(g') =
    h'\), then we find that \({(F\phi)}^{-1}(hh') = g g' = {(F\phi)}^{-1}(h) \cdot
    {(F\phi)}^{-1}(h')\). This implies in the existence of a naturally induced
    morphism of groups \(\phi^{-1}: H \to G\) defined by \(\phi^{-1}(h) =
    {(F\phi)}^{-1}(h)\). It is clear that \(\phi^{-1}\) is the right and left
    inverse of \(\phi\), thus \(\phi^{-1}\) is the inverse of \(\phi\) in \(\Grp\)
    and \(\phi\) is an isomorphism.
\end{proof}

\begin{definition}[Embedding]
    \label{def:grp-embedding}
    Let \(\phi: G \isoto H\) be an isomorphism of groups. We define the group \(\im
    \phi \subseteq H\) as an embedding of \(G\) on \(H\).
\end{definition}

\begin{proposition}\label{prop: iso-order-com}
    Let \(\phi: G \isoto H\) be an isomorphism of groups. Then:
    \begin{itemize}
        \setlength\itemsep{0em}
        \item For all \(g \in G\), we have \(|g| = |\phi(g)|\).
        \item \(G\) is commutative if and only if \(H\) is commutative.
    \end{itemize}
\end{proposition}

\begin{proof}
    Let \(g \in G\) be any element, then from \cref{prop: grp-morph-order} we have
    that \(|\phi(g)|\) divides \(|g|\). Since \(\phi^{-1}\) exists and is a
    morphism of groups, it also follows that \(|g|\) divides \(|\phi(g)|\) ---
    hence \(|g| = |\phi(g)|\).

    Let \(G\) be a commutative group and \(G \iso H\). Let \(h, h' \in H\) be any
    elements and consider \(\phi^{-1}(h) = g\) and \(\phi^{-1}(h') = g'\). From
    the structure preserving property of \(\phi\) we have
    \[
        h h' = \phi(g) \phi(g') = \phi(gg') = \phi(g'g) = \phi(g')\phi(g) = h' h.
    \]
    That is, \(H\) is commutative. The counter-implication is equivalent and will
    be omitted.
\end{proof}

\begin{lemma}[Inner automorphism]\label{lem: grp-inner-aut}
    Let \(G \in \Grp\). For each \(g \in G\), the map \(\gamma_g: G \to G\) given
    by \(\gamma_g(a) = gag^{-1}\) is an automorphism --- called inner
    automorphism of \(G\).
\end{lemma}

\begin{proof}
    Let \(g \in G\) be any element. Suppose \(a \in \ker\gamma_g\), then
    \(\gamma_g(a) = gag^{-1} = e_G\), hence, \(a = g^{-1} e_G g = g^{-1}g = e_G\),
    that is \(\ker\gamma_g = e_G\) --- \(\gamma_g\) is injective. Let \(g' \in G\)
    be any element, then, \(\gamma_g(g^{-1} g' g) = g (g^{-1} g' g) g^{-1} = g'\),
    that is, \(\gamma_g\) is surjective. We conclude that \(\gamma_g\) is a
    bijection --- hence an isomorphism, so \(\gamma_g \in \Aut_\Grp(G)\).
\end{proof}

\begin{lemma}[Inner automorphism correspondence]\label{lem: grp-inner-aut-cor}
    Let \(G \in \Grp\). The map \(\phi: G \to \Aut_\Grp(G)\) defined by the
    mapping \(\phi(g) = \gamma_g\) (where \(\gamma_g\) is defined in \cref{lem:
        grp-inner-aut}) is a morphism of groups.
\end{lemma}

\begin{proof}
    Let \(g, g' \in G\) be any elements, then
    \[
        \gamma_{gg'}(a) = (gg')a{(gg')}^{-1} = (g g') a (g'^{-1} g^{-1})
        = g(g' a g'^{-1}) g^{-1} = \gamma_g \gamma_{g'}(a).
    \]
    This being said, its easy to see that \(\phi\) preserves the group structure:
    \(\phi(gg') = \gamma_{gg'} = \gamma_g \gamma_{g'} = \phi(g) \phi(g')\). Thus
    \(\phi\) is a morphism of groups.
\end{proof}


\subsection{More Thoughts On Cyclic Groups}

We can now state the definition of a cyclic group in a formal manner, it goes as
follows:

\begin{definition}[Cyclic group]\label{def: cyclic-grp}
    A group \(G\) is said to be cyclic if \(G \iso \Z\) or \(G \iso \Z/n\Z\) for
    some \(n \in \N\).
\end{definition}

\begin{proposition}
    A finite group of order \(n \in \N\) is cyclic if and only if it contains an
    element of order \(n\).
\end{proposition}

\begin{proof}
    Let \(G\) be a cyclic group of order \(n\). Since \(G\) is finite, then there
    exists an isomorphism \(\phi: G \isoto \Z/n\Z\). Consider the element \(g =
    \phi^{-1}({[1]}_n) \in G\), from \cref{prop: iso-order-com} we see that \(|g| =
    n\).

    Let \(G\) be a finite group of order \(n\) and \(x \in G\) be such that \(|x|
    = n\). Let \(\phi: G \to \Z/n\Z\) be any morphism of groups sending \(x
    \mapsto {[1]}_n\). Consider the collection \(G' = \{e_G, x, x^2, \dots,
    x^{n-1}\} \subseteq G\), and notice that, together with the binary operation
    of \(G\), \(G'\) becomes a group of \(n\) elements --- that is, \(G' = G\) and
    every element \(g \in G\) can be written as \(g = x^k\) for some \(1 \leq k
    \leq n\). This implies in \(\phi\) injective --- thus a bijection. From
    \cref{prop: grp-iso-bij} we see that \(\phi\) is an isomorphism \(G \iso
    \Z/n\Z\) --- \(G\) is a cyclic group, which finishes our proof.
\end{proof}

\begin{proposition}
    \label{prop:order-cyclic-totient}
    The order of the cyclic group \(C_n\) is equal to \(\phi(n)\), where \(\phi\) is
    the Euler totient function --- that is, the number of positive integers less
    than \(n\) that are relatively prime to \(n\).
\end{proposition}

\begin{proof}
    Let \(x \in C_n\) be a generator of the group --- that is, \(x^n = e\) --- then
    for all \(d < n\) such that \(\gcd(d, n) = 1\) we have \(x^d \neq x\) and
    \((x^d)^n = e\) thus \(x^d\) is a generator of \(C_n\). Therefore the number
    of distinct elements of \(C_n\) is the same as the number of positive integers
    coprime of \(n\).
\end{proof}

\begin{proposition}[Subgroup]
    \label{prop:subgroup-of-cyclic-group}
    Any subgroup of a cyclic group is cyclic.
\end{proposition}

\begin{proof}
    Let \(G = \langle g \rangle\) be a cyclic group and \(H \subseteq G\) be any
    subgroup of \(G\). If \(h \in H\) then there exists \(n \in \Z\) such that \(h =
    g^n\), thus \(g^n \in H\) --- but then, given any other \(h' \in H\), there must
    exist \(m \in \Z\) such that \(h' = g^{n + m} = h^m\), thus \(H = \langle h
    \rangle\) is cyclic.
\end{proof}

\begin{lemma}
    \label{lem:inner-aut-subgroup}
    Given a group \(G\), the collection of inner automorphisms, which we'll denote
    by \(\Inn(G)\), is a subgroup of \(\Aut(G)\). Moreover, the following statements
    are equivalent:
    \begin{enumerate}[(a)]\setlength\itemsep{0em}
        \item \(\Inn(G)\) is cyclic.
        \item \(\Inn(G)\) is trivial.
        \item \(G\) is abelian.
    \end{enumerate}
    Therefore, if \(\Aut(G)\) is cyclic, the group \(G\) is abelian.
\end{lemma}

\begin{proof}
    From \cref{lem: grp-inner-aut} and \cref{lem: grp-inner-aut-cor} we find that
    \(\Inn(G) \subseteq \Aut(G)\) is indeed a subgroup. Now we prove the
    equivalences:
    \begin{itemize}\setlength\itemsep{0em}
        \item (a) \(\Leftrightarrow\) (b). If \(\Inn(G)\) is cyclic, then there must
              exist \(g \in G\) for which \(\Inn(G) = \langle \gamma_g \rangle\). Therefore,
              given any \(h \in G\), there exists \(n \in \Z\) for which \(\gamma_h =
              \gamma_g^n\) --- therefore \(h g h^{-1} = g^n g g^{-n} = g\), thus \(h g = g
              h\). We can thus conclude that \(\gamma_g\) is the identity morphism \(h \mapsto
              h\), which implies in \(\Inn(G)\) being trivial. Now, if we assume \(\Inn(G)\)
              to be trivial by hypothesis, then clearly \(\Inn(G)\) is cyclic.

        \item (b) \(\Leftrightarrow\) (c). If \(\Inn(G)\) is trivial, then given any
              pair of elements \(g, h \in G\) we have \(\gamma_g = \Id\) and thus \(g h g^{-1}
              = h\), which implies in \(g h = h g\), thus \(G\) is commutative. Now, if \(G\)
              is commutative by hypothesis, we find that \(g h g^{-1} = (g g^{-1}) h = h\)
              thus \(\Inn(G)\) is trivial.
    \end{itemize}
    Thus, if \(\Aut(G)\) is a cyclic group, we find that the subgroup \(\Inn(G)
    \subseteq \Aut(G)\) is also cyclic (from \cref{prop:subgroup-of-cyclic-group}),
    hence \(G\) is abelian.
\end{proof}

\subsection{Some Matrix Groups}

\begin{example}[Important matrix groups]
    \label{exp:important-matrix-groups}
    Let \(k\) be a field and \(\Mat_{n \times n}(k) = \End_{\Vect_k}(k^n)\) be the
    collection of all \(n \times n\) matrices over \(k\), for any \(n \in \Z_{>
        0}\). We define the following important groups of square matrices:
    \begin{enumerate}\setlength\itemsep{0em}
        \item General linear group: \(\GL_n(k) \coloneq \{T \in \Mat_{n \times n}(k)
              \colon \det T \neq 0\}\), the group of invertible matrices.

        \item Special linear group: \(\SL_n(k) \coloneq \{T \in \GL_n(k) \colon \det T =
              1\}\).

        \item Orthogonal group: \(\Orth_n(k) \coloneq \{T \in \GL_n(k) \colon T T^{*} =
              T^{*} T = \Id_n\}\)\footnote{The matrix \(T^{*}\) denotes the transpose of
                  \(T\), which is the same as the dual of \(T\).}.

        \item Special orthogonal group: \(\SO_n(k) \coloneq \{T \in \Orth_n(k) \colon
              \det T = 1\}\).

        \item Unitary group: \(\Unit_n(\CC) \coloneq \{T \in \GL_n(\CC) \colon T
              T^{\dag} = T^{\dag} T = \Id_n\}\)\footnote{The matrix \(T^{\dag}\) denotes the
                  complex transpose of \(T\).}.

        \item Special unitary group: \(\SU_n(\CC) \coloneq \{T \in \Unit_n(\CC) \colon
              \det T = 1\}\).

        \item Lie algebra of the general linear group: \(\lie{gl}_n(k) \subseteq
              \GL_n(k)\) such that
              \[
                  [T, M] \coloneq T M - M T \in \lie{gl}_n(k)
              \]
              for all \(T, M \in \lie{gl}_n(k)\).

        \item Lie algebra of the special linear group: \(\lie{sl}_n(k) \coloneq \{T \in
              \lie{gl}_n(k) \colon \Tr T = 0\}\).

        \item Lie algebra of the orthogonal group: \(\lie{o}_n(k) \coloneq \{ T \in
              \lie{gl}_n(k) \colon T + T^{*} = 0\}\)

        \item Lie algebra of the unitary group: \(\lie{u}_n(\CC) \coloneq \{T \in
              \lie{gl}_n(\CC) \colon T + T^{\dag} = 0\}\).

        \item Lie algebra of the special unitary group: \(\lie{su}_n(\CC) \coloneq \{T
              \in \lie{gl}_n(\CC) \colon \Tr T = 0\}\).
    \end{enumerate}
    The proof that such examples are indeed groups come immediately from \cref{prop:
        comp det} and \cref{def:dual-morphism}. Notice that all of the above groups are
    subgroups of the general linear group \(\GL_n(k)\). It should be noted, however,
    that the Lie groups presented do \emph{not} form a group under multiplication.
\end{example}

\begin{example}[Upper triangular matrices]
    \label{exp:upper-triangular-GL-subgroup}
    The collection of upper triangular matrices over a field \(k\) is a subgroup of
    \(\GL_n(k)\). Let \(A\) be any upper triangular matrix, then from definition
    \([a_{ij}]_{1 \leq i, j \leq n}\) is such that \(a_{ij} = 0\) for all \(i > j\)
    and \(\prod_{j=1}^n a_{jj} \neq 0\) --- therefore, any permutation \(\sigma \in
    S_n\) other than the identity will have at least one element \(i_0 > j\). Hence
    \[
        \det A = \sum_{\sigma \in S_n} \sign(\sigma) \prod_{j=1}^n a_{\sigma(j)\, j}
        = \prod_{j=1}^n a_{j j}
        \neq 0,
    \]
    which implies in \(A \in \GL_n(k)\). Moreover, since the sum of two upper
    triangular matrices is upper triangular, clearly the sum is in \(\GL_n(k)\) ---
    which proves that the collection of upper triangular matrices is a subgroup,
    since the existence of inverses is immediate.
\end{example}

\begin{example}
    \label{exp:SL_2(C)-form}
    Notice that any matrix of the form
    \begin{equation}\label{eq:matrix-form-SL_2(C)}
        A \coloneq
        \begin{bmatrix}
            a + bi   & c + di \\
            - c + di & a - bi
        \end{bmatrix}
    \end{equation}
    Is such that
    \begin{gather*}
        A A^{\dag} =
        \begin{bmatrix}
            a + bi   & c + di \\
            - c + di & a - bi
        \end{bmatrix}
        \begin{bmatrix}
            a - bi & -c - di \\
            c - di & a + bi
        \end{bmatrix}
        =
        \begin{bmatrix}
            a^2 + b^2 + c^2 + d^2 & 0                     \\
            0                     & a^2 + b^2 + c^2 + d^2
        \end{bmatrix},
        \\%
        A^{\dag} A =
        \begin{bmatrix}
            a - bi & -c - di \\
            c - di & a + bi
        \end{bmatrix}
        \begin{bmatrix}
            a + bi   & c + di \\
            - c + di & a - bi
        \end{bmatrix}
        =
        \begin{bmatrix}
            a^2 + b^2 + c^2 + d^2 & 0                     \\
            0                     & a^2 + b^2 + c^2 + d^2
        \end{bmatrix}.
    \end{gather*}
    Therefore, imposing \(a^2 + b^2 + c^2 + d^2 = 1\) in
    \cref{eq:matrix-form-SL_2(C)} makes \(A\) into a matrix of the group
    \(\SL_2(\CC)\). Furthermore, if \(T \in \GL_n(\CC)\) is a matrix with a form
    other than that of \cref{eq:matrix-form-SL_2(C)}, then it cannot be the case
    that \(\det T\) equal \(1\) --- thus, every \(\SL_2(\CC)\) matrix has the form
    of \(A\).
\end{example}

\begin{proposition}
    \label{prop:SL_2(Z)-generator-matrices}
    The group \(\SL_2(\Z)\) is generated by the pair matrices
    \[
        A \coloneq
        \begin{bmatrix}
            0 & -1 \\
            1 & 0
        \end{bmatrix}
        \quad
        \text{ and }
        \quad
        B \coloneq
        \begin{bmatrix}
            1 & 1 \\
            0 & 1
        \end{bmatrix}
    \]
\end{proposition}

\begin{proof}
    It should be noted that \(B^n = \big[ \begin{smallmatrix} 1 & n \\ 0 &
            1 \end{smallmatrix} \big]\) and \(A^2 = \big[ \begin{smallmatrix} -1 & 0 \\ 0 &
            -1 \end{smallmatrix} \big]\) --- thus, given any matrix \(X =
    \big[ \begin{smallmatrix} a & b \\ c & d \end{smallmatrix} \big] \in \SL_2(\Z)\)
    we have
    \[
        A X =
        \begin{bmatrix}
            -c & -d \\ a & b
        \end{bmatrix}
        \quad
        \text{ and }
        \quad
        B^n X =
        \begin{bmatrix}
            a + n c & b + n d \\ c & d
        \end{bmatrix}.
    \]
    If we assume that \(c \neq 0\), and \(|a| \geq |c|\) (otherwise, simply multiply
    from the left by \(A\)), we can apply the division algorithm to find \(q_0, r_0
    \in \Z\) for which \(a = q_0 c + r_0\), satisfying \(0 \leq r_0 \leq
    |c|\). Then we can apply
    \[
        A B^{-q_0}X = A
        \begin{bmatrix}
            r_0 & b - q_0 d \\ c & d
        \end{bmatrix}
        =
        \begin{bmatrix}
            - c & - d \\ r_0  & b - q_0 d
        \end{bmatrix} \coloneq X_0,
    \]
    Now, if \(r_0 \neq 0\), we can recursively apply the division algorithm (which
    I'll carry again just for the sake of clarity): we find \(q_1, r_1 \in \Z\) such
    that \(-c = q_1 r_0 + r_1\), where \(0 \leq r_1 \leq |r_0|\) --- we thus find a
    new power \(q_1\) for which
    \[
        A B^{-q_1} X_0 =
        \begin{bmatrix}
            -r_0 & q_0 d - b               \\
            r_1  & (q_0 q_1 - 1) d - q_1 b
        \end{bmatrix}
        = X_1.
    \]
    This process is ensured to terminate at some point with a zero lower left
    entry. Since the group is acting on the left of \(\SL_2(\Z)\) and the
    determinant is always zero, the resulting matrix will be of the form
    \(\big[
        \begin{smallmatrix}
            \pm 1 & m \\ 0 & \pm 1
        \end{smallmatrix}\big] \in \SL_2(\Z)\) --- which equals either \(B^m\) or
    \(-B^{-m}\). We conclude that there must exist some \(g \in \langle A, B
    \rangle\) such that \(g X = \pm B^t\) for some \(t \in \Z\) --- and since \(A^2
    = - I_2\), we obtain \(X = \pm g^{-1} B^t \in \SL_2(\Z)\) and \(\pm g^{-1} B^t
    \in \langle A, B \rangle\).
\end{proof}

\begin{proposition}[Union of subgroups]
    \label{prop:union-of-subgroups}
    Let \(G\) be any group, then:
    \begin{enumerate}[(a)]\setlength\itemsep{0em}
        \item Given subgroups \(H, Q \subseteq G\), the union \(H \cup Q\) is a subgroup
              of \(G\) if and only if either \(H \subseteq Q\) or \(Q \subseteq H\).

        \item If \(H_0 \subseteq H_1 \subseteq \dots\) is a collection of subgroups of
              \(G\), then the union \(\bigcup_{j \geq 0} H_j \subseteq G\) is a subgroup of
              \(G\).
    \end{enumerate}
\end{proposition}

\begin{proof}
    \begin{enumerate}[(a)]\setlength\itemsep{0em}
        \item If \(H \subseteq Q\) or \(Q \subseteq H\), then \(H \cup Q\) is clearly a
              subgroup. On the other hand, if \(H \cup Q\) is a subgroup, then for every \(h
              \in H\) and \(q \in Q\) we must have \(hq \in H \cup Q\) --- which implies that
              \(hq \in H\) or \(hq \in Q\), for the first case, we have \(h^{-1}(h q) = q
              \in H\) thus \(Q \subseteq H\), for the second case, \((h q) q^{-1} = h \in Q\)
              implying in \(H \subseteq Q\).

        \item Since the composition of elements is only defined for finitely many
              elements, the proposition follows immediately.
    \end{enumerate}
\end{proof}

\subsection{Group Products}

Let \((G, \cdot_G), (H, \cdot_H) \in \Grp\) be any objects. We define a
binary operation \(\cdot\,: (G \times H)^2 \to G \times H\) as the mapping
\begin{equation}\label{eq: grp-prod-bin}
    (g, h) \cdot (g', h') = (g \cdot_G g', h \cdot_H h').
\end{equation}
Such binary operation defines a group structure on \(G \times H\). Notice that,
given an element \((g, h) \in G \times H\), there exists an element \((g^{-1},
h^{-1}) \in G \times H\) such that \((g, h) \cdot (g^{-1}, h^{-1}) = (e_G,
e_H)\). Moreover, clearly \((e_G, e_H) \in G \times H\) is the identity element
of the structure. Hence \((G \times H, \cdot) \in \Grp\).

Also, the natural projections \(\pi_G: G \times H \to G\) and \(\pi_H: G \times
H \to H\) define morphisms of groups.

\begin{definition}[Direct product]
    Let \(\{G_{j}\}_{j \in J}\) be a collection of groups. We define the direct
    product of this family as the group \(\prod_{j \in J} G_j\) given by elements
    \((x_j)_{j \in J}\) such that \(x_j \in G_j\). The composition of elements of
    the direct product is defined component-wise, that is, if \((x_j)_{j \in J},
    (y_j)_{j \in J} \in \prod_{j \in J} G_j\), then \((x_j)_{j \in J} (y_j)_{j \in
            J} \coloneq (x_j y_j)_{j \in J}\). Moreover, inverses are also defined
    component-wise, \((x_j)_{j \in J}^{-1} \coloneq (x_j^{-1})_{j \in J}\).
\end{definition}

\begin{proposition}
    The direct products of groups are products on the category of groups,
    \(\Grp\). That is, for all group \(W\) and group morphisms \(f \in
    \Hom_\Grp(W, G)\) and \(g \in \Hom_\Grp(W, H)\), there exists a unique
    morphism \(\varphi \in \Hom_\Grp(W, G \times H)\) such that the following
    diagram commutes
    \[
        \begin{tikzcd}
            &W
            \ar[d, dashed, "\varphi"]
            \ar[ddr, bend left, "g"]
            \ar[ddl, swap, bend right, "f"]
            & \\
            &G \times H \ar[dr, swap, "\pi_H"] \ar[dl, "\pi_G"] & \\
            G & &H
        \end{tikzcd}
    \]
\end{proposition}

\begin{proof}
    We just take \(\varphi: W \to G \times H\) as the mapping \(w \xmapsto \varphi
    (f(w), g(w))\). We show that \(\varphi\) exists in \(\Grp\): let \(x, y \in
    W\) be any elements, then, since \(f\) and \(g\) are group morphisms, we find
    that
    \[
        \varphi(xy) = (f(xy), g(xy)) = (f(x) f(y), g(x) g(y))
        = (f(x), g(x)) (f(y), g(y)) = \varphi(x) \varphi(y).
    \]
    That is, \(\varphi\) is a group morphism. The uniqueness comes from the
    covariant functor \(F: \Grp \to \Set\), since \(\Set\) allows for products and
    hence the set-function \(F \varphi\) is unique.
\end{proof}

\begin{remark}
    We now show that if \(G, H \in \Grp\) are such that \(G \iso H \times G\),
    \emph{it does not follow that} \(H\) is trivial.
    Let \(G \coloneq \bigoplus_{j=0}^{\infty} \Z\) and \(H \coloneq \Z\) be abelian
    groups under the natural structure of addition. Notice that \(G \iso H \times
    G\) by the natural assignment of each element of \(G\) to itself in \(H \times
    G\). Since \(H\) is non-trivial, we found a counterexample.
\end{remark}

\begin{proposition}
    \label{prop:product-subgroups-isomorphism}
    Let \(G\) be a group and \(H, Q \subseteq G\) be subgroups for which \(H \cap Q
    = e\) and \(H Q = G\) --- that is, for every \(g \in G\), there exists \(h \in
    H\) and \(q \in Q\) such that \(g = h q\) --- we also impose that \(h q = q h\)
    for every \(h \in H\) and \(q \in Q\). Then, the morphism of groups \(H \times Q
    \isoto G\) defined by the mapping \((h, q) \mapsto hq\) is an isomorphism.
\end{proposition}

\begin{proof}
    Notice that \((h, q)(h', q') = (h h', q q') \mapsto (hh')(qq') = (hq)(h'q')\)
    thus the map is indeed a morphism of groups. Moreover, since every element of
    \(G\) can be written as a product \(HQ\), it follows that the map is
    surjective. Now, let \((h, q)\) be in the kernel of the morphism, then \(hq =
    e\) which implies in \(h = q^{-1}\) but then \(h \in H \cap Q\) and by
    hypothesis \(h = e\) --- thus the morphism is injective.
\end{proof}

\section{Quotient Groups --- The Birth of Normal Subgroups}

\subsection{Cosets}

\begin{definition}[Coset]
    \label{def:coset}
    Let \(G\) be a group and \(H\) be a subgroup of \(G\). Given any \(g \in G\), a
    left coset of \(H\) in \(G\) induced by \(g\) and denoted by \(g H\) is a set
    whose elements have the form \(g h\) for each \(h \in H\). A right coset of
    \(H\) in \(G\) induced by \(g\) is denoted \(H g\) and is a set consisting of
    elements of the form \(h g\) for each \(h \in H\). An element of a coset is
    commonly called \emph{coset representative}.
\end{definition}

\begin{corollary}
    Let \(G\) be a group and \(H\) be subgroup of \(G\), then, for every \(h \in
    H\), we have
    \[
        h H = H h = H.
    \]
\end{corollary}

\begin{proof}
    Notice that, given \(x \in H\), the element \(h (h^{-1} x) = x \in h H\), thus
    \(H \subseteq h H\), on the other hand, it's clear that \(hH \subseteq H\),
    since \(hH\) is composed of product of elements of \(H\), which itself is closed
    under products. The same analogous proof goes for \(Hh\) so I won't bother to
    write it down.
\end{proof}

\begin{corollary}[Equal cosets]
    \label{cor:equal-cosets}
    Let \(G\) be a group and \(H \subseteq G\) be a subgroup. Given \(x, y \in G\),
    if the cosets \(x H\) and \(y H\) share any common element, then \(x H = y H\).
\end{corollary}

\begin{proof}
    Let \(g \in xH \cap yHk\), then there exists \(x h \in xH\) and \(y h' \in yH\)
    such that \(x h = g = y h'\), then, in particular, \(x = y h' h^{-1}\) moreover,
    since \(H\) is a subgroup, it is clear that \(h' h^{-1} \in H\) then \(x H = (y
    h' h^{-1})H = y(h' h^{-1})H = y H\).
\end{proof}

\begin{definition}[Index]
    \label{def:grp-index}
    Let \(G\) be a group and \(H \subseteq G\) be a subgroup. The number of left
    cosets of \(H\) in \(G\) is denoted by \([G \colon H]\), which will be commonly
    referred to as the \emph{index} of \(H\) in \(G\).
\end{definition}

\begin{corollary}
    \label{cor:order-as-index}
    If we denote by \(*\) the trivial group, the \emph{order} of a group
    \(G\) is the same as \([G \colon *]\) --- that is, \(|G| = [G \colon *]\).
\end{corollary}

\begin{proof}
    One can view the trivial group \(*\) as a subgroup of \(G\) containing only the
    identity. Notice that the number of left cosets of \(*\) will be exactly the
    number of elements of \(G\), that is \([G \colon *] = |G|\).
\end{proof}

\begin{proposition}
    \label{prop:grp-index-subgroup}
    Let \(G\) be a group, and \(H \subseteq G\) be a subgroup, and \(Q \subseteq H\)
    be a subgroup. Then, if any two of the quantities \(\{[G \colon H], [H \colon
            Q], [G \colon Q]\}\) is finite, the third is also finite and the following
    equality holds
    \[
        [G \colon H] [H \colon Q] = [G \colon Q].
    \]
\end{proposition}

\begin{proof}
    Let \(\{x_{i}\}_{i \in I} \subseteq H\) be coset representatives of \(Q\), and
    \(\{y_{j}\}_{j \in J} \subseteq G\) be coset representatives of \(H\) --- that
    is, each one of the collections \(\{x_i Q\}_{i \in I}\) and \(\{y_{j} H\}_{j \in
    J}\) have pairwise disjoint elements, and \(H = \bigcup_{i \in I} x_i Q\), and
    \(G = \bigcup_{j \in J} y_j H\). Then we have that \(G = \bigcup_{(i, j) \in I
        \times J} y_j x_i Q\), and our goal will be to prove that \(y_j x_i Q \cap
    y_{i'} x_{j'} Q = \emptyset\). Suppose on the contrary that their intersection
    is non-empty, which by \cref{cor:equal-cosets} implies \(y_j x_i Q = y_{j'}
    x_{i'} Q\). Since \(x_j, x_{j'} \in H\), we have \(y_j x_i Q H = y_j x_i H = y_j
    H\) and analogously \(y_{j'} x_{i'} Q H = y_{j'} H\) --- thus \(y_j H = y_{j'}
    H\), which implies in \(y_j = y_{j'}\). This shows that the collection \(\{y_{j}
    x_i\}_{(i, j) \in I \times J} \subseteq G\) are coset representatives for \(Q\)
    and therefore \([G \colon H] [H \colon Q] = [G \colon Q]\).
\end{proof}

\begin{corollary}[Lagrange's theorem]
    \label{cor:order-subgroup-divides-order-group}
    Let \(G\) be a finite group, then the order of any subgroup \(H\) of \(G\)
    divides the order \(|G|\).
\end{corollary}

\begin{proof}
    Since \(|G| = [G \colon *]\) is finite, any subgroup \(H\) of \(G\) is also
    finite and therefore \([G \colon H] [H \colon *] = [G \colon *]\), which is
    exactly the same as \([G \colon H] |H| = |G|\).
\end{proof}

\begin{example}[Prime order]
    \label{exp:grp-prime-order-cyclic}
    Let \(G\) be a group with order \(|G| \coloneq p\) prime. Choose any \(g \in G\)
    with \(g \neq e\), and consider the subgroup \(H \coloneq \langle g
    \rangle\). From \cref{prop:grp-index-subgroup} we find that \([G \colon H] |H| =
    p\), hence \(|H|\) divides \(p\), but since \(|H| \leq p\), then \(|H| = p\) and
    therefore \(H = G\). This implies that any non-identity element of \(G\)
    generates the whole group, which is the same as to say that \(G\) is cyclic.
\end{example}

\subsection{Normal Subgroups}

\begin{definition}[Normal subgroup]
    \label{def:normal-subgroup}
    Let \(G\) be a group. We define a \emph{normal subgroup} to be the kernel of
    some morphism of groups in \(\Hom_{\Grp}(G, -)\)\footnote{In this case, we are
        using \(\Hom_{\Grp}(G, -)\) to denote the same as the collection of all group
        morphisms whose source is \(G\).}. In other words, a subgroup \(N \subseteq G\)
    is normal if there exists a morphism of groups \(\phi: G \to H\), for some group
    \(H\), for which \(\ker \phi = N\).
\end{definition}

\begin{definition}[Quotient group]
    \label{def:quotient-group}
    Let \(G\) be a group and \(N\) be a normal subgroup of \(G\). We denote by
    \(G/N\) the collection of all left cosets of \(N\) in \(G\), on the other hand,
    \(G \backslash N\) denotes the collection of all right cosets of \(N\) in
    \(G\). Moreover, we view \(G/N\) (and \(G \backslash N\)) as groups where:
    \begin{itemize}\setlength\itemsep{0em}
        \item The product of two cosets \(x N\) and \(y N\) (or, respectively, \(N
              x\) and \(N y\)) is given by \((x N) (y N) \coloneq (x y) N\) which is again a
              left coset in \(G/N\) (conversely, \((N x) (N y) \coloneq N (x y) \in G
              \backslash N\)).
        \item Given any coset \(x N\) (respectively, \(N x\)), its inverse is given by
              \(x^{-1} N\) (respectively, \(N x^{-1}\)).
        \item The identity of the group is \(N\).
    \end{itemize}
    The group \(G/N\) is commonly referred to as the \emph{quotient group} of \(G\)
    by \(H\).
\end{definition}

\begin{proposition}
    \label{prop:normal-subgroup-equivalence}
    Let \(G\) be a group. A subgroup \(N \subseteq G\) is normal if and only if, for
    every \(g \in G\), we have \(gNg^{-1} = N\).
\end{proposition}

\begin{proof}
    First, suppose that \(N\) is a normal group and \(\phi: G \to H\) is a morphism
    such that \(\ker \phi = N\), then if \(g \in G\) is any element, we see that for
    any \(n \in N\) we have \(\phi(g n) = \phi(g) \phi(n) = \phi(g)\) and
    analogously \(\phi(n g) = \phi(n) \phi(g) = \phi(g)\), thus in general \(g N = N
    g = \phi^{-1}(\phi(g))\). Note that if we multiply both groups on the right by
    \(g^{-1}\) we get \(g N g^{-1} = N\), as wanted.

    On the other hand, let \(N \subseteq G\) be a subgroup such that \(g N g^{-1} =
    N\) for every \(g \in G\), then in particular left cosets are equal to right
    cosets because, multiplying on the right by \(g\) we obtain \(g N = N
    g\). Consider the group of left cosets \(G/N\) (since right and left cosets are
    equivalent in this specific case, we could also have considered \(G \backslash
    N\)) and define the morphism of groups \(\pi: G \to G/N\) by the mapping
    \(\pi(g) = g N\). Notice that if \(g \in \ker \pi\), then \(g N = N\), which
    implies that \(g \in N\), moreover, if \(n \in N\) is any element, then
    \(\pi(n) = n N = N\) --- thus \(\ker \pi = N\).
\end{proof}

In fact the morphism \(\pi: G \epi G/N\), defined above by the map \(g \xmapsto
\psi g N\), is so important we are even going to distinguishably call it the
\emph{canonical projection map} of \(G\) onto the factor group \(G/N\). It is
trivial that such canonical projection \(\pi\) is surjective.

\begin{corollary}
    \label{prop:normal-and-inner-automorphism-group}
    Let \(G\) be a group and \(H \subseteq G\) be a subgroup. Then \(H\) is normal
    in \(G\) if and only if, for all given \(\gamma \in \Inn(G)\), we have
    \(\gamma(H) \subseteq H\). Therefore, the morphism \(\Inn(G) \to \Aut(H)\)
    mapping \(\gamma_g \mapsto \gamma_g|_H\) is well defined.
\end{corollary}

\begin{corollary}[Intersection of normal subgroups is normal]
    \label{cor:intersection-normal-subgroups}
    Let \(\{N_{j}\}_{j \in J}\) be any collection of normal subgroups of a given
    group \(G\). Then \(N \coloneq \bigcap_{j \in J} N_j\) is a normal subgroup of
    \(G\).
\end{corollary}

\begin{proof}
    Let \(n \in N\) and \(g \in G\) be any two elements, then \(n \in N_j\) for all
    \(j \in J\) and from the normal condition we obtain that \(g n g^{-1} \in N_j\)
    for all \(j \in J\) as well --- which implies in \(g n g^{-1} \in N\).
\end{proof}

\begin{proposition}
    \label{prop:index-2-is-normal}
    If \(G\) is a group and \(N \subseteq G\) is a subgroup with index
    \([G \colon N] = 2\), then \(N\) is normal in \(G\).
\end{proposition}

\begin{proof}
    Suppose \(N \subseteq G\) is a subgroup with index \(2\) --- that is, \(G/N\)
    consists only of two cosets. Thus there must exist \(g \in G\) such that
    \(g N \neq N\) and consequently \(N g \neq N\) --- moreover, since
    \(G = N \cup g N = N \cup N g\), one concludes that \(g N = N g\).
\end{proof}

\begin{definition}[Normal closure]
    \label{def:normal-closure}
    Given a group \(G\) and a subset \(S \subseteq G\), the \emph{normal closure} of
    \(S\) is the subgroup of \(G\) generated by all elements of the form \(g^{-1} s
    g\) --- for \(g \in G\) and \(s \in S\).
\end{definition}


\subsection{Quotient Group Properties}

We now study the properties of factorization of maps between groups and
sequences of maps from the viewpoint of quotientings. One interesting immediate
example is when \(H \subseteq G\) is a subgroup of a group \(G\), then
\[
    \begin{tikzcd}
        H \ar[r, tail, "\iota"] &G \ar[r, two heads, "\pi"] &G/H
    \end{tikzcd}
\]
which is a short exact sequence. Moreover, given any groups \(G\), \(Q\) and
\(K\), if we have an exact sequence
\[
    \begin{tikzcd}
        * \ar[r] &Q \ar[r, "f"] &G \ar[r, "g"] &K \ar[r] &*
    \end{tikzcd}
\]
we can conclude that \(f\) is injective, and \(g\) is surjective --- this comes
from the fact that \(\ker f = e_Q\) and, since the kernel of \(K \to 0\) is the
whole group \(K\), we necessarily have \(g(G) = K\). Moreover, if we define \(H
\coloneq \ker g\), there exists a natural identification
\[
    \begin{tikzcd}
        * \ar[r]
        &Q \ar[d, swap, "\iso"] \ar[r, tail, "f"]
        &G \ar[r, two heads, "g"] \ar[d, "\iso"]
        &K \ar[r] \ar[d, "\iso"]
        &* \\
        * \ar[r] &H \ar[r] &G \ar[r] &G/H \ar[r] &* \\
    \end{tikzcd}
\]

\begin{proposition}[Universal property of quotient groups]
    \label{prop:universal-property-quotients-grp}
    Let \(G\) be a group and \(N\) be a normal subgroup of \(G\). Then, given a
    group \(Q\) together with a morphism of groups \(\phi: G \to Q\) such that
    \(N\) is the kernel of \(\phi\), there exists a unique morphism
    \(\phi_{*}: G/N \to Q\) such that the diagram
    \[
        \begin{tikzcd}
            G \ar[d, two heads, swap, "\pi"] \ar[r, "\phi"] &Q \\
            G/N \ar[ur, dashed, swap, bend right, tail, "\phi_{*}"] &
        \end{tikzcd}
    \]
    is commutative. Moreover, the map \(\phi_*\) is injective. The morphism
    \(\phi_{*}\) induces an isomorphism of groups
    \(\overline{\phi}: G/N \isoto \im \phi\) and therefore we have the following
    factorization
    \[
        \begin{tikzcd}
            G \ar[d, two heads] \ar[r, "\phi"] &Q \\
            G/N \ar[r, "\overline{\phi}"', "\dis"]
            &\im \phi \ar[u, hook]
        \end{tikzcd}
    \]
\end{proposition}

\begin{proof}
    Define \(\phi_{*}\) simply as the mapping \(gH \to \phi(g)\), then clearly
    \(\phi_{*} \pi = \phi\). Moreover, given any \(x \in G/N\), the fiber
    \(\pi^{-1}(x)\) is non-empty, thus \(\phi_{*}\) cannot assume any other values
    beside those specified by \(\phi\), which implies in its uniqueness. Moreover,
    since \(\ker \phi = N\), if \(n \in N\) then
    \(\phi_{*} \pi(n) = \phi_{*}(n N) = e_{Q}\) but \(n N = N\) thus
    \(N \in \ker \phi_{*}\), moreover, if \(x N \in \ker \phi_{*}\) it follows that
    \(\phi(x) = e_Q\) then \(x \in N\) and therefore \(x N = N\) --- this implies
    that \(\ker \phi_{*} = N\), which is the identity element of \(G/N\), thus
    \(\phi_{*}\) is injective.
\end{proof}

\begin{corollary}[First isomorphism]
    \label{cor:first-iso-grp}
    Let \(\phi: G \epi H\) be a \emph{surjective} morphism of groups. There exists a
    canonical isomorphism of groups
    \[
        G/{\ker \phi} \iso H.
    \]
\end{corollary}

\begin{proof}
    By the universal property, the induced map \(\phi_{*}: G/{\ker \phi} \mono H\)
    is already injective. From hypothesis, \(\phi\) being surjective implies that
    \(\phi_{*}\) is surjective since \(\phi = \pi \phi_{*}\).
\end{proof}

\begin{corollary}
    \label{cor:universal-property-quotients-grp}
    Let \(G\) be a group and \(H\) be a subgroup of \(G\). Define \(S\) to be the
    subgroup of \(G\) consisting of the intersection of all \emph{normal} subgroups
    of \(G\) containing \(H\). Then \(S\) is normal in \(G\) and is the smallest
    normal subgroup of \(G\) containing \(H\). Let \(Q\) be a group and
    \(\phi: G \to Q\) be a morphism of groups such that \(H \subseteq \ker
    \phi\). Then we have \(S \subseteq \ker \phi\), and there exists a \emph{unique}
    morphism \(\phi_{*}: G/S \mono Q\) such that the following diagram commutes
    \[
        \begin{tikzcd}
            G \ar[d, two heads, swap, "\pi"] \ar[r, "\phi"] &Q \\
            G/S \ar[ur, dashed, swap, bend right, tail, "\phi_{*}"] &
        \end{tikzcd}
    \]
    moreover, \(\phi_{*}\) is \emph{injective}.
\end{corollary}

\begin{proof}
    As before, we just define \(\phi_{*}(x S) \coloneq \phi(x)\). The commutativity
    follow from construction, uniqueness follows from the universal
    property. Moreover, \(S\) is given by the intersection of arbitrarily many
    normal subgroups of \(G\) containing \(H\), thus in particular \(H \subseteq S\)
    and, for every \(g \in G\), we have \(g S g^{-1} = S\) --- thus \(S\) is indeed
    normal.
\end{proof}

\begin{corollary}[Third isomorphism]
    \label{cor:quotient-isomorphism}
    Let \(G\) be a group and \(H \subseteq G\) be a \emph{normal} subgroup of
    \(G\). Let \(Q \subseteq G\) be a subgroup \emph{containing} \(H\). Then,
    \(Q/H\) is normal in \(G/H\) if and only if \(Q\) is normal in \(G\), if that is
    the case, then we have a canonical isomorphism
    \[
        \frac{G/H}{Q/H} \iso G/Q.
    \]
\end{corollary}

\begin{proof}
    Suppose \(Q\) is normal in \(G\). Since \(H\) is contained in \(Q\), and
    \(\ker(G \epi G/Q) = Q\), we can apply the universal property of quotients
    \cref{cor:universal-property-quotients-grp} to the subgroup \(H\) of the kernel
    and find a unique induced injective morphism \(G/H \mono G/Q\) --- whose kernel,
    on the other hand, is \(Q/H\), thus \(Q/H\) is normal in \(G/H\).

    For the converse, let \(Q/H\) be normal in \(G/H\), and consider the morphism
    given by the composition of the canonical projections
    \[
        \begin{tikzcd}
            G \ar[r, two heads] &G/H \ar[r, two heads] &\frac{G/H}{Q/H}
        \end{tikzcd}
    \]
    which has a kernel given by \(Q\) --- thus \(Q\) is normal in \(G\). With that,
    using \cref{cor:universal-property-quotients-grp}, the induced map is the wanted
    canonical isomorphism.
\end{proof}

In the context of the last corollary, we can visualize the propositions by the
following diagram, which commutes
\[
    \begin{tikzcd}
        * \ar[r] &Q \ar[r, tail] \ar[d, "\iso"']
        &G \ar[r, two heads] \ar[d, "\iso"]
        &G/Q \ar[r] \ar[d, "\Id"] &*
        \\
        * \ar[r] &Q/H \ar[r, tail] &G/H \ar[r, two heads] &G/Q \ar[r] &*
    \end{tikzcd}
\]

\begin{corollary}[Second isomorphism]
    \label{cor:grp-intersection-coset-isomorphism}
    Let \(G\) be a group, and let \(H\) and \(Q\) be subgroups of \(G\) such that
    \(H \subseteq N_G(Q)\) --- that is, \(H\) is contained in the normalizer of
    \(Q\). Then we have the following canonical isomorphism
    \[
        \frac{H}{H \cap Q} \iso \frac{H Q}{Q}.
    \]
\end{corollary}

\begin{proof}
    Since \(H \subseteq N_G(Q)\), then \(hQh^{-1} = Q\) for every \(h \in H\), so
    that \(Q\) is normal in \(H\). Since both \(H\) and \(Q\) are normal in \(H\),
    their intersection \(H \cap Q\) is also normal in \(H\). Moreover, \(HQ = QH\)
    and, given \(hq \in HQ\), the element \(q^{-1} h^{-1} \in Q H = H Q\) exists, thus
    \(HQ\) is closed under inverses, and clearly closed under products, thus \(HQ\)
    is a subgroup of \(G\). Consider now the surjective morphism \(H \epi H Q / Q\)
    given by the mapping \(h \mapsto h Q\) --- whose kernel is \(H \cap Q\), and
    therefore, by the quotient universal property, there exists a unique injective
    morphism \(H/(H \cap Q) \mono HQ/Q\), which is also surjective by the
    construction of \(H \epi H Q / Q\).
\end{proof}

\begin{proposition}[Morphism preimage preserve normality]
    \label{prop:morphisms-preserve-normality}
    Let \(G\) and \(H\) be groups and \(N \subseteq H\) be a normal subgroup. If
    \(f: G \to H\) is a morphism, then \(f^{-1}(N) \subseteq G\) is a normal
    subgroup of \(G\).
\end{proposition}

\begin{proof}
    Let \(x \in G\) and \(y \in f^{-1}(N)\) be any two element of \(G\). We have
    \(f(x y x^{-1}) = f(x) f(y) f(x)^{-1} \in N\), thus \(x y x^{-1} \in f^{-1}(N)\).
\end{proof}

Notice that the previous proposition gets us the following commutative diagram
\[
    \begin{tikzcd}
        G \ar[r, "f"] &H \ar[r, two heads, "\pi"] &H/N \\
        f^{-1}(N) \ar[u, hook] \ar[r, "f"] &N \ar[u, hook]
    \end{tikzcd}
\]
Moreover, since \(\ker \pi f = f^{-1}(N)\), we can use the universal property to
obtain
\[
    \begin{tikzcd}
        G \ar[d, two heads] \ar[r, "\pi f"] &H/N \\
        G/f^{-1}(N) \ar[ru, swap, tail, bend right, dashed, "\phi"] &
    \end{tikzcd}
\]
If we now impose surjectivity to the morphism \(f\), we find that \(\phi\) must
also be surjective, giving rise to an isomorphism \(G/f^{-1}(N) \iso H/N\). The
last two diagrams can be encoded in a single commutative diagram:
\[
    \begin{tikzcd}
        0 \ar[r]
        &f^{-1}(N) \ar[d, "f"] \ar[r, tail]
        &G \ar[d, "f"] \ar[r, two heads]
        &G/f^{-1}(N) \ar[r] \ar[d, dashed, tail, "\phi"]
        &0 \\
        0 \ar[r]
        &N \ar[r, tail]
        &H \ar[r, two heads]
        &H/N \ar[r]
        &0
    \end{tikzcd}
\]

\subsection{Centralizers \& Normalizers}

\begin{definition}[Centralizers and normalizers]
    \label{def:normalizer-centralizer}
    Let \(G\) be a group and \(S \subseteq G\) be any set of elements. We define the
    following groups:
    \begin{enumerate}[(a)]\setlength\itemsep{0em}
        \item The \emph{normalizer} of \(S\) is defined as the group
              \(N(S) \coloneq \{g \in G \colon g S g^{-1} = S\}\).
        \item A \emph{centralizer} of \(S\) is defined as the group
              \(Z(S) \coloneq \{g \in G \colon g s g^{-1} = s \text{, for all } s \in
              S\}\). The centralizer \(Z(G)\) is commonly called the \emph{center} of \(G\).
    \end{enumerate}
\end{definition}

The normalizer and centralizer are indeed groups, notice that if \(g \in N(S)\),
then \(g^{-1} \in N(S)\) since \(g S g^{-1} = S\) it follows, by multiplying on
the left by \(g^{-1}\) and on the right by \(g\), that
\(S = g^{-1} S g\). Moreover, given any two elements \(g, h \in N(S)\), we have
\[
    (g h)^{-1} N (g h) = (h^{-1} g ^{-1}) N (g h) = (h^{-1}N h) (g^{-1} N g) = S.
\]
For the case of the centralizer the proof is analogous as the one just made for
normalizers, where instead of \(S\) we would be considering any element
\(s \in S\).

In the following lemmas, let \(G\) be a group and \(H\) be a subgroup of \(G\).

\begin{lemma}
    If \(Q \subseteq G\) is any subgroup containing \(H\), and \(H\) is normal in
    \(Q\), then \(Q \subseteq N_G(H)\) --- that is, the normalizer \(N_G(H)\) of
    \(H\) is the largest subgroup of \(G\) such that \(H\) is normal.
\end{lemma}

\begin{proof}
    Let \(K\) be a group and \(\phi: Q \to K\) be a morphism of groups such that
    \(\ker \phi = H\), then given any \(q \in Q\), and any \(h \in H\), we have
    \(\phi(q h q^{-1}) = \phi(q) \phi(h) \phi(q)^{-1} = \phi(q) \phi(q)^{-1} = e_K\)
    --- thus \(q h q^{-1} \in H\) and since \(q H q^{-1} = H\) from hypothesis, then
    \(q \in N(H)\). The last statement follows clearly from the proposition.
\end{proof}

\begin{lemma}
    If \(Q \subseteq N_G(H)\) is a subgroup, then \(Q H\) is a group and \(H\) is
    normal in \(Q H\).
\end{lemma}

\begin{proof}
    First we verify that \(Q H\) is indeed a group. Let \(q h \in Q H\) be any
    element. Since \(Q \subseteq N_G(H)\), then \(Q H Q^{-1} = H\) and therefore
    \(Q H = H Q\) --- hence there exists \(q' h' \in Q H\) such that
    \(q' h' = q^{-1} h^{-1}\) so that \(Q H\) is closed under inverses. Moreover,
    given any two elements \(q h, q' h' \in Q H\), we have
    \((q h)(q' h') = q (h q') h'\), but since there exists \(q'' h'' \in Q H\) such
    that \(q'' h'' = h q'\) then
    \(q(h q') h' = q (q'' h'') h' = (q q'') (h'' h') \in Q H\) --- thus \(Q H\) is
    also closed under products, and therefore \(Q H\) is a group.

    Let \(q h \in Q H\) be any element and consider the group
    \((q h) H (h^{-1} q^{-1})\). Given any element \(h' \in H\), we have that
    \((q^{-1} h^{-1}) h' (q h) \in H\), thus
    \((q h) [(q^{-1} h^{-1}) h' (q h)] (h^{-1} q ^{-1}) = h'\) thus
    \(H \subseteq (q h) H (h^{-1} q^{-1})\). Moreover, clearly
    \((q h) H (h^{-1} q^{-1}) \subseteq H\) thus the equality holds --- which
    implies that \(H\) is normal on the group \(Q H\).
\end{proof}

\subsection{Epimorphisms}

We lay out the most difficult part of \cref{prop:epic-in-grp} in the following
lemma.

\begin{lemma}[Epimorphisms are surjections]
    \label{lem:epimorphism-is-surjective-grp}
    In the category of groups, an epimorphism is a surjective set-function.
\end{lemma}

\begin{proof}
    Let \(\phi: H \epi G\) be an epimorphism in \(\Grp\) --- our goal will be to
    prove that \(\im \phi = G\).

    Suppose, for the sake of contradiction, that \([G \colon \im \phi] = 2\) --- so
    that, by \cref{prop:index-2-is-normal} \(\im \phi\) is normal in \(G\). Now, if
    we consider the canonical projection and the trivial morphism
    \(\pi, 0: G \rightrightarrows G/{\im \phi}\) we see that \(\pi \phi = 0 \phi\)
    --- however, since \(\phi\) is epic, this would imply in \(\pi = 0\) which
    cannot possibly be true. Hence \([G \colon \im \phi] \geq 3\) necessarily and
    thus \(\im \phi\) is .

    Define the \emph{set}-function \(\sigma: G/{\im \phi} \to G/{\im \phi}\) to be a
    permutation of the cosets of \(G/{\im \phi}\) such that
    \(\sigma(\im \phi) \coloneq \im \phi\) is the \emph{only} fixed point of
    \(\sigma\) --- which can occur only because \([G \colon \im \phi] > 2\). We now
    consider the \emph{set}-functions given by the canonical projection
    \(\pi: G \epi G/{\im \phi}\) and a map \(\eta: G/{\im \phi} \to G\) defined so
    that \(\pi \eta = \Id_{G/{\im \phi}}\) and \(\eta(\im \phi) \coloneq e_G\) ---
    which is possible since \(\pi\) is surjective and therefore has a right-inverse.

    One can define a \emph{set}-function \(\alpha: G \to \im \phi\) for which every
    \(g \in G\) can be written uniquely as
    \begin{equation}\label{eq:write-uniquely-g}
        g = \alpha(g) \eta \pi(g).
    \end{equation}
    Since \(G\) is a group, the existence of \(\alpha\) is ensured --- simply define
    \(\alpha(g) \coloneq g (\eta \pi(g))^{-1}\).  The unicity of \(\alpha\) comes
    from the fact that if \(\beta: G \to \im \phi\) is another map such that
    \(g = \beta(g) \eta \pi(g)\), then by the injectivity of \(\eta\) we conclude
    that \(\beta = \alpha\). In the special case of \(g \in \im \phi\) then
    \(\eta\pi(g) = \eta(\im \phi) = e_G\), and thus \(\alpha(g) \coloneq g\).

    Define \(\lambda: G \to G\) to be the \emph{set}-function given by
    \(\lambda(g) \coloneq \alpha(g) \eta (\sigma \pi(g))\), for all \(g \in
    G\). Notice that \(\lambda\) is nothing but a permutation on \(G\) since every
    element of \(G\) is uniquely written as in \cref{eq:write-uniquely-g} --- and
    \(\sigma\) merely permutes the cosets of \(G/\im \phi\). Let \(P\) be the group
    of all permutations of \(G\) and consider group morphisms
    \(k, \ell: G \rightrightarrows P\) defined by
    \begin{gather}
        \label{gath:k-definition-gx}
        k(g)(x)  \coloneq  g x, \text{ for all } g, x \in G, \\
        \label{gath:ell-definition-g}
        \ell(g) \coloneq \lambda^{-1} k(g) \lambda, \text{ for all } g \in G.
    \end{gather}
    Indeed, given \(g, g' \in G\) we have
    \begin{gather*}
        k(g g')(x) = (g g') x = g (g' x) = k(g)(k(g')(x)), \\
        \ell(g g') = \lambda^{-1} k(g g') \lambda
        = \lambda^{-1} k(g) k(g') \lambda = \ell(g) \ell(g')
    \end{gather*}
    thus both are group morphisms. The condition for \(k(g) = \ell(g)\) is the same
    as \(\lambda k(g) = k(g) \lambda\), in turn if \(x \in G\) is any element,
    the requirement that
    \begin{equation}\label{eq:k=l-any-x-in-G}
        \lambda(g x) = g \lambda(x)
    \end{equation}
    is also an equivalent condition.

    For the case where \(g \in \im \phi\) and \(x \in G\) is any element, one has
    \(\pi(g x) = \pi(g) \pi(x) = \pi(x)\) and \(\alpha(g x) = g
    \alpha(x)\). Therefore,
    \[
        \lambda(g x)
        = \alpha(g x) \eta(\sigma \pi(g x))
        = g \alpha(x) \eta(\sigma \pi(x))
        = g \lambda(x).
    \]


    Using \cref{eq:k=l-any-x-in-G} one concludes that \(k(g) = \ell(g)\) for all
    \(g \in \im \phi\) --- which in turn implies in \(k \phi = \ell \phi\). Note
    however that since \(\phi\) is an epimorphism of groups by hypothesis, we
    conclude that \(k = \ell\) in general --- thus \(\lambda(g x) = g \lambda(x)\)
    is true for all choices of \(g, x \in G\). Fix \(g_0 \in G\) and consider the
    particular case where \(x \coloneq e_G\), we thus have
    \(\lambda(g_0) = g_0 \lambda(e_G) = g_0\) --- therefore,
    \(g_0 = \lambda(g_0) = \alpha(g_0) \eta (\sigma \pi(g_0))\) from the definition
    of \(\lambda\), and \(g_0 = \alpha(g_0) \eta \pi(g_0)\) from
    \cref{eq:write-uniquely-g}. Combining both equations for \(g_0\) we conclude
    that \(\eta (\sigma \pi(g_0)) = \eta \pi(g_0)\), which in turn implies in
    \(\sigma \pi(g_0) = \pi(g_0)\) since \(\eta\) is injective. Furthermore,
    \(\sigma\) was constructed so that \(\im \phi\) was its only fixed point, that
    is, \(\pi(g_0) = \im \phi\) and therefore \(g_0 \in \im \phi\). This shows that
    \(\im \phi = G\) as wanted.
\end{proof}

\begin{remark}
    \label{rem:not-every-grp-epi-is-split}
    The clever reader may be tempted to think that every epimorphism in the category
    of groups is a split epimorphism --- so that the forgetful functor
    \(\Grp \to \Set\) still preserves it. However, things are not as bright as one
    might think.

    Consider for instance the natural projective group morphism
    \(\Z/5\Z \epi \Z/2\Z\) sending \([x]_5 \mapsto [x]_2\). Although an epimorphism,
    such projection is not a split epimorphism --- the lack of elements of order
    \(2\) in \(\Z/5\Z\) does not allow the existence of non-trivial morphisms of
    groups \(\Z/2\Z \to \Z/5\Z\).
\end{remark}

\begin{proposition}[Epimorphisms in \(\Grp\)]
    \label{prop:epic-in-grp}
    Let \(\phi: G \to H\) be a group morphism. The following propositions are
    equivalent:
    \begin{enumerate}[(a)]\setlength\itemsep{0em}
        \item The morphism \(\phi\) is an epimorphism in \(\Grp\).
        \item The set function \(\phi\) is surjective in \(\Set\).
    \end{enumerate}
\end{proposition}

\begin{proof}
    For (a) \(\implies\) (b), we have \cref{lem:epimorphism-is-surjective-grp}. On
    the other hand, the proof that (b) \(\implies\) (a) is straightforward: let
    \(f_1, f_2: H \rightrightarrows Q\) be any two group morphisms such that
    \(f_1 \phi = f_2 \phi\), then given any \(h \in H\) there exists \(g \in G\)
    such that \(\phi(g) = h\) and hence \(f_1\phi(g) = f_1(h) = f_2(h) = f_2
    \phi(g)\) --- since \(h\) was chosen arbitrarily over \(H\), we conclude that
    \(f_1 = f_2\).
\end{proof}

\subsection{Examples \& Consequences}

\begin{proposition}[Modular group \(\PSL_2(\Z)\)]
    \label{prop:PSL_2(Z)-generators}
    Let \(\sim\) be an equivalence relation on \(\SL_2(\Z)\) for which \(A \sim B\)
    if and only if \(A = \pm B\). We define the \emph{modular group} as the quotient
    of \(\SL_2(\Z)\) by this equivalence relation, that is
    \[
        \PSL_2(\Z) \coloneq \SL_2(\Z)/{\sim}
    \]
    The modular group so constructed is generated by the cosets of the matrices
    \[
        \begin{bmatrix}
            0 & -1 \\ 1 & 0
        \end{bmatrix}
        \quad
        \text{ and }
        \quad
        \begin{bmatrix}
            0 & -1 \\ 1 & 1
        \end{bmatrix}.
    \]
\end{proposition}

\begin{proof}
    Let's consider the canonical projection morphism
    \(\pi: \SL_2(\Z) \epi \PSL_2(\Z)\) and let \(g \coloneq \big[
        \begin{smallmatrix}
            0 & -1 \\ 1 & 0
        \end{smallmatrix} \big]
    \) and \(h \coloneq \big[
        \begin{smallmatrix}
            1 & 1 \\ 0 & 1
        \end{smallmatrix} \big]\), then since \(g\) and \(g h = \big[
        \begin{smallmatrix}
            0 & -1 \\ 1 & 1
        \end{smallmatrix} \big]\) generate \(\SL_2(\Z)\), the classes \(\pi(g)\) and
    \(\pi(gh)\) generate \(\PSL_2(\Z)\). Moreover, since \(g^2 = -I_2\), the order
    of \(\pi(g)\) is \(2\) --- on the other hand, \((g h)^3 = -I_2\) then
    \(\pi(gh)\) has order \(3\).
\end{proof}

\begin{proposition}
    \label{prop:finite-abelian-p-div-then-elem-of-p-order}
    Let \(G\) be a \emph{finite abelian group}. If \(p\) is a prime divisor of the
    order of \(G\), then there exists an element of \(G\) whose order is \(p\).
\end{proposition}

\begin{proof}
    Since \(p\) divides \(|G|\), let \(m \in \Z\) such that \(|G| = p m\). We
    proceed via strong induction on \(m\).
    \begin{itemize}\setlength\itemsep{0em}
        \item The base case \(m = 1\), one has \(|G| = p\). If \(g \in G\) is any
              element, then \(\langle g \rangle\) is a subgroup of \(G\) and, by Lagrange's
              theorem, we have that \(|\langle g \rangle|\) divides \(|G|\) --- thus it must
              be the case that \(|\langle g \rangle| = p\). Therefore \(g\) is an element of
              order \(p\), since \(p\) is prime.
        \item For the inductive hypothesis, let \(m > 1\) and assume that that the
              proposition is true for all \(m' < m\).
        \item We now prove the case for \(m\), that is, \(|G| = p m\). Let \(g \in G\)
              be any element and consider the subgroup \(\langle g \rangle\) of
              \(G\).

              If \(p\) divides the order of \(g\), then there exists \(a
              \in \Z\) such that \(p a = |g|\), hence the element \(a g \in G\) has order
              \(p\) --- if this is the case, we are done.

              Otherwise, since \(G\) is abelian, then \(\langle g \rangle\) is a subgroup
              and we may consider the quotient \(G/{\langle g \rangle}\). By
              \cref{prop:grp-index-subgroup} we find that
              \(|G/{\langle g \rangle}| = |G|/{|\langle g \rangle|} = p m / |g|\). Since
              \(p\) does not divide \(|g|\) by hypothesis, then \(|g|\) must divide \(m\),
              therefore \(m/|g| < m\). By the inductive hypothesis, the proposition is true
              for positive integers less than \(m\), hence there exists
              \(h + \langle g \rangle \in G/{\langle g \rangle}\) with order \(p\). If we
              consider the natural projection \(\pi: G \epi G/{\langle g \rangle}\), by
              \cref{prop: grp-morph-order} we have that
              \(|\pi(h)| = |h + \langle g \rangle| = p\) divides the order of \(h\). If
              \(|h| = p b\) for some \(b \in \Z\), then the element \(b h \in G\) has order
              \(p\) and we are done.
    \end{itemize}
\end{proof}

\todo[inline]{Solve Aluffi problems from section 8 (quotient groups)}

\section{Group Towers \& Solvability}

\todo[inline]{Change from ``towers'' to ``series''}

\begin{definition}[Tower of subgroups]
    \label{def:tower-subgroups}
    Let \(G\) be a group. A finite sequence of groups \((G_0, \dots, G_n)\) is said
    to be a \emph{tower of subgroups} of \(G\) if
    \[
        G = G_0 \supsetneq G_1 \supsetneq \dots \supsetneq G_n.
    \]
    The tower may be classified as a \emph{normal tower} if \(G_{j + 1}\) is normal
    in \(G_j\) for every \(0 \leq j \leq n - 1\). The tower is said to be
    \emph{abelian} (respectively, \emph{cyclic}) if it is a normal tower and each
    quotient \(G_j/G_{j+1}\) is abelian (respectively, cyclic).
\end{definition}

A direct corollary of \cref{prop:morphisms-preserve-normality} goes as follows.

\begin{corollary}
    \label{cor:morphism-normal-towers}
    Let \(f: G \to H\) be a group morphism, and \(H \supsetneq H_0 \supsetneq \dots
    \supsetneq H_n\) be a normal tower in \(H\). Then, if we define subgroups \(G_j
    \coloneq f^{-1}(H_j)\) for every \(0 \leq j \leq n\), the sequence \((G_0,
    \dots, G_n)\) forms a normal tower in \(G\).

    Moreover, if \((H_0, \dots, H_n)\) is an abelian tower (respectively, cyclic
    tower) in \(H\), then \((G_0, \dots, G_n)\) is an abelian tower (respectively,
    cyclic tower) in \(G\).
\end{corollary}

\begin{proof}
    The first part is clear. For the second, notice that since for all \(0 \leq j
    \leq n-1\) we have the following commutative diagram
    \[
        \begin{tikzcd}
            G_j \ar[d, two heads] \ar[r, "f"] &H_j \ar[r, two heads] & H_j/H_{j+1} \\
            G_j/G_{j+1} \ar[rru, bend right, tail, dashed]
        \end{tikzcd}
    \]
    then the injection \(G_j/G_{j+1} \mono H_j/H_{j+1}\) allow us to view
    \(G_j/G_{j+1}\) as a subgroup of \(H_j/H_{j+1}\) --- thus necessarily abelian
    (respectively, cyclic).
\end{proof}

\begin{definition}[Refinement]
    \label{def:refinement-tower}
    Given a tower \(G = G_0 \supsetneq \dots \supsetneq G_n\) of subgroups of \(G\),
    a \emph{refinement} of such tower is given by the insertion of a finite
    sequence of subgroups \((G_{n+1}, \dots, G_m)\) to the end of the tower --- that
    is,
    \[
        G = G_{0} \supsetneq \dots \supsetneq G_n \supsetneq G_{n+1} \supsetneq \dots
        \supsetneq G_{m}.
    \]
\end{definition}

\begin{definition}[Solvable group]
    \label{def:solvable-group}
    A group \(G\) is said to be \emph{solvable}, if there exists an \emph{abelian
        tower} in \(G\) such that the last element is the trivial subgroup.
\end{definition}

\begin{proposition}
    \label{prop:finite-grp-cyclic-refinement}
    The following propositions regard implications of the solvability of finite
    groups:
    \begin{enumerate}[(a)]\setlength\itemsep{0em}
        \item Let \(G\) be a \emph{finite} group. An abelian tower of \(G\) admits a
              \emph{cyclic refinement}.
        \item Let \(G\) be a \emph{finite solvable} group. Then \(G\) admits a
              \emph{cyclic tower} whose last element is the trivial subgroup.
    \end{enumerate}
\end{proposition}

\begin{proof}
    Let \(G\) be a finite abelian group. We'll set out to prove that \(G\) admits a
    cyclic tower ending with the trivial subgroup. We proceed by induction on the
    order of \(G\). If \(|G| = 1\), then the proposition follows trivially. Suppose,
    as the hypothesis of induction, that the proposition is true for \(1 \leq |G| <
    n\). Let now \(|G| = n\) and consider any element \(x \neq e\) of \(G\). Define
    the group \(X \coloneq \langle x \rangle\), and \(H \coloneq G/X\) --- since
    \(|H| < |G| = n\) it must be true, from hypothesis, that there exists a cyclic
    tower
    \[
        H \supsetneq H_0 \supsetneq \dots \supsetneq H_m = \{e\}.
    \]
    Let \(\pi: G \epi H\) be the canonical projection. From
    \cref{cor:morphism-normal-towers}, the sequence of subgroups \(G_0 \coloneq G\),
    and \(G_j \coloneq f^{-1}(H_j)\) for \(1 \leq j \leq m\) form a cyclic tower in
    \(G\), thus
    \[
        G = G_0 \supsetneq G_1 \supsetneq \dots \supsetneq G_m \supsetneq \{e\}.
    \]
    is the desired cyclic tower on \(G\). This proves both the first statement and
    the second.
\end{proof}

For the time being, the following theorem is a classic but probably still beyond
the scope of this text, so I shall only mention it.

\begin{theorem}[Feit-Thompson]
    \label{thm:feit-thompson}
    All finite groups of odd order are solvable.
\end{theorem}

\begin{theorem}
    \label{thm:solvable-iff-normal-and-quotient-solvable}
    Let \(N \subseteq G\) be a normal subgroup of the group \(G\). Then \(G\) is
    solvable if and only if both \(N\) and \(G/N\) are solvable.
\end{theorem}

\begin{proof}
    (\(\Rightarrow\)) Suppose \(G\) is solvable, and let \(G = G_0 \supsetneq \dots
    \supsetneq G_n = \{e\}\) be an abelian tower in \(G\).  For each \(1 \leq j \leq
    n\), define \(N_{j} \coloneq N \cap G_j\), and let \(\phi_j: G_j \to Q\) be a
    group morphism with kernel \(G_{j+1}\), then the induced morphism \(\phi_j|_{N
    \cap G_j}: N \cap G_j \to Q\) has kernel \(N \cap G_{j+1} = N_{j+1}\) --- which
    proves that \(N_{j+1}\) is normal in \(N_j\). Moreover, there exists a canonical
    embedding \(N_j/N_{j+1} \emb G_j/G_{j+1}\), which implies that \(N_j/N_{j+1}\)
    is abelian by \cref{prop:injective-morphism-abelian}. Therefore \(N = N_0
    \supsetneq \dots \supsetneq N_n = \{e\}\) is an abelian tower and hence \(N\) is
    solvable. For the group \(G/N\) we can simply consider the tower \(G/N = (G_0
    N)/N \supsetneq \dots \supsetneq (G_n N)/N = (\{e\} N)/N = N\) --- where
    \((G_{j+1} N)/N\) is surely normal in \((G_j N)/N\). Moreover, \([(G_j N)/N]
    \big/ [(G_{j+1} N)/N] \iso (G_j N)/(G_{j+1} N)\) which inherits the commutative
    structure from \(G_j/G_{j+1}\) --- thus the tower is abelian and \(G/N\) is
    solvable.

    (\(\Leftarrow\)) Let both \(N\) and \(G/N\) be solvable and consider the abelian
    towers \(G/N = H_0 \supsetneq \dots \supsetneq H_n = N\), and \(N = Q_0
    \supsetneq \dots \supsetneq Q_m = \{e\}\). Since every \(H_j\) is isomorphic to
    a subgroup \(H_j'\) of \(G\) containing \(N\), we see that
    \[
        G = H_0' \supsetneq \dots \supsetneq H_n' = N
    \]
    is an abelian tower. Moreover, we can append the abelian tower of \(N\) to
    obtain the abelian tower
    \[
        G = H_0' \supsetneq \dots \supsetneq H_n'
        = Q_0 \supsetneq \dots \supsetneq Q_m = \{e\}.
    \]
    This shows that \(G\) is a solvable group.
\end{proof}

\subsection{Commutator Group}

\begin{definition}[Commutator group]
    \label{def:commutator-group}
    Let \(G\) be a group, we define the \emph{commutator} group \([G, G]\) of \(G\)
    to be the subgroup of \(G\) generated by all elements of the form \(x y x^{-1}
    y^{-1}\), for \(x, y \in G\).
\end{definition}

\begin{lemma}
    \label{lem:commutator-is-normal}
    The commutator group \([G, G]\) is \emph{normal} in \(G\).
\end{lemma}

\begin{proof}
    We show that \(g [G, G] g^{-1} = [G, G]\) for all \(g \in G\). Let \(g, x, y \in
    G\) be any triple of elements. If we let \(p \coloneq g x g^{-1}\) and \(q
    \coloneq g y g^{-1}\) we obtain
    \[
        p q p^{-1} q ^{-1}
        = (g x g^{-1}) (g y g^{-1}) (g x g^{-1}) (g y g^{-1})
        = g x y x^{-1} y^{-1} g^{-1},
    \]
    therefore \(g [G, G] g^{-1} \subseteq [G, G]\). Moreover, if \(a \coloneq g^{-1}
    x g\) and \(b \coloneq g^{-1} y g\), then
    \begin{align*}
        g [a b a^{-1} b^{-1}] g^{-1}
         & = g [
                (g^{-1} x g) (g^{-1} y g) (g^{-1} x g)
                (g^{-1} x^{-1} g) (g^{-1} y^{-1} g)
        ] g^{-1}                                    \\
         & = g [ g^{-1} x y x^{-1} y^{-1} g] g^{-1} \\
         & = x y x^{-1} y^{-1},
    \end{align*}
    which implies in \(x y x^{-1} y^{-1} \in g [G, G] g^{-1}\) and hence \([G, G]
    \subseteq g [G, G] g^{-1}\).
\end{proof}

\begin{lemma}
    \label{lem:grp-modulo-commutator-is-commutative}
    The quotient group \(G/[G, G]\) is commutative.
\end{lemma}

\begin{proof}
    Notice that \(x y = y x\) is equivalent to \(x y (y x)^{-1} = x y x^{-1} y^{-1}
    = e\), thus, for any given \(x, y \in G\), since \(x y x^{-1} y^{-1} \in [G,
        G]\), then \([x y] = [y x] \in G/[G, G]\) --- which shows the commutativity.
\end{proof}

\begin{lemma}
    \label{lem:commutator-kernel}
    Let \(G\) be any group and \(H\) be a commutative group. Any morphism of groups
    \(f: G \to H\) has the commutator \([G, G]\) in its kernel. Therefore the
    following diagram commutes
    \[
        \begin{tikzcd}
            G \ar[r, "f"] \ar[d, two heads]            & H \\
            G/[G, G] \ar[ru, bend right, dashed, tail] &
        \end{tikzcd}
    \]
\end{lemma}

\begin{proof}
    If \(H\) is commutative, then given any pair \(x, y \in G\) we have
    \[
        f(x y x^{-1} y^{-1})
        = f(x) f(y) f(x^{-1}) f(y^{-1})
        = f(y) f(x) f(x)^{-1} f(y)^{-1}
        = e_H,
    \]
    thus indeed \([G, G] \subseteq \ker f\).
\end{proof}

%%% Local Variables:
%%% mode: latex
%%% TeX-master: "../../deep-dive"
%%% End:
